\section{Device groups}\label{sec:Basic Facilities of a Virtio Device / Device groups}

It is occasionally useful to have a device control a group of
other devices (the group may occasionally include the device
itself) within a group. The owner device itself is not a
member of the group (except in the special case of the self group).
Terminology used in such cases:

\begin{description}
\item[Device group]
        or just group, includes zero or more devices.
\item[Owner device]
        or owner, the device controlling the group.
\item[Member device]
        a device within a group. The owner device itself is not
	a member of the group except for the \field{Self group type}.
\item[Member identifier]
        each member has this identifier, unique within the group
	and used to address it through the owner device.
\item[Group type identifier]
	specifies what kind of member devices there are in a
	group, how the member identifier is interpreted
	and what kind of control the owner has.
	A given owner can control multiple groups
	of different types but only a single group of a given type,
	thus the type and the owner together identify the group.
	\footnote{Even though some group types only support
			specific transports, group type identifiers
			are global rather than transport-specific -
			a flood of new group types is not expected.}
\end{description}

\begin{note}
Each device only has a single driver, thus for the purposes of
this section, "the driver" is usually unambiguous and refers to
the driver of the owner device.  When there's ambiguity, "owner
driver" refers to the driver of the owner device, while "member
driver" refers to the driver of a member device.
\end{note}

The following group types, and their identifiers, are currently specified:
\begin{description}
\item[Self group type (0x0)]
This device group includes the owner device itself and no other devices.
The group type identifier for this group is 0x0.
The member identifier for this group has a value of 0x0.

\item[SR-IOV group type (0x1)]
This device group has a PCI Single Root I/O Virtualization
(SR-IOV) physical function (PF) device as the owner and includes
all its SR-IOV virtual functions (VFs) as members (see
\hyperref[intro:PCIe]{[PCIe]}).

The PF device itself is not a member of the group.

The group type identifier for this group is 0x1.

A member identifier for this group can have a value from 0x1 to
\field{NumVFs} as specified in the
SR-IOV Extended Capability of the owner device
and equals the SR-IOV VF number of the member device;
the group only exists when the \field{VF Enable} bit
in the SR-IOV Control Register within the
SR-IOV Extended Capability of the owner device is set
(see \hyperref[intro:PCIe]{[PCIe]}).

Both owner and member devices for this group type use the Virtio
PCI transport (see \ref{sec:Virtio Transport Options / Virtio Over PCI Bus}).
\end{description}

\subsection{Group administration commands}\label{sec:Basic Facilities of a Virtio Device / Device groups / Group administration commands}

The driver sends group administration commands to the owner device of
a group to control member devices of the group.
This mechanism can
be used, for example, to configure a member device before it is
initialized by its driver.
\footnote{The term "administration" is intended to be interpreted
widely to include any kind of control. See specific commands
for detail.}

All the group administration commands are of the following form:

\begin{lstlisting}
struct virtio_admin_cmd {
        /* Device-readable part */
        le16 opcode;
        /*
         * 0       - Self
         * 1       - SR-IOV
         * 2-65535 - reserved
         */
        le16 group_type;
        /* unused, reserved for future extensions */
        u8 reserved1[12];
        le64 group_member_id;
        le64 command_specific_data[];

        /* Device-writable part */
        le16 status;
        le16 status_qualifier;
        /* unused, reserved for future extensions */
        u8 reserved2[4];
        u8 command_specific_result[];
};
\end{lstlisting}

For all commands, \field{opcode}, \field{group_type} and if
necessary \field{group_member_id} and \field{command_specific_data} are
set by the driver, and the owner device sets \field{status} and if
needed \field{status_qualifier} and
\field{command_specific_result}.

Generally, any unused device-readable fields are set to zero by the driver
and ignored by the device.  Any unused device-writeable fields are set to zero
by the device and ignored by the driver.

\field{opcode} specifies the command. The valid
values for \field{opcode} can be found in the following table:

\begin{tabularx}{\textwidth}{ |l||l|X| }
\hline
opcode & Name & Command Description \\
\hline \hline
0x0000 & VIRTIO_ADMIN_CMD_LIST_QUERY & Provides to driver list of commands supported for this group type    \\
\hline
0x0001 & VIRTIO_ADMIN_CMD_LIST_USE & Provides to device list of commands used for this group type \\
\hline
0x0002 & \hyperref[par:Basic Facilities of a Virtio Device / Device groups / Group administration commands / Legacy Interface / VIRTIO-ADMIN-CMD-LEGACY-COMMON-CFG-WRITE]{VIRTIO_ADMIN_CMD_LEGACY_COMMON_CFG_WRITE} & Writes into the legacy common configuration structure \\
\hline
0x0003 & \hyperref[par:Basic Facilities of a Virtio Device / Device groups / Group administration commands / Legacy Interface / VIRTIO-ADMIN-CMD-LEGACY-COMMON-CFG-READ]{VIRTIO_ADMIN_CMD_LEGACY_COMMON_CFG_READ} & Reads from the legacy common configuration structure  \\
\hline
0x0004 & \hyperref[par:Basic Facilities of a Virtio Device / Device groups / Group administration commands / Legacy Interface / VIRTIO-ADMIN-CMD-LEGACY-DEV-CFG-WRITE]{VIRTIO_ADMIN_CMD_LEGACY_DEV_CFG_WRITE} & Writes into the legacy device configuration structure \\
\hline
0x0005 & \hyperref[par:Basic Facilities of a Virtio Device / Device groups / Group administration commands / Legacy Interface / VIRTIO-ADMIN-CMD-LEGACY-DEV-CFG-READ]{VIRTIO_ADMIN_CMD_LEGACY_DEV_CFG_READ} & Reads into the legacy device configuration structure \\
\hline
0x0006 & \hyperref[par:Basic Facilities of a Virtio Device / Device groups / Group administration commands / Legacy Interface / VIRTIO-ADMIN-CMD-LEGACY-NOTIFY-INFO]{VIRTIO_ADMIN_CMD_LEGACY_NOTIFY_INFO} & Query the notification region information \\
\hline
0x0007 & \hyperref[par:Basic Facilities of a Virtio Device / Device groups / Group administration commands / Device and driver capabilities / VIRTIO-ADMIN-CMD-CAP-ID-LIST-QUERY]{VIRTIO_ADMIN_CMD_CAP_ID_LIST_QUERY} & Query the supported device capabilities bitmap \\
\hline
0x0008 & \hyperref[par:Basic Facilities of a Virtio Device / Device groups / Group administration commands / Device and driver capabilities / VIRTIO-ADMIN-CMD-DEVICE-CAP-GET]{VIRTIO_ADMIN_CMD_DEVICE_CAP_GET} & Get the device capabilities \\
\hline
0x0009 & \hyperref[par:Basic Facilities of a Virtio Device / Device groups / Group administration commands / Device and driver capabilities / VIRTIO-ADMIN-CMD-DRIVER-CAP-SET]{VIRTIO_ADMIN_CMD_DRIVER_CAP_SET} & Set the driver capabilities \\
\hline
0x000a & \hyperref[par:Basic Facilities of a Virtio Device / Device groups / Group administration commands / Device resource objects / VIRTIO-ADMIN-CMD-RESOURCE-OBJ-CREATE]{VIRTIO_ADMIN_CMD_RESOURCE_OBJ_CREATE} & Create a device resource object \\
\hline
0x000c & \hyperref[par:Basic Facilities of a Virtio Device / Device groups / Group administration commands / Device resource objects / VIRTIO-ADMIN-CMD-RESOURCE-OBJ-MODIFY]{VIRTIO_ADMIN_CMD_RESOURCE_OBJ_MODIFY} & Modify a device resource object \\
\hline
0x000b & \hyperref[par:Basic Facilities of a Virtio Device / Device groups / Group administration commands / Device resource objects / VIRTIO-ADMIN-CMD-RESOURCE-OBJ-QUERY]{VIRTIO_ADMIN_CMD_RESOURCE_OBJ_QUERY} & Query a device resource object \\
\hline
0x000d & \hyperref[par:Basic Facilities of a Virtio Device / Device groups / Group administration commands / Device resource objects / VIRTIO-ADMIN-CMD-RESOURCE-OBJ-DESTROY]{VIRTIO_ADMIN_CMD_RESOURCE_OBJ_DESTROY} & Destroy a device resource object \\
\hline
0x000e & \hyperref[par:Basic Facilities of a Virtio Device / Device groups / Group administration commands / Device parts /  Device parts handling commands / VIRTIO-ADMIN-CMD-DEV-PARTS-METADATA-GET]{VIRTIO_ADMIN_CMD_DEV_PARTS_METADATA_GET} & Get the metadata of the device parts \\
\hline
0x000f & \hyperref[par:Basic Facilities of a Virtio Device / Device groups / Group administration commands / Device parts / Device parts handling commands / VIRTIO-ADMIN-CMD-DEV-PARTS-GET]{VIRTIO_ADMIN_CMD_DEV_PARTS_GET} & Get the device parts \\
\hline
0x0010 & \hyperref[par:Basic Facilities of a Virtio Device / Device groups / Group administration commands / Device parts / Device parts handling commands / VIRTIO-ADMIN-CMD-DEV-PARTS-SET]{VIRTIO_ADMIN_CMD_DEV_PARTS_SET} & Set the device parts \\
\hline
0x0011 & \hyperref[par:Basic Facilities of a Virtio Device / Device groups / Group administration commands / Device parts / Device parts handling commands / VIRTIO-ADMIN-CMD-DEV-MODE-SET]{VIRTIO_ADMIN_CMD_DEV_MODE_SET} & Stop or resume the device \\
\hline
\hline
0x0013 - 0x7FFF & - & Commands using \field{struct virtio_admin_cmd}    \\
\hline
0x8000 - 0xFFFF & - & Reserved for future commands (possibly using a different structure)    \\
\hline
\end{tabularx}

The \field{group_type} specifies the group type identifier.
The \field{group_member_id} specifies the member identifier within the group.
See section \ref{sec:Basic Facilities of a Virtio Device / Device groups}
for the definition of the group type identifier and group member
identifier.

The \field{status} describes the command result and possibly
failure reason at an abstract level, this is appropriate for
forwarding to applications. The \field{status_qualifier} describes
failures at a low virtio specific level, as appropriate for debugging.
The following table describes possible \field{status} values;
to simplify common implementations, they are intentionally
matching common \hyperref[intro:errno]{Linux error names and numbers}:

\begin{tabular}{|l|l|l|}
\hline
Status (decimal) & Name & Description \\
\hline \hline
00   & VIRTIO_ADMIN_STATUS_OK    & successful completion  \\
\hline
06   & VIRTIO_ADMIN_STATUS_ENXIO & no such capability or resource\\
\hline
11   & VIRTIO_ADMIN_STATUS_EAGAIN    & try again \\
\hline
12   & VIRTIO_ADMIN_STATUS_ENOMEM    & insufficient resources \\
\hline
16   & VIRTIO_ADMIN_STATUS_EBUSY     & device busy \\
\hline
22   & VIRTIO_ADMIN_STATUS_EINVAL    & invalid command \\
\hline
28   & VIRTIO_ADMIN_STATUS_ENOSPC    & resources exhausted on device \\
\hline
other   & -    & group administration command error  \\
\hline
\end{tabular}

When \field{status} is VIRTIO_ADMIN_STATUS_OK, \field{status_qualifier}
is reserved and set to zero by the device.

The following table describes possible \field{status_qualifier} values:

\begin{tabularx}{\textwidth}{ |l||l|X| }
\hline
Status & Name & Description \\
\hline \hline
0x00   & VIRTIO_ADMIN_STATUS_Q_OK               & used with VIRTIO_ADMIN_STATUS_OK  \\
\hline
0x01   & VIRTIO_ADMIN_STATUS_Q_INVALID_COMMAND  & command error: no additional information  \\
\hline
0x02   & VIRTIO_ADMIN_STATUS_Q_INVALID_OPCODE   & unsupported or invalid \field{opcode}  \\
\hline
0x03   & VIRTIO_ADMIN_STATUS_Q_INVALID_FIELD    & unsupported or invalid field within \field{command_specific_data}  \\
\hline
0x04   & VIRTIO_ADMIN_STATUS_Q_INVALID_GROUP    & unsupported or invalid \field{group_type} \\
\hline
0x05   & VIRTIO_ADMIN_STATUS_Q_INVALID_MEMBER   & unsupported or invalid \field{group_member_id} \\
\hline
0x06   & VIRTIO_ADMIN_STATUS_Q_NORESOURCE       & out of internal resources: ok to retry \\
\hline
0x07   & VIRTIO_ADMIN_STATUS_Q_TRYAGAIN         & command blocks for too long: should retry \\
\hline
0x08-0xFFFF   & -    & reserved for future use \\
\hline
\end{tabularx}

Each command uses a different \field{command_specific_data} and
\field{command_specific_result} structures and the length of
\field{command_specific_data} and \field{command_specific_result}
depends on these structures and is described separately or is
implicit in the structure description.

Before sending any group administration commands to the device, the driver
needs to communicate to the device which commands it is going to
use. Initially (after reset), only two commands are assumed to be used:
VIRTIO_ADMIN_CMD_LIST_QUERY and VIRTIO_ADMIN_CMD_LIST_USE.

Before sending any other commands for any member of a specific group to
the device, the driver queries the supported commands via
VIRTIO_ADMIN_CMD_LIST_QUERY and sends the commands it is
capable of using via VIRTIO_ADMIN_CMD_LIST_USE.

Commands VIRTIO_ADMIN_CMD_LIST_QUERY and
VIRTIO_ADMIN_CMD_LIST_USE
both use the following structure describing the
command opcodes:

\begin{lstlisting}
struct virtio_admin_cmd_list {
       /* Indicates which of the below fields were returned
       le64 device_admin_cmd_opcodes[];
};
\end{lstlisting}

This structure is an array of 64 bit values in little-endian byte
order, in which a bit is set if the specific command opcode
is supported. Thus, \field{device_admin_cmd_opcodes[0]} refers to the
first 64-bit value in this array corresponding to opcodes 0 to
63, \field{device_admin_cmd_opcodes[1]} is the second 64-bit value
corresponding to opcodes 64 to 127, etc.
For example, the array of size 2 including
the values 0x3 in \field{device_admin_cmd_opcodes[0]}
and 0x1 in \field{device_admin_cmd_opcodes[1]} indicates that only
opcodes 0, 1 and 64 are supported.
The length of the array depends on the supported opcodes - it is
large enough to include bits set for all supported opcodes,
that is the length can be calculated by starting with the largest
supported opcode adding one, dividing by 64 and rounding up.
In other words, for
VIRTIO_ADMIN_CMD_LIST_QUERY and VIRTIO_ADMIN_CMD_LIST_USE the
length of \field{command_specific_result} and
\field{command_specific_data} respectively will be
$DIV_ROUND_UP(max_cmd, 64) * 8$ where DIV_ROUND_UP is integer division
with round up and \field{max_cmd} is the largest available command opcode.

The array is also allowed to be larger and to additionally
include an arbitrary number of all-zero entries.

Accordingly, bits 0 and 1 corresponding to opcode 0
(VIRTIO_ADMIN_CMD_LIST_QUERY) and 1
(VIRTIO_ADMIN_CMD_LIST_USE) are
always set in \field{device_admin_cmd_opcodes[0]} returned by VIRTIO_ADMIN_CMD_LIST_QUERY.

For the command VIRTIO_ADMIN_CMD_LIST_QUERY, \field{opcode} is set to 0x0.
The \field{group_member_id} is unused. It is set to zero by driver.
This command has no command specific data.
The device, upon success, returns a result in
\field{command_specific_result} in the format
\field{struct virtio_admin_cmd_list} describing the
list of group administration commands supported for the group type
specified by \field{group_type}.

For the command VIRTIO_ADMIN_CMD_LIST_USE, \field{opcode}
is set to 0x1.
The \field{group_member_id} is unused. It is set to zero by driver.
The \field{command_specific_data} is in the format
\field{struct virtio_admin_cmd_list} describing the
list of group administration commands used by the driver
with the group type specified by \field{group_type}.

This command has no command specific result.

The driver issues the command VIRTIO_ADMIN_CMD_LIST_QUERY to
query the list of commands valid for this group and before sending
any commands for any member of a group.

The driver then enables use of some of the opcodes by sending to
the device the command VIRTIO_ADMIN_CMD_LIST_USE with a subset
of the list returned by VIRTIO_ADMIN_CMD_LIST_QUERY that is
both understood and used by the driver.

If the device supports the command list used by the driver, the
device completes the command with status VIRTIO_ADMIN_STATUS_OK.
If the device does not support the command list
(for example, if the driver is not capable to use
some required commands), the device
completes the command with status
VIRTIO_ADMIN_STATUS_INVALID_FIELD.

Note: the driver is assumed not to set bits in
device_admin_cmd_opcodes
if it is not familiar with how the command opcode
is used, since the device could have dependencies between
command opcodes.

It is assumed that all members in a group support and are used
with the same list of commands. However, for owner devices
supporting multiple group types, the list of supported commands
might differ between different group types.

\input{admin-cmds-legacy-interface.tex}
\input{admin-cmds-capabilities.tex}
\subsubsection{Device resource objects}\label{sec:Basic Facilities of a Virtio Device / Device groups / Group administration commands / Device resource objects}

Providing certain functionality consumes limited device resources such as
memory, processing units, buffer memory, or end-to-end credits. A device may
support multiple types of resource objects, each controlling different device
functionality. To manage this, virtio provides
\field{Device resource objects} that the driver can create, modify, and
destroy using administration commands with the self group type. Creating and
destroying a resource object consume and release device resources, respectively.
The device resource object query command returns the resource object as
maintained by the device.

For each resource type, the number of resource objects that can be created
is reported by the device as part of a device capability
\ref{sec:Basic Facilities of a Virtio Device / Device groups / Group administration commands / Device and driver capabilities}.
The driver reports the desired (same or lower) number of resource objects
as part of a driver capability \ref{sec:Basic Facilities of a Virtio Device / Device groups / Group administration commands / Device and driver capabilities}.
For each device object type, resource object limit is defined by field
\field{limit} using \field{Device and driver capabilities}.

\begin{lstlisting}
le32 limit; /* maximum resource id = limit - 1 */
\end{lstlisting}

Each resource object has a unique resource object ID - a driver-assigned number
in the range of 0 to \field{limit - 1}, where the \field{limit} is the maximum
number set by the driver for this resource object type. These resource IDs are unique within
each resource object type. The driver assigns the resource ID when creating a
device resource object. Once the resource object is successfully created,
subsequent resource modification, query, and destroy commands use this
resource object ID. No two resource objects share the same ID. Destroying a
resource object allows for the reuse of its ID for another resource object
of the same type.

A valid resource object id is \field{limit - 1}. For example, when a device
reports a \field{limit = 10} capability for a resource object, and drivers sets
\field{limit = 8}, the valid resource object id range for the device and the
driver is 0 to 7 for all the resource object commands. In this example,
the driver can only create 8 resource objects of a specified type.

A resource object of one type may depend on the resource object of another type.
Such dependency between resource objects is established by referring to the unique
resource ID in the administration commands. For example, a driver creates a
resource object identified by ID A of one type, then creates another resource
object identified by ID B of a different type, which depends on resource object
A. This dependency establishes the lifecycle of these resource objects. The driver
that creates the dependent resource object must destroy the resource objects in the
exact reverse order of their creation. In this example, the driver would
destroy resource object B before destroying resource object A.

Some resource object types are generic, common across multiple devices.
Others are specific for one device type.

\begin{tabular}{|l|l|}
\hline
Resource object type & Description \\
\hline \hline
0x000-0x1ff & Generic resource object type common across all devices \\
\hline
0x200-0x4ff & Device type specific resource object \\
\hline
0x500-0xffff & Reserved for future use  \\
\hline
\end{tabular}

Following generic resource objects are defined which are described separately.

\begin{xltabular}{\textwidth}{ |X||X|X| }
\hline
Resource object type & Name & Description \\
\hline \hline
0x000 & VIRTIO_RESOURCE_OBJ_DEV_PARTS & Device parts object, see \ref{par:Basic Facilities of a Virtio Device / Device groups / Group administration commands / Device parts / VIRTIO_RESOURCE_OBJ_DEV_PARTS} \\
\hline
0x001-0x1ff & - & Generic resource object range reserved \\
\hline
\hline
\end{xltabular}

When the device resets, it starts with zero resources of each type; the driver
can create resources up to the published \field{limit}. The driver can
destroy and recreate the resource one or multiple times. Upon device reset,
all resource objects created by the driver are destroyed within the device.

Following administration commands control device resource objects,
they are supported for the self group type, occasionally some resource
objects can be created for the SR-IOV group type as well. Such sr-iov group
type specific resource objects are listed where such objects is defined.

\begin{enumerate}
\item VIRTIO_ADMIN_CMD_RESOURCE_OBJ_CREATE
\item VIRTIO_ADMIN_CMD_RESOURCE_OBJ_MODIFY
\item VIRTIO_ADMIN_CMD_RESOURCE_OBJ_QUERY
\item VIRTIO_ADMIN_CMD_RESOURCE_OBJ_DESTROY
\end{enumerate}

Each resource object administration command uses a common header
\field{struct virtio_admin_cmd_resource_obj_cmd_hdr}.

\begin{lstlisting}
struct virtio_admin_cmd_resource_obj_cmd_hdr {
        le16 type;
        u8 reserved[2];
        le32 id; /* Indicates unique resource object id per resource object type */
};

\end{lstlisting}

\field{type} refers to the device resource object type.
\field{id} uniquely identifies the resource object of a specified \field{type}.

\paragraph{VIRTIO_ADMIN_CMD_RESOURCE_OBJ_CREATE}
\label{par:Basic Facilities of a Virtio Device / Device groups / Group administration commands / Device resource objects / VIRTIO_ADMIN_CMD_RESOURCE_OBJ_CREATE}

This command creates the specified resource object of \field{type} identified by the
resource id \field{id}. The valid range of \field{id} is defined by the
device in the related device capability. The driver assigns the unique \field{id}
for the resource for the specified \field{type}.

For the command VIRTIO_ADMIN_CMD_RESOURCE_OBJ_CREATE, \field{opcode} is set to 0xa.
\field{group_member_id} is set to zero for self-group type and set to
the member device to be accessed for the SR-IOV group type.
The \field{command_specific_data} is in the format
\field{struct virtio_admin_cmd_resource_obj_create_data}.
\field{resource_obj_specific_data} refers to the resource object specific data.
Each resource uses a different \field{resource_obj_specific_data} and is described
separately.

\field{flags} is reserved for future extension for optional resource object attributes and
is set to 0. Each resource object uses a different value for
\field{flags} and it is described separately.

\begin{lstlisting}
struct virtio_admin_cmd_resource_obj_create_data {
        struct virtio_admin_cmd_resource_obj_cmd_hdr hdr;
        le64 flags;
        u8 resource_obj_specific_data[];
};
\end{lstlisting}

When the command completes successfully, the resource object is created by the
device and the device can immediately begin using it.
This command has no command specific result.

\paragraph{VIRTIO_ADMIN_CMD_RESOURCE_OBJ_MODIFY}
\label{par:Basic Facilities of a Virtio Device / Device groups / Group administration commands / Device resource objects / VIRTIO_ADMIN_CMD_RESOURCE_OBJ_MODIFY}

This command modifies the attributes of an existing device resource object.
For the command VIRTIO_ADMIN_CMD_RESOURCE_OBJ_MODIFY, \field{opcode} is set to 0xb.
The \field{command_specific_data} is in the format
\field{struct virtio_admin_cmd_resource_modify_data}.
\field{group_member_id} is set to zero for self-group type and set to
the member device to be accessed for the SR-IOV group type.
\field{id} identifies the resource object of type \field{type} whose attributes
to modify.
This command modifies the attributes supplied in \field{resource_obj_specific_data}.

\begin{lstlisting}
struct virtio_admin_cmd_resource_modify_data {
        struct virtio_admin_cmd_resource_obj_cmd_hdr hdr;
        le64 flags;
        u8 resource_obj_specific_data[];
};
\end{lstlisting}

This command has no command specific result.
When the command completes successfully, attributes of the resource object is
set to the values supplied in \field{resource_obj_specific_data}.

\paragraph{VIRTIO_ADMIN_CMD_RESOURCE_OBJ_QUERY}
\label{par:Basic Facilities of a Virtio Device / Device groups / Group administration commands / Device resource objects / VIRTIO_ADMIN_CMD_RESOURCE_OBJ_QUERY}

This command queries attributes of the existing resource object.
For the command VIRTIO_ADMIN_CMD_RESOURCE_OBJ_QUERY, \field{opcode} is set to 0xc.
\field{group_member_id} is set to zero for self-group type and set to
the member device to be accessed for the SR-IOV group type.
The \field{command_specific_data} is in the format
\field{struct virtio_admin_cmd_resource_obj_query_data}.
\field{id} identifies the existing resource object of type \field{type} whose
attributes to query.

\begin{lstlisting}
struct virtio_admin_cmd_resource_obj_query_data {
        struct virtio_admin_cmd_resource_obj_cmd_hdr hdr;
        le64 flags;
};
\end{lstlisting}

\begin{lstlisting}
struct virtio_admin_cmd_resource_obj_query_result {
        u8 resource_obj_specific_result[];
};
\end{lstlisting}

\field{command_specific_result} is in the format
\field{virtio_admin_cmd_resource_obj_query_result}.

When the command completes successfully, the attributes of the specified
resource object are are set in \field{resource_obj_specific_data}.

\paragraph{VIRTIO_ADMIN_CMD_RESOURCE_OBJ_DESTROY}
\label{par:Basic Facilities of a Virtio Device / Device groups / Group administration commands / Device resource objects / VIRTIO_ADMIN_CMD_RESOURCE_OBJ_DESTROY}

This command destroys the previously created device resource object.
For the command VIRTIO_ADMIN_CMD_RESOURCE_OBJ_DESTROY, \field{opcode} is set to 0xd.
The \field{command_specific_data} is in the format
\field{struct virtio_admin_cmd_resource_obj_cmd_hdr}.
\field{group_member_id} is set to zero for self-group type and set to
the member device to be accessed for the SR-IOV group type.
\field{id} identifies the existing resource object of type \field{type}.

This command destroys the specified resource object of \field{type} identified
by \field{id}, which is previously created using
VIRTIO_ADMIN_CMD_RESOURCE_OBJ_CREATE command.

This command has no command specific result.
When the command completes successfully, the resource object is destroyed from the device.

\devicenormative{\paragraph}{Device resource objects}{Basic Facilities of a Virtio Device / Device groups / Group administration commands / Device resource objects}

The device SHOULD complete the command VIRTIO_ADMIN_CMD_RESOURCE_OBJ_CREATE
with \field{status} set to VIRTIO_ADMIN_STATUS_EEXIST if a resource object already exists
with supplied resource \field{id} for the specified \field{type}.

The device SHOULD complete the commands VIRTIO_ADMIN_CMD_RESOURCE_OBJ_MODIFY,
VIRTIO_ADMIN_CMD_RESOURCE_QUERY and
VIRTIO_ADMIN_CMD_RESOURCE_OBJ_DESTROY with \field{status} set to
VIRTIO_ADMIN_STATUS_ENXIO if the specified resource object does not exist.

The device SHOULD set \field{status} to VIRTIO_ADMIN_STATUS_ENOSPC for the
command VIRTIO_ADMIN_CMD_RESOURCE_OBJ_CREATE if the device fail to create the
resource object.

The device SHOULD complete the commands VIRTIO_ADMIN_CMD_RESOURCE_OBJ_MODIFY or
VIRTIO_ADMIN_CMD_RESOURCE_OBJ_DESTROY commands with \field{status} set to
VIRTIO_ADMIN_STATUS_EBUSY if other resource objects depend on the resource object
being modified or destroyed.

The device MUST allow recreating the resource object using the command
VIRTIO_ADMIN_CMD_RESOURCE_OBJ_CREATE which was previously
destroyed using the command VIRTIO_ADMIN_CMD_RESOURCE_OBJ_DESTROY respectively
without undergoing a device reset.

The device SHOULD allow creating the resource object using
the command VIRTIO_ADMIN_CMD_RESOURCE_OBJ_CREATE with any resource
id as long as the resource object is not created.

The device MAY fail the command VIRTIO_ADMIN_CMD_RESOURCE_OBJ_CREATE even if the
resources within the device have not reached up to the \field{max_limit}
but the device MAY have reached an internal limit.

When a capability represents a number of resource objects, the device SHOULD
allow creating as many resource objects as represented by the driver capability.

The device MUST NOT have any side effects on the resource object when the command
VIRTIO_ADMIN_CMD_RESOURCE_OBJ_MODIFY fails.

The device MUST complete the command VIRTIO_ADMIN_CMD_RESOURCE_OBJ_QUERY
with \field{resource_obj_specific_data} which is matching the
\field{resource_obj_specific_data} of last VIRTIO_ADMIN_CMD_RESOURCE_OBJ_CREATE
or VIRTIO_ADMIN_CMD_RESOURCE_OBJ_MODIFY command.

On device reset, the device MUST destroy all the resource objects which
have been created.

\drivernormative{\paragraph}{Device resource objects}{Basic Facilities of a Virtio Device / Device groups / Group administration commands / Device resource objects}

The driver MUST not create a second resource object of the same type with same
ID using command VIRTIO_ADMIN_CMD_RESOURCE_OBJ_CREATE before destroying the
previously created resource object.

The driver MUST NOT create more resource objects of a specified \field{type} using
command VIRTIO_ADMIN_CMD_RESOURCE_OBJ_CREATE than the maximum limit set by the
driver capability.

The driver SHOULD NOT modify, query and destroy the resource object which is
already destroyed previously by the driver.

The driver SHOULD NOT destroy the resource object on which other resource objects
are depending; the driver SHOULD destroy all the resource objects which do not depend
on other resource objects.

The driver MUST NOT set the capability related to the resource objects if the
resource objects have been created using the command VIRTIO_ADMIN_CMD_RESOURCE_OBJ_CREATE
and not yet destroyed.

The driver MUST send the command VIRTIO_ADMIN_CMD_DRIVER_CAP_SET before using
any resources related to such capability.

\subsubsection{Device parts}\label{sec:Basic Facilities of a Virtio Device / Device groups / Group administration commands / Device parts}

In some systems, there is a need to capture the state of all or part of
a device and subsequently restore either the same device or a
different one to this captured state. A group owner device can support
administration commands to facilitate these get and set operations for
the group member devices.

For example, a hypervisor can use the administration commands to
capture parts of the device state and save the result as part of
a VM snapshot. Later, the hypervisor can retrieve the snapshot and use the
administration commands to restore parts of a device to resume
VM operation.

As another example, these commands can be used to facilitate VM migration by the
hypervisor: one (source) hypervisor can get parts of a device and send
the results to another (destination) hypervisor, which will in turn
set (restore) parts of (another) device to resume the VM operation on
the destination.

The device comprises many device parts which the driver can get and set.
Administration commands are provided to either get and set all the device
parts at once, or to get the device parts metadata that indicates which
device parts are present, and later to get and set specific device parts.
To get and set the device parts or their metadata, the driver first creates a
device parts resource object, indicating whether the object should
handle get or set operations but not both simultaneously. The device and the
driver indicate the device parts resource objects' limit using the capability
VIRTIO_DEV_PARTS_CAP.

The device can be stopped to prevent device parts from changing.
When the device is stopped, it does not initiate any transport requests.
For instance, the device refrains from sending any configuration or
virtqueue notifications and does not access any virtqueues or the driver's
buffer memory. While the driver may remain active and continue to send
notifications to the device, potentially updating some device parts,
the device itself will not initiate any transport requests.

\paragraph{VIRTIO_DEV_PARTS_CAP}
\label{par:Basic Facilities of a Virtio Device / Device groups / Group administration commands / Device parts / VIRTIO-DEV-PARTS-CAP}

The capability VIRTIO_DEV_PARTS_CAP indicates the device parts resource objects limit.
\field{cap_specific_data} is in the format \field{struct virtio_dev_parts_cap}.

\begin{lstlisting}
struct virtio_dev_parts_cap {
        u8 get_parts_resource_objects_limit;
        u8 set_parts_resource_objects_limit;
};
\end{lstlisting}

\field{get_parts_resource_objects_limit} indicates the supported device parts
resource objects for retrieving the device parts.
\field{set_parts_resource_objects_limit} indicates the supported device parts
resource objects for restoring the device parts.

\paragraph{VIRTIO_RESOURCE_OBJ_DEV_PARTS}\label{par:Basic Facilities of a Virtio Device / Device groups / Group administration commands / Device parts / VIRTIO-RESOURCE-OBJ-DEV-PARTS}

A device parts resource object is used to either get or set the device parts.
Before performing any get or set operation for the device parts, the driver
creates the device parts resource object
VIRTIO_RESOURCE_OBJ_DEV_PARTS using the administration command
\nameref{par:Basic Facilities of a Virtio Device / Device groups / Group administration commands / Device resource objects / VIRTIO-ADMIN-CMD-RESOURCE-OBJ-CREATE}.
The driver indicates the intended purpose (get or set) at the time of creating the
device parts resource object.
For the device parts resource object, both \field{resource_obj_specific_data} and
\field{resource_obj_specific_result} are in the format
\field{struct virtio_resource_obj_dev_parts}.

\begin{lstlisting}
struct virtio_resource_obj_dev_parts {
        u8 type;
#define VIRTIO_RESOURCE_OBJ_DEV_PARTS_TYPE_GET 0
#define VIRTIO_RESOURCE_OBJ_DEV_PARTS_TYPE_SET 1
        u8 reserved[7];
};
\end{lstlisting}

When \field{type} is set to VIRTIO_RESOURCE_OBJ_DEV_PARTS_TYPE_GET,
the driver can use the object to capture the device parts and the metadata of
these device parts. When \field{type} is set to
VIRTIO_RESOURCE_OBJ_DEV_PARTS_TYPE_SET, the driver can use the
object to restore the device parts.

\paragraph{Device parts handling commands}\label{par:Basic Facilities of a Virtio Device / Device groups / Group administration commands / Device parts / Device parts handling commands}

The owner driver uses the following resource object handling administration
commands. These commands are only used for the device parts resource
object after the driver creates the VIRTIO_RESOURCE_OBJ_DEV_PARTS object.
These commands are currently only defined for the SR-IOV group type:

\begin{enumerate}
\item VIRTIO_ADMIN_CMD_DEV_PARTS_METADATA_GET
\item VIRTIO_ADMIN_CMD_DEV_PARTS_GET
\item VIRTIO_ADMIN_CMD_DEV_PARTS_SET
\end{enumerate}

\subparagraph{VIRTIO_ADMIN_CMD_DEV_PARTS_METADATA_GET}
\label{par:Basic Facilities of a Virtio Device / Device groups / Group administration commands / Device parts / Device parts handling commands / VIRTIO-ADMIN-CMD-DEV-PARTS-METADATA-GET}

This command obtains the metadata of the device parts. This metadata includes
the maximum size of the device parts, the count of device parts, and a list of
the device part headers.

For the command VIRTIO_ADMIN_CMD_DEV_PARTS_METADATA_GET, \field{opcode} is set
to 0xe. The \field{command_specific_data} is in the format
\field{struct virtio_admin_cmd_dev_parts_metadata_data}.
\field{group_member_id} refers to the member device to be accessed.
The resource object \field{type} in the \field{hdr} is set to
VIRTIO_RESOURCE_OBJ_DEV_PARTS and \field{id} is set to the ID of the
device parts resource object.

\begin{lstlisting}
struct virtio_admin_cmd_dev_parts_metadata_data {
        struct virtio_admin_cmd_resource_obj_cmd_hdr hdr;
        u8 type;
        u8 reserved[7];
};

#define VIRTIO_ADMIN_CMD_DEV_PARTS_METADATA_TYPE_SIZE 0
#define VIRTIO_ADMIN_CMD_DEV_PARTS_METADATA_TYPE_COUNT 1
#define VIRTIO_ADMIN_CMD_DEV_PARTS_METADATA_TYPE_LIST  2

struct virtio_admin_cmd_dev_parts_metadata_result {
        union {
                struct {
                        le32 size;
                        le32 reserved;
                } parts_size;
                struct {
                        le32 count;
                        le32 reserved;
                } hdr_list_count;
                struct {
                        le32 count;
                        le32 reserved;
                        struct virtio_dev_part_hdr hdrs[];
                } hdr_list;
        };
};
\end{lstlisting}

When the command completes successfully, the
\field{command_specific_result} is in the format
\field{struct virtio_admin_cmd_dev_parts_metadata_result}.

When \field{type} is set to VIRTIO_ADMIN_CMD_DEV_PARTS_METADATA_TYPE_SIZE,
the device responds with \field{parts_size}. \field{parts_size.size} indicates
the maximum size in bytes for all the device parts.

When \field{type} is set to VIRTIO_ADMIN_CMD_DEV_PARTS_METADATA_TYPE_COUNT, the
device responds with \field{hdr_list_count.count}. The
\field{hdr_list_count.count} indicates an count of
\field{struct virtio_dev_part_hdr} metadata entries that the device can
provide when the \field{type} is set to VIRTIO_ADMIN_CMD_DEV_PARTS_METADATA_TYPE_LIST
in a subsequent VIRTIO_ADMIN_CMD_DEV_PARTS_METADATA_GET command.

When \field{type} is set to VIRTIO_ADMIN_CMD_DEV_PARTS_METADATA_TYPE_LIST,
the device responds with \field{hdr_list}. \field{hdr_list}
indicates the device parts metadata.

\field{reserved} is reserved and set to 0.

The command responds with the \field{status} VIRTIO_ADMIN_STATUS_ENOMEM
when the size of \field{command_specific_result} is not sufficient enough
for the response.

\subparagraph{VIRTIO_ADMIN_CMD_DEV_PARTS_GET}
\label{par:Basic Facilities of a Virtio Device / Device groups / Group administration commands / Device parts /  Device parts handling commands / VIRTIO-ADMIN-CMD-DEV-PARTS-GET}

This command captures the device parts. For the command
VIRTIO_ADMIN_CMD_DEV_PARTS_GET, \field{opcode} is set to 0xf.
The \field{command_specific_data} is in the format
\field{struct virtio_admin_cmd_dev_parts_get_data}.
\field{group_member_id} refers to the member device to be accessed.
The resource object \field{type} in the \field{hdr} is set to
VIRTIO_RESOURCE_OBJ_DEV_PARTS and \field{id} is set to the ID of the
device parts resource object.

\begin{lstlisting}
struct virtio_admin_cmd_dev_parts_get_data {
        struct virtio_admin_cmd_resource_obj_cmd_hdr hdr;
        u8 type;
        u8 reserved[7];
        struct virtio_dev_part_hdr hdr_list[];
};

#define VIRTIO_ADMIN_CMD_DEV_PARTS_GET_TYPE_SELECTED 0
#define VIRTIO_ADMIN_CMD_DEV_PARTS_GET_TYPE_ALL 1

struct virtio_admin_cmd_dev_parts_get_result {
        struct virtio_dev_part parts[];
};

\end{lstlisting}

When the driver wants to capture specific device parts, \field{type} is set to
VIRTIO_ADMIN_CMD_DEV_PARTS_GET_TYPE_SELECTED and \field{hdr_list} is set to the
device parts of interest.

When the driver wants to retrieve all the device parts, \field{type} is set to
VIRTIO_ADMIN_CMD_DEV_PARTS_GET_TYPE_ALL, and \field{hdr_list} is empty.

\field{reserved} is reserved and set to 0.

When the command completes successfully, the \field{command_specific_result} is
in the format \field{struct virtio_admin_cmd_dev_parts_get_result}, containing
either the selected device parts or all the device parts.

If the requested device part does not exist, the device skips the device part
without any error.

\subparagraph{VIRTIO_ADMIN_CMD_DEV_PARTS_SET}\label{par:Basic Facilities of a Virtio Device / Device groups / Group administration commands / Device parts / Device parts handling commands / VIRTIO-ADMIN-CMD-DEV-PARTS-SET}

This command sets one or multiple device parts. For the command
VIRTIO_ADMIN_CMD_DEV_PARTS_SET, \field{opcode} is set to 0x10.
The \field{group_member_id} refers to the member device to be accessed.
The resource object \field{type} in the \field{hdr} is set to
VIRTIO_RESOURCE_OBJ_DEV_PARTS and \field{id} is set to the ID of the
device parts resource object.

\begin{lstlisting}
struct virtio_admin_cmd_dev_parts_set_data {
        struct virtio_admin_cmd_resource_obj_cmd_hdr hdr;
        struct virtio_dev_part parts[];
};
\end{lstlisting}

The \field{command_specific_data} is in the format
\field{struct virtio_admin_cmd_dev_parts_set_data}.

This command has no command specific result.

The driver stops the device before setting any device parts.

When the command completes successfully, the device has updated device
parts to the value supplied in \field{virtio_admin_cmd_dev_parts_set_data}.

The device parts set by this command take effect when the device is resumed
using the VIRTIO_ADMIN_CMD_DEV_MODE_SET command.

When the command fails with a status other than VIRTIO_ADMIN_STATUS_OK, the
device does not have any side effects.

\subparagraph{VIRTIO_ADMIN_CMD_DEV_MODE_SET}\label{par:Basic Facilities of a Virtio Device / Device groups / Group administration commands / Device parts / Device parts handling commands / VIRTIO-ADMIN-CMD-DEV-MODE-SET}

This command either stops the device from initiating any transport requests or
resumes the device operation. For the command VIRTIO_ADMIN_CMD_DEV_MODE_SET,
\field{opcode} is set to 0x11. \field{group_member_id} indicates the member
device to be accessed.

The \field{command_specific_data} is in the format
\field{struct virtio_admin_cmd_dev_mode_set_data}.

\begin{lstlisting}
struct virtio_admin_cmd_dev_mode_set_data {
        u8 flags;
};

#define VIRTIO_ADMIN_CMD_DEV_MODE_F_STOPPED 0
\end{lstlisting}

This command has no command specific result.

When the command completes successfully and if the \field{flags} field is set
to VIRTIO_ADMIN_CMD_DEV_MODE_F_STOPPED (bit 0), the device is stopped.
When the device is stopped, the device stops initiating all transport
communications, which includes:

\begin{enumerate}
\item stopping configuration change notifications
\item stopping all virtqueue notifications
\item stops accessing all virtqueues and the driver buffer memory
\end{enumerate}

After the device is stopped, the device parts remain unchanged unless
the driver initiates any transport requests.

When the device is stopped, it writes back any associated descriptors for all
observed buffers to prevent out-of-order processing if the device is resumed.

When the command completes successfully and if the \field{flags} field
is set to zero, the device resumes its operation. If the command completes
with an error, it does not produce any side effects on the device.

\paragraph{Device parts order}\label{par:Basic Facilities of a Virtio Device / Device groups / Group administration commands / Device parts / Device parts order}

Device parts are usually captured and restored using get and set administration
commands respectively; when multiple device parts are captured or restored,
they are arranged in the specific order listed:

Some of the device parts do not need to be written to the device when restored, such
device parts are listed as \field{O}. When a such an optional device part is
exchanged using \field{struct virtio_dev_part}, it is marked as optional by
setting VIRTIO_DEV_PART_F_OPTIONAL(bit 0) in the \field{flags}.

\begin{table}[H]
\caption{Device parts order}
\label{table:Basic Facilities of a Virtio Device / Device groups / Group administration commands / Device parts / Device parts order/ Device parts order}
\begin{tabularx}{\textwidth}{ |l|l|X| }
\hline
Part name & Optional & Mandatory preceding parts \\
\hline \hline
\hline
VIRTIO_DEV_PART_DEV_FEATURES & O & Nil \\
\hline
VIRTIO_DEV_PART_DRV_FEATURES & - & Nil \\
\hline
VIRTIO_DEV_PART_PCI_COMMON_CFG & - & VIRTIO_DEV_PART_DEV_FEATURES, VIRTIO_DEV_PART_DRV_FEATURES \\
\hline
VIRTIO_DEV_PART_DEVICE_STATUS & - & VIRTIO_DEV_PART_DEV_FEATURES, VIRTIO_DEV_PART_DRV_FEATURES, VIRTIO_DEV_PART_PCI_COMMON_CFG \\
\hline
VIRTIO_DEV_PART_VQ_CFG & - & VIRTIO_DEV_PART_DEV_FEATURES, VIRTIO_DEV_PART_DRV_FEATURES, VIRTIO_DEV_PART_PCI_COMMON_CFG,
                             VIRTIO_DEV_PART_DEVICE_STATUS \\
\hline
VIRTIO_DEV_PART_VQ_NOTIFY_CFG & - & VIRTIO_DEV_PART_DEV_FEATURES, VIRTIO_DEV_PART_DRV_FEATURES, VIRTIO_DEV_PART_PCI_COMMON_CFG,
                             VIRTIO_DEV_PART_DEVICE_STATUS, VIRTIO_DEV_PART_VQ_CFG \\
\hline
\hline
\end{tabularx}
\end{table}

\devicenormative{\subparagraph}{Device parts}{Basic Facilities of a Virtio Device / Device groups / Group administration commands / Device parts}

A device MUST either support all of, or none of
VIRTIO_ADMIN_CMD_DEV_PARTS_METADATA_GET,
VIRTIO_ADMIN_CMD_DEV_PARTS_GET, VIRTIO_ADMIN_CMD_DEV_PARTS_SET,
VIRTIO_ADMIN_CMD_RESOURCE_OBJ_CREATE,
VIRTIO_ADMIN_CMD_RESOURCE_OBJ_DESTROY, VIRTIO_ADMIN_CMD_RESOURCE_OBJ_MODIFY
VIRTIO_ADMIN_CMD_RESOURCE_OBJ_QUERY, and
VIRTIO_ADMIN_CMD_DEV_MODE_SET commands, where resource commands apply to
the resource object VIRTIO_RESOURCE_OBJ_DEV_PARTS.

The device MUST support getting the device parts multiple times
with the command VIRTIO_ADMIN_CMD_DEV_PARTS_GET.

When there are multiple device parts in the command
VIRTIO_ADMIN_CMD_DEV_PARTS_GET, the device MUST respond the device parts in the
same order as listed in the table
\nameref{table:Basic Facilities of a Virtio Device / Device groups / Group administration commands / Device parts / Device parts order/ Device parts order}.

The device SHOULD respond with an error status for the command
VIRTIO_ADMIN_CMD_DEV_PARTS_SET if the device is not stopped.

The device MUST support the command VIRTIO_ADMIN_CMD_DEV_PARTS_SET,
allowing the same or different device parts to be set multiple times.

The device MUST respond with an error for the command
VIRTIO_ADMIN_CMD_DEV_PARTS_SET, if there is a mismatch between the
device part length supplied in the VIRTIO_ADMIN_CMD_DEV_PARTS_SET
and the device part length in the device.

The device MUST NOT set the device part VIRTIO_DEV_PART_DEV_FEATURES in
the command VIRTIO_ADMIN_CMD_DEV_PARTS_SET; instead,
it must verify that the device features supplied in
VIRTIO_DEV_PART_DEV_FEATURES match those the device has.

The device may ignore the setting of a device part that has the
VIRTIO_DEV_PART_F_OPTIONAL bit set.

For the SR-IOV group type, when the device is stopped using the command
VIRTIO_ADMIN_CMD_DEV_MODE_SET,
\begin{itemize}
\item the device MUST not initiate any PCI transaction,
\item the device MUST finish all the outstanding PCI transactions before completing
      the command VIRTIO_ADMIN_CMD_DEV_MODE_SET,
\item the device MUST write any associated descriptors to the driver memory for
      all the observed buffers,
\item the device MUST accept driver notifications and the device MAY update any
      device parts,
\item the device MUST respond with valid values for PCI read requests,
\item the device MUST operate in the same way for the PCI architected interfaces
      regardless of the device mode.
\item the device MUST not generate any PCI PME.
\end{itemize}

When the device is stopped,
\begin{itemize}
\item the device MUST not access any virtqueue memory or any memory referred
      by the virtqueue.
\item the device MUST not generate any configuration change notification
      or any virtqueue notification.
\end{itemize}

For the SR-IOV group type,
\begin{itemize}
\item the device MUST respond to the commands
VIRTIO_ADMIN_CMD_DEV_MODE_SET, VIRTIO_ADMIN_CMD_DEV_PARTS_SET
after the member device completes FLR, if the FLR is in progress on the device
when the device receives any of these commands.

\item the member device MUST respond to the commands
VIRTIO_ADMIN_CMD_DEV_MODE_SET and VIRTIO_ADMIN_CMD_DEV_PARTS_SET
after the device reset completes in the device, if the
device reset is in progress when the device receives any of these commands.

\item the member device MUST respond to commands
VIRTIO_ADMIN_CMD_DEV_MODE_SET and VIRTIO_ADMIN_CMD_DEV_PARTS_SET
after the device power management state
transition completes on the device, if the power management state transition
is in progress when the device receives any of these commands.
\end{itemize}

When the \field{flags} is set to VIRTIO_ADMIN_CMD_DEV_MODE_FLAGS_STOPPED
in the command VIRTIO_ADMIN_CMD_DEV_MODE_SET, and if the device is already
stopped before, the device MUST complete the command successfully.

When the VIRTIO_ADMIN_CMD_DEV_MODE_FLAGS_STOPPED \field{flags} clear,
in the command VIRTIO_ADMIN_CMD_DEV_MODE_SET, and if the device is
not stopped before, the device MUST complete the command successfully.

For the SR-IOV group type, the device MUST clear all the device parts to
the default value when the member device is reset or undergo an PCI FLR.

The device MAY NOT respond to the selected device part in \field{hdr_list}
in the command VIRTIO_ADMIN_CMD_DEV_PARTS_GET if the device part is invalid
in the device.

For the commands VIRTIO_ADMIN_CMD_DEV_PARTS_GET and
VIRTIO_ADMIN_CMD_DEV_PARTS_METADATA_GET, when the device responds with:
\begin{itemize}
\item
VIRTIO_DEV_PART_DRV_FEATURES or VIRTIO_DEV_PART_PCI_COMMON_CFG, it MUST be
preceded by VIRTIO_DEV_PART_DEV_FEATURES.

\item VIRTIO_DEV_PART_PCI_COMMON_CFG, it MUST be preceded by
VIRTIO_DEV_PART_DEV_FEATURES.

\item VIRTIO_DEV_PART_PCI_COMMON_CFG, it MUST be preceded by
VIRTIO_DEV_PART_DEV_FEATURES and VIRTIO_DEV_PART_DRV_FEATURES.

\item VIRTIO_DEV_PART_DEV_CFG, it MUST be preceded by VIRTIO_DEV_PART_DEV_FEATURES.

\item VIRTIO_DEV_PART_DRV_CFG, it be preceded by VIRTIO_DEV_PART_DEV_FEATURES,
VIRTIO_DEV_PART_DRV_FEATURES and VIRTIO_DEV_PART_DEV_CFG.

\item VIRTIO_DEV_PART_DEVICE_STAtUS, it is preceded by VIRTIO_DEV_PART_DEV_FEATURES,
VIRTIO_DEV_PART_DRV_FEATURES, and VIRTIO_DEV_PART_DEV_CFG.
\end{itemize}

When the device receives a VIRTIO_ADMIN_CMD_DEV_PARTS_SET command containing the
parts VIRTIO_DEV_PART_DEV_FEATURES, VIRTIO_DEV_PART_PCI_COMMON_CFG and
VIRTIO_DEV_PART_DEV_CFG, the device SHOULD only verify that the provided configuration is
correct but SHOULD NOT apply it, especially for the fields that are designated
as read-only and invariant. This ensures that the device respects the
immutability of certain configuration aspects while still performing necessary
validation checks.

\drivernormative{\subparagraph}{Device parts}{Basic Facilities of a Virtio Device / Device groups / Group administration commands / Device parts}

The driver MUST set the mode to VIRTIO_ADMIN_CMD_DEV_MODE_F_STOPPED in
the command VIRTIO_ADMIN_CMD_DEV_MODE_SET before setting parts using the command
VIRTIO_ADMIN_CMD_DEV_PARTS_SET.

When there are multiple device parts in the command
VIRTIO_ADMIN_CMD_DEV_PARTS_SET, the driver MUST set the device parts in the same
order as listed in the table
\nameref{table:Basic Facilities of a Virtio Device / Device groups / Group administration commands / Device parts / Device parts order/ Device parts order}.

For the SR-IOV group type, the driver SHOULD NOT access the device configuration
space described in section
\ref{sec:Basic Facilities of a Virtio Device / Device Configuration Space}
when the device is stopped.

The driver SHOULD allocate sufficient response buffer to receive all the device
parts metadata in the command VIRTIO_ADMIN_CMD_DEV_PARTS_METADATA_GET.

The driver SHOULD allocate sufficient response buffer to receive all the device
parts in the command VIRTIO_ADMIN_CMD_DEV_PARTS_GET.


\devicenormative{\subsubsection}{Group administration commands}{Basic Facilities of a Virtio Device / Device groups / Group administration commands}

The device MUST validate \field{opcode}, \field{group_type} and
\field{group_member_id}, and if any of these has an invalid or
unsupported value, set \field{status} to
VIRTIO_ADMIN_STATUS_EINVAL and set \field{status_qualifier}
accordingly:
\begin{itemize}
\item if \field{group_type} is invalid, \field{status_qualifier}
	MUST be set to VIRTIO_ADMIN_STATUS_Q_INVALID_GROUP;
\item otherwise, if \field{opcode} is invalid,
	\field{status_qualifier} MUST be set to
	VIRTIO_ADMIN_STATUS_Q_INVALID_OPCODE;
\item otherwise, if \field{group_member_id} is used by the
	specific command and is invalid, \field{status_qualifier} MUST be
	set to VIRTIO_ADMIN_STATUS_Q_INVALID_MEMBER.
\end{itemize}

If a command completes successfully, the device MUST set
\field{status} to VIRTIO_ADMIN_STATUS_OK.

If a command fails, the device MUST set
\field{status} to a value different from VIRTIO_ADMIN_STATUS_OK.

If \field{status} is set to VIRTIO_ADMIN_STATUS_EINVAL, the
device state MUST NOT change, that is the command MUST NOT have
any side effects on the device, in particular the device MUST NOT
enter an error state as a result of this command.

If a command fails, the device state generally SHOULD NOT change,
as far as possible.

The device MAY enforce additional restrictions and dependencies on
opcodes used by the driver and MAY fail the command
VIRTIO_ADMIN_CMD_LIST_USE with \field{status} set to VIRTIO_ADMIN_STATUS_EINVAL
and \field{status_qualifier} set to VIRTIO_ADMIN_STATUS_Q_INVALID_FIELD
if the list of commands used violate internal device dependencies.

If the device supports multiple group types, commands for each group
type MUST operate independently of each other, in particular,
the device MAY return different results for VIRTIO_ADMIN_CMD_LIST_QUERY
for different group types.

After reset, if the device supports a given group type
and before receiving VIRTIO_ADMIN_CMD_LIST_USE for this group type
the device MUST assume
that the list of legal commands used by the driver consists of
the two commands VIRTIO_ADMIN_CMD_LIST_QUERY and VIRTIO_ADMIN_CMD_LIST_USE.

After completing VIRTIO_ADMIN_CMD_LIST_USE successfully,
the device MUST set the list of legal commands used by the driver
to the one supplied in \field{command_specific_data}.

The device MUST validate commands against the list used by
the driver and MUST fail any commands not in the list with
\field{status} set to VIRTIO_ADMIN_STATUS_EINVAL
and \field{status_qualifier} set to
VIRTIO_ADMIN_STATUS_Q_INVALID_OPCODE.

The list of supported commands reported by the device MUST NOT
shrink (but MAY expand): after reporting a given command as
supported through VIRTIO_ADMIN_CMD_LIST_QUERY the device MUST NOT
later report it as unsupported.  Further, after a given set of
commands has been used (via a successful
VIRTIO_ADMIN_CMD_LIST_USE), then after a device or system reset
the device SHOULD complete successfully any following calls to
VIRTIO_ADMIN_CMD_LIST_USE with the same list of commands; if this
command VIRTIO_ADMIN_CMD_LIST_USE fails after a device or system
reset, the device MUST not fail it solely because of the command
list used.  Failure to do so would interfere with resuming from
suspend and error recovery. Exceptions MAY apply if the system
configuration assures, in some way, that the driver does not
cache the previous value of VIRTIO_ADMIN_CMD_LIST_USE,
such as in the case of a firmware upgrade or downgrade.

When processing a command with the SR-IOV group type,
if the device does not have an SR-IOV Extended Capability or
if \field{VF Enable} is clear
then the device MUST fail all commands with
\field{status} set to VIRTIO_ADMIN_STATUS_EINVAL and
\field{status_qualifier} set to
VIRTIO_ADMIN_STATUS_Q_INVALID_GROUP;
otherwise, if \field{group_member_id} is not
between $1$ and \field{NumVFs} inclusive,
the device MUST fail all commands with
\field{status} set to VIRTIO_ADMIN_STATUS_EINVAL and
\field{status_qualifier} set to
VIRTIO_ADMIN_STATUS_Q_INVALID_MEMBER;
\field{NumVFs}, \field{VF Migration Capable}  and
\field{VF Enable} refer to registers within the SR-IOV Extended
Capability as specified by \hyperref[intro:PCIe]{[PCIe]}.

\drivernormative{\subsubsection}{Group administration commands}{Basic Facilities of a Virtio Device / Device groups / Group administration commands}

The driver MAY discover whether device supports a specific group type
by issuing VIRTIO_ADMIN_CMD_LIST_QUERY with the matching
\field{group_type}.

The driver MUST issue VIRTIO_ADMIN_CMD_LIST_USE
and wait for it to be completed with status
VIRTIO_ADMIN_STATUS_OK before issuing any commands
(except for the initial VIRTIO_ADMIN_CMD_LIST_QUERY
and VIRTIO_ADMIN_CMD_LIST_USE).

The driver MAY issue VIRTIO_ADMIN_CMD_LIST_USE any number
of times but MUST NOT issue VIRTIO_ADMIN_CMD_LIST_USE commands
if any other command has been submitted to the
device and has not yet completed processing by the device.

The driver SHOULD NOT set bits in device_admin_cmd_opcodes
if it is not familiar with how the command opcode
is used, as dependencies between command opcodes might exist.

The driver MUST NOT request (via VIRTIO_ADMIN_CMD_LIST_USE)
the use of any commands not previously reported as
supported for the same group type by VIRTIO_ADMIN_CMD_LIST_QUERY.

The driver MUST NOT use any commands for a given group type
before sending VIRTIO_ADMIN_CMD_LIST_USE with the correct
list of command opcodes and group type.

The driver MAY block use of VIRTIO_ADMIN_CMD_LIST_QUERY and
VIRTIO_ADMIN_CMD_LIST_USE by issuing VIRTIO_ADMIN_CMD_LIST_USE
with respective bits cleared in \field{command_specific_data}.

The driver MUST handle a command error with a reserved \field{status}
value in the same way as \field{status} set to VIRTIO_ADMIN_STATUS_EINVAL
(except possibly for different error reporting/diagnostic messages).

The driver MUST handle a command error with a reserved
\field{status_qualifier} value in the same way as
\field{status_qualifier} set to
VIRTIO_ADMIN_STATUS_Q_INVALID_COMMAND (except possibly for
different error reporting/diagnostic messages).

When sending commands with the SR-IOV group type,
the driver specify a value for \field{group_member_id}
between $1$ and \field{NumVFs} inclusive,
the driver MUST also make sure that as long as any such command
is outstanding, \field{VF Migration Capable} is clear and
\field{VF Enable} is set;
\field{NumVFs}, \field{VF Migration Capable}  and
\field{VF Enable} refer to registers within the SR-IOV Extended
Capability as specified by \hyperref[intro:PCIe]{[PCIe]}.

\section{Administration Virtqueues}\label{sec:Basic Facilities of a Virtio Device / Administration Virtqueues}

An administration virtqueue of an owner device is used to submit
group administration commands. An owner device can have more
than one administration virtqueue.

If VIRTIO_F_ADMIN_VQ has been negotiated, an owner device exposes one
or more administration virtqueues. The number and locations of the
administration virtqueues are exposed by the owner device in a transport
specific manner.

The driver enqueues requests to an arbitrary administration
virtqueue, and they are used by the device on that same
virtqueue. It is the responsibility of the driver to ensure
strict request ordering for commands, because they will be
consumed with no order constraints.  For example, if consistency
is required then the driver can wait for the processing of a
first command by the device to be completed before submitting
another command depending on the first one.

Administration virtqueues are used as follows:
\begin{itemize}
\item The driver submits the command using the \field{struct virtio_admin_cmd}
structure using a buffer consisting of two parts: a device-readable one followed by a
device-writable one.
\item the device-readable part includes fields from \field{opcode}
through \field{command_specific_data}.
\item the device-writeable buffer includes fields from \field{status}
through \field{command_specific_result} inclusive.
\end{itemize}

For each command, this specification describes a distinct
format structure used for \field{command_specific_data} and
\field{command_specific_result}, the length of these fields
depends on the command.

However, to ensure forward compatibility
\begin{itemize}
\item drivers are allowed to submit buffers that are longer
than the device expects
(that is, longer than the length of
\field{opcode} through \field{command_specific_data}).
This allows the driver to maintain
a single format structure even if some structure fields are
unused by the device.
\item drivers are allowed to submit buffers that are shorter
than what the device expects
(that is, shorter than the length of \field{status} through
\field{command_specific_result}). This allows the device to maintain
a single format structure even if some structure fields are
unused by the driver.
\end{itemize}

The device compares the length of each part (device-readable and
device-writeable) of the buffer as submitted by driver to what it
expects and then silently truncates the structures to either the
length submitted by the driver, or the length described in this
specification, whichever is shorter.  The device silently ignores
any data falling outside the shorter of the two lengths. Any
missing fields are interpreted as set to zero.

Similarly, the driver compares the used buffer length
of the buffer to what it expects and then silently
truncates the structure to the used buffer length.
The driver silently ignores any data falling outside
the used buffer length reported by the device.  Any missing
fields are interpreted as set to zero.

This simplifies driver and device implementations since the
driver/device can simply maintain a single large structure (such
as a C structure) for a command and its result. As new versions
of the specification are designed, new fields can be added to the
tail of a structure, with the driver/device using the full
structure without concern for versioning.

\devicenormative{\subsection}{Group administration commands}{Basic Facilities of a Virtio Device / Administration virtqueues}

The device MUST support device-readable and device-writeable buffers
shorter than described in this specification, by
\begin{enumerate}
\item acting as if any data that would be read outside the
device-readable buffers is set to zero, and
\item discarding data that would be written outside the
specified device-writeable buffers.
\end{enumerate}

The device MUST support device-readable and device-writeable buffers
longer than described in this specification, by
\begin{enumerate}
\item ignoring any data in device-readable buffers outside
the expected length, and
\item only writing the expected structure to the device-writeable
buffers, ignoring any extra buffers, and reporting the
actual length of data written, in bytes,
as buffer used length.
\end{enumerate}

The device SHOULD initialize the device-writeable buffer
up to the length of the structure described by this specification or
the length of the buffer supplied by the driver (even if the buffer is
all set to zero), whichever is shorter.

The device MUST NOT fail a command solely because the buffers
provided are shorter or longer than described in this
specification.

The device MUST initialize the device-writeable part of
\field{struct virtio_admin_cmd} that is a multiple of 64 bit in
size.

The device MUST initialize \field{status} and
\field{status_qualifier} in \field{struct virtio_admin_cmd}.

The device MUST process commands on a given administration virtqueue
in the order in which they are queued.

If multiple administration virtqueues have been configured,
device MAY process commands on distinct virtqueues with
no order constraints.

If the device sets \field{status} to either VIRTIO_ADMIN_STATUS_EAGAIN
or VIRTIO_ADMIN_STATUS_ENOMEM, then the command MUST NOT
have any side effects, making it safe to retry.

\drivernormative{\subsection}{Group administration commands}{Basic Facilities of a Virtio Device / Administration virtqueues}

The driver MAY supply device-readable or device-writeable parts
of \field{struct virtio_admin_cmd} that are longer than described in
this specification.

The driver SHOULD supply device-readable part of
\field{struct virtio_admin_cmd} that is at least as
large as the structure described by this specification
(even if the structure is all set to zero).

The driver MUST supply both device-readable or device-writeable parts
of \field{struct virtio_admin_cmd} that are a multiple of 64 bit
in length.

The device MUST supply both device-readable or device-writeable parts
of \field{struct virtio_admin_cmd} that are larger than zero in
length. However, \field{command_specific_data} and
\field{command_specific_result} MAY be zero in length, unless
specified otherwise for the command.

The driver MUST NOT assume that the device will initialize the whole
device-writeable part of \field{struct virtio_admin_cmd} as described in the specification; instead,
the driver MUST act as if the structure
outside the part of the buffer used by the device
is set to zero.

If multiple administration virtqueues have been configured,
the driver MUST ensure ordering for commands
placed on different administration virtqueues.

The driver SHOULD retry a command that completed with
\field{status} VIRTIO_ADMIN_STATUS_EAGAIN.
