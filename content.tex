\chapter{Basic Facilities of a Virtio Device}\label{sec:Basic Facilities of a Virtio Device}

A virtio device is discovered and identified by a bus-specific method
(see the bus specific sections: \ref{sec:Virtio Transport Options / Virtio Over PCI Bus}~\nameref{sec:Virtio Transport Options / Virtio Over PCI Bus},
\ref{sec:Virtio Transport Options / Virtio Over MMIO}~\nameref{sec:Virtio Transport Options / Virtio Over MMIO} and \ref{sec:Virtio Transport Options / Virtio Over Channel I/O}~\nameref{sec:Virtio Transport Options / Virtio Over Channel I/O}).  Each
device consists of the following parts:

\begin{itemize}
\item Device status field
\item Feature bits
\item Notifications
\item Device Configuration space
\item One or more virtqueues
\end{itemize}

\section{\field{Device Status} Field}\label{sec:Basic Facilities of a Virtio Device / Device Status Field}
During device initialization by a driver,
the driver follows the sequence of steps specified in
\ref{sec:General Initialization And Device Operation / Device
Initialization}.

The \field{device status} field provides a simple low-level
indication of the completed steps of this sequence.
It's most useful to imagine it hooked up to traffic
lights on the console indicating the status of each device.  The
following bits are defined (listed below in the order in which
they would be typically set):
\begin{description}
\item[ACKNOWLEDGE (1)] Indicates that the guest OS has found the
  device and recognized it as a valid virtio device.

\item[DRIVER (2)] Indicates that the guest OS knows how to drive the
  device.
  \begin{note}
    There could be a significant (or infinite) delay before setting
    this bit.  For example, under Linux, drivers can be loadable modules.
  \end{note}

\item[FAILED (128)] Indicates that something went wrong in the guest,
  and it has given up on the device. This could be an internal
  error, or the driver didn't like the device for some reason, or
  even a fatal error during device operation.

\item[FEATURES_OK (8)] Indicates that the driver has acknowledged all the
  features it understands, and feature negotiation is complete.

\item[DRIVER_OK (4)] Indicates that the driver is set up and ready to
  drive the device.

\item[DEVICE_NEEDS_RESET (64)] Indicates that the device has experienced
  an error from which it can't recover.
\end{description}

The \field{device status} field starts out as 0, and is reinitialized to 0 by
the device during reset.

\drivernormative{\subsection}{Device Status Field}{Basic Facilities of a Virtio Device / Device Status Field}
The driver MUST update \field{device status},
setting bits to indicate the completed steps of the driver
initialization sequence specified in
\ref{sec:General Initialization And Device Operation / Device
Initialization}.
The driver MUST NOT clear a
\field{device status} bit.  If the driver sets the FAILED bit,
the driver MUST later reset the device before attempting to re-initialize.

The driver SHOULD NOT rely on completion of operations of a
device if DEVICE_NEEDS_RESET is set.
\begin{note}
For example, the driver can't assume requests in flight will be
completed if DEVICE_NEEDS_RESET is set, nor can it assume that
they have not been completed.  A good implementation will try to
recover by issuing a reset.
\end{note}

\devicenormative{\subsection}{Device Status Field}{Basic Facilities of a Virtio Device / Device Status Field}

The device MUST NOT consume buffers or send any used buffer
notifications to the driver before DRIVER_OK.

\label{sec:Basic Facilities of a Virtio Device / Device Status Field / DEVICENEEDSRESET}The device SHOULD set DEVICE_NEEDS_RESET when it enters an error state
that a reset is needed.  If DRIVER_OK is set, after it sets DEVICE_NEEDS_RESET, the device
MUST send a device configuration change notification to the driver.

\section{Feature Bits}\label{sec:Basic Facilities of a Virtio Device / Feature Bits}

Each virtio device offers all the features it understands.  During
device initialization, the driver reads this and tells the device the
subset that it accepts.  The only way to renegotiate is to reset
the device.

This allows for forwards and backwards compatibility: if the device is
enhanced with a new feature bit, older drivers will not write that
feature bit back to the device.  Similarly, if a driver is enhanced with a feature
that the device doesn't support, it see the new feature is not offered.

Feature bits are allocated as follows:

\begin{description}
\item[0 to 23, and 50 to 127] Feature bits for the specific device type

\item[24 to 41] Feature bits reserved for extensions to the queue and
  feature negotiation mechanisms, see \ref{sec:Reserved Feature Bits}

\item[42 to 49, and 128 and above] Feature bits reserved for future extensions.
\end{description}

\begin{note}
For example, feature bit 0 for a network device (i.e.
Device ID 1) indicates that the device supports checksumming of
packets.
\end{note}

In particular, new fields in the device configuration space are
indicated by offering a new feature bit.

To keep the feature negotiation mechanism extensible, it is
important that devices \em{do not} offer any feature bits that
they would not be able to handle if the driver accepted them
(even though drivers are not supposed to accept any unspecified,
reserved, or unsupported features even if offered, according to
the specification.) Likewise, it is important that drivers \em{do
not} accept feature bits they do not know how to handle (even
though devices are not supposed to offer any unspecified,
reserved, or unsupported features in the first place,
according to the specification.) The preferred
way for handling reserved and unexpected features is that the
driver ignores them.

In particular, this is
especially important for features limited to specific transports,
as enabling these for more transports in future versions of the
specification is highly likely to require changing the behaviour
from drivers and devices.  Drivers and devices supporting
multiple transports need to carefully maintain per-transport
lists of allowed features.

\drivernormative{\subsection}{Feature Bits}{Basic Facilities of a Virtio Device / Feature Bits}
The driver MUST NOT accept a feature which the device did not offer,
and MUST NOT accept a feature which requires another feature which was
not accepted.

The driver MUST validate the feature bits offered by the device.
The driver MUST ignore and MUST NOT accept any feature bit that is
\begin{itemize}
\item not described in this specification,
\item marked as reserved,
\item not supported for the specific transport,
\item not defined for the device type.
\end{itemize}

The driver SHOULD go into backwards compatibility mode
if the device does not offer a feature it understands, otherwise MUST
set the FAILED \field{device status} bit and cease initialization.

By contrast, the driver MUST NOT fail solely because a feature
it does not understand has been offered by the device.

\devicenormative{\subsection}{Feature Bits}{Basic Facilities of a Virtio Device / Feature Bits}
The device MUST NOT offer a feature which requires another feature
which was not offered.  The device SHOULD accept any valid subset
of features the driver accepts, otherwise it MUST fail to set the
FEATURES_OK \field{device status} bit when the driver writes it.

The device MUST NOT offer feature bits corresponding to features
it would not support if accepted by the driver (even if the
driver is prohibited from accepting the feature bits by the
specification); for the sake of clarity, this refers to feature
bits not described in this specification, reserved feature bits
and feature bits reserved or not supported for the specific
transport or the specific device type, but this does not preclude
devices written to a future version of this specification from
offering such feature bits should such a specification have a
provision for devices to support the corresponding features.

If a device has successfully negotiated a set of features
at least once (by accepting the FEATURES_OK \field{device
status} bit during device initialization), then it SHOULD
NOT fail re-negotiation of the same set of features after
a device or system reset.  Failure to do so would interfere
with resuming from suspend and error recovery.

\subsection{Legacy Interface: A Note on Feature
Bits}\label{sec:Basic Facilities of a Virtio Device / Feature
Bits / Legacy Interface: A Note on Feature Bits}

Transitional Drivers MUST detect Legacy Devices by detecting that
the feature bit VIRTIO_F_VERSION_1 is not offered.
Transitional devices MUST detect Legacy drivers by detecting that
VIRTIO_F_VERSION_1 has not been acknowledged by the driver.

In this case device is used through the legacy interface.

Legacy interface support is OPTIONAL.
Thus, both transitional and non-transitional devices and
drivers are compliant with this specification.

Requirements pertaining to transitional devices and drivers
is contained in sections named 'Legacy Interface' like this one.

When device is used through the legacy interface, transitional
devices and transitional drivers MUST operate according to the
requirements documented within these legacy interface sections.
Specification text within these sections generally does not apply
to non-transitional devices.

\section{Notifications}\label{sec:Basic Facilities of a Virtio Device
/ Notifications}

The notion of sending a notification (driver to device or device
to driver) plays an important role in this specification. The
modus operandi of the notifications is transport specific.

There are three types of notifications: 
\begin{itemize}
\item configuration change notification
\item available buffer notification
\item used buffer notification. 
\end{itemize}

Configuration change notifications and used buffer notifications are sent
by the device, the recipient is the driver. A configuration change
notification indicates that the device configuration space has changed; a
used buffer notification indicates that a buffer may have been made used
on the virtqueue designated by the notification.

Available buffer notifications are sent by the driver, the recipient is
the device. This type of notification indicates that a buffer may have
been made available on the virtqueue designated by the notification.

The semantics, the transport-specific implementations, and other
important aspects of the different notifications are specified in detail
in the following chapters.

Most transports implement notifications sent by the device to the
driver using interrupts. Therefore, in previous versions of this
specification, these notifications were often called interrupts.
Some names defined in this specification still retain this interrupt
terminology. Occasionally, the term event is used to refer to
a notification or a receipt of a notification.

\section{Device Reset}\label{sec:Basic Facilities of a Virtio Device / Device Reset}

The driver may want to initiate a device reset at various times; notably,
it is required to do so during device initialization and device cleanup.

The mechanism used by the driver to initiate the reset is transport specific.

\devicenormative{\subsection}{Device Reset}{Basic Facilities of a Virtio Device / Device Reset}

A device MUST reinitialize \field{device status} to 0 after receiving a reset.

A device MUST NOT send notifications or interact with the queues after
indicating completion of the reset by reinitializing \field{device status}
to 0, until the driver re-initializes the device.

\drivernormative{\subsection}{Device Reset}{Basic Facilities of a Virtio Device / Device Reset}

The driver SHOULD consider a driver-initiated reset complete when it
reads \field{device status} as 0.

\section{Device Configuration Space}\label{sec:Basic Facilities of a Virtio Device / Device Configuration Space}

Device configuration space is generally used for rarely-changing or
initialization-time parameters.  Where configuration fields are
optional, their existence is indicated by feature bits: Future
versions of this specification will likely extend the device
configuration space by adding extra fields at the tail.

\begin{note}
The device configuration space uses the little-endian format
for multi-byte fields.
\end{note}

Each transport also provides a generation count for the device configuration
space, which will change whenever there is a possibility that two
accesses to the device configuration space can see different versions of that
space.

\drivernormative{\subsection}{Device Configuration Space}{Basic Facilities of a Virtio Device / Device Configuration Space}
Drivers MUST NOT assume reads from
fields greater than 32 bits wide are atomic, nor are reads from
multiple fields: drivers SHOULD read device configuration space fields like so:

\begin{lstlisting}
u32 before, after;
do {
        before = get_config_generation(device);
        // read config entry/entries.
        after = get_config_generation(device);
} while (after != before);
\end{lstlisting}

For optional configuration space fields, the driver MUST check that the
corresponding feature is offered before accessing that part of the configuration
space.
\begin{note}
See section \ref{sec:General Initialization And Device Operation / Device Initialization} for details on feature negotiation.
\end{note}

Drivers MUST
NOT limit structure size and device configuration space size.  Instead,
drivers SHOULD only check that device configuration space is {\em large enough} to
contain the fields necessary for device operation.

\begin{note}
For example, if the specification states that device configuration
space 'includes a single 8-bit field' drivers should understand this to mean that
the device configuration space might also include an arbitrary amount of
tail padding, and accept any device configuration space size equal to or
greater than the specified 8-bit size.
\end{note}

\devicenormative{\subsection}{Device Configuration Space}{Basic Facilities of a Virtio Device / Device Configuration Space}
The device MUST allow reading of any device-specific configuration
field before FEATURES_OK is set by the driver.  This includes fields which are
conditional on feature bits, as long as those feature bits are offered
by the device.

\subsection{Legacy Interface: A Note on Device Configuration Space endian-ness}\label{sec:Basic Facilities of a Virtio Device / Device Configuration Space / Legacy Interface: A Note on Configuration Space endian-ness}

Note that for legacy interfaces, device configuration space is generally the
guest's native endian, rather than PCI's little-endian.
The correct endian-ness is documented for each device.

\subsection{Legacy Interface: Device Configuration Space}\label{sec:Basic Facilities of a Virtio Device / Device Configuration Space / Legacy Interface: Device Configuration Space}

Legacy devices did not have a configuration generation field, thus are
susceptible to race conditions if configuration is updated.  This
affects the block \field{capacity} (see \ref{sec:Device Types /
Block Device / Device configuration layout}) and
network \field{mac} (see \ref{sec:Device Types / Network Device /
Device configuration layout}) fields;
when using the legacy interface, drivers SHOULD
read these fields multiple times until two reads generate a consistent
result.

\section{Virtqueues}\label{sec:Basic Facilities of a Virtio Device / Virtqueues}

The mechanism for bulk data transport on virtio devices is
pretentiously called a virtqueue. Each device can have zero or more
virtqueues\footnote{For example, the simplest network device has one virtqueue for
transmit and one for receive.}.

A virtio device can have maximum of 65536 virtqueues. Each virtqueue is
identified by a virtqueue index. A virtqueue index has a value in the
range of 0 to 65535.

Driver makes requests available to device by adding
an available buffer to the queue, i.e., adding a buffer
describing the request to a virtqueue, and optionally triggering
a driver event, i.e., sending an available buffer notification
to the device.

Device executes the requests and - when complete - adds
a used buffer to the queue, i.e., lets the driver
know by marking the buffer as used. Device can then trigger
a device event, i.e., send a used buffer notification to the driver.

Device reports the number of bytes it has written to memory for
each buffer it uses. This is referred to as ``used length''.

Device is not generally required to use buffers in
the same order in which they have been made available
by the driver.

Some devices always use descriptors in the same order in which
they have been made available. These devices can offer the
VIRTIO_F_IN_ORDER feature. If negotiated, this knowledge
might allow optimizations or simplify driver and/or device code.

Each virtqueue can consist of up to 3 parts:
\begin{itemize}
\item Descriptor Area - used for describing buffers
\item Driver Area - extra data supplied by driver to the device
\item Device Area - extra data supplied by device to driver
\end{itemize}

\begin{note}
Note that previous versions of this spec used different names for
these parts (following \ref{sec:Basic Facilities of a Virtio Device / Split Virtqueues}):
\begin{itemize}
\item Descriptor Table - for the Descriptor Area
\item Available Ring - for the Driver Area
\item Used Ring - for the Device Area
\end{itemize}

\end{note}

Two formats are supported: Split Virtqueues (see \ref{sec:Basic
Facilities of a Virtio Device / Split
Virtqueues}~\nameref{sec:Basic Facilities of a Virtio Device /
Split Virtqueues}) and Packed Virtqueues (see \ref{sec:Basic
Facilities of a Virtio Device / Packed
Virtqueues}~\nameref{sec:Basic Facilities of a Virtio Device /
Packed Virtqueues}).

Every driver and device supports either the Packed or the Split
Virtqueue format, or both.

\subsection{Virtqueue Reset}\label{sec:Basic Facilities of a Virtio Device / Virtqueues / Virtqueue Reset}

When VIRTIO_F_RING_RESET is negotiated, the driver can reset a virtqueue
individually. The way to reset the virtqueue is transport specific.

Virtqueue reset is divided into two parts. The driver first resets a queue and
can afterwards optionally re-enable it.

\subsubsection{Virtqueue Reset}\label{sec:Basic Facilities of a Virtio Device / Virtqueues / Virtqueue Reset / Virtqueue Reset}

\devicenormative{\paragraph}{Virtqueue Reset}{Basic Facilities of a Virtio Device / Virtqueues / Virtqueue Reset / Virtqueue Reset}

After a queue has been reset by the driver, the device MUST NOT execute
any requests from that virtqueue, or notify the driver for it.

The device MUST reset any state of a virtqueue to the default state,
including the available state and the used state.

\drivernormative{\paragraph}{Virtqueue Reset}{Basic Facilities of a Virtio Device / Virtqueues / Virtqueue Reset / Virtqueue Reset}

After the driver tells the device to reset a queue, the driver MUST verify that
the queue has actually been reset.

After the queue has been successfully reset, the driver MAY release any
resource associated with that virtqueue.

\subsubsection{Virtqueue Re-enable}\label{sec:Basic Facilities of a Virtio Device / Virtqueues / Virtqueue Reset / Virtqueue Re-enable}

This process is the same as the initialization process of a single queue during
the initialization of the entire device.

\devicenormative{\paragraph}{Virtqueue Re-enable}{Basic Facilities of a Virtio Device / Virtqueues / Virtqueue Reset / Virtqueue Re-enable}

The device MUST observe any queue configuration that may have been
changed by the driver, like the maximum queue size.

\drivernormative{\paragraph}{Virtqueue Re-enable}{Basic Facilities of a Virtio Device / Virtqueues / Virtqueue Reset / Virtqueue Re-enable}

When re-enabling a queue, the driver MUST configure the queue resources
as during initial virtqueue discovery, but optionally with different
parameters.

\input{split-ring.tex}

\input{packed-ring.tex}

\section{Driver Notifications} \label{sec:Basic Facilities of a Virtio Device / Driver notifications}
The driver is sometimes required to send an available buffer
notification to the device.

When VIRTIO_F_NOTIFICATION_DATA has not been negotiated,
this notification contains either a virtqueue index if
VIRTIO_F_NOTIF_CONFIG_DATA is not negotiated or device supplied virtqueue
notification config data if VIRTIO_F_NOTIF_CONFIG_DATA is negotiated.

The notification method and supplying any such virtqueue notification config data
is transport specific.

However, some devices benefit from the ability to find out the
amount of available data in the queue without accessing the virtqueue in memory:
for efficiency or as a debugging aid.

To help with these optimizations, when VIRTIO_F_NOTIFICATION_DATA
has been negotiated, driver notifications to the device include
the following information:

\begin{description}
\item [vq_index or vq_notif_config_data] Either virtqueue index or device
      supplied queue notification config data corresponding to a virtqueue.
\item [next_off] Offset
      within the ring where the next available ring entry
      will be written.
      When VIRTIO_F_RING_PACKED has not been negotiated this refers to the
      15 least significant bits of the available index.
      When VIRTIO_F_RING_PACKED has been negotiated this refers to the offset
      (in units of descriptor entries)
      within the descriptor ring where the next available
      descriptor will be written.
\item [next_wrap] Wrap Counter.
      With VIRTIO_F_RING_PACKED this is the wrap counter
      referring to the next available descriptor.
      Without VIRTIO_F_RING_PACKED this is the most significant bit
      (bit 15) of the available index.
\end{description}

Note that the driver can send multiple notifications even without
making any more buffers available. When VIRTIO_F_NOTIFICATION_DATA
has been negotiated, these notifications would then have
identical \field{next_off} and \field{next_wrap} values.

\input{shared-mem.tex}

\section{Exporting Objects}\label{sec:Basic Facilities of a Virtio Device / Exporting Objects}

When an object created by one virtio device needs to be
shared with a separate virtio device, the first device can
export the object by generating a UUID which can then
be passed to the second device to identify the object.

What constitutes an object, how to export objects, and
how to import objects are defined by the individual device
types. It is RECOMMENDED that devices generate version 4
UUIDs as specified by \hyperref[intro:rfc4122]{[RFC4122]}.

\section{Device groups}\label{sec:Basic Facilities of a Virtio Device / Device groups}

It is occasionally useful to have a device control a group of
other devices (the group may occasionally include the device
itself) within a group. The owner device itself is not a
member of the group (except in the special case of the self group).
Terminology used in such cases:

\begin{description}
\item[Device group]
        or just group, includes zero or more devices.
\item[Owner device]
        or owner, the device controlling the group.
\item[Member device]
        a device within a group. The owner device itself is not
	a member of the group except for the \field{Self group type}.
\item[Member identifier]
        each member has this identifier, unique within the group
	and used to address it through the owner device.
\item[Group type identifier]
	specifies what kind of member devices there are in a
	group, how the member identifier is interpreted
	and what kind of control the owner has.
	A given owner can control multiple groups
	of different types but only a single group of a given type,
	thus the type and the owner together identify the group.
	\footnote{Even though some group types only support
			specific transports, group type identifiers
			are global rather than transport-specific -
			a flood of new group types is not expected.}
\end{description}

\begin{note}
Each device only has a single driver, thus for the purposes of
this section, "the driver" is usually unambiguous and refers to
the driver of the owner device.  When there's ambiguity, "owner
driver" refers to the driver of the owner device, while "member
driver" refers to the driver of a member device.
\end{note}

The following group types, and their identifiers, are currently specified:
\begin{description}
\item[Self group type (0x0)]
This device group includes the owner device itself and no other devices.
The group type identifier for this group is 0x0.
The member identifier for this group has a value of 0x0.

\item[SR-IOV group type (0x1)]
This device group has a PCI Single Root I/O Virtualization
(SR-IOV) physical function (PF) device as the owner and includes
all its SR-IOV virtual functions (VFs) as members (see
\hyperref[intro:PCIe]{[PCIe]}).

The PF device itself is not a member of the group.

The group type identifier for this group is 0x1.

A member identifier for this group can have a value from 0x1 to
\field{NumVFs} as specified in the
SR-IOV Extended Capability of the owner device
and equals the SR-IOV VF number of the member device;
the group only exists when the \field{VF Enable} bit
in the SR-IOV Control Register within the
SR-IOV Extended Capability of the owner device is set
(see \hyperref[intro:PCIe]{[PCIe]}).

Both owner and member devices for this group type use the Virtio
PCI transport (see \ref{sec:Virtio Transport Options / Virtio Over PCI Bus}).
\end{description}

\subsection{Group administration commands}\label{sec:Basic Facilities of a Virtio Device / Device groups / Group administration commands}

The driver sends group administration commands to the owner device of
a group to control member devices of the group.
This mechanism can
be used, for example, to configure a member device before it is
initialized by its driver.
\footnote{The term "administration" is intended to be interpreted
widely to include any kind of control. See specific commands
for detail.}

All the group administration commands are of the following form:

\begin{lstlisting}
struct virtio_admin_cmd {
        /* Device-readable part */
        le16 opcode;
        /*
         * 0       - Self
         * 1       - SR-IOV
         * 2-65535 - reserved
         */
        le16 group_type;
        /* unused, reserved for future extensions */
        u8 reserved1[12];
        le64 group_member_id;
        le64 command_specific_data[];

        /* Device-writable part */
        le16 status;
        le16 status_qualifier;
        /* unused, reserved for future extensions */
        u8 reserved2[4];
        u8 command_specific_result[];
};
\end{lstlisting}

For all commands, \field{opcode}, \field{group_type} and if
necessary \field{group_member_id} and \field{command_specific_data} are
set by the driver, and the owner device sets \field{status} and if
needed \field{status_qualifier} and
\field{command_specific_result}.

Generally, any unused device-readable fields are set to zero by the driver
and ignored by the device.  Any unused device-writeable fields are set to zero
by the device and ignored by the driver.

\field{opcode} specifies the command. The valid
values for \field{opcode} can be found in the following table:

\begin{tabularx}{\textwidth}{ |l||l|X| }
\hline
opcode & Name & Command Description \\
\hline \hline
0x0000 & VIRTIO_ADMIN_CMD_LIST_QUERY & Provides to driver list of commands supported for this group type    \\
\hline
0x0001 & VIRTIO_ADMIN_CMD_LIST_USE & Provides to device list of commands used for this group type \\
\hline
0x0002 & \hyperref[par:Basic Facilities of a Virtio Device / Device groups / Group administration commands / Legacy Interface / VIRTIO-ADMIN-CMD-LEGACY-COMMON-CFG-WRITE]{VIRTIO_ADMIN_CMD_LEGACY_COMMON_CFG_WRITE} & Writes into the legacy common configuration structure \\
\hline
0x0003 & \hyperref[par:Basic Facilities of a Virtio Device / Device groups / Group administration commands / Legacy Interface / VIRTIO-ADMIN-CMD-LEGACY-COMMON-CFG-READ]{VIRTIO_ADMIN_CMD_LEGACY_COMMON_CFG_READ} & Reads from the legacy common configuration structure  \\
\hline
0x0004 & \hyperref[par:Basic Facilities of a Virtio Device / Device groups / Group administration commands / Legacy Interface / VIRTIO-ADMIN-CMD-LEGACY-DEV-CFG-WRITE]{VIRTIO_ADMIN_CMD_LEGACY_DEV_CFG_WRITE} & Writes into the legacy device configuration structure \\
\hline
0x0005 & \hyperref[par:Basic Facilities of a Virtio Device / Device groups / Group administration commands / Legacy Interface / VIRTIO-ADMIN-CMD-LEGACY-DEV-CFG-READ]{VIRTIO_ADMIN_CMD_LEGACY_DEV_CFG_READ} & Reads into the legacy device configuration structure \\
\hline
0x0006 & \hyperref[par:Basic Facilities of a Virtio Device / Device groups / Group administration commands / Legacy Interface / VIRTIO-ADMIN-CMD-LEGACY-NOTIFY-INFO]{VIRTIO_ADMIN_CMD_LEGACY_NOTIFY_INFO} & Query the notification region information \\
\hline
0x0007 & \hyperref[par:Basic Facilities of a Virtio Device / Device groups / Group administration commands / Device and driver capabilities / VIRTIO-ADMIN-CMD-CAP-ID-LIST-QUERY]{VIRTIO_ADMIN_CMD_CAP_ID_LIST_QUERY} & Query the supported device capabilities bitmap \\
\hline
0x0008 & \hyperref[par:Basic Facilities of a Virtio Device / Device groups / Group administration commands / Device and driver capabilities / VIRTIO-ADMIN-CMD-DEVICE-CAP-GET]{VIRTIO_ADMIN_CMD_DEVICE_CAP_GET} & Get the device capabilities \\
\hline
0x0009 & \hyperref[par:Basic Facilities of a Virtio Device / Device groups / Group administration commands / Device and driver capabilities / VIRTIO-ADMIN-CMD-DRIVER-CAP-SET]{VIRTIO_ADMIN_CMD_DRIVER_CAP_SET} & Set the driver capabilities \\
\hline
0x000a & \hyperref[par:Basic Facilities of a Virtio Device / Device groups / Group administration commands / Device resource objects / VIRTIO-ADMIN-CMD-RESOURCE-OBJ-CREATE]{VIRTIO_ADMIN_CMD_RESOURCE_OBJ_CREATE} & Create a device resource object \\
\hline
0x000c & \hyperref[par:Basic Facilities of a Virtio Device / Device groups / Group administration commands / Device resource objects / VIRTIO-ADMIN-CMD-RESOURCE-OBJ-MODIFY]{VIRTIO_ADMIN_CMD_RESOURCE_OBJ_MODIFY} & Modify a device resource object \\
\hline
0x000b & \hyperref[par:Basic Facilities of a Virtio Device / Device groups / Group administration commands / Device resource objects / VIRTIO-ADMIN-CMD-RESOURCE-OBJ-QUERY]{VIRTIO_ADMIN_CMD_RESOURCE_OBJ_QUERY} & Query a device resource object \\
\hline
0x000d & \hyperref[par:Basic Facilities of a Virtio Device / Device groups / Group administration commands / Device resource objects / VIRTIO-ADMIN-CMD-RESOURCE-OBJ-DESTROY]{VIRTIO_ADMIN_CMD_RESOURCE_OBJ_DESTROY} & Destroy a device resource object \\
\hline
0x000e & \hyperref[par:Basic Facilities of a Virtio Device / Device groups / Group administration commands / Device parts /  Device parts handling commands / VIRTIO-ADMIN-CMD-DEV-PARTS-METADATA-GET]{VIRTIO_ADMIN_CMD_DEV_PARTS_METADATA_GET} & Get the metadata of the device parts \\
\hline
0x000f & \hyperref[par:Basic Facilities of a Virtio Device / Device groups / Group administration commands / Device parts / Device parts handling commands / VIRTIO-ADMIN-CMD-DEV-PARTS-GET]{VIRTIO_ADMIN_CMD_DEV_PARTS_GET} & Get the device parts \\
\hline
0x0010 & \hyperref[par:Basic Facilities of a Virtio Device / Device groups / Group administration commands / Device parts / Device parts handling commands / VIRTIO-ADMIN-CMD-DEV-PARTS-SET]{VIRTIO_ADMIN_CMD_DEV_PARTS_SET} & Set the device parts \\
\hline
0x0011 & \hyperref[par:Basic Facilities of a Virtio Device / Device groups / Group administration commands / Device parts / Device parts handling commands / VIRTIO-ADMIN-CMD-DEV-MODE-SET]{VIRTIO_ADMIN_CMD_DEV_MODE_SET} & Stop or resume the device \\
\hline
\hline
0x0013 - 0x7FFF & - & Commands using \field{struct virtio_admin_cmd}    \\
\hline
0x8000 - 0xFFFF & - & Reserved for future commands (possibly using a different structure)    \\
\hline
\end{tabularx}

The \field{group_type} specifies the group type identifier.
The \field{group_member_id} specifies the member identifier within the group.
See section \ref{sec:Basic Facilities of a Virtio Device / Device groups}
for the definition of the group type identifier and group member
identifier.

The \field{status} describes the command result and possibly
failure reason at an abstract level, this is appropriate for
forwarding to applications. The \field{status_qualifier} describes
failures at a low virtio specific level, as appropriate for debugging.
The following table describes possible \field{status} values;
to simplify common implementations, they are intentionally
matching common \hyperref[intro:errno]{Linux error names and numbers}:

\begin{tabular}{|l|l|l|}
\hline
Status (decimal) & Name & Description \\
\hline \hline
00   & VIRTIO_ADMIN_STATUS_OK    & successful completion  \\
\hline
06   & VIRTIO_ADMIN_STATUS_ENXIO & no such capability or resource\\
\hline
11   & VIRTIO_ADMIN_STATUS_EAGAIN    & try again \\
\hline
12   & VIRTIO_ADMIN_STATUS_ENOMEM    & insufficient resources \\
\hline
16   & VIRTIO_ADMIN_STATUS_EBUSY     & device busy \\
\hline
22   & VIRTIO_ADMIN_STATUS_EINVAL    & invalid command \\
\hline
28   & VIRTIO_ADMIN_STATUS_ENOSPC    & resources exhausted on device \\
\hline
other   & -    & group administration command error  \\
\hline
\end{tabular}

When \field{status} is VIRTIO_ADMIN_STATUS_OK, \field{status_qualifier}
is reserved and set to zero by the device.

The following table describes possible \field{status_qualifier} values:

\begin{tabularx}{\textwidth}{ |l||l|X| }
\hline
Status & Name & Description \\
\hline \hline
0x00   & VIRTIO_ADMIN_STATUS_Q_OK               & used with VIRTIO_ADMIN_STATUS_OK  \\
\hline
0x01   & VIRTIO_ADMIN_STATUS_Q_INVALID_COMMAND  & command error: no additional information  \\
\hline
0x02   & VIRTIO_ADMIN_STATUS_Q_INVALID_OPCODE   & unsupported or invalid \field{opcode}  \\
\hline
0x03   & VIRTIO_ADMIN_STATUS_Q_INVALID_FIELD    & unsupported or invalid field within \field{command_specific_data}  \\
\hline
0x04   & VIRTIO_ADMIN_STATUS_Q_INVALID_GROUP    & unsupported or invalid \field{group_type} \\
\hline
0x05   & VIRTIO_ADMIN_STATUS_Q_INVALID_MEMBER   & unsupported or invalid \field{group_member_id} \\
\hline
0x06   & VIRTIO_ADMIN_STATUS_Q_NORESOURCE       & out of internal resources: ok to retry \\
\hline
0x07   & VIRTIO_ADMIN_STATUS_Q_TRYAGAIN         & command blocks for too long: should retry \\
\hline
0x08-0xFFFF   & -    & reserved for future use \\
\hline
\end{tabularx}

Each command uses a different \field{command_specific_data} and
\field{command_specific_result} structures and the length of
\field{command_specific_data} and \field{command_specific_result}
depends on these structures and is described separately or is
implicit in the structure description.

Before sending any group administration commands to the device, the driver
needs to communicate to the device which commands it is going to
use. Initially (after reset), only two commands are assumed to be used:
VIRTIO_ADMIN_CMD_LIST_QUERY and VIRTIO_ADMIN_CMD_LIST_USE.

Before sending any other commands for any member of a specific group to
the device, the driver queries the supported commands via
VIRTIO_ADMIN_CMD_LIST_QUERY and sends the commands it is
capable of using via VIRTIO_ADMIN_CMD_LIST_USE.

Commands VIRTIO_ADMIN_CMD_LIST_QUERY and
VIRTIO_ADMIN_CMD_LIST_USE
both use the following structure describing the
command opcodes:

\begin{lstlisting}
struct virtio_admin_cmd_list {
       /* Indicates which of the below fields were returned
       le64 device_admin_cmd_opcodes[];
};
\end{lstlisting}

This structure is an array of 64 bit values in little-endian byte
order, in which a bit is set if the specific command opcode
is supported. Thus, \field{device_admin_cmd_opcodes[0]} refers to the
first 64-bit value in this array corresponding to opcodes 0 to
63, \field{device_admin_cmd_opcodes[1]} is the second 64-bit value
corresponding to opcodes 64 to 127, etc.
For example, the array of size 2 including
the values 0x3 in \field{device_admin_cmd_opcodes[0]}
and 0x1 in \field{device_admin_cmd_opcodes[1]} indicates that only
opcodes 0, 1 and 64 are supported.
The length of the array depends on the supported opcodes - it is
large enough to include bits set for all supported opcodes,
that is the length can be calculated by starting with the largest
supported opcode adding one, dividing by 64 and rounding up.
In other words, for
VIRTIO_ADMIN_CMD_LIST_QUERY and VIRTIO_ADMIN_CMD_LIST_USE the
length of \field{command_specific_result} and
\field{command_specific_data} respectively will be
$DIV_ROUND_UP(max_cmd, 64) * 8$ where DIV_ROUND_UP is integer division
with round up and \field{max_cmd} is the largest available command opcode.

The array is also allowed to be larger and to additionally
include an arbitrary number of all-zero entries.

Accordingly, bits 0 and 1 corresponding to opcode 0
(VIRTIO_ADMIN_CMD_LIST_QUERY) and 1
(VIRTIO_ADMIN_CMD_LIST_USE) are
always set in \field{device_admin_cmd_opcodes[0]} returned by VIRTIO_ADMIN_CMD_LIST_QUERY.

For the command VIRTIO_ADMIN_CMD_LIST_QUERY, \field{opcode} is set to 0x0.
The \field{group_member_id} is unused. It is set to zero by driver.
This command has no command specific data.
The device, upon success, returns a result in
\field{command_specific_result} in the format
\field{struct virtio_admin_cmd_list} describing the
list of group administration commands supported for the group type
specified by \field{group_type}.

For the command VIRTIO_ADMIN_CMD_LIST_USE, \field{opcode}
is set to 0x1.
The \field{group_member_id} is unused. It is set to zero by driver.
The \field{command_specific_data} is in the format
\field{struct virtio_admin_cmd_list} describing the
list of group administration commands used by the driver
with the group type specified by \field{group_type}.

This command has no command specific result.

The driver issues the command VIRTIO_ADMIN_CMD_LIST_QUERY to
query the list of commands valid for this group and before sending
any commands for any member of a group.

The driver then enables use of some of the opcodes by sending to
the device the command VIRTIO_ADMIN_CMD_LIST_USE with a subset
of the list returned by VIRTIO_ADMIN_CMD_LIST_QUERY that is
both understood and used by the driver.

If the device supports the command list used by the driver, the
device completes the command with status VIRTIO_ADMIN_STATUS_OK.
If the device does not support the command list
(for example, if the driver is not capable to use
some required commands), the device
completes the command with status
VIRTIO_ADMIN_STATUS_INVALID_FIELD.

Note: the driver is assumed not to set bits in
device_admin_cmd_opcodes
if it is not familiar with how the command opcode
is used, since the device could have dependencies between
command opcodes.

It is assumed that all members in a group support and are used
with the same list of commands. However, for owner devices
supporting multiple group types, the list of supported commands
might differ between different group types.

\input{admin-cmds-legacy-interface.tex}
\input{admin-cmds-capabilities.tex}
\subsubsection{Device resource objects}\label{sec:Basic Facilities of a Virtio Device / Device groups / Group administration commands / Device resource objects}

Providing certain functionality consumes limited device resources such as
memory, processing units, buffer memory, or end-to-end credits. A device may
support multiple types of resource objects, each controlling different device
functionality. To manage this, virtio provides
\field{Device resource objects} that the driver can create, modify, and
destroy using administration commands with the self group type. Creating and
destroying a resource object consume and release device resources, respectively.
The device resource object query command returns the resource object as
maintained by the device.

For each resource type, the number of resource objects that can be created
is reported by the device as part of a device capability
\ref{sec:Basic Facilities of a Virtio Device / Device groups / Group administration commands / Device and driver capabilities}.
The driver reports the desired (same or lower) number of resource objects
as part of a driver capability \ref{sec:Basic Facilities of a Virtio Device / Device groups / Group administration commands / Device and driver capabilities}.
For each device object type, resource object limit is defined by field
\field{limit} using \field{Device and driver capabilities}.

\begin{lstlisting}
le32 limit; /* maximum resource id = limit - 1 */
\end{lstlisting}

Each resource object has a unique resource object ID - a driver-assigned number
in the range of 0 to \field{limit - 1}, where the \field{limit} is the maximum
number set by the driver for this resource object type. These resource IDs are unique within
each resource object type. The driver assigns the resource ID when creating a
device resource object. Once the resource object is successfully created,
subsequent resource modification, query, and destroy commands use this
resource object ID. No two resource objects share the same ID. Destroying a
resource object allows for the reuse of its ID for another resource object
of the same type.

A valid resource object id is \field{limit - 1}. For example, when a device
reports a \field{limit = 10} capability for a resource object, and drivers sets
\field{limit = 8}, the valid resource object id range for the device and the
driver is 0 to 7 for all the resource object commands. In this example,
the driver can only create 8 resource objects of a specified type.

A resource object of one type may depend on the resource object of another type.
Such dependency between resource objects is established by referring to the unique
resource ID in the administration commands. For example, a driver creates a
resource object identified by ID A of one type, then creates another resource
object identified by ID B of a different type, which depends on resource object
A. This dependency establishes the lifecycle of these resource objects. The driver
that creates the dependent resource object must destroy the resource objects in the
exact reverse order of their creation. In this example, the driver would
destroy resource object B before destroying resource object A.

Some resource object types are generic, common across multiple devices.
Others are specific for one device type.

\begin{tabular}{|l|l|}
\hline
Resource object type & Description \\
\hline \hline
0x000-0x1ff & Generic resource object type common across all devices \\
\hline
0x200-0x4ff & Device type specific resource object \\
\hline
0x500-0xffff & Reserved for future use  \\
\hline
\end{tabular}

Following generic resource objects are defined which are described separately.

\begin{xltabular}{\textwidth}{ |X||X|X| }
\hline
Resource object type & Name & Description \\
\hline \hline
0x000 & VIRTIO_RESOURCE_OBJ_DEV_PARTS & Device parts object, see \ref{par:Basic Facilities of a Virtio Device / Device groups / Group administration commands / Device parts / VIRTIO_RESOURCE_OBJ_DEV_PARTS} \\
\hline
0x001-0x1ff & - & Generic resource object range reserved \\
\hline
\hline
\end{xltabular}

When the device resets, it starts with zero resources of each type; the driver
can create resources up to the published \field{limit}. The driver can
destroy and recreate the resource one or multiple times. Upon device reset,
all resource objects created by the driver are destroyed within the device.

Following administration commands control device resource objects,
they are supported for the self group type, occasionally some resource
objects can be created for the SR-IOV group type as well. Such sr-iov group
type specific resource objects are listed where such objects is defined.

\begin{enumerate}
\item VIRTIO_ADMIN_CMD_RESOURCE_OBJ_CREATE
\item VIRTIO_ADMIN_CMD_RESOURCE_OBJ_MODIFY
\item VIRTIO_ADMIN_CMD_RESOURCE_OBJ_QUERY
\item VIRTIO_ADMIN_CMD_RESOURCE_OBJ_DESTROY
\end{enumerate}

Each resource object administration command uses a common header
\field{struct virtio_admin_cmd_resource_obj_cmd_hdr}.

\begin{lstlisting}
struct virtio_admin_cmd_resource_obj_cmd_hdr {
        le16 type;
        u8 reserved[2];
        le32 id; /* Indicates unique resource object id per resource object type */
};

\end{lstlisting}

\field{type} refers to the device resource object type.
\field{id} uniquely identifies the resource object of a specified \field{type}.

\paragraph{VIRTIO_ADMIN_CMD_RESOURCE_OBJ_CREATE}
\label{par:Basic Facilities of a Virtio Device / Device groups / Group administration commands / Device resource objects / VIRTIO_ADMIN_CMD_RESOURCE_OBJ_CREATE}

This command creates the specified resource object of \field{type} identified by the
resource id \field{id}. The valid range of \field{id} is defined by the
device in the related device capability. The driver assigns the unique \field{id}
for the resource for the specified \field{type}.

For the command VIRTIO_ADMIN_CMD_RESOURCE_OBJ_CREATE, \field{opcode} is set to 0xa.
\field{group_member_id} is set to zero for self-group type and set to
the member device to be accessed for the SR-IOV group type.
The \field{command_specific_data} is in the format
\field{struct virtio_admin_cmd_resource_obj_create_data}.
\field{resource_obj_specific_data} refers to the resource object specific data.
Each resource uses a different \field{resource_obj_specific_data} and is described
separately.

\field{flags} is reserved for future extension for optional resource object attributes and
is set to 0. Each resource object uses a different value for
\field{flags} and it is described separately.

\begin{lstlisting}
struct virtio_admin_cmd_resource_obj_create_data {
        struct virtio_admin_cmd_resource_obj_cmd_hdr hdr;
        le64 flags;
        u8 resource_obj_specific_data[];
};
\end{lstlisting}

When the command completes successfully, the resource object is created by the
device and the device can immediately begin using it.
This command has no command specific result.

\paragraph{VIRTIO_ADMIN_CMD_RESOURCE_OBJ_MODIFY}
\label{par:Basic Facilities of a Virtio Device / Device groups / Group administration commands / Device resource objects / VIRTIO_ADMIN_CMD_RESOURCE_OBJ_MODIFY}

This command modifies the attributes of an existing device resource object.
For the command VIRTIO_ADMIN_CMD_RESOURCE_OBJ_MODIFY, \field{opcode} is set to 0xb.
The \field{command_specific_data} is in the format
\field{struct virtio_admin_cmd_resource_modify_data}.
\field{group_member_id} is set to zero for self-group type and set to
the member device to be accessed for the SR-IOV group type.
\field{id} identifies the resource object of type \field{type} whose attributes
to modify.
This command modifies the attributes supplied in \field{resource_obj_specific_data}.

\begin{lstlisting}
struct virtio_admin_cmd_resource_modify_data {
        struct virtio_admin_cmd_resource_obj_cmd_hdr hdr;
        le64 flags;
        u8 resource_obj_specific_data[];
};
\end{lstlisting}

This command has no command specific result.
When the command completes successfully, attributes of the resource object is
set to the values supplied in \field{resource_obj_specific_data}.

\paragraph{VIRTIO_ADMIN_CMD_RESOURCE_OBJ_QUERY}
\label{par:Basic Facilities of a Virtio Device / Device groups / Group administration commands / Device resource objects / VIRTIO_ADMIN_CMD_RESOURCE_OBJ_QUERY}

This command queries attributes of the existing resource object.
For the command VIRTIO_ADMIN_CMD_RESOURCE_OBJ_QUERY, \field{opcode} is set to 0xc.
\field{group_member_id} is set to zero for self-group type and set to
the member device to be accessed for the SR-IOV group type.
The \field{command_specific_data} is in the format
\field{struct virtio_admin_cmd_resource_obj_query_data}.
\field{id} identifies the existing resource object of type \field{type} whose
attributes to query.

\begin{lstlisting}
struct virtio_admin_cmd_resource_obj_query_data {
        struct virtio_admin_cmd_resource_obj_cmd_hdr hdr;
        le64 flags;
};
\end{lstlisting}

\begin{lstlisting}
struct virtio_admin_cmd_resource_obj_query_result {
        u8 resource_obj_specific_result[];
};
\end{lstlisting}

\field{command_specific_result} is in the format
\field{virtio_admin_cmd_resource_obj_query_result}.

When the command completes successfully, the attributes of the specified
resource object are are set in \field{resource_obj_specific_data}.

\paragraph{VIRTIO_ADMIN_CMD_RESOURCE_OBJ_DESTROY}
\label{par:Basic Facilities of a Virtio Device / Device groups / Group administration commands / Device resource objects / VIRTIO_ADMIN_CMD_RESOURCE_OBJ_DESTROY}

This command destroys the previously created device resource object.
For the command VIRTIO_ADMIN_CMD_RESOURCE_OBJ_DESTROY, \field{opcode} is set to 0xd.
The \field{command_specific_data} is in the format
\field{struct virtio_admin_cmd_resource_obj_cmd_hdr}.
\field{group_member_id} is set to zero for self-group type and set to
the member device to be accessed for the SR-IOV group type.
\field{id} identifies the existing resource object of type \field{type}.

This command destroys the specified resource object of \field{type} identified
by \field{id}, which is previously created using
VIRTIO_ADMIN_CMD_RESOURCE_OBJ_CREATE command.

This command has no command specific result.
When the command completes successfully, the resource object is destroyed from the device.

\devicenormative{\paragraph}{Device resource objects}{Basic Facilities of a Virtio Device / Device groups / Group administration commands / Device resource objects}

The device SHOULD complete the command VIRTIO_ADMIN_CMD_RESOURCE_OBJ_CREATE
with \field{status} set to VIRTIO_ADMIN_STATUS_EEXIST if a resource object already exists
with supplied resource \field{id} for the specified \field{type}.

The device SHOULD complete the commands VIRTIO_ADMIN_CMD_RESOURCE_OBJ_MODIFY,
VIRTIO_ADMIN_CMD_RESOURCE_QUERY and
VIRTIO_ADMIN_CMD_RESOURCE_OBJ_DESTROY with \field{status} set to
VIRTIO_ADMIN_STATUS_ENXIO if the specified resource object does not exist.

The device SHOULD set \field{status} to VIRTIO_ADMIN_STATUS_ENOSPC for the
command VIRTIO_ADMIN_CMD_RESOURCE_OBJ_CREATE if the device fail to create the
resource object.

The device SHOULD complete the commands VIRTIO_ADMIN_CMD_RESOURCE_OBJ_MODIFY or
VIRTIO_ADMIN_CMD_RESOURCE_OBJ_DESTROY commands with \field{status} set to
VIRTIO_ADMIN_STATUS_EBUSY if other resource objects depend on the resource object
being modified or destroyed.

The device MUST allow recreating the resource object using the command
VIRTIO_ADMIN_CMD_RESOURCE_OBJ_CREATE which was previously
destroyed using the command VIRTIO_ADMIN_CMD_RESOURCE_OBJ_DESTROY respectively
without undergoing a device reset.

The device SHOULD allow creating the resource object using
the command VIRTIO_ADMIN_CMD_RESOURCE_OBJ_CREATE with any resource
id as long as the resource object is not created.

The device MAY fail the command VIRTIO_ADMIN_CMD_RESOURCE_OBJ_CREATE even if the
resources within the device have not reached up to the \field{max_limit}
but the device MAY have reached an internal limit.

When a capability represents a number of resource objects, the device SHOULD
allow creating as many resource objects as represented by the driver capability.

The device MUST NOT have any side effects on the resource object when the command
VIRTIO_ADMIN_CMD_RESOURCE_OBJ_MODIFY fails.

The device MUST complete the command VIRTIO_ADMIN_CMD_RESOURCE_OBJ_QUERY
with \field{resource_obj_specific_data} which is matching the
\field{resource_obj_specific_data} of last VIRTIO_ADMIN_CMD_RESOURCE_OBJ_CREATE
or VIRTIO_ADMIN_CMD_RESOURCE_OBJ_MODIFY command.

On device reset, the device MUST destroy all the resource objects which
have been created.

\drivernormative{\paragraph}{Device resource objects}{Basic Facilities of a Virtio Device / Device groups / Group administration commands / Device resource objects}

The driver MUST not create a second resource object of the same type with same
ID using command VIRTIO_ADMIN_CMD_RESOURCE_OBJ_CREATE before destroying the
previously created resource object.

The driver MUST NOT create more resource objects of a specified \field{type} using
command VIRTIO_ADMIN_CMD_RESOURCE_OBJ_CREATE than the maximum limit set by the
driver capability.

The driver SHOULD NOT modify, query and destroy the resource object which is
already destroyed previously by the driver.

The driver SHOULD NOT destroy the resource object on which other resource objects
are depending; the driver SHOULD destroy all the resource objects which do not depend
on other resource objects.

The driver MUST NOT set the capability related to the resource objects if the
resource objects have been created using the command VIRTIO_ADMIN_CMD_RESOURCE_OBJ_CREATE
and not yet destroyed.

The driver MUST send the command VIRTIO_ADMIN_CMD_DRIVER_CAP_SET before using
any resources related to such capability.

\subsubsection{Device parts}\label{sec:Basic Facilities of a Virtio Device / Device groups / Group administration commands / Device parts}

In some systems, there is a need to capture the state of all or part of
a device and subsequently restore either the same device or a
different one to this captured state. A group owner device can support
administration commands to facilitate these get and set operations for
the group member devices.

For example, a hypervisor can use the administration commands to
capture parts of the device state and save the result as part of
a VM snapshot. Later, the hypervisor can retrieve the snapshot and use the
administration commands to restore parts of a device to resume
VM operation.

As another example, these commands can be used to facilitate VM migration by the
hypervisor: one (source) hypervisor can get parts of a device and send
the results to another (destination) hypervisor, which will in turn
set (restore) parts of (another) device to resume the VM operation on
the destination.

The device comprises many device parts which the driver can get and set.
Administration commands are provided to either get and set all the device
parts at once, or to get the device parts metadata that indicates which
device parts are present, and later to get and set specific device parts.
To get and set the device parts or their metadata, the driver first creates a
device parts resource object, indicating whether the object should
handle get or set operations but not both simultaneously. The device and the
driver indicate the device parts resource objects' limit using the capability
VIRTIO_DEV_PARTS_CAP.

The device can be stopped to prevent device parts from changing.
When the device is stopped, it does not initiate any transport requests.
For instance, the device refrains from sending any configuration or
virtqueue notifications and does not access any virtqueues or the driver's
buffer memory. While the driver may remain active and continue to send
notifications to the device, potentially updating some device parts,
the device itself will not initiate any transport requests.

\paragraph{VIRTIO_DEV_PARTS_CAP}
\label{par:Basic Facilities of a Virtio Device / Device groups / Group administration commands / Device parts / VIRTIO-DEV-PARTS-CAP}

The capability VIRTIO_DEV_PARTS_CAP indicates the device parts resource objects limit.
\field{cap_specific_data} is in the format \field{struct virtio_dev_parts_cap}.

\begin{lstlisting}
struct virtio_dev_parts_cap {
        u8 get_parts_resource_objects_limit;
        u8 set_parts_resource_objects_limit;
};
\end{lstlisting}

\field{get_parts_resource_objects_limit} indicates the supported device parts
resource objects for retrieving the device parts.
\field{set_parts_resource_objects_limit} indicates the supported device parts
resource objects for restoring the device parts.

\paragraph{VIRTIO_RESOURCE_OBJ_DEV_PARTS}\label{par:Basic Facilities of a Virtio Device / Device groups / Group administration commands / Device parts / VIRTIO-RESOURCE-OBJ-DEV-PARTS}

A device parts resource object is used to either get or set the device parts.
Before performing any get or set operation for the device parts, the driver
creates the device parts resource object
VIRTIO_RESOURCE_OBJ_DEV_PARTS using the administration command
\nameref{par:Basic Facilities of a Virtio Device / Device groups / Group administration commands / Device resource objects / VIRTIO-ADMIN-CMD-RESOURCE-OBJ-CREATE}.
The driver indicates the intended purpose (get or set) at the time of creating the
device parts resource object.
For the device parts resource object, both \field{resource_obj_specific_data} and
\field{resource_obj_specific_result} are in the format
\field{struct virtio_resource_obj_dev_parts}.

\begin{lstlisting}
struct virtio_resource_obj_dev_parts {
        u8 type;
#define VIRTIO_RESOURCE_OBJ_DEV_PARTS_TYPE_GET 0
#define VIRTIO_RESOURCE_OBJ_DEV_PARTS_TYPE_SET 1
        u8 reserved[7];
};
\end{lstlisting}

When \field{type} is set to VIRTIO_RESOURCE_OBJ_DEV_PARTS_TYPE_GET,
the driver can use the object to capture the device parts and the metadata of
these device parts. When \field{type} is set to
VIRTIO_RESOURCE_OBJ_DEV_PARTS_TYPE_SET, the driver can use the
object to restore the device parts.

\paragraph{Device parts handling commands}\label{par:Basic Facilities of a Virtio Device / Device groups / Group administration commands / Device parts / Device parts handling commands}

The owner driver uses the following resource object handling administration
commands. These commands are only used for the device parts resource
object after the driver creates the VIRTIO_RESOURCE_OBJ_DEV_PARTS object.
These commands are currently only defined for the SR-IOV group type:

\begin{enumerate}
\item VIRTIO_ADMIN_CMD_DEV_PARTS_METADATA_GET
\item VIRTIO_ADMIN_CMD_DEV_PARTS_GET
\item VIRTIO_ADMIN_CMD_DEV_PARTS_SET
\end{enumerate}

\subparagraph{VIRTIO_ADMIN_CMD_DEV_PARTS_METADATA_GET}
\label{par:Basic Facilities of a Virtio Device / Device groups / Group administration commands / Device parts / Device parts handling commands / VIRTIO-ADMIN-CMD-DEV-PARTS-METADATA-GET}

This command obtains the metadata of the device parts. This metadata includes
the maximum size of the device parts, the count of device parts, and a list of
the device part headers.

For the command VIRTIO_ADMIN_CMD_DEV_PARTS_METADATA_GET, \field{opcode} is set
to 0xe. The \field{command_specific_data} is in the format
\field{struct virtio_admin_cmd_dev_parts_metadata_data}.
\field{group_member_id} refers to the member device to be accessed.
The resource object \field{type} in the \field{hdr} is set to
VIRTIO_RESOURCE_OBJ_DEV_PARTS and \field{id} is set to the ID of the
device parts resource object.

\begin{lstlisting}
struct virtio_admin_cmd_dev_parts_metadata_data {
        struct virtio_admin_cmd_resource_obj_cmd_hdr hdr;
        u8 type;
        u8 reserved[7];
};

#define VIRTIO_ADMIN_CMD_DEV_PARTS_METADATA_TYPE_SIZE 0
#define VIRTIO_ADMIN_CMD_DEV_PARTS_METADATA_TYPE_COUNT 1
#define VIRTIO_ADMIN_CMD_DEV_PARTS_METADATA_TYPE_LIST  2

struct virtio_admin_cmd_dev_parts_metadata_result {
        union {
                struct {
                        le32 size;
                        le32 reserved;
                } parts_size;
                struct {
                        le32 count;
                        le32 reserved;
                } hdr_list_count;
                struct {
                        le32 count;
                        le32 reserved;
                        struct virtio_dev_part_hdr hdrs[];
                } hdr_list;
        };
};
\end{lstlisting}

When the command completes successfully, the
\field{command_specific_result} is in the format
\field{struct virtio_admin_cmd_dev_parts_metadata_result}.

When \field{type} is set to VIRTIO_ADMIN_CMD_DEV_PARTS_METADATA_TYPE_SIZE,
the device responds with \field{parts_size}. \field{parts_size.size} indicates
the maximum size in bytes for all the device parts.

When \field{type} is set to VIRTIO_ADMIN_CMD_DEV_PARTS_METADATA_TYPE_COUNT, the
device responds with \field{hdr_list_count.count}. The
\field{hdr_list_count.count} indicates an count of
\field{struct virtio_dev_part_hdr} metadata entries that the device can
provide when the \field{type} is set to VIRTIO_ADMIN_CMD_DEV_PARTS_METADATA_TYPE_LIST
in a subsequent VIRTIO_ADMIN_CMD_DEV_PARTS_METADATA_GET command.

When \field{type} is set to VIRTIO_ADMIN_CMD_DEV_PARTS_METADATA_TYPE_LIST,
the device responds with \field{hdr_list}. \field{hdr_list}
indicates the device parts metadata.

\field{reserved} is reserved and set to 0.

The command responds with the \field{status} VIRTIO_ADMIN_STATUS_ENOMEM
when the size of \field{command_specific_result} is not sufficient enough
for the response.

\subparagraph{VIRTIO_ADMIN_CMD_DEV_PARTS_GET}
\label{par:Basic Facilities of a Virtio Device / Device groups / Group administration commands / Device parts /  Device parts handling commands / VIRTIO-ADMIN-CMD-DEV-PARTS-GET}

This command captures the device parts. For the command
VIRTIO_ADMIN_CMD_DEV_PARTS_GET, \field{opcode} is set to 0xf.
The \field{command_specific_data} is in the format
\field{struct virtio_admin_cmd_dev_parts_get_data}.
\field{group_member_id} refers to the member device to be accessed.
The resource object \field{type} in the \field{hdr} is set to
VIRTIO_RESOURCE_OBJ_DEV_PARTS and \field{id} is set to the ID of the
device parts resource object.

\begin{lstlisting}
struct virtio_admin_cmd_dev_parts_get_data {
        struct virtio_admin_cmd_resource_obj_cmd_hdr hdr;
        u8 type;
        u8 reserved[7];
        struct virtio_dev_part_hdr hdr_list[];
};

#define VIRTIO_ADMIN_CMD_DEV_PARTS_GET_TYPE_SELECTED 0
#define VIRTIO_ADMIN_CMD_DEV_PARTS_GET_TYPE_ALL 1

struct virtio_admin_cmd_dev_parts_get_result {
        struct virtio_dev_part parts[];
};

\end{lstlisting}

When the driver wants to capture specific device parts, \field{type} is set to
VIRTIO_ADMIN_CMD_DEV_PARTS_GET_TYPE_SELECTED and \field{hdr_list} is set to the
device parts of interest.

When the driver wants to retrieve all the device parts, \field{type} is set to
VIRTIO_ADMIN_CMD_DEV_PARTS_GET_TYPE_ALL, and \field{hdr_list} is empty.

\field{reserved} is reserved and set to 0.

When the command completes successfully, the \field{command_specific_result} is
in the format \field{struct virtio_admin_cmd_dev_parts_get_result}, containing
either the selected device parts or all the device parts.

If the requested device part does not exist, the device skips the device part
without any error.

\subparagraph{VIRTIO_ADMIN_CMD_DEV_PARTS_SET}\label{par:Basic Facilities of a Virtio Device / Device groups / Group administration commands / Device parts / Device parts handling commands / VIRTIO-ADMIN-CMD-DEV-PARTS-SET}

This command sets one or multiple device parts. For the command
VIRTIO_ADMIN_CMD_DEV_PARTS_SET, \field{opcode} is set to 0x10.
The \field{group_member_id} refers to the member device to be accessed.
The resource object \field{type} in the \field{hdr} is set to
VIRTIO_RESOURCE_OBJ_DEV_PARTS and \field{id} is set to the ID of the
device parts resource object.

\begin{lstlisting}
struct virtio_admin_cmd_dev_parts_set_data {
        struct virtio_admin_cmd_resource_obj_cmd_hdr hdr;
        struct virtio_dev_part parts[];
};
\end{lstlisting}

The \field{command_specific_data} is in the format
\field{struct virtio_admin_cmd_dev_parts_set_data}.

This command has no command specific result.

The driver stops the device before setting any device parts.

When the command completes successfully, the device has updated device
parts to the value supplied in \field{virtio_admin_cmd_dev_parts_set_data}.

The device parts set by this command take effect when the device is resumed
using the VIRTIO_ADMIN_CMD_DEV_MODE_SET command.

When the command fails with a status other than VIRTIO_ADMIN_STATUS_OK, the
device does not have any side effects.

\subparagraph{VIRTIO_ADMIN_CMD_DEV_MODE_SET}\label{par:Basic Facilities of a Virtio Device / Device groups / Group administration commands / Device parts / Device parts handling commands / VIRTIO-ADMIN-CMD-DEV-MODE-SET}

This command either stops the device from initiating any transport requests or
resumes the device operation. For the command VIRTIO_ADMIN_CMD_DEV_MODE_SET,
\field{opcode} is set to 0x11. \field{group_member_id} indicates the member
device to be accessed.

The \field{command_specific_data} is in the format
\field{struct virtio_admin_cmd_dev_mode_set_data}.

\begin{lstlisting}
struct virtio_admin_cmd_dev_mode_set_data {
        u8 flags;
};

#define VIRTIO_ADMIN_CMD_DEV_MODE_F_STOPPED 0
\end{lstlisting}

This command has no command specific result.

When the command completes successfully and if the \field{flags} field is set
to VIRTIO_ADMIN_CMD_DEV_MODE_F_STOPPED (bit 0), the device is stopped.
When the device is stopped, the device stops initiating all transport
communications, which includes:

\begin{enumerate}
\item stopping configuration change notifications
\item stopping all virtqueue notifications
\item stops accessing all virtqueues and the driver buffer memory
\end{enumerate}

After the device is stopped, the device parts remain unchanged unless
the driver initiates any transport requests.

When the device is stopped, it writes back any associated descriptors for all
observed buffers to prevent out-of-order processing if the device is resumed.

When the command completes successfully and if the \field{flags} field
is set to zero, the device resumes its operation. If the command completes
with an error, it does not produce any side effects on the device.

\paragraph{Device parts order}\label{par:Basic Facilities of a Virtio Device / Device groups / Group administration commands / Device parts / Device parts order}

Device parts are usually captured and restored using get and set administration
commands respectively; when multiple device parts are captured or restored,
they are arranged in the specific order listed:

Some of the device parts do not need to be written to the device when restored, such
device parts are listed as \field{O}. When a such an optional device part is
exchanged using \field{struct virtio_dev_part}, it is marked as optional by
setting VIRTIO_DEV_PART_F_OPTIONAL(bit 0) in the \field{flags}.

\begin{table}[H]
\caption{Device parts order}
\label{table:Basic Facilities of a Virtio Device / Device groups / Group administration commands / Device parts / Device parts order/ Device parts order}
\begin{tabularx}{\textwidth}{ |l|l|X| }
\hline
Part name & Optional & Mandatory preceding parts \\
\hline \hline
\hline
VIRTIO_DEV_PART_DEV_FEATURES & O & Nil \\
\hline
VIRTIO_DEV_PART_DRV_FEATURES & - & Nil \\
\hline
VIRTIO_DEV_PART_PCI_COMMON_CFG & - & VIRTIO_DEV_PART_DEV_FEATURES, VIRTIO_DEV_PART_DRV_FEATURES \\
\hline
VIRTIO_DEV_PART_DEVICE_STATUS & - & VIRTIO_DEV_PART_DEV_FEATURES, VIRTIO_DEV_PART_DRV_FEATURES, VIRTIO_DEV_PART_PCI_COMMON_CFG \\
\hline
VIRTIO_DEV_PART_VQ_CFG & - & VIRTIO_DEV_PART_DEV_FEATURES, VIRTIO_DEV_PART_DRV_FEATURES, VIRTIO_DEV_PART_PCI_COMMON_CFG,
                             VIRTIO_DEV_PART_DEVICE_STATUS \\
\hline
VIRTIO_DEV_PART_VQ_NOTIFY_CFG & - & VIRTIO_DEV_PART_DEV_FEATURES, VIRTIO_DEV_PART_DRV_FEATURES, VIRTIO_DEV_PART_PCI_COMMON_CFG,
                             VIRTIO_DEV_PART_DEVICE_STATUS, VIRTIO_DEV_PART_VQ_CFG \\
\hline
\hline
\end{tabularx}
\end{table}

\devicenormative{\subparagraph}{Device parts}{Basic Facilities of a Virtio Device / Device groups / Group administration commands / Device parts}

A device MUST either support all of, or none of
VIRTIO_ADMIN_CMD_DEV_PARTS_METADATA_GET,
VIRTIO_ADMIN_CMD_DEV_PARTS_GET, VIRTIO_ADMIN_CMD_DEV_PARTS_SET,
VIRTIO_ADMIN_CMD_RESOURCE_OBJ_CREATE,
VIRTIO_ADMIN_CMD_RESOURCE_OBJ_DESTROY, VIRTIO_ADMIN_CMD_RESOURCE_OBJ_MODIFY
VIRTIO_ADMIN_CMD_RESOURCE_OBJ_QUERY, and
VIRTIO_ADMIN_CMD_DEV_MODE_SET commands, where resource commands apply to
the resource object VIRTIO_RESOURCE_OBJ_DEV_PARTS.

The device MUST support getting the device parts multiple times
with the command VIRTIO_ADMIN_CMD_DEV_PARTS_GET.

When there are multiple device parts in the command
VIRTIO_ADMIN_CMD_DEV_PARTS_GET, the device MUST respond the device parts in the
same order as listed in the table
\nameref{table:Basic Facilities of a Virtio Device / Device groups / Group administration commands / Device parts / Device parts order/ Device parts order}.

The device SHOULD respond with an error status for the command
VIRTIO_ADMIN_CMD_DEV_PARTS_SET if the device is not stopped.

The device MUST support the command VIRTIO_ADMIN_CMD_DEV_PARTS_SET,
allowing the same or different device parts to be set multiple times.

The device MUST respond with an error for the command
VIRTIO_ADMIN_CMD_DEV_PARTS_SET, if there is a mismatch between the
device part length supplied in the VIRTIO_ADMIN_CMD_DEV_PARTS_SET
and the device part length in the device.

The device MUST NOT set the device part VIRTIO_DEV_PART_DEV_FEATURES in
the command VIRTIO_ADMIN_CMD_DEV_PARTS_SET; instead,
it must verify that the device features supplied in
VIRTIO_DEV_PART_DEV_FEATURES match those the device has.

The device may ignore the setting of a device part that has the
VIRTIO_DEV_PART_F_OPTIONAL bit set.

For the SR-IOV group type, when the device is stopped using the command
VIRTIO_ADMIN_CMD_DEV_MODE_SET,
\begin{itemize}
\item the device MUST not initiate any PCI transaction,
\item the device MUST finish all the outstanding PCI transactions before completing
      the command VIRTIO_ADMIN_CMD_DEV_MODE_SET,
\item the device MUST write any associated descriptors to the driver memory for
      all the observed buffers,
\item the device MUST accept driver notifications and the device MAY update any
      device parts,
\item the device MUST respond with valid values for PCI read requests,
\item the device MUST operate in the same way for the PCI architected interfaces
      regardless of the device mode.
\item the device MUST not generate any PCI PME.
\end{itemize}

When the device is stopped,
\begin{itemize}
\item the device MUST not access any virtqueue memory or any memory referred
      by the virtqueue.
\item the device MUST not generate any configuration change notification
      or any virtqueue notification.
\end{itemize}

For the SR-IOV group type,
\begin{itemize}
\item the device MUST respond to the commands
VIRTIO_ADMIN_CMD_DEV_MODE_SET, VIRTIO_ADMIN_CMD_DEV_PARTS_SET
after the member device completes FLR, if the FLR is in progress on the device
when the device receives any of these commands.

\item the member device MUST respond to the commands
VIRTIO_ADMIN_CMD_DEV_MODE_SET and VIRTIO_ADMIN_CMD_DEV_PARTS_SET
after the device reset completes in the device, if the
device reset is in progress when the device receives any of these commands.

\item the member device MUST respond to commands
VIRTIO_ADMIN_CMD_DEV_MODE_SET and VIRTIO_ADMIN_CMD_DEV_PARTS_SET
after the device power management state
transition completes on the device, if the power management state transition
is in progress when the device receives any of these commands.
\end{itemize}

When the \field{flags} is set to VIRTIO_ADMIN_CMD_DEV_MODE_FLAGS_STOPPED
in the command VIRTIO_ADMIN_CMD_DEV_MODE_SET, and if the device is already
stopped before, the device MUST complete the command successfully.

When the VIRTIO_ADMIN_CMD_DEV_MODE_FLAGS_STOPPED \field{flags} clear,
in the command VIRTIO_ADMIN_CMD_DEV_MODE_SET, and if the device is
not stopped before, the device MUST complete the command successfully.

For the SR-IOV group type, the device MUST clear all the device parts to
the default value when the member device is reset or undergo an PCI FLR.

The device MAY NOT respond to the selected device part in \field{hdr_list}
in the command VIRTIO_ADMIN_CMD_DEV_PARTS_GET if the device part is invalid
in the device.

For the commands VIRTIO_ADMIN_CMD_DEV_PARTS_GET and
VIRTIO_ADMIN_CMD_DEV_PARTS_METADATA_GET, when the device responds with:
\begin{itemize}
\item
VIRTIO_DEV_PART_DRV_FEATURES or VIRTIO_DEV_PART_PCI_COMMON_CFG, it MUST be
preceded by VIRTIO_DEV_PART_DEV_FEATURES.

\item VIRTIO_DEV_PART_PCI_COMMON_CFG, it MUST be preceded by
VIRTIO_DEV_PART_DEV_FEATURES.

\item VIRTIO_DEV_PART_PCI_COMMON_CFG, it MUST be preceded by
VIRTIO_DEV_PART_DEV_FEATURES and VIRTIO_DEV_PART_DRV_FEATURES.

\item VIRTIO_DEV_PART_DEV_CFG, it MUST be preceded by VIRTIO_DEV_PART_DEV_FEATURES.

\item VIRTIO_DEV_PART_DRV_CFG, it be preceded by VIRTIO_DEV_PART_DEV_FEATURES,
VIRTIO_DEV_PART_DRV_FEATURES and VIRTIO_DEV_PART_DEV_CFG.

\item VIRTIO_DEV_PART_DEVICE_STAtUS, it is preceded by VIRTIO_DEV_PART_DEV_FEATURES,
VIRTIO_DEV_PART_DRV_FEATURES, and VIRTIO_DEV_PART_DEV_CFG.
\end{itemize}

When the device receives a VIRTIO_ADMIN_CMD_DEV_PARTS_SET command containing the
parts VIRTIO_DEV_PART_DEV_FEATURES, VIRTIO_DEV_PART_PCI_COMMON_CFG and
VIRTIO_DEV_PART_DEV_CFG, the device SHOULD only verify that the provided configuration is
correct but SHOULD NOT apply it, especially for the fields that are designated
as read-only and invariant. This ensures that the device respects the
immutability of certain configuration aspects while still performing necessary
validation checks.

\drivernormative{\subparagraph}{Device parts}{Basic Facilities of a Virtio Device / Device groups / Group administration commands / Device parts}

The driver MUST set the mode to VIRTIO_ADMIN_CMD_DEV_MODE_F_STOPPED in
the command VIRTIO_ADMIN_CMD_DEV_MODE_SET before setting parts using the command
VIRTIO_ADMIN_CMD_DEV_PARTS_SET.

When there are multiple device parts in the command
VIRTIO_ADMIN_CMD_DEV_PARTS_SET, the driver MUST set the device parts in the same
order as listed in the table
\nameref{table:Basic Facilities of a Virtio Device / Device groups / Group administration commands / Device parts / Device parts order/ Device parts order}.

For the SR-IOV group type, the driver SHOULD NOT access the device configuration
space described in section
\ref{sec:Basic Facilities of a Virtio Device / Device Configuration Space}
when the device is stopped.

The driver SHOULD allocate sufficient response buffer to receive all the device
parts metadata in the command VIRTIO_ADMIN_CMD_DEV_PARTS_METADATA_GET.

The driver SHOULD allocate sufficient response buffer to receive all the device
parts in the command VIRTIO_ADMIN_CMD_DEV_PARTS_GET.


\devicenormative{\subsubsection}{Group administration commands}{Basic Facilities of a Virtio Device / Device groups / Group administration commands}

The device MUST validate \field{opcode}, \field{group_type} and
\field{group_member_id}, and if any of these has an invalid or
unsupported value, set \field{status} to
VIRTIO_ADMIN_STATUS_EINVAL and set \field{status_qualifier}
accordingly:
\begin{itemize}
\item if \field{group_type} is invalid, \field{status_qualifier}
	MUST be set to VIRTIO_ADMIN_STATUS_Q_INVALID_GROUP;
\item otherwise, if \field{opcode} is invalid,
	\field{status_qualifier} MUST be set to
	VIRTIO_ADMIN_STATUS_Q_INVALID_OPCODE;
\item otherwise, if \field{group_member_id} is used by the
	specific command and is invalid, \field{status_qualifier} MUST be
	set to VIRTIO_ADMIN_STATUS_Q_INVALID_MEMBER.
\end{itemize}

If a command completes successfully, the device MUST set
\field{status} to VIRTIO_ADMIN_STATUS_OK.

If a command fails, the device MUST set
\field{status} to a value different from VIRTIO_ADMIN_STATUS_OK.

If \field{status} is set to VIRTIO_ADMIN_STATUS_EINVAL, the
device state MUST NOT change, that is the command MUST NOT have
any side effects on the device, in particular the device MUST NOT
enter an error state as a result of this command.

If a command fails, the device state generally SHOULD NOT change,
as far as possible.

The device MAY enforce additional restrictions and dependencies on
opcodes used by the driver and MAY fail the command
VIRTIO_ADMIN_CMD_LIST_USE with \field{status} set to VIRTIO_ADMIN_STATUS_EINVAL
and \field{status_qualifier} set to VIRTIO_ADMIN_STATUS_Q_INVALID_FIELD
if the list of commands used violate internal device dependencies.

If the device supports multiple group types, commands for each group
type MUST operate independently of each other, in particular,
the device MAY return different results for VIRTIO_ADMIN_CMD_LIST_QUERY
for different group types.

After reset, if the device supports a given group type
and before receiving VIRTIO_ADMIN_CMD_LIST_USE for this group type
the device MUST assume
that the list of legal commands used by the driver consists of
the two commands VIRTIO_ADMIN_CMD_LIST_QUERY and VIRTIO_ADMIN_CMD_LIST_USE.

After completing VIRTIO_ADMIN_CMD_LIST_USE successfully,
the device MUST set the list of legal commands used by the driver
to the one supplied in \field{command_specific_data}.

The device MUST validate commands against the list used by
the driver and MUST fail any commands not in the list with
\field{status} set to VIRTIO_ADMIN_STATUS_EINVAL
and \field{status_qualifier} set to
VIRTIO_ADMIN_STATUS_Q_INVALID_OPCODE.

The list of supported commands reported by the device MUST NOT
shrink (but MAY expand): after reporting a given command as
supported through VIRTIO_ADMIN_CMD_LIST_QUERY the device MUST NOT
later report it as unsupported.  Further, after a given set of
commands has been used (via a successful
VIRTIO_ADMIN_CMD_LIST_USE), then after a device or system reset
the device SHOULD complete successfully any following calls to
VIRTIO_ADMIN_CMD_LIST_USE with the same list of commands; if this
command VIRTIO_ADMIN_CMD_LIST_USE fails after a device or system
reset, the device MUST not fail it solely because of the command
list used.  Failure to do so would interfere with resuming from
suspend and error recovery. Exceptions MAY apply if the system
configuration assures, in some way, that the driver does not
cache the previous value of VIRTIO_ADMIN_CMD_LIST_USE,
such as in the case of a firmware upgrade or downgrade.

When processing a command with the SR-IOV group type,
if the device does not have an SR-IOV Extended Capability or
if \field{VF Enable} is clear
then the device MUST fail all commands with
\field{status} set to VIRTIO_ADMIN_STATUS_EINVAL and
\field{status_qualifier} set to
VIRTIO_ADMIN_STATUS_Q_INVALID_GROUP;
otherwise, if \field{group_member_id} is not
between $1$ and \field{NumVFs} inclusive,
the device MUST fail all commands with
\field{status} set to VIRTIO_ADMIN_STATUS_EINVAL and
\field{status_qualifier} set to
VIRTIO_ADMIN_STATUS_Q_INVALID_MEMBER;
\field{NumVFs}, \field{VF Migration Capable}  and
\field{VF Enable} refer to registers within the SR-IOV Extended
Capability as specified by \hyperref[intro:PCIe]{[PCIe]}.

\drivernormative{\subsubsection}{Group administration commands}{Basic Facilities of a Virtio Device / Device groups / Group administration commands}

The driver MAY discover whether device supports a specific group type
by issuing VIRTIO_ADMIN_CMD_LIST_QUERY with the matching
\field{group_type}.

The driver MUST issue VIRTIO_ADMIN_CMD_LIST_USE
and wait for it to be completed with status
VIRTIO_ADMIN_STATUS_OK before issuing any commands
(except for the initial VIRTIO_ADMIN_CMD_LIST_QUERY
and VIRTIO_ADMIN_CMD_LIST_USE).

The driver MAY issue VIRTIO_ADMIN_CMD_LIST_USE any number
of times but MUST NOT issue VIRTIO_ADMIN_CMD_LIST_USE commands
if any other command has been submitted to the
device and has not yet completed processing by the device.

The driver SHOULD NOT set bits in device_admin_cmd_opcodes
if it is not familiar with how the command opcode
is used, as dependencies between command opcodes might exist.

The driver MUST NOT request (via VIRTIO_ADMIN_CMD_LIST_USE)
the use of any commands not previously reported as
supported for the same group type by VIRTIO_ADMIN_CMD_LIST_QUERY.

The driver MUST NOT use any commands for a given group type
before sending VIRTIO_ADMIN_CMD_LIST_USE with the correct
list of command opcodes and group type.

The driver MAY block use of VIRTIO_ADMIN_CMD_LIST_QUERY and
VIRTIO_ADMIN_CMD_LIST_USE by issuing VIRTIO_ADMIN_CMD_LIST_USE
with respective bits cleared in \field{command_specific_data}.

The driver MUST handle a command error with a reserved \field{status}
value in the same way as \field{status} set to VIRTIO_ADMIN_STATUS_EINVAL
(except possibly for different error reporting/diagnostic messages).

The driver MUST handle a command error with a reserved
\field{status_qualifier} value in the same way as
\field{status_qualifier} set to
VIRTIO_ADMIN_STATUS_Q_INVALID_COMMAND (except possibly for
different error reporting/diagnostic messages).

When sending commands with the SR-IOV group type,
the driver specify a value for \field{group_member_id}
between $1$ and \field{NumVFs} inclusive,
the driver MUST also make sure that as long as any such command
is outstanding, \field{VF Migration Capable} is clear and
\field{VF Enable} is set;
\field{NumVFs}, \field{VF Migration Capable}  and
\field{VF Enable} refer to registers within the SR-IOV Extended
Capability as specified by \hyperref[intro:PCIe]{[PCIe]}.

\section{Administration Virtqueues}\label{sec:Basic Facilities of a Virtio Device / Administration Virtqueues}

An administration virtqueue of an owner device is used to submit
group administration commands. An owner device can have more
than one administration virtqueue.

If VIRTIO_F_ADMIN_VQ has been negotiated, an owner device exposes one
or more administration virtqueues. The number and locations of the
administration virtqueues are exposed by the owner device in a transport
specific manner.

The driver enqueues requests to an arbitrary administration
virtqueue, and they are used by the device on that same
virtqueue. It is the responsibility of the driver to ensure
strict request ordering for commands, because they will be
consumed with no order constraints.  For example, if consistency
is required then the driver can wait for the processing of a
first command by the device to be completed before submitting
another command depending on the first one.

Administration virtqueues are used as follows:
\begin{itemize}
\item The driver submits the command using the \field{struct virtio_admin_cmd}
structure using a buffer consisting of two parts: a device-readable one followed by a
device-writable one.
\item the device-readable part includes fields from \field{opcode}
through \field{command_specific_data}.
\item the device-writeable buffer includes fields from \field{status}
through \field{command_specific_result} inclusive.
\end{itemize}

For each command, this specification describes a distinct
format structure used for \field{command_specific_data} and
\field{command_specific_result}, the length of these fields
depends on the command.

However, to ensure forward compatibility
\begin{itemize}
\item drivers are allowed to submit buffers that are longer
than the device expects
(that is, longer than the length of
\field{opcode} through \field{command_specific_data}).
This allows the driver to maintain
a single format structure even if some structure fields are
unused by the device.
\item drivers are allowed to submit buffers that are shorter
than what the device expects
(that is, shorter than the length of \field{status} through
\field{command_specific_result}). This allows the device to maintain
a single format structure even if some structure fields are
unused by the driver.
\end{itemize}

The device compares the length of each part (device-readable and
device-writeable) of the buffer as submitted by driver to what it
expects and then silently truncates the structures to either the
length submitted by the driver, or the length described in this
specification, whichever is shorter.  The device silently ignores
any data falling outside the shorter of the two lengths. Any
missing fields are interpreted as set to zero.

Similarly, the driver compares the used buffer length
of the buffer to what it expects and then silently
truncates the structure to the used buffer length.
The driver silently ignores any data falling outside
the used buffer length reported by the device.  Any missing
fields are interpreted as set to zero.

This simplifies driver and device implementations since the
driver/device can simply maintain a single large structure (such
as a C structure) for a command and its result. As new versions
of the specification are designed, new fields can be added to the
tail of a structure, with the driver/device using the full
structure without concern for versioning.

\devicenormative{\subsection}{Group administration commands}{Basic Facilities of a Virtio Device / Administration virtqueues}

The device MUST support device-readable and device-writeable buffers
shorter than described in this specification, by
\begin{enumerate}
\item acting as if any data that would be read outside the
device-readable buffers is set to zero, and
\item discarding data that would be written outside the
specified device-writeable buffers.
\end{enumerate}

The device MUST support device-readable and device-writeable buffers
longer than described in this specification, by
\begin{enumerate}
\item ignoring any data in device-readable buffers outside
the expected length, and
\item only writing the expected structure to the device-writeable
buffers, ignoring any extra buffers, and reporting the
actual length of data written, in bytes,
as buffer used length.
\end{enumerate}

The device SHOULD initialize the device-writeable buffer
up to the length of the structure described by this specification or
the length of the buffer supplied by the driver (even if the buffer is
all set to zero), whichever is shorter.

The device MUST NOT fail a command solely because the buffers
provided are shorter or longer than described in this
specification.

The device MUST initialize the device-writeable part of
\field{struct virtio_admin_cmd} that is a multiple of 64 bit in
size.

The device MUST initialize \field{status} and
\field{status_qualifier} in \field{struct virtio_admin_cmd}.

The device MUST process commands on a given administration virtqueue
in the order in which they are queued.

If multiple administration virtqueues have been configured,
device MAY process commands on distinct virtqueues with
no order constraints.

If the device sets \field{status} to either VIRTIO_ADMIN_STATUS_EAGAIN
or VIRTIO_ADMIN_STATUS_ENOMEM, then the command MUST NOT
have any side effects, making it safe to retry.

\drivernormative{\subsection}{Group administration commands}{Basic Facilities of a Virtio Device / Administration virtqueues}

The driver MAY supply device-readable or device-writeable parts
of \field{struct virtio_admin_cmd} that are longer than described in
this specification.

The driver SHOULD supply device-readable part of
\field{struct virtio_admin_cmd} that is at least as
large as the structure described by this specification
(even if the structure is all set to zero).

The driver MUST supply both device-readable or device-writeable parts
of \field{struct virtio_admin_cmd} that are a multiple of 64 bit
in length.

The device MUST supply both device-readable or device-writeable parts
of \field{struct virtio_admin_cmd} that are larger than zero in
length. However, \field{command_specific_data} and
\field{command_specific_result} MAY be zero in length, unless
specified otherwise for the command.

The driver MUST NOT assume that the device will initialize the whole
device-writeable part of \field{struct virtio_admin_cmd} as described in the specification; instead,
the driver MUST act as if the structure
outside the part of the buffer used by the device
is set to zero.

If multiple administration virtqueues have been configured,
the driver MUST ensure ordering for commands
placed on different administration virtqueues.

The driver SHOULD retry a command that completed with
\field{status} VIRTIO_ADMIN_STATUS_EAGAIN.

\section{Device parts}\label{sec:Basic Facilities of a Virtio Device / Device parts}

Device parts represent the device state, with parts for basic
device facilities such as driver features, as well as transport specific
and device type specific parts. In memory, each device part consists
of a header \field{struct virtio_dev_part_hdr} followed by
the device part data in \field{value}. The driver can get and set
these device parts using administration commands.

\begin{lstlisting}
struct virtio_dev_part_hdr {
        le16 part_type;
        u8 flags;
        u8 reserved;
        union {
                struct {
                        le32 offset;
                        le32 reserved;
                } pci_common_cfg;
                struct  {
                        le16 index;
                        u8 reserved[6];
                } vq_index;
        } selector;
        le32 length;
};

#define VIRTIO_DEV_PART_F_OPTIONAL 0

struct virtio_dev_part {
        struct virtio_dev_part_hdr hdr;
        u8 value[];
};

\end{lstlisting}

Each device part consists of a fixed size \field{hdr} followed by optional
part data in field \field{value}. The device parts are divided into
two categories and identified by \field{part_type}. The common device parts are
independent of the device type and, are in the range \field{0x0000 - 0x01FF}. Common
device parts are listed in
\ref{table:Basic Facilities of a Virtio Device / Device parts / Common device parts}
The device parts in the range \field{0x0200 - 0x05FF} are specific to a device type
such as a network or console device.
The device part is identified by the \field{part_type} field as listed:

\begin{description}
\item[0x0000 - 0x01FF] - common part - used to describe a part of the device that
                         is independent of the device type
\item[0x0200 - 0x05FF] - device type specific part - used to indicate parts
                         that are device type specific
\item[0x0600 - 0xFFFF] - reserved
\end{description}

Some device parts are optional, the device can function without them.
For example, such parts can help improve performance, with the device working
slower, yet still correctly, even without the parts. In another example,
optional parts can be used for validation, with the device being able to deduce
the part itself, the part being helpful to detect driver or user errors.
Such device parts are marked optional by setting bit 0
(VIRTIO_DEV_PART_F_OPTIONAL) in the \field{flags}.

\field{reserved} is reserved and set to zero.

\field{length} indicates the length of the \field{value} in bytes. The length
of the device part depends on the device part itself and is described separately.
The device part data is in \field{value} and is \field{part_type} specific.

\field{selector} further specifies the part. It is only used for some
\field{part_type} values.

\field{selector.pci_common_cfg.offset} is the offset of the
field in the \nameref{sec:Virtio Transport Options / Virtio Over PCI Bus / PCI Device Layout / Common configuration structure layout}. It is valid only when the \field{part_type} is set to VIRTIO_DEV_PART_PCI_COMMON_CFG,
otherwise it is reserved and set to 0.

\field{selector.vq_index.index} is the index of the virtqueue. It is valid
only when the \field{part_type} is VIRTIO_DEV_PART_VQ_CFG or
VIRTIO_DEV_PART_VQ_CFG.

\subsection{Common device parts}\label{sec:Basic Facilities of a Virtio Device / Device parts / Common device parts}

Common parts are independent of the device type.
\field{part_type} and \field{value} for each part are documented as follows:

\begin{table}
\caption{Common device parts}
\label{table:Basic Facilities of a Virtio Device / Device parts / Common device parts}
\begin{xltabular}{\textwidth}{ |l||l|X| }
\hline
Type & Name & Description \\
\hline \hline
0x100 & VIRTIO_DEV_PART_DEV_FEATURES & Device features, see \ref{sec:Basic Facilities of a Virtio Device / Device parts / Common device parts / VIRTIO_DEV_PART_DEV_FEATURES} \\
\hline
0x101 & VIRTIO_DEV_PART_DRV_FEATURES & Driver features, \ref{sec:Basic Facilities of a Virtio Device / Device parts / Common device parts / VIRTIO_DEV_PART_DRV_FEATURES} \\
\hline
0x102 & VIRTIO_DEV_PART_PCI_COMMON_CFG & PCI common configuration, see \ref{sec:Basic Facilities of a Virtio Device / Device parts / Common device parts / VIRTIO_DEV_PART_PCI_COMMON_CFG} \\
\hline
0x103 & VIRTIO_DEV_PART_DEVICE_STATUS & Device status, see \ref{sec:Basic Facilities of a Virtio Device / Device parts / Common device parts / VIRTIO_DEV_PART_DEVICE_STATUS} \\
\hline
0x104 & VIRTIO_DEV_PART_VQ_CFG & Virtqueue configuration, see \ref{sec:Basic Facilities of a Virtio Device / Device parts / Common device parts / VIRTIO_DEV_PART_VQ_CFG} \\
\hline
0x105 & VIRTIO_DEV_PART_VQ_NOTIFY_CFG & Virtqueue notification configuration, see \ref{sec:Basic Facilities of a Virtio Device / Device parts / Common device parts / VIRTIO_DEV_PART_VQ_NOTIFY_CFG} \\
\hline
0x106 - 0x2FF & - & Common device parts range reserved for future \\
\hline
\hline
\end{xltabular}
\end{table}

\subsubsection{VIRTIO_DEV_PART_DEV_FEATURES}
\label{sec:Basic Facilities of a Virtio Device / Device parts / Common device parts / VIRTIO_DEV_PART_DEV_FEATURES}

For VIRTIO_DEV_PART_DEV_FEATURES, \field{part_type} is set to 0x100.
The VIRTIO_DEV_PART_DEV_FEATURES field indicates features offered by the device.
\field{value} is in the format of \field{struct virtio_dev_part_features}.
\field{feature_bits} is in the format listed in
\ref{sec:Basic Facilities of a Virtio Device / Feature Bits}.
\field{length} is the length of the \field{struct virtio_dev_part_features}.

If the VIRTIO_DEV_PART_DEV_FEATURES device part is present, there is exactly
one instance of it in the get or set commands.

The VIRTIO_DEV_PART_DEV_FEATURES part is optional for which
the VIRTIO_DEV_PART_F_OPTIONAL (bit 0) \field{flags} is set.

\begin{lstlisting}
struct virtio_dev_part_features {
        le64 feature_bits[];
};
\end{lstlisting}

\subsubsection{VIRTIO_DEV_PART_DRV_FEATURES}
\label{sec:Basic Facilities of a Virtio Device / Device parts / Common device parts / VIRTIO_DEV_PART_DRV_FEATURES}

For VIRTIO_DEV_PART_DRV_FEATURES, \field{part_type} is set to 0x101.
The VIRTIO_DEV_PART_DRV_FEATURES field indicates features set by the driver.
\field{value} is in the format of \field{struct virtio_dev_part_features}.
\field{feature_bits} is in the format listed in
\ref{sec:Basic Facilities of a Virtio Device / Feature Bits}.
\field{length} is the length of the \field{struct virtio_dev_part_features}.

If the VIRTIO_DEV_PART_DEV_FEATURES device part present, there is exactly
one instance of it in the get or set commands.

\subsubsection{VIRTIO_DEV_PART_PCI_COMMON_CFG}
\label{sec:Basic Facilities of a Virtio Device / Device parts / Common device parts / VIRTIO_DEV_PART_PCI_COMMON_CFG}

For VIRTIO_DEV_PART_PCI_COMMON_CFG, \field{part_type} is set to 0x102.
VIRTIO_DEV_PART_PCI_COMMON_CFG refers to the common device configuration
fields. \field{offset} refers to the
byte offset of single field in the common configuration layout described in
\field{struct virtio_pci_common_cfg}. \field{value} is in the format depending on
the \field{offset}, for example when \field{cfg_offset = 18}, \field{value}
is in the format of \field{num_queues}. \field{length} is the length of
\field{value} in bytes of a single structure field whose offset is \field{offset}.

One or multiple VIRTIO_DEV_PART_PCI_COMMON_CFG parts may exist in the
get or set commands; each such part corresponds to a unique \field{offset}.

\subsubsection{VIRTIO_DEV_PART_DEVICE_STATUS}
\label{sec:Basic Facilities of a Virtio Device / Device parts / Common device parts / VIRTIO_DEV_PART_DEVICE_STATUS}

For VIRTIO_DEV_PART_DEVICE_STATUS, \field{part_type} is set to 0x103.
The VIRTIO_DEV_PART_DEVICE_STATUS field indicates the device status as listed in
\ref{sec:Basic Facilities of a Virtio Device / Device Status Field}.
\field{value} is in the format \field{device_status} of
\field{struct virtio_pci_common_cfg}.

If the VIRTIO_DEV_PART_DEV_FEATURES device part is present, there is exactly
one instance of it in the get or set commands.

There is exactly one part may exist in the get or set
commands.

\subsubsection{VIRTIO_DEV_PART_VQ_CFG}
\label{sec:Basic Facilities of a Virtio Device / Device parts / Common device parts / VIRTIO_DEV_PART_VQ_CFG}

For VIRTIO_DEV_PART_VQ_CFG, \field{part_type} is set to 0x104.
\field{value} is in the format \field{struct virtio_dev_part_vq_cfg}.
\field{length} is the length of \field{struct virtio_dev_part_vq_cfg}.

\begin{lstlisting}
struct virtio_dev_part_vq_cfg {
        le16 queue_size;
        le16 vector;
        le16 enabled;
        le16 reserved;
        le64 queue_desc;
        le64 queue_driver;
        le64 queue_device;
};
\end{lstlisting}

\field{queue_size}, \field{vector}, \field{queue_desc},
\field{queue_driver} and \field{queue_device} correspond to the
fields of \field{struct virtio_pci_common_cfg} when used for PCI transport.

One or multiple instances of the device part VIRTIO_DEV_PART_VQ_CFG may exist in
the get and set commands. Each such device part corresponds to a unique virtqueue identified
by the \field{vq_index.index}.

\subsubsection{VIRTIO_DEV_PART_VQ_NOTIFY_CFG}
\label{sec:Basic Facilities of a Virtio Device / Device parts / Common device parts / VIRTIO_DEV_PART_VQ_NOTIFY_CFG}

For VIRTIO_DEV_PART_VQ_NOTIFY_CFG, \field{part_type} is set to 0x105.
\field{value} is in the format \field{struct virtio_dev_part_vq_notify_data}.
\field{length} is the length of \field{struct virtio_dev_part_vq_notify_data}.

\begin{lstlisting}
struct virtio_dev_part_vq_notify_data {
        le16 queue_notify_off;
        le16 queue_notif_config_data;
        u8 reserved[4];
};
\end{lstlisting}

\field{queue_notify_off} and \field{queue_notif_config_data} corresponds to the
fields in \field{struct virtio_pci_common_cfg} described in the
\nameref{sec:Virtio Transport Options / Virtio Over PCI Bus / PCI Device Layout / Common configuration structure layout}.

One or multiple instance of the device part VIRTIO_DEV_PART_VQ_NOTIFY_CFG may exist
in the get and set commands, each such device part corresponds to a unique
virtqueue identified by the \field{vq_index.index}.

\field{reserved} is reserved and set to 0.

\subsection{Assumptions}
For the SR-IOV group type, some hypervisors do not allow the driver to access
the PCI configuration space and the MSI-X Table space directly. Such hypervisors
query and save these fields without the need for this device parts.
Therefore, this version of the specification does not have it in the device parts. A future
extension of the device part may further include them as new device part.


\chapter{General Initialization And Device Operation}\label{sec:General Initialization And Device Operation}

We start with an overview of device initialization, then expand on the
details of the device and how each step is performed.  This section
is best read along with the bus-specific section which describes
how to communicate with the specific device.

\section{Device Initialization}\label{sec:General Initialization And Device Operation / Device Initialization}

\drivernormative{\subsection}{Device Initialization}{General Initialization And Device Operation / Device Initialization}
The driver MUST follow this sequence to initialize a device:

\begin{enumerate}
\item Reset the device.

\item Set the ACKNOWLEDGE status bit: the guest OS has noticed the device.

\item Set the DRIVER status bit: the guest OS knows how to drive the device.

\item\label{itm:General Initialization And Device Operation /
Device Initialization / Read feature bits} Read device feature bits, and write the subset of feature bits
   understood by the OS and driver to the device.  During this step the
   driver MAY read (but MUST NOT write) the device-specific configuration fields to check that it can support the device before accepting it.

\item\label{itm:General Initialization And Device Operation / Device Initialization / Set FEATURES-OK} Set the FEATURES_OK status bit.  The driver MUST NOT accept
   new feature bits after this step.

\item\label{itm:General Initialization And Device Operation / Device Initialization / Re-read FEATURES-OK} Re-read \field{device status} to ensure the FEATURES_OK bit is still
   set: otherwise, the device does not support our subset of features
   and the device is unusable.

\item\label{itm:General Initialization And Device Operation / Device Initialization / Device-specific Setup} Perform device-specific setup, including discovery of virtqueues for the
   device, optional per-bus setup, reading and possibly writing the
   device's virtio configuration space, and population of virtqueues.

\item\label{itm:General Initialization And Device Operation / Device Initialization / Set DRIVER-OK} Set the DRIVER_OK status bit.  At this point the device is
   ``live''.
\end{enumerate}

If any of these steps go irrecoverably wrong, the driver SHOULD
set the FAILED status bit to indicate that it has given up on the
device (it can reset the device later to restart if desired).  The
driver MUST NOT continue initialization in that case.

The driver MUST NOT send any buffer available notifications to
the device before setting DRIVER_OK.

\subsection{Legacy Interface: Device Initialization}\label{sec:General Initialization And Device Operation / Device Initialization / Legacy Interface: Device Initialization}
Legacy devices did not support the FEATURES_OK status bit, and thus did
not have a graceful way for the device to indicate unsupported feature
combinations.  They also did not provide a clear mechanism to end
feature negotiation, which meant that devices finalized features on
first-use, and no features could be introduced which radically changed
the initial operation of the device.

Legacy driver implementations often used the device before setting the
DRIVER_OK bit, and sometimes even before writing the feature bits
to the device.

The result was the steps \ref{itm:General Initialization And
Device Operation / Device Initialization / Set FEATURES-OK} and
\ref{itm:General Initialization And Device Operation / Device
Initialization / Re-read FEATURES-OK} were omitted, and steps
\ref{itm:General Initialization And Device Operation /
Device Initialization / Read feature bits},
\ref{itm:General Initialization And Device Operation / Device Initialization / Device-specific Setup} and \ref{itm:General Initialization And Device Operation / Device Initialization / Set DRIVER-OK}
were conflated.

Therefore, when using the legacy interface:
\begin{itemize}
\item
The transitional driver MUST execute the initialization
sequence as described in \ref{sec:General Initialization And Device
Operation / Device Initialization}
but omitting the steps \ref{itm:General Initialization And Device
Operation / Device Initialization / Set FEATURES-OK} and
\ref{itm:General Initialization And Device Operation / Device
Initialization / Re-read FEATURES-OK}.

\item
The transitional device MUST support the driver
writing device configuration fields
before the step \ref{itm:General Initialization And Device Operation /
Device Initialization / Read feature bits}.
\item
The transitional device MUST support the driver
using the device before the step \ref{itm:General Initialization
And Device Operation / Device Initialization / Set DRIVER-OK}.
\end{itemize}

\section{Device Operation}\label{sec:General Initialization And Device Operation / Device Operation}

When operating the device, each field in the device configuration
space can be changed by either the driver or the device.

Whenever such a configuration change is triggered by the device,
driver is notified. This makes it possible for drivers to
cache device configuration, avoiding expensive configuration
reads unless notified.


\subsection{Notification of Device Configuration Changes}\label{sec:General Initialization And Device Operation / Device Operation / Notification of Device Configuration Changes}

For devices where the device-specific configuration information can be
changed, a configuration change notification is sent when a
device-specific configuration change occurs.

In addition, this notification is triggered by the device setting
DEVICE_NEEDS_RESET (see \ref{sec:Basic Facilities of a Virtio Device / Device Status Field / DEVICENEEDSRESET}).

\section{Device Cleanup}\label{sec:General Initialization And Device Operation / Device Cleanup}

Once the driver has set the DRIVER_OK status bit, all the configured
virtqueue of the device are considered live.  None of the virtqueues
of a device are live once the device has been reset.

\drivernormative{\subsection}{Device Cleanup}{General Initialization And Device Operation / Device Cleanup}

A driver MUST NOT alter virtqueue entries for exposed buffers,
i.e., buffers which have been
made available to the device (and not been used by the device)
of a live virtqueue.

Thus a driver MUST ensure a virtqueue isn't live (by device reset) before removing exposed buffers.

\chapter{Virtio Transport Options}\label{sec:Virtio Transport Options}

Virtio can use various different buses, thus the standard is split
into virtio general and bus-specific sections.

\input{transport-pci.tex}
\input{transport-mmio.tex}
\input{transport-ccw.tex}

\chapter{Device Types}\label{sec:Device Types}

On top of the queues, config space and feature negotiation facilities
built into virtio, several devices are defined.

The following device IDs are used to identify different types of virtio
devices.  Some device IDs are reserved for devices which are not currently
defined in this standard.

Discovering what devices are available and their type is bus-dependent.

\begin{longtable} { |l|c| }
\hline
Device ID  &  Virtio Device    \\
\hline \hline
0          & reserved (invalid) \\
\hline
1          &   network device     \\
\hline
2          &   block device     \\
\hline
3          &      console       \\
\hline
4          &  entropy source    \\
\hline
5          & memory ballooning (traditional)  \\
\hline
6          &     ioMemory       \\
\hline
7          &       rpmsg        \\
\hline
8          &     SCSI host      \\
\hline
9          &   9P transport     \\
\hline
10         &   mac80211 wlan    \\
\hline
11         &   rproc serial     \\
\hline
12         &   virtio CAIF      \\
\hline
13         &  memory balloon    \\
\hline
16         &   GPU device       \\
\hline
17         &   Timer/Clock device \\
\hline
18         &   Input device \\
\hline
19         &   Socket device \\
\hline
20         &   Crypto device \\
\hline
21         &   Signal Distribution Module \\
\hline
22         &   pstore device \\
\hline
23         &   IOMMU device \\
\hline
24         &   Memory device \\
\hline
25         &   Sound device \\
\hline
26         &   file system device \\
\hline
27         &   PMEM device \\
\hline
28         &   RPMB device \\
\hline
29         &   mac80211 hwsim wireless simulation device \\
\hline
30         &   Video encoder device \\
\hline
31         &   Video decoder device \\
\hline
32         &   SCMI device \\
\hline
33         &   NitroSecureModule \\
\hline
34         &   I2C adapter \\
\hline
35         &   Watchdog \\
\hline
36         &   CAN device \\
\hline
38         &   Parameter Server \\
\hline
39         &   Audio policy device \\
\hline
40         &   Bluetooth device \\
\hline
41         &   GPIO device \\
\hline
42         &   RDMA device \\
\hline
43         &   Camera device \\
\hline
44         &   ISM device \\
\hline
45         &   SPI controller \\
\hline
46         &   TEE device \\
\hline
47         &   CPU balloon device \\
\hline
\end{longtable}

Some of the devices above are unspecified by this document,
because they are seen as immature or especially niche.  Be warned
that some are only specified by the sole existing implementation;
they could become part of a future specification, be abandoned
entirely, or live on outside this standard.  We shall speak of
them no further.

\section{Crypto Device}\label{sec:Device Types / Crypto Device}

The virtio crypto device is a virtual cryptography device as well as a
virtual cryptographic accelerator. The virtio crypto device provides the
following crypto services: CIPHER, MAC, HASH, AEAD and AKCIPHER. Virtio crypto
devices have a single control queue and at least one data queue. Crypto
operation requests are placed into a data queue, and serviced by the
device. Some crypto operation requests are only valid in the context of a
session. The role of the control queue is facilitating control operation
requests. Sessions management is realized with control operation
requests.

\subsection{Device ID}\label{sec:Device Types / Crypto Device / Device ID}

20

\subsection{Virtqueues}\label{sec:Device Types / Crypto Device / Virtqueues}

\begin{description}
\item[0] dataq1
\item[\ldots]
\item[N-1] dataqN
\item[N] controlq
\end{description}

N is set by \field{max_dataqueues}.

\subsection{Feature bits}\label{sec:Device Types / Crypto Device / Feature bits}

\begin{description}
\item VIRTIO_CRYPTO_F_REVISION_1 (0) revision 1. Revision 1 has a specific
    request format and other enhancements (which result in some additional
    requirements).
\item VIRTIO_CRYPTO_F_CIPHER_STATELESS_MODE (1) stateless mode requests are
    supported by the CIPHER service.
\item VIRTIO_CRYPTO_F_HASH_STATELESS_MODE (2) stateless mode requests are
    supported by the HASH service.
\item VIRTIO_CRYPTO_F_MAC_STATELESS_MODE (3) stateless mode requests are
    supported by the MAC service.
\item VIRTIO_CRYPTO_F_AEAD_STATELESS_MODE (4) stateless mode requests are
    supported by the AEAD service.
\item VIRTIO_CRYPTO_F_AKCIPHER_STATELESS_MODE (5) stateless mode requests are
    supported by the AKCIPHER service.
\end{description}


\subsubsection{Feature bit requirements}\label{sec:Device Types / Crypto Device / Feature bit requirements}

Some crypto feature bits require other crypto feature bits
(see \ref{drivernormative:Basic Facilities of a Virtio Device / Feature Bits}):

\begin{description}
\item[VIRTIO_CRYPTO_F_CIPHER_STATELESS_MODE] Requires VIRTIO_CRYPTO_F_REVISION_1.
\item[VIRTIO_CRYPTO_F_HASH_STATELESS_MODE] Requires VIRTIO_CRYPTO_F_REVISION_1.
\item[VIRTIO_CRYPTO_F_MAC_STATELESS_MODE] Requires VIRTIO_CRYPTO_F_REVISION_1.
\item[VIRTIO_CRYPTO_F_AEAD_STATELESS_MODE] Requires VIRTIO_CRYPTO_F_REVISION_1.
\item[VIRTIO_CRYPTO_F_AKCIPHER_STATELESS_MODE] Requires VIRTIO_CRYPTO_F_REVISION_1.
\end{description}

\subsection{Supported crypto services}\label{sec:Device Types / Crypto Device / Supported crypto services}

The following crypto services are defined:

\begin{lstlisting}
/* CIPHER (Symmetric Key Cipher) service */
#define VIRTIO_CRYPTO_SERVICE_CIPHER 0
/* HASH service */
#define VIRTIO_CRYPTO_SERVICE_HASH   1
/* MAC (Message Authentication Codes) service */
#define VIRTIO_CRYPTO_SERVICE_MAC    2
/* AEAD (Authenticated Encryption with Associated Data) service */
#define VIRTIO_CRYPTO_SERVICE_AEAD   3
/* AKCIPHER (Asymmetric Key Cipher) service */
#define VIRTIO_CRYPTO_SERVICE_AKCIPHER 4
\end{lstlisting}

The above constants designate bits used to indicate the which of crypto services are
offered by the device as described in, see \ref{sec:Device Types / Crypto Device / Device configuration layout}.

\subsubsection{CIPHER services}\label{sec:Device Types / Crypto Device / Supported crypto services / CIPHER services}

The following CIPHER algorithms are defined:

\begin{lstlisting}
#define VIRTIO_CRYPTO_NO_CIPHER                 0
#define VIRTIO_CRYPTO_CIPHER_ARC4               1
#define VIRTIO_CRYPTO_CIPHER_AES_ECB            2
#define VIRTIO_CRYPTO_CIPHER_AES_CBC            3
#define VIRTIO_CRYPTO_CIPHER_AES_CTR            4
#define VIRTIO_CRYPTO_CIPHER_DES_ECB            5
#define VIRTIO_CRYPTO_CIPHER_DES_CBC            6
#define VIRTIO_CRYPTO_CIPHER_3DES_ECB           7
#define VIRTIO_CRYPTO_CIPHER_3DES_CBC           8
#define VIRTIO_CRYPTO_CIPHER_3DES_CTR           9
#define VIRTIO_CRYPTO_CIPHER_KASUMI_F8          10
#define VIRTIO_CRYPTO_CIPHER_SNOW3G_UEA2        11
#define VIRTIO_CRYPTO_CIPHER_AES_F8             12
#define VIRTIO_CRYPTO_CIPHER_AES_XTS            13
#define VIRTIO_CRYPTO_CIPHER_ZUC_EEA3           14
\end{lstlisting}

The above constants have two usages:
\begin{enumerate}
\item As bit numbers, used to tell the driver which CIPHER algorithms
are supported by the device, see \ref{sec:Device Types / Crypto Device / Device configuration layout}.
\item As values, used to designate the algorithm in (CIPHER type) crypto
operation requests, see \ref{sec:Device Types / Crypto Device / Device Operation / Control Virtqueue / Session operation}.
\end{enumerate}

\subsubsection{HASH services}\label{sec:Device Types / Crypto Device / Supported crypto services / HASH services}

The following HASH algorithms are defined:

\begin{lstlisting}
#define VIRTIO_CRYPTO_NO_HASH            0
#define VIRTIO_CRYPTO_HASH_MD5           1
#define VIRTIO_CRYPTO_HASH_SHA1          2
#define VIRTIO_CRYPTO_HASH_SHA_224       3
#define VIRTIO_CRYPTO_HASH_SHA_256       4
#define VIRTIO_CRYPTO_HASH_SHA_384       5
#define VIRTIO_CRYPTO_HASH_SHA_512       6
#define VIRTIO_CRYPTO_HASH_SHA3_224      7
#define VIRTIO_CRYPTO_HASH_SHA3_256      8
#define VIRTIO_CRYPTO_HASH_SHA3_384      9
#define VIRTIO_CRYPTO_HASH_SHA3_512      10
#define VIRTIO_CRYPTO_HASH_SHA3_SHAKE128      11
#define VIRTIO_CRYPTO_HASH_SHA3_SHAKE256      12
\end{lstlisting}

The above constants have two usages:
\begin{enumerate}
\item As bit numbers, used to tell the driver which HASH algorithms
are supported by the device, see \ref{sec:Device Types / Crypto Device / Device configuration layout}.
\item As values, used to designate the algorithm in (HASH type) crypto
operation requires, see \ref{sec:Device Types / Crypto Device / Device Operation / Control Virtqueue / Session operation}.
\end{enumerate}

\subsubsection{MAC services}\label{sec:Device Types / Crypto Device / Supported crypto services / MAC services}

The following MAC algorithms are defined:

\begin{lstlisting}
#define VIRTIO_CRYPTO_NO_MAC                       0
#define VIRTIO_CRYPTO_MAC_HMAC_MD5                 1
#define VIRTIO_CRYPTO_MAC_HMAC_SHA1                2
#define VIRTIO_CRYPTO_MAC_HMAC_SHA_224             3
#define VIRTIO_CRYPTO_MAC_HMAC_SHA_256             4
#define VIRTIO_CRYPTO_MAC_HMAC_SHA_384             5
#define VIRTIO_CRYPTO_MAC_HMAC_SHA_512             6
#define VIRTIO_CRYPTO_MAC_CMAC_3DES                25
#define VIRTIO_CRYPTO_MAC_CMAC_AES                 26
#define VIRTIO_CRYPTO_MAC_KASUMI_F9                27
#define VIRTIO_CRYPTO_MAC_SNOW3G_UIA2              28
#define VIRTIO_CRYPTO_MAC_GMAC_AES                 41
#define VIRTIO_CRYPTO_MAC_GMAC_TWOFISH             42
#define VIRTIO_CRYPTO_MAC_CBCMAC_AES               49
#define VIRTIO_CRYPTO_MAC_CBCMAC_KASUMI_F9         50
#define VIRTIO_CRYPTO_MAC_XCBC_AES                 53
#define VIRTIO_CRYPTO_MAC_ZUC_EIA3                 54
\end{lstlisting}

The above constants have two usages:
\begin{enumerate}
\item As bit numbers, used to tell the driver which MAC algorithms
are supported by the device, see \ref{sec:Device Types / Crypto Device / Device configuration layout}.
\item As values, used to designate the algorithm in (MAC type) crypto
operation requests, see \ref{sec:Device Types / Crypto Device / Device Operation / Control Virtqueue / Session operation}.
\end{enumerate}

\subsubsection{AEAD services}\label{sec:Device Types / Crypto Device / Supported crypto services / AEAD services}

The following AEAD algorithms are defined:

\begin{lstlisting}
#define VIRTIO_CRYPTO_NO_AEAD     0
#define VIRTIO_CRYPTO_AEAD_GCM    1
#define VIRTIO_CRYPTO_AEAD_CCM    2
#define VIRTIO_CRYPTO_AEAD_CHACHA20_POLY1305  3
\end{lstlisting}

The above constants have two usages:
\begin{enumerate}
\item As bit numbers, used to tell the driver which AEAD algorithms
are supported by the device, see \ref{sec:Device Types / Crypto Device / Device configuration layout}.
\item As values, used to designate the algorithm in (DEAD type) crypto
operation requests, see \ref{sec:Device Types / Crypto Device / Device Operation / Control Virtqueue / Session operation}.
\end{enumerate}

\subsubsection{AKCIPHER services}\label{sec: Device Types / Crypto Device / Supported crypto services / AKCIPHER services}

The following AKCIPHER algorithms are defined:
\begin{lstlisting}
#define VIRTIO_CRYPTO_NO_AKCIPHER 0
#define VIRTIO_CRYPTO_AKCIPHER_RSA   1
#define VIRTIO_CRYPTO_AKCIPHER_ECDSA 2
\end{lstlisting}

The above constants have two usages:
\begin{enumerate}
\item As bit numbers, used to tell the driver which AKCIPHER algorithms
are supported by the device, see \ref{sec:Device Types / Crypto Device / Device configuration layout}.
\item As values, used to designate the algorithm in asymmetric crypto operation requests,
see \ref{sec:Device Types / Crypto Device / Device Operation / Control Virtqueue / Session operation}.
\end{enumerate}


\subsection{Device configuration layout}\label{sec:Device Types / Crypto Device / Device configuration layout}

Crypto device configuration uses the following layout structure:

\begin{lstlisting}
struct virtio_crypto_config {
    le32 status;
    le32 max_dataqueues;
    le32 crypto_services;
    /* Detailed algorithms mask */
    le32 cipher_algo_l;
    le32 cipher_algo_h;
    le32 hash_algo;
    le32 mac_algo_l;
    le32 mac_algo_h;
    le32 aead_algo;
    /* Maximum length of cipher key in bytes */
    le32 max_cipher_key_len;
    /* Maximum length of authenticated key in bytes */
    le32 max_auth_key_len;
    le32 akcipher_algo;
    /* Maximum size of each crypto request's content in bytes */
    le64 max_size;
};
\end{lstlisting}

\begin{description}
\item Currently, only one \field{status} bit is defined: VIRTIO_CRYPTO_S_HW_READY
    set indicates that the device is ready to process requests, this bit is read-only
    for the driver
\begin{lstlisting}
#define VIRTIO_CRYPTO_S_HW_READY  (1 << 0)
\end{lstlisting}

\item [\field{max_dataqueues}] is the maximum number of data virtqueues that can
    be configured by the device. The driver MAY use only one data queue, or it
    can use more to achieve better performance.

\item [\field{crypto_services}] crypto service offered, see \ref{sec:Device Types / Crypto Device / Supported crypto services}.

\item [\field{cipher_algo_l}] CIPHER algorithms bits 0-31, see \ref{sec:Device Types / Crypto Device / Supported crypto services  / CIPHER services}.

\item [\field{cipher_algo_h}] CIPHER algorithms bits 32-63, see \ref{sec:Device Types / Crypto Device / Supported crypto services  / CIPHER services}.

\item [\field{hash_algo}] HASH algorithms bits, see \ref{sec:Device Types / Crypto Device / Supported crypto services  / HASH services}.

\item [\field{mac_algo_l}] MAC algorithms bits 0-31, see \ref{sec:Device Types / Crypto Device / Supported crypto services  / MAC services}.

\item [\field{mac_algo_h}] MAC algorithms bits 32-63, see \ref{sec:Device Types / Crypto Device / Supported crypto services  / MAC services}.

\item [\field{aead_algo}] AEAD algorithms bits, see \ref{sec:Device Types / Crypto Device / Supported crypto services  / AEAD services}.

\item [\field{max_cipher_key_len}] is the maximum length of cipher key supported by the device.

\item [\field{max_auth_key_len}] is the maximum length of authenticated key supported by the device.

\item [\field{akcipher_algo}] AKCIPHER algorithms bit 0-31, see \ref{sec: Device Types / Crypto Device / Supported crypto services / AKCIPHER services}.

\item [\field{max_size}] is the maximum size of the variable-length parameters of
    data operation of each crypto request's content supported by the device.
\end{description}

\begin{note}
Unless explicitly stated otherwise all lengths and sizes are in bytes.
\end{note}

\devicenormative{\subsubsection}{Device configuration layout}{Device Types / Crypto Device / Device configuration layout}

\begin{itemize*}
\item The device MUST set \field{max_dataqueues} to between 1 and 65535 inclusive.
\item The device MUST set the \field{status} with valid flags, undefined flags MUST NOT be set.
\item The device MUST accept and handle requests after \field{status} is set to VIRTIO_CRYPTO_S_HW_READY.
\item The device MUST set \field{crypto_services} based on the crypto services the device offers.
\item The device MUST set detailed algorithms masks for each service advertised by \field{crypto_services}.
    The device MUST NOT set the not defined algorithms bits.
\item The device MUST set \field{max_size} to show the maximum size of crypto request the device supports.
\item The device MUST set \field{max_cipher_key_len} to show the maximum length of cipher key if the
    device supports CIPHER service.
\item The device MUST set \field{max_auth_key_len} to show the maximum length of authenticated key if
    the device supports MAC service.
\end{itemize*}

\drivernormative{\subsubsection}{Device configuration layout}{Device Types / Crypto Device / Device configuration layout}

\begin{itemize*}
\item The driver MUST read the \field{status} from the bottom bit of status to check whether the
    VIRTIO_CRYPTO_S_HW_READY is set, and the driver MUST reread it after device reset.
\item The driver MUST NOT transmit any requests to the device if the VIRTIO_CRYPTO_S_HW_READY is not set.
\item The driver MUST read \field{max_dataqueues} field to discover the number of data queues the device supports.
\item The driver MUST read \field{crypto_services} field to discover which services the device is able to offer.
\item The driver SHOULD ignore the not defined algorithms bits.
\item The driver MUST read the detailed algorithms fields based on \field{crypto_services} field.
\item The driver SHOULD read \field{max_size} to discover the maximum size of the variable-length
    parameters of data operation of the crypto request's content the device supports and MUST
    guarantee that the size of each crypto request's content is within the \field{max_size}, otherwise
    the request will fail and the driver MUST reset the device.
\item The driver SHOULD read \field{max_cipher_key_len} to discover the maximum length of cipher key
    the device supports and MUST guarantee that the \field{key_len} (CIPHER service or AEAD service) is within
    the \field{max_cipher_key_len} of the device configuration, otherwise the request will fail.
\item The driver SHOULD read \field{max_auth_key_len} to discover the maximum length of authenticated
    key the device supports and MUST guarantee that the \field{auth_key_len} (MAC service) is within the
    \field{max_auth_key_len} of the device configuration, otherwise the request will fail.
\end{itemize*}

\subsection{Device Initialization}\label{sec:Device Types / Crypto Device / Device Initialization}

\drivernormative{\subsubsection}{Device Initialization}{Device Types / Crypto Device / Device Initialization}

\begin{itemize*}
\item The driver MUST configure and initialize all virtqueues.
\item The driver MUST read the supported crypto services from bits of \field{crypto_services}.
\item The driver MUST read the supported algorithms based on \field{crypto_services} field.
\end{itemize*}

\subsection{Device Operation}\label{sec:Device Types / Crypto Device / Device Operation}

The operation of a virtio crypto device is driven by requests placed on the virtqueues.
Requests consist of a queue-type specific header (specifying among others the operation)
and an operation specific payload.

If VIRTIO_CRYPTO_F_REVISION_1 is negotiated the device may support both session mode
(See \ref{sec:Device Types / Crypto Device / Device Operation / Control Virtqueue / Session operation})
and stateless mode operation requests.
In stateless mode all operation parameters are supplied as a part of each request,
while in session mode, some or all operation parameters are managed within the
session. Stateless mode is guarded by feature bits 0-4 on a service level. If
stateless mode is negotiated for a service, the service accepts both session
mode and stateless requests; otherwise stateless mode requests are rejected
(via operation status).

\subsubsection{Operation Status}\label{sec:Device Types / Crypto Device / Device Operation / Operation status}
The device MUST return a status code as part of the operation (both session
operation and service operation) result. The valid operation status as follows:

\begin{lstlisting}
enum VIRTIO_CRYPTO_STATUS {
    VIRTIO_CRYPTO_OK = 0,
    VIRTIO_CRYPTO_ERR = 1,
    VIRTIO_CRYPTO_BADMSG = 2,
    VIRTIO_CRYPTO_NOTSUPP = 3,
    VIRTIO_CRYPTO_INVSESS = 4,
    VIRTIO_CRYPTO_NOSPC = 5,
    VIRTIO_CRYPTO_KEY_REJECTED = 6,
    VIRTIO_CRYPTO_MAX
};
\end{lstlisting}

\begin{itemize*}
\item VIRTIO_CRYPTO_OK: success.
\item VIRTIO_CRYPTO_BADMSG: authentication failed (only when AEAD decryption).
\item VIRTIO_CRYPTO_NOTSUPP: operation or algorithm is unsupported.
\item VIRTIO_CRYPTO_INVSESS: invalid session ID when executing crypto operations.
\item VIRTIO_CRYPTO_NOSPC: no free session ID (only when the VIRTIO_CRYPTO_F_REVISION_1
    feature bit is negotiated).
\item VIRTIO_CRYPTO_KEY_REJECTED: signature verification failed (only when AKCIPHER verification).
\item VIRTIO_CRYPTO_ERR: any failure not mentioned above occurs.
\end{itemize*}

\subsubsection{Control Virtqueue}\label{sec:Device Types / Crypto Device / Device Operation / Control Virtqueue}

The driver uses the control virtqueue to send control commands to the
device, such as session operations (See \ref{sec:Device Types / Crypto Device / Device
Operation / Control Virtqueue / Session operation}).

The header for controlq is of the following form:
\begin{lstlisting}
#define VIRTIO_CRYPTO_OPCODE(service, op)   (((service) << 8) | (op))

struct virtio_crypto_ctrl_header {
#define VIRTIO_CRYPTO_CIPHER_CREATE_SESSION \
       VIRTIO_CRYPTO_OPCODE(VIRTIO_CRYPTO_SERVICE_CIPHER, 0x02)
#define VIRTIO_CRYPTO_CIPHER_DESTROY_SESSION \
       VIRTIO_CRYPTO_OPCODE(VIRTIO_CRYPTO_SERVICE_CIPHER, 0x03)
#define VIRTIO_CRYPTO_HASH_CREATE_SESSION \
       VIRTIO_CRYPTO_OPCODE(VIRTIO_CRYPTO_SERVICE_HASH, 0x02)
#define VIRTIO_CRYPTO_HASH_DESTROY_SESSION \
       VIRTIO_CRYPTO_OPCODE(VIRTIO_CRYPTO_SERVICE_HASH, 0x03)
#define VIRTIO_CRYPTO_MAC_CREATE_SESSION \
       VIRTIO_CRYPTO_OPCODE(VIRTIO_CRYPTO_SERVICE_MAC, 0x02)
#define VIRTIO_CRYPTO_MAC_DESTROY_SESSION \
       VIRTIO_CRYPTO_OPCODE(VIRTIO_CRYPTO_SERVICE_MAC, 0x03)
#define VIRTIO_CRYPTO_AEAD_CREATE_SESSION \
       VIRTIO_CRYPTO_OPCODE(VIRTIO_CRYPTO_SERVICE_AEAD, 0x02)
#define VIRTIO_CRYPTO_AEAD_DESTROY_SESSION \
       VIRTIO_CRYPTO_OPCODE(VIRTIO_CRYPTO_SERVICE_AEAD, 0x03)
#define VIRTIO_CRYPTO_AKCIPHER_CREATE_SESSION \
       VIRTIO_CRYPTO_OPCODE(VIRTIO_CRYPTO_SERVICE_AKCIPHER, 0x04)
#define VIRTIO_CRYPTO_AKCIPHER_DESTROY_SESSION \
       VIRTIO_CRYPTO_OPCDE(VIRTIO_CRYPTO_SERVICE_AKCIPHER, 0x05)
    le32 opcode;
    /* algo should be service-specific algorithms */
    le32 algo;
    le32 flag;
    le32 reserved;
};
\end{lstlisting}

The controlq request is composed of four parts:
\begin{lstlisting}
struct virtio_crypto_op_ctrl_req {
    /* Device read only portion */

    struct virtio_crypto_ctrl_header header;

#define VIRTIO_CRYPTO_CTRLQ_OP_SPEC_HDR_LEGACY 56
    /* fixed length fields, opcode specific */
    u8 op_flf[flf_len];

    /* variable length fields, opcode specific */
    u8 op_vlf[vlf_len];

    /* Device write only portion */

    /* op result or completion status */
    u8 op_outcome[outcome_len];
};
\end{lstlisting}

\field{header} is a general header (see above).

\field{op_flf} is the opcode (in \field{header}) specific fixed-length parameters.

\field{flf_len} depends on the VIRTIO_CRYPTO_F_REVISION_1 feature bit (see below).

\field{op_vlf} is the opcode (in \field{header}) specific variable-length parameters.

\field{vlf_len} is the size of the specific structure used.
\begin{note}
The \field{vlf_len} of session-destroy operation and the hash-session-create
operation is ZERO.
\end{note}

\begin{itemize*}
\item If the opcode (in \field{header}) is VIRTIO_CRYPTO_CIPHER_CREATE_SESSION
    then \field{op_flf} is struct virtio_crypto_sym_create_session_flf if
    VIRTIO_CRYPTO_F_REVISION_1 is negotiated and struct virtio_crypto_sym_create_session_flf is
    padded to 56 bytes if NOT negotiated, and \field{op_vlf} is struct
    virtio_crypto_sym_create_session_vlf.
\item If the opcode (in \field{header}) is VIRTIO_CRYPTO_HASH_CREATE_SESSION
    then \field{op_flf} is struct virtio_crypto_hash_create_session_flf if
    VIRTIO_CRYPTO_F_REVISION_1 is negotiated and struct virtio_crypto_hash_create_session_flf is
    padded to 56 bytes if NOT negotiated.
\item If the opcode (in \field{header}) is VIRTIO_CRYPTO_MAC_CREATE_SESSION
    then \field{op_flf} is struct virtio_crypto_mac_create_session_flf if
    VIRTIO_CRYPTO_F_REVISION_1 is negotiated and struct virtio_crypto_mac_create_session_flf is
    padded to 56 bytes if NOT negotiated, and \field{op_vlf} is struct
    virtio_crypto_mac_create_session_vlf.
\item If the opcode (in \field{header}) is VIRTIO_CRYPTO_AEAD_CREATE_SESSION
    then \field{op_flf} is struct virtio_crypto_aead_create_session_flf if
    VIRTIO_CRYPTO_F_REVISION_1 is negotiated and struct virtio_crypto_aead_create_session_flf is
    padded to 56 bytes if NOT negotiated, and \field{op_vlf} is struct
    virtio_crypto_aead_create_session_vlf.
\item If the opcode (in \field{header}) is VIRTIO_CRYPTO_AKCIPHER_CREATE_SESSION
    then \field{op_flf} is struct virtio_crypto_akcipher_create_session_flf if
    VIRTIO_CRYPTO_F_REVISION_1 is negotiated and struct virtio_crypto_akcipher_create_session_flf is
    padded to 56 bytes if NOT negotiated, and \field{op_vlf} is struct
    virtio_crypto_akcipher_create_session_vlf.
\item If the opcode (in \field{header}) is VIRTIO_CRYPTO_CIPHER_DESTROY_SESSION
    or VIRTIO_CRYPTO_HASH_DESTROY_SESSION or VIRTIO_CRYPTO_MAC_DESTROY_SESSION or
    VIRTIO_CRYPTO_AEAD_DESTROY_SESSION then \field{op_flf} is struct
    virtio_crypto_destroy_session_flf if VIRTIO_CRYPTO_F_REVISION_1 is negotiated and
    struct virtio_crypto_destroy_session_flf is padded to 56 bytes if NOT negotiated.
\end{itemize*}

\field{op_outcome} stores the result of operation and must be struct
virtio_crypto_destroy_session_input for destroy session or
struct virtio_crypto_create_session_input for create session.

\field{outcome_len} is the size of the structure used.


\paragraph{Session operation}\label{sec:Device Types / Crypto Device / Device
Operation / Control Virtqueue / Session operation}

The session is a handle which describes the cryptographic parameters to be
applied to a number of buffers.

The following structure stores the result of session creation set by the device:

\begin{lstlisting}
struct virtio_crypto_create_session_input {
    le64 session_id;
    le32 status;
    le32 padding;
};
\end{lstlisting}

A request to destroy a session includes the following information:

\begin{lstlisting}
struct virtio_crypto_destroy_session_flf {
    /* Device read only portion */
    le64  session_id;
};

struct virtio_crypto_destroy_session_input {
    /* Device write only portion */
    u8  status;
};
\end{lstlisting}


\subparagraph{Session operation: HASH session}\label{sec:Device Types / Crypto Device / Device
Operation / Control Virtqueue / Session operation / Session operation: HASH session}

The fixed-length parameters of HASH session requests is as follows:

\begin{lstlisting}
struct virtio_crypto_hash_create_session_flf {
    /* Device read only portion */

    /* See VIRTIO_CRYPTO_HASH_* above */
    le32 algo;
    /* hash result length */
    le32 hash_result_len;
};
\end{lstlisting}


\subparagraph{Session operation: MAC session}\label{sec:Device Types / Crypto Device / Device
Operation / Control Virtqueue / Session operation / Session operation: MAC session}

The fixed-length and the variable-length parameters of MAC session requests are as follows:

\begin{lstlisting}
struct virtio_crypto_mac_create_session_flf {
    /* Device read only portion */

    /* See VIRTIO_CRYPTO_MAC_* above */
    le32 algo;
    /* hash result length */
    le32 hash_result_len;
    /* length of authenticated key */
    le32 auth_key_len;
    le32 padding;
};

struct virtio_crypto_mac_create_session_vlf {
    /* Device read only portion */

    /* The authenticated key */
    u8 auth_key[auth_key_len];
};
\end{lstlisting}

The length of \field{auth_key} is specified in \field{auth_key_len} in the struct
virtio_crypto_mac_create_session_flf.


\subparagraph{Session operation: Symmetric algorithms session}\label{sec:Device Types / Crypto Device / Device
Operation / Control Virtqueue / Session operation / Session operation: Symmetric algorithms session}

The request of symmetric session could be the CIPHER algorithms request
or the chain algorithms (chaining CIPHER and HASH/MAC) request.

The fixed-length and the variable-length parameters of CIPHER session requests are as follows:

\begin{lstlisting}
struct virtio_crypto_cipher_session_flf {
    /* Device read only portion */

    /* See VIRTIO_CRYPTO_CIPHER* above */
    le32 algo;
    /* length of key */
    le32 key_len;
#define VIRTIO_CRYPTO_OP_ENCRYPT  1
#define VIRTIO_CRYPTO_OP_DECRYPT  2
    /* encryption or decryption */
    le32 op;
    le32 padding;
};

struct virtio_crypto_cipher_session_vlf {
    /* Device read only portion */

    /* The cipher key */
    u8 cipher_key[key_len];
};
\end{lstlisting}

The length of \field{cipher_key} is specified in \field{key_len} in the struct
virtio_crypto_cipher_session_flf.

The fixed-length and the variable-length parameters of Chain session requests are as follows:

\begin{lstlisting}
struct virtio_crypto_alg_chain_session_flf {
    /* Device read only portion */

#define VIRTIO_CRYPTO_SYM_ALG_CHAIN_ORDER_HASH_THEN_CIPHER  1
#define VIRTIO_CRYPTO_SYM_ALG_CHAIN_ORDER_CIPHER_THEN_HASH  2
    le32 alg_chain_order;
/* Plain hash */
#define VIRTIO_CRYPTO_SYM_HASH_MODE_PLAIN    1
/* Authenticated hash (mac) */
#define VIRTIO_CRYPTO_SYM_HASH_MODE_AUTH     2
/* Nested hash */
#define VIRTIO_CRYPTO_SYM_HASH_MODE_NESTED   3
    le32 hash_mode;
    struct virtio_crypto_cipher_session_flf cipher_hdr;

#define VIRTIO_CRYPTO_ALG_CHAIN_SESS_OP_SPEC_HDR_SIZE  16
    /* fixed length fields, algo specific */
    u8 algo_flf[VIRTIO_CRYPTO_ALG_CHAIN_SESS_OP_SPEC_HDR_SIZE];

    /* length of the additional authenticated data (AAD) in bytes */
    le32 aad_len;
    le32 padding;
};

struct virtio_crypto_alg_chain_session_vlf {
    /* Device read only portion */

    /* The cipher key */
    u8 cipher_key[key_len];
    /* The authenticated key */
    u8 auth_key[auth_key_len];
};
\end{lstlisting}

\field{hash_mode} decides the type used by \field{algo_flf}.

\field{algo_flf} is fixed to 16 bytes and MUST contains or be one of
the following types:
\begin{itemize*}
\item struct virtio_crypto_hash_create_session_flf
\item struct virtio_crypto_mac_create_session_flf
\end{itemize*}
The data of unused part (if has) in \field{algo_flf} will be ignored.

The length of \field{cipher_key} is specified in \field{key_len} in \field{cipher_hdr}.

The length of \field{auth_key} is specified in \field{auth_key_len} in struct
virtio_crypto_mac_create_session_flf.

The fixed-length parameters of Symmetric session requests are as follows:

\begin{lstlisting}
struct virtio_crypto_sym_create_session_flf {
    /* Device read only portion */

#define VIRTIO_CRYPTO_SYM_SESS_OP_SPEC_HDR_SIZE  48
    /* fixed length fields, opcode specific */
    u8 op_flf[VIRTIO_CRYPTO_SYM_SESS_OP_SPEC_HDR_SIZE];

/* No operation */
#define VIRTIO_CRYPTO_SYM_OP_NONE  0
/* Cipher only operation on the data */
#define VIRTIO_CRYPTO_SYM_OP_CIPHER  1
/* Chain any cipher with any hash or mac operation. The order
   depends on the value of alg_chain_order param */
#define VIRTIO_CRYPTO_SYM_OP_ALGORITHM_CHAINING  2
    le32 op_type;
    le32 padding;
};
\end{lstlisting}

\field{op_flf} is fixed to 48 bytes, MUST contains or be one of
the following types:
\begin{itemize*}
\item struct virtio_crypto_cipher_session_flf
\item struct virtio_crypto_alg_chain_session_flf
\end{itemize*}
The data of unused part (if has) in \field{op_flf} will be ignored.

\field{op_type} decides the type used by \field{op_flf}.

The variable-length parameters of Symmetric session requests are as follows:

\begin{lstlisting}
struct virtio_crypto_sym_create_session_vlf {
    /* Device read only portion */
    /* variable length fields, opcode specific */
    u8 op_vlf[vlf_len];
};
\end{lstlisting}

\field{op_vlf} MUST contains or be one of the following types:
\begin{itemize*}
\item struct virtio_crypto_cipher_session_vlf
\item struct virtio_crypto_alg_chain_session_vlf
\end{itemize*}

\field{op_type} in struct virtio_crypto_sym_create_session_flf decides the
type used by \field{op_vlf}.

\field{vlf_len} is the size of the specific structure used.


\subparagraph{Session operation: AEAD session}\label{sec:Device Types / Crypto Device / Device
Operation / Control Virtqueue / Session operation / Session operation: AEAD session}

The fixed-length and the variable-length parameters of AEAD session requests are as follows:

\begin{lstlisting}
struct virtio_crypto_aead_create_session_flf {
    /* Device read only portion */

    /* See VIRTIO_CRYPTO_AEAD_* above */
    le32 algo;
    /* length of key */
    le32 key_len;
    /* Authentication tag length */
    le32 tag_len;
    /* The length of the additional authenticated data (AAD) in bytes */
    le32 aad_len;
    /* encryption or decryption, See above VIRTIO_CRYPTO_OP_* */
    le32 op;
    le32 padding;
};

struct virtio_crypto_aead_create_session_vlf {
    /* Device read only portion */
    u8 key[key_len];
};
\end{lstlisting}

The length of \field{key} is specified in \field{key_len} in struct
virtio_crypto_aead_create_session_flf.

\subparagraph{Session operation: AKCIPHER session}\label{sec:Device Types / Crypto Device / Device
Operation / Control Virtqueue / Session operation / Session operation: AKCIPHER session}

Due to the complexity of asymmetric key algorithms, different algorithms
require different parameters. The following data structures are used as
supplementary parameters to describe the asymmetric algorithm sessions.

For the RSA algorithm, the extra parameters are as follows:
\begin{lstlisting}
struct virtio_crypto_rsa_session_para {
#define VIRTIO_CRYPTO_RSA_RAW_PADDING   0
#define VIRTIO_CRYPTO_RSA_PKCS1_PADDING 1
    le32 padding_algo;

#define VIRTIO_CRYPTO_RSA_NO_HASH   0
#define VIRTIO_CRYPTO_RSA_MD2       1
#define VIRTIO_CRYPTO_RSA_MD3       2
#define VIRTIO_CRYPTO_RSA_MD4       3
#define VIRTIO_CRYPTO_RSA_MD5       4
#define VIRTIO_CRYPTO_RSA_SHA1      5
#define VIRTIO_CRYPTO_RSA_SHA256    6
#define VIRTIO_CRYPTO_RSA_SHA384    7
#define VIRTIO_CRYPTO_RSA_SHA512    8
#define VIRTIO_CRYPTO_RSA_SHA224    9
    le32 hash_algo;
};
\end{lstlisting}

\field{padding_algo} specifies the padding method used by RSA sessions.
\begin{itemize*}
\item If VIRTIO_CRYPTO_RSA_RAW_PADDING is specified, 1) \field{hash_algo}
is ignored, 2) ciphertext and plaintext MUST be padded with leading zeros,
3) and RSA sessions with VIRTIO_CRYPTO_RSA_RAW_PADDING MUST not be used
for verification and signing operations.
\item If VIRTIO_CRYPTO_RSA_PKCS1_PADDING is specified, EMSA-PKCS1-v1_5 padding method
is used (see \hyperref[intro:rfc3447]{PKCS\#1}), \field{hash_algo} specifies how the
digest of the data passed to RSA sessions is calculated when verifying and signing.
It only affects the padding algorithm and is ignored during encryption and decryption.
\end{itemize*}

The ECC algorithms such as the ECDSA algorithm, cannot use custom curves, only the
following known curves can be used (see \hyperref[intro:NIST]{NIST-recommended curves}).

\begin{lstlisting}
#define VIRTIO_CRYPTO_CURVE_UNKNOWN   0
#define VIRTIO_CRYPTO_CURVE_NIST_P192 1
#define VIRTIO_CRYPTO_CURVE_NIST_P224 2
#define VIRTIO_CRYPTO_CURVE_NIST_P256 3
#define VIRTIO_CRYPTO_CURVE_NIST_P384 4
#define VIRTIO_CRYPTO_CURVE_NIST_P521 5
\end{lstlisting}

For the ECDSA algorithm, the extra parameters are as follows:
\begin{lstlisting}
struct virtio_crypto_ecdsa_session_para {
    /* See VIRTIO_CRYPTO_CURVE_* above */
    le32 curve_id;
};
\end{lstlisting}

The fixed-length and the variable-length parameters of AKCIPHER session requests are as follows:
\begin{lstlisting}
struct virtio_crypto_akcipher_create_session_flf {
    /* Device read only portion */

    /* See VIRTIO_CRYPTO_AKCIPHER_* above */
    le32 algo;
#define VIRTIO_CRYPTO_AKCIPHER_KEY_TYPE_PUBLIC 1
#define VIRTIO_CRYPTO_AKCIPHER_KEY_TYPE_PRIVATE 2
    le32 key_type;
    /* length of key */
    le32 key_len;

#define VIRTIO_CRYPTO_AKCIPHER_SESS_ALGO_SPEC_HDR_SIZE 44
    u8 algo_flf[VIRTIO_CRYPTO_AKCIPHER_SESS_ALGO_SPEC_HDR_SIZE];
};

struct virtio_crypto_akcipher_create_session_vlf {
    /* Device read only portion */
    u8 key[key_len];
};
\end{lstlisting}

\field{algo} decides the type used by \field{algo_flf}.
\field{algo_flf} is fixed to 44 bytes and MUST contains of be one the
following structures:
\begin{itemize*}
\item struct virtio_crypto_rsa_session_para
\item struct virtio_crypto_ecdsa_session_para
\end{itemize*}

The length of \field{key} is specified in \field{key_len} in the struct
virtio_crypto_akcipher_create_session_flf.

For the RSA algorithm, the key needs to be encoded according to
\hyperref[intro:rfc3447]{PKCS\#1}. The private key is described with the
RSAPrivateKey structure, and the public key is described with the RSAPublicKey
structure. These ASN.1 structures are encoded in DER encoding rules (see
\hyperref[intro:rfc6025]{rfc6025}).

\begin{lstlisting}
RSAPrivateKey ::= SEQUENCE {
    version          INTEGER,
    modulus          INTEGER,
    publicExponent   INTEGER,
    privateExponent  INTEGER,
    prime1           INTEGER,
    prime2           INTEGER,
    exponent1        INTEGER,
    exponent1        INTEGER,
    coefficient      INTEGER,
    otherPrimeInfos  OtherPrimeInfos OPTIONAL
}

OtherPrimeInfos ::= SEQUENCE SIZE(1...MAX) OF OtherPrimeInfo

OtherPrimeINfo ::= SEQUENCE {
    prime           INTEGER,
    exponent        INTEGER,
    coefficient     INTEGER
}

RSAPublicKey ::= SEQUENCE {
    modulus         INTEGER,
    publicExponent  INTEGER
}
\end{lstlisting}

For the ECDSA algorithm, the private key is encoded according to
\hyperref[intro:rfc5915]{RFC5915}, the private key of the ECDSA algorithm
is described by the ASN.1 structure ECPrivateKey and encoded with DER
encoding rules (see \hyperref[intro:rfc6025]{rfc6025}).

\begin{lstlisting}
ECPrivateKey ::= SEQUNCE {
    version         INTEGER,
    privateKey      OCTET STRING,
    parameters [0]  ECParameters {{ NamedCurve }} OPTIONAL,
    publicKey  [1]  BIT STRING OPTIONAL
}
\end{lstlisting}

The public key of the ECDSA algorithm is encoded according to \hyperref[intro:SEC1]{SEC1},
and the public key of ECDSA is described by the ASN.1 structure ECPoint.
When initializing a session with ECDSA public key, the ECPoint is DER encoded and the
\field{key} only contains the value part of ECPoint, that is, the header part of the
OCTET STRING will be omitted (see \hyperref[intro:rfc6025]{rfc6025}).

\begin{lstlisting}
ECPoint ::= OCTET STRING
\end{lstlisting}

The length of \field{key} is specified in \field{key_len} in
struct virtio_crypto_akcipher_create_session_flf.

\drivernormative{\subparagraph}{Session operation: create session}{Device Types / Crypto Device / Device
Operation / Control Virtqueue / Session operation / Session operation: create session}

\begin{itemize*}
\item The driver MUST set the \field{opcode} field based on service type: CIPHER, HASH, MAC, AEAD or AKCIPHER.
\item The driver MUST set the control general header, the opcode specific header,
    the opcode specific extra parameters and the opcode specific outcome buffer in turn.
    See \ref{sec:Device Types / Crypto Device / Device Operation / Control Virtqueue}.
\item The driver MUST set the \field{reversed} field to zero.
\end{itemize*}

\devicenormative{\subparagraph}{Session operation: create session}{Device Types / Crypto Device / Device
Operation / Control Virtqueue / Session operation / Session operation: create session}

\begin{itemize*}
\item The device MUST use the corresponding opcode specific structure according to the
    \field{opcode} in the control general header.
\item The device MUST extract extra parameters according to the structures used.
\item The device MUST set the \field{status} field to one of the following values of enum
    VIRTIO_CRYPTO_STATUS after finish a session creation:
\begin{itemize*}
\item VIRTIO_CRYPTO_OK if a session is created successfully.
\item VIRTIO_CRYPTO_NOTSUPP if the requested algorithm or operation is unsupported.
\item VIRTIO_CRYPTO_NOSPC if no free session ID (only when the VIRTIO_CRYPTO_F_REVISION_1
    feature bit is negotiated).
\item VIRTIO_CRYPTO_ERR if failure not mentioned above occurs.
\end{itemize*}
\item The device MUST set the \field{session_id} field to a unique session identifier only
    if the status is set to VIRTIO_CRYPTO_OK.
\end{itemize*}

\drivernormative{\subparagraph}{Session operation: destroy session}{Device Types / Crypto Device / Device
Operation / Control Virtqueue / Session operation / Session operation: destroy session}

\begin{itemize*}
\item The driver MUST set the \field{opcode} field based on service type: CIPHER, HASH, MAC, AEAD or AKCIPHER.
\item The driver MUST set the \field{session_id} to a valid value assigned by the device
    when the session was created.
\end{itemize*}

\devicenormative{\subparagraph}{Session operation: destroy session}{Device Types / Crypto Device / Device
Operation / Control Virtqueue / Session operation / Session operation: destroy session}

\begin{itemize*}
\item The device MUST set the \field{status} field to one of the following values of enum VIRTIO_CRYPTO_STATUS.
\begin{itemize*}
\item VIRTIO_CRYPTO_OK if a session is created successfully.
\item VIRTIO_CRYPTO_ERR if any failure occurs.
\end{itemize*}
\end{itemize*}


\subsubsection{Data Virtqueue}\label{sec:Device Types / Crypto Device / Device Operation / Data Virtqueue}

The driver uses the data virtqueues to transmit crypto operation requests to the device,
and completes the crypto operations.

The header for dataq is as follows:

\begin{lstlisting}
struct virtio_crypto_op_header {
#define VIRTIO_CRYPTO_CIPHER_ENCRYPT \
    VIRTIO_CRYPTO_OPCODE(VIRTIO_CRYPTO_SERVICE_CIPHER, 0x00)
#define VIRTIO_CRYPTO_CIPHER_DECRYPT \
    VIRTIO_CRYPTO_OPCODE(VIRTIO_CRYPTO_SERVICE_CIPHER, 0x01)
#define VIRTIO_CRYPTO_HASH \
    VIRTIO_CRYPTO_OPCODE(VIRTIO_CRYPTO_SERVICE_HASH, 0x00)
#define VIRTIO_CRYPTO_MAC \
    VIRTIO_CRYPTO_OPCODE(VIRTIO_CRYPTO_SERVICE_MAC, 0x00)
#define VIRTIO_CRYPTO_AEAD_ENCRYPT \
    VIRTIO_CRYPTO_OPCODE(VIRTIO_CRYPTO_SERVICE_AEAD, 0x00)
#define VIRTIO_CRYPTO_AEAD_DECRYPT \
    VIRTIO_CRYPTO_OPCODE(VIRTIO_CRYPTO_SERVICE_AEAD, 0x01)
#define VIRTIO_CRYPTO_AKCIPHER_ENCRYPT \
    VIRTIO_CRYPTO_OPCODE(VIRTIO_CRYPTO_SERVICE_AKCIPHER, 0x00)
#define VIRTIO_CRYPTO_AKCIPHER_DECRYPT \
    VIRTIO_CRYPTO_OPCODE(VIRTIO_CRYPTO_SERVICE_AKCIPHER, 0x01)
#define VIRTIO_CRYPTO_AKCIPHER_SIGN \
    VIRTIO_CRYPTO_OPCODE(VIRTIO_CRYPTO_SERVICE_AKCIPHER, 0x02)
#define VIRTIO_CRYPTO_AKCIPHER_VERIFY \
    VIRTIO_CRYPTO_OPCODE(VIRTIO_CRYPTO_SERVICE_AKCIPHER, 0x03)
    le32 opcode;
    /* algo should be service-specific algorithms */
    le32 algo;
    le64 session_id;
#define VIRTIO_CRYPTO_FLAG_SESSION_MODE 1
    /* control flag to control the request */
    le32 flag;
    le32 padding;
};
\end{lstlisting}

\begin{note}
If VIRTIO_CRYPTO_F_REVISION_1 is not negotiated the \field{flag} is ignored.

If VIRTIO_CRYPTO_F_REVISION_1 is negotiated but VIRTIO_CRYPTO_F_<SERVICE>_STATELESS_MODE
is not negotiated, then the device SHOULD reject <SERVICE> requests if
VIRTIO_CRYPTO_FLAG_SESSION_MODE is not set (in \field{flag}).
\end{note}

The dataq request is composed of four parts:
\begin{lstlisting}
struct virtio_crypto_op_data_req {
    /* Device read only portion */

    struct virtio_crypto_op_header header;

#define VIRTIO_CRYPTO_DATAQ_OP_SPEC_HDR_LEGACY 48
    /* fixed length fields, opcode specific */
    u8 op_flf[flf_len];

    /* Device read && write portion */
    /* variable length fields, opcode specific */
    u8 op_vlf[vlf_len];

    /* Device write only portion */
    struct virtio_crypto_inhdr inhdr;
};
\end{lstlisting}

\field{header} is a general header (see above).

\field{op_flf} is the opcode (in \field{header}) specific header.

\field{flf_len} depends on the VIRTIO_CRYPTO_F_REVISION_1 feature bit
(see below).

\field{op_vlf} is the opcode (in \field{header}) specific parameters.

\field{vlf_len} is the size of the specific structure used.

\begin{itemize*}
\item If the the opcode (in \field{header}) is VIRTIO_CRYPTO_CIPHER_ENCRYPT
    or VIRTIO_CRYPTO_CIPHER_DECRYPT then:
    \begin{itemize*}
    \item If VIRTIO_CRYPTO_F_CIPHER_STATELESS_MODE is negotiated, \field{op_flf} is
        struct virtio_crypto_sym_data_flf_stateless, and \field{op_vlf} is struct
        virtio_crypto_sym_data_vlf_stateless.
    \item If VIRTIO_CRYPTO_F_CIPHER_STATELESS_MODE is NOT negotiated, \field{op_flf}
        is struct virtio_crypto_sym_data_flf if VIRTIO_CRYPTO_F_REVISION_1 is negotiated
        and struct virtio_crypto_sym_data_flf is padded to 48 bytes if NOT negotiated,
        and \field{op_vlf} is struct virtio_crypto_sym_data_vlf.
    \end{itemize*}
\item If the the opcode (in \field{header}) is VIRTIO_CRYPTO_HASH:
    \begin{itemize*}
    \item If VIRTIO_CRYPTO_F_HASH_STATELESS_MODE is negotiated, \field{op_flf} is
        struct virtio_crypto_hash_data_flf_stateless, and \field{op_vlf} is struct
        virtio_crypto_hash_data_vlf_stateless.
    \item If VIRTIO_CRYPTO_F_HASH_STATELESS_MODE is NOT negotiated, \field{op_flf}
        is struct virtio_crypto_hash_data_flf if VIRTIO_CRYPTO_F_REVISION_1 is negotiated
        and struct virtio_crypto_hash_data_flf is padded to 48 bytes if NOT negotiated,
        and \field{op_vlf} is struct virtio_crypto_hash_data_vlf.
    \end{itemize*}
\item If the the opcode (in \field{header}) is VIRTIO_CRYPTO_MAC:
    \begin{itemize*}
    \item If VIRTIO_CRYPTO_F_MAC_STATELESS_MODE is negotiated, \field{op_flf} is
        struct virtio_crypto_mac_data_flf_stateless, and \field{op_vlf} is struct
        virtio_crypto_mac_data_vlf_stateless.
    \item If VIRTIO_CRYPTO_F_MAC_STATELESS_MODE is NOT negotiated, \field{op_flf}
        is struct virtio_crypto_mac_data_flf if VIRTIO_CRYPTO_F_REVISION_1 is negotiated
        and struct virtio_crypto_mac_data_flf is padded to 48 bytes if NOT negotiated,
        and \field{op_vlf} is struct virtio_crypto_mac_data_vlf.
    \end{itemize*}
\item If the the opcode (in \field{header}) is VIRTIO_CRYPTO_AEAD_ENCRYPT
    or VIRTIO_CRYPTO_AEAD_DECRYPT then:
    \begin{itemize*}
    \item If VIRTIO_CRYPTO_F_AEAD_STATELESS_MODE is negotiated, \field{op_flf} is
        struct virtio_crypto_aead_data_flf_stateless, and \field{op_vlf} is struct
        virtio_crypto_aead_data_vlf_stateless.
    \item If VIRTIO_CRYPTO_F_AEAD_STATELESS_MODE is NOT negotiated, \field{op_flf}
        is struct virtio_crypto_aead_data_flf if VIRTIO_CRYPTO_F_REVISION_1 is negotiated
        and struct virtio_crypto_aead_data_flf is padded to 48 bytes if NOT negotiated,
        and \field{op_vlf} is struct virtio_crypto_aead_data_vlf.
    \end{itemize*}
\item If the opcode (in \field{header}) is VIRTIO_CRYPTO_AKCIPHER_ENCRYPT, VIRTIO_CRYPTO_AKCIPHER_DECRYPT,
    VIRTIO_CRYPTO_AKCIPHER_SIGN or VIRTIO_CRYPTO_AKCIPHER_VERIFY then:
    \begin{itemize*}
    \item If VIRTIO_CRYPTO_F_AKCIPHER_STATELESS_MODE is negotiated, \field{op_flf} is
        struct virtio_crypto_akcipher_data_flf_statless, and \field{op_vlf} is struct
        virtio_crypto_akcipher_data_vlf_stateless.
    \item If VIRTIO_CRYPTO_F_AKCIPHER_STATELESS_MODE is NOT negotiated, \field{op_flf}
        is struct virtio_crypto_akcipher_data_flf if VIRTIO_CRYPTO_F_REVISION_1 is negotiated
        and struct virtio_crypto_akcipher_data_flf is padded to 48 bytes if NOT negotiated,
        and \field{op_vlf} is struct virtio_crypto_akcipher_data_vlf.
    \end{itemize*}
\end{itemize*}

\field{inhdr} is a unified input header that used to return the status of
the operations, is defined as follows:

\begin{lstlisting}
struct virtio_crypto_inhdr {
    u8 status;
};
\end{lstlisting}

\subsubsection{HASH Service Operation}\label{sec:Device Types / Crypto Device / Device Operation / HASH Service Operation}

Session mode HASH service requests are as follows:

\begin{lstlisting}
struct virtio_crypto_hash_data_flf {
    /* length of source data */
    le32 src_data_len;
    /* hash result length */
    le32 hash_result_len;
};

struct virtio_crypto_hash_data_vlf {
    /* Device read only portion */
    /* Source data */
    u8 src_data[src_data_len];

    /* Device write only portion */
    /* Hash result data */
    u8 hash_result[hash_result_len];
};
\end{lstlisting}

Each data request uses the virtio_crypto_hash_data_flf structure and the
virtio_crypto_hash_data_vlf structure to store information used to run the
HASH operations.

\field{src_data} is the source data that will be processed.
\field{src_data_len} is the length of source data.
\field{hash_result} is the result data and \field{hash_result_len} is the length
of it.

Stateless mode HASH service requests are as follows:

\begin{lstlisting}
struct virtio_crypto_hash_data_flf_stateless {
    struct {
        /* See VIRTIO_CRYPTO_HASH_* above */
        le32 algo;
    } sess_para;

    /* length of source data */
    le32 src_data_len;
    /* hash result length */
    le32 hash_result_len;
    le32 reserved;
};
struct virtio_crypto_hash_data_vlf_stateless {
    /* Device read only portion */
    /* Source data */
    u8 src_data[src_data_len];

    /* Device write only portion */
    /* Hash result data */
    u8 hash_result[hash_result_len];
};
\end{lstlisting}

\drivernormative{\paragraph}{HASH Service Operation}{Device Types / Crypto Device / Device Operation / HASH Service Operation}

\begin{itemize*}
\item If the driver uses the session mode, then the driver MUST set \field{session_id}
    in struct virtio_crypto_op_header to a valid value assigned by the device when the
    session was created.
\item If the VIRTIO_CRYPTO_F_HASH_STATELESS_MODE feature bit is negotiated, 1) if the
    driver uses the stateless mode, then the driver MUST set the \field{flag} field in
    struct virtio_crypto_op_header to ZERO and MUST set the fields in struct
    virtio_crypto_hash_data_flf_stateless.sess_para, 2) if the driver uses the session
    mode, then the driver MUST set the \field{flag} field in struct virtio_crypto_op_header
    to VIRTIO_CRYPTO_FLAG_SESSION_MODE.
\item The driver MUST set \field{opcode} in struct virtio_crypto_op_header to VIRTIO_CRYPTO_HASH.
\end{itemize*}

\devicenormative{\paragraph}{HASH Service Operation}{Device Types / Crypto Device / Device Operation / HASH Service Operation}

\begin{itemize*}
\item The device MUST use the corresponding structure according to the \field{opcode}
    in the data general header.
\item If the VIRTIO_CRYPTO_F_HASH_STATELESS_MODE feature bit is negotiated, the device
    MUST parse \field{flag} field in struct virtio_crypto_op_header in order to decide
    which mode the driver uses.
\item The device MUST copy the results of HASH operations in the hash_result[] if HASH
    operations success.
\item The device MUST set \field{status} in struct virtio_crypto_inhdr to one of the
    following values of enum VIRTIO_CRYPTO_STATUS:
\begin{itemize*}
\item VIRTIO_CRYPTO_OK if the operation success.
\item VIRTIO_CRYPTO_NOTSUPP if the requested algorithm or operation is unsupported.
\item VIRTIO_CRYPTO_INVSESS if the session ID invalid when in session mode.
\item VIRTIO_CRYPTO_ERR if any failure not mentioned above occurs.
\end{itemize*}
\end{itemize*}


\subsubsection{MAC Service Operation}\label{sec:Device Types / Crypto Device / Device Operation / MAC Service Operation}

Session mode MAC service requests are as follows:

\begin{lstlisting}
struct virtio_crypto_mac_data_flf {
    struct virtio_crypto_hash_data_flf hdr;
};

struct virtio_crypto_mac_data_vlf {
    /* Device read only portion */
    /* Source data */
    u8 src_data[src_data_len];

    /* Device write only portion */
    /* Hash result data */
    u8 hash_result[hash_result_len];
};
\end{lstlisting}

Each request uses the virtio_crypto_mac_data_flf structure and the
virtio_crypto_mac_data_vlf structure to store information used to run the
MAC operations.

\field{src_data} is the source data that will be processed.
\field{src_data_len} is the length of source data.
\field{hash_result} is the result data and \field{hash_result_len} is the length
of it.

Stateless mode MAC service requests are as follows:

\begin{lstlisting}
struct virtio_crypto_mac_data_flf_stateless {
    struct {
        /* See VIRTIO_CRYPTO_MAC_* above */
        le32 algo;
        /* length of authenticated key */
        le32 auth_key_len;
    } sess_para;

    /* length of source data */
    le32 src_data_len;
    /* hash result length */
    le32 hash_result_len;
};

struct virtio_crypto_mac_data_vlf_stateless {
    /* Device read only portion */
    /* The authenticated key */
    u8 auth_key[auth_key_len];
    /* Source data */
    u8 src_data[src_data_len];

    /* Device write only portion */
    /* Hash result data */
    u8 hash_result[hash_result_len];
};
\end{lstlisting}

\field{auth_key} is the authenticated key that will be used during the process.
\field{auth_key_len} is the length of the key.

\drivernormative{\paragraph}{MAC Service Operation}{Device Types / Crypto Device / Device Operation / MAC Service Operation}

\begin{itemize*}
\item If the driver uses the session mode, then the driver MUST set \field{session_id}
    in struct virtio_crypto_op_header to a valid value assigned by the device when the
    session was created.
\item If the VIRTIO_CRYPTO_F_MAC_STATELESS_MODE feature bit is negotiated, 1) if the
    driver uses the stateless mode, then the driver MUST set the \field{flag} field
    in struct virtio_crypto_op_header to ZERO and MUST set the fields in struct
    virtio_crypto_mac_data_flf_stateless.sess_para, 2) if the driver uses the session
    mode, then the driver MUST set the \field{flag} field in struct virtio_crypto_op_header
    to VIRTIO_CRYPTO_FLAG_SESSION_MODE.
\item The driver MUST set \field{opcode} in struct virtio_crypto_op_header to VIRTIO_CRYPTO_MAC.
\end{itemize*}

\devicenormative{\paragraph}{MAC Service Operation}{Device Types / Crypto Device / Device Operation / MAC Service Operation}

\begin{itemize*}
\item If the VIRTIO_CRYPTO_F_MAC_STATELESS_MODE feature bit is negotiated, the device
    MUST parse \field{flag} field in struct virtio_crypto_op_header in order to decide
	which mode the driver uses.
\item The device MUST copy the results of MAC operations in the hash_result[] if HASH
    operations success.
\item The device MUST set \field{status} in struct virtio_crypto_inhdr to one of the
    following values of enum VIRTIO_CRYPTO_STATUS:
\begin{itemize*}
\item VIRTIO_CRYPTO_OK if the operation success.
\item VIRTIO_CRYPTO_NOTSUPP if the requested algorithm or operation is unsupported.
\item VIRTIO_CRYPTO_INVSESS if the session ID invalid when in session mode.
\item VIRTIO_CRYPTO_ERR if any failure not mentioned above occurs.
\end{itemize*}
\end{itemize*}

\subsubsection{Symmetric algorithms Operation}\label{sec:Device Types / Crypto Device / Device Operation / Symmetric algorithms Operation}

Session mode CIPHER service requests are as follows:

\begin{lstlisting}
struct virtio_crypto_cipher_data_flf {
    /*
     * Byte Length of valid IV/Counter data pointed to by the below iv data.
     *
     * For block ciphers in CBC or F8 mode, or for Kasumi in F8 mode, or for
     *   SNOW3G in UEA2 mode, this is the length of the IV (which
     *   must be the same as the block length of the cipher).
     * For block ciphers in CTR mode, this is the length of the counter
     *   (which must be the same as the block length of the cipher).
     */
    le32 iv_len;
    /* length of source data */
    le32 src_data_len;
    /* length of destination data */
    le32 dst_data_len;
    le32 padding;
};

struct virtio_crypto_cipher_data_vlf {
    /* Device read only portion */

    /*
     * Initialization Vector or Counter data.
     *
     * For block ciphers in CBC or F8 mode, or for Kasumi in F8 mode, or for
     *   SNOW3G in UEA2 mode, this is the Initialization Vector (IV)
     *   value.
     * For block ciphers in CTR mode, this is the counter.
     * For AES-XTS, this is the 128bit tweak, i, from IEEE Std 1619-2007.
     *
     * The IV/Counter will be updated after every partial cryptographic
     * operation.
     */
    u8 iv[iv_len];
    /* Source data */
    u8 src_data[src_data_len];

    /* Device write only portion */
    /* Destination data */
    u8 dst_data[dst_data_len];
};
\end{lstlisting}

Session mode requests of algorithm chaining are as follows:

\begin{lstlisting}
struct virtio_crypto_alg_chain_data_flf {
    le32 iv_len;
    /* Length of source data */
    le32 src_data_len;
    /* Length of destination data */
    le32 dst_data_len;
    /* Starting point for cipher processing in source data */
    le32 cipher_start_src_offset;
    /* Length of the source data that the cipher will be computed on */
    le32 len_to_cipher;
    /* Starting point for hash processing in source data */
    le32 hash_start_src_offset;
    /* Length of the source data that the hash will be computed on */
    le32 len_to_hash;
    /* Length of the additional auth data */
    le32 aad_len;
    /* Length of the hash result */
    le32 hash_result_len;
    le32 reserved;
};

struct virtio_crypto_alg_chain_data_vlf {
    /* Device read only portion */

    /* Initialization Vector or Counter data */
    u8 iv[iv_len];
    /* Source data */
    u8 src_data[src_data_len];
    /* Additional authenticated data if exists */
    u8 aad[aad_len];

    /* Device write only portion */

    /* Destination data */
    u8 dst_data[dst_data_len];
    /* Hash result data */
    u8 hash_result[hash_result_len];
};
\end{lstlisting}

Session mode requests of symmetric algorithm are as follows:

\begin{lstlisting}
struct virtio_crypto_sym_data_flf {
    /* Device read only portion */

#define VIRTIO_CRYPTO_SYM_DATA_REQ_HDR_SIZE    40
    u8 op_type_flf[VIRTIO_CRYPTO_SYM_DATA_REQ_HDR_SIZE];

    /* See above VIRTIO_CRYPTO_SYM_OP_* */
    le32 op_type;
    le32 padding;
};

struct virtio_crypto_sym_data_vlf {
    u8 op_type_vlf[sym_para_len];
};
\end{lstlisting}

Each request uses the virtio_crypto_sym_data_flf structure and the
virtio_crypto_sym_data_flf structure to store information used to run the
CIPHER operations.

\field{op_type_flf} is the \field{op_type} specific header, it MUST starts
with or be one of the following structures:
\begin{itemize*}
\item struct virtio_crypto_cipher_data_flf
\item struct virtio_crypto_alg_chain_data_flf
\end{itemize*}

The length of \field{op_type_flf} is fixed to 40 bytes, the data of unused
part (if has) will be ignored.

\field{op_type_vlf} is the \field{op_type} specific parameters, it MUST starts
with or be one of the following structures:
\begin{itemize*}
\item struct virtio_crypto_cipher_data_vlf
\item struct virtio_crypto_alg_chain_data_vlf
\end{itemize*}

\field{sym_para_len} is the size of the specific structure used.

Stateless mode CIPHER service requests are as follows:

\begin{lstlisting}
struct virtio_crypto_cipher_data_flf_stateless {
    struct {
        /* See VIRTIO_CRYPTO_CIPHER* above */
        le32 algo;
        /* length of key */
        le32 key_len;

        /* See VIRTIO_CRYPTO_OP_* above */
        le32 op;
    } sess_para;

    /*
     * Byte Length of valid IV/Counter data pointed to by the below iv data.
     */
    le32 iv_len;
    /* length of source data */
    le32 src_data_len;
    /* length of destination data */
    le32 dst_data_len;
};

struct virtio_crypto_cipher_data_vlf_stateless {
    /* Device read only portion */

    /* The cipher key */
    u8 cipher_key[key_len];

    /* Initialization Vector or Counter data. */
    u8 iv[iv_len];
    /* Source data */
    u8 src_data[src_data_len];

    /* Device write only portion */
    /* Destination data */
    u8 dst_data[dst_data_len];
};
\end{lstlisting}

Stateless mode requests of algorithm chaining are as follows:

\begin{lstlisting}
struct virtio_crypto_alg_chain_data_flf_stateless {
    struct {
        /* See VIRTIO_CRYPTO_SYM_ALG_CHAIN_ORDER_* above */
        le32 alg_chain_order;
        /* length of the additional authenticated data in bytes */
        le32 aad_len;

        struct {
            /* See VIRTIO_CRYPTO_CIPHER* above */
            le32 algo;
            /* length of key */
            le32 key_len;
            /* See VIRTIO_CRYPTO_OP_* above */
            le32 op;
        } cipher;

        struct {
            /* See VIRTIO_CRYPTO_HASH_* or VIRTIO_CRYPTO_MAC_* above */
            le32 algo;
            /* length of authenticated key */
            le32 auth_key_len;
            /* See VIRTIO_CRYPTO_SYM_HASH_MODE_* above */
            le32 hash_mode;
        } hash;
    } sess_para;

    le32 iv_len;
    /* Length of source data */
    le32 src_data_len;
    /* Length of destination data */
    le32 dst_data_len;
    /* Starting point for cipher processing in source data */
    le32 cipher_start_src_offset;
    /* Length of the source data that the cipher will be computed on */
    le32 len_to_cipher;
    /* Starting point for hash processing in source data */
    le32 hash_start_src_offset;
    /* Length of the source data that the hash will be computed on */
    le32 len_to_hash;
    /* Length of the additional auth data */
    le32 aad_len;
    /* Length of the hash result */
    le32 hash_result_len;
    le32 reserved;
};

struct virtio_crypto_alg_chain_data_vlf_stateless {
    /* Device read only portion */

    /* The cipher key */
    u8 cipher_key[key_len];
    /* The auth key */
    u8 auth_key[auth_key_len];
    /* Initialization Vector or Counter data */
    u8 iv[iv_len];
    /* Additional authenticated data if exists */
    u8 aad[aad_len];
    /* Source data */
    u8 src_data[src_data_len];

    /* Device write only portion */

    /* Destination data */
    u8 dst_data[dst_data_len];
    /* Hash result data */
    u8 hash_result[hash_result_len];
};
\end{lstlisting}

Stateless mode requests of symmetric algorithm are as follows:

\begin{lstlisting}
struct virtio_crypto_sym_data_flf_stateless {
    /* Device read only portion */
#define VIRTIO_CRYPTO_SYM_DATE_REQ_HDR_STATELESS_SIZE    72
    u8 op_type_flf[VIRTIO_CRYPTO_SYM_DATE_REQ_HDR_STATELESS_SIZE];

    /* Device write only portion */
    /* See above VIRTIO_CRYPTO_SYM_OP_* */
    le32 op_type;
};

struct virtio_crypto_sym_data_vlf_stateless {
    u8 op_type_vlf[sym_para_len];
};
\end{lstlisting}

\field{op_type_flf} is the \field{op_type} specific header, it MUST starts
with or be one of the following structures:
\begin{itemize*}
\item struct virtio_crypto_cipher_data_flf_stateless
\item struct virtio_crypto_alg_chain_data_flf_stateless
\end{itemize*}

The length of \field{op_type_flf} is fixed to 72 bytes, the data of unused
part (if has) will be ignored.

\field{op_type_vlf} is the \field{op_type} specific parameters, it MUST starts
with or be one of the following structures:
\begin{itemize*}
\item struct virtio_crypto_cipher_data_vlf_stateless
\item struct virtio_crypto_alg_chain_data_vlf_stateless
\end{itemize*}

\field{sym_para_len} is the size of the specific structure used.

\drivernormative{\paragraph}{Symmetric algorithms Operation}{Device Types / Crypto Device / Device Operation / Symmetric algorithms Operation}

\begin{itemize*}
\item If the driver uses the session mode, then the driver MUST set \field{session_id}
    in struct virtio_crypto_op_header to a valid value assigned by the device when the
    session was created.
\item If the VIRTIO_CRYPTO_F_CIPHER_STATELESS_MODE feature bit is negotiated, 1) if the
    driver uses the stateless mode, then the driver MUST set the \field{flag} field in
    struct virtio_crypto_op_header to ZERO and MUST set the fields in struct
    virtio_crypto_cipher_data_flf_stateless.sess_para or struct
    virtio_crypto_alg_chain_data_flf_stateless.sess_para, 2) if the driver uses the
    session mode, then the driver MUST set the \field{flag} field in struct
    virtio_crypto_op_header to VIRTIO_CRYPTO_FLAG_SESSION_MODE.
\item The driver MUST set the \field{opcode} field in struct virtio_crypto_op_header
    to VIRTIO_CRYPTO_CIPHER_ENCRYPT or VIRTIO_CRYPTO_CIPHER_DECRYPT.
\item The driver MUST specify the fields of struct virtio_crypto_cipher_data_flf in
    struct virtio_crypto_sym_data_flf and struct virtio_crypto_cipher_data_vlf in
    struct virtio_crypto_sym_data_vlf if the request is based on VIRTIO_CRYPTO_SYM_OP_CIPHER.
\item The driver MUST specify the fields of struct virtio_crypto_alg_chain_data_flf
    in struct virtio_crypto_sym_data_flf and struct virtio_crypto_alg_chain_data_vlf
    in struct virtio_crypto_sym_data_vlf if the request is of the VIRTIO_CRYPTO_SYM_OP_ALGORITHM_CHAINING
    type.
\end{itemize*}

\devicenormative{\paragraph}{Symmetric algorithms Operation}{Device Types / Crypto Device / Device Operation / Symmetric algorithms Operation}

\begin{itemize*}
\item If the VIRTIO_CRYPTO_F_CIPHER_STATELESS_MODE feature bit is negotiated, the device
    MUST parse \field{flag} field in struct virtio_crypto_op_header in order to decide
	which mode the driver uses.
\item The device MUST parse the virtio_crypto_sym_data_req based on the \field{opcode}
    field in general header.
\item The device MUST parse the fields of struct virtio_crypto_cipher_data_flf in
    struct virtio_crypto_sym_data_flf and struct virtio_crypto_cipher_data_vlf in
    struct virtio_crypto_sym_data_vlf if the request is based on VIRTIO_CRYPTO_SYM_OP_CIPHER.
\item The device MUST parse the fields of struct virtio_crypto_alg_chain_data_flf
    in struct virtio_crypto_sym_data_flf and struct virtio_crypto_alg_chain_data_vlf
    in struct virtio_crypto_sym_data_vlf if the request is of the VIRTIO_CRYPTO_SYM_OP_ALGORITHM_CHAINING
    type.
\item The device MUST copy the result of cryptographic operation in the dst_data[] in
    both plain CIPHER mode and algorithms chain mode.
\item The device MUST check the \field{para}.\field{add_len} is bigger than 0 before
    parse the additional authenticated data in plain algorithms chain mode.
\item The device MUST copy the result of HASH/MAC operation in the hash_result[] is
    of the VIRTIO_CRYPTO_SYM_OP_ALGORITHM_CHAINING type.
\item The device MUST set the \field{status} field in struct virtio_crypto_inhdr to
    one of the following values of enum VIRTIO_CRYPTO_STATUS:
\begin{itemize*}
\item VIRTIO_CRYPTO_OK if the operation success.
\item VIRTIO_CRYPTO_NOTSUPP if the requested algorithm or operation is unsupported.
\item VIRTIO_CRYPTO_INVSESS if the session ID is invalid in session mode.
\item VIRTIO_CRYPTO_ERR if failure not mentioned above occurs.
\end{itemize*}
\end{itemize*}

\subsubsection{AEAD Service Operation}\label{sec:Device Types / Crypto Device / Device Operation / AEAD Service Operation}

Session mode requests of symmetric algorithm are as follows:

\begin{lstlisting}
struct virtio_crypto_aead_data_flf {
    /*
     * Byte Length of valid IV data.
     *
     * For GCM mode, this is either 12 (for 96-bit IVs) or 16, in which
     *   case iv points to J0.
     * For CCM mode, this is the length of the nonce, which can be in the
     *   range 7 to 13 inclusive.
     */
    le32 iv_len;
    /* length of additional auth data */
    le32 aad_len;
    /* length of source data */
    le32 src_data_len;
    /* length of dst data, this should be at least src_data_len + tag_len */
    le32 dst_data_len;
    /* Authentication tag length */
    le32 tag_len;
    le32 reserved;
};

struct virtio_crypto_aead_data_vlf {
    /* Device read only portion */

    /*
     * Initialization Vector data.
     *
     * For GCM mode, this is either the IV (if the length is 96 bits) or J0
     *   (for other sizes), where J0 is as defined by NIST SP800-38D.
     *   Regardless of the IV length, a full 16 bytes needs to be allocated.
     * For CCM mode, the first byte is reserved, and the nonce should be
     *   written starting at &iv[1] (to allow space for the implementation
     *   to write in the flags in the first byte).  Note that a full 16 bytes
     *   should be allocated, even though the iv_len field will have
     *   a value less than this.
     *
     * The IV will be updated after every partial cryptographic operation.
     */
    u8 iv[iv_len];
    /* Source data */
    u8 src_data[src_data_len];
    /* Additional authenticated data if exists */
    u8 aad[aad_len];

    /* Device write only portion */
    /* Pointer to output data */
    u8 dst_data[dst_data_len];
};
\end{lstlisting}

Each request uses the virtio_crypto_aead_data_flf structure and the
virtio_crypto_aead_data_flf structure to store information used to run the
AEAD operations.

Stateless mode AEAD service requests are as follows:

\begin{lstlisting}
struct virtio_crypto_aead_data_flf_stateless {
    struct {
        /* See VIRTIO_CRYPTO_AEAD_* above */
        le32 algo;
        /* length of key */
        le32 key_len;
        /* encrypt or decrypt, See above VIRTIO_CRYPTO_OP_* */
        le32 op;
    } sess_para;

    /* Byte Length of valid IV data. */
    le32 iv_len;
    /* Authentication tag length */
    le32 tag_len;
    /* length of additional auth data */
    le32 aad_len;
    /* length of source data */
    le32 src_data_len;
    /* length of dst data, this should be at least src_data_len + tag_len */
    le32 dst_data_len;
};

struct virtio_crypto_aead_data_vlf_stateless {
    /* Device read only portion */

    /* The cipher key */
    u8 key[key_len];
    /* Initialization Vector data. */
    u8 iv[iv_len];
    /* Source data */
    u8 src_data[src_data_len];
    /* Additional authenticated data if exists */
    u8 aad[aad_len];

    /* Device write only portion */
    /* Pointer to output data */
    u8 dst_data[dst_data_len];
};
\end{lstlisting}

\drivernormative{\paragraph}{AEAD Service Operation}{Device Types / Crypto Device / Device Operation / AEAD Service Operation}

\begin{itemize*}
\item If the driver uses the session mode, then the driver MUST set
    \field{session_id} in struct virtio_crypto_op_header to a valid value assigned
    by the device when the session was created.
\item If the VIRTIO_CRYPTO_F_AEAD_STATELESS_MODE feature bit is negotiated, 1) if
    the driver uses the stateless mode, then the driver MUST set the \field{flag}
    field in struct virtio_crypto_op_header to ZERO and MUST set the fields in
    struct virtio_crypto_aead_data_flf_stateless.sess_para, 2) if the driver uses
    the session mode, then the driver MUST set the \field{flag} field in struct
    virtio_crypto_op_header to VIRTIO_CRYPTO_FLAG_SESSION_MODE.
\item The driver MUST set the \field{opcode} field in struct virtio_crypto_op_header
    to VIRTIO_CRYPTO_AEAD_ENCRYPT or VIRTIO_CRYPTO_AEAD_DECRYPT.
\end{itemize*}

\devicenormative{\paragraph}{AEAD Service Operation}{Device Types / Crypto Device / Device Operation / AEAD Service Operation}

\begin{itemize*}
\item If the VIRTIO_CRYPTO_F_AEAD_STATELESS_MODE feature bit is negotiated, the
    device MUST parse the virtio_crypto_aead_data_vlf_stateless based on the \field{opcode}
	field in general header.
\item The device MUST copy the result of cryptographic operation in the dst_data[].
\item The device MUST copy the authentication tag in the dst_data[] offset the cipher result.
\item The device MUST set the \field{status} field in struct virtio_crypto_inhdr to
    one of the following values of enum VIRTIO_CRYPTO_STATUS:
\item When the \field{opcode} field is VIRTIO_CRYPTO_AEAD_DECRYPT, the device MUST
    verify and return the verification result to the driver.
\begin{itemize*}
\item VIRTIO_CRYPTO_OK if the operation success.
\item VIRTIO_CRYPTO_NOTSUPP if the requested algorithm or operation is unsupported.
\item VIRTIO_CRYPTO_BADMSG if the verification result is incorrect.
\item VIRTIO_CRYPTO_INVSESS if the session ID invalid when in session mode.
\item VIRTIO_CRYPTO_ERR if any failure not mentioned above occurs.
\end{itemize*}
\end{itemize*}

\subsubsection{AKCIPHER Service Operation}\label{sec:Device Types / Crypto Device / Device Operation / AKCIPHER Service Operation}

Session mode AKCIPHER requests are as follows:

\begin{lstlisting}
struct virtio_crypto_akcipher_data_flf {
    /* length of source data */
    le32 src_data_len;
    /* length of dst data */
    le32 dst_data_len;
};

struct virtio_crypto_akcipher_data_vlf {
    /* Device read only portion */
    /* Source data */
    u8 src_data[src_data_len];

    /* Device write only portion */
    /* Pointer to output data */
    u8 dst_data[dst_data_len];
};
\end{lstlisting}

Each data request uses the virtio_crypto_akcipher_flf structure and the virtio_crypto_akcipher_data_vlf
structure to store information used to run the AKCIPHER operations.

For encryption, decryption, and signing:
\field{src_data} is the source data that will be processed, note that for signing operations,
src_data stores the data to be signed, which usually is the digest of some data rather than the
data itself.
\field{src_data_len} is the length of source data.
\field{dst_result} is the result data and \field{dst_data_len} is the length of it. Note that the
length of the result is not always exactly equal to dst_data_len, the driver needs to check how
many bytes the device has written and calculate the actual length of the result.

For verification:
\field{src_data_len} refers to the length of the signature, and \field{dst_data_len} refers to
the length of signed data, where the signed data is usually the digest of some data.
\field{src_data} is spliced by the signature and the signed data, the src_data with the lower
address stores the signature, and the higher address stores the signed data.
\field{dst_data} is always empty for verification.

Different algorithms have different signature formats.
For the RSA algorithm, the result is determined by the padding algorithm specified by
\field{padding_algo} in structure virtio_crypto_rsa_session_para.

For the ECDSA algorithm, the signature is composed of the following
ASN.1 structure (see \hyperref[intro:rfc3279]{RFC3279})
and MUST be DER encoded (see \hyperref[intro:rfc6025]{rfc6025}).

\begin{lstlisting}
Ecdsa-Sig-Value ::= SEQUENCE {
    r INTEGER,
    s INTEGER
}
\end{lstlisting}

Stateless mode AKCIPHER service requests are as follows:
\begin{lstlisting}
struct virtio_crypto_akcipher_data_flf_stateless {
    struct {
        /* See VIRTIO_CYRPTO_AKCIPHER* above */
        le32 algo;
        /* See VIRTIO_CRYPTO_AKCIPHER_KEY_TYPE_* above */
        le32 key_type;
        /* length of key */
        le32 key_len;

        /* algothrim specific parameters described above */
        union {
            struct virtio_crypto_rsa_session_para rsa;
            struct virtio_crypto_ecdsa_session_para ecdsa;
        } u;
    } sess_para;

    /* length of source data */
    le32 src_data_len;
    /* length of destination data */
    le32 dst_data_len;
};

struct virtio_crypto_akcipher_data_vlf_stateless {
    /* Device read only portion */
    u8 akcipher_key[key_len];

    /* Source data */
    u8 src_data[src_data_len];

    /* Device write only portion */
    u8 dst_data[dst_data_len];
};
\end{lstlisting}

In stateless mode, the format of key and signature, the meaning of src_data and dst_data, are all the same
with session mode.

\drivernormative{\paragraph}{AKCIPHER Service Operation}{Device Types / Crypto Device / Device Operation / AKCIPHER Service Operation}

\begin{itemize*}
\item If the driver uses the session mode, then the driver MUST set
    \field{session_id} in struct virtio_crypto_op_header to a valid
    value assigned by the device when the session was created.
\item If the VIRTIO_CRYPTO_F_AKCIPHER_STATELESS_MODE feature bit is negotiated, 1) if the
    driver uses the stateless mode, then the driver MUST set the \field{flag} field in
    struct virtio_crypto_op_header to ZERO and MUST set the fields in struct
    virtio_crypto_akcipher_flf_stateless.sess_para, 2) if the driver uses the session
    mode, then the driver MUST set the \field{flag} field in struct virtio_crypto_op_header
    to VIRTIO_CRYPTO_FLAG_SESSION_MODE.
\item The driver MUST set the \field{opcode} field in struct virtio_crypto_op_header
    to one of VIRTIO_CRYPTO_AKCIPHER_ENCRYPT, VIRTIO_CRYPTO_AKCIPHER_DESTROY_SESSION,
    VIRTIO_CRYPTO_AKCIPHER_SIGN, and VIRTIO_CRYPTO_AKCIPHER_VERIFY.
\end{itemize*}

\devicenormative{\paragraph}{AKCIPHER Service Operation}{Device Types / Crypto Device / Device Operation / AKCIPHER Service Operation}

\begin{itemize*}
\item If the VIRTIO_CRYPTO_F_AKCIPHER_STATELESS_MODE feature bit is negotiated, the
    device MUST parse the virtio_crypto_akcipher_data_vlf_stateless based on the \field{opcode}
    field in general header.
\item The device MUST copy the result of cryptographic operation in the dst_data[].
\item The device MUST set the \field{status} field in struct virtio_crypto_inhdr to
    one of the following values of enum VIRTIO_CRYPTO_STATUS:
\begin{itemize*}
\item VIRTIO_CRYPTO_OK if the operation success.
\item VIRTIO_CRYPTO_NOTSUPP if the requested algorithm or operation is unsupported.
\item VIRTIO_CRYPTO_BADMSG if the verification result is incorrect.
\item VIRTIO_CRYPTO_INVSESS if the session ID invalid when in session mode.
\item VIRTIO_CRYPTO_KEY_REJECTED if the signature verification failed.
\item VIRTIO_CRYPTO_ERR if any failure not mentioned above occurs.
\end{itemize*}
\end{itemize*}

\section{Crypto Device}\label{sec:Device Types / Crypto Device}

The virtio crypto device is a virtual cryptography device as well as a
virtual cryptographic accelerator. The virtio crypto device provides the
following crypto services: CIPHER, MAC, HASH, AEAD and AKCIPHER. Virtio crypto
devices have a single control queue and at least one data queue. Crypto
operation requests are placed into a data queue, and serviced by the
device. Some crypto operation requests are only valid in the context of a
session. The role of the control queue is facilitating control operation
requests. Sessions management is realized with control operation
requests.

\subsection{Device ID}\label{sec:Device Types / Crypto Device / Device ID}

20

\subsection{Virtqueues}\label{sec:Device Types / Crypto Device / Virtqueues}

\begin{description}
\item[0] dataq1
\item[\ldots]
\item[N-1] dataqN
\item[N] controlq
\end{description}

N is set by \field{max_dataqueues}.

\subsection{Feature bits}\label{sec:Device Types / Crypto Device / Feature bits}

\begin{description}
\item VIRTIO_CRYPTO_F_REVISION_1 (0) revision 1. Revision 1 has a specific
    request format and other enhancements (which result in some additional
    requirements).
\item VIRTIO_CRYPTO_F_CIPHER_STATELESS_MODE (1) stateless mode requests are
    supported by the CIPHER service.
\item VIRTIO_CRYPTO_F_HASH_STATELESS_MODE (2) stateless mode requests are
    supported by the HASH service.
\item VIRTIO_CRYPTO_F_MAC_STATELESS_MODE (3) stateless mode requests are
    supported by the MAC service.
\item VIRTIO_CRYPTO_F_AEAD_STATELESS_MODE (4) stateless mode requests are
    supported by the AEAD service.
\item VIRTIO_CRYPTO_F_AKCIPHER_STATELESS_MODE (5) stateless mode requests are
    supported by the AKCIPHER service.
\end{description}


\subsubsection{Feature bit requirements}\label{sec:Device Types / Crypto Device / Feature bit requirements}

Some crypto feature bits require other crypto feature bits
(see \ref{drivernormative:Basic Facilities of a Virtio Device / Feature Bits}):

\begin{description}
\item[VIRTIO_CRYPTO_F_CIPHER_STATELESS_MODE] Requires VIRTIO_CRYPTO_F_REVISION_1.
\item[VIRTIO_CRYPTO_F_HASH_STATELESS_MODE] Requires VIRTIO_CRYPTO_F_REVISION_1.
\item[VIRTIO_CRYPTO_F_MAC_STATELESS_MODE] Requires VIRTIO_CRYPTO_F_REVISION_1.
\item[VIRTIO_CRYPTO_F_AEAD_STATELESS_MODE] Requires VIRTIO_CRYPTO_F_REVISION_1.
\item[VIRTIO_CRYPTO_F_AKCIPHER_STATELESS_MODE] Requires VIRTIO_CRYPTO_F_REVISION_1.
\end{description}

\subsection{Supported crypto services}\label{sec:Device Types / Crypto Device / Supported crypto services}

The following crypto services are defined:

\begin{lstlisting}
/* CIPHER (Symmetric Key Cipher) service */
#define VIRTIO_CRYPTO_SERVICE_CIPHER 0
/* HASH service */
#define VIRTIO_CRYPTO_SERVICE_HASH   1
/* MAC (Message Authentication Codes) service */
#define VIRTIO_CRYPTO_SERVICE_MAC    2
/* AEAD (Authenticated Encryption with Associated Data) service */
#define VIRTIO_CRYPTO_SERVICE_AEAD   3
/* AKCIPHER (Asymmetric Key Cipher) service */
#define VIRTIO_CRYPTO_SERVICE_AKCIPHER 4
\end{lstlisting}

The above constants designate bits used to indicate the which of crypto services are
offered by the device as described in, see \ref{sec:Device Types / Crypto Device / Device configuration layout}.

\subsubsection{CIPHER services}\label{sec:Device Types / Crypto Device / Supported crypto services / CIPHER services}

The following CIPHER algorithms are defined:

\begin{lstlisting}
#define VIRTIO_CRYPTO_NO_CIPHER                 0
#define VIRTIO_CRYPTO_CIPHER_ARC4               1
#define VIRTIO_CRYPTO_CIPHER_AES_ECB            2
#define VIRTIO_CRYPTO_CIPHER_AES_CBC            3
#define VIRTIO_CRYPTO_CIPHER_AES_CTR            4
#define VIRTIO_CRYPTO_CIPHER_DES_ECB            5
#define VIRTIO_CRYPTO_CIPHER_DES_CBC            6
#define VIRTIO_CRYPTO_CIPHER_3DES_ECB           7
#define VIRTIO_CRYPTO_CIPHER_3DES_CBC           8
#define VIRTIO_CRYPTO_CIPHER_3DES_CTR           9
#define VIRTIO_CRYPTO_CIPHER_KASUMI_F8          10
#define VIRTIO_CRYPTO_CIPHER_SNOW3G_UEA2        11
#define VIRTIO_CRYPTO_CIPHER_AES_F8             12
#define VIRTIO_CRYPTO_CIPHER_AES_XTS            13
#define VIRTIO_CRYPTO_CIPHER_ZUC_EEA3           14
\end{lstlisting}

The above constants have two usages:
\begin{enumerate}
\item As bit numbers, used to tell the driver which CIPHER algorithms
are supported by the device, see \ref{sec:Device Types / Crypto Device / Device configuration layout}.
\item As values, used to designate the algorithm in (CIPHER type) crypto
operation requests, see \ref{sec:Device Types / Crypto Device / Device Operation / Control Virtqueue / Session operation}.
\end{enumerate}

\subsubsection{HASH services}\label{sec:Device Types / Crypto Device / Supported crypto services / HASH services}

The following HASH algorithms are defined:

\begin{lstlisting}
#define VIRTIO_CRYPTO_NO_HASH            0
#define VIRTIO_CRYPTO_HASH_MD5           1
#define VIRTIO_CRYPTO_HASH_SHA1          2
#define VIRTIO_CRYPTO_HASH_SHA_224       3
#define VIRTIO_CRYPTO_HASH_SHA_256       4
#define VIRTIO_CRYPTO_HASH_SHA_384       5
#define VIRTIO_CRYPTO_HASH_SHA_512       6
#define VIRTIO_CRYPTO_HASH_SHA3_224      7
#define VIRTIO_CRYPTO_HASH_SHA3_256      8
#define VIRTIO_CRYPTO_HASH_SHA3_384      9
#define VIRTIO_CRYPTO_HASH_SHA3_512      10
#define VIRTIO_CRYPTO_HASH_SHA3_SHAKE128      11
#define VIRTIO_CRYPTO_HASH_SHA3_SHAKE256      12
\end{lstlisting}

The above constants have two usages:
\begin{enumerate}
\item As bit numbers, used to tell the driver which HASH algorithms
are supported by the device, see \ref{sec:Device Types / Crypto Device / Device configuration layout}.
\item As values, used to designate the algorithm in (HASH type) crypto
operation requires, see \ref{sec:Device Types / Crypto Device / Device Operation / Control Virtqueue / Session operation}.
\end{enumerate}

\subsubsection{MAC services}\label{sec:Device Types / Crypto Device / Supported crypto services / MAC services}

The following MAC algorithms are defined:

\begin{lstlisting}
#define VIRTIO_CRYPTO_NO_MAC                       0
#define VIRTIO_CRYPTO_MAC_HMAC_MD5                 1
#define VIRTIO_CRYPTO_MAC_HMAC_SHA1                2
#define VIRTIO_CRYPTO_MAC_HMAC_SHA_224             3
#define VIRTIO_CRYPTO_MAC_HMAC_SHA_256             4
#define VIRTIO_CRYPTO_MAC_HMAC_SHA_384             5
#define VIRTIO_CRYPTO_MAC_HMAC_SHA_512             6
#define VIRTIO_CRYPTO_MAC_CMAC_3DES                25
#define VIRTIO_CRYPTO_MAC_CMAC_AES                 26
#define VIRTIO_CRYPTO_MAC_KASUMI_F9                27
#define VIRTIO_CRYPTO_MAC_SNOW3G_UIA2              28
#define VIRTIO_CRYPTO_MAC_GMAC_AES                 41
#define VIRTIO_CRYPTO_MAC_GMAC_TWOFISH             42
#define VIRTIO_CRYPTO_MAC_CBCMAC_AES               49
#define VIRTIO_CRYPTO_MAC_CBCMAC_KASUMI_F9         50
#define VIRTIO_CRYPTO_MAC_XCBC_AES                 53
#define VIRTIO_CRYPTO_MAC_ZUC_EIA3                 54
\end{lstlisting}

The above constants have two usages:
\begin{enumerate}
\item As bit numbers, used to tell the driver which MAC algorithms
are supported by the device, see \ref{sec:Device Types / Crypto Device / Device configuration layout}.
\item As values, used to designate the algorithm in (MAC type) crypto
operation requests, see \ref{sec:Device Types / Crypto Device / Device Operation / Control Virtqueue / Session operation}.
\end{enumerate}

\subsubsection{AEAD services}\label{sec:Device Types / Crypto Device / Supported crypto services / AEAD services}

The following AEAD algorithms are defined:

\begin{lstlisting}
#define VIRTIO_CRYPTO_NO_AEAD     0
#define VIRTIO_CRYPTO_AEAD_GCM    1
#define VIRTIO_CRYPTO_AEAD_CCM    2
#define VIRTIO_CRYPTO_AEAD_CHACHA20_POLY1305  3
\end{lstlisting}

The above constants have two usages:
\begin{enumerate}
\item As bit numbers, used to tell the driver which AEAD algorithms
are supported by the device, see \ref{sec:Device Types / Crypto Device / Device configuration layout}.
\item As values, used to designate the algorithm in (DEAD type) crypto
operation requests, see \ref{sec:Device Types / Crypto Device / Device Operation / Control Virtqueue / Session operation}.
\end{enumerate}

\subsubsection{AKCIPHER services}\label{sec: Device Types / Crypto Device / Supported crypto services / AKCIPHER services}

The following AKCIPHER algorithms are defined:
\begin{lstlisting}
#define VIRTIO_CRYPTO_NO_AKCIPHER 0
#define VIRTIO_CRYPTO_AKCIPHER_RSA   1
#define VIRTIO_CRYPTO_AKCIPHER_ECDSA 2
\end{lstlisting}

The above constants have two usages:
\begin{enumerate}
\item As bit numbers, used to tell the driver which AKCIPHER algorithms
are supported by the device, see \ref{sec:Device Types / Crypto Device / Device configuration layout}.
\item As values, used to designate the algorithm in asymmetric crypto operation requests,
see \ref{sec:Device Types / Crypto Device / Device Operation / Control Virtqueue / Session operation}.
\end{enumerate}


\subsection{Device configuration layout}\label{sec:Device Types / Crypto Device / Device configuration layout}

Crypto device configuration uses the following layout structure:

\begin{lstlisting}
struct virtio_crypto_config {
    le32 status;
    le32 max_dataqueues;
    le32 crypto_services;
    /* Detailed algorithms mask */
    le32 cipher_algo_l;
    le32 cipher_algo_h;
    le32 hash_algo;
    le32 mac_algo_l;
    le32 mac_algo_h;
    le32 aead_algo;
    /* Maximum length of cipher key in bytes */
    le32 max_cipher_key_len;
    /* Maximum length of authenticated key in bytes */
    le32 max_auth_key_len;
    le32 akcipher_algo;
    /* Maximum size of each crypto request's content in bytes */
    le64 max_size;
};
\end{lstlisting}

\begin{description}
\item Currently, only one \field{status} bit is defined: VIRTIO_CRYPTO_S_HW_READY
    set indicates that the device is ready to process requests, this bit is read-only
    for the driver
\begin{lstlisting}
#define VIRTIO_CRYPTO_S_HW_READY  (1 << 0)
\end{lstlisting}

\item [\field{max_dataqueues}] is the maximum number of data virtqueues that can
    be configured by the device. The driver MAY use only one data queue, or it
    can use more to achieve better performance.

\item [\field{crypto_services}] crypto service offered, see \ref{sec:Device Types / Crypto Device / Supported crypto services}.

\item [\field{cipher_algo_l}] CIPHER algorithms bits 0-31, see \ref{sec:Device Types / Crypto Device / Supported crypto services  / CIPHER services}.

\item [\field{cipher_algo_h}] CIPHER algorithms bits 32-63, see \ref{sec:Device Types / Crypto Device / Supported crypto services  / CIPHER services}.

\item [\field{hash_algo}] HASH algorithms bits, see \ref{sec:Device Types / Crypto Device / Supported crypto services  / HASH services}.

\item [\field{mac_algo_l}] MAC algorithms bits 0-31, see \ref{sec:Device Types / Crypto Device / Supported crypto services  / MAC services}.

\item [\field{mac_algo_h}] MAC algorithms bits 32-63, see \ref{sec:Device Types / Crypto Device / Supported crypto services  / MAC services}.

\item [\field{aead_algo}] AEAD algorithms bits, see \ref{sec:Device Types / Crypto Device / Supported crypto services  / AEAD services}.

\item [\field{max_cipher_key_len}] is the maximum length of cipher key supported by the device.

\item [\field{max_auth_key_len}] is the maximum length of authenticated key supported by the device.

\item [\field{akcipher_algo}] AKCIPHER algorithms bit 0-31, see \ref{sec: Device Types / Crypto Device / Supported crypto services / AKCIPHER services}.

\item [\field{max_size}] is the maximum size of the variable-length parameters of
    data operation of each crypto request's content supported by the device.
\end{description}

\begin{note}
Unless explicitly stated otherwise all lengths and sizes are in bytes.
\end{note}

\devicenormative{\subsubsection}{Device configuration layout}{Device Types / Crypto Device / Device configuration layout}

\begin{itemize*}
\item The device MUST set \field{max_dataqueues} to between 1 and 65535 inclusive.
\item The device MUST set the \field{status} with valid flags, undefined flags MUST NOT be set.
\item The device MUST accept and handle requests after \field{status} is set to VIRTIO_CRYPTO_S_HW_READY.
\item The device MUST set \field{crypto_services} based on the crypto services the device offers.
\item The device MUST set detailed algorithms masks for each service advertised by \field{crypto_services}.
    The device MUST NOT set the not defined algorithms bits.
\item The device MUST set \field{max_size} to show the maximum size of crypto request the device supports.
\item The device MUST set \field{max_cipher_key_len} to show the maximum length of cipher key if the
    device supports CIPHER service.
\item The device MUST set \field{max_auth_key_len} to show the maximum length of authenticated key if
    the device supports MAC service.
\end{itemize*}

\drivernormative{\subsubsection}{Device configuration layout}{Device Types / Crypto Device / Device configuration layout}

\begin{itemize*}
\item The driver MUST read the \field{status} from the bottom bit of status to check whether the
    VIRTIO_CRYPTO_S_HW_READY is set, and the driver MUST reread it after device reset.
\item The driver MUST NOT transmit any requests to the device if the VIRTIO_CRYPTO_S_HW_READY is not set.
\item The driver MUST read \field{max_dataqueues} field to discover the number of data queues the device supports.
\item The driver MUST read \field{crypto_services} field to discover which services the device is able to offer.
\item The driver SHOULD ignore the not defined algorithms bits.
\item The driver MUST read the detailed algorithms fields based on \field{crypto_services} field.
\item The driver SHOULD read \field{max_size} to discover the maximum size of the variable-length
    parameters of data operation of the crypto request's content the device supports and MUST
    guarantee that the size of each crypto request's content is within the \field{max_size}, otherwise
    the request will fail and the driver MUST reset the device.
\item The driver SHOULD read \field{max_cipher_key_len} to discover the maximum length of cipher key
    the device supports and MUST guarantee that the \field{key_len} (CIPHER service or AEAD service) is within
    the \field{max_cipher_key_len} of the device configuration, otherwise the request will fail.
\item The driver SHOULD read \field{max_auth_key_len} to discover the maximum length of authenticated
    key the device supports and MUST guarantee that the \field{auth_key_len} (MAC service) is within the
    \field{max_auth_key_len} of the device configuration, otherwise the request will fail.
\end{itemize*}

\subsection{Device Initialization}\label{sec:Device Types / Crypto Device / Device Initialization}

\drivernormative{\subsubsection}{Device Initialization}{Device Types / Crypto Device / Device Initialization}

\begin{itemize*}
\item The driver MUST configure and initialize all virtqueues.
\item The driver MUST read the supported crypto services from bits of \field{crypto_services}.
\item The driver MUST read the supported algorithms based on \field{crypto_services} field.
\end{itemize*}

\subsection{Device Operation}\label{sec:Device Types / Crypto Device / Device Operation}

The operation of a virtio crypto device is driven by requests placed on the virtqueues.
Requests consist of a queue-type specific header (specifying among others the operation)
and an operation specific payload.

If VIRTIO_CRYPTO_F_REVISION_1 is negotiated the device may support both session mode
(See \ref{sec:Device Types / Crypto Device / Device Operation / Control Virtqueue / Session operation})
and stateless mode operation requests.
In stateless mode all operation parameters are supplied as a part of each request,
while in session mode, some or all operation parameters are managed within the
session. Stateless mode is guarded by feature bits 0-4 on a service level. If
stateless mode is negotiated for a service, the service accepts both session
mode and stateless requests; otherwise stateless mode requests are rejected
(via operation status).

\subsubsection{Operation Status}\label{sec:Device Types / Crypto Device / Device Operation / Operation status}
The device MUST return a status code as part of the operation (both session
operation and service operation) result. The valid operation status as follows:

\begin{lstlisting}
enum VIRTIO_CRYPTO_STATUS {
    VIRTIO_CRYPTO_OK = 0,
    VIRTIO_CRYPTO_ERR = 1,
    VIRTIO_CRYPTO_BADMSG = 2,
    VIRTIO_CRYPTO_NOTSUPP = 3,
    VIRTIO_CRYPTO_INVSESS = 4,
    VIRTIO_CRYPTO_NOSPC = 5,
    VIRTIO_CRYPTO_KEY_REJECTED = 6,
    VIRTIO_CRYPTO_MAX
};
\end{lstlisting}

\begin{itemize*}
\item VIRTIO_CRYPTO_OK: success.
\item VIRTIO_CRYPTO_BADMSG: authentication failed (only when AEAD decryption).
\item VIRTIO_CRYPTO_NOTSUPP: operation or algorithm is unsupported.
\item VIRTIO_CRYPTO_INVSESS: invalid session ID when executing crypto operations.
\item VIRTIO_CRYPTO_NOSPC: no free session ID (only when the VIRTIO_CRYPTO_F_REVISION_1
    feature bit is negotiated).
\item VIRTIO_CRYPTO_KEY_REJECTED: signature verification failed (only when AKCIPHER verification).
\item VIRTIO_CRYPTO_ERR: any failure not mentioned above occurs.
\end{itemize*}

\subsubsection{Control Virtqueue}\label{sec:Device Types / Crypto Device / Device Operation / Control Virtqueue}

The driver uses the control virtqueue to send control commands to the
device, such as session operations (See \ref{sec:Device Types / Crypto Device / Device
Operation / Control Virtqueue / Session operation}).

The header for controlq is of the following form:
\begin{lstlisting}
#define VIRTIO_CRYPTO_OPCODE(service, op)   (((service) << 8) | (op))

struct virtio_crypto_ctrl_header {
#define VIRTIO_CRYPTO_CIPHER_CREATE_SESSION \
       VIRTIO_CRYPTO_OPCODE(VIRTIO_CRYPTO_SERVICE_CIPHER, 0x02)
#define VIRTIO_CRYPTO_CIPHER_DESTROY_SESSION \
       VIRTIO_CRYPTO_OPCODE(VIRTIO_CRYPTO_SERVICE_CIPHER, 0x03)
#define VIRTIO_CRYPTO_HASH_CREATE_SESSION \
       VIRTIO_CRYPTO_OPCODE(VIRTIO_CRYPTO_SERVICE_HASH, 0x02)
#define VIRTIO_CRYPTO_HASH_DESTROY_SESSION \
       VIRTIO_CRYPTO_OPCODE(VIRTIO_CRYPTO_SERVICE_HASH, 0x03)
#define VIRTIO_CRYPTO_MAC_CREATE_SESSION \
       VIRTIO_CRYPTO_OPCODE(VIRTIO_CRYPTO_SERVICE_MAC, 0x02)
#define VIRTIO_CRYPTO_MAC_DESTROY_SESSION \
       VIRTIO_CRYPTO_OPCODE(VIRTIO_CRYPTO_SERVICE_MAC, 0x03)
#define VIRTIO_CRYPTO_AEAD_CREATE_SESSION \
       VIRTIO_CRYPTO_OPCODE(VIRTIO_CRYPTO_SERVICE_AEAD, 0x02)
#define VIRTIO_CRYPTO_AEAD_DESTROY_SESSION \
       VIRTIO_CRYPTO_OPCODE(VIRTIO_CRYPTO_SERVICE_AEAD, 0x03)
#define VIRTIO_CRYPTO_AKCIPHER_CREATE_SESSION \
       VIRTIO_CRYPTO_OPCODE(VIRTIO_CRYPTO_SERVICE_AKCIPHER, 0x04)
#define VIRTIO_CRYPTO_AKCIPHER_DESTROY_SESSION \
       VIRTIO_CRYPTO_OPCDE(VIRTIO_CRYPTO_SERVICE_AKCIPHER, 0x05)
    le32 opcode;
    /* algo should be service-specific algorithms */
    le32 algo;
    le32 flag;
    le32 reserved;
};
\end{lstlisting}

The controlq request is composed of four parts:
\begin{lstlisting}
struct virtio_crypto_op_ctrl_req {
    /* Device read only portion */

    struct virtio_crypto_ctrl_header header;

#define VIRTIO_CRYPTO_CTRLQ_OP_SPEC_HDR_LEGACY 56
    /* fixed length fields, opcode specific */
    u8 op_flf[flf_len];

    /* variable length fields, opcode specific */
    u8 op_vlf[vlf_len];

    /* Device write only portion */

    /* op result or completion status */
    u8 op_outcome[outcome_len];
};
\end{lstlisting}

\field{header} is a general header (see above).

\field{op_flf} is the opcode (in \field{header}) specific fixed-length parameters.

\field{flf_len} depends on the VIRTIO_CRYPTO_F_REVISION_1 feature bit (see below).

\field{op_vlf} is the opcode (in \field{header}) specific variable-length parameters.

\field{vlf_len} is the size of the specific structure used.
\begin{note}
The \field{vlf_len} of session-destroy operation and the hash-session-create
operation is ZERO.
\end{note}

\begin{itemize*}
\item If the opcode (in \field{header}) is VIRTIO_CRYPTO_CIPHER_CREATE_SESSION
    then \field{op_flf} is struct virtio_crypto_sym_create_session_flf if
    VIRTIO_CRYPTO_F_REVISION_1 is negotiated and struct virtio_crypto_sym_create_session_flf is
    padded to 56 bytes if NOT negotiated, and \field{op_vlf} is struct
    virtio_crypto_sym_create_session_vlf.
\item If the opcode (in \field{header}) is VIRTIO_CRYPTO_HASH_CREATE_SESSION
    then \field{op_flf} is struct virtio_crypto_hash_create_session_flf if
    VIRTIO_CRYPTO_F_REVISION_1 is negotiated and struct virtio_crypto_hash_create_session_flf is
    padded to 56 bytes if NOT negotiated.
\item If the opcode (in \field{header}) is VIRTIO_CRYPTO_MAC_CREATE_SESSION
    then \field{op_flf} is struct virtio_crypto_mac_create_session_flf if
    VIRTIO_CRYPTO_F_REVISION_1 is negotiated and struct virtio_crypto_mac_create_session_flf is
    padded to 56 bytes if NOT negotiated, and \field{op_vlf} is struct
    virtio_crypto_mac_create_session_vlf.
\item If the opcode (in \field{header}) is VIRTIO_CRYPTO_AEAD_CREATE_SESSION
    then \field{op_flf} is struct virtio_crypto_aead_create_session_flf if
    VIRTIO_CRYPTO_F_REVISION_1 is negotiated and struct virtio_crypto_aead_create_session_flf is
    padded to 56 bytes if NOT negotiated, and \field{op_vlf} is struct
    virtio_crypto_aead_create_session_vlf.
\item If the opcode (in \field{header}) is VIRTIO_CRYPTO_AKCIPHER_CREATE_SESSION
    then \field{op_flf} is struct virtio_crypto_akcipher_create_session_flf if
    VIRTIO_CRYPTO_F_REVISION_1 is negotiated and struct virtio_crypto_akcipher_create_session_flf is
    padded to 56 bytes if NOT negotiated, and \field{op_vlf} is struct
    virtio_crypto_akcipher_create_session_vlf.
\item If the opcode (in \field{header}) is VIRTIO_CRYPTO_CIPHER_DESTROY_SESSION
    or VIRTIO_CRYPTO_HASH_DESTROY_SESSION or VIRTIO_CRYPTO_MAC_DESTROY_SESSION or
    VIRTIO_CRYPTO_AEAD_DESTROY_SESSION then \field{op_flf} is struct
    virtio_crypto_destroy_session_flf if VIRTIO_CRYPTO_F_REVISION_1 is negotiated and
    struct virtio_crypto_destroy_session_flf is padded to 56 bytes if NOT negotiated.
\end{itemize*}

\field{op_outcome} stores the result of operation and must be struct
virtio_crypto_destroy_session_input for destroy session or
struct virtio_crypto_create_session_input for create session.

\field{outcome_len} is the size of the structure used.


\paragraph{Session operation}\label{sec:Device Types / Crypto Device / Device
Operation / Control Virtqueue / Session operation}

The session is a handle which describes the cryptographic parameters to be
applied to a number of buffers.

The following structure stores the result of session creation set by the device:

\begin{lstlisting}
struct virtio_crypto_create_session_input {
    le64 session_id;
    le32 status;
    le32 padding;
};
\end{lstlisting}

A request to destroy a session includes the following information:

\begin{lstlisting}
struct virtio_crypto_destroy_session_flf {
    /* Device read only portion */
    le64  session_id;
};

struct virtio_crypto_destroy_session_input {
    /* Device write only portion */
    u8  status;
};
\end{lstlisting}


\subparagraph{Session operation: HASH session}\label{sec:Device Types / Crypto Device / Device
Operation / Control Virtqueue / Session operation / Session operation: HASH session}

The fixed-length parameters of HASH session requests is as follows:

\begin{lstlisting}
struct virtio_crypto_hash_create_session_flf {
    /* Device read only portion */

    /* See VIRTIO_CRYPTO_HASH_* above */
    le32 algo;
    /* hash result length */
    le32 hash_result_len;
};
\end{lstlisting}


\subparagraph{Session operation: MAC session}\label{sec:Device Types / Crypto Device / Device
Operation / Control Virtqueue / Session operation / Session operation: MAC session}

The fixed-length and the variable-length parameters of MAC session requests are as follows:

\begin{lstlisting}
struct virtio_crypto_mac_create_session_flf {
    /* Device read only portion */

    /* See VIRTIO_CRYPTO_MAC_* above */
    le32 algo;
    /* hash result length */
    le32 hash_result_len;
    /* length of authenticated key */
    le32 auth_key_len;
    le32 padding;
};

struct virtio_crypto_mac_create_session_vlf {
    /* Device read only portion */

    /* The authenticated key */
    u8 auth_key[auth_key_len];
};
\end{lstlisting}

The length of \field{auth_key} is specified in \field{auth_key_len} in the struct
virtio_crypto_mac_create_session_flf.


\subparagraph{Session operation: Symmetric algorithms session}\label{sec:Device Types / Crypto Device / Device
Operation / Control Virtqueue / Session operation / Session operation: Symmetric algorithms session}

The request of symmetric session could be the CIPHER algorithms request
or the chain algorithms (chaining CIPHER and HASH/MAC) request.

The fixed-length and the variable-length parameters of CIPHER session requests are as follows:

\begin{lstlisting}
struct virtio_crypto_cipher_session_flf {
    /* Device read only portion */

    /* See VIRTIO_CRYPTO_CIPHER* above */
    le32 algo;
    /* length of key */
    le32 key_len;
#define VIRTIO_CRYPTO_OP_ENCRYPT  1
#define VIRTIO_CRYPTO_OP_DECRYPT  2
    /* encryption or decryption */
    le32 op;
    le32 padding;
};

struct virtio_crypto_cipher_session_vlf {
    /* Device read only portion */

    /* The cipher key */
    u8 cipher_key[key_len];
};
\end{lstlisting}

The length of \field{cipher_key} is specified in \field{key_len} in the struct
virtio_crypto_cipher_session_flf.

The fixed-length and the variable-length parameters of Chain session requests are as follows:

\begin{lstlisting}
struct virtio_crypto_alg_chain_session_flf {
    /* Device read only portion */

#define VIRTIO_CRYPTO_SYM_ALG_CHAIN_ORDER_HASH_THEN_CIPHER  1
#define VIRTIO_CRYPTO_SYM_ALG_CHAIN_ORDER_CIPHER_THEN_HASH  2
    le32 alg_chain_order;
/* Plain hash */
#define VIRTIO_CRYPTO_SYM_HASH_MODE_PLAIN    1
/* Authenticated hash (mac) */
#define VIRTIO_CRYPTO_SYM_HASH_MODE_AUTH     2
/* Nested hash */
#define VIRTIO_CRYPTO_SYM_HASH_MODE_NESTED   3
    le32 hash_mode;
    struct virtio_crypto_cipher_session_flf cipher_hdr;

#define VIRTIO_CRYPTO_ALG_CHAIN_SESS_OP_SPEC_HDR_SIZE  16
    /* fixed length fields, algo specific */
    u8 algo_flf[VIRTIO_CRYPTO_ALG_CHAIN_SESS_OP_SPEC_HDR_SIZE];

    /* length of the additional authenticated data (AAD) in bytes */
    le32 aad_len;
    le32 padding;
};

struct virtio_crypto_alg_chain_session_vlf {
    /* Device read only portion */

    /* The cipher key */
    u8 cipher_key[key_len];
    /* The authenticated key */
    u8 auth_key[auth_key_len];
};
\end{lstlisting}

\field{hash_mode} decides the type used by \field{algo_flf}.

\field{algo_flf} is fixed to 16 bytes and MUST contains or be one of
the following types:
\begin{itemize*}
\item struct virtio_crypto_hash_create_session_flf
\item struct virtio_crypto_mac_create_session_flf
\end{itemize*}
The data of unused part (if has) in \field{algo_flf} will be ignored.

The length of \field{cipher_key} is specified in \field{key_len} in \field{cipher_hdr}.

The length of \field{auth_key} is specified in \field{auth_key_len} in struct
virtio_crypto_mac_create_session_flf.

The fixed-length parameters of Symmetric session requests are as follows:

\begin{lstlisting}
struct virtio_crypto_sym_create_session_flf {
    /* Device read only portion */

#define VIRTIO_CRYPTO_SYM_SESS_OP_SPEC_HDR_SIZE  48
    /* fixed length fields, opcode specific */
    u8 op_flf[VIRTIO_CRYPTO_SYM_SESS_OP_SPEC_HDR_SIZE];

/* No operation */
#define VIRTIO_CRYPTO_SYM_OP_NONE  0
/* Cipher only operation on the data */
#define VIRTIO_CRYPTO_SYM_OP_CIPHER  1
/* Chain any cipher with any hash or mac operation. The order
   depends on the value of alg_chain_order param */
#define VIRTIO_CRYPTO_SYM_OP_ALGORITHM_CHAINING  2
    le32 op_type;
    le32 padding;
};
\end{lstlisting}

\field{op_flf} is fixed to 48 bytes, MUST contains or be one of
the following types:
\begin{itemize*}
\item struct virtio_crypto_cipher_session_flf
\item struct virtio_crypto_alg_chain_session_flf
\end{itemize*}
The data of unused part (if has) in \field{op_flf} will be ignored.

\field{op_type} decides the type used by \field{op_flf}.

The variable-length parameters of Symmetric session requests are as follows:

\begin{lstlisting}
struct virtio_crypto_sym_create_session_vlf {
    /* Device read only portion */
    /* variable length fields, opcode specific */
    u8 op_vlf[vlf_len];
};
\end{lstlisting}

\field{op_vlf} MUST contains or be one of the following types:
\begin{itemize*}
\item struct virtio_crypto_cipher_session_vlf
\item struct virtio_crypto_alg_chain_session_vlf
\end{itemize*}

\field{op_type} in struct virtio_crypto_sym_create_session_flf decides the
type used by \field{op_vlf}.

\field{vlf_len} is the size of the specific structure used.


\subparagraph{Session operation: AEAD session}\label{sec:Device Types / Crypto Device / Device
Operation / Control Virtqueue / Session operation / Session operation: AEAD session}

The fixed-length and the variable-length parameters of AEAD session requests are as follows:

\begin{lstlisting}
struct virtio_crypto_aead_create_session_flf {
    /* Device read only portion */

    /* See VIRTIO_CRYPTO_AEAD_* above */
    le32 algo;
    /* length of key */
    le32 key_len;
    /* Authentication tag length */
    le32 tag_len;
    /* The length of the additional authenticated data (AAD) in bytes */
    le32 aad_len;
    /* encryption or decryption, See above VIRTIO_CRYPTO_OP_* */
    le32 op;
    le32 padding;
};

struct virtio_crypto_aead_create_session_vlf {
    /* Device read only portion */
    u8 key[key_len];
};
\end{lstlisting}

The length of \field{key} is specified in \field{key_len} in struct
virtio_crypto_aead_create_session_flf.

\subparagraph{Session operation: AKCIPHER session}\label{sec:Device Types / Crypto Device / Device
Operation / Control Virtqueue / Session operation / Session operation: AKCIPHER session}

Due to the complexity of asymmetric key algorithms, different algorithms
require different parameters. The following data structures are used as
supplementary parameters to describe the asymmetric algorithm sessions.

For the RSA algorithm, the extra parameters are as follows:
\begin{lstlisting}
struct virtio_crypto_rsa_session_para {
#define VIRTIO_CRYPTO_RSA_RAW_PADDING   0
#define VIRTIO_CRYPTO_RSA_PKCS1_PADDING 1
    le32 padding_algo;

#define VIRTIO_CRYPTO_RSA_NO_HASH   0
#define VIRTIO_CRYPTO_RSA_MD2       1
#define VIRTIO_CRYPTO_RSA_MD3       2
#define VIRTIO_CRYPTO_RSA_MD4       3
#define VIRTIO_CRYPTO_RSA_MD5       4
#define VIRTIO_CRYPTO_RSA_SHA1      5
#define VIRTIO_CRYPTO_RSA_SHA256    6
#define VIRTIO_CRYPTO_RSA_SHA384    7
#define VIRTIO_CRYPTO_RSA_SHA512    8
#define VIRTIO_CRYPTO_RSA_SHA224    9
    le32 hash_algo;
};
\end{lstlisting}

\field{padding_algo} specifies the padding method used by RSA sessions.
\begin{itemize*}
\item If VIRTIO_CRYPTO_RSA_RAW_PADDING is specified, 1) \field{hash_algo}
is ignored, 2) ciphertext and plaintext MUST be padded with leading zeros,
3) and RSA sessions with VIRTIO_CRYPTO_RSA_RAW_PADDING MUST not be used
for verification and signing operations.
\item If VIRTIO_CRYPTO_RSA_PKCS1_PADDING is specified, EMSA-PKCS1-v1_5 padding method
is used (see \hyperref[intro:rfc3447]{PKCS\#1}), \field{hash_algo} specifies how the
digest of the data passed to RSA sessions is calculated when verifying and signing.
It only affects the padding algorithm and is ignored during encryption and decryption.
\end{itemize*}

The ECC algorithms such as the ECDSA algorithm, cannot use custom curves, only the
following known curves can be used (see \hyperref[intro:NIST]{NIST-recommended curves}).

\begin{lstlisting}
#define VIRTIO_CRYPTO_CURVE_UNKNOWN   0
#define VIRTIO_CRYPTO_CURVE_NIST_P192 1
#define VIRTIO_CRYPTO_CURVE_NIST_P224 2
#define VIRTIO_CRYPTO_CURVE_NIST_P256 3
#define VIRTIO_CRYPTO_CURVE_NIST_P384 4
#define VIRTIO_CRYPTO_CURVE_NIST_P521 5
\end{lstlisting}

For the ECDSA algorithm, the extra parameters are as follows:
\begin{lstlisting}
struct virtio_crypto_ecdsa_session_para {
    /* See VIRTIO_CRYPTO_CURVE_* above */
    le32 curve_id;
};
\end{lstlisting}

The fixed-length and the variable-length parameters of AKCIPHER session requests are as follows:
\begin{lstlisting}
struct virtio_crypto_akcipher_create_session_flf {
    /* Device read only portion */

    /* See VIRTIO_CRYPTO_AKCIPHER_* above */
    le32 algo;
#define VIRTIO_CRYPTO_AKCIPHER_KEY_TYPE_PUBLIC 1
#define VIRTIO_CRYPTO_AKCIPHER_KEY_TYPE_PRIVATE 2
    le32 key_type;
    /* length of key */
    le32 key_len;

#define VIRTIO_CRYPTO_AKCIPHER_SESS_ALGO_SPEC_HDR_SIZE 44
    u8 algo_flf[VIRTIO_CRYPTO_AKCIPHER_SESS_ALGO_SPEC_HDR_SIZE];
};

struct virtio_crypto_akcipher_create_session_vlf {
    /* Device read only portion */
    u8 key[key_len];
};
\end{lstlisting}

\field{algo} decides the type used by \field{algo_flf}.
\field{algo_flf} is fixed to 44 bytes and MUST contains of be one the
following structures:
\begin{itemize*}
\item struct virtio_crypto_rsa_session_para
\item struct virtio_crypto_ecdsa_session_para
\end{itemize*}

The length of \field{key} is specified in \field{key_len} in the struct
virtio_crypto_akcipher_create_session_flf.

For the RSA algorithm, the key needs to be encoded according to
\hyperref[intro:rfc3447]{PKCS\#1}. The private key is described with the
RSAPrivateKey structure, and the public key is described with the RSAPublicKey
structure. These ASN.1 structures are encoded in DER encoding rules (see
\hyperref[intro:rfc6025]{rfc6025}).

\begin{lstlisting}
RSAPrivateKey ::= SEQUENCE {
    version          INTEGER,
    modulus          INTEGER,
    publicExponent   INTEGER,
    privateExponent  INTEGER,
    prime1           INTEGER,
    prime2           INTEGER,
    exponent1        INTEGER,
    exponent1        INTEGER,
    coefficient      INTEGER,
    otherPrimeInfos  OtherPrimeInfos OPTIONAL
}

OtherPrimeInfos ::= SEQUENCE SIZE(1...MAX) OF OtherPrimeInfo

OtherPrimeINfo ::= SEQUENCE {
    prime           INTEGER,
    exponent        INTEGER,
    coefficient     INTEGER
}

RSAPublicKey ::= SEQUENCE {
    modulus         INTEGER,
    publicExponent  INTEGER
}
\end{lstlisting}

For the ECDSA algorithm, the private key is encoded according to
\hyperref[intro:rfc5915]{RFC5915}, the private key of the ECDSA algorithm
is described by the ASN.1 structure ECPrivateKey and encoded with DER
encoding rules (see \hyperref[intro:rfc6025]{rfc6025}).

\begin{lstlisting}
ECPrivateKey ::= SEQUNCE {
    version         INTEGER,
    privateKey      OCTET STRING,
    parameters [0]  ECParameters {{ NamedCurve }} OPTIONAL,
    publicKey  [1]  BIT STRING OPTIONAL
}
\end{lstlisting}

The public key of the ECDSA algorithm is encoded according to \hyperref[intro:SEC1]{SEC1},
and the public key of ECDSA is described by the ASN.1 structure ECPoint.
When initializing a session with ECDSA public key, the ECPoint is DER encoded and the
\field{key} only contains the value part of ECPoint, that is, the header part of the
OCTET STRING will be omitted (see \hyperref[intro:rfc6025]{rfc6025}).

\begin{lstlisting}
ECPoint ::= OCTET STRING
\end{lstlisting}

The length of \field{key} is specified in \field{key_len} in
struct virtio_crypto_akcipher_create_session_flf.

\drivernormative{\subparagraph}{Session operation: create session}{Device Types / Crypto Device / Device
Operation / Control Virtqueue / Session operation / Session operation: create session}

\begin{itemize*}
\item The driver MUST set the \field{opcode} field based on service type: CIPHER, HASH, MAC, AEAD or AKCIPHER.
\item The driver MUST set the control general header, the opcode specific header,
    the opcode specific extra parameters and the opcode specific outcome buffer in turn.
    See \ref{sec:Device Types / Crypto Device / Device Operation / Control Virtqueue}.
\item The driver MUST set the \field{reversed} field to zero.
\end{itemize*}

\devicenormative{\subparagraph}{Session operation: create session}{Device Types / Crypto Device / Device
Operation / Control Virtqueue / Session operation / Session operation: create session}

\begin{itemize*}
\item The device MUST use the corresponding opcode specific structure according to the
    \field{opcode} in the control general header.
\item The device MUST extract extra parameters according to the structures used.
\item The device MUST set the \field{status} field to one of the following values of enum
    VIRTIO_CRYPTO_STATUS after finish a session creation:
\begin{itemize*}
\item VIRTIO_CRYPTO_OK if a session is created successfully.
\item VIRTIO_CRYPTO_NOTSUPP if the requested algorithm or operation is unsupported.
\item VIRTIO_CRYPTO_NOSPC if no free session ID (only when the VIRTIO_CRYPTO_F_REVISION_1
    feature bit is negotiated).
\item VIRTIO_CRYPTO_ERR if failure not mentioned above occurs.
\end{itemize*}
\item The device MUST set the \field{session_id} field to a unique session identifier only
    if the status is set to VIRTIO_CRYPTO_OK.
\end{itemize*}

\drivernormative{\subparagraph}{Session operation: destroy session}{Device Types / Crypto Device / Device
Operation / Control Virtqueue / Session operation / Session operation: destroy session}

\begin{itemize*}
\item The driver MUST set the \field{opcode} field based on service type: CIPHER, HASH, MAC, AEAD or AKCIPHER.
\item The driver MUST set the \field{session_id} to a valid value assigned by the device
    when the session was created.
\end{itemize*}

\devicenormative{\subparagraph}{Session operation: destroy session}{Device Types / Crypto Device / Device
Operation / Control Virtqueue / Session operation / Session operation: destroy session}

\begin{itemize*}
\item The device MUST set the \field{status} field to one of the following values of enum VIRTIO_CRYPTO_STATUS.
\begin{itemize*}
\item VIRTIO_CRYPTO_OK if a session is created successfully.
\item VIRTIO_CRYPTO_ERR if any failure occurs.
\end{itemize*}
\end{itemize*}


\subsubsection{Data Virtqueue}\label{sec:Device Types / Crypto Device / Device Operation / Data Virtqueue}

The driver uses the data virtqueues to transmit crypto operation requests to the device,
and completes the crypto operations.

The header for dataq is as follows:

\begin{lstlisting}
struct virtio_crypto_op_header {
#define VIRTIO_CRYPTO_CIPHER_ENCRYPT \
    VIRTIO_CRYPTO_OPCODE(VIRTIO_CRYPTO_SERVICE_CIPHER, 0x00)
#define VIRTIO_CRYPTO_CIPHER_DECRYPT \
    VIRTIO_CRYPTO_OPCODE(VIRTIO_CRYPTO_SERVICE_CIPHER, 0x01)
#define VIRTIO_CRYPTO_HASH \
    VIRTIO_CRYPTO_OPCODE(VIRTIO_CRYPTO_SERVICE_HASH, 0x00)
#define VIRTIO_CRYPTO_MAC \
    VIRTIO_CRYPTO_OPCODE(VIRTIO_CRYPTO_SERVICE_MAC, 0x00)
#define VIRTIO_CRYPTO_AEAD_ENCRYPT \
    VIRTIO_CRYPTO_OPCODE(VIRTIO_CRYPTO_SERVICE_AEAD, 0x00)
#define VIRTIO_CRYPTO_AEAD_DECRYPT \
    VIRTIO_CRYPTO_OPCODE(VIRTIO_CRYPTO_SERVICE_AEAD, 0x01)
#define VIRTIO_CRYPTO_AKCIPHER_ENCRYPT \
    VIRTIO_CRYPTO_OPCODE(VIRTIO_CRYPTO_SERVICE_AKCIPHER, 0x00)
#define VIRTIO_CRYPTO_AKCIPHER_DECRYPT \
    VIRTIO_CRYPTO_OPCODE(VIRTIO_CRYPTO_SERVICE_AKCIPHER, 0x01)
#define VIRTIO_CRYPTO_AKCIPHER_SIGN \
    VIRTIO_CRYPTO_OPCODE(VIRTIO_CRYPTO_SERVICE_AKCIPHER, 0x02)
#define VIRTIO_CRYPTO_AKCIPHER_VERIFY \
    VIRTIO_CRYPTO_OPCODE(VIRTIO_CRYPTO_SERVICE_AKCIPHER, 0x03)
    le32 opcode;
    /* algo should be service-specific algorithms */
    le32 algo;
    le64 session_id;
#define VIRTIO_CRYPTO_FLAG_SESSION_MODE 1
    /* control flag to control the request */
    le32 flag;
    le32 padding;
};
\end{lstlisting}

\begin{note}
If VIRTIO_CRYPTO_F_REVISION_1 is not negotiated the \field{flag} is ignored.

If VIRTIO_CRYPTO_F_REVISION_1 is negotiated but VIRTIO_CRYPTO_F_<SERVICE>_STATELESS_MODE
is not negotiated, then the device SHOULD reject <SERVICE> requests if
VIRTIO_CRYPTO_FLAG_SESSION_MODE is not set (in \field{flag}).
\end{note}

The dataq request is composed of four parts:
\begin{lstlisting}
struct virtio_crypto_op_data_req {
    /* Device read only portion */

    struct virtio_crypto_op_header header;

#define VIRTIO_CRYPTO_DATAQ_OP_SPEC_HDR_LEGACY 48
    /* fixed length fields, opcode specific */
    u8 op_flf[flf_len];

    /* Device read && write portion */
    /* variable length fields, opcode specific */
    u8 op_vlf[vlf_len];

    /* Device write only portion */
    struct virtio_crypto_inhdr inhdr;
};
\end{lstlisting}

\field{header} is a general header (see above).

\field{op_flf} is the opcode (in \field{header}) specific header.

\field{flf_len} depends on the VIRTIO_CRYPTO_F_REVISION_1 feature bit
(see below).

\field{op_vlf} is the opcode (in \field{header}) specific parameters.

\field{vlf_len} is the size of the specific structure used.

\begin{itemize*}
\item If the the opcode (in \field{header}) is VIRTIO_CRYPTO_CIPHER_ENCRYPT
    or VIRTIO_CRYPTO_CIPHER_DECRYPT then:
    \begin{itemize*}
    \item If VIRTIO_CRYPTO_F_CIPHER_STATELESS_MODE is negotiated, \field{op_flf} is
        struct virtio_crypto_sym_data_flf_stateless, and \field{op_vlf} is struct
        virtio_crypto_sym_data_vlf_stateless.
    \item If VIRTIO_CRYPTO_F_CIPHER_STATELESS_MODE is NOT negotiated, \field{op_flf}
        is struct virtio_crypto_sym_data_flf if VIRTIO_CRYPTO_F_REVISION_1 is negotiated
        and struct virtio_crypto_sym_data_flf is padded to 48 bytes if NOT negotiated,
        and \field{op_vlf} is struct virtio_crypto_sym_data_vlf.
    \end{itemize*}
\item If the the opcode (in \field{header}) is VIRTIO_CRYPTO_HASH:
    \begin{itemize*}
    \item If VIRTIO_CRYPTO_F_HASH_STATELESS_MODE is negotiated, \field{op_flf} is
        struct virtio_crypto_hash_data_flf_stateless, and \field{op_vlf} is struct
        virtio_crypto_hash_data_vlf_stateless.
    \item If VIRTIO_CRYPTO_F_HASH_STATELESS_MODE is NOT negotiated, \field{op_flf}
        is struct virtio_crypto_hash_data_flf if VIRTIO_CRYPTO_F_REVISION_1 is negotiated
        and struct virtio_crypto_hash_data_flf is padded to 48 bytes if NOT negotiated,
        and \field{op_vlf} is struct virtio_crypto_hash_data_vlf.
    \end{itemize*}
\item If the the opcode (in \field{header}) is VIRTIO_CRYPTO_MAC:
    \begin{itemize*}
    \item If VIRTIO_CRYPTO_F_MAC_STATELESS_MODE is negotiated, \field{op_flf} is
        struct virtio_crypto_mac_data_flf_stateless, and \field{op_vlf} is struct
        virtio_crypto_mac_data_vlf_stateless.
    \item If VIRTIO_CRYPTO_F_MAC_STATELESS_MODE is NOT negotiated, \field{op_flf}
        is struct virtio_crypto_mac_data_flf if VIRTIO_CRYPTO_F_REVISION_1 is negotiated
        and struct virtio_crypto_mac_data_flf is padded to 48 bytes if NOT negotiated,
        and \field{op_vlf} is struct virtio_crypto_mac_data_vlf.
    \end{itemize*}
\item If the the opcode (in \field{header}) is VIRTIO_CRYPTO_AEAD_ENCRYPT
    or VIRTIO_CRYPTO_AEAD_DECRYPT then:
    \begin{itemize*}
    \item If VIRTIO_CRYPTO_F_AEAD_STATELESS_MODE is negotiated, \field{op_flf} is
        struct virtio_crypto_aead_data_flf_stateless, and \field{op_vlf} is struct
        virtio_crypto_aead_data_vlf_stateless.
    \item If VIRTIO_CRYPTO_F_AEAD_STATELESS_MODE is NOT negotiated, \field{op_flf}
        is struct virtio_crypto_aead_data_flf if VIRTIO_CRYPTO_F_REVISION_1 is negotiated
        and struct virtio_crypto_aead_data_flf is padded to 48 bytes if NOT negotiated,
        and \field{op_vlf} is struct virtio_crypto_aead_data_vlf.
    \end{itemize*}
\item If the opcode (in \field{header}) is VIRTIO_CRYPTO_AKCIPHER_ENCRYPT, VIRTIO_CRYPTO_AKCIPHER_DECRYPT,
    VIRTIO_CRYPTO_AKCIPHER_SIGN or VIRTIO_CRYPTO_AKCIPHER_VERIFY then:
    \begin{itemize*}
    \item If VIRTIO_CRYPTO_F_AKCIPHER_STATELESS_MODE is negotiated, \field{op_flf} is
        struct virtio_crypto_akcipher_data_flf_statless, and \field{op_vlf} is struct
        virtio_crypto_akcipher_data_vlf_stateless.
    \item If VIRTIO_CRYPTO_F_AKCIPHER_STATELESS_MODE is NOT negotiated, \field{op_flf}
        is struct virtio_crypto_akcipher_data_flf if VIRTIO_CRYPTO_F_REVISION_1 is negotiated
        and struct virtio_crypto_akcipher_data_flf is padded to 48 bytes if NOT negotiated,
        and \field{op_vlf} is struct virtio_crypto_akcipher_data_vlf.
    \end{itemize*}
\end{itemize*}

\field{inhdr} is a unified input header that used to return the status of
the operations, is defined as follows:

\begin{lstlisting}
struct virtio_crypto_inhdr {
    u8 status;
};
\end{lstlisting}

\subsubsection{HASH Service Operation}\label{sec:Device Types / Crypto Device / Device Operation / HASH Service Operation}

Session mode HASH service requests are as follows:

\begin{lstlisting}
struct virtio_crypto_hash_data_flf {
    /* length of source data */
    le32 src_data_len;
    /* hash result length */
    le32 hash_result_len;
};

struct virtio_crypto_hash_data_vlf {
    /* Device read only portion */
    /* Source data */
    u8 src_data[src_data_len];

    /* Device write only portion */
    /* Hash result data */
    u8 hash_result[hash_result_len];
};
\end{lstlisting}

Each data request uses the virtio_crypto_hash_data_flf structure and the
virtio_crypto_hash_data_vlf structure to store information used to run the
HASH operations.

\field{src_data} is the source data that will be processed.
\field{src_data_len} is the length of source data.
\field{hash_result} is the result data and \field{hash_result_len} is the length
of it.

Stateless mode HASH service requests are as follows:

\begin{lstlisting}
struct virtio_crypto_hash_data_flf_stateless {
    struct {
        /* See VIRTIO_CRYPTO_HASH_* above */
        le32 algo;
    } sess_para;

    /* length of source data */
    le32 src_data_len;
    /* hash result length */
    le32 hash_result_len;
    le32 reserved;
};
struct virtio_crypto_hash_data_vlf_stateless {
    /* Device read only portion */
    /* Source data */
    u8 src_data[src_data_len];

    /* Device write only portion */
    /* Hash result data */
    u8 hash_result[hash_result_len];
};
\end{lstlisting}

\drivernormative{\paragraph}{HASH Service Operation}{Device Types / Crypto Device / Device Operation / HASH Service Operation}

\begin{itemize*}
\item If the driver uses the session mode, then the driver MUST set \field{session_id}
    in struct virtio_crypto_op_header to a valid value assigned by the device when the
    session was created.
\item If the VIRTIO_CRYPTO_F_HASH_STATELESS_MODE feature bit is negotiated, 1) if the
    driver uses the stateless mode, then the driver MUST set the \field{flag} field in
    struct virtio_crypto_op_header to ZERO and MUST set the fields in struct
    virtio_crypto_hash_data_flf_stateless.sess_para, 2) if the driver uses the session
    mode, then the driver MUST set the \field{flag} field in struct virtio_crypto_op_header
    to VIRTIO_CRYPTO_FLAG_SESSION_MODE.
\item The driver MUST set \field{opcode} in struct virtio_crypto_op_header to VIRTIO_CRYPTO_HASH.
\end{itemize*}

\devicenormative{\paragraph}{HASH Service Operation}{Device Types / Crypto Device / Device Operation / HASH Service Operation}

\begin{itemize*}
\item The device MUST use the corresponding structure according to the \field{opcode}
    in the data general header.
\item If the VIRTIO_CRYPTO_F_HASH_STATELESS_MODE feature bit is negotiated, the device
    MUST parse \field{flag} field in struct virtio_crypto_op_header in order to decide
    which mode the driver uses.
\item The device MUST copy the results of HASH operations in the hash_result[] if HASH
    operations success.
\item The device MUST set \field{status} in struct virtio_crypto_inhdr to one of the
    following values of enum VIRTIO_CRYPTO_STATUS:
\begin{itemize*}
\item VIRTIO_CRYPTO_OK if the operation success.
\item VIRTIO_CRYPTO_NOTSUPP if the requested algorithm or operation is unsupported.
\item VIRTIO_CRYPTO_INVSESS if the session ID invalid when in session mode.
\item VIRTIO_CRYPTO_ERR if any failure not mentioned above occurs.
\end{itemize*}
\end{itemize*}


\subsubsection{MAC Service Operation}\label{sec:Device Types / Crypto Device / Device Operation / MAC Service Operation}

Session mode MAC service requests are as follows:

\begin{lstlisting}
struct virtio_crypto_mac_data_flf {
    struct virtio_crypto_hash_data_flf hdr;
};

struct virtio_crypto_mac_data_vlf {
    /* Device read only portion */
    /* Source data */
    u8 src_data[src_data_len];

    /* Device write only portion */
    /* Hash result data */
    u8 hash_result[hash_result_len];
};
\end{lstlisting}

Each request uses the virtio_crypto_mac_data_flf structure and the
virtio_crypto_mac_data_vlf structure to store information used to run the
MAC operations.

\field{src_data} is the source data that will be processed.
\field{src_data_len} is the length of source data.
\field{hash_result} is the result data and \field{hash_result_len} is the length
of it.

Stateless mode MAC service requests are as follows:

\begin{lstlisting}
struct virtio_crypto_mac_data_flf_stateless {
    struct {
        /* See VIRTIO_CRYPTO_MAC_* above */
        le32 algo;
        /* length of authenticated key */
        le32 auth_key_len;
    } sess_para;

    /* length of source data */
    le32 src_data_len;
    /* hash result length */
    le32 hash_result_len;
};

struct virtio_crypto_mac_data_vlf_stateless {
    /* Device read only portion */
    /* The authenticated key */
    u8 auth_key[auth_key_len];
    /* Source data */
    u8 src_data[src_data_len];

    /* Device write only portion */
    /* Hash result data */
    u8 hash_result[hash_result_len];
};
\end{lstlisting}

\field{auth_key} is the authenticated key that will be used during the process.
\field{auth_key_len} is the length of the key.

\drivernormative{\paragraph}{MAC Service Operation}{Device Types / Crypto Device / Device Operation / MAC Service Operation}

\begin{itemize*}
\item If the driver uses the session mode, then the driver MUST set \field{session_id}
    in struct virtio_crypto_op_header to a valid value assigned by the device when the
    session was created.
\item If the VIRTIO_CRYPTO_F_MAC_STATELESS_MODE feature bit is negotiated, 1) if the
    driver uses the stateless mode, then the driver MUST set the \field{flag} field
    in struct virtio_crypto_op_header to ZERO and MUST set the fields in struct
    virtio_crypto_mac_data_flf_stateless.sess_para, 2) if the driver uses the session
    mode, then the driver MUST set the \field{flag} field in struct virtio_crypto_op_header
    to VIRTIO_CRYPTO_FLAG_SESSION_MODE.
\item The driver MUST set \field{opcode} in struct virtio_crypto_op_header to VIRTIO_CRYPTO_MAC.
\end{itemize*}

\devicenormative{\paragraph}{MAC Service Operation}{Device Types / Crypto Device / Device Operation / MAC Service Operation}

\begin{itemize*}
\item If the VIRTIO_CRYPTO_F_MAC_STATELESS_MODE feature bit is negotiated, the device
    MUST parse \field{flag} field in struct virtio_crypto_op_header in order to decide
	which mode the driver uses.
\item The device MUST copy the results of MAC operations in the hash_result[] if HASH
    operations success.
\item The device MUST set \field{status} in struct virtio_crypto_inhdr to one of the
    following values of enum VIRTIO_CRYPTO_STATUS:
\begin{itemize*}
\item VIRTIO_CRYPTO_OK if the operation success.
\item VIRTIO_CRYPTO_NOTSUPP if the requested algorithm or operation is unsupported.
\item VIRTIO_CRYPTO_INVSESS if the session ID invalid when in session mode.
\item VIRTIO_CRYPTO_ERR if any failure not mentioned above occurs.
\end{itemize*}
\end{itemize*}

\subsubsection{Symmetric algorithms Operation}\label{sec:Device Types / Crypto Device / Device Operation / Symmetric algorithms Operation}

Session mode CIPHER service requests are as follows:

\begin{lstlisting}
struct virtio_crypto_cipher_data_flf {
    /*
     * Byte Length of valid IV/Counter data pointed to by the below iv data.
     *
     * For block ciphers in CBC or F8 mode, or for Kasumi in F8 mode, or for
     *   SNOW3G in UEA2 mode, this is the length of the IV (which
     *   must be the same as the block length of the cipher).
     * For block ciphers in CTR mode, this is the length of the counter
     *   (which must be the same as the block length of the cipher).
     */
    le32 iv_len;
    /* length of source data */
    le32 src_data_len;
    /* length of destination data */
    le32 dst_data_len;
    le32 padding;
};

struct virtio_crypto_cipher_data_vlf {
    /* Device read only portion */

    /*
     * Initialization Vector or Counter data.
     *
     * For block ciphers in CBC or F8 mode, or for Kasumi in F8 mode, or for
     *   SNOW3G in UEA2 mode, this is the Initialization Vector (IV)
     *   value.
     * For block ciphers in CTR mode, this is the counter.
     * For AES-XTS, this is the 128bit tweak, i, from IEEE Std 1619-2007.
     *
     * The IV/Counter will be updated after every partial cryptographic
     * operation.
     */
    u8 iv[iv_len];
    /* Source data */
    u8 src_data[src_data_len];

    /* Device write only portion */
    /* Destination data */
    u8 dst_data[dst_data_len];
};
\end{lstlisting}

Session mode requests of algorithm chaining are as follows:

\begin{lstlisting}
struct virtio_crypto_alg_chain_data_flf {
    le32 iv_len;
    /* Length of source data */
    le32 src_data_len;
    /* Length of destination data */
    le32 dst_data_len;
    /* Starting point for cipher processing in source data */
    le32 cipher_start_src_offset;
    /* Length of the source data that the cipher will be computed on */
    le32 len_to_cipher;
    /* Starting point for hash processing in source data */
    le32 hash_start_src_offset;
    /* Length of the source data that the hash will be computed on */
    le32 len_to_hash;
    /* Length of the additional auth data */
    le32 aad_len;
    /* Length of the hash result */
    le32 hash_result_len;
    le32 reserved;
};

struct virtio_crypto_alg_chain_data_vlf {
    /* Device read only portion */

    /* Initialization Vector or Counter data */
    u8 iv[iv_len];
    /* Source data */
    u8 src_data[src_data_len];
    /* Additional authenticated data if exists */
    u8 aad[aad_len];

    /* Device write only portion */

    /* Destination data */
    u8 dst_data[dst_data_len];
    /* Hash result data */
    u8 hash_result[hash_result_len];
};
\end{lstlisting}

Session mode requests of symmetric algorithm are as follows:

\begin{lstlisting}
struct virtio_crypto_sym_data_flf {
    /* Device read only portion */

#define VIRTIO_CRYPTO_SYM_DATA_REQ_HDR_SIZE    40
    u8 op_type_flf[VIRTIO_CRYPTO_SYM_DATA_REQ_HDR_SIZE];

    /* See above VIRTIO_CRYPTO_SYM_OP_* */
    le32 op_type;
    le32 padding;
};

struct virtio_crypto_sym_data_vlf {
    u8 op_type_vlf[sym_para_len];
};
\end{lstlisting}

Each request uses the virtio_crypto_sym_data_flf structure and the
virtio_crypto_sym_data_flf structure to store information used to run the
CIPHER operations.

\field{op_type_flf} is the \field{op_type} specific header, it MUST starts
with or be one of the following structures:
\begin{itemize*}
\item struct virtio_crypto_cipher_data_flf
\item struct virtio_crypto_alg_chain_data_flf
\end{itemize*}

The length of \field{op_type_flf} is fixed to 40 bytes, the data of unused
part (if has) will be ignored.

\field{op_type_vlf} is the \field{op_type} specific parameters, it MUST starts
with or be one of the following structures:
\begin{itemize*}
\item struct virtio_crypto_cipher_data_vlf
\item struct virtio_crypto_alg_chain_data_vlf
\end{itemize*}

\field{sym_para_len} is the size of the specific structure used.

Stateless mode CIPHER service requests are as follows:

\begin{lstlisting}
struct virtio_crypto_cipher_data_flf_stateless {
    struct {
        /* See VIRTIO_CRYPTO_CIPHER* above */
        le32 algo;
        /* length of key */
        le32 key_len;

        /* See VIRTIO_CRYPTO_OP_* above */
        le32 op;
    } sess_para;

    /*
     * Byte Length of valid IV/Counter data pointed to by the below iv data.
     */
    le32 iv_len;
    /* length of source data */
    le32 src_data_len;
    /* length of destination data */
    le32 dst_data_len;
};

struct virtio_crypto_cipher_data_vlf_stateless {
    /* Device read only portion */

    /* The cipher key */
    u8 cipher_key[key_len];

    /* Initialization Vector or Counter data. */
    u8 iv[iv_len];
    /* Source data */
    u8 src_data[src_data_len];

    /* Device write only portion */
    /* Destination data */
    u8 dst_data[dst_data_len];
};
\end{lstlisting}

Stateless mode requests of algorithm chaining are as follows:

\begin{lstlisting}
struct virtio_crypto_alg_chain_data_flf_stateless {
    struct {
        /* See VIRTIO_CRYPTO_SYM_ALG_CHAIN_ORDER_* above */
        le32 alg_chain_order;
        /* length of the additional authenticated data in bytes */
        le32 aad_len;

        struct {
            /* See VIRTIO_CRYPTO_CIPHER* above */
            le32 algo;
            /* length of key */
            le32 key_len;
            /* See VIRTIO_CRYPTO_OP_* above */
            le32 op;
        } cipher;

        struct {
            /* See VIRTIO_CRYPTO_HASH_* or VIRTIO_CRYPTO_MAC_* above */
            le32 algo;
            /* length of authenticated key */
            le32 auth_key_len;
            /* See VIRTIO_CRYPTO_SYM_HASH_MODE_* above */
            le32 hash_mode;
        } hash;
    } sess_para;

    le32 iv_len;
    /* Length of source data */
    le32 src_data_len;
    /* Length of destination data */
    le32 dst_data_len;
    /* Starting point for cipher processing in source data */
    le32 cipher_start_src_offset;
    /* Length of the source data that the cipher will be computed on */
    le32 len_to_cipher;
    /* Starting point for hash processing in source data */
    le32 hash_start_src_offset;
    /* Length of the source data that the hash will be computed on */
    le32 len_to_hash;
    /* Length of the additional auth data */
    le32 aad_len;
    /* Length of the hash result */
    le32 hash_result_len;
    le32 reserved;
};

struct virtio_crypto_alg_chain_data_vlf_stateless {
    /* Device read only portion */

    /* The cipher key */
    u8 cipher_key[key_len];
    /* The auth key */
    u8 auth_key[auth_key_len];
    /* Initialization Vector or Counter data */
    u8 iv[iv_len];
    /* Additional authenticated data if exists */
    u8 aad[aad_len];
    /* Source data */
    u8 src_data[src_data_len];

    /* Device write only portion */

    /* Destination data */
    u8 dst_data[dst_data_len];
    /* Hash result data */
    u8 hash_result[hash_result_len];
};
\end{lstlisting}

Stateless mode requests of symmetric algorithm are as follows:

\begin{lstlisting}
struct virtio_crypto_sym_data_flf_stateless {
    /* Device read only portion */
#define VIRTIO_CRYPTO_SYM_DATE_REQ_HDR_STATELESS_SIZE    72
    u8 op_type_flf[VIRTIO_CRYPTO_SYM_DATE_REQ_HDR_STATELESS_SIZE];

    /* Device write only portion */
    /* See above VIRTIO_CRYPTO_SYM_OP_* */
    le32 op_type;
};

struct virtio_crypto_sym_data_vlf_stateless {
    u8 op_type_vlf[sym_para_len];
};
\end{lstlisting}

\field{op_type_flf} is the \field{op_type} specific header, it MUST starts
with or be one of the following structures:
\begin{itemize*}
\item struct virtio_crypto_cipher_data_flf_stateless
\item struct virtio_crypto_alg_chain_data_flf_stateless
\end{itemize*}

The length of \field{op_type_flf} is fixed to 72 bytes, the data of unused
part (if has) will be ignored.

\field{op_type_vlf} is the \field{op_type} specific parameters, it MUST starts
with or be one of the following structures:
\begin{itemize*}
\item struct virtio_crypto_cipher_data_vlf_stateless
\item struct virtio_crypto_alg_chain_data_vlf_stateless
\end{itemize*}

\field{sym_para_len} is the size of the specific structure used.

\drivernormative{\paragraph}{Symmetric algorithms Operation}{Device Types / Crypto Device / Device Operation / Symmetric algorithms Operation}

\begin{itemize*}
\item If the driver uses the session mode, then the driver MUST set \field{session_id}
    in struct virtio_crypto_op_header to a valid value assigned by the device when the
    session was created.
\item If the VIRTIO_CRYPTO_F_CIPHER_STATELESS_MODE feature bit is negotiated, 1) if the
    driver uses the stateless mode, then the driver MUST set the \field{flag} field in
    struct virtio_crypto_op_header to ZERO and MUST set the fields in struct
    virtio_crypto_cipher_data_flf_stateless.sess_para or struct
    virtio_crypto_alg_chain_data_flf_stateless.sess_para, 2) if the driver uses the
    session mode, then the driver MUST set the \field{flag} field in struct
    virtio_crypto_op_header to VIRTIO_CRYPTO_FLAG_SESSION_MODE.
\item The driver MUST set the \field{opcode} field in struct virtio_crypto_op_header
    to VIRTIO_CRYPTO_CIPHER_ENCRYPT or VIRTIO_CRYPTO_CIPHER_DECRYPT.
\item The driver MUST specify the fields of struct virtio_crypto_cipher_data_flf in
    struct virtio_crypto_sym_data_flf and struct virtio_crypto_cipher_data_vlf in
    struct virtio_crypto_sym_data_vlf if the request is based on VIRTIO_CRYPTO_SYM_OP_CIPHER.
\item The driver MUST specify the fields of struct virtio_crypto_alg_chain_data_flf
    in struct virtio_crypto_sym_data_flf and struct virtio_crypto_alg_chain_data_vlf
    in struct virtio_crypto_sym_data_vlf if the request is of the VIRTIO_CRYPTO_SYM_OP_ALGORITHM_CHAINING
    type.
\end{itemize*}

\devicenormative{\paragraph}{Symmetric algorithms Operation}{Device Types / Crypto Device / Device Operation / Symmetric algorithms Operation}

\begin{itemize*}
\item If the VIRTIO_CRYPTO_F_CIPHER_STATELESS_MODE feature bit is negotiated, the device
    MUST parse \field{flag} field in struct virtio_crypto_op_header in order to decide
	which mode the driver uses.
\item The device MUST parse the virtio_crypto_sym_data_req based on the \field{opcode}
    field in general header.
\item The device MUST parse the fields of struct virtio_crypto_cipher_data_flf in
    struct virtio_crypto_sym_data_flf and struct virtio_crypto_cipher_data_vlf in
    struct virtio_crypto_sym_data_vlf if the request is based on VIRTIO_CRYPTO_SYM_OP_CIPHER.
\item The device MUST parse the fields of struct virtio_crypto_alg_chain_data_flf
    in struct virtio_crypto_sym_data_flf and struct virtio_crypto_alg_chain_data_vlf
    in struct virtio_crypto_sym_data_vlf if the request is of the VIRTIO_CRYPTO_SYM_OP_ALGORITHM_CHAINING
    type.
\item The device MUST copy the result of cryptographic operation in the dst_data[] in
    both plain CIPHER mode and algorithms chain mode.
\item The device MUST check the \field{para}.\field{add_len} is bigger than 0 before
    parse the additional authenticated data in plain algorithms chain mode.
\item The device MUST copy the result of HASH/MAC operation in the hash_result[] is
    of the VIRTIO_CRYPTO_SYM_OP_ALGORITHM_CHAINING type.
\item The device MUST set the \field{status} field in struct virtio_crypto_inhdr to
    one of the following values of enum VIRTIO_CRYPTO_STATUS:
\begin{itemize*}
\item VIRTIO_CRYPTO_OK if the operation success.
\item VIRTIO_CRYPTO_NOTSUPP if the requested algorithm or operation is unsupported.
\item VIRTIO_CRYPTO_INVSESS if the session ID is invalid in session mode.
\item VIRTIO_CRYPTO_ERR if failure not mentioned above occurs.
\end{itemize*}
\end{itemize*}

\subsubsection{AEAD Service Operation}\label{sec:Device Types / Crypto Device / Device Operation / AEAD Service Operation}

Session mode requests of symmetric algorithm are as follows:

\begin{lstlisting}
struct virtio_crypto_aead_data_flf {
    /*
     * Byte Length of valid IV data.
     *
     * For GCM mode, this is either 12 (for 96-bit IVs) or 16, in which
     *   case iv points to J0.
     * For CCM mode, this is the length of the nonce, which can be in the
     *   range 7 to 13 inclusive.
     */
    le32 iv_len;
    /* length of additional auth data */
    le32 aad_len;
    /* length of source data */
    le32 src_data_len;
    /* length of dst data, this should be at least src_data_len + tag_len */
    le32 dst_data_len;
    /* Authentication tag length */
    le32 tag_len;
    le32 reserved;
};

struct virtio_crypto_aead_data_vlf {
    /* Device read only portion */

    /*
     * Initialization Vector data.
     *
     * For GCM mode, this is either the IV (if the length is 96 bits) or J0
     *   (for other sizes), where J0 is as defined by NIST SP800-38D.
     *   Regardless of the IV length, a full 16 bytes needs to be allocated.
     * For CCM mode, the first byte is reserved, and the nonce should be
     *   written starting at &iv[1] (to allow space for the implementation
     *   to write in the flags in the first byte).  Note that a full 16 bytes
     *   should be allocated, even though the iv_len field will have
     *   a value less than this.
     *
     * The IV will be updated after every partial cryptographic operation.
     */
    u8 iv[iv_len];
    /* Source data */
    u8 src_data[src_data_len];
    /* Additional authenticated data if exists */
    u8 aad[aad_len];

    /* Device write only portion */
    /* Pointer to output data */
    u8 dst_data[dst_data_len];
};
\end{lstlisting}

Each request uses the virtio_crypto_aead_data_flf structure and the
virtio_crypto_aead_data_flf structure to store information used to run the
AEAD operations.

Stateless mode AEAD service requests are as follows:

\begin{lstlisting}
struct virtio_crypto_aead_data_flf_stateless {
    struct {
        /* See VIRTIO_CRYPTO_AEAD_* above */
        le32 algo;
        /* length of key */
        le32 key_len;
        /* encrypt or decrypt, See above VIRTIO_CRYPTO_OP_* */
        le32 op;
    } sess_para;

    /* Byte Length of valid IV data. */
    le32 iv_len;
    /* Authentication tag length */
    le32 tag_len;
    /* length of additional auth data */
    le32 aad_len;
    /* length of source data */
    le32 src_data_len;
    /* length of dst data, this should be at least src_data_len + tag_len */
    le32 dst_data_len;
};

struct virtio_crypto_aead_data_vlf_stateless {
    /* Device read only portion */

    /* The cipher key */
    u8 key[key_len];
    /* Initialization Vector data. */
    u8 iv[iv_len];
    /* Source data */
    u8 src_data[src_data_len];
    /* Additional authenticated data if exists */
    u8 aad[aad_len];

    /* Device write only portion */
    /* Pointer to output data */
    u8 dst_data[dst_data_len];
};
\end{lstlisting}

\drivernormative{\paragraph}{AEAD Service Operation}{Device Types / Crypto Device / Device Operation / AEAD Service Operation}

\begin{itemize*}
\item If the driver uses the session mode, then the driver MUST set
    \field{session_id} in struct virtio_crypto_op_header to a valid value assigned
    by the device when the session was created.
\item If the VIRTIO_CRYPTO_F_AEAD_STATELESS_MODE feature bit is negotiated, 1) if
    the driver uses the stateless mode, then the driver MUST set the \field{flag}
    field in struct virtio_crypto_op_header to ZERO and MUST set the fields in
    struct virtio_crypto_aead_data_flf_stateless.sess_para, 2) if the driver uses
    the session mode, then the driver MUST set the \field{flag} field in struct
    virtio_crypto_op_header to VIRTIO_CRYPTO_FLAG_SESSION_MODE.
\item The driver MUST set the \field{opcode} field in struct virtio_crypto_op_header
    to VIRTIO_CRYPTO_AEAD_ENCRYPT or VIRTIO_CRYPTO_AEAD_DECRYPT.
\end{itemize*}

\devicenormative{\paragraph}{AEAD Service Operation}{Device Types / Crypto Device / Device Operation / AEAD Service Operation}

\begin{itemize*}
\item If the VIRTIO_CRYPTO_F_AEAD_STATELESS_MODE feature bit is negotiated, the
    device MUST parse the virtio_crypto_aead_data_vlf_stateless based on the \field{opcode}
	field in general header.
\item The device MUST copy the result of cryptographic operation in the dst_data[].
\item The device MUST copy the authentication tag in the dst_data[] offset the cipher result.
\item The device MUST set the \field{status} field in struct virtio_crypto_inhdr to
    one of the following values of enum VIRTIO_CRYPTO_STATUS:
\item When the \field{opcode} field is VIRTIO_CRYPTO_AEAD_DECRYPT, the device MUST
    verify and return the verification result to the driver.
\begin{itemize*}
\item VIRTIO_CRYPTO_OK if the operation success.
\item VIRTIO_CRYPTO_NOTSUPP if the requested algorithm or operation is unsupported.
\item VIRTIO_CRYPTO_BADMSG if the verification result is incorrect.
\item VIRTIO_CRYPTO_INVSESS if the session ID invalid when in session mode.
\item VIRTIO_CRYPTO_ERR if any failure not mentioned above occurs.
\end{itemize*}
\end{itemize*}

\subsubsection{AKCIPHER Service Operation}\label{sec:Device Types / Crypto Device / Device Operation / AKCIPHER Service Operation}

Session mode AKCIPHER requests are as follows:

\begin{lstlisting}
struct virtio_crypto_akcipher_data_flf {
    /* length of source data */
    le32 src_data_len;
    /* length of dst data */
    le32 dst_data_len;
};

struct virtio_crypto_akcipher_data_vlf {
    /* Device read only portion */
    /* Source data */
    u8 src_data[src_data_len];

    /* Device write only portion */
    /* Pointer to output data */
    u8 dst_data[dst_data_len];
};
\end{lstlisting}

Each data request uses the virtio_crypto_akcipher_flf structure and the virtio_crypto_akcipher_data_vlf
structure to store information used to run the AKCIPHER operations.

For encryption, decryption, and signing:
\field{src_data} is the source data that will be processed, note that for signing operations,
src_data stores the data to be signed, which usually is the digest of some data rather than the
data itself.
\field{src_data_len} is the length of source data.
\field{dst_result} is the result data and \field{dst_data_len} is the length of it. Note that the
length of the result is not always exactly equal to dst_data_len, the driver needs to check how
many bytes the device has written and calculate the actual length of the result.

For verification:
\field{src_data_len} refers to the length of the signature, and \field{dst_data_len} refers to
the length of signed data, where the signed data is usually the digest of some data.
\field{src_data} is spliced by the signature and the signed data, the src_data with the lower
address stores the signature, and the higher address stores the signed data.
\field{dst_data} is always empty for verification.

Different algorithms have different signature formats.
For the RSA algorithm, the result is determined by the padding algorithm specified by
\field{padding_algo} in structure virtio_crypto_rsa_session_para.

For the ECDSA algorithm, the signature is composed of the following
ASN.1 structure (see \hyperref[intro:rfc3279]{RFC3279})
and MUST be DER encoded (see \hyperref[intro:rfc6025]{rfc6025}).

\begin{lstlisting}
Ecdsa-Sig-Value ::= SEQUENCE {
    r INTEGER,
    s INTEGER
}
\end{lstlisting}

Stateless mode AKCIPHER service requests are as follows:
\begin{lstlisting}
struct virtio_crypto_akcipher_data_flf_stateless {
    struct {
        /* See VIRTIO_CYRPTO_AKCIPHER* above */
        le32 algo;
        /* See VIRTIO_CRYPTO_AKCIPHER_KEY_TYPE_* above */
        le32 key_type;
        /* length of key */
        le32 key_len;

        /* algothrim specific parameters described above */
        union {
            struct virtio_crypto_rsa_session_para rsa;
            struct virtio_crypto_ecdsa_session_para ecdsa;
        } u;
    } sess_para;

    /* length of source data */
    le32 src_data_len;
    /* length of destination data */
    le32 dst_data_len;
};

struct virtio_crypto_akcipher_data_vlf_stateless {
    /* Device read only portion */
    u8 akcipher_key[key_len];

    /* Source data */
    u8 src_data[src_data_len];

    /* Device write only portion */
    u8 dst_data[dst_data_len];
};
\end{lstlisting}

In stateless mode, the format of key and signature, the meaning of src_data and dst_data, are all the same
with session mode.

\drivernormative{\paragraph}{AKCIPHER Service Operation}{Device Types / Crypto Device / Device Operation / AKCIPHER Service Operation}

\begin{itemize*}
\item If the driver uses the session mode, then the driver MUST set
    \field{session_id} in struct virtio_crypto_op_header to a valid
    value assigned by the device when the session was created.
\item If the VIRTIO_CRYPTO_F_AKCIPHER_STATELESS_MODE feature bit is negotiated, 1) if the
    driver uses the stateless mode, then the driver MUST set the \field{flag} field in
    struct virtio_crypto_op_header to ZERO and MUST set the fields in struct
    virtio_crypto_akcipher_flf_stateless.sess_para, 2) if the driver uses the session
    mode, then the driver MUST set the \field{flag} field in struct virtio_crypto_op_header
    to VIRTIO_CRYPTO_FLAG_SESSION_MODE.
\item The driver MUST set the \field{opcode} field in struct virtio_crypto_op_header
    to one of VIRTIO_CRYPTO_AKCIPHER_ENCRYPT, VIRTIO_CRYPTO_AKCIPHER_DESTROY_SESSION,
    VIRTIO_CRYPTO_AKCIPHER_SIGN, and VIRTIO_CRYPTO_AKCIPHER_VERIFY.
\end{itemize*}

\devicenormative{\paragraph}{AKCIPHER Service Operation}{Device Types / Crypto Device / Device Operation / AKCIPHER Service Operation}

\begin{itemize*}
\item If the VIRTIO_CRYPTO_F_AKCIPHER_STATELESS_MODE feature bit is negotiated, the
    device MUST parse the virtio_crypto_akcipher_data_vlf_stateless based on the \field{opcode}
    field in general header.
\item The device MUST copy the result of cryptographic operation in the dst_data[].
\item The device MUST set the \field{status} field in struct virtio_crypto_inhdr to
    one of the following values of enum VIRTIO_CRYPTO_STATUS:
\begin{itemize*}
\item VIRTIO_CRYPTO_OK if the operation success.
\item VIRTIO_CRYPTO_NOTSUPP if the requested algorithm or operation is unsupported.
\item VIRTIO_CRYPTO_BADMSG if the verification result is incorrect.
\item VIRTIO_CRYPTO_INVSESS if the session ID invalid when in session mode.
\item VIRTIO_CRYPTO_KEY_REJECTED if the signature verification failed.
\item VIRTIO_CRYPTO_ERR if any failure not mentioned above occurs.
\end{itemize*}
\end{itemize*}

\section{Crypto Device}\label{sec:Device Types / Crypto Device}

The virtio crypto device is a virtual cryptography device as well as a
virtual cryptographic accelerator. The virtio crypto device provides the
following crypto services: CIPHER, MAC, HASH, AEAD and AKCIPHER. Virtio crypto
devices have a single control queue and at least one data queue. Crypto
operation requests are placed into a data queue, and serviced by the
device. Some crypto operation requests are only valid in the context of a
session. The role of the control queue is facilitating control operation
requests. Sessions management is realized with control operation
requests.

\subsection{Device ID}\label{sec:Device Types / Crypto Device / Device ID}

20

\subsection{Virtqueues}\label{sec:Device Types / Crypto Device / Virtqueues}

\begin{description}
\item[0] dataq1
\item[\ldots]
\item[N-1] dataqN
\item[N] controlq
\end{description}

N is set by \field{max_dataqueues}.

\subsection{Feature bits}\label{sec:Device Types / Crypto Device / Feature bits}

\begin{description}
\item VIRTIO_CRYPTO_F_REVISION_1 (0) revision 1. Revision 1 has a specific
    request format and other enhancements (which result in some additional
    requirements).
\item VIRTIO_CRYPTO_F_CIPHER_STATELESS_MODE (1) stateless mode requests are
    supported by the CIPHER service.
\item VIRTIO_CRYPTO_F_HASH_STATELESS_MODE (2) stateless mode requests are
    supported by the HASH service.
\item VIRTIO_CRYPTO_F_MAC_STATELESS_MODE (3) stateless mode requests are
    supported by the MAC service.
\item VIRTIO_CRYPTO_F_AEAD_STATELESS_MODE (4) stateless mode requests are
    supported by the AEAD service.
\item VIRTIO_CRYPTO_F_AKCIPHER_STATELESS_MODE (5) stateless mode requests are
    supported by the AKCIPHER service.
\end{description}


\subsubsection{Feature bit requirements}\label{sec:Device Types / Crypto Device / Feature bit requirements}

Some crypto feature bits require other crypto feature bits
(see \ref{drivernormative:Basic Facilities of a Virtio Device / Feature Bits}):

\begin{description}
\item[VIRTIO_CRYPTO_F_CIPHER_STATELESS_MODE] Requires VIRTIO_CRYPTO_F_REVISION_1.
\item[VIRTIO_CRYPTO_F_HASH_STATELESS_MODE] Requires VIRTIO_CRYPTO_F_REVISION_1.
\item[VIRTIO_CRYPTO_F_MAC_STATELESS_MODE] Requires VIRTIO_CRYPTO_F_REVISION_1.
\item[VIRTIO_CRYPTO_F_AEAD_STATELESS_MODE] Requires VIRTIO_CRYPTO_F_REVISION_1.
\item[VIRTIO_CRYPTO_F_AKCIPHER_STATELESS_MODE] Requires VIRTIO_CRYPTO_F_REVISION_1.
\end{description}

\subsection{Supported crypto services}\label{sec:Device Types / Crypto Device / Supported crypto services}

The following crypto services are defined:

\begin{lstlisting}
/* CIPHER (Symmetric Key Cipher) service */
#define VIRTIO_CRYPTO_SERVICE_CIPHER 0
/* HASH service */
#define VIRTIO_CRYPTO_SERVICE_HASH   1
/* MAC (Message Authentication Codes) service */
#define VIRTIO_CRYPTO_SERVICE_MAC    2
/* AEAD (Authenticated Encryption with Associated Data) service */
#define VIRTIO_CRYPTO_SERVICE_AEAD   3
/* AKCIPHER (Asymmetric Key Cipher) service */
#define VIRTIO_CRYPTO_SERVICE_AKCIPHER 4
\end{lstlisting}

The above constants designate bits used to indicate the which of crypto services are
offered by the device as described in, see \ref{sec:Device Types / Crypto Device / Device configuration layout}.

\subsubsection{CIPHER services}\label{sec:Device Types / Crypto Device / Supported crypto services / CIPHER services}

The following CIPHER algorithms are defined:

\begin{lstlisting}
#define VIRTIO_CRYPTO_NO_CIPHER                 0
#define VIRTIO_CRYPTO_CIPHER_ARC4               1
#define VIRTIO_CRYPTO_CIPHER_AES_ECB            2
#define VIRTIO_CRYPTO_CIPHER_AES_CBC            3
#define VIRTIO_CRYPTO_CIPHER_AES_CTR            4
#define VIRTIO_CRYPTO_CIPHER_DES_ECB            5
#define VIRTIO_CRYPTO_CIPHER_DES_CBC            6
#define VIRTIO_CRYPTO_CIPHER_3DES_ECB           7
#define VIRTIO_CRYPTO_CIPHER_3DES_CBC           8
#define VIRTIO_CRYPTO_CIPHER_3DES_CTR           9
#define VIRTIO_CRYPTO_CIPHER_KASUMI_F8          10
#define VIRTIO_CRYPTO_CIPHER_SNOW3G_UEA2        11
#define VIRTIO_CRYPTO_CIPHER_AES_F8             12
#define VIRTIO_CRYPTO_CIPHER_AES_XTS            13
#define VIRTIO_CRYPTO_CIPHER_ZUC_EEA3           14
\end{lstlisting}

The above constants have two usages:
\begin{enumerate}
\item As bit numbers, used to tell the driver which CIPHER algorithms
are supported by the device, see \ref{sec:Device Types / Crypto Device / Device configuration layout}.
\item As values, used to designate the algorithm in (CIPHER type) crypto
operation requests, see \ref{sec:Device Types / Crypto Device / Device Operation / Control Virtqueue / Session operation}.
\end{enumerate}

\subsubsection{HASH services}\label{sec:Device Types / Crypto Device / Supported crypto services / HASH services}

The following HASH algorithms are defined:

\begin{lstlisting}
#define VIRTIO_CRYPTO_NO_HASH            0
#define VIRTIO_CRYPTO_HASH_MD5           1
#define VIRTIO_CRYPTO_HASH_SHA1          2
#define VIRTIO_CRYPTO_HASH_SHA_224       3
#define VIRTIO_CRYPTO_HASH_SHA_256       4
#define VIRTIO_CRYPTO_HASH_SHA_384       5
#define VIRTIO_CRYPTO_HASH_SHA_512       6
#define VIRTIO_CRYPTO_HASH_SHA3_224      7
#define VIRTIO_CRYPTO_HASH_SHA3_256      8
#define VIRTIO_CRYPTO_HASH_SHA3_384      9
#define VIRTIO_CRYPTO_HASH_SHA3_512      10
#define VIRTIO_CRYPTO_HASH_SHA3_SHAKE128      11
#define VIRTIO_CRYPTO_HASH_SHA3_SHAKE256      12
\end{lstlisting}

The above constants have two usages:
\begin{enumerate}
\item As bit numbers, used to tell the driver which HASH algorithms
are supported by the device, see \ref{sec:Device Types / Crypto Device / Device configuration layout}.
\item As values, used to designate the algorithm in (HASH type) crypto
operation requires, see \ref{sec:Device Types / Crypto Device / Device Operation / Control Virtqueue / Session operation}.
\end{enumerate}

\subsubsection{MAC services}\label{sec:Device Types / Crypto Device / Supported crypto services / MAC services}

The following MAC algorithms are defined:

\begin{lstlisting}
#define VIRTIO_CRYPTO_NO_MAC                       0
#define VIRTIO_CRYPTO_MAC_HMAC_MD5                 1
#define VIRTIO_CRYPTO_MAC_HMAC_SHA1                2
#define VIRTIO_CRYPTO_MAC_HMAC_SHA_224             3
#define VIRTIO_CRYPTO_MAC_HMAC_SHA_256             4
#define VIRTIO_CRYPTO_MAC_HMAC_SHA_384             5
#define VIRTIO_CRYPTO_MAC_HMAC_SHA_512             6
#define VIRTIO_CRYPTO_MAC_CMAC_3DES                25
#define VIRTIO_CRYPTO_MAC_CMAC_AES                 26
#define VIRTIO_CRYPTO_MAC_KASUMI_F9                27
#define VIRTIO_CRYPTO_MAC_SNOW3G_UIA2              28
#define VIRTIO_CRYPTO_MAC_GMAC_AES                 41
#define VIRTIO_CRYPTO_MAC_GMAC_TWOFISH             42
#define VIRTIO_CRYPTO_MAC_CBCMAC_AES               49
#define VIRTIO_CRYPTO_MAC_CBCMAC_KASUMI_F9         50
#define VIRTIO_CRYPTO_MAC_XCBC_AES                 53
#define VIRTIO_CRYPTO_MAC_ZUC_EIA3                 54
\end{lstlisting}

The above constants have two usages:
\begin{enumerate}
\item As bit numbers, used to tell the driver which MAC algorithms
are supported by the device, see \ref{sec:Device Types / Crypto Device / Device configuration layout}.
\item As values, used to designate the algorithm in (MAC type) crypto
operation requests, see \ref{sec:Device Types / Crypto Device / Device Operation / Control Virtqueue / Session operation}.
\end{enumerate}

\subsubsection{AEAD services}\label{sec:Device Types / Crypto Device / Supported crypto services / AEAD services}

The following AEAD algorithms are defined:

\begin{lstlisting}
#define VIRTIO_CRYPTO_NO_AEAD     0
#define VIRTIO_CRYPTO_AEAD_GCM    1
#define VIRTIO_CRYPTO_AEAD_CCM    2
#define VIRTIO_CRYPTO_AEAD_CHACHA20_POLY1305  3
\end{lstlisting}

The above constants have two usages:
\begin{enumerate}
\item As bit numbers, used to tell the driver which AEAD algorithms
are supported by the device, see \ref{sec:Device Types / Crypto Device / Device configuration layout}.
\item As values, used to designate the algorithm in (DEAD type) crypto
operation requests, see \ref{sec:Device Types / Crypto Device / Device Operation / Control Virtqueue / Session operation}.
\end{enumerate}

\subsubsection{AKCIPHER services}\label{sec: Device Types / Crypto Device / Supported crypto services / AKCIPHER services}

The following AKCIPHER algorithms are defined:
\begin{lstlisting}
#define VIRTIO_CRYPTO_NO_AKCIPHER 0
#define VIRTIO_CRYPTO_AKCIPHER_RSA   1
#define VIRTIO_CRYPTO_AKCIPHER_ECDSA 2
\end{lstlisting}

The above constants have two usages:
\begin{enumerate}
\item As bit numbers, used to tell the driver which AKCIPHER algorithms
are supported by the device, see \ref{sec:Device Types / Crypto Device / Device configuration layout}.
\item As values, used to designate the algorithm in asymmetric crypto operation requests,
see \ref{sec:Device Types / Crypto Device / Device Operation / Control Virtqueue / Session operation}.
\end{enumerate}


\subsection{Device configuration layout}\label{sec:Device Types / Crypto Device / Device configuration layout}

Crypto device configuration uses the following layout structure:

\begin{lstlisting}
struct virtio_crypto_config {
    le32 status;
    le32 max_dataqueues;
    le32 crypto_services;
    /* Detailed algorithms mask */
    le32 cipher_algo_l;
    le32 cipher_algo_h;
    le32 hash_algo;
    le32 mac_algo_l;
    le32 mac_algo_h;
    le32 aead_algo;
    /* Maximum length of cipher key in bytes */
    le32 max_cipher_key_len;
    /* Maximum length of authenticated key in bytes */
    le32 max_auth_key_len;
    le32 akcipher_algo;
    /* Maximum size of each crypto request's content in bytes */
    le64 max_size;
};
\end{lstlisting}

\begin{description}
\item Currently, only one \field{status} bit is defined: VIRTIO_CRYPTO_S_HW_READY
    set indicates that the device is ready to process requests, this bit is read-only
    for the driver
\begin{lstlisting}
#define VIRTIO_CRYPTO_S_HW_READY  (1 << 0)
\end{lstlisting}

\item [\field{max_dataqueues}] is the maximum number of data virtqueues that can
    be configured by the device. The driver MAY use only one data queue, or it
    can use more to achieve better performance.

\item [\field{crypto_services}] crypto service offered, see \ref{sec:Device Types / Crypto Device / Supported crypto services}.

\item [\field{cipher_algo_l}] CIPHER algorithms bits 0-31, see \ref{sec:Device Types / Crypto Device / Supported crypto services  / CIPHER services}.

\item [\field{cipher_algo_h}] CIPHER algorithms bits 32-63, see \ref{sec:Device Types / Crypto Device / Supported crypto services  / CIPHER services}.

\item [\field{hash_algo}] HASH algorithms bits, see \ref{sec:Device Types / Crypto Device / Supported crypto services  / HASH services}.

\item [\field{mac_algo_l}] MAC algorithms bits 0-31, see \ref{sec:Device Types / Crypto Device / Supported crypto services  / MAC services}.

\item [\field{mac_algo_h}] MAC algorithms bits 32-63, see \ref{sec:Device Types / Crypto Device / Supported crypto services  / MAC services}.

\item [\field{aead_algo}] AEAD algorithms bits, see \ref{sec:Device Types / Crypto Device / Supported crypto services  / AEAD services}.

\item [\field{max_cipher_key_len}] is the maximum length of cipher key supported by the device.

\item [\field{max_auth_key_len}] is the maximum length of authenticated key supported by the device.

\item [\field{akcipher_algo}] AKCIPHER algorithms bit 0-31, see \ref{sec: Device Types / Crypto Device / Supported crypto services / AKCIPHER services}.

\item [\field{max_size}] is the maximum size of the variable-length parameters of
    data operation of each crypto request's content supported by the device.
\end{description}

\begin{note}
Unless explicitly stated otherwise all lengths and sizes are in bytes.
\end{note}

\devicenormative{\subsubsection}{Device configuration layout}{Device Types / Crypto Device / Device configuration layout}

\begin{itemize*}
\item The device MUST set \field{max_dataqueues} to between 1 and 65535 inclusive.
\item The device MUST set the \field{status} with valid flags, undefined flags MUST NOT be set.
\item The device MUST accept and handle requests after \field{status} is set to VIRTIO_CRYPTO_S_HW_READY.
\item The device MUST set \field{crypto_services} based on the crypto services the device offers.
\item The device MUST set detailed algorithms masks for each service advertised by \field{crypto_services}.
    The device MUST NOT set the not defined algorithms bits.
\item The device MUST set \field{max_size} to show the maximum size of crypto request the device supports.
\item The device MUST set \field{max_cipher_key_len} to show the maximum length of cipher key if the
    device supports CIPHER service.
\item The device MUST set \field{max_auth_key_len} to show the maximum length of authenticated key if
    the device supports MAC service.
\end{itemize*}

\drivernormative{\subsubsection}{Device configuration layout}{Device Types / Crypto Device / Device configuration layout}

\begin{itemize*}
\item The driver MUST read the \field{status} from the bottom bit of status to check whether the
    VIRTIO_CRYPTO_S_HW_READY is set, and the driver MUST reread it after device reset.
\item The driver MUST NOT transmit any requests to the device if the VIRTIO_CRYPTO_S_HW_READY is not set.
\item The driver MUST read \field{max_dataqueues} field to discover the number of data queues the device supports.
\item The driver MUST read \field{crypto_services} field to discover which services the device is able to offer.
\item The driver SHOULD ignore the not defined algorithms bits.
\item The driver MUST read the detailed algorithms fields based on \field{crypto_services} field.
\item The driver SHOULD read \field{max_size} to discover the maximum size of the variable-length
    parameters of data operation of the crypto request's content the device supports and MUST
    guarantee that the size of each crypto request's content is within the \field{max_size}, otherwise
    the request will fail and the driver MUST reset the device.
\item The driver SHOULD read \field{max_cipher_key_len} to discover the maximum length of cipher key
    the device supports and MUST guarantee that the \field{key_len} (CIPHER service or AEAD service) is within
    the \field{max_cipher_key_len} of the device configuration, otherwise the request will fail.
\item The driver SHOULD read \field{max_auth_key_len} to discover the maximum length of authenticated
    key the device supports and MUST guarantee that the \field{auth_key_len} (MAC service) is within the
    \field{max_auth_key_len} of the device configuration, otherwise the request will fail.
\end{itemize*}

\subsection{Device Initialization}\label{sec:Device Types / Crypto Device / Device Initialization}

\drivernormative{\subsubsection}{Device Initialization}{Device Types / Crypto Device / Device Initialization}

\begin{itemize*}
\item The driver MUST configure and initialize all virtqueues.
\item The driver MUST read the supported crypto services from bits of \field{crypto_services}.
\item The driver MUST read the supported algorithms based on \field{crypto_services} field.
\end{itemize*}

\subsection{Device Operation}\label{sec:Device Types / Crypto Device / Device Operation}

The operation of a virtio crypto device is driven by requests placed on the virtqueues.
Requests consist of a queue-type specific header (specifying among others the operation)
and an operation specific payload.

If VIRTIO_CRYPTO_F_REVISION_1 is negotiated the device may support both session mode
(See \ref{sec:Device Types / Crypto Device / Device Operation / Control Virtqueue / Session operation})
and stateless mode operation requests.
In stateless mode all operation parameters are supplied as a part of each request,
while in session mode, some or all operation parameters are managed within the
session. Stateless mode is guarded by feature bits 0-4 on a service level. If
stateless mode is negotiated for a service, the service accepts both session
mode and stateless requests; otherwise stateless mode requests are rejected
(via operation status).

\subsubsection{Operation Status}\label{sec:Device Types / Crypto Device / Device Operation / Operation status}
The device MUST return a status code as part of the operation (both session
operation and service operation) result. The valid operation status as follows:

\begin{lstlisting}
enum VIRTIO_CRYPTO_STATUS {
    VIRTIO_CRYPTO_OK = 0,
    VIRTIO_CRYPTO_ERR = 1,
    VIRTIO_CRYPTO_BADMSG = 2,
    VIRTIO_CRYPTO_NOTSUPP = 3,
    VIRTIO_CRYPTO_INVSESS = 4,
    VIRTIO_CRYPTO_NOSPC = 5,
    VIRTIO_CRYPTO_KEY_REJECTED = 6,
    VIRTIO_CRYPTO_MAX
};
\end{lstlisting}

\begin{itemize*}
\item VIRTIO_CRYPTO_OK: success.
\item VIRTIO_CRYPTO_BADMSG: authentication failed (only when AEAD decryption).
\item VIRTIO_CRYPTO_NOTSUPP: operation or algorithm is unsupported.
\item VIRTIO_CRYPTO_INVSESS: invalid session ID when executing crypto operations.
\item VIRTIO_CRYPTO_NOSPC: no free session ID (only when the VIRTIO_CRYPTO_F_REVISION_1
    feature bit is negotiated).
\item VIRTIO_CRYPTO_KEY_REJECTED: signature verification failed (only when AKCIPHER verification).
\item VIRTIO_CRYPTO_ERR: any failure not mentioned above occurs.
\end{itemize*}

\subsubsection{Control Virtqueue}\label{sec:Device Types / Crypto Device / Device Operation / Control Virtqueue}

The driver uses the control virtqueue to send control commands to the
device, such as session operations (See \ref{sec:Device Types / Crypto Device / Device
Operation / Control Virtqueue / Session operation}).

The header for controlq is of the following form:
\begin{lstlisting}
#define VIRTIO_CRYPTO_OPCODE(service, op)   (((service) << 8) | (op))

struct virtio_crypto_ctrl_header {
#define VIRTIO_CRYPTO_CIPHER_CREATE_SESSION \
       VIRTIO_CRYPTO_OPCODE(VIRTIO_CRYPTO_SERVICE_CIPHER, 0x02)
#define VIRTIO_CRYPTO_CIPHER_DESTROY_SESSION \
       VIRTIO_CRYPTO_OPCODE(VIRTIO_CRYPTO_SERVICE_CIPHER, 0x03)
#define VIRTIO_CRYPTO_HASH_CREATE_SESSION \
       VIRTIO_CRYPTO_OPCODE(VIRTIO_CRYPTO_SERVICE_HASH, 0x02)
#define VIRTIO_CRYPTO_HASH_DESTROY_SESSION \
       VIRTIO_CRYPTO_OPCODE(VIRTIO_CRYPTO_SERVICE_HASH, 0x03)
#define VIRTIO_CRYPTO_MAC_CREATE_SESSION \
       VIRTIO_CRYPTO_OPCODE(VIRTIO_CRYPTO_SERVICE_MAC, 0x02)
#define VIRTIO_CRYPTO_MAC_DESTROY_SESSION \
       VIRTIO_CRYPTO_OPCODE(VIRTIO_CRYPTO_SERVICE_MAC, 0x03)
#define VIRTIO_CRYPTO_AEAD_CREATE_SESSION \
       VIRTIO_CRYPTO_OPCODE(VIRTIO_CRYPTO_SERVICE_AEAD, 0x02)
#define VIRTIO_CRYPTO_AEAD_DESTROY_SESSION \
       VIRTIO_CRYPTO_OPCODE(VIRTIO_CRYPTO_SERVICE_AEAD, 0x03)
#define VIRTIO_CRYPTO_AKCIPHER_CREATE_SESSION \
       VIRTIO_CRYPTO_OPCODE(VIRTIO_CRYPTO_SERVICE_AKCIPHER, 0x04)
#define VIRTIO_CRYPTO_AKCIPHER_DESTROY_SESSION \
       VIRTIO_CRYPTO_OPCDE(VIRTIO_CRYPTO_SERVICE_AKCIPHER, 0x05)
    le32 opcode;
    /* algo should be service-specific algorithms */
    le32 algo;
    le32 flag;
    le32 reserved;
};
\end{lstlisting}

The controlq request is composed of four parts:
\begin{lstlisting}
struct virtio_crypto_op_ctrl_req {
    /* Device read only portion */

    struct virtio_crypto_ctrl_header header;

#define VIRTIO_CRYPTO_CTRLQ_OP_SPEC_HDR_LEGACY 56
    /* fixed length fields, opcode specific */
    u8 op_flf[flf_len];

    /* variable length fields, opcode specific */
    u8 op_vlf[vlf_len];

    /* Device write only portion */

    /* op result or completion status */
    u8 op_outcome[outcome_len];
};
\end{lstlisting}

\field{header} is a general header (see above).

\field{op_flf} is the opcode (in \field{header}) specific fixed-length parameters.

\field{flf_len} depends on the VIRTIO_CRYPTO_F_REVISION_1 feature bit (see below).

\field{op_vlf} is the opcode (in \field{header}) specific variable-length parameters.

\field{vlf_len} is the size of the specific structure used.
\begin{note}
The \field{vlf_len} of session-destroy operation and the hash-session-create
operation is ZERO.
\end{note}

\begin{itemize*}
\item If the opcode (in \field{header}) is VIRTIO_CRYPTO_CIPHER_CREATE_SESSION
    then \field{op_flf} is struct virtio_crypto_sym_create_session_flf if
    VIRTIO_CRYPTO_F_REVISION_1 is negotiated and struct virtio_crypto_sym_create_session_flf is
    padded to 56 bytes if NOT negotiated, and \field{op_vlf} is struct
    virtio_crypto_sym_create_session_vlf.
\item If the opcode (in \field{header}) is VIRTIO_CRYPTO_HASH_CREATE_SESSION
    then \field{op_flf} is struct virtio_crypto_hash_create_session_flf if
    VIRTIO_CRYPTO_F_REVISION_1 is negotiated and struct virtio_crypto_hash_create_session_flf is
    padded to 56 bytes if NOT negotiated.
\item If the opcode (in \field{header}) is VIRTIO_CRYPTO_MAC_CREATE_SESSION
    then \field{op_flf} is struct virtio_crypto_mac_create_session_flf if
    VIRTIO_CRYPTO_F_REVISION_1 is negotiated and struct virtio_crypto_mac_create_session_flf is
    padded to 56 bytes if NOT negotiated, and \field{op_vlf} is struct
    virtio_crypto_mac_create_session_vlf.
\item If the opcode (in \field{header}) is VIRTIO_CRYPTO_AEAD_CREATE_SESSION
    then \field{op_flf} is struct virtio_crypto_aead_create_session_flf if
    VIRTIO_CRYPTO_F_REVISION_1 is negotiated and struct virtio_crypto_aead_create_session_flf is
    padded to 56 bytes if NOT negotiated, and \field{op_vlf} is struct
    virtio_crypto_aead_create_session_vlf.
\item If the opcode (in \field{header}) is VIRTIO_CRYPTO_AKCIPHER_CREATE_SESSION
    then \field{op_flf} is struct virtio_crypto_akcipher_create_session_flf if
    VIRTIO_CRYPTO_F_REVISION_1 is negotiated and struct virtio_crypto_akcipher_create_session_flf is
    padded to 56 bytes if NOT negotiated, and \field{op_vlf} is struct
    virtio_crypto_akcipher_create_session_vlf.
\item If the opcode (in \field{header}) is VIRTIO_CRYPTO_CIPHER_DESTROY_SESSION
    or VIRTIO_CRYPTO_HASH_DESTROY_SESSION or VIRTIO_CRYPTO_MAC_DESTROY_SESSION or
    VIRTIO_CRYPTO_AEAD_DESTROY_SESSION then \field{op_flf} is struct
    virtio_crypto_destroy_session_flf if VIRTIO_CRYPTO_F_REVISION_1 is negotiated and
    struct virtio_crypto_destroy_session_flf is padded to 56 bytes if NOT negotiated.
\end{itemize*}

\field{op_outcome} stores the result of operation and must be struct
virtio_crypto_destroy_session_input for destroy session or
struct virtio_crypto_create_session_input for create session.

\field{outcome_len} is the size of the structure used.


\paragraph{Session operation}\label{sec:Device Types / Crypto Device / Device
Operation / Control Virtqueue / Session operation}

The session is a handle which describes the cryptographic parameters to be
applied to a number of buffers.

The following structure stores the result of session creation set by the device:

\begin{lstlisting}
struct virtio_crypto_create_session_input {
    le64 session_id;
    le32 status;
    le32 padding;
};
\end{lstlisting}

A request to destroy a session includes the following information:

\begin{lstlisting}
struct virtio_crypto_destroy_session_flf {
    /* Device read only portion */
    le64  session_id;
};

struct virtio_crypto_destroy_session_input {
    /* Device write only portion */
    u8  status;
};
\end{lstlisting}


\subparagraph{Session operation: HASH session}\label{sec:Device Types / Crypto Device / Device
Operation / Control Virtqueue / Session operation / Session operation: HASH session}

The fixed-length parameters of HASH session requests is as follows:

\begin{lstlisting}
struct virtio_crypto_hash_create_session_flf {
    /* Device read only portion */

    /* See VIRTIO_CRYPTO_HASH_* above */
    le32 algo;
    /* hash result length */
    le32 hash_result_len;
};
\end{lstlisting}


\subparagraph{Session operation: MAC session}\label{sec:Device Types / Crypto Device / Device
Operation / Control Virtqueue / Session operation / Session operation: MAC session}

The fixed-length and the variable-length parameters of MAC session requests are as follows:

\begin{lstlisting}
struct virtio_crypto_mac_create_session_flf {
    /* Device read only portion */

    /* See VIRTIO_CRYPTO_MAC_* above */
    le32 algo;
    /* hash result length */
    le32 hash_result_len;
    /* length of authenticated key */
    le32 auth_key_len;
    le32 padding;
};

struct virtio_crypto_mac_create_session_vlf {
    /* Device read only portion */

    /* The authenticated key */
    u8 auth_key[auth_key_len];
};
\end{lstlisting}

The length of \field{auth_key} is specified in \field{auth_key_len} in the struct
virtio_crypto_mac_create_session_flf.


\subparagraph{Session operation: Symmetric algorithms session}\label{sec:Device Types / Crypto Device / Device
Operation / Control Virtqueue / Session operation / Session operation: Symmetric algorithms session}

The request of symmetric session could be the CIPHER algorithms request
or the chain algorithms (chaining CIPHER and HASH/MAC) request.

The fixed-length and the variable-length parameters of CIPHER session requests are as follows:

\begin{lstlisting}
struct virtio_crypto_cipher_session_flf {
    /* Device read only portion */

    /* See VIRTIO_CRYPTO_CIPHER* above */
    le32 algo;
    /* length of key */
    le32 key_len;
#define VIRTIO_CRYPTO_OP_ENCRYPT  1
#define VIRTIO_CRYPTO_OP_DECRYPT  2
    /* encryption or decryption */
    le32 op;
    le32 padding;
};

struct virtio_crypto_cipher_session_vlf {
    /* Device read only portion */

    /* The cipher key */
    u8 cipher_key[key_len];
};
\end{lstlisting}

The length of \field{cipher_key} is specified in \field{key_len} in the struct
virtio_crypto_cipher_session_flf.

The fixed-length and the variable-length parameters of Chain session requests are as follows:

\begin{lstlisting}
struct virtio_crypto_alg_chain_session_flf {
    /* Device read only portion */

#define VIRTIO_CRYPTO_SYM_ALG_CHAIN_ORDER_HASH_THEN_CIPHER  1
#define VIRTIO_CRYPTO_SYM_ALG_CHAIN_ORDER_CIPHER_THEN_HASH  2
    le32 alg_chain_order;
/* Plain hash */
#define VIRTIO_CRYPTO_SYM_HASH_MODE_PLAIN    1
/* Authenticated hash (mac) */
#define VIRTIO_CRYPTO_SYM_HASH_MODE_AUTH     2
/* Nested hash */
#define VIRTIO_CRYPTO_SYM_HASH_MODE_NESTED   3
    le32 hash_mode;
    struct virtio_crypto_cipher_session_flf cipher_hdr;

#define VIRTIO_CRYPTO_ALG_CHAIN_SESS_OP_SPEC_HDR_SIZE  16
    /* fixed length fields, algo specific */
    u8 algo_flf[VIRTIO_CRYPTO_ALG_CHAIN_SESS_OP_SPEC_HDR_SIZE];

    /* length of the additional authenticated data (AAD) in bytes */
    le32 aad_len;
    le32 padding;
};

struct virtio_crypto_alg_chain_session_vlf {
    /* Device read only portion */

    /* The cipher key */
    u8 cipher_key[key_len];
    /* The authenticated key */
    u8 auth_key[auth_key_len];
};
\end{lstlisting}

\field{hash_mode} decides the type used by \field{algo_flf}.

\field{algo_flf} is fixed to 16 bytes and MUST contains or be one of
the following types:
\begin{itemize*}
\item struct virtio_crypto_hash_create_session_flf
\item struct virtio_crypto_mac_create_session_flf
\end{itemize*}
The data of unused part (if has) in \field{algo_flf} will be ignored.

The length of \field{cipher_key} is specified in \field{key_len} in \field{cipher_hdr}.

The length of \field{auth_key} is specified in \field{auth_key_len} in struct
virtio_crypto_mac_create_session_flf.

The fixed-length parameters of Symmetric session requests are as follows:

\begin{lstlisting}
struct virtio_crypto_sym_create_session_flf {
    /* Device read only portion */

#define VIRTIO_CRYPTO_SYM_SESS_OP_SPEC_HDR_SIZE  48
    /* fixed length fields, opcode specific */
    u8 op_flf[VIRTIO_CRYPTO_SYM_SESS_OP_SPEC_HDR_SIZE];

/* No operation */
#define VIRTIO_CRYPTO_SYM_OP_NONE  0
/* Cipher only operation on the data */
#define VIRTIO_CRYPTO_SYM_OP_CIPHER  1
/* Chain any cipher with any hash or mac operation. The order
   depends on the value of alg_chain_order param */
#define VIRTIO_CRYPTO_SYM_OP_ALGORITHM_CHAINING  2
    le32 op_type;
    le32 padding;
};
\end{lstlisting}

\field{op_flf} is fixed to 48 bytes, MUST contains or be one of
the following types:
\begin{itemize*}
\item struct virtio_crypto_cipher_session_flf
\item struct virtio_crypto_alg_chain_session_flf
\end{itemize*}
The data of unused part (if has) in \field{op_flf} will be ignored.

\field{op_type} decides the type used by \field{op_flf}.

The variable-length parameters of Symmetric session requests are as follows:

\begin{lstlisting}
struct virtio_crypto_sym_create_session_vlf {
    /* Device read only portion */
    /* variable length fields, opcode specific */
    u8 op_vlf[vlf_len];
};
\end{lstlisting}

\field{op_vlf} MUST contains or be one of the following types:
\begin{itemize*}
\item struct virtio_crypto_cipher_session_vlf
\item struct virtio_crypto_alg_chain_session_vlf
\end{itemize*}

\field{op_type} in struct virtio_crypto_sym_create_session_flf decides the
type used by \field{op_vlf}.

\field{vlf_len} is the size of the specific structure used.


\subparagraph{Session operation: AEAD session}\label{sec:Device Types / Crypto Device / Device
Operation / Control Virtqueue / Session operation / Session operation: AEAD session}

The fixed-length and the variable-length parameters of AEAD session requests are as follows:

\begin{lstlisting}
struct virtio_crypto_aead_create_session_flf {
    /* Device read only portion */

    /* See VIRTIO_CRYPTO_AEAD_* above */
    le32 algo;
    /* length of key */
    le32 key_len;
    /* Authentication tag length */
    le32 tag_len;
    /* The length of the additional authenticated data (AAD) in bytes */
    le32 aad_len;
    /* encryption or decryption, See above VIRTIO_CRYPTO_OP_* */
    le32 op;
    le32 padding;
};

struct virtio_crypto_aead_create_session_vlf {
    /* Device read only portion */
    u8 key[key_len];
};
\end{lstlisting}

The length of \field{key} is specified in \field{key_len} in struct
virtio_crypto_aead_create_session_flf.

\subparagraph{Session operation: AKCIPHER session}\label{sec:Device Types / Crypto Device / Device
Operation / Control Virtqueue / Session operation / Session operation: AKCIPHER session}

Due to the complexity of asymmetric key algorithms, different algorithms
require different parameters. The following data structures are used as
supplementary parameters to describe the asymmetric algorithm sessions.

For the RSA algorithm, the extra parameters are as follows:
\begin{lstlisting}
struct virtio_crypto_rsa_session_para {
#define VIRTIO_CRYPTO_RSA_RAW_PADDING   0
#define VIRTIO_CRYPTO_RSA_PKCS1_PADDING 1
    le32 padding_algo;

#define VIRTIO_CRYPTO_RSA_NO_HASH   0
#define VIRTIO_CRYPTO_RSA_MD2       1
#define VIRTIO_CRYPTO_RSA_MD3       2
#define VIRTIO_CRYPTO_RSA_MD4       3
#define VIRTIO_CRYPTO_RSA_MD5       4
#define VIRTIO_CRYPTO_RSA_SHA1      5
#define VIRTIO_CRYPTO_RSA_SHA256    6
#define VIRTIO_CRYPTO_RSA_SHA384    7
#define VIRTIO_CRYPTO_RSA_SHA512    8
#define VIRTIO_CRYPTO_RSA_SHA224    9
    le32 hash_algo;
};
\end{lstlisting}

\field{padding_algo} specifies the padding method used by RSA sessions.
\begin{itemize*}
\item If VIRTIO_CRYPTO_RSA_RAW_PADDING is specified, 1) \field{hash_algo}
is ignored, 2) ciphertext and plaintext MUST be padded with leading zeros,
3) and RSA sessions with VIRTIO_CRYPTO_RSA_RAW_PADDING MUST not be used
for verification and signing operations.
\item If VIRTIO_CRYPTO_RSA_PKCS1_PADDING is specified, EMSA-PKCS1-v1_5 padding method
is used (see \hyperref[intro:rfc3447]{PKCS\#1}), \field{hash_algo} specifies how the
digest of the data passed to RSA sessions is calculated when verifying and signing.
It only affects the padding algorithm and is ignored during encryption and decryption.
\end{itemize*}

The ECC algorithms such as the ECDSA algorithm, cannot use custom curves, only the
following known curves can be used (see \hyperref[intro:NIST]{NIST-recommended curves}).

\begin{lstlisting}
#define VIRTIO_CRYPTO_CURVE_UNKNOWN   0
#define VIRTIO_CRYPTO_CURVE_NIST_P192 1
#define VIRTIO_CRYPTO_CURVE_NIST_P224 2
#define VIRTIO_CRYPTO_CURVE_NIST_P256 3
#define VIRTIO_CRYPTO_CURVE_NIST_P384 4
#define VIRTIO_CRYPTO_CURVE_NIST_P521 5
\end{lstlisting}

For the ECDSA algorithm, the extra parameters are as follows:
\begin{lstlisting}
struct virtio_crypto_ecdsa_session_para {
    /* See VIRTIO_CRYPTO_CURVE_* above */
    le32 curve_id;
};
\end{lstlisting}

The fixed-length and the variable-length parameters of AKCIPHER session requests are as follows:
\begin{lstlisting}
struct virtio_crypto_akcipher_create_session_flf {
    /* Device read only portion */

    /* See VIRTIO_CRYPTO_AKCIPHER_* above */
    le32 algo;
#define VIRTIO_CRYPTO_AKCIPHER_KEY_TYPE_PUBLIC 1
#define VIRTIO_CRYPTO_AKCIPHER_KEY_TYPE_PRIVATE 2
    le32 key_type;
    /* length of key */
    le32 key_len;

#define VIRTIO_CRYPTO_AKCIPHER_SESS_ALGO_SPEC_HDR_SIZE 44
    u8 algo_flf[VIRTIO_CRYPTO_AKCIPHER_SESS_ALGO_SPEC_HDR_SIZE];
};

struct virtio_crypto_akcipher_create_session_vlf {
    /* Device read only portion */
    u8 key[key_len];
};
\end{lstlisting}

\field{algo} decides the type used by \field{algo_flf}.
\field{algo_flf} is fixed to 44 bytes and MUST contains of be one the
following structures:
\begin{itemize*}
\item struct virtio_crypto_rsa_session_para
\item struct virtio_crypto_ecdsa_session_para
\end{itemize*}

The length of \field{key} is specified in \field{key_len} in the struct
virtio_crypto_akcipher_create_session_flf.

For the RSA algorithm, the key needs to be encoded according to
\hyperref[intro:rfc3447]{PKCS\#1}. The private key is described with the
RSAPrivateKey structure, and the public key is described with the RSAPublicKey
structure. These ASN.1 structures are encoded in DER encoding rules (see
\hyperref[intro:rfc6025]{rfc6025}).

\begin{lstlisting}
RSAPrivateKey ::= SEQUENCE {
    version          INTEGER,
    modulus          INTEGER,
    publicExponent   INTEGER,
    privateExponent  INTEGER,
    prime1           INTEGER,
    prime2           INTEGER,
    exponent1        INTEGER,
    exponent1        INTEGER,
    coefficient      INTEGER,
    otherPrimeInfos  OtherPrimeInfos OPTIONAL
}

OtherPrimeInfos ::= SEQUENCE SIZE(1...MAX) OF OtherPrimeInfo

OtherPrimeINfo ::= SEQUENCE {
    prime           INTEGER,
    exponent        INTEGER,
    coefficient     INTEGER
}

RSAPublicKey ::= SEQUENCE {
    modulus         INTEGER,
    publicExponent  INTEGER
}
\end{lstlisting}

For the ECDSA algorithm, the private key is encoded according to
\hyperref[intro:rfc5915]{RFC5915}, the private key of the ECDSA algorithm
is described by the ASN.1 structure ECPrivateKey and encoded with DER
encoding rules (see \hyperref[intro:rfc6025]{rfc6025}).

\begin{lstlisting}
ECPrivateKey ::= SEQUNCE {
    version         INTEGER,
    privateKey      OCTET STRING,
    parameters [0]  ECParameters {{ NamedCurve }} OPTIONAL,
    publicKey  [1]  BIT STRING OPTIONAL
}
\end{lstlisting}

The public key of the ECDSA algorithm is encoded according to \hyperref[intro:SEC1]{SEC1},
and the public key of ECDSA is described by the ASN.1 structure ECPoint.
When initializing a session with ECDSA public key, the ECPoint is DER encoded and the
\field{key} only contains the value part of ECPoint, that is, the header part of the
OCTET STRING will be omitted (see \hyperref[intro:rfc6025]{rfc6025}).

\begin{lstlisting}
ECPoint ::= OCTET STRING
\end{lstlisting}

The length of \field{key} is specified in \field{key_len} in
struct virtio_crypto_akcipher_create_session_flf.

\drivernormative{\subparagraph}{Session operation: create session}{Device Types / Crypto Device / Device
Operation / Control Virtqueue / Session operation / Session operation: create session}

\begin{itemize*}
\item The driver MUST set the \field{opcode} field based on service type: CIPHER, HASH, MAC, AEAD or AKCIPHER.
\item The driver MUST set the control general header, the opcode specific header,
    the opcode specific extra parameters and the opcode specific outcome buffer in turn.
    See \ref{sec:Device Types / Crypto Device / Device Operation / Control Virtqueue}.
\item The driver MUST set the \field{reversed} field to zero.
\end{itemize*}

\devicenormative{\subparagraph}{Session operation: create session}{Device Types / Crypto Device / Device
Operation / Control Virtqueue / Session operation / Session operation: create session}

\begin{itemize*}
\item The device MUST use the corresponding opcode specific structure according to the
    \field{opcode} in the control general header.
\item The device MUST extract extra parameters according to the structures used.
\item The device MUST set the \field{status} field to one of the following values of enum
    VIRTIO_CRYPTO_STATUS after finish a session creation:
\begin{itemize*}
\item VIRTIO_CRYPTO_OK if a session is created successfully.
\item VIRTIO_CRYPTO_NOTSUPP if the requested algorithm or operation is unsupported.
\item VIRTIO_CRYPTO_NOSPC if no free session ID (only when the VIRTIO_CRYPTO_F_REVISION_1
    feature bit is negotiated).
\item VIRTIO_CRYPTO_ERR if failure not mentioned above occurs.
\end{itemize*}
\item The device MUST set the \field{session_id} field to a unique session identifier only
    if the status is set to VIRTIO_CRYPTO_OK.
\end{itemize*}

\drivernormative{\subparagraph}{Session operation: destroy session}{Device Types / Crypto Device / Device
Operation / Control Virtqueue / Session operation / Session operation: destroy session}

\begin{itemize*}
\item The driver MUST set the \field{opcode} field based on service type: CIPHER, HASH, MAC, AEAD or AKCIPHER.
\item The driver MUST set the \field{session_id} to a valid value assigned by the device
    when the session was created.
\end{itemize*}

\devicenormative{\subparagraph}{Session operation: destroy session}{Device Types / Crypto Device / Device
Operation / Control Virtqueue / Session operation / Session operation: destroy session}

\begin{itemize*}
\item The device MUST set the \field{status} field to one of the following values of enum VIRTIO_CRYPTO_STATUS.
\begin{itemize*}
\item VIRTIO_CRYPTO_OK if a session is created successfully.
\item VIRTIO_CRYPTO_ERR if any failure occurs.
\end{itemize*}
\end{itemize*}


\subsubsection{Data Virtqueue}\label{sec:Device Types / Crypto Device / Device Operation / Data Virtqueue}

The driver uses the data virtqueues to transmit crypto operation requests to the device,
and completes the crypto operations.

The header for dataq is as follows:

\begin{lstlisting}
struct virtio_crypto_op_header {
#define VIRTIO_CRYPTO_CIPHER_ENCRYPT \
    VIRTIO_CRYPTO_OPCODE(VIRTIO_CRYPTO_SERVICE_CIPHER, 0x00)
#define VIRTIO_CRYPTO_CIPHER_DECRYPT \
    VIRTIO_CRYPTO_OPCODE(VIRTIO_CRYPTO_SERVICE_CIPHER, 0x01)
#define VIRTIO_CRYPTO_HASH \
    VIRTIO_CRYPTO_OPCODE(VIRTIO_CRYPTO_SERVICE_HASH, 0x00)
#define VIRTIO_CRYPTO_MAC \
    VIRTIO_CRYPTO_OPCODE(VIRTIO_CRYPTO_SERVICE_MAC, 0x00)
#define VIRTIO_CRYPTO_AEAD_ENCRYPT \
    VIRTIO_CRYPTO_OPCODE(VIRTIO_CRYPTO_SERVICE_AEAD, 0x00)
#define VIRTIO_CRYPTO_AEAD_DECRYPT \
    VIRTIO_CRYPTO_OPCODE(VIRTIO_CRYPTO_SERVICE_AEAD, 0x01)
#define VIRTIO_CRYPTO_AKCIPHER_ENCRYPT \
    VIRTIO_CRYPTO_OPCODE(VIRTIO_CRYPTO_SERVICE_AKCIPHER, 0x00)
#define VIRTIO_CRYPTO_AKCIPHER_DECRYPT \
    VIRTIO_CRYPTO_OPCODE(VIRTIO_CRYPTO_SERVICE_AKCIPHER, 0x01)
#define VIRTIO_CRYPTO_AKCIPHER_SIGN \
    VIRTIO_CRYPTO_OPCODE(VIRTIO_CRYPTO_SERVICE_AKCIPHER, 0x02)
#define VIRTIO_CRYPTO_AKCIPHER_VERIFY \
    VIRTIO_CRYPTO_OPCODE(VIRTIO_CRYPTO_SERVICE_AKCIPHER, 0x03)
    le32 opcode;
    /* algo should be service-specific algorithms */
    le32 algo;
    le64 session_id;
#define VIRTIO_CRYPTO_FLAG_SESSION_MODE 1
    /* control flag to control the request */
    le32 flag;
    le32 padding;
};
\end{lstlisting}

\begin{note}
If VIRTIO_CRYPTO_F_REVISION_1 is not negotiated the \field{flag} is ignored.

If VIRTIO_CRYPTO_F_REVISION_1 is negotiated but VIRTIO_CRYPTO_F_<SERVICE>_STATELESS_MODE
is not negotiated, then the device SHOULD reject <SERVICE> requests if
VIRTIO_CRYPTO_FLAG_SESSION_MODE is not set (in \field{flag}).
\end{note}

The dataq request is composed of four parts:
\begin{lstlisting}
struct virtio_crypto_op_data_req {
    /* Device read only portion */

    struct virtio_crypto_op_header header;

#define VIRTIO_CRYPTO_DATAQ_OP_SPEC_HDR_LEGACY 48
    /* fixed length fields, opcode specific */
    u8 op_flf[flf_len];

    /* Device read && write portion */
    /* variable length fields, opcode specific */
    u8 op_vlf[vlf_len];

    /* Device write only portion */
    struct virtio_crypto_inhdr inhdr;
};
\end{lstlisting}

\field{header} is a general header (see above).

\field{op_flf} is the opcode (in \field{header}) specific header.

\field{flf_len} depends on the VIRTIO_CRYPTO_F_REVISION_1 feature bit
(see below).

\field{op_vlf} is the opcode (in \field{header}) specific parameters.

\field{vlf_len} is the size of the specific structure used.

\begin{itemize*}
\item If the the opcode (in \field{header}) is VIRTIO_CRYPTO_CIPHER_ENCRYPT
    or VIRTIO_CRYPTO_CIPHER_DECRYPT then:
    \begin{itemize*}
    \item If VIRTIO_CRYPTO_F_CIPHER_STATELESS_MODE is negotiated, \field{op_flf} is
        struct virtio_crypto_sym_data_flf_stateless, and \field{op_vlf} is struct
        virtio_crypto_sym_data_vlf_stateless.
    \item If VIRTIO_CRYPTO_F_CIPHER_STATELESS_MODE is NOT negotiated, \field{op_flf}
        is struct virtio_crypto_sym_data_flf if VIRTIO_CRYPTO_F_REVISION_1 is negotiated
        and struct virtio_crypto_sym_data_flf is padded to 48 bytes if NOT negotiated,
        and \field{op_vlf} is struct virtio_crypto_sym_data_vlf.
    \end{itemize*}
\item If the the opcode (in \field{header}) is VIRTIO_CRYPTO_HASH:
    \begin{itemize*}
    \item If VIRTIO_CRYPTO_F_HASH_STATELESS_MODE is negotiated, \field{op_flf} is
        struct virtio_crypto_hash_data_flf_stateless, and \field{op_vlf} is struct
        virtio_crypto_hash_data_vlf_stateless.
    \item If VIRTIO_CRYPTO_F_HASH_STATELESS_MODE is NOT negotiated, \field{op_flf}
        is struct virtio_crypto_hash_data_flf if VIRTIO_CRYPTO_F_REVISION_1 is negotiated
        and struct virtio_crypto_hash_data_flf is padded to 48 bytes if NOT negotiated,
        and \field{op_vlf} is struct virtio_crypto_hash_data_vlf.
    \end{itemize*}
\item If the the opcode (in \field{header}) is VIRTIO_CRYPTO_MAC:
    \begin{itemize*}
    \item If VIRTIO_CRYPTO_F_MAC_STATELESS_MODE is negotiated, \field{op_flf} is
        struct virtio_crypto_mac_data_flf_stateless, and \field{op_vlf} is struct
        virtio_crypto_mac_data_vlf_stateless.
    \item If VIRTIO_CRYPTO_F_MAC_STATELESS_MODE is NOT negotiated, \field{op_flf}
        is struct virtio_crypto_mac_data_flf if VIRTIO_CRYPTO_F_REVISION_1 is negotiated
        and struct virtio_crypto_mac_data_flf is padded to 48 bytes if NOT negotiated,
        and \field{op_vlf} is struct virtio_crypto_mac_data_vlf.
    \end{itemize*}
\item If the the opcode (in \field{header}) is VIRTIO_CRYPTO_AEAD_ENCRYPT
    or VIRTIO_CRYPTO_AEAD_DECRYPT then:
    \begin{itemize*}
    \item If VIRTIO_CRYPTO_F_AEAD_STATELESS_MODE is negotiated, \field{op_flf} is
        struct virtio_crypto_aead_data_flf_stateless, and \field{op_vlf} is struct
        virtio_crypto_aead_data_vlf_stateless.
    \item If VIRTIO_CRYPTO_F_AEAD_STATELESS_MODE is NOT negotiated, \field{op_flf}
        is struct virtio_crypto_aead_data_flf if VIRTIO_CRYPTO_F_REVISION_1 is negotiated
        and struct virtio_crypto_aead_data_flf is padded to 48 bytes if NOT negotiated,
        and \field{op_vlf} is struct virtio_crypto_aead_data_vlf.
    \end{itemize*}
\item If the opcode (in \field{header}) is VIRTIO_CRYPTO_AKCIPHER_ENCRYPT, VIRTIO_CRYPTO_AKCIPHER_DECRYPT,
    VIRTIO_CRYPTO_AKCIPHER_SIGN or VIRTIO_CRYPTO_AKCIPHER_VERIFY then:
    \begin{itemize*}
    \item If VIRTIO_CRYPTO_F_AKCIPHER_STATELESS_MODE is negotiated, \field{op_flf} is
        struct virtio_crypto_akcipher_data_flf_statless, and \field{op_vlf} is struct
        virtio_crypto_akcipher_data_vlf_stateless.
    \item If VIRTIO_CRYPTO_F_AKCIPHER_STATELESS_MODE is NOT negotiated, \field{op_flf}
        is struct virtio_crypto_akcipher_data_flf if VIRTIO_CRYPTO_F_REVISION_1 is negotiated
        and struct virtio_crypto_akcipher_data_flf is padded to 48 bytes if NOT negotiated,
        and \field{op_vlf} is struct virtio_crypto_akcipher_data_vlf.
    \end{itemize*}
\end{itemize*}

\field{inhdr} is a unified input header that used to return the status of
the operations, is defined as follows:

\begin{lstlisting}
struct virtio_crypto_inhdr {
    u8 status;
};
\end{lstlisting}

\subsubsection{HASH Service Operation}\label{sec:Device Types / Crypto Device / Device Operation / HASH Service Operation}

Session mode HASH service requests are as follows:

\begin{lstlisting}
struct virtio_crypto_hash_data_flf {
    /* length of source data */
    le32 src_data_len;
    /* hash result length */
    le32 hash_result_len;
};

struct virtio_crypto_hash_data_vlf {
    /* Device read only portion */
    /* Source data */
    u8 src_data[src_data_len];

    /* Device write only portion */
    /* Hash result data */
    u8 hash_result[hash_result_len];
};
\end{lstlisting}

Each data request uses the virtio_crypto_hash_data_flf structure and the
virtio_crypto_hash_data_vlf structure to store information used to run the
HASH operations.

\field{src_data} is the source data that will be processed.
\field{src_data_len} is the length of source data.
\field{hash_result} is the result data and \field{hash_result_len} is the length
of it.

Stateless mode HASH service requests are as follows:

\begin{lstlisting}
struct virtio_crypto_hash_data_flf_stateless {
    struct {
        /* See VIRTIO_CRYPTO_HASH_* above */
        le32 algo;
    } sess_para;

    /* length of source data */
    le32 src_data_len;
    /* hash result length */
    le32 hash_result_len;
    le32 reserved;
};
struct virtio_crypto_hash_data_vlf_stateless {
    /* Device read only portion */
    /* Source data */
    u8 src_data[src_data_len];

    /* Device write only portion */
    /* Hash result data */
    u8 hash_result[hash_result_len];
};
\end{lstlisting}

\drivernormative{\paragraph}{HASH Service Operation}{Device Types / Crypto Device / Device Operation / HASH Service Operation}

\begin{itemize*}
\item If the driver uses the session mode, then the driver MUST set \field{session_id}
    in struct virtio_crypto_op_header to a valid value assigned by the device when the
    session was created.
\item If the VIRTIO_CRYPTO_F_HASH_STATELESS_MODE feature bit is negotiated, 1) if the
    driver uses the stateless mode, then the driver MUST set the \field{flag} field in
    struct virtio_crypto_op_header to ZERO and MUST set the fields in struct
    virtio_crypto_hash_data_flf_stateless.sess_para, 2) if the driver uses the session
    mode, then the driver MUST set the \field{flag} field in struct virtio_crypto_op_header
    to VIRTIO_CRYPTO_FLAG_SESSION_MODE.
\item The driver MUST set \field{opcode} in struct virtio_crypto_op_header to VIRTIO_CRYPTO_HASH.
\end{itemize*}

\devicenormative{\paragraph}{HASH Service Operation}{Device Types / Crypto Device / Device Operation / HASH Service Operation}

\begin{itemize*}
\item The device MUST use the corresponding structure according to the \field{opcode}
    in the data general header.
\item If the VIRTIO_CRYPTO_F_HASH_STATELESS_MODE feature bit is negotiated, the device
    MUST parse \field{flag} field in struct virtio_crypto_op_header in order to decide
    which mode the driver uses.
\item The device MUST copy the results of HASH operations in the hash_result[] if HASH
    operations success.
\item The device MUST set \field{status} in struct virtio_crypto_inhdr to one of the
    following values of enum VIRTIO_CRYPTO_STATUS:
\begin{itemize*}
\item VIRTIO_CRYPTO_OK if the operation success.
\item VIRTIO_CRYPTO_NOTSUPP if the requested algorithm or operation is unsupported.
\item VIRTIO_CRYPTO_INVSESS if the session ID invalid when in session mode.
\item VIRTIO_CRYPTO_ERR if any failure not mentioned above occurs.
\end{itemize*}
\end{itemize*}


\subsubsection{MAC Service Operation}\label{sec:Device Types / Crypto Device / Device Operation / MAC Service Operation}

Session mode MAC service requests are as follows:

\begin{lstlisting}
struct virtio_crypto_mac_data_flf {
    struct virtio_crypto_hash_data_flf hdr;
};

struct virtio_crypto_mac_data_vlf {
    /* Device read only portion */
    /* Source data */
    u8 src_data[src_data_len];

    /* Device write only portion */
    /* Hash result data */
    u8 hash_result[hash_result_len];
};
\end{lstlisting}

Each request uses the virtio_crypto_mac_data_flf structure and the
virtio_crypto_mac_data_vlf structure to store information used to run the
MAC operations.

\field{src_data} is the source data that will be processed.
\field{src_data_len} is the length of source data.
\field{hash_result} is the result data and \field{hash_result_len} is the length
of it.

Stateless mode MAC service requests are as follows:

\begin{lstlisting}
struct virtio_crypto_mac_data_flf_stateless {
    struct {
        /* See VIRTIO_CRYPTO_MAC_* above */
        le32 algo;
        /* length of authenticated key */
        le32 auth_key_len;
    } sess_para;

    /* length of source data */
    le32 src_data_len;
    /* hash result length */
    le32 hash_result_len;
};

struct virtio_crypto_mac_data_vlf_stateless {
    /* Device read only portion */
    /* The authenticated key */
    u8 auth_key[auth_key_len];
    /* Source data */
    u8 src_data[src_data_len];

    /* Device write only portion */
    /* Hash result data */
    u8 hash_result[hash_result_len];
};
\end{lstlisting}

\field{auth_key} is the authenticated key that will be used during the process.
\field{auth_key_len} is the length of the key.

\drivernormative{\paragraph}{MAC Service Operation}{Device Types / Crypto Device / Device Operation / MAC Service Operation}

\begin{itemize*}
\item If the driver uses the session mode, then the driver MUST set \field{session_id}
    in struct virtio_crypto_op_header to a valid value assigned by the device when the
    session was created.
\item If the VIRTIO_CRYPTO_F_MAC_STATELESS_MODE feature bit is negotiated, 1) if the
    driver uses the stateless mode, then the driver MUST set the \field{flag} field
    in struct virtio_crypto_op_header to ZERO and MUST set the fields in struct
    virtio_crypto_mac_data_flf_stateless.sess_para, 2) if the driver uses the session
    mode, then the driver MUST set the \field{flag} field in struct virtio_crypto_op_header
    to VIRTIO_CRYPTO_FLAG_SESSION_MODE.
\item The driver MUST set \field{opcode} in struct virtio_crypto_op_header to VIRTIO_CRYPTO_MAC.
\end{itemize*}

\devicenormative{\paragraph}{MAC Service Operation}{Device Types / Crypto Device / Device Operation / MAC Service Operation}

\begin{itemize*}
\item If the VIRTIO_CRYPTO_F_MAC_STATELESS_MODE feature bit is negotiated, the device
    MUST parse \field{flag} field in struct virtio_crypto_op_header in order to decide
	which mode the driver uses.
\item The device MUST copy the results of MAC operations in the hash_result[] if HASH
    operations success.
\item The device MUST set \field{status} in struct virtio_crypto_inhdr to one of the
    following values of enum VIRTIO_CRYPTO_STATUS:
\begin{itemize*}
\item VIRTIO_CRYPTO_OK if the operation success.
\item VIRTIO_CRYPTO_NOTSUPP if the requested algorithm or operation is unsupported.
\item VIRTIO_CRYPTO_INVSESS if the session ID invalid when in session mode.
\item VIRTIO_CRYPTO_ERR if any failure not mentioned above occurs.
\end{itemize*}
\end{itemize*}

\subsubsection{Symmetric algorithms Operation}\label{sec:Device Types / Crypto Device / Device Operation / Symmetric algorithms Operation}

Session mode CIPHER service requests are as follows:

\begin{lstlisting}
struct virtio_crypto_cipher_data_flf {
    /*
     * Byte Length of valid IV/Counter data pointed to by the below iv data.
     *
     * For block ciphers in CBC or F8 mode, or for Kasumi in F8 mode, or for
     *   SNOW3G in UEA2 mode, this is the length of the IV (which
     *   must be the same as the block length of the cipher).
     * For block ciphers in CTR mode, this is the length of the counter
     *   (which must be the same as the block length of the cipher).
     */
    le32 iv_len;
    /* length of source data */
    le32 src_data_len;
    /* length of destination data */
    le32 dst_data_len;
    le32 padding;
};

struct virtio_crypto_cipher_data_vlf {
    /* Device read only portion */

    /*
     * Initialization Vector or Counter data.
     *
     * For block ciphers in CBC or F8 mode, or for Kasumi in F8 mode, or for
     *   SNOW3G in UEA2 mode, this is the Initialization Vector (IV)
     *   value.
     * For block ciphers in CTR mode, this is the counter.
     * For AES-XTS, this is the 128bit tweak, i, from IEEE Std 1619-2007.
     *
     * The IV/Counter will be updated after every partial cryptographic
     * operation.
     */
    u8 iv[iv_len];
    /* Source data */
    u8 src_data[src_data_len];

    /* Device write only portion */
    /* Destination data */
    u8 dst_data[dst_data_len];
};
\end{lstlisting}

Session mode requests of algorithm chaining are as follows:

\begin{lstlisting}
struct virtio_crypto_alg_chain_data_flf {
    le32 iv_len;
    /* Length of source data */
    le32 src_data_len;
    /* Length of destination data */
    le32 dst_data_len;
    /* Starting point for cipher processing in source data */
    le32 cipher_start_src_offset;
    /* Length of the source data that the cipher will be computed on */
    le32 len_to_cipher;
    /* Starting point for hash processing in source data */
    le32 hash_start_src_offset;
    /* Length of the source data that the hash will be computed on */
    le32 len_to_hash;
    /* Length of the additional auth data */
    le32 aad_len;
    /* Length of the hash result */
    le32 hash_result_len;
    le32 reserved;
};

struct virtio_crypto_alg_chain_data_vlf {
    /* Device read only portion */

    /* Initialization Vector or Counter data */
    u8 iv[iv_len];
    /* Source data */
    u8 src_data[src_data_len];
    /* Additional authenticated data if exists */
    u8 aad[aad_len];

    /* Device write only portion */

    /* Destination data */
    u8 dst_data[dst_data_len];
    /* Hash result data */
    u8 hash_result[hash_result_len];
};
\end{lstlisting}

Session mode requests of symmetric algorithm are as follows:

\begin{lstlisting}
struct virtio_crypto_sym_data_flf {
    /* Device read only portion */

#define VIRTIO_CRYPTO_SYM_DATA_REQ_HDR_SIZE    40
    u8 op_type_flf[VIRTIO_CRYPTO_SYM_DATA_REQ_HDR_SIZE];

    /* See above VIRTIO_CRYPTO_SYM_OP_* */
    le32 op_type;
    le32 padding;
};

struct virtio_crypto_sym_data_vlf {
    u8 op_type_vlf[sym_para_len];
};
\end{lstlisting}

Each request uses the virtio_crypto_sym_data_flf structure and the
virtio_crypto_sym_data_flf structure to store information used to run the
CIPHER operations.

\field{op_type_flf} is the \field{op_type} specific header, it MUST starts
with or be one of the following structures:
\begin{itemize*}
\item struct virtio_crypto_cipher_data_flf
\item struct virtio_crypto_alg_chain_data_flf
\end{itemize*}

The length of \field{op_type_flf} is fixed to 40 bytes, the data of unused
part (if has) will be ignored.

\field{op_type_vlf} is the \field{op_type} specific parameters, it MUST starts
with or be one of the following structures:
\begin{itemize*}
\item struct virtio_crypto_cipher_data_vlf
\item struct virtio_crypto_alg_chain_data_vlf
\end{itemize*}

\field{sym_para_len} is the size of the specific structure used.

Stateless mode CIPHER service requests are as follows:

\begin{lstlisting}
struct virtio_crypto_cipher_data_flf_stateless {
    struct {
        /* See VIRTIO_CRYPTO_CIPHER* above */
        le32 algo;
        /* length of key */
        le32 key_len;

        /* See VIRTIO_CRYPTO_OP_* above */
        le32 op;
    } sess_para;

    /*
     * Byte Length of valid IV/Counter data pointed to by the below iv data.
     */
    le32 iv_len;
    /* length of source data */
    le32 src_data_len;
    /* length of destination data */
    le32 dst_data_len;
};

struct virtio_crypto_cipher_data_vlf_stateless {
    /* Device read only portion */

    /* The cipher key */
    u8 cipher_key[key_len];

    /* Initialization Vector or Counter data. */
    u8 iv[iv_len];
    /* Source data */
    u8 src_data[src_data_len];

    /* Device write only portion */
    /* Destination data */
    u8 dst_data[dst_data_len];
};
\end{lstlisting}

Stateless mode requests of algorithm chaining are as follows:

\begin{lstlisting}
struct virtio_crypto_alg_chain_data_flf_stateless {
    struct {
        /* See VIRTIO_CRYPTO_SYM_ALG_CHAIN_ORDER_* above */
        le32 alg_chain_order;
        /* length of the additional authenticated data in bytes */
        le32 aad_len;

        struct {
            /* See VIRTIO_CRYPTO_CIPHER* above */
            le32 algo;
            /* length of key */
            le32 key_len;
            /* See VIRTIO_CRYPTO_OP_* above */
            le32 op;
        } cipher;

        struct {
            /* See VIRTIO_CRYPTO_HASH_* or VIRTIO_CRYPTO_MAC_* above */
            le32 algo;
            /* length of authenticated key */
            le32 auth_key_len;
            /* See VIRTIO_CRYPTO_SYM_HASH_MODE_* above */
            le32 hash_mode;
        } hash;
    } sess_para;

    le32 iv_len;
    /* Length of source data */
    le32 src_data_len;
    /* Length of destination data */
    le32 dst_data_len;
    /* Starting point for cipher processing in source data */
    le32 cipher_start_src_offset;
    /* Length of the source data that the cipher will be computed on */
    le32 len_to_cipher;
    /* Starting point for hash processing in source data */
    le32 hash_start_src_offset;
    /* Length of the source data that the hash will be computed on */
    le32 len_to_hash;
    /* Length of the additional auth data */
    le32 aad_len;
    /* Length of the hash result */
    le32 hash_result_len;
    le32 reserved;
};

struct virtio_crypto_alg_chain_data_vlf_stateless {
    /* Device read only portion */

    /* The cipher key */
    u8 cipher_key[key_len];
    /* The auth key */
    u8 auth_key[auth_key_len];
    /* Initialization Vector or Counter data */
    u8 iv[iv_len];
    /* Additional authenticated data if exists */
    u8 aad[aad_len];
    /* Source data */
    u8 src_data[src_data_len];

    /* Device write only portion */

    /* Destination data */
    u8 dst_data[dst_data_len];
    /* Hash result data */
    u8 hash_result[hash_result_len];
};
\end{lstlisting}

Stateless mode requests of symmetric algorithm are as follows:

\begin{lstlisting}
struct virtio_crypto_sym_data_flf_stateless {
    /* Device read only portion */
#define VIRTIO_CRYPTO_SYM_DATE_REQ_HDR_STATELESS_SIZE    72
    u8 op_type_flf[VIRTIO_CRYPTO_SYM_DATE_REQ_HDR_STATELESS_SIZE];

    /* Device write only portion */
    /* See above VIRTIO_CRYPTO_SYM_OP_* */
    le32 op_type;
};

struct virtio_crypto_sym_data_vlf_stateless {
    u8 op_type_vlf[sym_para_len];
};
\end{lstlisting}

\field{op_type_flf} is the \field{op_type} specific header, it MUST starts
with or be one of the following structures:
\begin{itemize*}
\item struct virtio_crypto_cipher_data_flf_stateless
\item struct virtio_crypto_alg_chain_data_flf_stateless
\end{itemize*}

The length of \field{op_type_flf} is fixed to 72 bytes, the data of unused
part (if has) will be ignored.

\field{op_type_vlf} is the \field{op_type} specific parameters, it MUST starts
with or be one of the following structures:
\begin{itemize*}
\item struct virtio_crypto_cipher_data_vlf_stateless
\item struct virtio_crypto_alg_chain_data_vlf_stateless
\end{itemize*}

\field{sym_para_len} is the size of the specific structure used.

\drivernormative{\paragraph}{Symmetric algorithms Operation}{Device Types / Crypto Device / Device Operation / Symmetric algorithms Operation}

\begin{itemize*}
\item If the driver uses the session mode, then the driver MUST set \field{session_id}
    in struct virtio_crypto_op_header to a valid value assigned by the device when the
    session was created.
\item If the VIRTIO_CRYPTO_F_CIPHER_STATELESS_MODE feature bit is negotiated, 1) if the
    driver uses the stateless mode, then the driver MUST set the \field{flag} field in
    struct virtio_crypto_op_header to ZERO and MUST set the fields in struct
    virtio_crypto_cipher_data_flf_stateless.sess_para or struct
    virtio_crypto_alg_chain_data_flf_stateless.sess_para, 2) if the driver uses the
    session mode, then the driver MUST set the \field{flag} field in struct
    virtio_crypto_op_header to VIRTIO_CRYPTO_FLAG_SESSION_MODE.
\item The driver MUST set the \field{opcode} field in struct virtio_crypto_op_header
    to VIRTIO_CRYPTO_CIPHER_ENCRYPT or VIRTIO_CRYPTO_CIPHER_DECRYPT.
\item The driver MUST specify the fields of struct virtio_crypto_cipher_data_flf in
    struct virtio_crypto_sym_data_flf and struct virtio_crypto_cipher_data_vlf in
    struct virtio_crypto_sym_data_vlf if the request is based on VIRTIO_CRYPTO_SYM_OP_CIPHER.
\item The driver MUST specify the fields of struct virtio_crypto_alg_chain_data_flf
    in struct virtio_crypto_sym_data_flf and struct virtio_crypto_alg_chain_data_vlf
    in struct virtio_crypto_sym_data_vlf if the request is of the VIRTIO_CRYPTO_SYM_OP_ALGORITHM_CHAINING
    type.
\end{itemize*}

\devicenormative{\paragraph}{Symmetric algorithms Operation}{Device Types / Crypto Device / Device Operation / Symmetric algorithms Operation}

\begin{itemize*}
\item If the VIRTIO_CRYPTO_F_CIPHER_STATELESS_MODE feature bit is negotiated, the device
    MUST parse \field{flag} field in struct virtio_crypto_op_header in order to decide
	which mode the driver uses.
\item The device MUST parse the virtio_crypto_sym_data_req based on the \field{opcode}
    field in general header.
\item The device MUST parse the fields of struct virtio_crypto_cipher_data_flf in
    struct virtio_crypto_sym_data_flf and struct virtio_crypto_cipher_data_vlf in
    struct virtio_crypto_sym_data_vlf if the request is based on VIRTIO_CRYPTO_SYM_OP_CIPHER.
\item The device MUST parse the fields of struct virtio_crypto_alg_chain_data_flf
    in struct virtio_crypto_sym_data_flf and struct virtio_crypto_alg_chain_data_vlf
    in struct virtio_crypto_sym_data_vlf if the request is of the VIRTIO_CRYPTO_SYM_OP_ALGORITHM_CHAINING
    type.
\item The device MUST copy the result of cryptographic operation in the dst_data[] in
    both plain CIPHER mode and algorithms chain mode.
\item The device MUST check the \field{para}.\field{add_len} is bigger than 0 before
    parse the additional authenticated data in plain algorithms chain mode.
\item The device MUST copy the result of HASH/MAC operation in the hash_result[] is
    of the VIRTIO_CRYPTO_SYM_OP_ALGORITHM_CHAINING type.
\item The device MUST set the \field{status} field in struct virtio_crypto_inhdr to
    one of the following values of enum VIRTIO_CRYPTO_STATUS:
\begin{itemize*}
\item VIRTIO_CRYPTO_OK if the operation success.
\item VIRTIO_CRYPTO_NOTSUPP if the requested algorithm or operation is unsupported.
\item VIRTIO_CRYPTO_INVSESS if the session ID is invalid in session mode.
\item VIRTIO_CRYPTO_ERR if failure not mentioned above occurs.
\end{itemize*}
\end{itemize*}

\subsubsection{AEAD Service Operation}\label{sec:Device Types / Crypto Device / Device Operation / AEAD Service Operation}

Session mode requests of symmetric algorithm are as follows:

\begin{lstlisting}
struct virtio_crypto_aead_data_flf {
    /*
     * Byte Length of valid IV data.
     *
     * For GCM mode, this is either 12 (for 96-bit IVs) or 16, in which
     *   case iv points to J0.
     * For CCM mode, this is the length of the nonce, which can be in the
     *   range 7 to 13 inclusive.
     */
    le32 iv_len;
    /* length of additional auth data */
    le32 aad_len;
    /* length of source data */
    le32 src_data_len;
    /* length of dst data, this should be at least src_data_len + tag_len */
    le32 dst_data_len;
    /* Authentication tag length */
    le32 tag_len;
    le32 reserved;
};

struct virtio_crypto_aead_data_vlf {
    /* Device read only portion */

    /*
     * Initialization Vector data.
     *
     * For GCM mode, this is either the IV (if the length is 96 bits) or J0
     *   (for other sizes), where J0 is as defined by NIST SP800-38D.
     *   Regardless of the IV length, a full 16 bytes needs to be allocated.
     * For CCM mode, the first byte is reserved, and the nonce should be
     *   written starting at &iv[1] (to allow space for the implementation
     *   to write in the flags in the first byte).  Note that a full 16 bytes
     *   should be allocated, even though the iv_len field will have
     *   a value less than this.
     *
     * The IV will be updated after every partial cryptographic operation.
     */
    u8 iv[iv_len];
    /* Source data */
    u8 src_data[src_data_len];
    /* Additional authenticated data if exists */
    u8 aad[aad_len];

    /* Device write only portion */
    /* Pointer to output data */
    u8 dst_data[dst_data_len];
};
\end{lstlisting}

Each request uses the virtio_crypto_aead_data_flf structure and the
virtio_crypto_aead_data_flf structure to store information used to run the
AEAD operations.

Stateless mode AEAD service requests are as follows:

\begin{lstlisting}
struct virtio_crypto_aead_data_flf_stateless {
    struct {
        /* See VIRTIO_CRYPTO_AEAD_* above */
        le32 algo;
        /* length of key */
        le32 key_len;
        /* encrypt or decrypt, See above VIRTIO_CRYPTO_OP_* */
        le32 op;
    } sess_para;

    /* Byte Length of valid IV data. */
    le32 iv_len;
    /* Authentication tag length */
    le32 tag_len;
    /* length of additional auth data */
    le32 aad_len;
    /* length of source data */
    le32 src_data_len;
    /* length of dst data, this should be at least src_data_len + tag_len */
    le32 dst_data_len;
};

struct virtio_crypto_aead_data_vlf_stateless {
    /* Device read only portion */

    /* The cipher key */
    u8 key[key_len];
    /* Initialization Vector data. */
    u8 iv[iv_len];
    /* Source data */
    u8 src_data[src_data_len];
    /* Additional authenticated data if exists */
    u8 aad[aad_len];

    /* Device write only portion */
    /* Pointer to output data */
    u8 dst_data[dst_data_len];
};
\end{lstlisting}

\drivernormative{\paragraph}{AEAD Service Operation}{Device Types / Crypto Device / Device Operation / AEAD Service Operation}

\begin{itemize*}
\item If the driver uses the session mode, then the driver MUST set
    \field{session_id} in struct virtio_crypto_op_header to a valid value assigned
    by the device when the session was created.
\item If the VIRTIO_CRYPTO_F_AEAD_STATELESS_MODE feature bit is negotiated, 1) if
    the driver uses the stateless mode, then the driver MUST set the \field{flag}
    field in struct virtio_crypto_op_header to ZERO and MUST set the fields in
    struct virtio_crypto_aead_data_flf_stateless.sess_para, 2) if the driver uses
    the session mode, then the driver MUST set the \field{flag} field in struct
    virtio_crypto_op_header to VIRTIO_CRYPTO_FLAG_SESSION_MODE.
\item The driver MUST set the \field{opcode} field in struct virtio_crypto_op_header
    to VIRTIO_CRYPTO_AEAD_ENCRYPT or VIRTIO_CRYPTO_AEAD_DECRYPT.
\end{itemize*}

\devicenormative{\paragraph}{AEAD Service Operation}{Device Types / Crypto Device / Device Operation / AEAD Service Operation}

\begin{itemize*}
\item If the VIRTIO_CRYPTO_F_AEAD_STATELESS_MODE feature bit is negotiated, the
    device MUST parse the virtio_crypto_aead_data_vlf_stateless based on the \field{opcode}
	field in general header.
\item The device MUST copy the result of cryptographic operation in the dst_data[].
\item The device MUST copy the authentication tag in the dst_data[] offset the cipher result.
\item The device MUST set the \field{status} field in struct virtio_crypto_inhdr to
    one of the following values of enum VIRTIO_CRYPTO_STATUS:
\item When the \field{opcode} field is VIRTIO_CRYPTO_AEAD_DECRYPT, the device MUST
    verify and return the verification result to the driver.
\begin{itemize*}
\item VIRTIO_CRYPTO_OK if the operation success.
\item VIRTIO_CRYPTO_NOTSUPP if the requested algorithm or operation is unsupported.
\item VIRTIO_CRYPTO_BADMSG if the verification result is incorrect.
\item VIRTIO_CRYPTO_INVSESS if the session ID invalid when in session mode.
\item VIRTIO_CRYPTO_ERR if any failure not mentioned above occurs.
\end{itemize*}
\end{itemize*}

\subsubsection{AKCIPHER Service Operation}\label{sec:Device Types / Crypto Device / Device Operation / AKCIPHER Service Operation}

Session mode AKCIPHER requests are as follows:

\begin{lstlisting}
struct virtio_crypto_akcipher_data_flf {
    /* length of source data */
    le32 src_data_len;
    /* length of dst data */
    le32 dst_data_len;
};

struct virtio_crypto_akcipher_data_vlf {
    /* Device read only portion */
    /* Source data */
    u8 src_data[src_data_len];

    /* Device write only portion */
    /* Pointer to output data */
    u8 dst_data[dst_data_len];
};
\end{lstlisting}

Each data request uses the virtio_crypto_akcipher_flf structure and the virtio_crypto_akcipher_data_vlf
structure to store information used to run the AKCIPHER operations.

For encryption, decryption, and signing:
\field{src_data} is the source data that will be processed, note that for signing operations,
src_data stores the data to be signed, which usually is the digest of some data rather than the
data itself.
\field{src_data_len} is the length of source data.
\field{dst_result} is the result data and \field{dst_data_len} is the length of it. Note that the
length of the result is not always exactly equal to dst_data_len, the driver needs to check how
many bytes the device has written and calculate the actual length of the result.

For verification:
\field{src_data_len} refers to the length of the signature, and \field{dst_data_len} refers to
the length of signed data, where the signed data is usually the digest of some data.
\field{src_data} is spliced by the signature and the signed data, the src_data with the lower
address stores the signature, and the higher address stores the signed data.
\field{dst_data} is always empty for verification.

Different algorithms have different signature formats.
For the RSA algorithm, the result is determined by the padding algorithm specified by
\field{padding_algo} in structure virtio_crypto_rsa_session_para.

For the ECDSA algorithm, the signature is composed of the following
ASN.1 structure (see \hyperref[intro:rfc3279]{RFC3279})
and MUST be DER encoded (see \hyperref[intro:rfc6025]{rfc6025}).

\begin{lstlisting}
Ecdsa-Sig-Value ::= SEQUENCE {
    r INTEGER,
    s INTEGER
}
\end{lstlisting}

Stateless mode AKCIPHER service requests are as follows:
\begin{lstlisting}
struct virtio_crypto_akcipher_data_flf_stateless {
    struct {
        /* See VIRTIO_CYRPTO_AKCIPHER* above */
        le32 algo;
        /* See VIRTIO_CRYPTO_AKCIPHER_KEY_TYPE_* above */
        le32 key_type;
        /* length of key */
        le32 key_len;

        /* algothrim specific parameters described above */
        union {
            struct virtio_crypto_rsa_session_para rsa;
            struct virtio_crypto_ecdsa_session_para ecdsa;
        } u;
    } sess_para;

    /* length of source data */
    le32 src_data_len;
    /* length of destination data */
    le32 dst_data_len;
};

struct virtio_crypto_akcipher_data_vlf_stateless {
    /* Device read only portion */
    u8 akcipher_key[key_len];

    /* Source data */
    u8 src_data[src_data_len];

    /* Device write only portion */
    u8 dst_data[dst_data_len];
};
\end{lstlisting}

In stateless mode, the format of key and signature, the meaning of src_data and dst_data, are all the same
with session mode.

\drivernormative{\paragraph}{AKCIPHER Service Operation}{Device Types / Crypto Device / Device Operation / AKCIPHER Service Operation}

\begin{itemize*}
\item If the driver uses the session mode, then the driver MUST set
    \field{session_id} in struct virtio_crypto_op_header to a valid
    value assigned by the device when the session was created.
\item If the VIRTIO_CRYPTO_F_AKCIPHER_STATELESS_MODE feature bit is negotiated, 1) if the
    driver uses the stateless mode, then the driver MUST set the \field{flag} field in
    struct virtio_crypto_op_header to ZERO and MUST set the fields in struct
    virtio_crypto_akcipher_flf_stateless.sess_para, 2) if the driver uses the session
    mode, then the driver MUST set the \field{flag} field in struct virtio_crypto_op_header
    to VIRTIO_CRYPTO_FLAG_SESSION_MODE.
\item The driver MUST set the \field{opcode} field in struct virtio_crypto_op_header
    to one of VIRTIO_CRYPTO_AKCIPHER_ENCRYPT, VIRTIO_CRYPTO_AKCIPHER_DESTROY_SESSION,
    VIRTIO_CRYPTO_AKCIPHER_SIGN, and VIRTIO_CRYPTO_AKCIPHER_VERIFY.
\end{itemize*}

\devicenormative{\paragraph}{AKCIPHER Service Operation}{Device Types / Crypto Device / Device Operation / AKCIPHER Service Operation}

\begin{itemize*}
\item If the VIRTIO_CRYPTO_F_AKCIPHER_STATELESS_MODE feature bit is negotiated, the
    device MUST parse the virtio_crypto_akcipher_data_vlf_stateless based on the \field{opcode}
    field in general header.
\item The device MUST copy the result of cryptographic operation in the dst_data[].
\item The device MUST set the \field{status} field in struct virtio_crypto_inhdr to
    one of the following values of enum VIRTIO_CRYPTO_STATUS:
\begin{itemize*}
\item VIRTIO_CRYPTO_OK if the operation success.
\item VIRTIO_CRYPTO_NOTSUPP if the requested algorithm or operation is unsupported.
\item VIRTIO_CRYPTO_BADMSG if the verification result is incorrect.
\item VIRTIO_CRYPTO_INVSESS if the session ID invalid when in session mode.
\item VIRTIO_CRYPTO_KEY_REJECTED if the signature verification failed.
\item VIRTIO_CRYPTO_ERR if any failure not mentioned above occurs.
\end{itemize*}
\end{itemize*}

\section{Crypto Device}\label{sec:Device Types / Crypto Device}

The virtio crypto device is a virtual cryptography device as well as a
virtual cryptographic accelerator. The virtio crypto device provides the
following crypto services: CIPHER, MAC, HASH, AEAD and AKCIPHER. Virtio crypto
devices have a single control queue and at least one data queue. Crypto
operation requests are placed into a data queue, and serviced by the
device. Some crypto operation requests are only valid in the context of a
session. The role of the control queue is facilitating control operation
requests. Sessions management is realized with control operation
requests.

\subsection{Device ID}\label{sec:Device Types / Crypto Device / Device ID}

20

\subsection{Virtqueues}\label{sec:Device Types / Crypto Device / Virtqueues}

\begin{description}
\item[0] dataq1
\item[\ldots]
\item[N-1] dataqN
\item[N] controlq
\end{description}

N is set by \field{max_dataqueues}.

\subsection{Feature bits}\label{sec:Device Types / Crypto Device / Feature bits}

\begin{description}
\item VIRTIO_CRYPTO_F_REVISION_1 (0) revision 1. Revision 1 has a specific
    request format and other enhancements (which result in some additional
    requirements).
\item VIRTIO_CRYPTO_F_CIPHER_STATELESS_MODE (1) stateless mode requests are
    supported by the CIPHER service.
\item VIRTIO_CRYPTO_F_HASH_STATELESS_MODE (2) stateless mode requests are
    supported by the HASH service.
\item VIRTIO_CRYPTO_F_MAC_STATELESS_MODE (3) stateless mode requests are
    supported by the MAC service.
\item VIRTIO_CRYPTO_F_AEAD_STATELESS_MODE (4) stateless mode requests are
    supported by the AEAD service.
\item VIRTIO_CRYPTO_F_AKCIPHER_STATELESS_MODE (5) stateless mode requests are
    supported by the AKCIPHER service.
\end{description}


\subsubsection{Feature bit requirements}\label{sec:Device Types / Crypto Device / Feature bit requirements}

Some crypto feature bits require other crypto feature bits
(see \ref{drivernormative:Basic Facilities of a Virtio Device / Feature Bits}):

\begin{description}
\item[VIRTIO_CRYPTO_F_CIPHER_STATELESS_MODE] Requires VIRTIO_CRYPTO_F_REVISION_1.
\item[VIRTIO_CRYPTO_F_HASH_STATELESS_MODE] Requires VIRTIO_CRYPTO_F_REVISION_1.
\item[VIRTIO_CRYPTO_F_MAC_STATELESS_MODE] Requires VIRTIO_CRYPTO_F_REVISION_1.
\item[VIRTIO_CRYPTO_F_AEAD_STATELESS_MODE] Requires VIRTIO_CRYPTO_F_REVISION_1.
\item[VIRTIO_CRYPTO_F_AKCIPHER_STATELESS_MODE] Requires VIRTIO_CRYPTO_F_REVISION_1.
\end{description}

\subsection{Supported crypto services}\label{sec:Device Types / Crypto Device / Supported crypto services}

The following crypto services are defined:

\begin{lstlisting}
/* CIPHER (Symmetric Key Cipher) service */
#define VIRTIO_CRYPTO_SERVICE_CIPHER 0
/* HASH service */
#define VIRTIO_CRYPTO_SERVICE_HASH   1
/* MAC (Message Authentication Codes) service */
#define VIRTIO_CRYPTO_SERVICE_MAC    2
/* AEAD (Authenticated Encryption with Associated Data) service */
#define VIRTIO_CRYPTO_SERVICE_AEAD   3
/* AKCIPHER (Asymmetric Key Cipher) service */
#define VIRTIO_CRYPTO_SERVICE_AKCIPHER 4
\end{lstlisting}

The above constants designate bits used to indicate the which of crypto services are
offered by the device as described in, see \ref{sec:Device Types / Crypto Device / Device configuration layout}.

\subsubsection{CIPHER services}\label{sec:Device Types / Crypto Device / Supported crypto services / CIPHER services}

The following CIPHER algorithms are defined:

\begin{lstlisting}
#define VIRTIO_CRYPTO_NO_CIPHER                 0
#define VIRTIO_CRYPTO_CIPHER_ARC4               1
#define VIRTIO_CRYPTO_CIPHER_AES_ECB            2
#define VIRTIO_CRYPTO_CIPHER_AES_CBC            3
#define VIRTIO_CRYPTO_CIPHER_AES_CTR            4
#define VIRTIO_CRYPTO_CIPHER_DES_ECB            5
#define VIRTIO_CRYPTO_CIPHER_DES_CBC            6
#define VIRTIO_CRYPTO_CIPHER_3DES_ECB           7
#define VIRTIO_CRYPTO_CIPHER_3DES_CBC           8
#define VIRTIO_CRYPTO_CIPHER_3DES_CTR           9
#define VIRTIO_CRYPTO_CIPHER_KASUMI_F8          10
#define VIRTIO_CRYPTO_CIPHER_SNOW3G_UEA2        11
#define VIRTIO_CRYPTO_CIPHER_AES_F8             12
#define VIRTIO_CRYPTO_CIPHER_AES_XTS            13
#define VIRTIO_CRYPTO_CIPHER_ZUC_EEA3           14
\end{lstlisting}

The above constants have two usages:
\begin{enumerate}
\item As bit numbers, used to tell the driver which CIPHER algorithms
are supported by the device, see \ref{sec:Device Types / Crypto Device / Device configuration layout}.
\item As values, used to designate the algorithm in (CIPHER type) crypto
operation requests, see \ref{sec:Device Types / Crypto Device / Device Operation / Control Virtqueue / Session operation}.
\end{enumerate}

\subsubsection{HASH services}\label{sec:Device Types / Crypto Device / Supported crypto services / HASH services}

The following HASH algorithms are defined:

\begin{lstlisting}
#define VIRTIO_CRYPTO_NO_HASH            0
#define VIRTIO_CRYPTO_HASH_MD5           1
#define VIRTIO_CRYPTO_HASH_SHA1          2
#define VIRTIO_CRYPTO_HASH_SHA_224       3
#define VIRTIO_CRYPTO_HASH_SHA_256       4
#define VIRTIO_CRYPTO_HASH_SHA_384       5
#define VIRTIO_CRYPTO_HASH_SHA_512       6
#define VIRTIO_CRYPTO_HASH_SHA3_224      7
#define VIRTIO_CRYPTO_HASH_SHA3_256      8
#define VIRTIO_CRYPTO_HASH_SHA3_384      9
#define VIRTIO_CRYPTO_HASH_SHA3_512      10
#define VIRTIO_CRYPTO_HASH_SHA3_SHAKE128      11
#define VIRTIO_CRYPTO_HASH_SHA3_SHAKE256      12
\end{lstlisting}

The above constants have two usages:
\begin{enumerate}
\item As bit numbers, used to tell the driver which HASH algorithms
are supported by the device, see \ref{sec:Device Types / Crypto Device / Device configuration layout}.
\item As values, used to designate the algorithm in (HASH type) crypto
operation requires, see \ref{sec:Device Types / Crypto Device / Device Operation / Control Virtqueue / Session operation}.
\end{enumerate}

\subsubsection{MAC services}\label{sec:Device Types / Crypto Device / Supported crypto services / MAC services}

The following MAC algorithms are defined:

\begin{lstlisting}
#define VIRTIO_CRYPTO_NO_MAC                       0
#define VIRTIO_CRYPTO_MAC_HMAC_MD5                 1
#define VIRTIO_CRYPTO_MAC_HMAC_SHA1                2
#define VIRTIO_CRYPTO_MAC_HMAC_SHA_224             3
#define VIRTIO_CRYPTO_MAC_HMAC_SHA_256             4
#define VIRTIO_CRYPTO_MAC_HMAC_SHA_384             5
#define VIRTIO_CRYPTO_MAC_HMAC_SHA_512             6
#define VIRTIO_CRYPTO_MAC_CMAC_3DES                25
#define VIRTIO_CRYPTO_MAC_CMAC_AES                 26
#define VIRTIO_CRYPTO_MAC_KASUMI_F9                27
#define VIRTIO_CRYPTO_MAC_SNOW3G_UIA2              28
#define VIRTIO_CRYPTO_MAC_GMAC_AES                 41
#define VIRTIO_CRYPTO_MAC_GMAC_TWOFISH             42
#define VIRTIO_CRYPTO_MAC_CBCMAC_AES               49
#define VIRTIO_CRYPTO_MAC_CBCMAC_KASUMI_F9         50
#define VIRTIO_CRYPTO_MAC_XCBC_AES                 53
#define VIRTIO_CRYPTO_MAC_ZUC_EIA3                 54
\end{lstlisting}

The above constants have two usages:
\begin{enumerate}
\item As bit numbers, used to tell the driver which MAC algorithms
are supported by the device, see \ref{sec:Device Types / Crypto Device / Device configuration layout}.
\item As values, used to designate the algorithm in (MAC type) crypto
operation requests, see \ref{sec:Device Types / Crypto Device / Device Operation / Control Virtqueue / Session operation}.
\end{enumerate}

\subsubsection{AEAD services}\label{sec:Device Types / Crypto Device / Supported crypto services / AEAD services}

The following AEAD algorithms are defined:

\begin{lstlisting}
#define VIRTIO_CRYPTO_NO_AEAD     0
#define VIRTIO_CRYPTO_AEAD_GCM    1
#define VIRTIO_CRYPTO_AEAD_CCM    2
#define VIRTIO_CRYPTO_AEAD_CHACHA20_POLY1305  3
\end{lstlisting}

The above constants have two usages:
\begin{enumerate}
\item As bit numbers, used to tell the driver which AEAD algorithms
are supported by the device, see \ref{sec:Device Types / Crypto Device / Device configuration layout}.
\item As values, used to designate the algorithm in (DEAD type) crypto
operation requests, see \ref{sec:Device Types / Crypto Device / Device Operation / Control Virtqueue / Session operation}.
\end{enumerate}

\subsubsection{AKCIPHER services}\label{sec: Device Types / Crypto Device / Supported crypto services / AKCIPHER services}

The following AKCIPHER algorithms are defined:
\begin{lstlisting}
#define VIRTIO_CRYPTO_NO_AKCIPHER 0
#define VIRTIO_CRYPTO_AKCIPHER_RSA   1
#define VIRTIO_CRYPTO_AKCIPHER_ECDSA 2
\end{lstlisting}

The above constants have two usages:
\begin{enumerate}
\item As bit numbers, used to tell the driver which AKCIPHER algorithms
are supported by the device, see \ref{sec:Device Types / Crypto Device / Device configuration layout}.
\item As values, used to designate the algorithm in asymmetric crypto operation requests,
see \ref{sec:Device Types / Crypto Device / Device Operation / Control Virtqueue / Session operation}.
\end{enumerate}


\subsection{Device configuration layout}\label{sec:Device Types / Crypto Device / Device configuration layout}

Crypto device configuration uses the following layout structure:

\begin{lstlisting}
struct virtio_crypto_config {
    le32 status;
    le32 max_dataqueues;
    le32 crypto_services;
    /* Detailed algorithms mask */
    le32 cipher_algo_l;
    le32 cipher_algo_h;
    le32 hash_algo;
    le32 mac_algo_l;
    le32 mac_algo_h;
    le32 aead_algo;
    /* Maximum length of cipher key in bytes */
    le32 max_cipher_key_len;
    /* Maximum length of authenticated key in bytes */
    le32 max_auth_key_len;
    le32 akcipher_algo;
    /* Maximum size of each crypto request's content in bytes */
    le64 max_size;
};
\end{lstlisting}

\begin{description}
\item Currently, only one \field{status} bit is defined: VIRTIO_CRYPTO_S_HW_READY
    set indicates that the device is ready to process requests, this bit is read-only
    for the driver
\begin{lstlisting}
#define VIRTIO_CRYPTO_S_HW_READY  (1 << 0)
\end{lstlisting}

\item [\field{max_dataqueues}] is the maximum number of data virtqueues that can
    be configured by the device. The driver MAY use only one data queue, or it
    can use more to achieve better performance.

\item [\field{crypto_services}] crypto service offered, see \ref{sec:Device Types / Crypto Device / Supported crypto services}.

\item [\field{cipher_algo_l}] CIPHER algorithms bits 0-31, see \ref{sec:Device Types / Crypto Device / Supported crypto services  / CIPHER services}.

\item [\field{cipher_algo_h}] CIPHER algorithms bits 32-63, see \ref{sec:Device Types / Crypto Device / Supported crypto services  / CIPHER services}.

\item [\field{hash_algo}] HASH algorithms bits, see \ref{sec:Device Types / Crypto Device / Supported crypto services  / HASH services}.

\item [\field{mac_algo_l}] MAC algorithms bits 0-31, see \ref{sec:Device Types / Crypto Device / Supported crypto services  / MAC services}.

\item [\field{mac_algo_h}] MAC algorithms bits 32-63, see \ref{sec:Device Types / Crypto Device / Supported crypto services  / MAC services}.

\item [\field{aead_algo}] AEAD algorithms bits, see \ref{sec:Device Types / Crypto Device / Supported crypto services  / AEAD services}.

\item [\field{max_cipher_key_len}] is the maximum length of cipher key supported by the device.

\item [\field{max_auth_key_len}] is the maximum length of authenticated key supported by the device.

\item [\field{akcipher_algo}] AKCIPHER algorithms bit 0-31, see \ref{sec: Device Types / Crypto Device / Supported crypto services / AKCIPHER services}.

\item [\field{max_size}] is the maximum size of the variable-length parameters of
    data operation of each crypto request's content supported by the device.
\end{description}

\begin{note}
Unless explicitly stated otherwise all lengths and sizes are in bytes.
\end{note}

\devicenormative{\subsubsection}{Device configuration layout}{Device Types / Crypto Device / Device configuration layout}

\begin{itemize*}
\item The device MUST set \field{max_dataqueues} to between 1 and 65535 inclusive.
\item The device MUST set the \field{status} with valid flags, undefined flags MUST NOT be set.
\item The device MUST accept and handle requests after \field{status} is set to VIRTIO_CRYPTO_S_HW_READY.
\item The device MUST set \field{crypto_services} based on the crypto services the device offers.
\item The device MUST set detailed algorithms masks for each service advertised by \field{crypto_services}.
    The device MUST NOT set the not defined algorithms bits.
\item The device MUST set \field{max_size} to show the maximum size of crypto request the device supports.
\item The device MUST set \field{max_cipher_key_len} to show the maximum length of cipher key if the
    device supports CIPHER service.
\item The device MUST set \field{max_auth_key_len} to show the maximum length of authenticated key if
    the device supports MAC service.
\end{itemize*}

\drivernormative{\subsubsection}{Device configuration layout}{Device Types / Crypto Device / Device configuration layout}

\begin{itemize*}
\item The driver MUST read the \field{status} from the bottom bit of status to check whether the
    VIRTIO_CRYPTO_S_HW_READY is set, and the driver MUST reread it after device reset.
\item The driver MUST NOT transmit any requests to the device if the VIRTIO_CRYPTO_S_HW_READY is not set.
\item The driver MUST read \field{max_dataqueues} field to discover the number of data queues the device supports.
\item The driver MUST read \field{crypto_services} field to discover which services the device is able to offer.
\item The driver SHOULD ignore the not defined algorithms bits.
\item The driver MUST read the detailed algorithms fields based on \field{crypto_services} field.
\item The driver SHOULD read \field{max_size} to discover the maximum size of the variable-length
    parameters of data operation of the crypto request's content the device supports and MUST
    guarantee that the size of each crypto request's content is within the \field{max_size}, otherwise
    the request will fail and the driver MUST reset the device.
\item The driver SHOULD read \field{max_cipher_key_len} to discover the maximum length of cipher key
    the device supports and MUST guarantee that the \field{key_len} (CIPHER service or AEAD service) is within
    the \field{max_cipher_key_len} of the device configuration, otherwise the request will fail.
\item The driver SHOULD read \field{max_auth_key_len} to discover the maximum length of authenticated
    key the device supports and MUST guarantee that the \field{auth_key_len} (MAC service) is within the
    \field{max_auth_key_len} of the device configuration, otherwise the request will fail.
\end{itemize*}

\subsection{Device Initialization}\label{sec:Device Types / Crypto Device / Device Initialization}

\drivernormative{\subsubsection}{Device Initialization}{Device Types / Crypto Device / Device Initialization}

\begin{itemize*}
\item The driver MUST configure and initialize all virtqueues.
\item The driver MUST read the supported crypto services from bits of \field{crypto_services}.
\item The driver MUST read the supported algorithms based on \field{crypto_services} field.
\end{itemize*}

\subsection{Device Operation}\label{sec:Device Types / Crypto Device / Device Operation}

The operation of a virtio crypto device is driven by requests placed on the virtqueues.
Requests consist of a queue-type specific header (specifying among others the operation)
and an operation specific payload.

If VIRTIO_CRYPTO_F_REVISION_1 is negotiated the device may support both session mode
(See \ref{sec:Device Types / Crypto Device / Device Operation / Control Virtqueue / Session operation})
and stateless mode operation requests.
In stateless mode all operation parameters are supplied as a part of each request,
while in session mode, some or all operation parameters are managed within the
session. Stateless mode is guarded by feature bits 0-4 on a service level. If
stateless mode is negotiated for a service, the service accepts both session
mode and stateless requests; otherwise stateless mode requests are rejected
(via operation status).

\subsubsection{Operation Status}\label{sec:Device Types / Crypto Device / Device Operation / Operation status}
The device MUST return a status code as part of the operation (both session
operation and service operation) result. The valid operation status as follows:

\begin{lstlisting}
enum VIRTIO_CRYPTO_STATUS {
    VIRTIO_CRYPTO_OK = 0,
    VIRTIO_CRYPTO_ERR = 1,
    VIRTIO_CRYPTO_BADMSG = 2,
    VIRTIO_CRYPTO_NOTSUPP = 3,
    VIRTIO_CRYPTO_INVSESS = 4,
    VIRTIO_CRYPTO_NOSPC = 5,
    VIRTIO_CRYPTO_KEY_REJECTED = 6,
    VIRTIO_CRYPTO_MAX
};
\end{lstlisting}

\begin{itemize*}
\item VIRTIO_CRYPTO_OK: success.
\item VIRTIO_CRYPTO_BADMSG: authentication failed (only when AEAD decryption).
\item VIRTIO_CRYPTO_NOTSUPP: operation or algorithm is unsupported.
\item VIRTIO_CRYPTO_INVSESS: invalid session ID when executing crypto operations.
\item VIRTIO_CRYPTO_NOSPC: no free session ID (only when the VIRTIO_CRYPTO_F_REVISION_1
    feature bit is negotiated).
\item VIRTIO_CRYPTO_KEY_REJECTED: signature verification failed (only when AKCIPHER verification).
\item VIRTIO_CRYPTO_ERR: any failure not mentioned above occurs.
\end{itemize*}

\subsubsection{Control Virtqueue}\label{sec:Device Types / Crypto Device / Device Operation / Control Virtqueue}

The driver uses the control virtqueue to send control commands to the
device, such as session operations (See \ref{sec:Device Types / Crypto Device / Device
Operation / Control Virtqueue / Session operation}).

The header for controlq is of the following form:
\begin{lstlisting}
#define VIRTIO_CRYPTO_OPCODE(service, op)   (((service) << 8) | (op))

struct virtio_crypto_ctrl_header {
#define VIRTIO_CRYPTO_CIPHER_CREATE_SESSION \
       VIRTIO_CRYPTO_OPCODE(VIRTIO_CRYPTO_SERVICE_CIPHER, 0x02)
#define VIRTIO_CRYPTO_CIPHER_DESTROY_SESSION \
       VIRTIO_CRYPTO_OPCODE(VIRTIO_CRYPTO_SERVICE_CIPHER, 0x03)
#define VIRTIO_CRYPTO_HASH_CREATE_SESSION \
       VIRTIO_CRYPTO_OPCODE(VIRTIO_CRYPTO_SERVICE_HASH, 0x02)
#define VIRTIO_CRYPTO_HASH_DESTROY_SESSION \
       VIRTIO_CRYPTO_OPCODE(VIRTIO_CRYPTO_SERVICE_HASH, 0x03)
#define VIRTIO_CRYPTO_MAC_CREATE_SESSION \
       VIRTIO_CRYPTO_OPCODE(VIRTIO_CRYPTO_SERVICE_MAC, 0x02)
#define VIRTIO_CRYPTO_MAC_DESTROY_SESSION \
       VIRTIO_CRYPTO_OPCODE(VIRTIO_CRYPTO_SERVICE_MAC, 0x03)
#define VIRTIO_CRYPTO_AEAD_CREATE_SESSION \
       VIRTIO_CRYPTO_OPCODE(VIRTIO_CRYPTO_SERVICE_AEAD, 0x02)
#define VIRTIO_CRYPTO_AEAD_DESTROY_SESSION \
       VIRTIO_CRYPTO_OPCODE(VIRTIO_CRYPTO_SERVICE_AEAD, 0x03)
#define VIRTIO_CRYPTO_AKCIPHER_CREATE_SESSION \
       VIRTIO_CRYPTO_OPCODE(VIRTIO_CRYPTO_SERVICE_AKCIPHER, 0x04)
#define VIRTIO_CRYPTO_AKCIPHER_DESTROY_SESSION \
       VIRTIO_CRYPTO_OPCDE(VIRTIO_CRYPTO_SERVICE_AKCIPHER, 0x05)
    le32 opcode;
    /* algo should be service-specific algorithms */
    le32 algo;
    le32 flag;
    le32 reserved;
};
\end{lstlisting}

The controlq request is composed of four parts:
\begin{lstlisting}
struct virtio_crypto_op_ctrl_req {
    /* Device read only portion */

    struct virtio_crypto_ctrl_header header;

#define VIRTIO_CRYPTO_CTRLQ_OP_SPEC_HDR_LEGACY 56
    /* fixed length fields, opcode specific */
    u8 op_flf[flf_len];

    /* variable length fields, opcode specific */
    u8 op_vlf[vlf_len];

    /* Device write only portion */

    /* op result or completion status */
    u8 op_outcome[outcome_len];
};
\end{lstlisting}

\field{header} is a general header (see above).

\field{op_flf} is the opcode (in \field{header}) specific fixed-length parameters.

\field{flf_len} depends on the VIRTIO_CRYPTO_F_REVISION_1 feature bit (see below).

\field{op_vlf} is the opcode (in \field{header}) specific variable-length parameters.

\field{vlf_len} is the size of the specific structure used.
\begin{note}
The \field{vlf_len} of session-destroy operation and the hash-session-create
operation is ZERO.
\end{note}

\begin{itemize*}
\item If the opcode (in \field{header}) is VIRTIO_CRYPTO_CIPHER_CREATE_SESSION
    then \field{op_flf} is struct virtio_crypto_sym_create_session_flf if
    VIRTIO_CRYPTO_F_REVISION_1 is negotiated and struct virtio_crypto_sym_create_session_flf is
    padded to 56 bytes if NOT negotiated, and \field{op_vlf} is struct
    virtio_crypto_sym_create_session_vlf.
\item If the opcode (in \field{header}) is VIRTIO_CRYPTO_HASH_CREATE_SESSION
    then \field{op_flf} is struct virtio_crypto_hash_create_session_flf if
    VIRTIO_CRYPTO_F_REVISION_1 is negotiated and struct virtio_crypto_hash_create_session_flf is
    padded to 56 bytes if NOT negotiated.
\item If the opcode (in \field{header}) is VIRTIO_CRYPTO_MAC_CREATE_SESSION
    then \field{op_flf} is struct virtio_crypto_mac_create_session_flf if
    VIRTIO_CRYPTO_F_REVISION_1 is negotiated and struct virtio_crypto_mac_create_session_flf is
    padded to 56 bytes if NOT negotiated, and \field{op_vlf} is struct
    virtio_crypto_mac_create_session_vlf.
\item If the opcode (in \field{header}) is VIRTIO_CRYPTO_AEAD_CREATE_SESSION
    then \field{op_flf} is struct virtio_crypto_aead_create_session_flf if
    VIRTIO_CRYPTO_F_REVISION_1 is negotiated and struct virtio_crypto_aead_create_session_flf is
    padded to 56 bytes if NOT negotiated, and \field{op_vlf} is struct
    virtio_crypto_aead_create_session_vlf.
\item If the opcode (in \field{header}) is VIRTIO_CRYPTO_AKCIPHER_CREATE_SESSION
    then \field{op_flf} is struct virtio_crypto_akcipher_create_session_flf if
    VIRTIO_CRYPTO_F_REVISION_1 is negotiated and struct virtio_crypto_akcipher_create_session_flf is
    padded to 56 bytes if NOT negotiated, and \field{op_vlf} is struct
    virtio_crypto_akcipher_create_session_vlf.
\item If the opcode (in \field{header}) is VIRTIO_CRYPTO_CIPHER_DESTROY_SESSION
    or VIRTIO_CRYPTO_HASH_DESTROY_SESSION or VIRTIO_CRYPTO_MAC_DESTROY_SESSION or
    VIRTIO_CRYPTO_AEAD_DESTROY_SESSION then \field{op_flf} is struct
    virtio_crypto_destroy_session_flf if VIRTIO_CRYPTO_F_REVISION_1 is negotiated and
    struct virtio_crypto_destroy_session_flf is padded to 56 bytes if NOT negotiated.
\end{itemize*}

\field{op_outcome} stores the result of operation and must be struct
virtio_crypto_destroy_session_input for destroy session or
struct virtio_crypto_create_session_input for create session.

\field{outcome_len} is the size of the structure used.


\paragraph{Session operation}\label{sec:Device Types / Crypto Device / Device
Operation / Control Virtqueue / Session operation}

The session is a handle which describes the cryptographic parameters to be
applied to a number of buffers.

The following structure stores the result of session creation set by the device:

\begin{lstlisting}
struct virtio_crypto_create_session_input {
    le64 session_id;
    le32 status;
    le32 padding;
};
\end{lstlisting}

A request to destroy a session includes the following information:

\begin{lstlisting}
struct virtio_crypto_destroy_session_flf {
    /* Device read only portion */
    le64  session_id;
};

struct virtio_crypto_destroy_session_input {
    /* Device write only portion */
    u8  status;
};
\end{lstlisting}


\subparagraph{Session operation: HASH session}\label{sec:Device Types / Crypto Device / Device
Operation / Control Virtqueue / Session operation / Session operation: HASH session}

The fixed-length parameters of HASH session requests is as follows:

\begin{lstlisting}
struct virtio_crypto_hash_create_session_flf {
    /* Device read only portion */

    /* See VIRTIO_CRYPTO_HASH_* above */
    le32 algo;
    /* hash result length */
    le32 hash_result_len;
};
\end{lstlisting}


\subparagraph{Session operation: MAC session}\label{sec:Device Types / Crypto Device / Device
Operation / Control Virtqueue / Session operation / Session operation: MAC session}

The fixed-length and the variable-length parameters of MAC session requests are as follows:

\begin{lstlisting}
struct virtio_crypto_mac_create_session_flf {
    /* Device read only portion */

    /* See VIRTIO_CRYPTO_MAC_* above */
    le32 algo;
    /* hash result length */
    le32 hash_result_len;
    /* length of authenticated key */
    le32 auth_key_len;
    le32 padding;
};

struct virtio_crypto_mac_create_session_vlf {
    /* Device read only portion */

    /* The authenticated key */
    u8 auth_key[auth_key_len];
};
\end{lstlisting}

The length of \field{auth_key} is specified in \field{auth_key_len} in the struct
virtio_crypto_mac_create_session_flf.


\subparagraph{Session operation: Symmetric algorithms session}\label{sec:Device Types / Crypto Device / Device
Operation / Control Virtqueue / Session operation / Session operation: Symmetric algorithms session}

The request of symmetric session could be the CIPHER algorithms request
or the chain algorithms (chaining CIPHER and HASH/MAC) request.

The fixed-length and the variable-length parameters of CIPHER session requests are as follows:

\begin{lstlisting}
struct virtio_crypto_cipher_session_flf {
    /* Device read only portion */

    /* See VIRTIO_CRYPTO_CIPHER* above */
    le32 algo;
    /* length of key */
    le32 key_len;
#define VIRTIO_CRYPTO_OP_ENCRYPT  1
#define VIRTIO_CRYPTO_OP_DECRYPT  2
    /* encryption or decryption */
    le32 op;
    le32 padding;
};

struct virtio_crypto_cipher_session_vlf {
    /* Device read only portion */

    /* The cipher key */
    u8 cipher_key[key_len];
};
\end{lstlisting}

The length of \field{cipher_key} is specified in \field{key_len} in the struct
virtio_crypto_cipher_session_flf.

The fixed-length and the variable-length parameters of Chain session requests are as follows:

\begin{lstlisting}
struct virtio_crypto_alg_chain_session_flf {
    /* Device read only portion */

#define VIRTIO_CRYPTO_SYM_ALG_CHAIN_ORDER_HASH_THEN_CIPHER  1
#define VIRTIO_CRYPTO_SYM_ALG_CHAIN_ORDER_CIPHER_THEN_HASH  2
    le32 alg_chain_order;
/* Plain hash */
#define VIRTIO_CRYPTO_SYM_HASH_MODE_PLAIN    1
/* Authenticated hash (mac) */
#define VIRTIO_CRYPTO_SYM_HASH_MODE_AUTH     2
/* Nested hash */
#define VIRTIO_CRYPTO_SYM_HASH_MODE_NESTED   3
    le32 hash_mode;
    struct virtio_crypto_cipher_session_flf cipher_hdr;

#define VIRTIO_CRYPTO_ALG_CHAIN_SESS_OP_SPEC_HDR_SIZE  16
    /* fixed length fields, algo specific */
    u8 algo_flf[VIRTIO_CRYPTO_ALG_CHAIN_SESS_OP_SPEC_HDR_SIZE];

    /* length of the additional authenticated data (AAD) in bytes */
    le32 aad_len;
    le32 padding;
};

struct virtio_crypto_alg_chain_session_vlf {
    /* Device read only portion */

    /* The cipher key */
    u8 cipher_key[key_len];
    /* The authenticated key */
    u8 auth_key[auth_key_len];
};
\end{lstlisting}

\field{hash_mode} decides the type used by \field{algo_flf}.

\field{algo_flf} is fixed to 16 bytes and MUST contains or be one of
the following types:
\begin{itemize*}
\item struct virtio_crypto_hash_create_session_flf
\item struct virtio_crypto_mac_create_session_flf
\end{itemize*}
The data of unused part (if has) in \field{algo_flf} will be ignored.

The length of \field{cipher_key} is specified in \field{key_len} in \field{cipher_hdr}.

The length of \field{auth_key} is specified in \field{auth_key_len} in struct
virtio_crypto_mac_create_session_flf.

The fixed-length parameters of Symmetric session requests are as follows:

\begin{lstlisting}
struct virtio_crypto_sym_create_session_flf {
    /* Device read only portion */

#define VIRTIO_CRYPTO_SYM_SESS_OP_SPEC_HDR_SIZE  48
    /* fixed length fields, opcode specific */
    u8 op_flf[VIRTIO_CRYPTO_SYM_SESS_OP_SPEC_HDR_SIZE];

/* No operation */
#define VIRTIO_CRYPTO_SYM_OP_NONE  0
/* Cipher only operation on the data */
#define VIRTIO_CRYPTO_SYM_OP_CIPHER  1
/* Chain any cipher with any hash or mac operation. The order
   depends on the value of alg_chain_order param */
#define VIRTIO_CRYPTO_SYM_OP_ALGORITHM_CHAINING  2
    le32 op_type;
    le32 padding;
};
\end{lstlisting}

\field{op_flf} is fixed to 48 bytes, MUST contains or be one of
the following types:
\begin{itemize*}
\item struct virtio_crypto_cipher_session_flf
\item struct virtio_crypto_alg_chain_session_flf
\end{itemize*}
The data of unused part (if has) in \field{op_flf} will be ignored.

\field{op_type} decides the type used by \field{op_flf}.

The variable-length parameters of Symmetric session requests are as follows:

\begin{lstlisting}
struct virtio_crypto_sym_create_session_vlf {
    /* Device read only portion */
    /* variable length fields, opcode specific */
    u8 op_vlf[vlf_len];
};
\end{lstlisting}

\field{op_vlf} MUST contains or be one of the following types:
\begin{itemize*}
\item struct virtio_crypto_cipher_session_vlf
\item struct virtio_crypto_alg_chain_session_vlf
\end{itemize*}

\field{op_type} in struct virtio_crypto_sym_create_session_flf decides the
type used by \field{op_vlf}.

\field{vlf_len} is the size of the specific structure used.


\subparagraph{Session operation: AEAD session}\label{sec:Device Types / Crypto Device / Device
Operation / Control Virtqueue / Session operation / Session operation: AEAD session}

The fixed-length and the variable-length parameters of AEAD session requests are as follows:

\begin{lstlisting}
struct virtio_crypto_aead_create_session_flf {
    /* Device read only portion */

    /* See VIRTIO_CRYPTO_AEAD_* above */
    le32 algo;
    /* length of key */
    le32 key_len;
    /* Authentication tag length */
    le32 tag_len;
    /* The length of the additional authenticated data (AAD) in bytes */
    le32 aad_len;
    /* encryption or decryption, See above VIRTIO_CRYPTO_OP_* */
    le32 op;
    le32 padding;
};

struct virtio_crypto_aead_create_session_vlf {
    /* Device read only portion */
    u8 key[key_len];
};
\end{lstlisting}

The length of \field{key} is specified in \field{key_len} in struct
virtio_crypto_aead_create_session_flf.

\subparagraph{Session operation: AKCIPHER session}\label{sec:Device Types / Crypto Device / Device
Operation / Control Virtqueue / Session operation / Session operation: AKCIPHER session}

Due to the complexity of asymmetric key algorithms, different algorithms
require different parameters. The following data structures are used as
supplementary parameters to describe the asymmetric algorithm sessions.

For the RSA algorithm, the extra parameters are as follows:
\begin{lstlisting}
struct virtio_crypto_rsa_session_para {
#define VIRTIO_CRYPTO_RSA_RAW_PADDING   0
#define VIRTIO_CRYPTO_RSA_PKCS1_PADDING 1
    le32 padding_algo;

#define VIRTIO_CRYPTO_RSA_NO_HASH   0
#define VIRTIO_CRYPTO_RSA_MD2       1
#define VIRTIO_CRYPTO_RSA_MD3       2
#define VIRTIO_CRYPTO_RSA_MD4       3
#define VIRTIO_CRYPTO_RSA_MD5       4
#define VIRTIO_CRYPTO_RSA_SHA1      5
#define VIRTIO_CRYPTO_RSA_SHA256    6
#define VIRTIO_CRYPTO_RSA_SHA384    7
#define VIRTIO_CRYPTO_RSA_SHA512    8
#define VIRTIO_CRYPTO_RSA_SHA224    9
    le32 hash_algo;
};
\end{lstlisting}

\field{padding_algo} specifies the padding method used by RSA sessions.
\begin{itemize*}
\item If VIRTIO_CRYPTO_RSA_RAW_PADDING is specified, 1) \field{hash_algo}
is ignored, 2) ciphertext and plaintext MUST be padded with leading zeros,
3) and RSA sessions with VIRTIO_CRYPTO_RSA_RAW_PADDING MUST not be used
for verification and signing operations.
\item If VIRTIO_CRYPTO_RSA_PKCS1_PADDING is specified, EMSA-PKCS1-v1_5 padding method
is used (see \hyperref[intro:rfc3447]{PKCS\#1}), \field{hash_algo} specifies how the
digest of the data passed to RSA sessions is calculated when verifying and signing.
It only affects the padding algorithm and is ignored during encryption and decryption.
\end{itemize*}

The ECC algorithms such as the ECDSA algorithm, cannot use custom curves, only the
following known curves can be used (see \hyperref[intro:NIST]{NIST-recommended curves}).

\begin{lstlisting}
#define VIRTIO_CRYPTO_CURVE_UNKNOWN   0
#define VIRTIO_CRYPTO_CURVE_NIST_P192 1
#define VIRTIO_CRYPTO_CURVE_NIST_P224 2
#define VIRTIO_CRYPTO_CURVE_NIST_P256 3
#define VIRTIO_CRYPTO_CURVE_NIST_P384 4
#define VIRTIO_CRYPTO_CURVE_NIST_P521 5
\end{lstlisting}

For the ECDSA algorithm, the extra parameters are as follows:
\begin{lstlisting}
struct virtio_crypto_ecdsa_session_para {
    /* See VIRTIO_CRYPTO_CURVE_* above */
    le32 curve_id;
};
\end{lstlisting}

The fixed-length and the variable-length parameters of AKCIPHER session requests are as follows:
\begin{lstlisting}
struct virtio_crypto_akcipher_create_session_flf {
    /* Device read only portion */

    /* See VIRTIO_CRYPTO_AKCIPHER_* above */
    le32 algo;
#define VIRTIO_CRYPTO_AKCIPHER_KEY_TYPE_PUBLIC 1
#define VIRTIO_CRYPTO_AKCIPHER_KEY_TYPE_PRIVATE 2
    le32 key_type;
    /* length of key */
    le32 key_len;

#define VIRTIO_CRYPTO_AKCIPHER_SESS_ALGO_SPEC_HDR_SIZE 44
    u8 algo_flf[VIRTIO_CRYPTO_AKCIPHER_SESS_ALGO_SPEC_HDR_SIZE];
};

struct virtio_crypto_akcipher_create_session_vlf {
    /* Device read only portion */
    u8 key[key_len];
};
\end{lstlisting}

\field{algo} decides the type used by \field{algo_flf}.
\field{algo_flf} is fixed to 44 bytes and MUST contains of be one the
following structures:
\begin{itemize*}
\item struct virtio_crypto_rsa_session_para
\item struct virtio_crypto_ecdsa_session_para
\end{itemize*}

The length of \field{key} is specified in \field{key_len} in the struct
virtio_crypto_akcipher_create_session_flf.

For the RSA algorithm, the key needs to be encoded according to
\hyperref[intro:rfc3447]{PKCS\#1}. The private key is described with the
RSAPrivateKey structure, and the public key is described with the RSAPublicKey
structure. These ASN.1 structures are encoded in DER encoding rules (see
\hyperref[intro:rfc6025]{rfc6025}).

\begin{lstlisting}
RSAPrivateKey ::= SEQUENCE {
    version          INTEGER,
    modulus          INTEGER,
    publicExponent   INTEGER,
    privateExponent  INTEGER,
    prime1           INTEGER,
    prime2           INTEGER,
    exponent1        INTEGER,
    exponent1        INTEGER,
    coefficient      INTEGER,
    otherPrimeInfos  OtherPrimeInfos OPTIONAL
}

OtherPrimeInfos ::= SEQUENCE SIZE(1...MAX) OF OtherPrimeInfo

OtherPrimeINfo ::= SEQUENCE {
    prime           INTEGER,
    exponent        INTEGER,
    coefficient     INTEGER
}

RSAPublicKey ::= SEQUENCE {
    modulus         INTEGER,
    publicExponent  INTEGER
}
\end{lstlisting}

For the ECDSA algorithm, the private key is encoded according to
\hyperref[intro:rfc5915]{RFC5915}, the private key of the ECDSA algorithm
is described by the ASN.1 structure ECPrivateKey and encoded with DER
encoding rules (see \hyperref[intro:rfc6025]{rfc6025}).

\begin{lstlisting}
ECPrivateKey ::= SEQUNCE {
    version         INTEGER,
    privateKey      OCTET STRING,
    parameters [0]  ECParameters {{ NamedCurve }} OPTIONAL,
    publicKey  [1]  BIT STRING OPTIONAL
}
\end{lstlisting}

The public key of the ECDSA algorithm is encoded according to \hyperref[intro:SEC1]{SEC1},
and the public key of ECDSA is described by the ASN.1 structure ECPoint.
When initializing a session with ECDSA public key, the ECPoint is DER encoded and the
\field{key} only contains the value part of ECPoint, that is, the header part of the
OCTET STRING will be omitted (see \hyperref[intro:rfc6025]{rfc6025}).

\begin{lstlisting}
ECPoint ::= OCTET STRING
\end{lstlisting}

The length of \field{key} is specified in \field{key_len} in
struct virtio_crypto_akcipher_create_session_flf.

\drivernormative{\subparagraph}{Session operation: create session}{Device Types / Crypto Device / Device
Operation / Control Virtqueue / Session operation / Session operation: create session}

\begin{itemize*}
\item The driver MUST set the \field{opcode} field based on service type: CIPHER, HASH, MAC, AEAD or AKCIPHER.
\item The driver MUST set the control general header, the opcode specific header,
    the opcode specific extra parameters and the opcode specific outcome buffer in turn.
    See \ref{sec:Device Types / Crypto Device / Device Operation / Control Virtqueue}.
\item The driver MUST set the \field{reversed} field to zero.
\end{itemize*}

\devicenormative{\subparagraph}{Session operation: create session}{Device Types / Crypto Device / Device
Operation / Control Virtqueue / Session operation / Session operation: create session}

\begin{itemize*}
\item The device MUST use the corresponding opcode specific structure according to the
    \field{opcode} in the control general header.
\item The device MUST extract extra parameters according to the structures used.
\item The device MUST set the \field{status} field to one of the following values of enum
    VIRTIO_CRYPTO_STATUS after finish a session creation:
\begin{itemize*}
\item VIRTIO_CRYPTO_OK if a session is created successfully.
\item VIRTIO_CRYPTO_NOTSUPP if the requested algorithm or operation is unsupported.
\item VIRTIO_CRYPTO_NOSPC if no free session ID (only when the VIRTIO_CRYPTO_F_REVISION_1
    feature bit is negotiated).
\item VIRTIO_CRYPTO_ERR if failure not mentioned above occurs.
\end{itemize*}
\item The device MUST set the \field{session_id} field to a unique session identifier only
    if the status is set to VIRTIO_CRYPTO_OK.
\end{itemize*}

\drivernormative{\subparagraph}{Session operation: destroy session}{Device Types / Crypto Device / Device
Operation / Control Virtqueue / Session operation / Session operation: destroy session}

\begin{itemize*}
\item The driver MUST set the \field{opcode} field based on service type: CIPHER, HASH, MAC, AEAD or AKCIPHER.
\item The driver MUST set the \field{session_id} to a valid value assigned by the device
    when the session was created.
\end{itemize*}

\devicenormative{\subparagraph}{Session operation: destroy session}{Device Types / Crypto Device / Device
Operation / Control Virtqueue / Session operation / Session operation: destroy session}

\begin{itemize*}
\item The device MUST set the \field{status} field to one of the following values of enum VIRTIO_CRYPTO_STATUS.
\begin{itemize*}
\item VIRTIO_CRYPTO_OK if a session is created successfully.
\item VIRTIO_CRYPTO_ERR if any failure occurs.
\end{itemize*}
\end{itemize*}


\subsubsection{Data Virtqueue}\label{sec:Device Types / Crypto Device / Device Operation / Data Virtqueue}

The driver uses the data virtqueues to transmit crypto operation requests to the device,
and completes the crypto operations.

The header for dataq is as follows:

\begin{lstlisting}
struct virtio_crypto_op_header {
#define VIRTIO_CRYPTO_CIPHER_ENCRYPT \
    VIRTIO_CRYPTO_OPCODE(VIRTIO_CRYPTO_SERVICE_CIPHER, 0x00)
#define VIRTIO_CRYPTO_CIPHER_DECRYPT \
    VIRTIO_CRYPTO_OPCODE(VIRTIO_CRYPTO_SERVICE_CIPHER, 0x01)
#define VIRTIO_CRYPTO_HASH \
    VIRTIO_CRYPTO_OPCODE(VIRTIO_CRYPTO_SERVICE_HASH, 0x00)
#define VIRTIO_CRYPTO_MAC \
    VIRTIO_CRYPTO_OPCODE(VIRTIO_CRYPTO_SERVICE_MAC, 0x00)
#define VIRTIO_CRYPTO_AEAD_ENCRYPT \
    VIRTIO_CRYPTO_OPCODE(VIRTIO_CRYPTO_SERVICE_AEAD, 0x00)
#define VIRTIO_CRYPTO_AEAD_DECRYPT \
    VIRTIO_CRYPTO_OPCODE(VIRTIO_CRYPTO_SERVICE_AEAD, 0x01)
#define VIRTIO_CRYPTO_AKCIPHER_ENCRYPT \
    VIRTIO_CRYPTO_OPCODE(VIRTIO_CRYPTO_SERVICE_AKCIPHER, 0x00)
#define VIRTIO_CRYPTO_AKCIPHER_DECRYPT \
    VIRTIO_CRYPTO_OPCODE(VIRTIO_CRYPTO_SERVICE_AKCIPHER, 0x01)
#define VIRTIO_CRYPTO_AKCIPHER_SIGN \
    VIRTIO_CRYPTO_OPCODE(VIRTIO_CRYPTO_SERVICE_AKCIPHER, 0x02)
#define VIRTIO_CRYPTO_AKCIPHER_VERIFY \
    VIRTIO_CRYPTO_OPCODE(VIRTIO_CRYPTO_SERVICE_AKCIPHER, 0x03)
    le32 opcode;
    /* algo should be service-specific algorithms */
    le32 algo;
    le64 session_id;
#define VIRTIO_CRYPTO_FLAG_SESSION_MODE 1
    /* control flag to control the request */
    le32 flag;
    le32 padding;
};
\end{lstlisting}

\begin{note}
If VIRTIO_CRYPTO_F_REVISION_1 is not negotiated the \field{flag} is ignored.

If VIRTIO_CRYPTO_F_REVISION_1 is negotiated but VIRTIO_CRYPTO_F_<SERVICE>_STATELESS_MODE
is not negotiated, then the device SHOULD reject <SERVICE> requests if
VIRTIO_CRYPTO_FLAG_SESSION_MODE is not set (in \field{flag}).
\end{note}

The dataq request is composed of four parts:
\begin{lstlisting}
struct virtio_crypto_op_data_req {
    /* Device read only portion */

    struct virtio_crypto_op_header header;

#define VIRTIO_CRYPTO_DATAQ_OP_SPEC_HDR_LEGACY 48
    /* fixed length fields, opcode specific */
    u8 op_flf[flf_len];

    /* Device read && write portion */
    /* variable length fields, opcode specific */
    u8 op_vlf[vlf_len];

    /* Device write only portion */
    struct virtio_crypto_inhdr inhdr;
};
\end{lstlisting}

\field{header} is a general header (see above).

\field{op_flf} is the opcode (in \field{header}) specific header.

\field{flf_len} depends on the VIRTIO_CRYPTO_F_REVISION_1 feature bit
(see below).

\field{op_vlf} is the opcode (in \field{header}) specific parameters.

\field{vlf_len} is the size of the specific structure used.

\begin{itemize*}
\item If the the opcode (in \field{header}) is VIRTIO_CRYPTO_CIPHER_ENCRYPT
    or VIRTIO_CRYPTO_CIPHER_DECRYPT then:
    \begin{itemize*}
    \item If VIRTIO_CRYPTO_F_CIPHER_STATELESS_MODE is negotiated, \field{op_flf} is
        struct virtio_crypto_sym_data_flf_stateless, and \field{op_vlf} is struct
        virtio_crypto_sym_data_vlf_stateless.
    \item If VIRTIO_CRYPTO_F_CIPHER_STATELESS_MODE is NOT negotiated, \field{op_flf}
        is struct virtio_crypto_sym_data_flf if VIRTIO_CRYPTO_F_REVISION_1 is negotiated
        and struct virtio_crypto_sym_data_flf is padded to 48 bytes if NOT negotiated,
        and \field{op_vlf} is struct virtio_crypto_sym_data_vlf.
    \end{itemize*}
\item If the the opcode (in \field{header}) is VIRTIO_CRYPTO_HASH:
    \begin{itemize*}
    \item If VIRTIO_CRYPTO_F_HASH_STATELESS_MODE is negotiated, \field{op_flf} is
        struct virtio_crypto_hash_data_flf_stateless, and \field{op_vlf} is struct
        virtio_crypto_hash_data_vlf_stateless.
    \item If VIRTIO_CRYPTO_F_HASH_STATELESS_MODE is NOT negotiated, \field{op_flf}
        is struct virtio_crypto_hash_data_flf if VIRTIO_CRYPTO_F_REVISION_1 is negotiated
        and struct virtio_crypto_hash_data_flf is padded to 48 bytes if NOT negotiated,
        and \field{op_vlf} is struct virtio_crypto_hash_data_vlf.
    \end{itemize*}
\item If the the opcode (in \field{header}) is VIRTIO_CRYPTO_MAC:
    \begin{itemize*}
    \item If VIRTIO_CRYPTO_F_MAC_STATELESS_MODE is negotiated, \field{op_flf} is
        struct virtio_crypto_mac_data_flf_stateless, and \field{op_vlf} is struct
        virtio_crypto_mac_data_vlf_stateless.
    \item If VIRTIO_CRYPTO_F_MAC_STATELESS_MODE is NOT negotiated, \field{op_flf}
        is struct virtio_crypto_mac_data_flf if VIRTIO_CRYPTO_F_REVISION_1 is negotiated
        and struct virtio_crypto_mac_data_flf is padded to 48 bytes if NOT negotiated,
        and \field{op_vlf} is struct virtio_crypto_mac_data_vlf.
    \end{itemize*}
\item If the the opcode (in \field{header}) is VIRTIO_CRYPTO_AEAD_ENCRYPT
    or VIRTIO_CRYPTO_AEAD_DECRYPT then:
    \begin{itemize*}
    \item If VIRTIO_CRYPTO_F_AEAD_STATELESS_MODE is negotiated, \field{op_flf} is
        struct virtio_crypto_aead_data_flf_stateless, and \field{op_vlf} is struct
        virtio_crypto_aead_data_vlf_stateless.
    \item If VIRTIO_CRYPTO_F_AEAD_STATELESS_MODE is NOT negotiated, \field{op_flf}
        is struct virtio_crypto_aead_data_flf if VIRTIO_CRYPTO_F_REVISION_1 is negotiated
        and struct virtio_crypto_aead_data_flf is padded to 48 bytes if NOT negotiated,
        and \field{op_vlf} is struct virtio_crypto_aead_data_vlf.
    \end{itemize*}
\item If the opcode (in \field{header}) is VIRTIO_CRYPTO_AKCIPHER_ENCRYPT, VIRTIO_CRYPTO_AKCIPHER_DECRYPT,
    VIRTIO_CRYPTO_AKCIPHER_SIGN or VIRTIO_CRYPTO_AKCIPHER_VERIFY then:
    \begin{itemize*}
    \item If VIRTIO_CRYPTO_F_AKCIPHER_STATELESS_MODE is negotiated, \field{op_flf} is
        struct virtio_crypto_akcipher_data_flf_statless, and \field{op_vlf} is struct
        virtio_crypto_akcipher_data_vlf_stateless.
    \item If VIRTIO_CRYPTO_F_AKCIPHER_STATELESS_MODE is NOT negotiated, \field{op_flf}
        is struct virtio_crypto_akcipher_data_flf if VIRTIO_CRYPTO_F_REVISION_1 is negotiated
        and struct virtio_crypto_akcipher_data_flf is padded to 48 bytes if NOT negotiated,
        and \field{op_vlf} is struct virtio_crypto_akcipher_data_vlf.
    \end{itemize*}
\end{itemize*}

\field{inhdr} is a unified input header that used to return the status of
the operations, is defined as follows:

\begin{lstlisting}
struct virtio_crypto_inhdr {
    u8 status;
};
\end{lstlisting}

\subsubsection{HASH Service Operation}\label{sec:Device Types / Crypto Device / Device Operation / HASH Service Operation}

Session mode HASH service requests are as follows:

\begin{lstlisting}
struct virtio_crypto_hash_data_flf {
    /* length of source data */
    le32 src_data_len;
    /* hash result length */
    le32 hash_result_len;
};

struct virtio_crypto_hash_data_vlf {
    /* Device read only portion */
    /* Source data */
    u8 src_data[src_data_len];

    /* Device write only portion */
    /* Hash result data */
    u8 hash_result[hash_result_len];
};
\end{lstlisting}

Each data request uses the virtio_crypto_hash_data_flf structure and the
virtio_crypto_hash_data_vlf structure to store information used to run the
HASH operations.

\field{src_data} is the source data that will be processed.
\field{src_data_len} is the length of source data.
\field{hash_result} is the result data and \field{hash_result_len} is the length
of it.

Stateless mode HASH service requests are as follows:

\begin{lstlisting}
struct virtio_crypto_hash_data_flf_stateless {
    struct {
        /* See VIRTIO_CRYPTO_HASH_* above */
        le32 algo;
    } sess_para;

    /* length of source data */
    le32 src_data_len;
    /* hash result length */
    le32 hash_result_len;
    le32 reserved;
};
struct virtio_crypto_hash_data_vlf_stateless {
    /* Device read only portion */
    /* Source data */
    u8 src_data[src_data_len];

    /* Device write only portion */
    /* Hash result data */
    u8 hash_result[hash_result_len];
};
\end{lstlisting}

\drivernormative{\paragraph}{HASH Service Operation}{Device Types / Crypto Device / Device Operation / HASH Service Operation}

\begin{itemize*}
\item If the driver uses the session mode, then the driver MUST set \field{session_id}
    in struct virtio_crypto_op_header to a valid value assigned by the device when the
    session was created.
\item If the VIRTIO_CRYPTO_F_HASH_STATELESS_MODE feature bit is negotiated, 1) if the
    driver uses the stateless mode, then the driver MUST set the \field{flag} field in
    struct virtio_crypto_op_header to ZERO and MUST set the fields in struct
    virtio_crypto_hash_data_flf_stateless.sess_para, 2) if the driver uses the session
    mode, then the driver MUST set the \field{flag} field in struct virtio_crypto_op_header
    to VIRTIO_CRYPTO_FLAG_SESSION_MODE.
\item The driver MUST set \field{opcode} in struct virtio_crypto_op_header to VIRTIO_CRYPTO_HASH.
\end{itemize*}

\devicenormative{\paragraph}{HASH Service Operation}{Device Types / Crypto Device / Device Operation / HASH Service Operation}

\begin{itemize*}
\item The device MUST use the corresponding structure according to the \field{opcode}
    in the data general header.
\item If the VIRTIO_CRYPTO_F_HASH_STATELESS_MODE feature bit is negotiated, the device
    MUST parse \field{flag} field in struct virtio_crypto_op_header in order to decide
    which mode the driver uses.
\item The device MUST copy the results of HASH operations in the hash_result[] if HASH
    operations success.
\item The device MUST set \field{status} in struct virtio_crypto_inhdr to one of the
    following values of enum VIRTIO_CRYPTO_STATUS:
\begin{itemize*}
\item VIRTIO_CRYPTO_OK if the operation success.
\item VIRTIO_CRYPTO_NOTSUPP if the requested algorithm or operation is unsupported.
\item VIRTIO_CRYPTO_INVSESS if the session ID invalid when in session mode.
\item VIRTIO_CRYPTO_ERR if any failure not mentioned above occurs.
\end{itemize*}
\end{itemize*}


\subsubsection{MAC Service Operation}\label{sec:Device Types / Crypto Device / Device Operation / MAC Service Operation}

Session mode MAC service requests are as follows:

\begin{lstlisting}
struct virtio_crypto_mac_data_flf {
    struct virtio_crypto_hash_data_flf hdr;
};

struct virtio_crypto_mac_data_vlf {
    /* Device read only portion */
    /* Source data */
    u8 src_data[src_data_len];

    /* Device write only portion */
    /* Hash result data */
    u8 hash_result[hash_result_len];
};
\end{lstlisting}

Each request uses the virtio_crypto_mac_data_flf structure and the
virtio_crypto_mac_data_vlf structure to store information used to run the
MAC operations.

\field{src_data} is the source data that will be processed.
\field{src_data_len} is the length of source data.
\field{hash_result} is the result data and \field{hash_result_len} is the length
of it.

Stateless mode MAC service requests are as follows:

\begin{lstlisting}
struct virtio_crypto_mac_data_flf_stateless {
    struct {
        /* See VIRTIO_CRYPTO_MAC_* above */
        le32 algo;
        /* length of authenticated key */
        le32 auth_key_len;
    } sess_para;

    /* length of source data */
    le32 src_data_len;
    /* hash result length */
    le32 hash_result_len;
};

struct virtio_crypto_mac_data_vlf_stateless {
    /* Device read only portion */
    /* The authenticated key */
    u8 auth_key[auth_key_len];
    /* Source data */
    u8 src_data[src_data_len];

    /* Device write only portion */
    /* Hash result data */
    u8 hash_result[hash_result_len];
};
\end{lstlisting}

\field{auth_key} is the authenticated key that will be used during the process.
\field{auth_key_len} is the length of the key.

\drivernormative{\paragraph}{MAC Service Operation}{Device Types / Crypto Device / Device Operation / MAC Service Operation}

\begin{itemize*}
\item If the driver uses the session mode, then the driver MUST set \field{session_id}
    in struct virtio_crypto_op_header to a valid value assigned by the device when the
    session was created.
\item If the VIRTIO_CRYPTO_F_MAC_STATELESS_MODE feature bit is negotiated, 1) if the
    driver uses the stateless mode, then the driver MUST set the \field{flag} field
    in struct virtio_crypto_op_header to ZERO and MUST set the fields in struct
    virtio_crypto_mac_data_flf_stateless.sess_para, 2) if the driver uses the session
    mode, then the driver MUST set the \field{flag} field in struct virtio_crypto_op_header
    to VIRTIO_CRYPTO_FLAG_SESSION_MODE.
\item The driver MUST set \field{opcode} in struct virtio_crypto_op_header to VIRTIO_CRYPTO_MAC.
\end{itemize*}

\devicenormative{\paragraph}{MAC Service Operation}{Device Types / Crypto Device / Device Operation / MAC Service Operation}

\begin{itemize*}
\item If the VIRTIO_CRYPTO_F_MAC_STATELESS_MODE feature bit is negotiated, the device
    MUST parse \field{flag} field in struct virtio_crypto_op_header in order to decide
	which mode the driver uses.
\item The device MUST copy the results of MAC operations in the hash_result[] if HASH
    operations success.
\item The device MUST set \field{status} in struct virtio_crypto_inhdr to one of the
    following values of enum VIRTIO_CRYPTO_STATUS:
\begin{itemize*}
\item VIRTIO_CRYPTO_OK if the operation success.
\item VIRTIO_CRYPTO_NOTSUPP if the requested algorithm or operation is unsupported.
\item VIRTIO_CRYPTO_INVSESS if the session ID invalid when in session mode.
\item VIRTIO_CRYPTO_ERR if any failure not mentioned above occurs.
\end{itemize*}
\end{itemize*}

\subsubsection{Symmetric algorithms Operation}\label{sec:Device Types / Crypto Device / Device Operation / Symmetric algorithms Operation}

Session mode CIPHER service requests are as follows:

\begin{lstlisting}
struct virtio_crypto_cipher_data_flf {
    /*
     * Byte Length of valid IV/Counter data pointed to by the below iv data.
     *
     * For block ciphers in CBC or F8 mode, or for Kasumi in F8 mode, or for
     *   SNOW3G in UEA2 mode, this is the length of the IV (which
     *   must be the same as the block length of the cipher).
     * For block ciphers in CTR mode, this is the length of the counter
     *   (which must be the same as the block length of the cipher).
     */
    le32 iv_len;
    /* length of source data */
    le32 src_data_len;
    /* length of destination data */
    le32 dst_data_len;
    le32 padding;
};

struct virtio_crypto_cipher_data_vlf {
    /* Device read only portion */

    /*
     * Initialization Vector or Counter data.
     *
     * For block ciphers in CBC or F8 mode, or for Kasumi in F8 mode, or for
     *   SNOW3G in UEA2 mode, this is the Initialization Vector (IV)
     *   value.
     * For block ciphers in CTR mode, this is the counter.
     * For AES-XTS, this is the 128bit tweak, i, from IEEE Std 1619-2007.
     *
     * The IV/Counter will be updated after every partial cryptographic
     * operation.
     */
    u8 iv[iv_len];
    /* Source data */
    u8 src_data[src_data_len];

    /* Device write only portion */
    /* Destination data */
    u8 dst_data[dst_data_len];
};
\end{lstlisting}

Session mode requests of algorithm chaining are as follows:

\begin{lstlisting}
struct virtio_crypto_alg_chain_data_flf {
    le32 iv_len;
    /* Length of source data */
    le32 src_data_len;
    /* Length of destination data */
    le32 dst_data_len;
    /* Starting point for cipher processing in source data */
    le32 cipher_start_src_offset;
    /* Length of the source data that the cipher will be computed on */
    le32 len_to_cipher;
    /* Starting point for hash processing in source data */
    le32 hash_start_src_offset;
    /* Length of the source data that the hash will be computed on */
    le32 len_to_hash;
    /* Length of the additional auth data */
    le32 aad_len;
    /* Length of the hash result */
    le32 hash_result_len;
    le32 reserved;
};

struct virtio_crypto_alg_chain_data_vlf {
    /* Device read only portion */

    /* Initialization Vector or Counter data */
    u8 iv[iv_len];
    /* Source data */
    u8 src_data[src_data_len];
    /* Additional authenticated data if exists */
    u8 aad[aad_len];

    /* Device write only portion */

    /* Destination data */
    u8 dst_data[dst_data_len];
    /* Hash result data */
    u8 hash_result[hash_result_len];
};
\end{lstlisting}

Session mode requests of symmetric algorithm are as follows:

\begin{lstlisting}
struct virtio_crypto_sym_data_flf {
    /* Device read only portion */

#define VIRTIO_CRYPTO_SYM_DATA_REQ_HDR_SIZE    40
    u8 op_type_flf[VIRTIO_CRYPTO_SYM_DATA_REQ_HDR_SIZE];

    /* See above VIRTIO_CRYPTO_SYM_OP_* */
    le32 op_type;
    le32 padding;
};

struct virtio_crypto_sym_data_vlf {
    u8 op_type_vlf[sym_para_len];
};
\end{lstlisting}

Each request uses the virtio_crypto_sym_data_flf structure and the
virtio_crypto_sym_data_flf structure to store information used to run the
CIPHER operations.

\field{op_type_flf} is the \field{op_type} specific header, it MUST starts
with or be one of the following structures:
\begin{itemize*}
\item struct virtio_crypto_cipher_data_flf
\item struct virtio_crypto_alg_chain_data_flf
\end{itemize*}

The length of \field{op_type_flf} is fixed to 40 bytes, the data of unused
part (if has) will be ignored.

\field{op_type_vlf} is the \field{op_type} specific parameters, it MUST starts
with or be one of the following structures:
\begin{itemize*}
\item struct virtio_crypto_cipher_data_vlf
\item struct virtio_crypto_alg_chain_data_vlf
\end{itemize*}

\field{sym_para_len} is the size of the specific structure used.

Stateless mode CIPHER service requests are as follows:

\begin{lstlisting}
struct virtio_crypto_cipher_data_flf_stateless {
    struct {
        /* See VIRTIO_CRYPTO_CIPHER* above */
        le32 algo;
        /* length of key */
        le32 key_len;

        /* See VIRTIO_CRYPTO_OP_* above */
        le32 op;
    } sess_para;

    /*
     * Byte Length of valid IV/Counter data pointed to by the below iv data.
     */
    le32 iv_len;
    /* length of source data */
    le32 src_data_len;
    /* length of destination data */
    le32 dst_data_len;
};

struct virtio_crypto_cipher_data_vlf_stateless {
    /* Device read only portion */

    /* The cipher key */
    u8 cipher_key[key_len];

    /* Initialization Vector or Counter data. */
    u8 iv[iv_len];
    /* Source data */
    u8 src_data[src_data_len];

    /* Device write only portion */
    /* Destination data */
    u8 dst_data[dst_data_len];
};
\end{lstlisting}

Stateless mode requests of algorithm chaining are as follows:

\begin{lstlisting}
struct virtio_crypto_alg_chain_data_flf_stateless {
    struct {
        /* See VIRTIO_CRYPTO_SYM_ALG_CHAIN_ORDER_* above */
        le32 alg_chain_order;
        /* length of the additional authenticated data in bytes */
        le32 aad_len;

        struct {
            /* See VIRTIO_CRYPTO_CIPHER* above */
            le32 algo;
            /* length of key */
            le32 key_len;
            /* See VIRTIO_CRYPTO_OP_* above */
            le32 op;
        } cipher;

        struct {
            /* See VIRTIO_CRYPTO_HASH_* or VIRTIO_CRYPTO_MAC_* above */
            le32 algo;
            /* length of authenticated key */
            le32 auth_key_len;
            /* See VIRTIO_CRYPTO_SYM_HASH_MODE_* above */
            le32 hash_mode;
        } hash;
    } sess_para;

    le32 iv_len;
    /* Length of source data */
    le32 src_data_len;
    /* Length of destination data */
    le32 dst_data_len;
    /* Starting point for cipher processing in source data */
    le32 cipher_start_src_offset;
    /* Length of the source data that the cipher will be computed on */
    le32 len_to_cipher;
    /* Starting point for hash processing in source data */
    le32 hash_start_src_offset;
    /* Length of the source data that the hash will be computed on */
    le32 len_to_hash;
    /* Length of the additional auth data */
    le32 aad_len;
    /* Length of the hash result */
    le32 hash_result_len;
    le32 reserved;
};

struct virtio_crypto_alg_chain_data_vlf_stateless {
    /* Device read only portion */

    /* The cipher key */
    u8 cipher_key[key_len];
    /* The auth key */
    u8 auth_key[auth_key_len];
    /* Initialization Vector or Counter data */
    u8 iv[iv_len];
    /* Additional authenticated data if exists */
    u8 aad[aad_len];
    /* Source data */
    u8 src_data[src_data_len];

    /* Device write only portion */

    /* Destination data */
    u8 dst_data[dst_data_len];
    /* Hash result data */
    u8 hash_result[hash_result_len];
};
\end{lstlisting}

Stateless mode requests of symmetric algorithm are as follows:

\begin{lstlisting}
struct virtio_crypto_sym_data_flf_stateless {
    /* Device read only portion */
#define VIRTIO_CRYPTO_SYM_DATE_REQ_HDR_STATELESS_SIZE    72
    u8 op_type_flf[VIRTIO_CRYPTO_SYM_DATE_REQ_HDR_STATELESS_SIZE];

    /* Device write only portion */
    /* See above VIRTIO_CRYPTO_SYM_OP_* */
    le32 op_type;
};

struct virtio_crypto_sym_data_vlf_stateless {
    u8 op_type_vlf[sym_para_len];
};
\end{lstlisting}

\field{op_type_flf} is the \field{op_type} specific header, it MUST starts
with or be one of the following structures:
\begin{itemize*}
\item struct virtio_crypto_cipher_data_flf_stateless
\item struct virtio_crypto_alg_chain_data_flf_stateless
\end{itemize*}

The length of \field{op_type_flf} is fixed to 72 bytes, the data of unused
part (if has) will be ignored.

\field{op_type_vlf} is the \field{op_type} specific parameters, it MUST starts
with or be one of the following structures:
\begin{itemize*}
\item struct virtio_crypto_cipher_data_vlf_stateless
\item struct virtio_crypto_alg_chain_data_vlf_stateless
\end{itemize*}

\field{sym_para_len} is the size of the specific structure used.

\drivernormative{\paragraph}{Symmetric algorithms Operation}{Device Types / Crypto Device / Device Operation / Symmetric algorithms Operation}

\begin{itemize*}
\item If the driver uses the session mode, then the driver MUST set \field{session_id}
    in struct virtio_crypto_op_header to a valid value assigned by the device when the
    session was created.
\item If the VIRTIO_CRYPTO_F_CIPHER_STATELESS_MODE feature bit is negotiated, 1) if the
    driver uses the stateless mode, then the driver MUST set the \field{flag} field in
    struct virtio_crypto_op_header to ZERO and MUST set the fields in struct
    virtio_crypto_cipher_data_flf_stateless.sess_para or struct
    virtio_crypto_alg_chain_data_flf_stateless.sess_para, 2) if the driver uses the
    session mode, then the driver MUST set the \field{flag} field in struct
    virtio_crypto_op_header to VIRTIO_CRYPTO_FLAG_SESSION_MODE.
\item The driver MUST set the \field{opcode} field in struct virtio_crypto_op_header
    to VIRTIO_CRYPTO_CIPHER_ENCRYPT or VIRTIO_CRYPTO_CIPHER_DECRYPT.
\item The driver MUST specify the fields of struct virtio_crypto_cipher_data_flf in
    struct virtio_crypto_sym_data_flf and struct virtio_crypto_cipher_data_vlf in
    struct virtio_crypto_sym_data_vlf if the request is based on VIRTIO_CRYPTO_SYM_OP_CIPHER.
\item The driver MUST specify the fields of struct virtio_crypto_alg_chain_data_flf
    in struct virtio_crypto_sym_data_flf and struct virtio_crypto_alg_chain_data_vlf
    in struct virtio_crypto_sym_data_vlf if the request is of the VIRTIO_CRYPTO_SYM_OP_ALGORITHM_CHAINING
    type.
\end{itemize*}

\devicenormative{\paragraph}{Symmetric algorithms Operation}{Device Types / Crypto Device / Device Operation / Symmetric algorithms Operation}

\begin{itemize*}
\item If the VIRTIO_CRYPTO_F_CIPHER_STATELESS_MODE feature bit is negotiated, the device
    MUST parse \field{flag} field in struct virtio_crypto_op_header in order to decide
	which mode the driver uses.
\item The device MUST parse the virtio_crypto_sym_data_req based on the \field{opcode}
    field in general header.
\item The device MUST parse the fields of struct virtio_crypto_cipher_data_flf in
    struct virtio_crypto_sym_data_flf and struct virtio_crypto_cipher_data_vlf in
    struct virtio_crypto_sym_data_vlf if the request is based on VIRTIO_CRYPTO_SYM_OP_CIPHER.
\item The device MUST parse the fields of struct virtio_crypto_alg_chain_data_flf
    in struct virtio_crypto_sym_data_flf and struct virtio_crypto_alg_chain_data_vlf
    in struct virtio_crypto_sym_data_vlf if the request is of the VIRTIO_CRYPTO_SYM_OP_ALGORITHM_CHAINING
    type.
\item The device MUST copy the result of cryptographic operation in the dst_data[] in
    both plain CIPHER mode and algorithms chain mode.
\item The device MUST check the \field{para}.\field{add_len} is bigger than 0 before
    parse the additional authenticated data in plain algorithms chain mode.
\item The device MUST copy the result of HASH/MAC operation in the hash_result[] is
    of the VIRTIO_CRYPTO_SYM_OP_ALGORITHM_CHAINING type.
\item The device MUST set the \field{status} field in struct virtio_crypto_inhdr to
    one of the following values of enum VIRTIO_CRYPTO_STATUS:
\begin{itemize*}
\item VIRTIO_CRYPTO_OK if the operation success.
\item VIRTIO_CRYPTO_NOTSUPP if the requested algorithm or operation is unsupported.
\item VIRTIO_CRYPTO_INVSESS if the session ID is invalid in session mode.
\item VIRTIO_CRYPTO_ERR if failure not mentioned above occurs.
\end{itemize*}
\end{itemize*}

\subsubsection{AEAD Service Operation}\label{sec:Device Types / Crypto Device / Device Operation / AEAD Service Operation}

Session mode requests of symmetric algorithm are as follows:

\begin{lstlisting}
struct virtio_crypto_aead_data_flf {
    /*
     * Byte Length of valid IV data.
     *
     * For GCM mode, this is either 12 (for 96-bit IVs) or 16, in which
     *   case iv points to J0.
     * For CCM mode, this is the length of the nonce, which can be in the
     *   range 7 to 13 inclusive.
     */
    le32 iv_len;
    /* length of additional auth data */
    le32 aad_len;
    /* length of source data */
    le32 src_data_len;
    /* length of dst data, this should be at least src_data_len + tag_len */
    le32 dst_data_len;
    /* Authentication tag length */
    le32 tag_len;
    le32 reserved;
};

struct virtio_crypto_aead_data_vlf {
    /* Device read only portion */

    /*
     * Initialization Vector data.
     *
     * For GCM mode, this is either the IV (if the length is 96 bits) or J0
     *   (for other sizes), where J0 is as defined by NIST SP800-38D.
     *   Regardless of the IV length, a full 16 bytes needs to be allocated.
     * For CCM mode, the first byte is reserved, and the nonce should be
     *   written starting at &iv[1] (to allow space for the implementation
     *   to write in the flags in the first byte).  Note that a full 16 bytes
     *   should be allocated, even though the iv_len field will have
     *   a value less than this.
     *
     * The IV will be updated after every partial cryptographic operation.
     */
    u8 iv[iv_len];
    /* Source data */
    u8 src_data[src_data_len];
    /* Additional authenticated data if exists */
    u8 aad[aad_len];

    /* Device write only portion */
    /* Pointer to output data */
    u8 dst_data[dst_data_len];
};
\end{lstlisting}

Each request uses the virtio_crypto_aead_data_flf structure and the
virtio_crypto_aead_data_flf structure to store information used to run the
AEAD operations.

Stateless mode AEAD service requests are as follows:

\begin{lstlisting}
struct virtio_crypto_aead_data_flf_stateless {
    struct {
        /* See VIRTIO_CRYPTO_AEAD_* above */
        le32 algo;
        /* length of key */
        le32 key_len;
        /* encrypt or decrypt, See above VIRTIO_CRYPTO_OP_* */
        le32 op;
    } sess_para;

    /* Byte Length of valid IV data. */
    le32 iv_len;
    /* Authentication tag length */
    le32 tag_len;
    /* length of additional auth data */
    le32 aad_len;
    /* length of source data */
    le32 src_data_len;
    /* length of dst data, this should be at least src_data_len + tag_len */
    le32 dst_data_len;
};

struct virtio_crypto_aead_data_vlf_stateless {
    /* Device read only portion */

    /* The cipher key */
    u8 key[key_len];
    /* Initialization Vector data. */
    u8 iv[iv_len];
    /* Source data */
    u8 src_data[src_data_len];
    /* Additional authenticated data if exists */
    u8 aad[aad_len];

    /* Device write only portion */
    /* Pointer to output data */
    u8 dst_data[dst_data_len];
};
\end{lstlisting}

\drivernormative{\paragraph}{AEAD Service Operation}{Device Types / Crypto Device / Device Operation / AEAD Service Operation}

\begin{itemize*}
\item If the driver uses the session mode, then the driver MUST set
    \field{session_id} in struct virtio_crypto_op_header to a valid value assigned
    by the device when the session was created.
\item If the VIRTIO_CRYPTO_F_AEAD_STATELESS_MODE feature bit is negotiated, 1) if
    the driver uses the stateless mode, then the driver MUST set the \field{flag}
    field in struct virtio_crypto_op_header to ZERO and MUST set the fields in
    struct virtio_crypto_aead_data_flf_stateless.sess_para, 2) if the driver uses
    the session mode, then the driver MUST set the \field{flag} field in struct
    virtio_crypto_op_header to VIRTIO_CRYPTO_FLAG_SESSION_MODE.
\item The driver MUST set the \field{opcode} field in struct virtio_crypto_op_header
    to VIRTIO_CRYPTO_AEAD_ENCRYPT or VIRTIO_CRYPTO_AEAD_DECRYPT.
\end{itemize*}

\devicenormative{\paragraph}{AEAD Service Operation}{Device Types / Crypto Device / Device Operation / AEAD Service Operation}

\begin{itemize*}
\item If the VIRTIO_CRYPTO_F_AEAD_STATELESS_MODE feature bit is negotiated, the
    device MUST parse the virtio_crypto_aead_data_vlf_stateless based on the \field{opcode}
	field in general header.
\item The device MUST copy the result of cryptographic operation in the dst_data[].
\item The device MUST copy the authentication tag in the dst_data[] offset the cipher result.
\item The device MUST set the \field{status} field in struct virtio_crypto_inhdr to
    one of the following values of enum VIRTIO_CRYPTO_STATUS:
\item When the \field{opcode} field is VIRTIO_CRYPTO_AEAD_DECRYPT, the device MUST
    verify and return the verification result to the driver.
\begin{itemize*}
\item VIRTIO_CRYPTO_OK if the operation success.
\item VIRTIO_CRYPTO_NOTSUPP if the requested algorithm or operation is unsupported.
\item VIRTIO_CRYPTO_BADMSG if the verification result is incorrect.
\item VIRTIO_CRYPTO_INVSESS if the session ID invalid when in session mode.
\item VIRTIO_CRYPTO_ERR if any failure not mentioned above occurs.
\end{itemize*}
\end{itemize*}

\subsubsection{AKCIPHER Service Operation}\label{sec:Device Types / Crypto Device / Device Operation / AKCIPHER Service Operation}

Session mode AKCIPHER requests are as follows:

\begin{lstlisting}
struct virtio_crypto_akcipher_data_flf {
    /* length of source data */
    le32 src_data_len;
    /* length of dst data */
    le32 dst_data_len;
};

struct virtio_crypto_akcipher_data_vlf {
    /* Device read only portion */
    /* Source data */
    u8 src_data[src_data_len];

    /* Device write only portion */
    /* Pointer to output data */
    u8 dst_data[dst_data_len];
};
\end{lstlisting}

Each data request uses the virtio_crypto_akcipher_flf structure and the virtio_crypto_akcipher_data_vlf
structure to store information used to run the AKCIPHER operations.

For encryption, decryption, and signing:
\field{src_data} is the source data that will be processed, note that for signing operations,
src_data stores the data to be signed, which usually is the digest of some data rather than the
data itself.
\field{src_data_len} is the length of source data.
\field{dst_result} is the result data and \field{dst_data_len} is the length of it. Note that the
length of the result is not always exactly equal to dst_data_len, the driver needs to check how
many bytes the device has written and calculate the actual length of the result.

For verification:
\field{src_data_len} refers to the length of the signature, and \field{dst_data_len} refers to
the length of signed data, where the signed data is usually the digest of some data.
\field{src_data} is spliced by the signature and the signed data, the src_data with the lower
address stores the signature, and the higher address stores the signed data.
\field{dst_data} is always empty for verification.

Different algorithms have different signature formats.
For the RSA algorithm, the result is determined by the padding algorithm specified by
\field{padding_algo} in structure virtio_crypto_rsa_session_para.

For the ECDSA algorithm, the signature is composed of the following
ASN.1 structure (see \hyperref[intro:rfc3279]{RFC3279})
and MUST be DER encoded (see \hyperref[intro:rfc6025]{rfc6025}).

\begin{lstlisting}
Ecdsa-Sig-Value ::= SEQUENCE {
    r INTEGER,
    s INTEGER
}
\end{lstlisting}

Stateless mode AKCIPHER service requests are as follows:
\begin{lstlisting}
struct virtio_crypto_akcipher_data_flf_stateless {
    struct {
        /* See VIRTIO_CYRPTO_AKCIPHER* above */
        le32 algo;
        /* See VIRTIO_CRYPTO_AKCIPHER_KEY_TYPE_* above */
        le32 key_type;
        /* length of key */
        le32 key_len;

        /* algothrim specific parameters described above */
        union {
            struct virtio_crypto_rsa_session_para rsa;
            struct virtio_crypto_ecdsa_session_para ecdsa;
        } u;
    } sess_para;

    /* length of source data */
    le32 src_data_len;
    /* length of destination data */
    le32 dst_data_len;
};

struct virtio_crypto_akcipher_data_vlf_stateless {
    /* Device read only portion */
    u8 akcipher_key[key_len];

    /* Source data */
    u8 src_data[src_data_len];

    /* Device write only portion */
    u8 dst_data[dst_data_len];
};
\end{lstlisting}

In stateless mode, the format of key and signature, the meaning of src_data and dst_data, are all the same
with session mode.

\drivernormative{\paragraph}{AKCIPHER Service Operation}{Device Types / Crypto Device / Device Operation / AKCIPHER Service Operation}

\begin{itemize*}
\item If the driver uses the session mode, then the driver MUST set
    \field{session_id} in struct virtio_crypto_op_header to a valid
    value assigned by the device when the session was created.
\item If the VIRTIO_CRYPTO_F_AKCIPHER_STATELESS_MODE feature bit is negotiated, 1) if the
    driver uses the stateless mode, then the driver MUST set the \field{flag} field in
    struct virtio_crypto_op_header to ZERO and MUST set the fields in struct
    virtio_crypto_akcipher_flf_stateless.sess_para, 2) if the driver uses the session
    mode, then the driver MUST set the \field{flag} field in struct virtio_crypto_op_header
    to VIRTIO_CRYPTO_FLAG_SESSION_MODE.
\item The driver MUST set the \field{opcode} field in struct virtio_crypto_op_header
    to one of VIRTIO_CRYPTO_AKCIPHER_ENCRYPT, VIRTIO_CRYPTO_AKCIPHER_DESTROY_SESSION,
    VIRTIO_CRYPTO_AKCIPHER_SIGN, and VIRTIO_CRYPTO_AKCIPHER_VERIFY.
\end{itemize*}

\devicenormative{\paragraph}{AKCIPHER Service Operation}{Device Types / Crypto Device / Device Operation / AKCIPHER Service Operation}

\begin{itemize*}
\item If the VIRTIO_CRYPTO_F_AKCIPHER_STATELESS_MODE feature bit is negotiated, the
    device MUST parse the virtio_crypto_akcipher_data_vlf_stateless based on the \field{opcode}
    field in general header.
\item The device MUST copy the result of cryptographic operation in the dst_data[].
\item The device MUST set the \field{status} field in struct virtio_crypto_inhdr to
    one of the following values of enum VIRTIO_CRYPTO_STATUS:
\begin{itemize*}
\item VIRTIO_CRYPTO_OK if the operation success.
\item VIRTIO_CRYPTO_NOTSUPP if the requested algorithm or operation is unsupported.
\item VIRTIO_CRYPTO_BADMSG if the verification result is incorrect.
\item VIRTIO_CRYPTO_INVSESS if the session ID invalid when in session mode.
\item VIRTIO_CRYPTO_KEY_REJECTED if the signature verification failed.
\item VIRTIO_CRYPTO_ERR if any failure not mentioned above occurs.
\end{itemize*}
\end{itemize*}

\section{Crypto Device}\label{sec:Device Types / Crypto Device}

The virtio crypto device is a virtual cryptography device as well as a
virtual cryptographic accelerator. The virtio crypto device provides the
following crypto services: CIPHER, MAC, HASH, AEAD and AKCIPHER. Virtio crypto
devices have a single control queue and at least one data queue. Crypto
operation requests are placed into a data queue, and serviced by the
device. Some crypto operation requests are only valid in the context of a
session. The role of the control queue is facilitating control operation
requests. Sessions management is realized with control operation
requests.

\subsection{Device ID}\label{sec:Device Types / Crypto Device / Device ID}

20

\subsection{Virtqueues}\label{sec:Device Types / Crypto Device / Virtqueues}

\begin{description}
\item[0] dataq1
\item[\ldots]
\item[N-1] dataqN
\item[N] controlq
\end{description}

N is set by \field{max_dataqueues}.

\subsection{Feature bits}\label{sec:Device Types / Crypto Device / Feature bits}

\begin{description}
\item VIRTIO_CRYPTO_F_REVISION_1 (0) revision 1. Revision 1 has a specific
    request format and other enhancements (which result in some additional
    requirements).
\item VIRTIO_CRYPTO_F_CIPHER_STATELESS_MODE (1) stateless mode requests are
    supported by the CIPHER service.
\item VIRTIO_CRYPTO_F_HASH_STATELESS_MODE (2) stateless mode requests are
    supported by the HASH service.
\item VIRTIO_CRYPTO_F_MAC_STATELESS_MODE (3) stateless mode requests are
    supported by the MAC service.
\item VIRTIO_CRYPTO_F_AEAD_STATELESS_MODE (4) stateless mode requests are
    supported by the AEAD service.
\item VIRTIO_CRYPTO_F_AKCIPHER_STATELESS_MODE (5) stateless mode requests are
    supported by the AKCIPHER service.
\end{description}


\subsubsection{Feature bit requirements}\label{sec:Device Types / Crypto Device / Feature bit requirements}

Some crypto feature bits require other crypto feature bits
(see \ref{drivernormative:Basic Facilities of a Virtio Device / Feature Bits}):

\begin{description}
\item[VIRTIO_CRYPTO_F_CIPHER_STATELESS_MODE] Requires VIRTIO_CRYPTO_F_REVISION_1.
\item[VIRTIO_CRYPTO_F_HASH_STATELESS_MODE] Requires VIRTIO_CRYPTO_F_REVISION_1.
\item[VIRTIO_CRYPTO_F_MAC_STATELESS_MODE] Requires VIRTIO_CRYPTO_F_REVISION_1.
\item[VIRTIO_CRYPTO_F_AEAD_STATELESS_MODE] Requires VIRTIO_CRYPTO_F_REVISION_1.
\item[VIRTIO_CRYPTO_F_AKCIPHER_STATELESS_MODE] Requires VIRTIO_CRYPTO_F_REVISION_1.
\end{description}

\subsection{Supported crypto services}\label{sec:Device Types / Crypto Device / Supported crypto services}

The following crypto services are defined:

\begin{lstlisting}
/* CIPHER (Symmetric Key Cipher) service */
#define VIRTIO_CRYPTO_SERVICE_CIPHER 0
/* HASH service */
#define VIRTIO_CRYPTO_SERVICE_HASH   1
/* MAC (Message Authentication Codes) service */
#define VIRTIO_CRYPTO_SERVICE_MAC    2
/* AEAD (Authenticated Encryption with Associated Data) service */
#define VIRTIO_CRYPTO_SERVICE_AEAD   3
/* AKCIPHER (Asymmetric Key Cipher) service */
#define VIRTIO_CRYPTO_SERVICE_AKCIPHER 4
\end{lstlisting}

The above constants designate bits used to indicate the which of crypto services are
offered by the device as described in, see \ref{sec:Device Types / Crypto Device / Device configuration layout}.

\subsubsection{CIPHER services}\label{sec:Device Types / Crypto Device / Supported crypto services / CIPHER services}

The following CIPHER algorithms are defined:

\begin{lstlisting}
#define VIRTIO_CRYPTO_NO_CIPHER                 0
#define VIRTIO_CRYPTO_CIPHER_ARC4               1
#define VIRTIO_CRYPTO_CIPHER_AES_ECB            2
#define VIRTIO_CRYPTO_CIPHER_AES_CBC            3
#define VIRTIO_CRYPTO_CIPHER_AES_CTR            4
#define VIRTIO_CRYPTO_CIPHER_DES_ECB            5
#define VIRTIO_CRYPTO_CIPHER_DES_CBC            6
#define VIRTIO_CRYPTO_CIPHER_3DES_ECB           7
#define VIRTIO_CRYPTO_CIPHER_3DES_CBC           8
#define VIRTIO_CRYPTO_CIPHER_3DES_CTR           9
#define VIRTIO_CRYPTO_CIPHER_KASUMI_F8          10
#define VIRTIO_CRYPTO_CIPHER_SNOW3G_UEA2        11
#define VIRTIO_CRYPTO_CIPHER_AES_F8             12
#define VIRTIO_CRYPTO_CIPHER_AES_XTS            13
#define VIRTIO_CRYPTO_CIPHER_ZUC_EEA3           14
\end{lstlisting}

The above constants have two usages:
\begin{enumerate}
\item As bit numbers, used to tell the driver which CIPHER algorithms
are supported by the device, see \ref{sec:Device Types / Crypto Device / Device configuration layout}.
\item As values, used to designate the algorithm in (CIPHER type) crypto
operation requests, see \ref{sec:Device Types / Crypto Device / Device Operation / Control Virtqueue / Session operation}.
\end{enumerate}

\subsubsection{HASH services}\label{sec:Device Types / Crypto Device / Supported crypto services / HASH services}

The following HASH algorithms are defined:

\begin{lstlisting}
#define VIRTIO_CRYPTO_NO_HASH            0
#define VIRTIO_CRYPTO_HASH_MD5           1
#define VIRTIO_CRYPTO_HASH_SHA1          2
#define VIRTIO_CRYPTO_HASH_SHA_224       3
#define VIRTIO_CRYPTO_HASH_SHA_256       4
#define VIRTIO_CRYPTO_HASH_SHA_384       5
#define VIRTIO_CRYPTO_HASH_SHA_512       6
#define VIRTIO_CRYPTO_HASH_SHA3_224      7
#define VIRTIO_CRYPTO_HASH_SHA3_256      8
#define VIRTIO_CRYPTO_HASH_SHA3_384      9
#define VIRTIO_CRYPTO_HASH_SHA3_512      10
#define VIRTIO_CRYPTO_HASH_SHA3_SHAKE128      11
#define VIRTIO_CRYPTO_HASH_SHA3_SHAKE256      12
\end{lstlisting}

The above constants have two usages:
\begin{enumerate}
\item As bit numbers, used to tell the driver which HASH algorithms
are supported by the device, see \ref{sec:Device Types / Crypto Device / Device configuration layout}.
\item As values, used to designate the algorithm in (HASH type) crypto
operation requires, see \ref{sec:Device Types / Crypto Device / Device Operation / Control Virtqueue / Session operation}.
\end{enumerate}

\subsubsection{MAC services}\label{sec:Device Types / Crypto Device / Supported crypto services / MAC services}

The following MAC algorithms are defined:

\begin{lstlisting}
#define VIRTIO_CRYPTO_NO_MAC                       0
#define VIRTIO_CRYPTO_MAC_HMAC_MD5                 1
#define VIRTIO_CRYPTO_MAC_HMAC_SHA1                2
#define VIRTIO_CRYPTO_MAC_HMAC_SHA_224             3
#define VIRTIO_CRYPTO_MAC_HMAC_SHA_256             4
#define VIRTIO_CRYPTO_MAC_HMAC_SHA_384             5
#define VIRTIO_CRYPTO_MAC_HMAC_SHA_512             6
#define VIRTIO_CRYPTO_MAC_CMAC_3DES                25
#define VIRTIO_CRYPTO_MAC_CMAC_AES                 26
#define VIRTIO_CRYPTO_MAC_KASUMI_F9                27
#define VIRTIO_CRYPTO_MAC_SNOW3G_UIA2              28
#define VIRTIO_CRYPTO_MAC_GMAC_AES                 41
#define VIRTIO_CRYPTO_MAC_GMAC_TWOFISH             42
#define VIRTIO_CRYPTO_MAC_CBCMAC_AES               49
#define VIRTIO_CRYPTO_MAC_CBCMAC_KASUMI_F9         50
#define VIRTIO_CRYPTO_MAC_XCBC_AES                 53
#define VIRTIO_CRYPTO_MAC_ZUC_EIA3                 54
\end{lstlisting}

The above constants have two usages:
\begin{enumerate}
\item As bit numbers, used to tell the driver which MAC algorithms
are supported by the device, see \ref{sec:Device Types / Crypto Device / Device configuration layout}.
\item As values, used to designate the algorithm in (MAC type) crypto
operation requests, see \ref{sec:Device Types / Crypto Device / Device Operation / Control Virtqueue / Session operation}.
\end{enumerate}

\subsubsection{AEAD services}\label{sec:Device Types / Crypto Device / Supported crypto services / AEAD services}

The following AEAD algorithms are defined:

\begin{lstlisting}
#define VIRTIO_CRYPTO_NO_AEAD     0
#define VIRTIO_CRYPTO_AEAD_GCM    1
#define VIRTIO_CRYPTO_AEAD_CCM    2
#define VIRTIO_CRYPTO_AEAD_CHACHA20_POLY1305  3
\end{lstlisting}

The above constants have two usages:
\begin{enumerate}
\item As bit numbers, used to tell the driver which AEAD algorithms
are supported by the device, see \ref{sec:Device Types / Crypto Device / Device configuration layout}.
\item As values, used to designate the algorithm in (DEAD type) crypto
operation requests, see \ref{sec:Device Types / Crypto Device / Device Operation / Control Virtqueue / Session operation}.
\end{enumerate}

\subsubsection{AKCIPHER services}\label{sec: Device Types / Crypto Device / Supported crypto services / AKCIPHER services}

The following AKCIPHER algorithms are defined:
\begin{lstlisting}
#define VIRTIO_CRYPTO_NO_AKCIPHER 0
#define VIRTIO_CRYPTO_AKCIPHER_RSA   1
#define VIRTIO_CRYPTO_AKCIPHER_ECDSA 2
\end{lstlisting}

The above constants have two usages:
\begin{enumerate}
\item As bit numbers, used to tell the driver which AKCIPHER algorithms
are supported by the device, see \ref{sec:Device Types / Crypto Device / Device configuration layout}.
\item As values, used to designate the algorithm in asymmetric crypto operation requests,
see \ref{sec:Device Types / Crypto Device / Device Operation / Control Virtqueue / Session operation}.
\end{enumerate}


\subsection{Device configuration layout}\label{sec:Device Types / Crypto Device / Device configuration layout}

Crypto device configuration uses the following layout structure:

\begin{lstlisting}
struct virtio_crypto_config {
    le32 status;
    le32 max_dataqueues;
    le32 crypto_services;
    /* Detailed algorithms mask */
    le32 cipher_algo_l;
    le32 cipher_algo_h;
    le32 hash_algo;
    le32 mac_algo_l;
    le32 mac_algo_h;
    le32 aead_algo;
    /* Maximum length of cipher key in bytes */
    le32 max_cipher_key_len;
    /* Maximum length of authenticated key in bytes */
    le32 max_auth_key_len;
    le32 akcipher_algo;
    /* Maximum size of each crypto request's content in bytes */
    le64 max_size;
};
\end{lstlisting}

\begin{description}
\item Currently, only one \field{status} bit is defined: VIRTIO_CRYPTO_S_HW_READY
    set indicates that the device is ready to process requests, this bit is read-only
    for the driver
\begin{lstlisting}
#define VIRTIO_CRYPTO_S_HW_READY  (1 << 0)
\end{lstlisting}

\item [\field{max_dataqueues}] is the maximum number of data virtqueues that can
    be configured by the device. The driver MAY use only one data queue, or it
    can use more to achieve better performance.

\item [\field{crypto_services}] crypto service offered, see \ref{sec:Device Types / Crypto Device / Supported crypto services}.

\item [\field{cipher_algo_l}] CIPHER algorithms bits 0-31, see \ref{sec:Device Types / Crypto Device / Supported crypto services  / CIPHER services}.

\item [\field{cipher_algo_h}] CIPHER algorithms bits 32-63, see \ref{sec:Device Types / Crypto Device / Supported crypto services  / CIPHER services}.

\item [\field{hash_algo}] HASH algorithms bits, see \ref{sec:Device Types / Crypto Device / Supported crypto services  / HASH services}.

\item [\field{mac_algo_l}] MAC algorithms bits 0-31, see \ref{sec:Device Types / Crypto Device / Supported crypto services  / MAC services}.

\item [\field{mac_algo_h}] MAC algorithms bits 32-63, see \ref{sec:Device Types / Crypto Device / Supported crypto services  / MAC services}.

\item [\field{aead_algo}] AEAD algorithms bits, see \ref{sec:Device Types / Crypto Device / Supported crypto services  / AEAD services}.

\item [\field{max_cipher_key_len}] is the maximum length of cipher key supported by the device.

\item [\field{max_auth_key_len}] is the maximum length of authenticated key supported by the device.

\item [\field{akcipher_algo}] AKCIPHER algorithms bit 0-31, see \ref{sec: Device Types / Crypto Device / Supported crypto services / AKCIPHER services}.

\item [\field{max_size}] is the maximum size of the variable-length parameters of
    data operation of each crypto request's content supported by the device.
\end{description}

\begin{note}
Unless explicitly stated otherwise all lengths and sizes are in bytes.
\end{note}

\devicenormative{\subsubsection}{Device configuration layout}{Device Types / Crypto Device / Device configuration layout}

\begin{itemize*}
\item The device MUST set \field{max_dataqueues} to between 1 and 65535 inclusive.
\item The device MUST set the \field{status} with valid flags, undefined flags MUST NOT be set.
\item The device MUST accept and handle requests after \field{status} is set to VIRTIO_CRYPTO_S_HW_READY.
\item The device MUST set \field{crypto_services} based on the crypto services the device offers.
\item The device MUST set detailed algorithms masks for each service advertised by \field{crypto_services}.
    The device MUST NOT set the not defined algorithms bits.
\item The device MUST set \field{max_size} to show the maximum size of crypto request the device supports.
\item The device MUST set \field{max_cipher_key_len} to show the maximum length of cipher key if the
    device supports CIPHER service.
\item The device MUST set \field{max_auth_key_len} to show the maximum length of authenticated key if
    the device supports MAC service.
\end{itemize*}

\drivernormative{\subsubsection}{Device configuration layout}{Device Types / Crypto Device / Device configuration layout}

\begin{itemize*}
\item The driver MUST read the \field{status} from the bottom bit of status to check whether the
    VIRTIO_CRYPTO_S_HW_READY is set, and the driver MUST reread it after device reset.
\item The driver MUST NOT transmit any requests to the device if the VIRTIO_CRYPTO_S_HW_READY is not set.
\item The driver MUST read \field{max_dataqueues} field to discover the number of data queues the device supports.
\item The driver MUST read \field{crypto_services} field to discover which services the device is able to offer.
\item The driver SHOULD ignore the not defined algorithms bits.
\item The driver MUST read the detailed algorithms fields based on \field{crypto_services} field.
\item The driver SHOULD read \field{max_size} to discover the maximum size of the variable-length
    parameters of data operation of the crypto request's content the device supports and MUST
    guarantee that the size of each crypto request's content is within the \field{max_size}, otherwise
    the request will fail and the driver MUST reset the device.
\item The driver SHOULD read \field{max_cipher_key_len} to discover the maximum length of cipher key
    the device supports and MUST guarantee that the \field{key_len} (CIPHER service or AEAD service) is within
    the \field{max_cipher_key_len} of the device configuration, otherwise the request will fail.
\item The driver SHOULD read \field{max_auth_key_len} to discover the maximum length of authenticated
    key the device supports and MUST guarantee that the \field{auth_key_len} (MAC service) is within the
    \field{max_auth_key_len} of the device configuration, otherwise the request will fail.
\end{itemize*}

\subsection{Device Initialization}\label{sec:Device Types / Crypto Device / Device Initialization}

\drivernormative{\subsubsection}{Device Initialization}{Device Types / Crypto Device / Device Initialization}

\begin{itemize*}
\item The driver MUST configure and initialize all virtqueues.
\item The driver MUST read the supported crypto services from bits of \field{crypto_services}.
\item The driver MUST read the supported algorithms based on \field{crypto_services} field.
\end{itemize*}

\subsection{Device Operation}\label{sec:Device Types / Crypto Device / Device Operation}

The operation of a virtio crypto device is driven by requests placed on the virtqueues.
Requests consist of a queue-type specific header (specifying among others the operation)
and an operation specific payload.

If VIRTIO_CRYPTO_F_REVISION_1 is negotiated the device may support both session mode
(See \ref{sec:Device Types / Crypto Device / Device Operation / Control Virtqueue / Session operation})
and stateless mode operation requests.
In stateless mode all operation parameters are supplied as a part of each request,
while in session mode, some or all operation parameters are managed within the
session. Stateless mode is guarded by feature bits 0-4 on a service level. If
stateless mode is negotiated for a service, the service accepts both session
mode and stateless requests; otherwise stateless mode requests are rejected
(via operation status).

\subsubsection{Operation Status}\label{sec:Device Types / Crypto Device / Device Operation / Operation status}
The device MUST return a status code as part of the operation (both session
operation and service operation) result. The valid operation status as follows:

\begin{lstlisting}
enum VIRTIO_CRYPTO_STATUS {
    VIRTIO_CRYPTO_OK = 0,
    VIRTIO_CRYPTO_ERR = 1,
    VIRTIO_CRYPTO_BADMSG = 2,
    VIRTIO_CRYPTO_NOTSUPP = 3,
    VIRTIO_CRYPTO_INVSESS = 4,
    VIRTIO_CRYPTO_NOSPC = 5,
    VIRTIO_CRYPTO_KEY_REJECTED = 6,
    VIRTIO_CRYPTO_MAX
};
\end{lstlisting}

\begin{itemize*}
\item VIRTIO_CRYPTO_OK: success.
\item VIRTIO_CRYPTO_BADMSG: authentication failed (only when AEAD decryption).
\item VIRTIO_CRYPTO_NOTSUPP: operation or algorithm is unsupported.
\item VIRTIO_CRYPTO_INVSESS: invalid session ID when executing crypto operations.
\item VIRTIO_CRYPTO_NOSPC: no free session ID (only when the VIRTIO_CRYPTO_F_REVISION_1
    feature bit is negotiated).
\item VIRTIO_CRYPTO_KEY_REJECTED: signature verification failed (only when AKCIPHER verification).
\item VIRTIO_CRYPTO_ERR: any failure not mentioned above occurs.
\end{itemize*}

\subsubsection{Control Virtqueue}\label{sec:Device Types / Crypto Device / Device Operation / Control Virtqueue}

The driver uses the control virtqueue to send control commands to the
device, such as session operations (See \ref{sec:Device Types / Crypto Device / Device
Operation / Control Virtqueue / Session operation}).

The header for controlq is of the following form:
\begin{lstlisting}
#define VIRTIO_CRYPTO_OPCODE(service, op)   (((service) << 8) | (op))

struct virtio_crypto_ctrl_header {
#define VIRTIO_CRYPTO_CIPHER_CREATE_SESSION \
       VIRTIO_CRYPTO_OPCODE(VIRTIO_CRYPTO_SERVICE_CIPHER, 0x02)
#define VIRTIO_CRYPTO_CIPHER_DESTROY_SESSION \
       VIRTIO_CRYPTO_OPCODE(VIRTIO_CRYPTO_SERVICE_CIPHER, 0x03)
#define VIRTIO_CRYPTO_HASH_CREATE_SESSION \
       VIRTIO_CRYPTO_OPCODE(VIRTIO_CRYPTO_SERVICE_HASH, 0x02)
#define VIRTIO_CRYPTO_HASH_DESTROY_SESSION \
       VIRTIO_CRYPTO_OPCODE(VIRTIO_CRYPTO_SERVICE_HASH, 0x03)
#define VIRTIO_CRYPTO_MAC_CREATE_SESSION \
       VIRTIO_CRYPTO_OPCODE(VIRTIO_CRYPTO_SERVICE_MAC, 0x02)
#define VIRTIO_CRYPTO_MAC_DESTROY_SESSION \
       VIRTIO_CRYPTO_OPCODE(VIRTIO_CRYPTO_SERVICE_MAC, 0x03)
#define VIRTIO_CRYPTO_AEAD_CREATE_SESSION \
       VIRTIO_CRYPTO_OPCODE(VIRTIO_CRYPTO_SERVICE_AEAD, 0x02)
#define VIRTIO_CRYPTO_AEAD_DESTROY_SESSION \
       VIRTIO_CRYPTO_OPCODE(VIRTIO_CRYPTO_SERVICE_AEAD, 0x03)
#define VIRTIO_CRYPTO_AKCIPHER_CREATE_SESSION \
       VIRTIO_CRYPTO_OPCODE(VIRTIO_CRYPTO_SERVICE_AKCIPHER, 0x04)
#define VIRTIO_CRYPTO_AKCIPHER_DESTROY_SESSION \
       VIRTIO_CRYPTO_OPCDE(VIRTIO_CRYPTO_SERVICE_AKCIPHER, 0x05)
    le32 opcode;
    /* algo should be service-specific algorithms */
    le32 algo;
    le32 flag;
    le32 reserved;
};
\end{lstlisting}

The controlq request is composed of four parts:
\begin{lstlisting}
struct virtio_crypto_op_ctrl_req {
    /* Device read only portion */

    struct virtio_crypto_ctrl_header header;

#define VIRTIO_CRYPTO_CTRLQ_OP_SPEC_HDR_LEGACY 56
    /* fixed length fields, opcode specific */
    u8 op_flf[flf_len];

    /* variable length fields, opcode specific */
    u8 op_vlf[vlf_len];

    /* Device write only portion */

    /* op result or completion status */
    u8 op_outcome[outcome_len];
};
\end{lstlisting}

\field{header} is a general header (see above).

\field{op_flf} is the opcode (in \field{header}) specific fixed-length parameters.

\field{flf_len} depends on the VIRTIO_CRYPTO_F_REVISION_1 feature bit (see below).

\field{op_vlf} is the opcode (in \field{header}) specific variable-length parameters.

\field{vlf_len} is the size of the specific structure used.
\begin{note}
The \field{vlf_len} of session-destroy operation and the hash-session-create
operation is ZERO.
\end{note}

\begin{itemize*}
\item If the opcode (in \field{header}) is VIRTIO_CRYPTO_CIPHER_CREATE_SESSION
    then \field{op_flf} is struct virtio_crypto_sym_create_session_flf if
    VIRTIO_CRYPTO_F_REVISION_1 is negotiated and struct virtio_crypto_sym_create_session_flf is
    padded to 56 bytes if NOT negotiated, and \field{op_vlf} is struct
    virtio_crypto_sym_create_session_vlf.
\item If the opcode (in \field{header}) is VIRTIO_CRYPTO_HASH_CREATE_SESSION
    then \field{op_flf} is struct virtio_crypto_hash_create_session_flf if
    VIRTIO_CRYPTO_F_REVISION_1 is negotiated and struct virtio_crypto_hash_create_session_flf is
    padded to 56 bytes if NOT negotiated.
\item If the opcode (in \field{header}) is VIRTIO_CRYPTO_MAC_CREATE_SESSION
    then \field{op_flf} is struct virtio_crypto_mac_create_session_flf if
    VIRTIO_CRYPTO_F_REVISION_1 is negotiated and struct virtio_crypto_mac_create_session_flf is
    padded to 56 bytes if NOT negotiated, and \field{op_vlf} is struct
    virtio_crypto_mac_create_session_vlf.
\item If the opcode (in \field{header}) is VIRTIO_CRYPTO_AEAD_CREATE_SESSION
    then \field{op_flf} is struct virtio_crypto_aead_create_session_flf if
    VIRTIO_CRYPTO_F_REVISION_1 is negotiated and struct virtio_crypto_aead_create_session_flf is
    padded to 56 bytes if NOT negotiated, and \field{op_vlf} is struct
    virtio_crypto_aead_create_session_vlf.
\item If the opcode (in \field{header}) is VIRTIO_CRYPTO_AKCIPHER_CREATE_SESSION
    then \field{op_flf} is struct virtio_crypto_akcipher_create_session_flf if
    VIRTIO_CRYPTO_F_REVISION_1 is negotiated and struct virtio_crypto_akcipher_create_session_flf is
    padded to 56 bytes if NOT negotiated, and \field{op_vlf} is struct
    virtio_crypto_akcipher_create_session_vlf.
\item If the opcode (in \field{header}) is VIRTIO_CRYPTO_CIPHER_DESTROY_SESSION
    or VIRTIO_CRYPTO_HASH_DESTROY_SESSION or VIRTIO_CRYPTO_MAC_DESTROY_SESSION or
    VIRTIO_CRYPTO_AEAD_DESTROY_SESSION then \field{op_flf} is struct
    virtio_crypto_destroy_session_flf if VIRTIO_CRYPTO_F_REVISION_1 is negotiated and
    struct virtio_crypto_destroy_session_flf is padded to 56 bytes if NOT negotiated.
\end{itemize*}

\field{op_outcome} stores the result of operation and must be struct
virtio_crypto_destroy_session_input for destroy session or
struct virtio_crypto_create_session_input for create session.

\field{outcome_len} is the size of the structure used.


\paragraph{Session operation}\label{sec:Device Types / Crypto Device / Device
Operation / Control Virtqueue / Session operation}

The session is a handle which describes the cryptographic parameters to be
applied to a number of buffers.

The following structure stores the result of session creation set by the device:

\begin{lstlisting}
struct virtio_crypto_create_session_input {
    le64 session_id;
    le32 status;
    le32 padding;
};
\end{lstlisting}

A request to destroy a session includes the following information:

\begin{lstlisting}
struct virtio_crypto_destroy_session_flf {
    /* Device read only portion */
    le64  session_id;
};

struct virtio_crypto_destroy_session_input {
    /* Device write only portion */
    u8  status;
};
\end{lstlisting}


\subparagraph{Session operation: HASH session}\label{sec:Device Types / Crypto Device / Device
Operation / Control Virtqueue / Session operation / Session operation: HASH session}

The fixed-length parameters of HASH session requests is as follows:

\begin{lstlisting}
struct virtio_crypto_hash_create_session_flf {
    /* Device read only portion */

    /* See VIRTIO_CRYPTO_HASH_* above */
    le32 algo;
    /* hash result length */
    le32 hash_result_len;
};
\end{lstlisting}


\subparagraph{Session operation: MAC session}\label{sec:Device Types / Crypto Device / Device
Operation / Control Virtqueue / Session operation / Session operation: MAC session}

The fixed-length and the variable-length parameters of MAC session requests are as follows:

\begin{lstlisting}
struct virtio_crypto_mac_create_session_flf {
    /* Device read only portion */

    /* See VIRTIO_CRYPTO_MAC_* above */
    le32 algo;
    /* hash result length */
    le32 hash_result_len;
    /* length of authenticated key */
    le32 auth_key_len;
    le32 padding;
};

struct virtio_crypto_mac_create_session_vlf {
    /* Device read only portion */

    /* The authenticated key */
    u8 auth_key[auth_key_len];
};
\end{lstlisting}

The length of \field{auth_key} is specified in \field{auth_key_len} in the struct
virtio_crypto_mac_create_session_flf.


\subparagraph{Session operation: Symmetric algorithms session}\label{sec:Device Types / Crypto Device / Device
Operation / Control Virtqueue / Session operation / Session operation: Symmetric algorithms session}

The request of symmetric session could be the CIPHER algorithms request
or the chain algorithms (chaining CIPHER and HASH/MAC) request.

The fixed-length and the variable-length parameters of CIPHER session requests are as follows:

\begin{lstlisting}
struct virtio_crypto_cipher_session_flf {
    /* Device read only portion */

    /* See VIRTIO_CRYPTO_CIPHER* above */
    le32 algo;
    /* length of key */
    le32 key_len;
#define VIRTIO_CRYPTO_OP_ENCRYPT  1
#define VIRTIO_CRYPTO_OP_DECRYPT  2
    /* encryption or decryption */
    le32 op;
    le32 padding;
};

struct virtio_crypto_cipher_session_vlf {
    /* Device read only portion */

    /* The cipher key */
    u8 cipher_key[key_len];
};
\end{lstlisting}

The length of \field{cipher_key} is specified in \field{key_len} in the struct
virtio_crypto_cipher_session_flf.

The fixed-length and the variable-length parameters of Chain session requests are as follows:

\begin{lstlisting}
struct virtio_crypto_alg_chain_session_flf {
    /* Device read only portion */

#define VIRTIO_CRYPTO_SYM_ALG_CHAIN_ORDER_HASH_THEN_CIPHER  1
#define VIRTIO_CRYPTO_SYM_ALG_CHAIN_ORDER_CIPHER_THEN_HASH  2
    le32 alg_chain_order;
/* Plain hash */
#define VIRTIO_CRYPTO_SYM_HASH_MODE_PLAIN    1
/* Authenticated hash (mac) */
#define VIRTIO_CRYPTO_SYM_HASH_MODE_AUTH     2
/* Nested hash */
#define VIRTIO_CRYPTO_SYM_HASH_MODE_NESTED   3
    le32 hash_mode;
    struct virtio_crypto_cipher_session_flf cipher_hdr;

#define VIRTIO_CRYPTO_ALG_CHAIN_SESS_OP_SPEC_HDR_SIZE  16
    /* fixed length fields, algo specific */
    u8 algo_flf[VIRTIO_CRYPTO_ALG_CHAIN_SESS_OP_SPEC_HDR_SIZE];

    /* length of the additional authenticated data (AAD) in bytes */
    le32 aad_len;
    le32 padding;
};

struct virtio_crypto_alg_chain_session_vlf {
    /* Device read only portion */

    /* The cipher key */
    u8 cipher_key[key_len];
    /* The authenticated key */
    u8 auth_key[auth_key_len];
};
\end{lstlisting}

\field{hash_mode} decides the type used by \field{algo_flf}.

\field{algo_flf} is fixed to 16 bytes and MUST contains or be one of
the following types:
\begin{itemize*}
\item struct virtio_crypto_hash_create_session_flf
\item struct virtio_crypto_mac_create_session_flf
\end{itemize*}
The data of unused part (if has) in \field{algo_flf} will be ignored.

The length of \field{cipher_key} is specified in \field{key_len} in \field{cipher_hdr}.

The length of \field{auth_key} is specified in \field{auth_key_len} in struct
virtio_crypto_mac_create_session_flf.

The fixed-length parameters of Symmetric session requests are as follows:

\begin{lstlisting}
struct virtio_crypto_sym_create_session_flf {
    /* Device read only portion */

#define VIRTIO_CRYPTO_SYM_SESS_OP_SPEC_HDR_SIZE  48
    /* fixed length fields, opcode specific */
    u8 op_flf[VIRTIO_CRYPTO_SYM_SESS_OP_SPEC_HDR_SIZE];

/* No operation */
#define VIRTIO_CRYPTO_SYM_OP_NONE  0
/* Cipher only operation on the data */
#define VIRTIO_CRYPTO_SYM_OP_CIPHER  1
/* Chain any cipher with any hash or mac operation. The order
   depends on the value of alg_chain_order param */
#define VIRTIO_CRYPTO_SYM_OP_ALGORITHM_CHAINING  2
    le32 op_type;
    le32 padding;
};
\end{lstlisting}

\field{op_flf} is fixed to 48 bytes, MUST contains or be one of
the following types:
\begin{itemize*}
\item struct virtio_crypto_cipher_session_flf
\item struct virtio_crypto_alg_chain_session_flf
\end{itemize*}
The data of unused part (if has) in \field{op_flf} will be ignored.

\field{op_type} decides the type used by \field{op_flf}.

The variable-length parameters of Symmetric session requests are as follows:

\begin{lstlisting}
struct virtio_crypto_sym_create_session_vlf {
    /* Device read only portion */
    /* variable length fields, opcode specific */
    u8 op_vlf[vlf_len];
};
\end{lstlisting}

\field{op_vlf} MUST contains or be one of the following types:
\begin{itemize*}
\item struct virtio_crypto_cipher_session_vlf
\item struct virtio_crypto_alg_chain_session_vlf
\end{itemize*}

\field{op_type} in struct virtio_crypto_sym_create_session_flf decides the
type used by \field{op_vlf}.

\field{vlf_len} is the size of the specific structure used.


\subparagraph{Session operation: AEAD session}\label{sec:Device Types / Crypto Device / Device
Operation / Control Virtqueue / Session operation / Session operation: AEAD session}

The fixed-length and the variable-length parameters of AEAD session requests are as follows:

\begin{lstlisting}
struct virtio_crypto_aead_create_session_flf {
    /* Device read only portion */

    /* See VIRTIO_CRYPTO_AEAD_* above */
    le32 algo;
    /* length of key */
    le32 key_len;
    /* Authentication tag length */
    le32 tag_len;
    /* The length of the additional authenticated data (AAD) in bytes */
    le32 aad_len;
    /* encryption or decryption, See above VIRTIO_CRYPTO_OP_* */
    le32 op;
    le32 padding;
};

struct virtio_crypto_aead_create_session_vlf {
    /* Device read only portion */
    u8 key[key_len];
};
\end{lstlisting}

The length of \field{key} is specified in \field{key_len} in struct
virtio_crypto_aead_create_session_flf.

\subparagraph{Session operation: AKCIPHER session}\label{sec:Device Types / Crypto Device / Device
Operation / Control Virtqueue / Session operation / Session operation: AKCIPHER session}

Due to the complexity of asymmetric key algorithms, different algorithms
require different parameters. The following data structures are used as
supplementary parameters to describe the asymmetric algorithm sessions.

For the RSA algorithm, the extra parameters are as follows:
\begin{lstlisting}
struct virtio_crypto_rsa_session_para {
#define VIRTIO_CRYPTO_RSA_RAW_PADDING   0
#define VIRTIO_CRYPTO_RSA_PKCS1_PADDING 1
    le32 padding_algo;

#define VIRTIO_CRYPTO_RSA_NO_HASH   0
#define VIRTIO_CRYPTO_RSA_MD2       1
#define VIRTIO_CRYPTO_RSA_MD3       2
#define VIRTIO_CRYPTO_RSA_MD4       3
#define VIRTIO_CRYPTO_RSA_MD5       4
#define VIRTIO_CRYPTO_RSA_SHA1      5
#define VIRTIO_CRYPTO_RSA_SHA256    6
#define VIRTIO_CRYPTO_RSA_SHA384    7
#define VIRTIO_CRYPTO_RSA_SHA512    8
#define VIRTIO_CRYPTO_RSA_SHA224    9
    le32 hash_algo;
};
\end{lstlisting}

\field{padding_algo} specifies the padding method used by RSA sessions.
\begin{itemize*}
\item If VIRTIO_CRYPTO_RSA_RAW_PADDING is specified, 1) \field{hash_algo}
is ignored, 2) ciphertext and plaintext MUST be padded with leading zeros,
3) and RSA sessions with VIRTIO_CRYPTO_RSA_RAW_PADDING MUST not be used
for verification and signing operations.
\item If VIRTIO_CRYPTO_RSA_PKCS1_PADDING is specified, EMSA-PKCS1-v1_5 padding method
is used (see \hyperref[intro:rfc3447]{PKCS\#1}), \field{hash_algo} specifies how the
digest of the data passed to RSA sessions is calculated when verifying and signing.
It only affects the padding algorithm and is ignored during encryption and decryption.
\end{itemize*}

The ECC algorithms such as the ECDSA algorithm, cannot use custom curves, only the
following known curves can be used (see \hyperref[intro:NIST]{NIST-recommended curves}).

\begin{lstlisting}
#define VIRTIO_CRYPTO_CURVE_UNKNOWN   0
#define VIRTIO_CRYPTO_CURVE_NIST_P192 1
#define VIRTIO_CRYPTO_CURVE_NIST_P224 2
#define VIRTIO_CRYPTO_CURVE_NIST_P256 3
#define VIRTIO_CRYPTO_CURVE_NIST_P384 4
#define VIRTIO_CRYPTO_CURVE_NIST_P521 5
\end{lstlisting}

For the ECDSA algorithm, the extra parameters are as follows:
\begin{lstlisting}
struct virtio_crypto_ecdsa_session_para {
    /* See VIRTIO_CRYPTO_CURVE_* above */
    le32 curve_id;
};
\end{lstlisting}

The fixed-length and the variable-length parameters of AKCIPHER session requests are as follows:
\begin{lstlisting}
struct virtio_crypto_akcipher_create_session_flf {
    /* Device read only portion */

    /* See VIRTIO_CRYPTO_AKCIPHER_* above */
    le32 algo;
#define VIRTIO_CRYPTO_AKCIPHER_KEY_TYPE_PUBLIC 1
#define VIRTIO_CRYPTO_AKCIPHER_KEY_TYPE_PRIVATE 2
    le32 key_type;
    /* length of key */
    le32 key_len;

#define VIRTIO_CRYPTO_AKCIPHER_SESS_ALGO_SPEC_HDR_SIZE 44
    u8 algo_flf[VIRTIO_CRYPTO_AKCIPHER_SESS_ALGO_SPEC_HDR_SIZE];
};

struct virtio_crypto_akcipher_create_session_vlf {
    /* Device read only portion */
    u8 key[key_len];
};
\end{lstlisting}

\field{algo} decides the type used by \field{algo_flf}.
\field{algo_flf} is fixed to 44 bytes and MUST contains of be one the
following structures:
\begin{itemize*}
\item struct virtio_crypto_rsa_session_para
\item struct virtio_crypto_ecdsa_session_para
\end{itemize*}

The length of \field{key} is specified in \field{key_len} in the struct
virtio_crypto_akcipher_create_session_flf.

For the RSA algorithm, the key needs to be encoded according to
\hyperref[intro:rfc3447]{PKCS\#1}. The private key is described with the
RSAPrivateKey structure, and the public key is described with the RSAPublicKey
structure. These ASN.1 structures are encoded in DER encoding rules (see
\hyperref[intro:rfc6025]{rfc6025}).

\begin{lstlisting}
RSAPrivateKey ::= SEQUENCE {
    version          INTEGER,
    modulus          INTEGER,
    publicExponent   INTEGER,
    privateExponent  INTEGER,
    prime1           INTEGER,
    prime2           INTEGER,
    exponent1        INTEGER,
    exponent1        INTEGER,
    coefficient      INTEGER,
    otherPrimeInfos  OtherPrimeInfos OPTIONAL
}

OtherPrimeInfos ::= SEQUENCE SIZE(1...MAX) OF OtherPrimeInfo

OtherPrimeINfo ::= SEQUENCE {
    prime           INTEGER,
    exponent        INTEGER,
    coefficient     INTEGER
}

RSAPublicKey ::= SEQUENCE {
    modulus         INTEGER,
    publicExponent  INTEGER
}
\end{lstlisting}

For the ECDSA algorithm, the private key is encoded according to
\hyperref[intro:rfc5915]{RFC5915}, the private key of the ECDSA algorithm
is described by the ASN.1 structure ECPrivateKey and encoded with DER
encoding rules (see \hyperref[intro:rfc6025]{rfc6025}).

\begin{lstlisting}
ECPrivateKey ::= SEQUNCE {
    version         INTEGER,
    privateKey      OCTET STRING,
    parameters [0]  ECParameters {{ NamedCurve }} OPTIONAL,
    publicKey  [1]  BIT STRING OPTIONAL
}
\end{lstlisting}

The public key of the ECDSA algorithm is encoded according to \hyperref[intro:SEC1]{SEC1},
and the public key of ECDSA is described by the ASN.1 structure ECPoint.
When initializing a session with ECDSA public key, the ECPoint is DER encoded and the
\field{key} only contains the value part of ECPoint, that is, the header part of the
OCTET STRING will be omitted (see \hyperref[intro:rfc6025]{rfc6025}).

\begin{lstlisting}
ECPoint ::= OCTET STRING
\end{lstlisting}

The length of \field{key} is specified in \field{key_len} in
struct virtio_crypto_akcipher_create_session_flf.

\drivernormative{\subparagraph}{Session operation: create session}{Device Types / Crypto Device / Device
Operation / Control Virtqueue / Session operation / Session operation: create session}

\begin{itemize*}
\item The driver MUST set the \field{opcode} field based on service type: CIPHER, HASH, MAC, AEAD or AKCIPHER.
\item The driver MUST set the control general header, the opcode specific header,
    the opcode specific extra parameters and the opcode specific outcome buffer in turn.
    See \ref{sec:Device Types / Crypto Device / Device Operation / Control Virtqueue}.
\item The driver MUST set the \field{reversed} field to zero.
\end{itemize*}

\devicenormative{\subparagraph}{Session operation: create session}{Device Types / Crypto Device / Device
Operation / Control Virtqueue / Session operation / Session operation: create session}

\begin{itemize*}
\item The device MUST use the corresponding opcode specific structure according to the
    \field{opcode} in the control general header.
\item The device MUST extract extra parameters according to the structures used.
\item The device MUST set the \field{status} field to one of the following values of enum
    VIRTIO_CRYPTO_STATUS after finish a session creation:
\begin{itemize*}
\item VIRTIO_CRYPTO_OK if a session is created successfully.
\item VIRTIO_CRYPTO_NOTSUPP if the requested algorithm or operation is unsupported.
\item VIRTIO_CRYPTO_NOSPC if no free session ID (only when the VIRTIO_CRYPTO_F_REVISION_1
    feature bit is negotiated).
\item VIRTIO_CRYPTO_ERR if failure not mentioned above occurs.
\end{itemize*}
\item The device MUST set the \field{session_id} field to a unique session identifier only
    if the status is set to VIRTIO_CRYPTO_OK.
\end{itemize*}

\drivernormative{\subparagraph}{Session operation: destroy session}{Device Types / Crypto Device / Device
Operation / Control Virtqueue / Session operation / Session operation: destroy session}

\begin{itemize*}
\item The driver MUST set the \field{opcode} field based on service type: CIPHER, HASH, MAC, AEAD or AKCIPHER.
\item The driver MUST set the \field{session_id} to a valid value assigned by the device
    when the session was created.
\end{itemize*}

\devicenormative{\subparagraph}{Session operation: destroy session}{Device Types / Crypto Device / Device
Operation / Control Virtqueue / Session operation / Session operation: destroy session}

\begin{itemize*}
\item The device MUST set the \field{status} field to one of the following values of enum VIRTIO_CRYPTO_STATUS.
\begin{itemize*}
\item VIRTIO_CRYPTO_OK if a session is created successfully.
\item VIRTIO_CRYPTO_ERR if any failure occurs.
\end{itemize*}
\end{itemize*}


\subsubsection{Data Virtqueue}\label{sec:Device Types / Crypto Device / Device Operation / Data Virtqueue}

The driver uses the data virtqueues to transmit crypto operation requests to the device,
and completes the crypto operations.

The header for dataq is as follows:

\begin{lstlisting}
struct virtio_crypto_op_header {
#define VIRTIO_CRYPTO_CIPHER_ENCRYPT \
    VIRTIO_CRYPTO_OPCODE(VIRTIO_CRYPTO_SERVICE_CIPHER, 0x00)
#define VIRTIO_CRYPTO_CIPHER_DECRYPT \
    VIRTIO_CRYPTO_OPCODE(VIRTIO_CRYPTO_SERVICE_CIPHER, 0x01)
#define VIRTIO_CRYPTO_HASH \
    VIRTIO_CRYPTO_OPCODE(VIRTIO_CRYPTO_SERVICE_HASH, 0x00)
#define VIRTIO_CRYPTO_MAC \
    VIRTIO_CRYPTO_OPCODE(VIRTIO_CRYPTO_SERVICE_MAC, 0x00)
#define VIRTIO_CRYPTO_AEAD_ENCRYPT \
    VIRTIO_CRYPTO_OPCODE(VIRTIO_CRYPTO_SERVICE_AEAD, 0x00)
#define VIRTIO_CRYPTO_AEAD_DECRYPT \
    VIRTIO_CRYPTO_OPCODE(VIRTIO_CRYPTO_SERVICE_AEAD, 0x01)
#define VIRTIO_CRYPTO_AKCIPHER_ENCRYPT \
    VIRTIO_CRYPTO_OPCODE(VIRTIO_CRYPTO_SERVICE_AKCIPHER, 0x00)
#define VIRTIO_CRYPTO_AKCIPHER_DECRYPT \
    VIRTIO_CRYPTO_OPCODE(VIRTIO_CRYPTO_SERVICE_AKCIPHER, 0x01)
#define VIRTIO_CRYPTO_AKCIPHER_SIGN \
    VIRTIO_CRYPTO_OPCODE(VIRTIO_CRYPTO_SERVICE_AKCIPHER, 0x02)
#define VIRTIO_CRYPTO_AKCIPHER_VERIFY \
    VIRTIO_CRYPTO_OPCODE(VIRTIO_CRYPTO_SERVICE_AKCIPHER, 0x03)
    le32 opcode;
    /* algo should be service-specific algorithms */
    le32 algo;
    le64 session_id;
#define VIRTIO_CRYPTO_FLAG_SESSION_MODE 1
    /* control flag to control the request */
    le32 flag;
    le32 padding;
};
\end{lstlisting}

\begin{note}
If VIRTIO_CRYPTO_F_REVISION_1 is not negotiated the \field{flag} is ignored.

If VIRTIO_CRYPTO_F_REVISION_1 is negotiated but VIRTIO_CRYPTO_F_<SERVICE>_STATELESS_MODE
is not negotiated, then the device SHOULD reject <SERVICE> requests if
VIRTIO_CRYPTO_FLAG_SESSION_MODE is not set (in \field{flag}).
\end{note}

The dataq request is composed of four parts:
\begin{lstlisting}
struct virtio_crypto_op_data_req {
    /* Device read only portion */

    struct virtio_crypto_op_header header;

#define VIRTIO_CRYPTO_DATAQ_OP_SPEC_HDR_LEGACY 48
    /* fixed length fields, opcode specific */
    u8 op_flf[flf_len];

    /* Device read && write portion */
    /* variable length fields, opcode specific */
    u8 op_vlf[vlf_len];

    /* Device write only portion */
    struct virtio_crypto_inhdr inhdr;
};
\end{lstlisting}

\field{header} is a general header (see above).

\field{op_flf} is the opcode (in \field{header}) specific header.

\field{flf_len} depends on the VIRTIO_CRYPTO_F_REVISION_1 feature bit
(see below).

\field{op_vlf} is the opcode (in \field{header}) specific parameters.

\field{vlf_len} is the size of the specific structure used.

\begin{itemize*}
\item If the the opcode (in \field{header}) is VIRTIO_CRYPTO_CIPHER_ENCRYPT
    or VIRTIO_CRYPTO_CIPHER_DECRYPT then:
    \begin{itemize*}
    \item If VIRTIO_CRYPTO_F_CIPHER_STATELESS_MODE is negotiated, \field{op_flf} is
        struct virtio_crypto_sym_data_flf_stateless, and \field{op_vlf} is struct
        virtio_crypto_sym_data_vlf_stateless.
    \item If VIRTIO_CRYPTO_F_CIPHER_STATELESS_MODE is NOT negotiated, \field{op_flf}
        is struct virtio_crypto_sym_data_flf if VIRTIO_CRYPTO_F_REVISION_1 is negotiated
        and struct virtio_crypto_sym_data_flf is padded to 48 bytes if NOT negotiated,
        and \field{op_vlf} is struct virtio_crypto_sym_data_vlf.
    \end{itemize*}
\item If the the opcode (in \field{header}) is VIRTIO_CRYPTO_HASH:
    \begin{itemize*}
    \item If VIRTIO_CRYPTO_F_HASH_STATELESS_MODE is negotiated, \field{op_flf} is
        struct virtio_crypto_hash_data_flf_stateless, and \field{op_vlf} is struct
        virtio_crypto_hash_data_vlf_stateless.
    \item If VIRTIO_CRYPTO_F_HASH_STATELESS_MODE is NOT negotiated, \field{op_flf}
        is struct virtio_crypto_hash_data_flf if VIRTIO_CRYPTO_F_REVISION_1 is negotiated
        and struct virtio_crypto_hash_data_flf is padded to 48 bytes if NOT negotiated,
        and \field{op_vlf} is struct virtio_crypto_hash_data_vlf.
    \end{itemize*}
\item If the the opcode (in \field{header}) is VIRTIO_CRYPTO_MAC:
    \begin{itemize*}
    \item If VIRTIO_CRYPTO_F_MAC_STATELESS_MODE is negotiated, \field{op_flf} is
        struct virtio_crypto_mac_data_flf_stateless, and \field{op_vlf} is struct
        virtio_crypto_mac_data_vlf_stateless.
    \item If VIRTIO_CRYPTO_F_MAC_STATELESS_MODE is NOT negotiated, \field{op_flf}
        is struct virtio_crypto_mac_data_flf if VIRTIO_CRYPTO_F_REVISION_1 is negotiated
        and struct virtio_crypto_mac_data_flf is padded to 48 bytes if NOT negotiated,
        and \field{op_vlf} is struct virtio_crypto_mac_data_vlf.
    \end{itemize*}
\item If the the opcode (in \field{header}) is VIRTIO_CRYPTO_AEAD_ENCRYPT
    or VIRTIO_CRYPTO_AEAD_DECRYPT then:
    \begin{itemize*}
    \item If VIRTIO_CRYPTO_F_AEAD_STATELESS_MODE is negotiated, \field{op_flf} is
        struct virtio_crypto_aead_data_flf_stateless, and \field{op_vlf} is struct
        virtio_crypto_aead_data_vlf_stateless.
    \item If VIRTIO_CRYPTO_F_AEAD_STATELESS_MODE is NOT negotiated, \field{op_flf}
        is struct virtio_crypto_aead_data_flf if VIRTIO_CRYPTO_F_REVISION_1 is negotiated
        and struct virtio_crypto_aead_data_flf is padded to 48 bytes if NOT negotiated,
        and \field{op_vlf} is struct virtio_crypto_aead_data_vlf.
    \end{itemize*}
\item If the opcode (in \field{header}) is VIRTIO_CRYPTO_AKCIPHER_ENCRYPT, VIRTIO_CRYPTO_AKCIPHER_DECRYPT,
    VIRTIO_CRYPTO_AKCIPHER_SIGN or VIRTIO_CRYPTO_AKCIPHER_VERIFY then:
    \begin{itemize*}
    \item If VIRTIO_CRYPTO_F_AKCIPHER_STATELESS_MODE is negotiated, \field{op_flf} is
        struct virtio_crypto_akcipher_data_flf_statless, and \field{op_vlf} is struct
        virtio_crypto_akcipher_data_vlf_stateless.
    \item If VIRTIO_CRYPTO_F_AKCIPHER_STATELESS_MODE is NOT negotiated, \field{op_flf}
        is struct virtio_crypto_akcipher_data_flf if VIRTIO_CRYPTO_F_REVISION_1 is negotiated
        and struct virtio_crypto_akcipher_data_flf is padded to 48 bytes if NOT negotiated,
        and \field{op_vlf} is struct virtio_crypto_akcipher_data_vlf.
    \end{itemize*}
\end{itemize*}

\field{inhdr} is a unified input header that used to return the status of
the operations, is defined as follows:

\begin{lstlisting}
struct virtio_crypto_inhdr {
    u8 status;
};
\end{lstlisting}

\subsubsection{HASH Service Operation}\label{sec:Device Types / Crypto Device / Device Operation / HASH Service Operation}

Session mode HASH service requests are as follows:

\begin{lstlisting}
struct virtio_crypto_hash_data_flf {
    /* length of source data */
    le32 src_data_len;
    /* hash result length */
    le32 hash_result_len;
};

struct virtio_crypto_hash_data_vlf {
    /* Device read only portion */
    /* Source data */
    u8 src_data[src_data_len];

    /* Device write only portion */
    /* Hash result data */
    u8 hash_result[hash_result_len];
};
\end{lstlisting}

Each data request uses the virtio_crypto_hash_data_flf structure and the
virtio_crypto_hash_data_vlf structure to store information used to run the
HASH operations.

\field{src_data} is the source data that will be processed.
\field{src_data_len} is the length of source data.
\field{hash_result} is the result data and \field{hash_result_len} is the length
of it.

Stateless mode HASH service requests are as follows:

\begin{lstlisting}
struct virtio_crypto_hash_data_flf_stateless {
    struct {
        /* See VIRTIO_CRYPTO_HASH_* above */
        le32 algo;
    } sess_para;

    /* length of source data */
    le32 src_data_len;
    /* hash result length */
    le32 hash_result_len;
    le32 reserved;
};
struct virtio_crypto_hash_data_vlf_stateless {
    /* Device read only portion */
    /* Source data */
    u8 src_data[src_data_len];

    /* Device write only portion */
    /* Hash result data */
    u8 hash_result[hash_result_len];
};
\end{lstlisting}

\drivernormative{\paragraph}{HASH Service Operation}{Device Types / Crypto Device / Device Operation / HASH Service Operation}

\begin{itemize*}
\item If the driver uses the session mode, then the driver MUST set \field{session_id}
    in struct virtio_crypto_op_header to a valid value assigned by the device when the
    session was created.
\item If the VIRTIO_CRYPTO_F_HASH_STATELESS_MODE feature bit is negotiated, 1) if the
    driver uses the stateless mode, then the driver MUST set the \field{flag} field in
    struct virtio_crypto_op_header to ZERO and MUST set the fields in struct
    virtio_crypto_hash_data_flf_stateless.sess_para, 2) if the driver uses the session
    mode, then the driver MUST set the \field{flag} field in struct virtio_crypto_op_header
    to VIRTIO_CRYPTO_FLAG_SESSION_MODE.
\item The driver MUST set \field{opcode} in struct virtio_crypto_op_header to VIRTIO_CRYPTO_HASH.
\end{itemize*}

\devicenormative{\paragraph}{HASH Service Operation}{Device Types / Crypto Device / Device Operation / HASH Service Operation}

\begin{itemize*}
\item The device MUST use the corresponding structure according to the \field{opcode}
    in the data general header.
\item If the VIRTIO_CRYPTO_F_HASH_STATELESS_MODE feature bit is negotiated, the device
    MUST parse \field{flag} field in struct virtio_crypto_op_header in order to decide
    which mode the driver uses.
\item The device MUST copy the results of HASH operations in the hash_result[] if HASH
    operations success.
\item The device MUST set \field{status} in struct virtio_crypto_inhdr to one of the
    following values of enum VIRTIO_CRYPTO_STATUS:
\begin{itemize*}
\item VIRTIO_CRYPTO_OK if the operation success.
\item VIRTIO_CRYPTO_NOTSUPP if the requested algorithm or operation is unsupported.
\item VIRTIO_CRYPTO_INVSESS if the session ID invalid when in session mode.
\item VIRTIO_CRYPTO_ERR if any failure not mentioned above occurs.
\end{itemize*}
\end{itemize*}


\subsubsection{MAC Service Operation}\label{sec:Device Types / Crypto Device / Device Operation / MAC Service Operation}

Session mode MAC service requests are as follows:

\begin{lstlisting}
struct virtio_crypto_mac_data_flf {
    struct virtio_crypto_hash_data_flf hdr;
};

struct virtio_crypto_mac_data_vlf {
    /* Device read only portion */
    /* Source data */
    u8 src_data[src_data_len];

    /* Device write only portion */
    /* Hash result data */
    u8 hash_result[hash_result_len];
};
\end{lstlisting}

Each request uses the virtio_crypto_mac_data_flf structure and the
virtio_crypto_mac_data_vlf structure to store information used to run the
MAC operations.

\field{src_data} is the source data that will be processed.
\field{src_data_len} is the length of source data.
\field{hash_result} is the result data and \field{hash_result_len} is the length
of it.

Stateless mode MAC service requests are as follows:

\begin{lstlisting}
struct virtio_crypto_mac_data_flf_stateless {
    struct {
        /* See VIRTIO_CRYPTO_MAC_* above */
        le32 algo;
        /* length of authenticated key */
        le32 auth_key_len;
    } sess_para;

    /* length of source data */
    le32 src_data_len;
    /* hash result length */
    le32 hash_result_len;
};

struct virtio_crypto_mac_data_vlf_stateless {
    /* Device read only portion */
    /* The authenticated key */
    u8 auth_key[auth_key_len];
    /* Source data */
    u8 src_data[src_data_len];

    /* Device write only portion */
    /* Hash result data */
    u8 hash_result[hash_result_len];
};
\end{lstlisting}

\field{auth_key} is the authenticated key that will be used during the process.
\field{auth_key_len} is the length of the key.

\drivernormative{\paragraph}{MAC Service Operation}{Device Types / Crypto Device / Device Operation / MAC Service Operation}

\begin{itemize*}
\item If the driver uses the session mode, then the driver MUST set \field{session_id}
    in struct virtio_crypto_op_header to a valid value assigned by the device when the
    session was created.
\item If the VIRTIO_CRYPTO_F_MAC_STATELESS_MODE feature bit is negotiated, 1) if the
    driver uses the stateless mode, then the driver MUST set the \field{flag} field
    in struct virtio_crypto_op_header to ZERO and MUST set the fields in struct
    virtio_crypto_mac_data_flf_stateless.sess_para, 2) if the driver uses the session
    mode, then the driver MUST set the \field{flag} field in struct virtio_crypto_op_header
    to VIRTIO_CRYPTO_FLAG_SESSION_MODE.
\item The driver MUST set \field{opcode} in struct virtio_crypto_op_header to VIRTIO_CRYPTO_MAC.
\end{itemize*}

\devicenormative{\paragraph}{MAC Service Operation}{Device Types / Crypto Device / Device Operation / MAC Service Operation}

\begin{itemize*}
\item If the VIRTIO_CRYPTO_F_MAC_STATELESS_MODE feature bit is negotiated, the device
    MUST parse \field{flag} field in struct virtio_crypto_op_header in order to decide
	which mode the driver uses.
\item The device MUST copy the results of MAC operations in the hash_result[] if HASH
    operations success.
\item The device MUST set \field{status} in struct virtio_crypto_inhdr to one of the
    following values of enum VIRTIO_CRYPTO_STATUS:
\begin{itemize*}
\item VIRTIO_CRYPTO_OK if the operation success.
\item VIRTIO_CRYPTO_NOTSUPP if the requested algorithm or operation is unsupported.
\item VIRTIO_CRYPTO_INVSESS if the session ID invalid when in session mode.
\item VIRTIO_CRYPTO_ERR if any failure not mentioned above occurs.
\end{itemize*}
\end{itemize*}

\subsubsection{Symmetric algorithms Operation}\label{sec:Device Types / Crypto Device / Device Operation / Symmetric algorithms Operation}

Session mode CIPHER service requests are as follows:

\begin{lstlisting}
struct virtio_crypto_cipher_data_flf {
    /*
     * Byte Length of valid IV/Counter data pointed to by the below iv data.
     *
     * For block ciphers in CBC or F8 mode, or for Kasumi in F8 mode, or for
     *   SNOW3G in UEA2 mode, this is the length of the IV (which
     *   must be the same as the block length of the cipher).
     * For block ciphers in CTR mode, this is the length of the counter
     *   (which must be the same as the block length of the cipher).
     */
    le32 iv_len;
    /* length of source data */
    le32 src_data_len;
    /* length of destination data */
    le32 dst_data_len;
    le32 padding;
};

struct virtio_crypto_cipher_data_vlf {
    /* Device read only portion */

    /*
     * Initialization Vector or Counter data.
     *
     * For block ciphers in CBC or F8 mode, or for Kasumi in F8 mode, or for
     *   SNOW3G in UEA2 mode, this is the Initialization Vector (IV)
     *   value.
     * For block ciphers in CTR mode, this is the counter.
     * For AES-XTS, this is the 128bit tweak, i, from IEEE Std 1619-2007.
     *
     * The IV/Counter will be updated after every partial cryptographic
     * operation.
     */
    u8 iv[iv_len];
    /* Source data */
    u8 src_data[src_data_len];

    /* Device write only portion */
    /* Destination data */
    u8 dst_data[dst_data_len];
};
\end{lstlisting}

Session mode requests of algorithm chaining are as follows:

\begin{lstlisting}
struct virtio_crypto_alg_chain_data_flf {
    le32 iv_len;
    /* Length of source data */
    le32 src_data_len;
    /* Length of destination data */
    le32 dst_data_len;
    /* Starting point for cipher processing in source data */
    le32 cipher_start_src_offset;
    /* Length of the source data that the cipher will be computed on */
    le32 len_to_cipher;
    /* Starting point for hash processing in source data */
    le32 hash_start_src_offset;
    /* Length of the source data that the hash will be computed on */
    le32 len_to_hash;
    /* Length of the additional auth data */
    le32 aad_len;
    /* Length of the hash result */
    le32 hash_result_len;
    le32 reserved;
};

struct virtio_crypto_alg_chain_data_vlf {
    /* Device read only portion */

    /* Initialization Vector or Counter data */
    u8 iv[iv_len];
    /* Source data */
    u8 src_data[src_data_len];
    /* Additional authenticated data if exists */
    u8 aad[aad_len];

    /* Device write only portion */

    /* Destination data */
    u8 dst_data[dst_data_len];
    /* Hash result data */
    u8 hash_result[hash_result_len];
};
\end{lstlisting}

Session mode requests of symmetric algorithm are as follows:

\begin{lstlisting}
struct virtio_crypto_sym_data_flf {
    /* Device read only portion */

#define VIRTIO_CRYPTO_SYM_DATA_REQ_HDR_SIZE    40
    u8 op_type_flf[VIRTIO_CRYPTO_SYM_DATA_REQ_HDR_SIZE];

    /* See above VIRTIO_CRYPTO_SYM_OP_* */
    le32 op_type;
    le32 padding;
};

struct virtio_crypto_sym_data_vlf {
    u8 op_type_vlf[sym_para_len];
};
\end{lstlisting}

Each request uses the virtio_crypto_sym_data_flf structure and the
virtio_crypto_sym_data_flf structure to store information used to run the
CIPHER operations.

\field{op_type_flf} is the \field{op_type} specific header, it MUST starts
with or be one of the following structures:
\begin{itemize*}
\item struct virtio_crypto_cipher_data_flf
\item struct virtio_crypto_alg_chain_data_flf
\end{itemize*}

The length of \field{op_type_flf} is fixed to 40 bytes, the data of unused
part (if has) will be ignored.

\field{op_type_vlf} is the \field{op_type} specific parameters, it MUST starts
with or be one of the following structures:
\begin{itemize*}
\item struct virtio_crypto_cipher_data_vlf
\item struct virtio_crypto_alg_chain_data_vlf
\end{itemize*}

\field{sym_para_len} is the size of the specific structure used.

Stateless mode CIPHER service requests are as follows:

\begin{lstlisting}
struct virtio_crypto_cipher_data_flf_stateless {
    struct {
        /* See VIRTIO_CRYPTO_CIPHER* above */
        le32 algo;
        /* length of key */
        le32 key_len;

        /* See VIRTIO_CRYPTO_OP_* above */
        le32 op;
    } sess_para;

    /*
     * Byte Length of valid IV/Counter data pointed to by the below iv data.
     */
    le32 iv_len;
    /* length of source data */
    le32 src_data_len;
    /* length of destination data */
    le32 dst_data_len;
};

struct virtio_crypto_cipher_data_vlf_stateless {
    /* Device read only portion */

    /* The cipher key */
    u8 cipher_key[key_len];

    /* Initialization Vector or Counter data. */
    u8 iv[iv_len];
    /* Source data */
    u8 src_data[src_data_len];

    /* Device write only portion */
    /* Destination data */
    u8 dst_data[dst_data_len];
};
\end{lstlisting}

Stateless mode requests of algorithm chaining are as follows:

\begin{lstlisting}
struct virtio_crypto_alg_chain_data_flf_stateless {
    struct {
        /* See VIRTIO_CRYPTO_SYM_ALG_CHAIN_ORDER_* above */
        le32 alg_chain_order;
        /* length of the additional authenticated data in bytes */
        le32 aad_len;

        struct {
            /* See VIRTIO_CRYPTO_CIPHER* above */
            le32 algo;
            /* length of key */
            le32 key_len;
            /* See VIRTIO_CRYPTO_OP_* above */
            le32 op;
        } cipher;

        struct {
            /* See VIRTIO_CRYPTO_HASH_* or VIRTIO_CRYPTO_MAC_* above */
            le32 algo;
            /* length of authenticated key */
            le32 auth_key_len;
            /* See VIRTIO_CRYPTO_SYM_HASH_MODE_* above */
            le32 hash_mode;
        } hash;
    } sess_para;

    le32 iv_len;
    /* Length of source data */
    le32 src_data_len;
    /* Length of destination data */
    le32 dst_data_len;
    /* Starting point for cipher processing in source data */
    le32 cipher_start_src_offset;
    /* Length of the source data that the cipher will be computed on */
    le32 len_to_cipher;
    /* Starting point for hash processing in source data */
    le32 hash_start_src_offset;
    /* Length of the source data that the hash will be computed on */
    le32 len_to_hash;
    /* Length of the additional auth data */
    le32 aad_len;
    /* Length of the hash result */
    le32 hash_result_len;
    le32 reserved;
};

struct virtio_crypto_alg_chain_data_vlf_stateless {
    /* Device read only portion */

    /* The cipher key */
    u8 cipher_key[key_len];
    /* The auth key */
    u8 auth_key[auth_key_len];
    /* Initialization Vector or Counter data */
    u8 iv[iv_len];
    /* Additional authenticated data if exists */
    u8 aad[aad_len];
    /* Source data */
    u8 src_data[src_data_len];

    /* Device write only portion */

    /* Destination data */
    u8 dst_data[dst_data_len];
    /* Hash result data */
    u8 hash_result[hash_result_len];
};
\end{lstlisting}

Stateless mode requests of symmetric algorithm are as follows:

\begin{lstlisting}
struct virtio_crypto_sym_data_flf_stateless {
    /* Device read only portion */
#define VIRTIO_CRYPTO_SYM_DATE_REQ_HDR_STATELESS_SIZE    72
    u8 op_type_flf[VIRTIO_CRYPTO_SYM_DATE_REQ_HDR_STATELESS_SIZE];

    /* Device write only portion */
    /* See above VIRTIO_CRYPTO_SYM_OP_* */
    le32 op_type;
};

struct virtio_crypto_sym_data_vlf_stateless {
    u8 op_type_vlf[sym_para_len];
};
\end{lstlisting}

\field{op_type_flf} is the \field{op_type} specific header, it MUST starts
with or be one of the following structures:
\begin{itemize*}
\item struct virtio_crypto_cipher_data_flf_stateless
\item struct virtio_crypto_alg_chain_data_flf_stateless
\end{itemize*}

The length of \field{op_type_flf} is fixed to 72 bytes, the data of unused
part (if has) will be ignored.

\field{op_type_vlf} is the \field{op_type} specific parameters, it MUST starts
with or be one of the following structures:
\begin{itemize*}
\item struct virtio_crypto_cipher_data_vlf_stateless
\item struct virtio_crypto_alg_chain_data_vlf_stateless
\end{itemize*}

\field{sym_para_len} is the size of the specific structure used.

\drivernormative{\paragraph}{Symmetric algorithms Operation}{Device Types / Crypto Device / Device Operation / Symmetric algorithms Operation}

\begin{itemize*}
\item If the driver uses the session mode, then the driver MUST set \field{session_id}
    in struct virtio_crypto_op_header to a valid value assigned by the device when the
    session was created.
\item If the VIRTIO_CRYPTO_F_CIPHER_STATELESS_MODE feature bit is negotiated, 1) if the
    driver uses the stateless mode, then the driver MUST set the \field{flag} field in
    struct virtio_crypto_op_header to ZERO and MUST set the fields in struct
    virtio_crypto_cipher_data_flf_stateless.sess_para or struct
    virtio_crypto_alg_chain_data_flf_stateless.sess_para, 2) if the driver uses the
    session mode, then the driver MUST set the \field{flag} field in struct
    virtio_crypto_op_header to VIRTIO_CRYPTO_FLAG_SESSION_MODE.
\item The driver MUST set the \field{opcode} field in struct virtio_crypto_op_header
    to VIRTIO_CRYPTO_CIPHER_ENCRYPT or VIRTIO_CRYPTO_CIPHER_DECRYPT.
\item The driver MUST specify the fields of struct virtio_crypto_cipher_data_flf in
    struct virtio_crypto_sym_data_flf and struct virtio_crypto_cipher_data_vlf in
    struct virtio_crypto_sym_data_vlf if the request is based on VIRTIO_CRYPTO_SYM_OP_CIPHER.
\item The driver MUST specify the fields of struct virtio_crypto_alg_chain_data_flf
    in struct virtio_crypto_sym_data_flf and struct virtio_crypto_alg_chain_data_vlf
    in struct virtio_crypto_sym_data_vlf if the request is of the VIRTIO_CRYPTO_SYM_OP_ALGORITHM_CHAINING
    type.
\end{itemize*}

\devicenormative{\paragraph}{Symmetric algorithms Operation}{Device Types / Crypto Device / Device Operation / Symmetric algorithms Operation}

\begin{itemize*}
\item If the VIRTIO_CRYPTO_F_CIPHER_STATELESS_MODE feature bit is negotiated, the device
    MUST parse \field{flag} field in struct virtio_crypto_op_header in order to decide
	which mode the driver uses.
\item The device MUST parse the virtio_crypto_sym_data_req based on the \field{opcode}
    field in general header.
\item The device MUST parse the fields of struct virtio_crypto_cipher_data_flf in
    struct virtio_crypto_sym_data_flf and struct virtio_crypto_cipher_data_vlf in
    struct virtio_crypto_sym_data_vlf if the request is based on VIRTIO_CRYPTO_SYM_OP_CIPHER.
\item The device MUST parse the fields of struct virtio_crypto_alg_chain_data_flf
    in struct virtio_crypto_sym_data_flf and struct virtio_crypto_alg_chain_data_vlf
    in struct virtio_crypto_sym_data_vlf if the request is of the VIRTIO_CRYPTO_SYM_OP_ALGORITHM_CHAINING
    type.
\item The device MUST copy the result of cryptographic operation in the dst_data[] in
    both plain CIPHER mode and algorithms chain mode.
\item The device MUST check the \field{para}.\field{add_len} is bigger than 0 before
    parse the additional authenticated data in plain algorithms chain mode.
\item The device MUST copy the result of HASH/MAC operation in the hash_result[] is
    of the VIRTIO_CRYPTO_SYM_OP_ALGORITHM_CHAINING type.
\item The device MUST set the \field{status} field in struct virtio_crypto_inhdr to
    one of the following values of enum VIRTIO_CRYPTO_STATUS:
\begin{itemize*}
\item VIRTIO_CRYPTO_OK if the operation success.
\item VIRTIO_CRYPTO_NOTSUPP if the requested algorithm or operation is unsupported.
\item VIRTIO_CRYPTO_INVSESS if the session ID is invalid in session mode.
\item VIRTIO_CRYPTO_ERR if failure not mentioned above occurs.
\end{itemize*}
\end{itemize*}

\subsubsection{AEAD Service Operation}\label{sec:Device Types / Crypto Device / Device Operation / AEAD Service Operation}

Session mode requests of symmetric algorithm are as follows:

\begin{lstlisting}
struct virtio_crypto_aead_data_flf {
    /*
     * Byte Length of valid IV data.
     *
     * For GCM mode, this is either 12 (for 96-bit IVs) or 16, in which
     *   case iv points to J0.
     * For CCM mode, this is the length of the nonce, which can be in the
     *   range 7 to 13 inclusive.
     */
    le32 iv_len;
    /* length of additional auth data */
    le32 aad_len;
    /* length of source data */
    le32 src_data_len;
    /* length of dst data, this should be at least src_data_len + tag_len */
    le32 dst_data_len;
    /* Authentication tag length */
    le32 tag_len;
    le32 reserved;
};

struct virtio_crypto_aead_data_vlf {
    /* Device read only portion */

    /*
     * Initialization Vector data.
     *
     * For GCM mode, this is either the IV (if the length is 96 bits) or J0
     *   (for other sizes), where J0 is as defined by NIST SP800-38D.
     *   Regardless of the IV length, a full 16 bytes needs to be allocated.
     * For CCM mode, the first byte is reserved, and the nonce should be
     *   written starting at &iv[1] (to allow space for the implementation
     *   to write in the flags in the first byte).  Note that a full 16 bytes
     *   should be allocated, even though the iv_len field will have
     *   a value less than this.
     *
     * The IV will be updated after every partial cryptographic operation.
     */
    u8 iv[iv_len];
    /* Source data */
    u8 src_data[src_data_len];
    /* Additional authenticated data if exists */
    u8 aad[aad_len];

    /* Device write only portion */
    /* Pointer to output data */
    u8 dst_data[dst_data_len];
};
\end{lstlisting}

Each request uses the virtio_crypto_aead_data_flf structure and the
virtio_crypto_aead_data_flf structure to store information used to run the
AEAD operations.

Stateless mode AEAD service requests are as follows:

\begin{lstlisting}
struct virtio_crypto_aead_data_flf_stateless {
    struct {
        /* See VIRTIO_CRYPTO_AEAD_* above */
        le32 algo;
        /* length of key */
        le32 key_len;
        /* encrypt or decrypt, See above VIRTIO_CRYPTO_OP_* */
        le32 op;
    } sess_para;

    /* Byte Length of valid IV data. */
    le32 iv_len;
    /* Authentication tag length */
    le32 tag_len;
    /* length of additional auth data */
    le32 aad_len;
    /* length of source data */
    le32 src_data_len;
    /* length of dst data, this should be at least src_data_len + tag_len */
    le32 dst_data_len;
};

struct virtio_crypto_aead_data_vlf_stateless {
    /* Device read only portion */

    /* The cipher key */
    u8 key[key_len];
    /* Initialization Vector data. */
    u8 iv[iv_len];
    /* Source data */
    u8 src_data[src_data_len];
    /* Additional authenticated data if exists */
    u8 aad[aad_len];

    /* Device write only portion */
    /* Pointer to output data */
    u8 dst_data[dst_data_len];
};
\end{lstlisting}

\drivernormative{\paragraph}{AEAD Service Operation}{Device Types / Crypto Device / Device Operation / AEAD Service Operation}

\begin{itemize*}
\item If the driver uses the session mode, then the driver MUST set
    \field{session_id} in struct virtio_crypto_op_header to a valid value assigned
    by the device when the session was created.
\item If the VIRTIO_CRYPTO_F_AEAD_STATELESS_MODE feature bit is negotiated, 1) if
    the driver uses the stateless mode, then the driver MUST set the \field{flag}
    field in struct virtio_crypto_op_header to ZERO and MUST set the fields in
    struct virtio_crypto_aead_data_flf_stateless.sess_para, 2) if the driver uses
    the session mode, then the driver MUST set the \field{flag} field in struct
    virtio_crypto_op_header to VIRTIO_CRYPTO_FLAG_SESSION_MODE.
\item The driver MUST set the \field{opcode} field in struct virtio_crypto_op_header
    to VIRTIO_CRYPTO_AEAD_ENCRYPT or VIRTIO_CRYPTO_AEAD_DECRYPT.
\end{itemize*}

\devicenormative{\paragraph}{AEAD Service Operation}{Device Types / Crypto Device / Device Operation / AEAD Service Operation}

\begin{itemize*}
\item If the VIRTIO_CRYPTO_F_AEAD_STATELESS_MODE feature bit is negotiated, the
    device MUST parse the virtio_crypto_aead_data_vlf_stateless based on the \field{opcode}
	field in general header.
\item The device MUST copy the result of cryptographic operation in the dst_data[].
\item The device MUST copy the authentication tag in the dst_data[] offset the cipher result.
\item The device MUST set the \field{status} field in struct virtio_crypto_inhdr to
    one of the following values of enum VIRTIO_CRYPTO_STATUS:
\item When the \field{opcode} field is VIRTIO_CRYPTO_AEAD_DECRYPT, the device MUST
    verify and return the verification result to the driver.
\begin{itemize*}
\item VIRTIO_CRYPTO_OK if the operation success.
\item VIRTIO_CRYPTO_NOTSUPP if the requested algorithm or operation is unsupported.
\item VIRTIO_CRYPTO_BADMSG if the verification result is incorrect.
\item VIRTIO_CRYPTO_INVSESS if the session ID invalid when in session mode.
\item VIRTIO_CRYPTO_ERR if any failure not mentioned above occurs.
\end{itemize*}
\end{itemize*}

\subsubsection{AKCIPHER Service Operation}\label{sec:Device Types / Crypto Device / Device Operation / AKCIPHER Service Operation}

Session mode AKCIPHER requests are as follows:

\begin{lstlisting}
struct virtio_crypto_akcipher_data_flf {
    /* length of source data */
    le32 src_data_len;
    /* length of dst data */
    le32 dst_data_len;
};

struct virtio_crypto_akcipher_data_vlf {
    /* Device read only portion */
    /* Source data */
    u8 src_data[src_data_len];

    /* Device write only portion */
    /* Pointer to output data */
    u8 dst_data[dst_data_len];
};
\end{lstlisting}

Each data request uses the virtio_crypto_akcipher_flf structure and the virtio_crypto_akcipher_data_vlf
structure to store information used to run the AKCIPHER operations.

For encryption, decryption, and signing:
\field{src_data} is the source data that will be processed, note that for signing operations,
src_data stores the data to be signed, which usually is the digest of some data rather than the
data itself.
\field{src_data_len} is the length of source data.
\field{dst_result} is the result data and \field{dst_data_len} is the length of it. Note that the
length of the result is not always exactly equal to dst_data_len, the driver needs to check how
many bytes the device has written and calculate the actual length of the result.

For verification:
\field{src_data_len} refers to the length of the signature, and \field{dst_data_len} refers to
the length of signed data, where the signed data is usually the digest of some data.
\field{src_data} is spliced by the signature and the signed data, the src_data with the lower
address stores the signature, and the higher address stores the signed data.
\field{dst_data} is always empty for verification.

Different algorithms have different signature formats.
For the RSA algorithm, the result is determined by the padding algorithm specified by
\field{padding_algo} in structure virtio_crypto_rsa_session_para.

For the ECDSA algorithm, the signature is composed of the following
ASN.1 structure (see \hyperref[intro:rfc3279]{RFC3279})
and MUST be DER encoded (see \hyperref[intro:rfc6025]{rfc6025}).

\begin{lstlisting}
Ecdsa-Sig-Value ::= SEQUENCE {
    r INTEGER,
    s INTEGER
}
\end{lstlisting}

Stateless mode AKCIPHER service requests are as follows:
\begin{lstlisting}
struct virtio_crypto_akcipher_data_flf_stateless {
    struct {
        /* See VIRTIO_CYRPTO_AKCIPHER* above */
        le32 algo;
        /* See VIRTIO_CRYPTO_AKCIPHER_KEY_TYPE_* above */
        le32 key_type;
        /* length of key */
        le32 key_len;

        /* algothrim specific parameters described above */
        union {
            struct virtio_crypto_rsa_session_para rsa;
            struct virtio_crypto_ecdsa_session_para ecdsa;
        } u;
    } sess_para;

    /* length of source data */
    le32 src_data_len;
    /* length of destination data */
    le32 dst_data_len;
};

struct virtio_crypto_akcipher_data_vlf_stateless {
    /* Device read only portion */
    u8 akcipher_key[key_len];

    /* Source data */
    u8 src_data[src_data_len];

    /* Device write only portion */
    u8 dst_data[dst_data_len];
};
\end{lstlisting}

In stateless mode, the format of key and signature, the meaning of src_data and dst_data, are all the same
with session mode.

\drivernormative{\paragraph}{AKCIPHER Service Operation}{Device Types / Crypto Device / Device Operation / AKCIPHER Service Operation}

\begin{itemize*}
\item If the driver uses the session mode, then the driver MUST set
    \field{session_id} in struct virtio_crypto_op_header to a valid
    value assigned by the device when the session was created.
\item If the VIRTIO_CRYPTO_F_AKCIPHER_STATELESS_MODE feature bit is negotiated, 1) if the
    driver uses the stateless mode, then the driver MUST set the \field{flag} field in
    struct virtio_crypto_op_header to ZERO and MUST set the fields in struct
    virtio_crypto_akcipher_flf_stateless.sess_para, 2) if the driver uses the session
    mode, then the driver MUST set the \field{flag} field in struct virtio_crypto_op_header
    to VIRTIO_CRYPTO_FLAG_SESSION_MODE.
\item The driver MUST set the \field{opcode} field in struct virtio_crypto_op_header
    to one of VIRTIO_CRYPTO_AKCIPHER_ENCRYPT, VIRTIO_CRYPTO_AKCIPHER_DESTROY_SESSION,
    VIRTIO_CRYPTO_AKCIPHER_SIGN, and VIRTIO_CRYPTO_AKCIPHER_VERIFY.
\end{itemize*}

\devicenormative{\paragraph}{AKCIPHER Service Operation}{Device Types / Crypto Device / Device Operation / AKCIPHER Service Operation}

\begin{itemize*}
\item If the VIRTIO_CRYPTO_F_AKCIPHER_STATELESS_MODE feature bit is negotiated, the
    device MUST parse the virtio_crypto_akcipher_data_vlf_stateless based on the \field{opcode}
    field in general header.
\item The device MUST copy the result of cryptographic operation in the dst_data[].
\item The device MUST set the \field{status} field in struct virtio_crypto_inhdr to
    one of the following values of enum VIRTIO_CRYPTO_STATUS:
\begin{itemize*}
\item VIRTIO_CRYPTO_OK if the operation success.
\item VIRTIO_CRYPTO_NOTSUPP if the requested algorithm or operation is unsupported.
\item VIRTIO_CRYPTO_BADMSG if the verification result is incorrect.
\item VIRTIO_CRYPTO_INVSESS if the session ID invalid when in session mode.
\item VIRTIO_CRYPTO_KEY_REJECTED if the signature verification failed.
\item VIRTIO_CRYPTO_ERR if any failure not mentioned above occurs.
\end{itemize*}
\end{itemize*}

\section{Crypto Device}\label{sec:Device Types / Crypto Device}

The virtio crypto device is a virtual cryptography device as well as a
virtual cryptographic accelerator. The virtio crypto device provides the
following crypto services: CIPHER, MAC, HASH, AEAD and AKCIPHER. Virtio crypto
devices have a single control queue and at least one data queue. Crypto
operation requests are placed into a data queue, and serviced by the
device. Some crypto operation requests are only valid in the context of a
session. The role of the control queue is facilitating control operation
requests. Sessions management is realized with control operation
requests.

\subsection{Device ID}\label{sec:Device Types / Crypto Device / Device ID}

20

\subsection{Virtqueues}\label{sec:Device Types / Crypto Device / Virtqueues}

\begin{description}
\item[0] dataq1
\item[\ldots]
\item[N-1] dataqN
\item[N] controlq
\end{description}

N is set by \field{max_dataqueues}.

\subsection{Feature bits}\label{sec:Device Types / Crypto Device / Feature bits}

\begin{description}
\item VIRTIO_CRYPTO_F_REVISION_1 (0) revision 1. Revision 1 has a specific
    request format and other enhancements (which result in some additional
    requirements).
\item VIRTIO_CRYPTO_F_CIPHER_STATELESS_MODE (1) stateless mode requests are
    supported by the CIPHER service.
\item VIRTIO_CRYPTO_F_HASH_STATELESS_MODE (2) stateless mode requests are
    supported by the HASH service.
\item VIRTIO_CRYPTO_F_MAC_STATELESS_MODE (3) stateless mode requests are
    supported by the MAC service.
\item VIRTIO_CRYPTO_F_AEAD_STATELESS_MODE (4) stateless mode requests are
    supported by the AEAD service.
\item VIRTIO_CRYPTO_F_AKCIPHER_STATELESS_MODE (5) stateless mode requests are
    supported by the AKCIPHER service.
\end{description}


\subsubsection{Feature bit requirements}\label{sec:Device Types / Crypto Device / Feature bit requirements}

Some crypto feature bits require other crypto feature bits
(see \ref{drivernormative:Basic Facilities of a Virtio Device / Feature Bits}):

\begin{description}
\item[VIRTIO_CRYPTO_F_CIPHER_STATELESS_MODE] Requires VIRTIO_CRYPTO_F_REVISION_1.
\item[VIRTIO_CRYPTO_F_HASH_STATELESS_MODE] Requires VIRTIO_CRYPTO_F_REVISION_1.
\item[VIRTIO_CRYPTO_F_MAC_STATELESS_MODE] Requires VIRTIO_CRYPTO_F_REVISION_1.
\item[VIRTIO_CRYPTO_F_AEAD_STATELESS_MODE] Requires VIRTIO_CRYPTO_F_REVISION_1.
\item[VIRTIO_CRYPTO_F_AKCIPHER_STATELESS_MODE] Requires VIRTIO_CRYPTO_F_REVISION_1.
\end{description}

\subsection{Supported crypto services}\label{sec:Device Types / Crypto Device / Supported crypto services}

The following crypto services are defined:

\begin{lstlisting}
/* CIPHER (Symmetric Key Cipher) service */
#define VIRTIO_CRYPTO_SERVICE_CIPHER 0
/* HASH service */
#define VIRTIO_CRYPTO_SERVICE_HASH   1
/* MAC (Message Authentication Codes) service */
#define VIRTIO_CRYPTO_SERVICE_MAC    2
/* AEAD (Authenticated Encryption with Associated Data) service */
#define VIRTIO_CRYPTO_SERVICE_AEAD   3
/* AKCIPHER (Asymmetric Key Cipher) service */
#define VIRTIO_CRYPTO_SERVICE_AKCIPHER 4
\end{lstlisting}

The above constants designate bits used to indicate the which of crypto services are
offered by the device as described in, see \ref{sec:Device Types / Crypto Device / Device configuration layout}.

\subsubsection{CIPHER services}\label{sec:Device Types / Crypto Device / Supported crypto services / CIPHER services}

The following CIPHER algorithms are defined:

\begin{lstlisting}
#define VIRTIO_CRYPTO_NO_CIPHER                 0
#define VIRTIO_CRYPTO_CIPHER_ARC4               1
#define VIRTIO_CRYPTO_CIPHER_AES_ECB            2
#define VIRTIO_CRYPTO_CIPHER_AES_CBC            3
#define VIRTIO_CRYPTO_CIPHER_AES_CTR            4
#define VIRTIO_CRYPTO_CIPHER_DES_ECB            5
#define VIRTIO_CRYPTO_CIPHER_DES_CBC            6
#define VIRTIO_CRYPTO_CIPHER_3DES_ECB           7
#define VIRTIO_CRYPTO_CIPHER_3DES_CBC           8
#define VIRTIO_CRYPTO_CIPHER_3DES_CTR           9
#define VIRTIO_CRYPTO_CIPHER_KASUMI_F8          10
#define VIRTIO_CRYPTO_CIPHER_SNOW3G_UEA2        11
#define VIRTIO_CRYPTO_CIPHER_AES_F8             12
#define VIRTIO_CRYPTO_CIPHER_AES_XTS            13
#define VIRTIO_CRYPTO_CIPHER_ZUC_EEA3           14
\end{lstlisting}

The above constants have two usages:
\begin{enumerate}
\item As bit numbers, used to tell the driver which CIPHER algorithms
are supported by the device, see \ref{sec:Device Types / Crypto Device / Device configuration layout}.
\item As values, used to designate the algorithm in (CIPHER type) crypto
operation requests, see \ref{sec:Device Types / Crypto Device / Device Operation / Control Virtqueue / Session operation}.
\end{enumerate}

\subsubsection{HASH services}\label{sec:Device Types / Crypto Device / Supported crypto services / HASH services}

The following HASH algorithms are defined:

\begin{lstlisting}
#define VIRTIO_CRYPTO_NO_HASH            0
#define VIRTIO_CRYPTO_HASH_MD5           1
#define VIRTIO_CRYPTO_HASH_SHA1          2
#define VIRTIO_CRYPTO_HASH_SHA_224       3
#define VIRTIO_CRYPTO_HASH_SHA_256       4
#define VIRTIO_CRYPTO_HASH_SHA_384       5
#define VIRTIO_CRYPTO_HASH_SHA_512       6
#define VIRTIO_CRYPTO_HASH_SHA3_224      7
#define VIRTIO_CRYPTO_HASH_SHA3_256      8
#define VIRTIO_CRYPTO_HASH_SHA3_384      9
#define VIRTIO_CRYPTO_HASH_SHA3_512      10
#define VIRTIO_CRYPTO_HASH_SHA3_SHAKE128      11
#define VIRTIO_CRYPTO_HASH_SHA3_SHAKE256      12
\end{lstlisting}

The above constants have two usages:
\begin{enumerate}
\item As bit numbers, used to tell the driver which HASH algorithms
are supported by the device, see \ref{sec:Device Types / Crypto Device / Device configuration layout}.
\item As values, used to designate the algorithm in (HASH type) crypto
operation requires, see \ref{sec:Device Types / Crypto Device / Device Operation / Control Virtqueue / Session operation}.
\end{enumerate}

\subsubsection{MAC services}\label{sec:Device Types / Crypto Device / Supported crypto services / MAC services}

The following MAC algorithms are defined:

\begin{lstlisting}
#define VIRTIO_CRYPTO_NO_MAC                       0
#define VIRTIO_CRYPTO_MAC_HMAC_MD5                 1
#define VIRTIO_CRYPTO_MAC_HMAC_SHA1                2
#define VIRTIO_CRYPTO_MAC_HMAC_SHA_224             3
#define VIRTIO_CRYPTO_MAC_HMAC_SHA_256             4
#define VIRTIO_CRYPTO_MAC_HMAC_SHA_384             5
#define VIRTIO_CRYPTO_MAC_HMAC_SHA_512             6
#define VIRTIO_CRYPTO_MAC_CMAC_3DES                25
#define VIRTIO_CRYPTO_MAC_CMAC_AES                 26
#define VIRTIO_CRYPTO_MAC_KASUMI_F9                27
#define VIRTIO_CRYPTO_MAC_SNOW3G_UIA2              28
#define VIRTIO_CRYPTO_MAC_GMAC_AES                 41
#define VIRTIO_CRYPTO_MAC_GMAC_TWOFISH             42
#define VIRTIO_CRYPTO_MAC_CBCMAC_AES               49
#define VIRTIO_CRYPTO_MAC_CBCMAC_KASUMI_F9         50
#define VIRTIO_CRYPTO_MAC_XCBC_AES                 53
#define VIRTIO_CRYPTO_MAC_ZUC_EIA3                 54
\end{lstlisting}

The above constants have two usages:
\begin{enumerate}
\item As bit numbers, used to tell the driver which MAC algorithms
are supported by the device, see \ref{sec:Device Types / Crypto Device / Device configuration layout}.
\item As values, used to designate the algorithm in (MAC type) crypto
operation requests, see \ref{sec:Device Types / Crypto Device / Device Operation / Control Virtqueue / Session operation}.
\end{enumerate}

\subsubsection{AEAD services}\label{sec:Device Types / Crypto Device / Supported crypto services / AEAD services}

The following AEAD algorithms are defined:

\begin{lstlisting}
#define VIRTIO_CRYPTO_NO_AEAD     0
#define VIRTIO_CRYPTO_AEAD_GCM    1
#define VIRTIO_CRYPTO_AEAD_CCM    2
#define VIRTIO_CRYPTO_AEAD_CHACHA20_POLY1305  3
\end{lstlisting}

The above constants have two usages:
\begin{enumerate}
\item As bit numbers, used to tell the driver which AEAD algorithms
are supported by the device, see \ref{sec:Device Types / Crypto Device / Device configuration layout}.
\item As values, used to designate the algorithm in (DEAD type) crypto
operation requests, see \ref{sec:Device Types / Crypto Device / Device Operation / Control Virtqueue / Session operation}.
\end{enumerate}

\subsubsection{AKCIPHER services}\label{sec: Device Types / Crypto Device / Supported crypto services / AKCIPHER services}

The following AKCIPHER algorithms are defined:
\begin{lstlisting}
#define VIRTIO_CRYPTO_NO_AKCIPHER 0
#define VIRTIO_CRYPTO_AKCIPHER_RSA   1
#define VIRTIO_CRYPTO_AKCIPHER_ECDSA 2
\end{lstlisting}

The above constants have two usages:
\begin{enumerate}
\item As bit numbers, used to tell the driver which AKCIPHER algorithms
are supported by the device, see \ref{sec:Device Types / Crypto Device / Device configuration layout}.
\item As values, used to designate the algorithm in asymmetric crypto operation requests,
see \ref{sec:Device Types / Crypto Device / Device Operation / Control Virtqueue / Session operation}.
\end{enumerate}


\subsection{Device configuration layout}\label{sec:Device Types / Crypto Device / Device configuration layout}

Crypto device configuration uses the following layout structure:

\begin{lstlisting}
struct virtio_crypto_config {
    le32 status;
    le32 max_dataqueues;
    le32 crypto_services;
    /* Detailed algorithms mask */
    le32 cipher_algo_l;
    le32 cipher_algo_h;
    le32 hash_algo;
    le32 mac_algo_l;
    le32 mac_algo_h;
    le32 aead_algo;
    /* Maximum length of cipher key in bytes */
    le32 max_cipher_key_len;
    /* Maximum length of authenticated key in bytes */
    le32 max_auth_key_len;
    le32 akcipher_algo;
    /* Maximum size of each crypto request's content in bytes */
    le64 max_size;
};
\end{lstlisting}

\begin{description}
\item Currently, only one \field{status} bit is defined: VIRTIO_CRYPTO_S_HW_READY
    set indicates that the device is ready to process requests, this bit is read-only
    for the driver
\begin{lstlisting}
#define VIRTIO_CRYPTO_S_HW_READY  (1 << 0)
\end{lstlisting}

\item [\field{max_dataqueues}] is the maximum number of data virtqueues that can
    be configured by the device. The driver MAY use only one data queue, or it
    can use more to achieve better performance.

\item [\field{crypto_services}] crypto service offered, see \ref{sec:Device Types / Crypto Device / Supported crypto services}.

\item [\field{cipher_algo_l}] CIPHER algorithms bits 0-31, see \ref{sec:Device Types / Crypto Device / Supported crypto services  / CIPHER services}.

\item [\field{cipher_algo_h}] CIPHER algorithms bits 32-63, see \ref{sec:Device Types / Crypto Device / Supported crypto services  / CIPHER services}.

\item [\field{hash_algo}] HASH algorithms bits, see \ref{sec:Device Types / Crypto Device / Supported crypto services  / HASH services}.

\item [\field{mac_algo_l}] MAC algorithms bits 0-31, see \ref{sec:Device Types / Crypto Device / Supported crypto services  / MAC services}.

\item [\field{mac_algo_h}] MAC algorithms bits 32-63, see \ref{sec:Device Types / Crypto Device / Supported crypto services  / MAC services}.

\item [\field{aead_algo}] AEAD algorithms bits, see \ref{sec:Device Types / Crypto Device / Supported crypto services  / AEAD services}.

\item [\field{max_cipher_key_len}] is the maximum length of cipher key supported by the device.

\item [\field{max_auth_key_len}] is the maximum length of authenticated key supported by the device.

\item [\field{akcipher_algo}] AKCIPHER algorithms bit 0-31, see \ref{sec: Device Types / Crypto Device / Supported crypto services / AKCIPHER services}.

\item [\field{max_size}] is the maximum size of the variable-length parameters of
    data operation of each crypto request's content supported by the device.
\end{description}

\begin{note}
Unless explicitly stated otherwise all lengths and sizes are in bytes.
\end{note}

\devicenormative{\subsubsection}{Device configuration layout}{Device Types / Crypto Device / Device configuration layout}

\begin{itemize*}
\item The device MUST set \field{max_dataqueues} to between 1 and 65535 inclusive.
\item The device MUST set the \field{status} with valid flags, undefined flags MUST NOT be set.
\item The device MUST accept and handle requests after \field{status} is set to VIRTIO_CRYPTO_S_HW_READY.
\item The device MUST set \field{crypto_services} based on the crypto services the device offers.
\item The device MUST set detailed algorithms masks for each service advertised by \field{crypto_services}.
    The device MUST NOT set the not defined algorithms bits.
\item The device MUST set \field{max_size} to show the maximum size of crypto request the device supports.
\item The device MUST set \field{max_cipher_key_len} to show the maximum length of cipher key if the
    device supports CIPHER service.
\item The device MUST set \field{max_auth_key_len} to show the maximum length of authenticated key if
    the device supports MAC service.
\end{itemize*}

\drivernormative{\subsubsection}{Device configuration layout}{Device Types / Crypto Device / Device configuration layout}

\begin{itemize*}
\item The driver MUST read the \field{status} from the bottom bit of status to check whether the
    VIRTIO_CRYPTO_S_HW_READY is set, and the driver MUST reread it after device reset.
\item The driver MUST NOT transmit any requests to the device if the VIRTIO_CRYPTO_S_HW_READY is not set.
\item The driver MUST read \field{max_dataqueues} field to discover the number of data queues the device supports.
\item The driver MUST read \field{crypto_services} field to discover which services the device is able to offer.
\item The driver SHOULD ignore the not defined algorithms bits.
\item The driver MUST read the detailed algorithms fields based on \field{crypto_services} field.
\item The driver SHOULD read \field{max_size} to discover the maximum size of the variable-length
    parameters of data operation of the crypto request's content the device supports and MUST
    guarantee that the size of each crypto request's content is within the \field{max_size}, otherwise
    the request will fail and the driver MUST reset the device.
\item The driver SHOULD read \field{max_cipher_key_len} to discover the maximum length of cipher key
    the device supports and MUST guarantee that the \field{key_len} (CIPHER service or AEAD service) is within
    the \field{max_cipher_key_len} of the device configuration, otherwise the request will fail.
\item The driver SHOULD read \field{max_auth_key_len} to discover the maximum length of authenticated
    key the device supports and MUST guarantee that the \field{auth_key_len} (MAC service) is within the
    \field{max_auth_key_len} of the device configuration, otherwise the request will fail.
\end{itemize*}

\subsection{Device Initialization}\label{sec:Device Types / Crypto Device / Device Initialization}

\drivernormative{\subsubsection}{Device Initialization}{Device Types / Crypto Device / Device Initialization}

\begin{itemize*}
\item The driver MUST configure and initialize all virtqueues.
\item The driver MUST read the supported crypto services from bits of \field{crypto_services}.
\item The driver MUST read the supported algorithms based on \field{crypto_services} field.
\end{itemize*}

\subsection{Device Operation}\label{sec:Device Types / Crypto Device / Device Operation}

The operation of a virtio crypto device is driven by requests placed on the virtqueues.
Requests consist of a queue-type specific header (specifying among others the operation)
and an operation specific payload.

If VIRTIO_CRYPTO_F_REVISION_1 is negotiated the device may support both session mode
(See \ref{sec:Device Types / Crypto Device / Device Operation / Control Virtqueue / Session operation})
and stateless mode operation requests.
In stateless mode all operation parameters are supplied as a part of each request,
while in session mode, some or all operation parameters are managed within the
session. Stateless mode is guarded by feature bits 0-4 on a service level. If
stateless mode is negotiated for a service, the service accepts both session
mode and stateless requests; otherwise stateless mode requests are rejected
(via operation status).

\subsubsection{Operation Status}\label{sec:Device Types / Crypto Device / Device Operation / Operation status}
The device MUST return a status code as part of the operation (both session
operation and service operation) result. The valid operation status as follows:

\begin{lstlisting}
enum VIRTIO_CRYPTO_STATUS {
    VIRTIO_CRYPTO_OK = 0,
    VIRTIO_CRYPTO_ERR = 1,
    VIRTIO_CRYPTO_BADMSG = 2,
    VIRTIO_CRYPTO_NOTSUPP = 3,
    VIRTIO_CRYPTO_INVSESS = 4,
    VIRTIO_CRYPTO_NOSPC = 5,
    VIRTIO_CRYPTO_KEY_REJECTED = 6,
    VIRTIO_CRYPTO_MAX
};
\end{lstlisting}

\begin{itemize*}
\item VIRTIO_CRYPTO_OK: success.
\item VIRTIO_CRYPTO_BADMSG: authentication failed (only when AEAD decryption).
\item VIRTIO_CRYPTO_NOTSUPP: operation or algorithm is unsupported.
\item VIRTIO_CRYPTO_INVSESS: invalid session ID when executing crypto operations.
\item VIRTIO_CRYPTO_NOSPC: no free session ID (only when the VIRTIO_CRYPTO_F_REVISION_1
    feature bit is negotiated).
\item VIRTIO_CRYPTO_KEY_REJECTED: signature verification failed (only when AKCIPHER verification).
\item VIRTIO_CRYPTO_ERR: any failure not mentioned above occurs.
\end{itemize*}

\subsubsection{Control Virtqueue}\label{sec:Device Types / Crypto Device / Device Operation / Control Virtqueue}

The driver uses the control virtqueue to send control commands to the
device, such as session operations (See \ref{sec:Device Types / Crypto Device / Device
Operation / Control Virtqueue / Session operation}).

The header for controlq is of the following form:
\begin{lstlisting}
#define VIRTIO_CRYPTO_OPCODE(service, op)   (((service) << 8) | (op))

struct virtio_crypto_ctrl_header {
#define VIRTIO_CRYPTO_CIPHER_CREATE_SESSION \
       VIRTIO_CRYPTO_OPCODE(VIRTIO_CRYPTO_SERVICE_CIPHER, 0x02)
#define VIRTIO_CRYPTO_CIPHER_DESTROY_SESSION \
       VIRTIO_CRYPTO_OPCODE(VIRTIO_CRYPTO_SERVICE_CIPHER, 0x03)
#define VIRTIO_CRYPTO_HASH_CREATE_SESSION \
       VIRTIO_CRYPTO_OPCODE(VIRTIO_CRYPTO_SERVICE_HASH, 0x02)
#define VIRTIO_CRYPTO_HASH_DESTROY_SESSION \
       VIRTIO_CRYPTO_OPCODE(VIRTIO_CRYPTO_SERVICE_HASH, 0x03)
#define VIRTIO_CRYPTO_MAC_CREATE_SESSION \
       VIRTIO_CRYPTO_OPCODE(VIRTIO_CRYPTO_SERVICE_MAC, 0x02)
#define VIRTIO_CRYPTO_MAC_DESTROY_SESSION \
       VIRTIO_CRYPTO_OPCODE(VIRTIO_CRYPTO_SERVICE_MAC, 0x03)
#define VIRTIO_CRYPTO_AEAD_CREATE_SESSION \
       VIRTIO_CRYPTO_OPCODE(VIRTIO_CRYPTO_SERVICE_AEAD, 0x02)
#define VIRTIO_CRYPTO_AEAD_DESTROY_SESSION \
       VIRTIO_CRYPTO_OPCODE(VIRTIO_CRYPTO_SERVICE_AEAD, 0x03)
#define VIRTIO_CRYPTO_AKCIPHER_CREATE_SESSION \
       VIRTIO_CRYPTO_OPCODE(VIRTIO_CRYPTO_SERVICE_AKCIPHER, 0x04)
#define VIRTIO_CRYPTO_AKCIPHER_DESTROY_SESSION \
       VIRTIO_CRYPTO_OPCDE(VIRTIO_CRYPTO_SERVICE_AKCIPHER, 0x05)
    le32 opcode;
    /* algo should be service-specific algorithms */
    le32 algo;
    le32 flag;
    le32 reserved;
};
\end{lstlisting}

The controlq request is composed of four parts:
\begin{lstlisting}
struct virtio_crypto_op_ctrl_req {
    /* Device read only portion */

    struct virtio_crypto_ctrl_header header;

#define VIRTIO_CRYPTO_CTRLQ_OP_SPEC_HDR_LEGACY 56
    /* fixed length fields, opcode specific */
    u8 op_flf[flf_len];

    /* variable length fields, opcode specific */
    u8 op_vlf[vlf_len];

    /* Device write only portion */

    /* op result or completion status */
    u8 op_outcome[outcome_len];
};
\end{lstlisting}

\field{header} is a general header (see above).

\field{op_flf} is the opcode (in \field{header}) specific fixed-length parameters.

\field{flf_len} depends on the VIRTIO_CRYPTO_F_REVISION_1 feature bit (see below).

\field{op_vlf} is the opcode (in \field{header}) specific variable-length parameters.

\field{vlf_len} is the size of the specific structure used.
\begin{note}
The \field{vlf_len} of session-destroy operation and the hash-session-create
operation is ZERO.
\end{note}

\begin{itemize*}
\item If the opcode (in \field{header}) is VIRTIO_CRYPTO_CIPHER_CREATE_SESSION
    then \field{op_flf} is struct virtio_crypto_sym_create_session_flf if
    VIRTIO_CRYPTO_F_REVISION_1 is negotiated and struct virtio_crypto_sym_create_session_flf is
    padded to 56 bytes if NOT negotiated, and \field{op_vlf} is struct
    virtio_crypto_sym_create_session_vlf.
\item If the opcode (in \field{header}) is VIRTIO_CRYPTO_HASH_CREATE_SESSION
    then \field{op_flf} is struct virtio_crypto_hash_create_session_flf if
    VIRTIO_CRYPTO_F_REVISION_1 is negotiated and struct virtio_crypto_hash_create_session_flf is
    padded to 56 bytes if NOT negotiated.
\item If the opcode (in \field{header}) is VIRTIO_CRYPTO_MAC_CREATE_SESSION
    then \field{op_flf} is struct virtio_crypto_mac_create_session_flf if
    VIRTIO_CRYPTO_F_REVISION_1 is negotiated and struct virtio_crypto_mac_create_session_flf is
    padded to 56 bytes if NOT negotiated, and \field{op_vlf} is struct
    virtio_crypto_mac_create_session_vlf.
\item If the opcode (in \field{header}) is VIRTIO_CRYPTO_AEAD_CREATE_SESSION
    then \field{op_flf} is struct virtio_crypto_aead_create_session_flf if
    VIRTIO_CRYPTO_F_REVISION_1 is negotiated and struct virtio_crypto_aead_create_session_flf is
    padded to 56 bytes if NOT negotiated, and \field{op_vlf} is struct
    virtio_crypto_aead_create_session_vlf.
\item If the opcode (in \field{header}) is VIRTIO_CRYPTO_AKCIPHER_CREATE_SESSION
    then \field{op_flf} is struct virtio_crypto_akcipher_create_session_flf if
    VIRTIO_CRYPTO_F_REVISION_1 is negotiated and struct virtio_crypto_akcipher_create_session_flf is
    padded to 56 bytes if NOT negotiated, and \field{op_vlf} is struct
    virtio_crypto_akcipher_create_session_vlf.
\item If the opcode (in \field{header}) is VIRTIO_CRYPTO_CIPHER_DESTROY_SESSION
    or VIRTIO_CRYPTO_HASH_DESTROY_SESSION or VIRTIO_CRYPTO_MAC_DESTROY_SESSION or
    VIRTIO_CRYPTO_AEAD_DESTROY_SESSION then \field{op_flf} is struct
    virtio_crypto_destroy_session_flf if VIRTIO_CRYPTO_F_REVISION_1 is negotiated and
    struct virtio_crypto_destroy_session_flf is padded to 56 bytes if NOT negotiated.
\end{itemize*}

\field{op_outcome} stores the result of operation and must be struct
virtio_crypto_destroy_session_input for destroy session or
struct virtio_crypto_create_session_input for create session.

\field{outcome_len} is the size of the structure used.


\paragraph{Session operation}\label{sec:Device Types / Crypto Device / Device
Operation / Control Virtqueue / Session operation}

The session is a handle which describes the cryptographic parameters to be
applied to a number of buffers.

The following structure stores the result of session creation set by the device:

\begin{lstlisting}
struct virtio_crypto_create_session_input {
    le64 session_id;
    le32 status;
    le32 padding;
};
\end{lstlisting}

A request to destroy a session includes the following information:

\begin{lstlisting}
struct virtio_crypto_destroy_session_flf {
    /* Device read only portion */
    le64  session_id;
};

struct virtio_crypto_destroy_session_input {
    /* Device write only portion */
    u8  status;
};
\end{lstlisting}


\subparagraph{Session operation: HASH session}\label{sec:Device Types / Crypto Device / Device
Operation / Control Virtqueue / Session operation / Session operation: HASH session}

The fixed-length parameters of HASH session requests is as follows:

\begin{lstlisting}
struct virtio_crypto_hash_create_session_flf {
    /* Device read only portion */

    /* See VIRTIO_CRYPTO_HASH_* above */
    le32 algo;
    /* hash result length */
    le32 hash_result_len;
};
\end{lstlisting}


\subparagraph{Session operation: MAC session}\label{sec:Device Types / Crypto Device / Device
Operation / Control Virtqueue / Session operation / Session operation: MAC session}

The fixed-length and the variable-length parameters of MAC session requests are as follows:

\begin{lstlisting}
struct virtio_crypto_mac_create_session_flf {
    /* Device read only portion */

    /* See VIRTIO_CRYPTO_MAC_* above */
    le32 algo;
    /* hash result length */
    le32 hash_result_len;
    /* length of authenticated key */
    le32 auth_key_len;
    le32 padding;
};

struct virtio_crypto_mac_create_session_vlf {
    /* Device read only portion */

    /* The authenticated key */
    u8 auth_key[auth_key_len];
};
\end{lstlisting}

The length of \field{auth_key} is specified in \field{auth_key_len} in the struct
virtio_crypto_mac_create_session_flf.


\subparagraph{Session operation: Symmetric algorithms session}\label{sec:Device Types / Crypto Device / Device
Operation / Control Virtqueue / Session operation / Session operation: Symmetric algorithms session}

The request of symmetric session could be the CIPHER algorithms request
or the chain algorithms (chaining CIPHER and HASH/MAC) request.

The fixed-length and the variable-length parameters of CIPHER session requests are as follows:

\begin{lstlisting}
struct virtio_crypto_cipher_session_flf {
    /* Device read only portion */

    /* See VIRTIO_CRYPTO_CIPHER* above */
    le32 algo;
    /* length of key */
    le32 key_len;
#define VIRTIO_CRYPTO_OP_ENCRYPT  1
#define VIRTIO_CRYPTO_OP_DECRYPT  2
    /* encryption or decryption */
    le32 op;
    le32 padding;
};

struct virtio_crypto_cipher_session_vlf {
    /* Device read only portion */

    /* The cipher key */
    u8 cipher_key[key_len];
};
\end{lstlisting}

The length of \field{cipher_key} is specified in \field{key_len} in the struct
virtio_crypto_cipher_session_flf.

The fixed-length and the variable-length parameters of Chain session requests are as follows:

\begin{lstlisting}
struct virtio_crypto_alg_chain_session_flf {
    /* Device read only portion */

#define VIRTIO_CRYPTO_SYM_ALG_CHAIN_ORDER_HASH_THEN_CIPHER  1
#define VIRTIO_CRYPTO_SYM_ALG_CHAIN_ORDER_CIPHER_THEN_HASH  2
    le32 alg_chain_order;
/* Plain hash */
#define VIRTIO_CRYPTO_SYM_HASH_MODE_PLAIN    1
/* Authenticated hash (mac) */
#define VIRTIO_CRYPTO_SYM_HASH_MODE_AUTH     2
/* Nested hash */
#define VIRTIO_CRYPTO_SYM_HASH_MODE_NESTED   3
    le32 hash_mode;
    struct virtio_crypto_cipher_session_flf cipher_hdr;

#define VIRTIO_CRYPTO_ALG_CHAIN_SESS_OP_SPEC_HDR_SIZE  16
    /* fixed length fields, algo specific */
    u8 algo_flf[VIRTIO_CRYPTO_ALG_CHAIN_SESS_OP_SPEC_HDR_SIZE];

    /* length of the additional authenticated data (AAD) in bytes */
    le32 aad_len;
    le32 padding;
};

struct virtio_crypto_alg_chain_session_vlf {
    /* Device read only portion */

    /* The cipher key */
    u8 cipher_key[key_len];
    /* The authenticated key */
    u8 auth_key[auth_key_len];
};
\end{lstlisting}

\field{hash_mode} decides the type used by \field{algo_flf}.

\field{algo_flf} is fixed to 16 bytes and MUST contains or be one of
the following types:
\begin{itemize*}
\item struct virtio_crypto_hash_create_session_flf
\item struct virtio_crypto_mac_create_session_flf
\end{itemize*}
The data of unused part (if has) in \field{algo_flf} will be ignored.

The length of \field{cipher_key} is specified in \field{key_len} in \field{cipher_hdr}.

The length of \field{auth_key} is specified in \field{auth_key_len} in struct
virtio_crypto_mac_create_session_flf.

The fixed-length parameters of Symmetric session requests are as follows:

\begin{lstlisting}
struct virtio_crypto_sym_create_session_flf {
    /* Device read only portion */

#define VIRTIO_CRYPTO_SYM_SESS_OP_SPEC_HDR_SIZE  48
    /* fixed length fields, opcode specific */
    u8 op_flf[VIRTIO_CRYPTO_SYM_SESS_OP_SPEC_HDR_SIZE];

/* No operation */
#define VIRTIO_CRYPTO_SYM_OP_NONE  0
/* Cipher only operation on the data */
#define VIRTIO_CRYPTO_SYM_OP_CIPHER  1
/* Chain any cipher with any hash or mac operation. The order
   depends on the value of alg_chain_order param */
#define VIRTIO_CRYPTO_SYM_OP_ALGORITHM_CHAINING  2
    le32 op_type;
    le32 padding;
};
\end{lstlisting}

\field{op_flf} is fixed to 48 bytes, MUST contains or be one of
the following types:
\begin{itemize*}
\item struct virtio_crypto_cipher_session_flf
\item struct virtio_crypto_alg_chain_session_flf
\end{itemize*}
The data of unused part (if has) in \field{op_flf} will be ignored.

\field{op_type} decides the type used by \field{op_flf}.

The variable-length parameters of Symmetric session requests are as follows:

\begin{lstlisting}
struct virtio_crypto_sym_create_session_vlf {
    /* Device read only portion */
    /* variable length fields, opcode specific */
    u8 op_vlf[vlf_len];
};
\end{lstlisting}

\field{op_vlf} MUST contains or be one of the following types:
\begin{itemize*}
\item struct virtio_crypto_cipher_session_vlf
\item struct virtio_crypto_alg_chain_session_vlf
\end{itemize*}

\field{op_type} in struct virtio_crypto_sym_create_session_flf decides the
type used by \field{op_vlf}.

\field{vlf_len} is the size of the specific structure used.


\subparagraph{Session operation: AEAD session}\label{sec:Device Types / Crypto Device / Device
Operation / Control Virtqueue / Session operation / Session operation: AEAD session}

The fixed-length and the variable-length parameters of AEAD session requests are as follows:

\begin{lstlisting}
struct virtio_crypto_aead_create_session_flf {
    /* Device read only portion */

    /* See VIRTIO_CRYPTO_AEAD_* above */
    le32 algo;
    /* length of key */
    le32 key_len;
    /* Authentication tag length */
    le32 tag_len;
    /* The length of the additional authenticated data (AAD) in bytes */
    le32 aad_len;
    /* encryption or decryption, See above VIRTIO_CRYPTO_OP_* */
    le32 op;
    le32 padding;
};

struct virtio_crypto_aead_create_session_vlf {
    /* Device read only portion */
    u8 key[key_len];
};
\end{lstlisting}

The length of \field{key} is specified in \field{key_len} in struct
virtio_crypto_aead_create_session_flf.

\subparagraph{Session operation: AKCIPHER session}\label{sec:Device Types / Crypto Device / Device
Operation / Control Virtqueue / Session operation / Session operation: AKCIPHER session}

Due to the complexity of asymmetric key algorithms, different algorithms
require different parameters. The following data structures are used as
supplementary parameters to describe the asymmetric algorithm sessions.

For the RSA algorithm, the extra parameters are as follows:
\begin{lstlisting}
struct virtio_crypto_rsa_session_para {
#define VIRTIO_CRYPTO_RSA_RAW_PADDING   0
#define VIRTIO_CRYPTO_RSA_PKCS1_PADDING 1
    le32 padding_algo;

#define VIRTIO_CRYPTO_RSA_NO_HASH   0
#define VIRTIO_CRYPTO_RSA_MD2       1
#define VIRTIO_CRYPTO_RSA_MD3       2
#define VIRTIO_CRYPTO_RSA_MD4       3
#define VIRTIO_CRYPTO_RSA_MD5       4
#define VIRTIO_CRYPTO_RSA_SHA1      5
#define VIRTIO_CRYPTO_RSA_SHA256    6
#define VIRTIO_CRYPTO_RSA_SHA384    7
#define VIRTIO_CRYPTO_RSA_SHA512    8
#define VIRTIO_CRYPTO_RSA_SHA224    9
    le32 hash_algo;
};
\end{lstlisting}

\field{padding_algo} specifies the padding method used by RSA sessions.
\begin{itemize*}
\item If VIRTIO_CRYPTO_RSA_RAW_PADDING is specified, 1) \field{hash_algo}
is ignored, 2) ciphertext and plaintext MUST be padded with leading zeros,
3) and RSA sessions with VIRTIO_CRYPTO_RSA_RAW_PADDING MUST not be used
for verification and signing operations.
\item If VIRTIO_CRYPTO_RSA_PKCS1_PADDING is specified, EMSA-PKCS1-v1_5 padding method
is used (see \hyperref[intro:rfc3447]{PKCS\#1}), \field{hash_algo} specifies how the
digest of the data passed to RSA sessions is calculated when verifying and signing.
It only affects the padding algorithm and is ignored during encryption and decryption.
\end{itemize*}

The ECC algorithms such as the ECDSA algorithm, cannot use custom curves, only the
following known curves can be used (see \hyperref[intro:NIST]{NIST-recommended curves}).

\begin{lstlisting}
#define VIRTIO_CRYPTO_CURVE_UNKNOWN   0
#define VIRTIO_CRYPTO_CURVE_NIST_P192 1
#define VIRTIO_CRYPTO_CURVE_NIST_P224 2
#define VIRTIO_CRYPTO_CURVE_NIST_P256 3
#define VIRTIO_CRYPTO_CURVE_NIST_P384 4
#define VIRTIO_CRYPTO_CURVE_NIST_P521 5
\end{lstlisting}

For the ECDSA algorithm, the extra parameters are as follows:
\begin{lstlisting}
struct virtio_crypto_ecdsa_session_para {
    /* See VIRTIO_CRYPTO_CURVE_* above */
    le32 curve_id;
};
\end{lstlisting}

The fixed-length and the variable-length parameters of AKCIPHER session requests are as follows:
\begin{lstlisting}
struct virtio_crypto_akcipher_create_session_flf {
    /* Device read only portion */

    /* See VIRTIO_CRYPTO_AKCIPHER_* above */
    le32 algo;
#define VIRTIO_CRYPTO_AKCIPHER_KEY_TYPE_PUBLIC 1
#define VIRTIO_CRYPTO_AKCIPHER_KEY_TYPE_PRIVATE 2
    le32 key_type;
    /* length of key */
    le32 key_len;

#define VIRTIO_CRYPTO_AKCIPHER_SESS_ALGO_SPEC_HDR_SIZE 44
    u8 algo_flf[VIRTIO_CRYPTO_AKCIPHER_SESS_ALGO_SPEC_HDR_SIZE];
};

struct virtio_crypto_akcipher_create_session_vlf {
    /* Device read only portion */
    u8 key[key_len];
};
\end{lstlisting}

\field{algo} decides the type used by \field{algo_flf}.
\field{algo_flf} is fixed to 44 bytes and MUST contains of be one the
following structures:
\begin{itemize*}
\item struct virtio_crypto_rsa_session_para
\item struct virtio_crypto_ecdsa_session_para
\end{itemize*}

The length of \field{key} is specified in \field{key_len} in the struct
virtio_crypto_akcipher_create_session_flf.

For the RSA algorithm, the key needs to be encoded according to
\hyperref[intro:rfc3447]{PKCS\#1}. The private key is described with the
RSAPrivateKey structure, and the public key is described with the RSAPublicKey
structure. These ASN.1 structures are encoded in DER encoding rules (see
\hyperref[intro:rfc6025]{rfc6025}).

\begin{lstlisting}
RSAPrivateKey ::= SEQUENCE {
    version          INTEGER,
    modulus          INTEGER,
    publicExponent   INTEGER,
    privateExponent  INTEGER,
    prime1           INTEGER,
    prime2           INTEGER,
    exponent1        INTEGER,
    exponent1        INTEGER,
    coefficient      INTEGER,
    otherPrimeInfos  OtherPrimeInfos OPTIONAL
}

OtherPrimeInfos ::= SEQUENCE SIZE(1...MAX) OF OtherPrimeInfo

OtherPrimeINfo ::= SEQUENCE {
    prime           INTEGER,
    exponent        INTEGER,
    coefficient     INTEGER
}

RSAPublicKey ::= SEQUENCE {
    modulus         INTEGER,
    publicExponent  INTEGER
}
\end{lstlisting}

For the ECDSA algorithm, the private key is encoded according to
\hyperref[intro:rfc5915]{RFC5915}, the private key of the ECDSA algorithm
is described by the ASN.1 structure ECPrivateKey and encoded with DER
encoding rules (see \hyperref[intro:rfc6025]{rfc6025}).

\begin{lstlisting}
ECPrivateKey ::= SEQUNCE {
    version         INTEGER,
    privateKey      OCTET STRING,
    parameters [0]  ECParameters {{ NamedCurve }} OPTIONAL,
    publicKey  [1]  BIT STRING OPTIONAL
}
\end{lstlisting}

The public key of the ECDSA algorithm is encoded according to \hyperref[intro:SEC1]{SEC1},
and the public key of ECDSA is described by the ASN.1 structure ECPoint.
When initializing a session with ECDSA public key, the ECPoint is DER encoded and the
\field{key} only contains the value part of ECPoint, that is, the header part of the
OCTET STRING will be omitted (see \hyperref[intro:rfc6025]{rfc6025}).

\begin{lstlisting}
ECPoint ::= OCTET STRING
\end{lstlisting}

The length of \field{key} is specified in \field{key_len} in
struct virtio_crypto_akcipher_create_session_flf.

\drivernormative{\subparagraph}{Session operation: create session}{Device Types / Crypto Device / Device
Operation / Control Virtqueue / Session operation / Session operation: create session}

\begin{itemize*}
\item The driver MUST set the \field{opcode} field based on service type: CIPHER, HASH, MAC, AEAD or AKCIPHER.
\item The driver MUST set the control general header, the opcode specific header,
    the opcode specific extra parameters and the opcode specific outcome buffer in turn.
    See \ref{sec:Device Types / Crypto Device / Device Operation / Control Virtqueue}.
\item The driver MUST set the \field{reversed} field to zero.
\end{itemize*}

\devicenormative{\subparagraph}{Session operation: create session}{Device Types / Crypto Device / Device
Operation / Control Virtqueue / Session operation / Session operation: create session}

\begin{itemize*}
\item The device MUST use the corresponding opcode specific structure according to the
    \field{opcode} in the control general header.
\item The device MUST extract extra parameters according to the structures used.
\item The device MUST set the \field{status} field to one of the following values of enum
    VIRTIO_CRYPTO_STATUS after finish a session creation:
\begin{itemize*}
\item VIRTIO_CRYPTO_OK if a session is created successfully.
\item VIRTIO_CRYPTO_NOTSUPP if the requested algorithm or operation is unsupported.
\item VIRTIO_CRYPTO_NOSPC if no free session ID (only when the VIRTIO_CRYPTO_F_REVISION_1
    feature bit is negotiated).
\item VIRTIO_CRYPTO_ERR if failure not mentioned above occurs.
\end{itemize*}
\item The device MUST set the \field{session_id} field to a unique session identifier only
    if the status is set to VIRTIO_CRYPTO_OK.
\end{itemize*}

\drivernormative{\subparagraph}{Session operation: destroy session}{Device Types / Crypto Device / Device
Operation / Control Virtqueue / Session operation / Session operation: destroy session}

\begin{itemize*}
\item The driver MUST set the \field{opcode} field based on service type: CIPHER, HASH, MAC, AEAD or AKCIPHER.
\item The driver MUST set the \field{session_id} to a valid value assigned by the device
    when the session was created.
\end{itemize*}

\devicenormative{\subparagraph}{Session operation: destroy session}{Device Types / Crypto Device / Device
Operation / Control Virtqueue / Session operation / Session operation: destroy session}

\begin{itemize*}
\item The device MUST set the \field{status} field to one of the following values of enum VIRTIO_CRYPTO_STATUS.
\begin{itemize*}
\item VIRTIO_CRYPTO_OK if a session is created successfully.
\item VIRTIO_CRYPTO_ERR if any failure occurs.
\end{itemize*}
\end{itemize*}


\subsubsection{Data Virtqueue}\label{sec:Device Types / Crypto Device / Device Operation / Data Virtqueue}

The driver uses the data virtqueues to transmit crypto operation requests to the device,
and completes the crypto operations.

The header for dataq is as follows:

\begin{lstlisting}
struct virtio_crypto_op_header {
#define VIRTIO_CRYPTO_CIPHER_ENCRYPT \
    VIRTIO_CRYPTO_OPCODE(VIRTIO_CRYPTO_SERVICE_CIPHER, 0x00)
#define VIRTIO_CRYPTO_CIPHER_DECRYPT \
    VIRTIO_CRYPTO_OPCODE(VIRTIO_CRYPTO_SERVICE_CIPHER, 0x01)
#define VIRTIO_CRYPTO_HASH \
    VIRTIO_CRYPTO_OPCODE(VIRTIO_CRYPTO_SERVICE_HASH, 0x00)
#define VIRTIO_CRYPTO_MAC \
    VIRTIO_CRYPTO_OPCODE(VIRTIO_CRYPTO_SERVICE_MAC, 0x00)
#define VIRTIO_CRYPTO_AEAD_ENCRYPT \
    VIRTIO_CRYPTO_OPCODE(VIRTIO_CRYPTO_SERVICE_AEAD, 0x00)
#define VIRTIO_CRYPTO_AEAD_DECRYPT \
    VIRTIO_CRYPTO_OPCODE(VIRTIO_CRYPTO_SERVICE_AEAD, 0x01)
#define VIRTIO_CRYPTO_AKCIPHER_ENCRYPT \
    VIRTIO_CRYPTO_OPCODE(VIRTIO_CRYPTO_SERVICE_AKCIPHER, 0x00)
#define VIRTIO_CRYPTO_AKCIPHER_DECRYPT \
    VIRTIO_CRYPTO_OPCODE(VIRTIO_CRYPTO_SERVICE_AKCIPHER, 0x01)
#define VIRTIO_CRYPTO_AKCIPHER_SIGN \
    VIRTIO_CRYPTO_OPCODE(VIRTIO_CRYPTO_SERVICE_AKCIPHER, 0x02)
#define VIRTIO_CRYPTO_AKCIPHER_VERIFY \
    VIRTIO_CRYPTO_OPCODE(VIRTIO_CRYPTO_SERVICE_AKCIPHER, 0x03)
    le32 opcode;
    /* algo should be service-specific algorithms */
    le32 algo;
    le64 session_id;
#define VIRTIO_CRYPTO_FLAG_SESSION_MODE 1
    /* control flag to control the request */
    le32 flag;
    le32 padding;
};
\end{lstlisting}

\begin{note}
If VIRTIO_CRYPTO_F_REVISION_1 is not negotiated the \field{flag} is ignored.

If VIRTIO_CRYPTO_F_REVISION_1 is negotiated but VIRTIO_CRYPTO_F_<SERVICE>_STATELESS_MODE
is not negotiated, then the device SHOULD reject <SERVICE> requests if
VIRTIO_CRYPTO_FLAG_SESSION_MODE is not set (in \field{flag}).
\end{note}

The dataq request is composed of four parts:
\begin{lstlisting}
struct virtio_crypto_op_data_req {
    /* Device read only portion */

    struct virtio_crypto_op_header header;

#define VIRTIO_CRYPTO_DATAQ_OP_SPEC_HDR_LEGACY 48
    /* fixed length fields, opcode specific */
    u8 op_flf[flf_len];

    /* Device read && write portion */
    /* variable length fields, opcode specific */
    u8 op_vlf[vlf_len];

    /* Device write only portion */
    struct virtio_crypto_inhdr inhdr;
};
\end{lstlisting}

\field{header} is a general header (see above).

\field{op_flf} is the opcode (in \field{header}) specific header.

\field{flf_len} depends on the VIRTIO_CRYPTO_F_REVISION_1 feature bit
(see below).

\field{op_vlf} is the opcode (in \field{header}) specific parameters.

\field{vlf_len} is the size of the specific structure used.

\begin{itemize*}
\item If the the opcode (in \field{header}) is VIRTIO_CRYPTO_CIPHER_ENCRYPT
    or VIRTIO_CRYPTO_CIPHER_DECRYPT then:
    \begin{itemize*}
    \item If VIRTIO_CRYPTO_F_CIPHER_STATELESS_MODE is negotiated, \field{op_flf} is
        struct virtio_crypto_sym_data_flf_stateless, and \field{op_vlf} is struct
        virtio_crypto_sym_data_vlf_stateless.
    \item If VIRTIO_CRYPTO_F_CIPHER_STATELESS_MODE is NOT negotiated, \field{op_flf}
        is struct virtio_crypto_sym_data_flf if VIRTIO_CRYPTO_F_REVISION_1 is negotiated
        and struct virtio_crypto_sym_data_flf is padded to 48 bytes if NOT negotiated,
        and \field{op_vlf} is struct virtio_crypto_sym_data_vlf.
    \end{itemize*}
\item If the the opcode (in \field{header}) is VIRTIO_CRYPTO_HASH:
    \begin{itemize*}
    \item If VIRTIO_CRYPTO_F_HASH_STATELESS_MODE is negotiated, \field{op_flf} is
        struct virtio_crypto_hash_data_flf_stateless, and \field{op_vlf} is struct
        virtio_crypto_hash_data_vlf_stateless.
    \item If VIRTIO_CRYPTO_F_HASH_STATELESS_MODE is NOT negotiated, \field{op_flf}
        is struct virtio_crypto_hash_data_flf if VIRTIO_CRYPTO_F_REVISION_1 is negotiated
        and struct virtio_crypto_hash_data_flf is padded to 48 bytes if NOT negotiated,
        and \field{op_vlf} is struct virtio_crypto_hash_data_vlf.
    \end{itemize*}
\item If the the opcode (in \field{header}) is VIRTIO_CRYPTO_MAC:
    \begin{itemize*}
    \item If VIRTIO_CRYPTO_F_MAC_STATELESS_MODE is negotiated, \field{op_flf} is
        struct virtio_crypto_mac_data_flf_stateless, and \field{op_vlf} is struct
        virtio_crypto_mac_data_vlf_stateless.
    \item If VIRTIO_CRYPTO_F_MAC_STATELESS_MODE is NOT negotiated, \field{op_flf}
        is struct virtio_crypto_mac_data_flf if VIRTIO_CRYPTO_F_REVISION_1 is negotiated
        and struct virtio_crypto_mac_data_flf is padded to 48 bytes if NOT negotiated,
        and \field{op_vlf} is struct virtio_crypto_mac_data_vlf.
    \end{itemize*}
\item If the the opcode (in \field{header}) is VIRTIO_CRYPTO_AEAD_ENCRYPT
    or VIRTIO_CRYPTO_AEAD_DECRYPT then:
    \begin{itemize*}
    \item If VIRTIO_CRYPTO_F_AEAD_STATELESS_MODE is negotiated, \field{op_flf} is
        struct virtio_crypto_aead_data_flf_stateless, and \field{op_vlf} is struct
        virtio_crypto_aead_data_vlf_stateless.
    \item If VIRTIO_CRYPTO_F_AEAD_STATELESS_MODE is NOT negotiated, \field{op_flf}
        is struct virtio_crypto_aead_data_flf if VIRTIO_CRYPTO_F_REVISION_1 is negotiated
        and struct virtio_crypto_aead_data_flf is padded to 48 bytes if NOT negotiated,
        and \field{op_vlf} is struct virtio_crypto_aead_data_vlf.
    \end{itemize*}
\item If the opcode (in \field{header}) is VIRTIO_CRYPTO_AKCIPHER_ENCRYPT, VIRTIO_CRYPTO_AKCIPHER_DECRYPT,
    VIRTIO_CRYPTO_AKCIPHER_SIGN or VIRTIO_CRYPTO_AKCIPHER_VERIFY then:
    \begin{itemize*}
    \item If VIRTIO_CRYPTO_F_AKCIPHER_STATELESS_MODE is negotiated, \field{op_flf} is
        struct virtio_crypto_akcipher_data_flf_statless, and \field{op_vlf} is struct
        virtio_crypto_akcipher_data_vlf_stateless.
    \item If VIRTIO_CRYPTO_F_AKCIPHER_STATELESS_MODE is NOT negotiated, \field{op_flf}
        is struct virtio_crypto_akcipher_data_flf if VIRTIO_CRYPTO_F_REVISION_1 is negotiated
        and struct virtio_crypto_akcipher_data_flf is padded to 48 bytes if NOT negotiated,
        and \field{op_vlf} is struct virtio_crypto_akcipher_data_vlf.
    \end{itemize*}
\end{itemize*}

\field{inhdr} is a unified input header that used to return the status of
the operations, is defined as follows:

\begin{lstlisting}
struct virtio_crypto_inhdr {
    u8 status;
};
\end{lstlisting}

\subsubsection{HASH Service Operation}\label{sec:Device Types / Crypto Device / Device Operation / HASH Service Operation}

Session mode HASH service requests are as follows:

\begin{lstlisting}
struct virtio_crypto_hash_data_flf {
    /* length of source data */
    le32 src_data_len;
    /* hash result length */
    le32 hash_result_len;
};

struct virtio_crypto_hash_data_vlf {
    /* Device read only portion */
    /* Source data */
    u8 src_data[src_data_len];

    /* Device write only portion */
    /* Hash result data */
    u8 hash_result[hash_result_len];
};
\end{lstlisting}

Each data request uses the virtio_crypto_hash_data_flf structure and the
virtio_crypto_hash_data_vlf structure to store information used to run the
HASH operations.

\field{src_data} is the source data that will be processed.
\field{src_data_len} is the length of source data.
\field{hash_result} is the result data and \field{hash_result_len} is the length
of it.

Stateless mode HASH service requests are as follows:

\begin{lstlisting}
struct virtio_crypto_hash_data_flf_stateless {
    struct {
        /* See VIRTIO_CRYPTO_HASH_* above */
        le32 algo;
    } sess_para;

    /* length of source data */
    le32 src_data_len;
    /* hash result length */
    le32 hash_result_len;
    le32 reserved;
};
struct virtio_crypto_hash_data_vlf_stateless {
    /* Device read only portion */
    /* Source data */
    u8 src_data[src_data_len];

    /* Device write only portion */
    /* Hash result data */
    u8 hash_result[hash_result_len];
};
\end{lstlisting}

\drivernormative{\paragraph}{HASH Service Operation}{Device Types / Crypto Device / Device Operation / HASH Service Operation}

\begin{itemize*}
\item If the driver uses the session mode, then the driver MUST set \field{session_id}
    in struct virtio_crypto_op_header to a valid value assigned by the device when the
    session was created.
\item If the VIRTIO_CRYPTO_F_HASH_STATELESS_MODE feature bit is negotiated, 1) if the
    driver uses the stateless mode, then the driver MUST set the \field{flag} field in
    struct virtio_crypto_op_header to ZERO and MUST set the fields in struct
    virtio_crypto_hash_data_flf_stateless.sess_para, 2) if the driver uses the session
    mode, then the driver MUST set the \field{flag} field in struct virtio_crypto_op_header
    to VIRTIO_CRYPTO_FLAG_SESSION_MODE.
\item The driver MUST set \field{opcode} in struct virtio_crypto_op_header to VIRTIO_CRYPTO_HASH.
\end{itemize*}

\devicenormative{\paragraph}{HASH Service Operation}{Device Types / Crypto Device / Device Operation / HASH Service Operation}

\begin{itemize*}
\item The device MUST use the corresponding structure according to the \field{opcode}
    in the data general header.
\item If the VIRTIO_CRYPTO_F_HASH_STATELESS_MODE feature bit is negotiated, the device
    MUST parse \field{flag} field in struct virtio_crypto_op_header in order to decide
    which mode the driver uses.
\item The device MUST copy the results of HASH operations in the hash_result[] if HASH
    operations success.
\item The device MUST set \field{status} in struct virtio_crypto_inhdr to one of the
    following values of enum VIRTIO_CRYPTO_STATUS:
\begin{itemize*}
\item VIRTIO_CRYPTO_OK if the operation success.
\item VIRTIO_CRYPTO_NOTSUPP if the requested algorithm or operation is unsupported.
\item VIRTIO_CRYPTO_INVSESS if the session ID invalid when in session mode.
\item VIRTIO_CRYPTO_ERR if any failure not mentioned above occurs.
\end{itemize*}
\end{itemize*}


\subsubsection{MAC Service Operation}\label{sec:Device Types / Crypto Device / Device Operation / MAC Service Operation}

Session mode MAC service requests are as follows:

\begin{lstlisting}
struct virtio_crypto_mac_data_flf {
    struct virtio_crypto_hash_data_flf hdr;
};

struct virtio_crypto_mac_data_vlf {
    /* Device read only portion */
    /* Source data */
    u8 src_data[src_data_len];

    /* Device write only portion */
    /* Hash result data */
    u8 hash_result[hash_result_len];
};
\end{lstlisting}

Each request uses the virtio_crypto_mac_data_flf structure and the
virtio_crypto_mac_data_vlf structure to store information used to run the
MAC operations.

\field{src_data} is the source data that will be processed.
\field{src_data_len} is the length of source data.
\field{hash_result} is the result data and \field{hash_result_len} is the length
of it.

Stateless mode MAC service requests are as follows:

\begin{lstlisting}
struct virtio_crypto_mac_data_flf_stateless {
    struct {
        /* See VIRTIO_CRYPTO_MAC_* above */
        le32 algo;
        /* length of authenticated key */
        le32 auth_key_len;
    } sess_para;

    /* length of source data */
    le32 src_data_len;
    /* hash result length */
    le32 hash_result_len;
};

struct virtio_crypto_mac_data_vlf_stateless {
    /* Device read only portion */
    /* The authenticated key */
    u8 auth_key[auth_key_len];
    /* Source data */
    u8 src_data[src_data_len];

    /* Device write only portion */
    /* Hash result data */
    u8 hash_result[hash_result_len];
};
\end{lstlisting}

\field{auth_key} is the authenticated key that will be used during the process.
\field{auth_key_len} is the length of the key.

\drivernormative{\paragraph}{MAC Service Operation}{Device Types / Crypto Device / Device Operation / MAC Service Operation}

\begin{itemize*}
\item If the driver uses the session mode, then the driver MUST set \field{session_id}
    in struct virtio_crypto_op_header to a valid value assigned by the device when the
    session was created.
\item If the VIRTIO_CRYPTO_F_MAC_STATELESS_MODE feature bit is negotiated, 1) if the
    driver uses the stateless mode, then the driver MUST set the \field{flag} field
    in struct virtio_crypto_op_header to ZERO and MUST set the fields in struct
    virtio_crypto_mac_data_flf_stateless.sess_para, 2) if the driver uses the session
    mode, then the driver MUST set the \field{flag} field in struct virtio_crypto_op_header
    to VIRTIO_CRYPTO_FLAG_SESSION_MODE.
\item The driver MUST set \field{opcode} in struct virtio_crypto_op_header to VIRTIO_CRYPTO_MAC.
\end{itemize*}

\devicenormative{\paragraph}{MAC Service Operation}{Device Types / Crypto Device / Device Operation / MAC Service Operation}

\begin{itemize*}
\item If the VIRTIO_CRYPTO_F_MAC_STATELESS_MODE feature bit is negotiated, the device
    MUST parse \field{flag} field in struct virtio_crypto_op_header in order to decide
	which mode the driver uses.
\item The device MUST copy the results of MAC operations in the hash_result[] if HASH
    operations success.
\item The device MUST set \field{status} in struct virtio_crypto_inhdr to one of the
    following values of enum VIRTIO_CRYPTO_STATUS:
\begin{itemize*}
\item VIRTIO_CRYPTO_OK if the operation success.
\item VIRTIO_CRYPTO_NOTSUPP if the requested algorithm or operation is unsupported.
\item VIRTIO_CRYPTO_INVSESS if the session ID invalid when in session mode.
\item VIRTIO_CRYPTO_ERR if any failure not mentioned above occurs.
\end{itemize*}
\end{itemize*}

\subsubsection{Symmetric algorithms Operation}\label{sec:Device Types / Crypto Device / Device Operation / Symmetric algorithms Operation}

Session mode CIPHER service requests are as follows:

\begin{lstlisting}
struct virtio_crypto_cipher_data_flf {
    /*
     * Byte Length of valid IV/Counter data pointed to by the below iv data.
     *
     * For block ciphers in CBC or F8 mode, or for Kasumi in F8 mode, or for
     *   SNOW3G in UEA2 mode, this is the length of the IV (which
     *   must be the same as the block length of the cipher).
     * For block ciphers in CTR mode, this is the length of the counter
     *   (which must be the same as the block length of the cipher).
     */
    le32 iv_len;
    /* length of source data */
    le32 src_data_len;
    /* length of destination data */
    le32 dst_data_len;
    le32 padding;
};

struct virtio_crypto_cipher_data_vlf {
    /* Device read only portion */

    /*
     * Initialization Vector or Counter data.
     *
     * For block ciphers in CBC or F8 mode, or for Kasumi in F8 mode, or for
     *   SNOW3G in UEA2 mode, this is the Initialization Vector (IV)
     *   value.
     * For block ciphers in CTR mode, this is the counter.
     * For AES-XTS, this is the 128bit tweak, i, from IEEE Std 1619-2007.
     *
     * The IV/Counter will be updated after every partial cryptographic
     * operation.
     */
    u8 iv[iv_len];
    /* Source data */
    u8 src_data[src_data_len];

    /* Device write only portion */
    /* Destination data */
    u8 dst_data[dst_data_len];
};
\end{lstlisting}

Session mode requests of algorithm chaining are as follows:

\begin{lstlisting}
struct virtio_crypto_alg_chain_data_flf {
    le32 iv_len;
    /* Length of source data */
    le32 src_data_len;
    /* Length of destination data */
    le32 dst_data_len;
    /* Starting point for cipher processing in source data */
    le32 cipher_start_src_offset;
    /* Length of the source data that the cipher will be computed on */
    le32 len_to_cipher;
    /* Starting point for hash processing in source data */
    le32 hash_start_src_offset;
    /* Length of the source data that the hash will be computed on */
    le32 len_to_hash;
    /* Length of the additional auth data */
    le32 aad_len;
    /* Length of the hash result */
    le32 hash_result_len;
    le32 reserved;
};

struct virtio_crypto_alg_chain_data_vlf {
    /* Device read only portion */

    /* Initialization Vector or Counter data */
    u8 iv[iv_len];
    /* Source data */
    u8 src_data[src_data_len];
    /* Additional authenticated data if exists */
    u8 aad[aad_len];

    /* Device write only portion */

    /* Destination data */
    u8 dst_data[dst_data_len];
    /* Hash result data */
    u8 hash_result[hash_result_len];
};
\end{lstlisting}

Session mode requests of symmetric algorithm are as follows:

\begin{lstlisting}
struct virtio_crypto_sym_data_flf {
    /* Device read only portion */

#define VIRTIO_CRYPTO_SYM_DATA_REQ_HDR_SIZE    40
    u8 op_type_flf[VIRTIO_CRYPTO_SYM_DATA_REQ_HDR_SIZE];

    /* See above VIRTIO_CRYPTO_SYM_OP_* */
    le32 op_type;
    le32 padding;
};

struct virtio_crypto_sym_data_vlf {
    u8 op_type_vlf[sym_para_len];
};
\end{lstlisting}

Each request uses the virtio_crypto_sym_data_flf structure and the
virtio_crypto_sym_data_flf structure to store information used to run the
CIPHER operations.

\field{op_type_flf} is the \field{op_type} specific header, it MUST starts
with or be one of the following structures:
\begin{itemize*}
\item struct virtio_crypto_cipher_data_flf
\item struct virtio_crypto_alg_chain_data_flf
\end{itemize*}

The length of \field{op_type_flf} is fixed to 40 bytes, the data of unused
part (if has) will be ignored.

\field{op_type_vlf} is the \field{op_type} specific parameters, it MUST starts
with or be one of the following structures:
\begin{itemize*}
\item struct virtio_crypto_cipher_data_vlf
\item struct virtio_crypto_alg_chain_data_vlf
\end{itemize*}

\field{sym_para_len} is the size of the specific structure used.

Stateless mode CIPHER service requests are as follows:

\begin{lstlisting}
struct virtio_crypto_cipher_data_flf_stateless {
    struct {
        /* See VIRTIO_CRYPTO_CIPHER* above */
        le32 algo;
        /* length of key */
        le32 key_len;

        /* See VIRTIO_CRYPTO_OP_* above */
        le32 op;
    } sess_para;

    /*
     * Byte Length of valid IV/Counter data pointed to by the below iv data.
     */
    le32 iv_len;
    /* length of source data */
    le32 src_data_len;
    /* length of destination data */
    le32 dst_data_len;
};

struct virtio_crypto_cipher_data_vlf_stateless {
    /* Device read only portion */

    /* The cipher key */
    u8 cipher_key[key_len];

    /* Initialization Vector or Counter data. */
    u8 iv[iv_len];
    /* Source data */
    u8 src_data[src_data_len];

    /* Device write only portion */
    /* Destination data */
    u8 dst_data[dst_data_len];
};
\end{lstlisting}

Stateless mode requests of algorithm chaining are as follows:

\begin{lstlisting}
struct virtio_crypto_alg_chain_data_flf_stateless {
    struct {
        /* See VIRTIO_CRYPTO_SYM_ALG_CHAIN_ORDER_* above */
        le32 alg_chain_order;
        /* length of the additional authenticated data in bytes */
        le32 aad_len;

        struct {
            /* See VIRTIO_CRYPTO_CIPHER* above */
            le32 algo;
            /* length of key */
            le32 key_len;
            /* See VIRTIO_CRYPTO_OP_* above */
            le32 op;
        } cipher;

        struct {
            /* See VIRTIO_CRYPTO_HASH_* or VIRTIO_CRYPTO_MAC_* above */
            le32 algo;
            /* length of authenticated key */
            le32 auth_key_len;
            /* See VIRTIO_CRYPTO_SYM_HASH_MODE_* above */
            le32 hash_mode;
        } hash;
    } sess_para;

    le32 iv_len;
    /* Length of source data */
    le32 src_data_len;
    /* Length of destination data */
    le32 dst_data_len;
    /* Starting point for cipher processing in source data */
    le32 cipher_start_src_offset;
    /* Length of the source data that the cipher will be computed on */
    le32 len_to_cipher;
    /* Starting point for hash processing in source data */
    le32 hash_start_src_offset;
    /* Length of the source data that the hash will be computed on */
    le32 len_to_hash;
    /* Length of the additional auth data */
    le32 aad_len;
    /* Length of the hash result */
    le32 hash_result_len;
    le32 reserved;
};

struct virtio_crypto_alg_chain_data_vlf_stateless {
    /* Device read only portion */

    /* The cipher key */
    u8 cipher_key[key_len];
    /* The auth key */
    u8 auth_key[auth_key_len];
    /* Initialization Vector or Counter data */
    u8 iv[iv_len];
    /* Additional authenticated data if exists */
    u8 aad[aad_len];
    /* Source data */
    u8 src_data[src_data_len];

    /* Device write only portion */

    /* Destination data */
    u8 dst_data[dst_data_len];
    /* Hash result data */
    u8 hash_result[hash_result_len];
};
\end{lstlisting}

Stateless mode requests of symmetric algorithm are as follows:

\begin{lstlisting}
struct virtio_crypto_sym_data_flf_stateless {
    /* Device read only portion */
#define VIRTIO_CRYPTO_SYM_DATE_REQ_HDR_STATELESS_SIZE    72
    u8 op_type_flf[VIRTIO_CRYPTO_SYM_DATE_REQ_HDR_STATELESS_SIZE];

    /* Device write only portion */
    /* See above VIRTIO_CRYPTO_SYM_OP_* */
    le32 op_type;
};

struct virtio_crypto_sym_data_vlf_stateless {
    u8 op_type_vlf[sym_para_len];
};
\end{lstlisting}

\field{op_type_flf} is the \field{op_type} specific header, it MUST starts
with or be one of the following structures:
\begin{itemize*}
\item struct virtio_crypto_cipher_data_flf_stateless
\item struct virtio_crypto_alg_chain_data_flf_stateless
\end{itemize*}

The length of \field{op_type_flf} is fixed to 72 bytes, the data of unused
part (if has) will be ignored.

\field{op_type_vlf} is the \field{op_type} specific parameters, it MUST starts
with or be one of the following structures:
\begin{itemize*}
\item struct virtio_crypto_cipher_data_vlf_stateless
\item struct virtio_crypto_alg_chain_data_vlf_stateless
\end{itemize*}

\field{sym_para_len} is the size of the specific structure used.

\drivernormative{\paragraph}{Symmetric algorithms Operation}{Device Types / Crypto Device / Device Operation / Symmetric algorithms Operation}

\begin{itemize*}
\item If the driver uses the session mode, then the driver MUST set \field{session_id}
    in struct virtio_crypto_op_header to a valid value assigned by the device when the
    session was created.
\item If the VIRTIO_CRYPTO_F_CIPHER_STATELESS_MODE feature bit is negotiated, 1) if the
    driver uses the stateless mode, then the driver MUST set the \field{flag} field in
    struct virtio_crypto_op_header to ZERO and MUST set the fields in struct
    virtio_crypto_cipher_data_flf_stateless.sess_para or struct
    virtio_crypto_alg_chain_data_flf_stateless.sess_para, 2) if the driver uses the
    session mode, then the driver MUST set the \field{flag} field in struct
    virtio_crypto_op_header to VIRTIO_CRYPTO_FLAG_SESSION_MODE.
\item The driver MUST set the \field{opcode} field in struct virtio_crypto_op_header
    to VIRTIO_CRYPTO_CIPHER_ENCRYPT or VIRTIO_CRYPTO_CIPHER_DECRYPT.
\item The driver MUST specify the fields of struct virtio_crypto_cipher_data_flf in
    struct virtio_crypto_sym_data_flf and struct virtio_crypto_cipher_data_vlf in
    struct virtio_crypto_sym_data_vlf if the request is based on VIRTIO_CRYPTO_SYM_OP_CIPHER.
\item The driver MUST specify the fields of struct virtio_crypto_alg_chain_data_flf
    in struct virtio_crypto_sym_data_flf and struct virtio_crypto_alg_chain_data_vlf
    in struct virtio_crypto_sym_data_vlf if the request is of the VIRTIO_CRYPTO_SYM_OP_ALGORITHM_CHAINING
    type.
\end{itemize*}

\devicenormative{\paragraph}{Symmetric algorithms Operation}{Device Types / Crypto Device / Device Operation / Symmetric algorithms Operation}

\begin{itemize*}
\item If the VIRTIO_CRYPTO_F_CIPHER_STATELESS_MODE feature bit is negotiated, the device
    MUST parse \field{flag} field in struct virtio_crypto_op_header in order to decide
	which mode the driver uses.
\item The device MUST parse the virtio_crypto_sym_data_req based on the \field{opcode}
    field in general header.
\item The device MUST parse the fields of struct virtio_crypto_cipher_data_flf in
    struct virtio_crypto_sym_data_flf and struct virtio_crypto_cipher_data_vlf in
    struct virtio_crypto_sym_data_vlf if the request is based on VIRTIO_CRYPTO_SYM_OP_CIPHER.
\item The device MUST parse the fields of struct virtio_crypto_alg_chain_data_flf
    in struct virtio_crypto_sym_data_flf and struct virtio_crypto_alg_chain_data_vlf
    in struct virtio_crypto_sym_data_vlf if the request is of the VIRTIO_CRYPTO_SYM_OP_ALGORITHM_CHAINING
    type.
\item The device MUST copy the result of cryptographic operation in the dst_data[] in
    both plain CIPHER mode and algorithms chain mode.
\item The device MUST check the \field{para}.\field{add_len} is bigger than 0 before
    parse the additional authenticated data in plain algorithms chain mode.
\item The device MUST copy the result of HASH/MAC operation in the hash_result[] is
    of the VIRTIO_CRYPTO_SYM_OP_ALGORITHM_CHAINING type.
\item The device MUST set the \field{status} field in struct virtio_crypto_inhdr to
    one of the following values of enum VIRTIO_CRYPTO_STATUS:
\begin{itemize*}
\item VIRTIO_CRYPTO_OK if the operation success.
\item VIRTIO_CRYPTO_NOTSUPP if the requested algorithm or operation is unsupported.
\item VIRTIO_CRYPTO_INVSESS if the session ID is invalid in session mode.
\item VIRTIO_CRYPTO_ERR if failure not mentioned above occurs.
\end{itemize*}
\end{itemize*}

\subsubsection{AEAD Service Operation}\label{sec:Device Types / Crypto Device / Device Operation / AEAD Service Operation}

Session mode requests of symmetric algorithm are as follows:

\begin{lstlisting}
struct virtio_crypto_aead_data_flf {
    /*
     * Byte Length of valid IV data.
     *
     * For GCM mode, this is either 12 (for 96-bit IVs) or 16, in which
     *   case iv points to J0.
     * For CCM mode, this is the length of the nonce, which can be in the
     *   range 7 to 13 inclusive.
     */
    le32 iv_len;
    /* length of additional auth data */
    le32 aad_len;
    /* length of source data */
    le32 src_data_len;
    /* length of dst data, this should be at least src_data_len + tag_len */
    le32 dst_data_len;
    /* Authentication tag length */
    le32 tag_len;
    le32 reserved;
};

struct virtio_crypto_aead_data_vlf {
    /* Device read only portion */

    /*
     * Initialization Vector data.
     *
     * For GCM mode, this is either the IV (if the length is 96 bits) or J0
     *   (for other sizes), where J0 is as defined by NIST SP800-38D.
     *   Regardless of the IV length, a full 16 bytes needs to be allocated.
     * For CCM mode, the first byte is reserved, and the nonce should be
     *   written starting at &iv[1] (to allow space for the implementation
     *   to write in the flags in the first byte).  Note that a full 16 bytes
     *   should be allocated, even though the iv_len field will have
     *   a value less than this.
     *
     * The IV will be updated after every partial cryptographic operation.
     */
    u8 iv[iv_len];
    /* Source data */
    u8 src_data[src_data_len];
    /* Additional authenticated data if exists */
    u8 aad[aad_len];

    /* Device write only portion */
    /* Pointer to output data */
    u8 dst_data[dst_data_len];
};
\end{lstlisting}

Each request uses the virtio_crypto_aead_data_flf structure and the
virtio_crypto_aead_data_flf structure to store information used to run the
AEAD operations.

Stateless mode AEAD service requests are as follows:

\begin{lstlisting}
struct virtio_crypto_aead_data_flf_stateless {
    struct {
        /* See VIRTIO_CRYPTO_AEAD_* above */
        le32 algo;
        /* length of key */
        le32 key_len;
        /* encrypt or decrypt, See above VIRTIO_CRYPTO_OP_* */
        le32 op;
    } sess_para;

    /* Byte Length of valid IV data. */
    le32 iv_len;
    /* Authentication tag length */
    le32 tag_len;
    /* length of additional auth data */
    le32 aad_len;
    /* length of source data */
    le32 src_data_len;
    /* length of dst data, this should be at least src_data_len + tag_len */
    le32 dst_data_len;
};

struct virtio_crypto_aead_data_vlf_stateless {
    /* Device read only portion */

    /* The cipher key */
    u8 key[key_len];
    /* Initialization Vector data. */
    u8 iv[iv_len];
    /* Source data */
    u8 src_data[src_data_len];
    /* Additional authenticated data if exists */
    u8 aad[aad_len];

    /* Device write only portion */
    /* Pointer to output data */
    u8 dst_data[dst_data_len];
};
\end{lstlisting}

\drivernormative{\paragraph}{AEAD Service Operation}{Device Types / Crypto Device / Device Operation / AEAD Service Operation}

\begin{itemize*}
\item If the driver uses the session mode, then the driver MUST set
    \field{session_id} in struct virtio_crypto_op_header to a valid value assigned
    by the device when the session was created.
\item If the VIRTIO_CRYPTO_F_AEAD_STATELESS_MODE feature bit is negotiated, 1) if
    the driver uses the stateless mode, then the driver MUST set the \field{flag}
    field in struct virtio_crypto_op_header to ZERO and MUST set the fields in
    struct virtio_crypto_aead_data_flf_stateless.sess_para, 2) if the driver uses
    the session mode, then the driver MUST set the \field{flag} field in struct
    virtio_crypto_op_header to VIRTIO_CRYPTO_FLAG_SESSION_MODE.
\item The driver MUST set the \field{opcode} field in struct virtio_crypto_op_header
    to VIRTIO_CRYPTO_AEAD_ENCRYPT or VIRTIO_CRYPTO_AEAD_DECRYPT.
\end{itemize*}

\devicenormative{\paragraph}{AEAD Service Operation}{Device Types / Crypto Device / Device Operation / AEAD Service Operation}

\begin{itemize*}
\item If the VIRTIO_CRYPTO_F_AEAD_STATELESS_MODE feature bit is negotiated, the
    device MUST parse the virtio_crypto_aead_data_vlf_stateless based on the \field{opcode}
	field in general header.
\item The device MUST copy the result of cryptographic operation in the dst_data[].
\item The device MUST copy the authentication tag in the dst_data[] offset the cipher result.
\item The device MUST set the \field{status} field in struct virtio_crypto_inhdr to
    one of the following values of enum VIRTIO_CRYPTO_STATUS:
\item When the \field{opcode} field is VIRTIO_CRYPTO_AEAD_DECRYPT, the device MUST
    verify and return the verification result to the driver.
\begin{itemize*}
\item VIRTIO_CRYPTO_OK if the operation success.
\item VIRTIO_CRYPTO_NOTSUPP if the requested algorithm or operation is unsupported.
\item VIRTIO_CRYPTO_BADMSG if the verification result is incorrect.
\item VIRTIO_CRYPTO_INVSESS if the session ID invalid when in session mode.
\item VIRTIO_CRYPTO_ERR if any failure not mentioned above occurs.
\end{itemize*}
\end{itemize*}

\subsubsection{AKCIPHER Service Operation}\label{sec:Device Types / Crypto Device / Device Operation / AKCIPHER Service Operation}

Session mode AKCIPHER requests are as follows:

\begin{lstlisting}
struct virtio_crypto_akcipher_data_flf {
    /* length of source data */
    le32 src_data_len;
    /* length of dst data */
    le32 dst_data_len;
};

struct virtio_crypto_akcipher_data_vlf {
    /* Device read only portion */
    /* Source data */
    u8 src_data[src_data_len];

    /* Device write only portion */
    /* Pointer to output data */
    u8 dst_data[dst_data_len];
};
\end{lstlisting}

Each data request uses the virtio_crypto_akcipher_flf structure and the virtio_crypto_akcipher_data_vlf
structure to store information used to run the AKCIPHER operations.

For encryption, decryption, and signing:
\field{src_data} is the source data that will be processed, note that for signing operations,
src_data stores the data to be signed, which usually is the digest of some data rather than the
data itself.
\field{src_data_len} is the length of source data.
\field{dst_result} is the result data and \field{dst_data_len} is the length of it. Note that the
length of the result is not always exactly equal to dst_data_len, the driver needs to check how
many bytes the device has written and calculate the actual length of the result.

For verification:
\field{src_data_len} refers to the length of the signature, and \field{dst_data_len} refers to
the length of signed data, where the signed data is usually the digest of some data.
\field{src_data} is spliced by the signature and the signed data, the src_data with the lower
address stores the signature, and the higher address stores the signed data.
\field{dst_data} is always empty for verification.

Different algorithms have different signature formats.
For the RSA algorithm, the result is determined by the padding algorithm specified by
\field{padding_algo} in structure virtio_crypto_rsa_session_para.

For the ECDSA algorithm, the signature is composed of the following
ASN.1 structure (see \hyperref[intro:rfc3279]{RFC3279})
and MUST be DER encoded (see \hyperref[intro:rfc6025]{rfc6025}).

\begin{lstlisting}
Ecdsa-Sig-Value ::= SEQUENCE {
    r INTEGER,
    s INTEGER
}
\end{lstlisting}

Stateless mode AKCIPHER service requests are as follows:
\begin{lstlisting}
struct virtio_crypto_akcipher_data_flf_stateless {
    struct {
        /* See VIRTIO_CYRPTO_AKCIPHER* above */
        le32 algo;
        /* See VIRTIO_CRYPTO_AKCIPHER_KEY_TYPE_* above */
        le32 key_type;
        /* length of key */
        le32 key_len;

        /* algothrim specific parameters described above */
        union {
            struct virtio_crypto_rsa_session_para rsa;
            struct virtio_crypto_ecdsa_session_para ecdsa;
        } u;
    } sess_para;

    /* length of source data */
    le32 src_data_len;
    /* length of destination data */
    le32 dst_data_len;
};

struct virtio_crypto_akcipher_data_vlf_stateless {
    /* Device read only portion */
    u8 akcipher_key[key_len];

    /* Source data */
    u8 src_data[src_data_len];

    /* Device write only portion */
    u8 dst_data[dst_data_len];
};
\end{lstlisting}

In stateless mode, the format of key and signature, the meaning of src_data and dst_data, are all the same
with session mode.

\drivernormative{\paragraph}{AKCIPHER Service Operation}{Device Types / Crypto Device / Device Operation / AKCIPHER Service Operation}

\begin{itemize*}
\item If the driver uses the session mode, then the driver MUST set
    \field{session_id} in struct virtio_crypto_op_header to a valid
    value assigned by the device when the session was created.
\item If the VIRTIO_CRYPTO_F_AKCIPHER_STATELESS_MODE feature bit is negotiated, 1) if the
    driver uses the stateless mode, then the driver MUST set the \field{flag} field in
    struct virtio_crypto_op_header to ZERO and MUST set the fields in struct
    virtio_crypto_akcipher_flf_stateless.sess_para, 2) if the driver uses the session
    mode, then the driver MUST set the \field{flag} field in struct virtio_crypto_op_header
    to VIRTIO_CRYPTO_FLAG_SESSION_MODE.
\item The driver MUST set the \field{opcode} field in struct virtio_crypto_op_header
    to one of VIRTIO_CRYPTO_AKCIPHER_ENCRYPT, VIRTIO_CRYPTO_AKCIPHER_DESTROY_SESSION,
    VIRTIO_CRYPTO_AKCIPHER_SIGN, and VIRTIO_CRYPTO_AKCIPHER_VERIFY.
\end{itemize*}

\devicenormative{\paragraph}{AKCIPHER Service Operation}{Device Types / Crypto Device / Device Operation / AKCIPHER Service Operation}

\begin{itemize*}
\item If the VIRTIO_CRYPTO_F_AKCIPHER_STATELESS_MODE feature bit is negotiated, the
    device MUST parse the virtio_crypto_akcipher_data_vlf_stateless based on the \field{opcode}
    field in general header.
\item The device MUST copy the result of cryptographic operation in the dst_data[].
\item The device MUST set the \field{status} field in struct virtio_crypto_inhdr to
    one of the following values of enum VIRTIO_CRYPTO_STATUS:
\begin{itemize*}
\item VIRTIO_CRYPTO_OK if the operation success.
\item VIRTIO_CRYPTO_NOTSUPP if the requested algorithm or operation is unsupported.
\item VIRTIO_CRYPTO_BADMSG if the verification result is incorrect.
\item VIRTIO_CRYPTO_INVSESS if the session ID invalid when in session mode.
\item VIRTIO_CRYPTO_KEY_REJECTED if the signature verification failed.
\item VIRTIO_CRYPTO_ERR if any failure not mentioned above occurs.
\end{itemize*}
\end{itemize*}

\section{GPU Device}\label{sec:Device Types / GPU Device}

virtio-gpu is a virtio based graphics adapter.  It can operate in 2D
mode and in 3D mode.  3D mode will offload rendering ops to
the host gpu and therefore requires a gpu with 3D support on the host
machine.

In 2D mode the virtio-gpu device provides support for ARGB Hardware
cursors and multiple scanouts (aka heads).

\subsection{Device ID}\label{sec:Device Types / GPU Device / Device ID}

16

\subsection{Virtqueues}\label{sec:Device Types / GPU Device / Virtqueues}

\begin{description}
\item[0] controlq - queue for sending control commands
\item[1] cursorq - queue for sending cursor updates
\end{description}

Both queues have the same format.  Each request and each response have
a fixed header, followed by command specific data fields.  The
separate cursor queue is the "fast track" for cursor commands
(VIRTIO_GPU_CMD_UPDATE_CURSOR and VIRTIO_GPU_CMD_MOVE_CURSOR), so they
go through without being delayed by time-consuming commands in the
control queue.

\subsection{Feature bits}\label{sec:Device Types / GPU Device / Feature bits}

\begin{description}
\item[VIRTIO_GPU_F_VIRGL (0)] virgl 3D mode is supported.
\item[VIRTIO_GPU_F_EDID  (1)] EDID is supported.
\item[VIRTIO_GPU_F_RESOURCE_UUID (2)] assigning resources UUIDs for export
  to other virtio devices is supported.
\item[VIRTIO_GPU_F_RESOURCE_BLOB (3)] creating and using size-based blob
  resources is supported.
\item[VIRTIO_GPU_F_CONTEXT_INIT (4)] multiple context types and
  synchronization timelines supported.  Requires VIRTIO_GPU_F_VIRGL.
\end{description}

\subsection{Device configuration layout}\label{sec:Device Types / GPU Device / Device configuration layout}

GPU device configuration uses the following layout structure and
definitions:

\begin{lstlisting}
#define VIRTIO_GPU_EVENT_DISPLAY (1 << 0)

struct virtio_gpu_config {
        le32 events_read;
        le32 events_clear;
        le32 num_scanouts;
        le32 num_capsets;
};
\end{lstlisting}

\subsubsection{Device configuration fields}

\begin{description}
\item[\field{events_read}] signals pending events to the driver.  The
  driver MUST NOT write to this field.
\item[\field{events_clear}] clears pending events in the device.
  Writing a '1' into a bit will clear the corresponding bit in
  \field{events_read}, mimicking write-to-clear behavior.
\item[\field{num_scanouts}] specifies the maximum number of scanouts
  supported by the device.  Minimum value is 1, maximum value is 16.
\item[\field{num_capsets}] specifies the maximum number of capability
  sets supported by the device.  The minimum value is zero.
\end{description}

\subsubsection{Events}

\begin{description}
\item[VIRTIO_GPU_EVENT_DISPLAY] Display configuration has changed.
  The driver SHOULD use the VIRTIO_GPU_CMD_GET_DISPLAY_INFO command to
  fetch the information from the device.  In case EDID support is
  negotiated (VIRTIO_GPU_F_EDID feature flag) the device SHOULD also
  fetch the updated EDID blobs using the VIRTIO_GPU_CMD_GET_EDID
  command.
\end{description}

\devicenormative{\subsection}{Device Initialization}{Device Types / GPU Device / Device Initialization}

The driver SHOULD query the display information from the device using
the VIRTIO_GPU_CMD_GET_DISPLAY_INFO command and use that information
for the initial scanout setup.  In case EDID support is negotiated
(VIRTIO_GPU_F_EDID feature flag) the device SHOULD also fetch the EDID
information using the VIRTIO_GPU_CMD_GET_EDID command.  If no
information is available or all displays are disabled the driver MAY
choose to use a fallback, such as 1024x768 at display 0.

The driver SHOULD query all shared memory regions supported by the device.
If the device supports shared memory, the \field{shmid} of a region MUST
(see \ref{sec:Basic Facilities of a Virtio Device /
Shared Memory Regions}~\nameref{sec:Basic Facilities of a Virtio Device /
Shared Memory Regions}) be one of the following:

\begin{lstlisting}
enum virtio_gpu_shm_id {
        VIRTIO_GPU_SHM_ID_UNDEFINED = 0,
        VIRTIO_GPU_SHM_ID_HOST_VISIBLE = 1,
};
\end{lstlisting}

The shared memory region with VIRTIO_GPU_SHM_ID_HOST_VISIBLE is referred as
the "host visible memory region".  The device MUST support the
VIRTIO_GPU_CMD_RESOURCE_MAP_BLOB and VIRTIO_GPU_CMD_RESOURCE_UNMAP_BLOB
if the host visible memory region is available.

\subsection{Device Operation}\label{sec:Device Types / GPU Device / Device Operation}

The virtio-gpu is based around the concept of resources private to the
host.  The guest must DMA transfer into these resources, unless shared memory
regions are supported. This is a design requirement in order to interface with
future 3D rendering. In the unaccelerated 2D mode there is no support for DMA
transfers from resources, just to them.

Resources are initially simple 2D resources, consisting of a width,
height and format along with an identifier. The guest must then attach
backing store to the resources in order for DMA transfers to
work. This is like a GART in a real GPU.

\subsubsection{Device Operation: Create a framebuffer and configure scanout}

\begin{itemize*}
\item Create a host resource using VIRTIO_GPU_CMD_RESOURCE_CREATE_2D.
\item Allocate a framebuffer from guest ram, and attach it as backing
  storage to the resource just created, using
  VIRTIO_GPU_CMD_RESOURCE_ATTACH_BACKING.  Scatter lists are
  supported, so the framebuffer doesn't need to be contignous in guest
  physical memory.
\item Use VIRTIO_GPU_CMD_SET_SCANOUT to link the framebuffer to
  a display scanout.
\end{itemize*}

\subsubsection{Device Operation: Update a framebuffer and scanout}

\begin{itemize*}
\item Render to your framebuffer memory.
\item Use VIRTIO_GPU_CMD_TRANSFER_TO_HOST_2D to update the host resource
  from guest memory.
\item Use VIRTIO_GPU_CMD_RESOURCE_FLUSH to flush the updated resource
  to the display.
\end{itemize*}

\subsubsection{Device Operation: Using pageflip}

It is possible to create multiple framebuffers, flip between them
using VIRTIO_GPU_CMD_SET_SCANOUT and VIRTIO_GPU_CMD_RESOURCE_FLUSH,
and update the invisible framebuffer using
VIRTIO_GPU_CMD_TRANSFER_TO_HOST_2D.

\subsubsection{Device Operation: Multihead setup}

In case two or more displays are present there are different ways to
configure things:

\begin{itemize*}
\item Create a single framebuffer, link it to all displays
  (mirroring).
\item Create an framebuffer for each display.
\item Create one big framebuffer, configure scanouts to display a
  different rectangle of that framebuffer each.
\end{itemize*}

\devicenormative{\subsubsection}{Device Operation: Command lifecycle and fencing}{Device Types / GPU Device / Device Operation / Device Operation: Command lifecycle and fencing}

The device MAY process controlq commands asynchronously and return them
to the driver before the processing is complete.  If the driver needs
to know when the processing is finished it can set the
VIRTIO_GPU_FLAG_FENCE flag in the request.  The device MUST finish the
processing before returning the command then.

Note: current qemu implementation does asyncrounous processing only in
3d mode, when offloading the processing to the host gpu.

\subsubsection{Device Operation: Configure mouse cursor}

The mouse cursor image is a normal resource, except that it must be
64x64 in size.  The driver MUST create and populate the resource
(using the usual VIRTIO_GPU_CMD_RESOURCE_CREATE_2D,
VIRTIO_GPU_CMD_RESOURCE_ATTACH_BACKING and
VIRTIO_GPU_CMD_TRANSFER_TO_HOST_2D controlq commands) and make sure they
are completed (using VIRTIO_GPU_FLAG_FENCE).

Then VIRTIO_GPU_CMD_UPDATE_CURSOR can be sent to the cursorq to set
the pointer shape and position.  To move the pointer without updating
the shape use VIRTIO_GPU_CMD_MOVE_CURSOR instead.

\subsubsection{Device Operation: Request header}\label{sec:Device Types / GPU Device / Device Operation / Device Operation: Request header}

All requests and responses on the virtqueues have a fixed header
using the following layout structure and definitions:

\begin{lstlisting}
enum virtio_gpu_ctrl_type {

        /* 2d commands */
        VIRTIO_GPU_CMD_GET_DISPLAY_INFO = 0x0100,
        VIRTIO_GPU_CMD_RESOURCE_CREATE_2D,
        VIRTIO_GPU_CMD_RESOURCE_UNREF,
        VIRTIO_GPU_CMD_SET_SCANOUT,
        VIRTIO_GPU_CMD_RESOURCE_FLUSH,
        VIRTIO_GPU_CMD_TRANSFER_TO_HOST_2D,
        VIRTIO_GPU_CMD_RESOURCE_ATTACH_BACKING,
        VIRTIO_GPU_CMD_RESOURCE_DETACH_BACKING,
        VIRTIO_GPU_CMD_GET_CAPSET_INFO,
        VIRTIO_GPU_CMD_GET_CAPSET,
        VIRTIO_GPU_CMD_GET_EDID,
        VIRTIO_GPU_CMD_RESOURCE_ASSIGN_UUID,
        VIRTIO_GPU_CMD_RESOURCE_CREATE_BLOB,
        VIRTIO_GPU_CMD_SET_SCANOUT_BLOB,

        /* 3d commands */
        VIRTIO_GPU_CMD_CTX_CREATE = 0x0200,
        VIRTIO_GPU_CMD_CTX_DESTROY,
        VIRTIO_GPU_CMD_CTX_ATTACH_RESOURCE,
        VIRTIO_GPU_CMD_CTX_DETACH_RESOURCE,
        VIRTIO_GPU_CMD_RESOURCE_CREATE_3D,
        VIRTIO_GPU_CMD_TRANSFER_TO_HOST_3D,
        VIRTIO_GPU_CMD_TRANSFER_FROM_HOST_3D,
        VIRTIO_GPU_CMD_SUBMIT_3D,
        VIRTIO_GPU_CMD_RESOURCE_MAP_BLOB,
        VIRTIO_GPU_CMD_RESOURCE_UNMAP_BLOB,

        /* cursor commands */
        VIRTIO_GPU_CMD_UPDATE_CURSOR = 0x0300,
        VIRTIO_GPU_CMD_MOVE_CURSOR,

        /* success responses */
        VIRTIO_GPU_RESP_OK_NODATA = 0x1100,
        VIRTIO_GPU_RESP_OK_DISPLAY_INFO,
        VIRTIO_GPU_RESP_OK_CAPSET_INFO,
        VIRTIO_GPU_RESP_OK_CAPSET,
        VIRTIO_GPU_RESP_OK_EDID,
        VIRTIO_GPU_RESP_OK_RESOURCE_UUID,
        VIRTIO_GPU_RESP_OK_MAP_INFO,

        /* error responses */
        VIRTIO_GPU_RESP_ERR_UNSPEC = 0x1200,
        VIRTIO_GPU_RESP_ERR_OUT_OF_MEMORY,
        VIRTIO_GPU_RESP_ERR_INVALID_SCANOUT_ID,
        VIRTIO_GPU_RESP_ERR_INVALID_RESOURCE_ID,
        VIRTIO_GPU_RESP_ERR_INVALID_CONTEXT_ID,
        VIRTIO_GPU_RESP_ERR_INVALID_PARAMETER,
};

#define VIRTIO_GPU_FLAG_FENCE (1 << 0)
#define VIRTIO_GPU_FLAG_INFO_RING_IDX (1 << 1)

struct virtio_gpu_ctrl_hdr {
        le32 type;
        le32 flags;
        le64 fence_id;
        le32 ctx_id;
        u8 ring_idx;
        u8 padding[3];
};
\end{lstlisting}

The fixed header \field{struct virtio_gpu_ctrl_hdr} in each
request includes the following fields:

\begin{description}
\item[\field{type}] specifies the type of the driver request
  (VIRTIO_GPU_CMD_*) or device response (VIRTIO_GPU_RESP_*).
\item[\field{flags}] request / response flags.
\item[\field{fence_id}] If the driver sets the VIRTIO_GPU_FLAG_FENCE
  bit in the request \field{flags} field the device MUST:
  \begin{itemize*}
  \item set VIRTIO_GPU_FLAG_FENCE bit in the response,
  \item copy the content of the \field{fence_id} field from the
    request to the response, and
  \item send the response only after command processing is complete.
  \end{itemize*}
\item[\field{ctx_id}] Rendering context (used in 3D mode only).
\item[\field{ring_idx}] If VIRTIO_GPU_F_CONTEXT_INIT is supported, then
  the driver MAY set VIRTIO_GPU_FLAG_INFO_RING_IDX bit in the request
  \field{flags}.  In that case:
  \begin{itemize*}
  \item \field{ring_idx} indicates the value of a context-specific ring
   index.  The minimum value is 0 and maximum value is 63 (inclusive).
  \item If VIRTIO_GPU_FLAG_FENCE is set, \field{fence_id} acts as a
   sequence number on the synchronization timeline defined by
   \field{ctx_idx} and the ring index.
  \item If VIRTIO_GPU_FLAG_FENCE is set and when the command associated
   with \field{fence_id} is complete, the device MUST send a response for
   all outstanding commands with a sequence number less than or equal to
   \field{fence_id} on the same synchronization timeline.
  \end{itemize*}
\end{description}

On success the device will return VIRTIO_GPU_RESP_OK_NODATA in
case there is no payload.  Otherwise the \field{type} field will
indicate the kind of payload.

On error the device will return one of the
VIRTIO_GPU_RESP_ERR_* error codes.

\subsubsection{Device Operation: controlq}\label{sec:Device Types / GPU Device / Device Operation / Device Operation: controlq}

For any coordinates given 0,0 is top left, larger x moves right,
larger y moves down.

\begin{description}

\item[VIRTIO_GPU_CMD_GET_DISPLAY_INFO] Retrieve the current output
  configuration.  No request data (just bare \field{struct
    virtio_gpu_ctrl_hdr}).  Response type is
  VIRTIO_GPU_RESP_OK_DISPLAY_INFO, response data is \field{struct
    virtio_gpu_resp_display_info}.

\begin{lstlisting}
#define VIRTIO_GPU_MAX_SCANOUTS 16

struct virtio_gpu_rect {
        le32 x;
        le32 y;
        le32 width;
        le32 height;
};

struct virtio_gpu_resp_display_info {
        struct virtio_gpu_ctrl_hdr hdr;
        struct virtio_gpu_display_one {
                struct virtio_gpu_rect r;
                le32 enabled;
                le32 flags;
        } pmodes[VIRTIO_GPU_MAX_SCANOUTS];
};
\end{lstlisting}

The response contains a list of per-scanout information.  The info
contains whether the scanout is enabled and what its preferred
position and size is.

The size (fields \field{width} and \field{height}) is similar to the
native panel resolution in EDID display information, except that in
the virtual machine case the size can change when the host window
representing the guest display is gets resized.

The position (fields \field{x} and \field{y}) describe how the
displays are arranged (i.e. which is -- for example -- the left
display).

The \field{enabled} field is set when the user enabled the display.
It is roughly the same as the connected state of a physical display
connector.

\item[VIRTIO_GPU_CMD_GET_EDID] Retrieve the EDID data for a given
  scanout.  Request data is \field{struct virtio_gpu_get_edid}).
  Response type is VIRTIO_GPU_RESP_OK_EDID, response data is
  \field{struct virtio_gpu_resp_edid}.  Support is optional and
  negotiated using the VIRTIO_GPU_F_EDID feature flag.

\begin{lstlisting}
struct virtio_gpu_get_edid {
        struct virtio_gpu_ctrl_hdr hdr;
        le32 scanout;
        le32 padding;
};

struct virtio_gpu_resp_edid {
        struct virtio_gpu_ctrl_hdr hdr;
        le32 size;
        le32 padding;
        u8 edid[1024];
};
\end{lstlisting}

The response contains the EDID display data blob (as specified by
VESA) for the scanout.

\item[VIRTIO_GPU_CMD_RESOURCE_CREATE_2D] Create a 2D resource on the
  host.  Request data is \field{struct virtio_gpu_resource_create_2d}.
  Response type is VIRTIO_GPU_RESP_OK_NODATA.

\begin{lstlisting}
enum virtio_gpu_formats {
        VIRTIO_GPU_FORMAT_B8G8R8A8_UNORM  = 1,
        VIRTIO_GPU_FORMAT_B8G8R8X8_UNORM  = 2,
        VIRTIO_GPU_FORMAT_A8R8G8B8_UNORM  = 3,
        VIRTIO_GPU_FORMAT_X8R8G8B8_UNORM  = 4,

        VIRTIO_GPU_FORMAT_R8G8B8A8_UNORM  = 67,
        VIRTIO_GPU_FORMAT_X8B8G8R8_UNORM  = 68,

        VIRTIO_GPU_FORMAT_A8B8G8R8_UNORM  = 121,
        VIRTIO_GPU_FORMAT_R8G8B8X8_UNORM  = 134,
};

struct virtio_gpu_resource_create_2d {
        struct virtio_gpu_ctrl_hdr hdr;
        le32 resource_id;
        le32 format;
        le32 width;
        le32 height;
};
\end{lstlisting}

This creates a 2D resource on the host with the specified width,
height and format.  The resource ids are generated by the guest.

\item[VIRTIO_GPU_CMD_RESOURCE_UNREF] Destroy a resource.  Request data
  is \field{struct virtio_gpu_resource_unref}.  Response type is
  VIRTIO_GPU_RESP_OK_NODATA.

\begin{lstlisting}
struct virtio_gpu_resource_unref {
        struct virtio_gpu_ctrl_hdr hdr;
        le32 resource_id;
        le32 padding;
};
\end{lstlisting}

This informs the host that a resource is no longer required by the
guest.

\item[VIRTIO_GPU_CMD_SET_SCANOUT] Set the scanout parameters for a
  single output.  Request data is \field{struct
    virtio_gpu_set_scanout}.  Response type is
  VIRTIO_GPU_RESP_OK_NODATA.

\begin{lstlisting}
struct virtio_gpu_set_scanout {
        struct virtio_gpu_ctrl_hdr hdr;
        struct virtio_gpu_rect r;
        le32 scanout_id;
        le32 resource_id;
};
\end{lstlisting}

This sets the scanout parameters for a single scanout. The resource_id
is the resource to be scanned out from, along with a rectangle.

Scanout rectangles must be completely covered by the underlying
resource.  Overlapping (or identical) scanouts are allowed, typical
use case is screen mirroring.

The driver can use resource_id = 0 to disable a scanout.

\item[VIRTIO_GPU_CMD_RESOURCE_FLUSH] Flush a scanout resource Request
  data is \field{struct virtio_gpu_resource_flush}.  Response type is
  VIRTIO_GPU_RESP_OK_NODATA.

\begin{lstlisting}
struct virtio_gpu_resource_flush {
        struct virtio_gpu_ctrl_hdr hdr;
        struct virtio_gpu_rect r;
        le32 resource_id;
        le32 padding;
};
\end{lstlisting}

This flushes a resource to screen.  It takes a rectangle and a
resource id, and flushes any scanouts the resource is being used on.

\item[VIRTIO_GPU_CMD_TRANSFER_TO_HOST_2D] Transfer from guest memory
  to host resource.  Request data is \field{struct
    virtio_gpu_transfer_to_host_2d}.  Response type is
  VIRTIO_GPU_RESP_OK_NODATA.

\begin{lstlisting}
struct virtio_gpu_transfer_to_host_2d {
        struct virtio_gpu_ctrl_hdr hdr;
        struct virtio_gpu_rect r;
        le64 offset;
        le32 resource_id;
        le32 padding;
};
\end{lstlisting}

This takes a resource id along with an destination offset into the
resource, and a box to transfer to the host backing for the resource.

\item[VIRTIO_GPU_CMD_RESOURCE_ATTACH_BACKING] Assign backing pages to
  a resource.  Request data is \field{struct
    virtio_gpu_resource_attach_backing}, followed by \field{struct
    virtio_gpu_mem_entry} entries.  Response type is
  VIRTIO_GPU_RESP_OK_NODATA.

\begin{lstlisting}
struct virtio_gpu_resource_attach_backing {
        struct virtio_gpu_ctrl_hdr hdr;
        le32 resource_id;
        le32 nr_entries;
};

struct virtio_gpu_mem_entry {
        le64 addr;
        le32 length;
        le32 padding;
};
\end{lstlisting}

This assign an array of guest pages as the backing store for a
resource. These pages are then used for the transfer operations for
that resource from that point on.

\item[VIRTIO_GPU_CMD_RESOURCE_DETACH_BACKING] Detach backing pages
  from a resource.  Request data is \field{struct
    virtio_gpu_resource_detach_backing}.  Response type is
  VIRTIO_GPU_RESP_OK_NODATA.

\begin{lstlisting}
struct virtio_gpu_resource_detach_backing {
        struct virtio_gpu_ctrl_hdr hdr;
        le32 resource_id;
        le32 padding;
};
\end{lstlisting}

This detaches any backing pages from a resource, to be used in case of
guest swapping or object destruction.

\item[VIRTIO_GPU_CMD_GET_CAPSET_INFO] Gets the information associated with
  a particular \field{capset_index}, which MUST less than \field{num_capsets}
  defined in the device configuration.  Request data is
  \field{struct virtio_gpu_get_capset_info}.  Response type is
  VIRTIO_GPU_RESP_OK_CAPSET_INFO.

  On success, \field{struct virtio_gpu_resp_capset_info} contains the
  \field{capset_id}, \field{capset_max_version}, \field{capset_max_size}
  associated with capset at the specified {capset_idex}.  field{capset_id} MUST
  be one of the following (see listing for values):

  \begin{itemize*}
  \item \href{https://gitlab.freedesktop.org/virgl/virglrenderer/-/blob/master/src/virgl_hw.h#L526}{VIRTIO_GPU_CAPSET_VIRGL} --
	the first edition of Virgl (Gallium OpenGL) protocol.
  \item \href{https://gitlab.freedesktop.org/virgl/virglrenderer/-/blob/master/src/virgl_hw.h#L550}{VIRTIO_GPU_CAPSET_VIRGL2} --
	the second edition of Virgl (Gallium OpenGL) protocol after the capset fix.
  \item \href{https://android.googlesource.com/device/generic/vulkan-cereal/+/refs/heads/master/protocols/}{VIRTIO_GPU_CAPSET_GFXSTREAM} --
	gfxtream's (mostly) autogenerated GLES and Vulkan streaming protocols.
  \item \href{https://gitlab.freedesktop.org/olv/venus-protocol}{VIRTIO_GPU_CAPSET_VENUS} --
	Mesa's (mostly) autogenerated Vulkan protocol.
  \item \href{https://chromium.googlesource.com/chromiumos/platform/crosvm/+/refs/heads/main/rutabaga_gfx/src/cross_domain/cross_domain_protocol.rs}{VIRTIO_GPU_CAPSET_CROSS_DOMAIN} --
	protocol for display virtualization via Wayland proxying.
  \end{itemize*}

\begin{lstlisting}
struct virtio_gpu_get_capset_info {
        struct virtio_gpu_ctrl_hdr hdr;
        le32 capset_index;
        le32 padding;
};

#define VIRTIO_GPU_CAPSET_VIRGL 1
#define VIRTIO_GPU_CAPSET_VIRGL2 2
#define VIRTIO_GPU_CAPSET_GFXSTREAM 3
#define VIRTIO_GPU_CAPSET_VENUS 4
#define VIRTIO_GPU_CAPSET_CROSS_DOMAIN 5
struct virtio_gpu_resp_capset_info {
        struct virtio_gpu_ctrl_hdr hdr;
        le32 capset_id;
        le32 capset_max_version;
        le32 capset_max_size;
        le32 padding;
};
\end{lstlisting}

\item[VIRTIO_GPU_CMD_GET_CAPSET] Gets the capset associated with a
  particular \field{capset_id} and \field{capset_version}.  Request data is
  \field{struct virtio_gpu_get_capset}.  Response type is
  VIRTIO_GPU_RESP_OK_CAPSET.

\begin{lstlisting}
struct virtio_gpu_get_capset {
        struct virtio_gpu_ctrl_hdr hdr;
        le32 capset_id;
        le32 capset_version;
};

struct virtio_gpu_resp_capset {
        struct virtio_gpu_ctrl_hdr hdr;
        u8 capset_data[];
};
\end{lstlisting}

\item[VIRTIO_GPU_CMD_RESOURCE_ASSIGN_UUID] Creates an exported object from
  a resource. Request data is \field{struct
    virtio_gpu_resource_assign_uuid}.  Response type is
  VIRTIO_GPU_RESP_OK_RESOURCE_UUID, response data is \field{struct
    virtio_gpu_resp_resource_uuid}. Support is optional and negotiated
    using the VIRTIO_GPU_F_RESOURCE_UUID feature flag.

\begin{lstlisting}
struct virtio_gpu_resource_assign_uuid {
        struct virtio_gpu_ctrl_hdr hdr;
        le32 resource_id;
        le32 padding;
};

struct virtio_gpu_resp_resource_uuid {
        struct virtio_gpu_ctrl_hdr hdr;
        u8 uuid[16];
};
\end{lstlisting}

The response contains a UUID which identifies the exported object created from
the host private resource. Note that if the resource has an attached backing,
modifications made to the host private resource through the exported object by
other devices are not visible in the attached backing until they are transferred
into the backing.

\item[VIRTIO_GPU_CMD_RESOURCE_CREATE_BLOB] Creates a virtio-gpu blob
  resource. Request data is \field{struct
  virtio_gpu_resource_create_blob}, followed by \field{struct
  virtio_gpu_mem_entry} entries. Response type is
  VIRTIO_GPU_RESP_OK_NODATA. Support is optional and negotiated
  using the VIRTIO_GPU_F_RESOURCE_BLOB feature flag.

\begin{lstlisting}
#define VIRTIO_GPU_BLOB_MEM_GUEST             0x0001
#define VIRTIO_GPU_BLOB_MEM_HOST3D            0x0002
#define VIRTIO_GPU_BLOB_MEM_HOST3D_GUEST      0x0003

#define VIRTIO_GPU_BLOB_FLAG_USE_MAPPABLE     0x0001
#define VIRTIO_GPU_BLOB_FLAG_USE_SHAREABLE    0x0002
#define VIRTIO_GPU_BLOB_FLAG_USE_CROSS_DEVICE 0x0004

struct virtio_gpu_resource_create_blob {
       struct virtio_gpu_ctrl_hdr hdr;
       le32 resource_id;
       le32 blob_mem;
       le32 blob_flags;
       le32 nr_entries;
       le64 blob_id;
       le64 size;
};

\end{lstlisting}

A blob resource is a container for:

  \begin{itemize*}
  \item a guest memory allocation (referred to as a
  "guest-only blob resource").
  \item a host memory allocation (referred to as a
  "host-only blob resource").
  \item a guest memory and host memory allocation (referred
  to as a "default blob resource").
  \end{itemize*}

The memory properties of the blob resource MUST be described by
\field{blob_mem}, which MUST be non-zero.

For default and guest-only blob resources, \field{nr_entries} guest
memory entries may be assigned to the resource.  For default blob resources
(i.e, when \field{blob_mem} is VIRTIO_GPU_BLOB_MEM_HOST3D_GUEST), these
memory entries are used as a shadow buffer for the host memory. To
facilitate drivers that support swap-in and swap-out, \field{nr_entries} may
be zero and VIRTIO_GPU_CMD_RESOURCE_ATTACH_BACKING may be subsequently used.
VIRTIO_GPU_CMD_RESOURCE_DETACH_BACKING may be used to unassign memory entries.

\field{blob_mem} can only be VIRTIO_GPU_BLOB_MEM_HOST3D and
VIRTIO_GPU_BLOB_MEM_HOST3D_GUEST if VIRTIO_GPU_F_VIRGL is supported.
VIRTIO_GPU_BLOB_MEM_GUEST is valid regardless whether VIRTIO_GPU_F_VIRGL
is supported or not.

For VIRTIO_GPU_BLOB_MEM_HOST3D and VIRTIO_GPU_BLOB_MEM_HOST3D_GUEST, the
virtio-gpu resource MUST be created from the rendering context local object
identified by the \field{blob_id}. The actual allocation is done via
VIRTIO_GPU_CMD_SUBMIT_3D.

The driver MUST inform the device if the blob resource is used for
memory access, sharing between driver instances and/or sharing with
other devices. This is done via the \field{blob_flags} field.

If VIRTIO_GPU_F_VIRGL is set, both VIRTIO_GPU_CMD_TRANSFER_TO_HOST_3D
and VIRTIO_GPU_CMD_TRANSFER_FROM_HOST_3D may be used to update the
resource. There is no restriction on the image/buffer view the driver
has on the blob resource.

\item[VIRTIO_GPU_CMD_SET_SCANOUT_BLOB] sets scanout parameters for a
   blob resource. Request data is
  \field{struct virtio_gpu_set_scanout_blob}. Response type is
  VIRTIO_GPU_RESP_OK_NODATA. Support is optional and negotiated
  using the VIRTIO_GPU_F_RESOURCE_BLOB feature flag.

\begin{lstlisting}
struct virtio_gpu_set_scanout_blob {
       struct virtio_gpu_ctrl_hdr hdr;
       struct virtio_gpu_rect r;
       le32 scanout_id;
       le32 resource_id;
       le32 width;
       le32 height;
       le32 format;
       le32 padding;
       le32 strides[4];
       le32 offsets[4];
};
\end{lstlisting}

The rectangle \field{r} represents the portion of the blob resource being
displayed. The rest is the metadata associated with the blob resource. The
format MUST be one of \field{enum virtio_gpu_formats}.  The format MAY be
compressed with header and data planes.

\end{description}

\subsubsection{Device Operation: controlq (3d)}\label{sec:Device Types / GPU Device / Device Operation / Device Operation: controlq (3d)}

These commands are supported by the device if the VIRTIO_GPU_F_VIRGL
feature flag is set.

\begin{description}

\item[VIRTIO_GPU_CMD_CTX_CREATE] creates a context for submitting an opaque
  command stream.  Request data is \field{struct virtio_gpu_ctx_create}.
  Response type is VIRTIO_GPU_RESP_OK_NODATA.

\begin{lstlisting}
#define VIRTIO_GPU_CONTEXT_INIT_CAPSET_ID_MASK 0x000000ff;
struct virtio_gpu_ctx_create {
       struct virtio_gpu_ctrl_hdr hdr;
       le32 nlen;
       le32 context_init;
       char debug_name[64];
};
\end{lstlisting}

The implementation MUST create a context for the given \field{ctx_id} in
the \field{hdr}.  For debugging purposes, a \field{debug_name} and it's
length \field{nlen} is provided by the driver.  If
VIRTIO_GPU_F_CONTEXT_INIT is supported, then lower 8 bits of
\field{context_init} MAY contain the \field{capset_id} associated with
context.  In that case, then the device MUST create a context that can
handle the specified command stream.

If the lower 8-bits of the \field{context_init} are zero, then the type of
the context is determined by the device.

\item[VIRTIO_GPU_CMD_CTX_DESTROY]
\item[VIRTIO_GPU_CMD_CTX_ATTACH_RESOURCE]
\item[VIRTIO_GPU_CMD_CTX_DETACH_RESOURCE]
  Manage virtio-gpu 3d contexts.

\item[VIRTIO_GPU_CMD_RESOURCE_CREATE_3D]
  Create virtio-gpu 3d resources.

\item[VIRTIO_GPU_CMD_TRANSFER_TO_HOST_3D]
\item[VIRTIO_GPU_CMD_TRANSFER_FROM_HOST_3D]
  Transfer data from and to virtio-gpu 3d resources.

\item[VIRTIO_GPU_CMD_SUBMIT_3D]
  Submit an opaque command stream.  The type of the command stream is
  determined when creating a context.

\item[VIRTIO_GPU_CMD_RESOURCE_MAP_BLOB] maps a host-only
  blob resource into an offset in the host visible memory region. Request
  data is \field{struct virtio_gpu_resource_map_blob}.  The driver MUST
  not map a blob resource that is already mapped.  Response type is
  VIRTIO_GPU_RESP_OK_MAP_INFO. Support is optional and negotiated
  using the VIRTIO_GPU_F_RESOURCE_BLOB feature flag and checking for
  the presence of the host visible memory region.

\begin{lstlisting}
struct virtio_gpu_resource_map_blob {
        struct virtio_gpu_ctrl_hdr hdr;
        le32 resource_id;
        le32 padding;
        le64 offset;
};

#define VIRTIO_GPU_MAP_CACHE_MASK      0x0f
#define VIRTIO_GPU_MAP_CACHE_NONE      0x00
#define VIRTIO_GPU_MAP_CACHE_CACHED    0x01
#define VIRTIO_GPU_MAP_CACHE_UNCACHED  0x02
#define VIRTIO_GPU_MAP_CACHE_WC        0x03
struct virtio_gpu_resp_map_info {
        struct virtio_gpu_ctrl_hdr hdr;
        u32 map_info;
        u32 padding;
};
\end{lstlisting}

\item[VIRTIO_GPU_CMD_RESOURCE_UNMAP_BLOB] unmaps a
  host-only blob resource from the host visible memory region. Request data
  is \field{struct virtio_gpu_resource_unmap_blob}.  Response type is
  VIRTIO_GPU_RESP_OK_NODATA.  Support is optional and negotiated
  using the VIRTIO_GPU_F_RESOURCE_BLOB feature flag and checking for
  the presence of the host visible memory region.

\begin{lstlisting}
struct virtio_gpu_resource_unmap_blob {
        struct virtio_gpu_ctrl_hdr hdr;
        le32 resource_id;
        le32 padding;
};
\end{lstlisting}

\end{description}

\subsubsection{Device Operation: cursorq}\label{sec:Device Types / GPU Device / Device Operation / Device Operation: cursorq}

Both cursorq commands use the same command struct.

\begin{lstlisting}
struct virtio_gpu_cursor_pos {
        le32 scanout_id;
        le32 x;
        le32 y;
        le32 padding;
};

struct virtio_gpu_update_cursor {
        struct virtio_gpu_ctrl_hdr hdr;
        struct virtio_gpu_cursor_pos pos;
        le32 resource_id;
        le32 hot_x;
        le32 hot_y;
        le32 padding;
};
\end{lstlisting}

\begin{description}

\item[VIRTIO_GPU_CMD_UPDATE_CURSOR]
Update cursor.
Request data is \field{struct virtio_gpu_update_cursor}.
Response type is VIRTIO_GPU_RESP_OK_NODATA.

Full cursor update.  Cursor will be loaded from the specified
\field{resource_id} and will be moved to \field{pos}.  The driver must
transfer the cursor into the resource beforehand (using control queue
commands) and make sure the commands to fill the resource are actually
processed (using fencing).

\item[VIRTIO_GPU_CMD_MOVE_CURSOR]
Move cursor.
Request data is \field{struct virtio_gpu_update_cursor}.
Response type is VIRTIO_GPU_RESP_OK_NODATA.

Move cursor to the place specified in \field{pos}.  The other fields
are not used and will be ignored by the device.

\end{description}

\subsection{VGA Compatibility}\label{sec:Device Types / GPU Device / VGA Compatibility}

Applies to Virtio Over PCI only.  The GPU device can come with and
without VGA compatibility.  The PCI class should be DISPLAY_VGA if VGA
compatibility is present and DISPLAY_OTHER otherwise.

VGA compatibility: PCI region 0 has the linear framebuffer, standard
vga registers are present.  Configuring a scanout
(VIRTIO_GPU_CMD_SET_SCANOUT) switches the device from vga
compatibility mode into native virtio mode.  A reset switches it back
into vga compatibility mode.

Note: qemu implementation also provides bochs dispi interface io ports
and mmio bar at pci region 1 and is therefore fully compatible with
the qemu stdvga (see \href{https://git.qemu-project.org/?p=qemu.git;a=blob;f=docs/specs/standard-vga.txt;hb=HEAD}{docs/specs/standard-vga.txt} in the qemu source tree).

\section{Crypto Device}\label{sec:Device Types / Crypto Device}

The virtio crypto device is a virtual cryptography device as well as a
virtual cryptographic accelerator. The virtio crypto device provides the
following crypto services: CIPHER, MAC, HASH, AEAD and AKCIPHER. Virtio crypto
devices have a single control queue and at least one data queue. Crypto
operation requests are placed into a data queue, and serviced by the
device. Some crypto operation requests are only valid in the context of a
session. The role of the control queue is facilitating control operation
requests. Sessions management is realized with control operation
requests.

\subsection{Device ID}\label{sec:Device Types / Crypto Device / Device ID}

20

\subsection{Virtqueues}\label{sec:Device Types / Crypto Device / Virtqueues}

\begin{description}
\item[0] dataq1
\item[\ldots]
\item[N-1] dataqN
\item[N] controlq
\end{description}

N is set by \field{max_dataqueues}.

\subsection{Feature bits}\label{sec:Device Types / Crypto Device / Feature bits}

\begin{description}
\item VIRTIO_CRYPTO_F_REVISION_1 (0) revision 1. Revision 1 has a specific
    request format and other enhancements (which result in some additional
    requirements).
\item VIRTIO_CRYPTO_F_CIPHER_STATELESS_MODE (1) stateless mode requests are
    supported by the CIPHER service.
\item VIRTIO_CRYPTO_F_HASH_STATELESS_MODE (2) stateless mode requests are
    supported by the HASH service.
\item VIRTIO_CRYPTO_F_MAC_STATELESS_MODE (3) stateless mode requests are
    supported by the MAC service.
\item VIRTIO_CRYPTO_F_AEAD_STATELESS_MODE (4) stateless mode requests are
    supported by the AEAD service.
\item VIRTIO_CRYPTO_F_AKCIPHER_STATELESS_MODE (5) stateless mode requests are
    supported by the AKCIPHER service.
\end{description}


\subsubsection{Feature bit requirements}\label{sec:Device Types / Crypto Device / Feature bit requirements}

Some crypto feature bits require other crypto feature bits
(see \ref{drivernormative:Basic Facilities of a Virtio Device / Feature Bits}):

\begin{description}
\item[VIRTIO_CRYPTO_F_CIPHER_STATELESS_MODE] Requires VIRTIO_CRYPTO_F_REVISION_1.
\item[VIRTIO_CRYPTO_F_HASH_STATELESS_MODE] Requires VIRTIO_CRYPTO_F_REVISION_1.
\item[VIRTIO_CRYPTO_F_MAC_STATELESS_MODE] Requires VIRTIO_CRYPTO_F_REVISION_1.
\item[VIRTIO_CRYPTO_F_AEAD_STATELESS_MODE] Requires VIRTIO_CRYPTO_F_REVISION_1.
\item[VIRTIO_CRYPTO_F_AKCIPHER_STATELESS_MODE] Requires VIRTIO_CRYPTO_F_REVISION_1.
\end{description}

\subsection{Supported crypto services}\label{sec:Device Types / Crypto Device / Supported crypto services}

The following crypto services are defined:

\begin{lstlisting}
/* CIPHER (Symmetric Key Cipher) service */
#define VIRTIO_CRYPTO_SERVICE_CIPHER 0
/* HASH service */
#define VIRTIO_CRYPTO_SERVICE_HASH   1
/* MAC (Message Authentication Codes) service */
#define VIRTIO_CRYPTO_SERVICE_MAC    2
/* AEAD (Authenticated Encryption with Associated Data) service */
#define VIRTIO_CRYPTO_SERVICE_AEAD   3
/* AKCIPHER (Asymmetric Key Cipher) service */
#define VIRTIO_CRYPTO_SERVICE_AKCIPHER 4
\end{lstlisting}

The above constants designate bits used to indicate the which of crypto services are
offered by the device as described in, see \ref{sec:Device Types / Crypto Device / Device configuration layout}.

\subsubsection{CIPHER services}\label{sec:Device Types / Crypto Device / Supported crypto services / CIPHER services}

The following CIPHER algorithms are defined:

\begin{lstlisting}
#define VIRTIO_CRYPTO_NO_CIPHER                 0
#define VIRTIO_CRYPTO_CIPHER_ARC4               1
#define VIRTIO_CRYPTO_CIPHER_AES_ECB            2
#define VIRTIO_CRYPTO_CIPHER_AES_CBC            3
#define VIRTIO_CRYPTO_CIPHER_AES_CTR            4
#define VIRTIO_CRYPTO_CIPHER_DES_ECB            5
#define VIRTIO_CRYPTO_CIPHER_DES_CBC            6
#define VIRTIO_CRYPTO_CIPHER_3DES_ECB           7
#define VIRTIO_CRYPTO_CIPHER_3DES_CBC           8
#define VIRTIO_CRYPTO_CIPHER_3DES_CTR           9
#define VIRTIO_CRYPTO_CIPHER_KASUMI_F8          10
#define VIRTIO_CRYPTO_CIPHER_SNOW3G_UEA2        11
#define VIRTIO_CRYPTO_CIPHER_AES_F8             12
#define VIRTIO_CRYPTO_CIPHER_AES_XTS            13
#define VIRTIO_CRYPTO_CIPHER_ZUC_EEA3           14
\end{lstlisting}

The above constants have two usages:
\begin{enumerate}
\item As bit numbers, used to tell the driver which CIPHER algorithms
are supported by the device, see \ref{sec:Device Types / Crypto Device / Device configuration layout}.
\item As values, used to designate the algorithm in (CIPHER type) crypto
operation requests, see \ref{sec:Device Types / Crypto Device / Device Operation / Control Virtqueue / Session operation}.
\end{enumerate}

\subsubsection{HASH services}\label{sec:Device Types / Crypto Device / Supported crypto services / HASH services}

The following HASH algorithms are defined:

\begin{lstlisting}
#define VIRTIO_CRYPTO_NO_HASH            0
#define VIRTIO_CRYPTO_HASH_MD5           1
#define VIRTIO_CRYPTO_HASH_SHA1          2
#define VIRTIO_CRYPTO_HASH_SHA_224       3
#define VIRTIO_CRYPTO_HASH_SHA_256       4
#define VIRTIO_CRYPTO_HASH_SHA_384       5
#define VIRTIO_CRYPTO_HASH_SHA_512       6
#define VIRTIO_CRYPTO_HASH_SHA3_224      7
#define VIRTIO_CRYPTO_HASH_SHA3_256      8
#define VIRTIO_CRYPTO_HASH_SHA3_384      9
#define VIRTIO_CRYPTO_HASH_SHA3_512      10
#define VIRTIO_CRYPTO_HASH_SHA3_SHAKE128      11
#define VIRTIO_CRYPTO_HASH_SHA3_SHAKE256      12
\end{lstlisting}

The above constants have two usages:
\begin{enumerate}
\item As bit numbers, used to tell the driver which HASH algorithms
are supported by the device, see \ref{sec:Device Types / Crypto Device / Device configuration layout}.
\item As values, used to designate the algorithm in (HASH type) crypto
operation requires, see \ref{sec:Device Types / Crypto Device / Device Operation / Control Virtqueue / Session operation}.
\end{enumerate}

\subsubsection{MAC services}\label{sec:Device Types / Crypto Device / Supported crypto services / MAC services}

The following MAC algorithms are defined:

\begin{lstlisting}
#define VIRTIO_CRYPTO_NO_MAC                       0
#define VIRTIO_CRYPTO_MAC_HMAC_MD5                 1
#define VIRTIO_CRYPTO_MAC_HMAC_SHA1                2
#define VIRTIO_CRYPTO_MAC_HMAC_SHA_224             3
#define VIRTIO_CRYPTO_MAC_HMAC_SHA_256             4
#define VIRTIO_CRYPTO_MAC_HMAC_SHA_384             5
#define VIRTIO_CRYPTO_MAC_HMAC_SHA_512             6
#define VIRTIO_CRYPTO_MAC_CMAC_3DES                25
#define VIRTIO_CRYPTO_MAC_CMAC_AES                 26
#define VIRTIO_CRYPTO_MAC_KASUMI_F9                27
#define VIRTIO_CRYPTO_MAC_SNOW3G_UIA2              28
#define VIRTIO_CRYPTO_MAC_GMAC_AES                 41
#define VIRTIO_CRYPTO_MAC_GMAC_TWOFISH             42
#define VIRTIO_CRYPTO_MAC_CBCMAC_AES               49
#define VIRTIO_CRYPTO_MAC_CBCMAC_KASUMI_F9         50
#define VIRTIO_CRYPTO_MAC_XCBC_AES                 53
#define VIRTIO_CRYPTO_MAC_ZUC_EIA3                 54
\end{lstlisting}

The above constants have two usages:
\begin{enumerate}
\item As bit numbers, used to tell the driver which MAC algorithms
are supported by the device, see \ref{sec:Device Types / Crypto Device / Device configuration layout}.
\item As values, used to designate the algorithm in (MAC type) crypto
operation requests, see \ref{sec:Device Types / Crypto Device / Device Operation / Control Virtqueue / Session operation}.
\end{enumerate}

\subsubsection{AEAD services}\label{sec:Device Types / Crypto Device / Supported crypto services / AEAD services}

The following AEAD algorithms are defined:

\begin{lstlisting}
#define VIRTIO_CRYPTO_NO_AEAD     0
#define VIRTIO_CRYPTO_AEAD_GCM    1
#define VIRTIO_CRYPTO_AEAD_CCM    2
#define VIRTIO_CRYPTO_AEAD_CHACHA20_POLY1305  3
\end{lstlisting}

The above constants have two usages:
\begin{enumerate}
\item As bit numbers, used to tell the driver which AEAD algorithms
are supported by the device, see \ref{sec:Device Types / Crypto Device / Device configuration layout}.
\item As values, used to designate the algorithm in (DEAD type) crypto
operation requests, see \ref{sec:Device Types / Crypto Device / Device Operation / Control Virtqueue / Session operation}.
\end{enumerate}

\subsubsection{AKCIPHER services}\label{sec: Device Types / Crypto Device / Supported crypto services / AKCIPHER services}

The following AKCIPHER algorithms are defined:
\begin{lstlisting}
#define VIRTIO_CRYPTO_NO_AKCIPHER 0
#define VIRTIO_CRYPTO_AKCIPHER_RSA   1
#define VIRTIO_CRYPTO_AKCIPHER_ECDSA 2
\end{lstlisting}

The above constants have two usages:
\begin{enumerate}
\item As bit numbers, used to tell the driver which AKCIPHER algorithms
are supported by the device, see \ref{sec:Device Types / Crypto Device / Device configuration layout}.
\item As values, used to designate the algorithm in asymmetric crypto operation requests,
see \ref{sec:Device Types / Crypto Device / Device Operation / Control Virtqueue / Session operation}.
\end{enumerate}


\subsection{Device configuration layout}\label{sec:Device Types / Crypto Device / Device configuration layout}

Crypto device configuration uses the following layout structure:

\begin{lstlisting}
struct virtio_crypto_config {
    le32 status;
    le32 max_dataqueues;
    le32 crypto_services;
    /* Detailed algorithms mask */
    le32 cipher_algo_l;
    le32 cipher_algo_h;
    le32 hash_algo;
    le32 mac_algo_l;
    le32 mac_algo_h;
    le32 aead_algo;
    /* Maximum length of cipher key in bytes */
    le32 max_cipher_key_len;
    /* Maximum length of authenticated key in bytes */
    le32 max_auth_key_len;
    le32 akcipher_algo;
    /* Maximum size of each crypto request's content in bytes */
    le64 max_size;
};
\end{lstlisting}

\begin{description}
\item Currently, only one \field{status} bit is defined: VIRTIO_CRYPTO_S_HW_READY
    set indicates that the device is ready to process requests, this bit is read-only
    for the driver
\begin{lstlisting}
#define VIRTIO_CRYPTO_S_HW_READY  (1 << 0)
\end{lstlisting}

\item [\field{max_dataqueues}] is the maximum number of data virtqueues that can
    be configured by the device. The driver MAY use only one data queue, or it
    can use more to achieve better performance.

\item [\field{crypto_services}] crypto service offered, see \ref{sec:Device Types / Crypto Device / Supported crypto services}.

\item [\field{cipher_algo_l}] CIPHER algorithms bits 0-31, see \ref{sec:Device Types / Crypto Device / Supported crypto services  / CIPHER services}.

\item [\field{cipher_algo_h}] CIPHER algorithms bits 32-63, see \ref{sec:Device Types / Crypto Device / Supported crypto services  / CIPHER services}.

\item [\field{hash_algo}] HASH algorithms bits, see \ref{sec:Device Types / Crypto Device / Supported crypto services  / HASH services}.

\item [\field{mac_algo_l}] MAC algorithms bits 0-31, see \ref{sec:Device Types / Crypto Device / Supported crypto services  / MAC services}.

\item [\field{mac_algo_h}] MAC algorithms bits 32-63, see \ref{sec:Device Types / Crypto Device / Supported crypto services  / MAC services}.

\item [\field{aead_algo}] AEAD algorithms bits, see \ref{sec:Device Types / Crypto Device / Supported crypto services  / AEAD services}.

\item [\field{max_cipher_key_len}] is the maximum length of cipher key supported by the device.

\item [\field{max_auth_key_len}] is the maximum length of authenticated key supported by the device.

\item [\field{akcipher_algo}] AKCIPHER algorithms bit 0-31, see \ref{sec: Device Types / Crypto Device / Supported crypto services / AKCIPHER services}.

\item [\field{max_size}] is the maximum size of the variable-length parameters of
    data operation of each crypto request's content supported by the device.
\end{description}

\begin{note}
Unless explicitly stated otherwise all lengths and sizes are in bytes.
\end{note}

\devicenormative{\subsubsection}{Device configuration layout}{Device Types / Crypto Device / Device configuration layout}

\begin{itemize*}
\item The device MUST set \field{max_dataqueues} to between 1 and 65535 inclusive.
\item The device MUST set the \field{status} with valid flags, undefined flags MUST NOT be set.
\item The device MUST accept and handle requests after \field{status} is set to VIRTIO_CRYPTO_S_HW_READY.
\item The device MUST set \field{crypto_services} based on the crypto services the device offers.
\item The device MUST set detailed algorithms masks for each service advertised by \field{crypto_services}.
    The device MUST NOT set the not defined algorithms bits.
\item The device MUST set \field{max_size} to show the maximum size of crypto request the device supports.
\item The device MUST set \field{max_cipher_key_len} to show the maximum length of cipher key if the
    device supports CIPHER service.
\item The device MUST set \field{max_auth_key_len} to show the maximum length of authenticated key if
    the device supports MAC service.
\end{itemize*}

\drivernormative{\subsubsection}{Device configuration layout}{Device Types / Crypto Device / Device configuration layout}

\begin{itemize*}
\item The driver MUST read the \field{status} from the bottom bit of status to check whether the
    VIRTIO_CRYPTO_S_HW_READY is set, and the driver MUST reread it after device reset.
\item The driver MUST NOT transmit any requests to the device if the VIRTIO_CRYPTO_S_HW_READY is not set.
\item The driver MUST read \field{max_dataqueues} field to discover the number of data queues the device supports.
\item The driver MUST read \field{crypto_services} field to discover which services the device is able to offer.
\item The driver SHOULD ignore the not defined algorithms bits.
\item The driver MUST read the detailed algorithms fields based on \field{crypto_services} field.
\item The driver SHOULD read \field{max_size} to discover the maximum size of the variable-length
    parameters of data operation of the crypto request's content the device supports and MUST
    guarantee that the size of each crypto request's content is within the \field{max_size}, otherwise
    the request will fail and the driver MUST reset the device.
\item The driver SHOULD read \field{max_cipher_key_len} to discover the maximum length of cipher key
    the device supports and MUST guarantee that the \field{key_len} (CIPHER service or AEAD service) is within
    the \field{max_cipher_key_len} of the device configuration, otherwise the request will fail.
\item The driver SHOULD read \field{max_auth_key_len} to discover the maximum length of authenticated
    key the device supports and MUST guarantee that the \field{auth_key_len} (MAC service) is within the
    \field{max_auth_key_len} of the device configuration, otherwise the request will fail.
\end{itemize*}

\subsection{Device Initialization}\label{sec:Device Types / Crypto Device / Device Initialization}

\drivernormative{\subsubsection}{Device Initialization}{Device Types / Crypto Device / Device Initialization}

\begin{itemize*}
\item The driver MUST configure and initialize all virtqueues.
\item The driver MUST read the supported crypto services from bits of \field{crypto_services}.
\item The driver MUST read the supported algorithms based on \field{crypto_services} field.
\end{itemize*}

\subsection{Device Operation}\label{sec:Device Types / Crypto Device / Device Operation}

The operation of a virtio crypto device is driven by requests placed on the virtqueues.
Requests consist of a queue-type specific header (specifying among others the operation)
and an operation specific payload.

If VIRTIO_CRYPTO_F_REVISION_1 is negotiated the device may support both session mode
(See \ref{sec:Device Types / Crypto Device / Device Operation / Control Virtqueue / Session operation})
and stateless mode operation requests.
In stateless mode all operation parameters are supplied as a part of each request,
while in session mode, some or all operation parameters are managed within the
session. Stateless mode is guarded by feature bits 0-4 on a service level. If
stateless mode is negotiated for a service, the service accepts both session
mode and stateless requests; otherwise stateless mode requests are rejected
(via operation status).

\subsubsection{Operation Status}\label{sec:Device Types / Crypto Device / Device Operation / Operation status}
The device MUST return a status code as part of the operation (both session
operation and service operation) result. The valid operation status as follows:

\begin{lstlisting}
enum VIRTIO_CRYPTO_STATUS {
    VIRTIO_CRYPTO_OK = 0,
    VIRTIO_CRYPTO_ERR = 1,
    VIRTIO_CRYPTO_BADMSG = 2,
    VIRTIO_CRYPTO_NOTSUPP = 3,
    VIRTIO_CRYPTO_INVSESS = 4,
    VIRTIO_CRYPTO_NOSPC = 5,
    VIRTIO_CRYPTO_KEY_REJECTED = 6,
    VIRTIO_CRYPTO_MAX
};
\end{lstlisting}

\begin{itemize*}
\item VIRTIO_CRYPTO_OK: success.
\item VIRTIO_CRYPTO_BADMSG: authentication failed (only when AEAD decryption).
\item VIRTIO_CRYPTO_NOTSUPP: operation or algorithm is unsupported.
\item VIRTIO_CRYPTO_INVSESS: invalid session ID when executing crypto operations.
\item VIRTIO_CRYPTO_NOSPC: no free session ID (only when the VIRTIO_CRYPTO_F_REVISION_1
    feature bit is negotiated).
\item VIRTIO_CRYPTO_KEY_REJECTED: signature verification failed (only when AKCIPHER verification).
\item VIRTIO_CRYPTO_ERR: any failure not mentioned above occurs.
\end{itemize*}

\subsubsection{Control Virtqueue}\label{sec:Device Types / Crypto Device / Device Operation / Control Virtqueue}

The driver uses the control virtqueue to send control commands to the
device, such as session operations (See \ref{sec:Device Types / Crypto Device / Device
Operation / Control Virtqueue / Session operation}).

The header for controlq is of the following form:
\begin{lstlisting}
#define VIRTIO_CRYPTO_OPCODE(service, op)   (((service) << 8) | (op))

struct virtio_crypto_ctrl_header {
#define VIRTIO_CRYPTO_CIPHER_CREATE_SESSION \
       VIRTIO_CRYPTO_OPCODE(VIRTIO_CRYPTO_SERVICE_CIPHER, 0x02)
#define VIRTIO_CRYPTO_CIPHER_DESTROY_SESSION \
       VIRTIO_CRYPTO_OPCODE(VIRTIO_CRYPTO_SERVICE_CIPHER, 0x03)
#define VIRTIO_CRYPTO_HASH_CREATE_SESSION \
       VIRTIO_CRYPTO_OPCODE(VIRTIO_CRYPTO_SERVICE_HASH, 0x02)
#define VIRTIO_CRYPTO_HASH_DESTROY_SESSION \
       VIRTIO_CRYPTO_OPCODE(VIRTIO_CRYPTO_SERVICE_HASH, 0x03)
#define VIRTIO_CRYPTO_MAC_CREATE_SESSION \
       VIRTIO_CRYPTO_OPCODE(VIRTIO_CRYPTO_SERVICE_MAC, 0x02)
#define VIRTIO_CRYPTO_MAC_DESTROY_SESSION \
       VIRTIO_CRYPTO_OPCODE(VIRTIO_CRYPTO_SERVICE_MAC, 0x03)
#define VIRTIO_CRYPTO_AEAD_CREATE_SESSION \
       VIRTIO_CRYPTO_OPCODE(VIRTIO_CRYPTO_SERVICE_AEAD, 0x02)
#define VIRTIO_CRYPTO_AEAD_DESTROY_SESSION \
       VIRTIO_CRYPTO_OPCODE(VIRTIO_CRYPTO_SERVICE_AEAD, 0x03)
#define VIRTIO_CRYPTO_AKCIPHER_CREATE_SESSION \
       VIRTIO_CRYPTO_OPCODE(VIRTIO_CRYPTO_SERVICE_AKCIPHER, 0x04)
#define VIRTIO_CRYPTO_AKCIPHER_DESTROY_SESSION \
       VIRTIO_CRYPTO_OPCDE(VIRTIO_CRYPTO_SERVICE_AKCIPHER, 0x05)
    le32 opcode;
    /* algo should be service-specific algorithms */
    le32 algo;
    le32 flag;
    le32 reserved;
};
\end{lstlisting}

The controlq request is composed of four parts:
\begin{lstlisting}
struct virtio_crypto_op_ctrl_req {
    /* Device read only portion */

    struct virtio_crypto_ctrl_header header;

#define VIRTIO_CRYPTO_CTRLQ_OP_SPEC_HDR_LEGACY 56
    /* fixed length fields, opcode specific */
    u8 op_flf[flf_len];

    /* variable length fields, opcode specific */
    u8 op_vlf[vlf_len];

    /* Device write only portion */

    /* op result or completion status */
    u8 op_outcome[outcome_len];
};
\end{lstlisting}

\field{header} is a general header (see above).

\field{op_flf} is the opcode (in \field{header}) specific fixed-length parameters.

\field{flf_len} depends on the VIRTIO_CRYPTO_F_REVISION_1 feature bit (see below).

\field{op_vlf} is the opcode (in \field{header}) specific variable-length parameters.

\field{vlf_len} is the size of the specific structure used.
\begin{note}
The \field{vlf_len} of session-destroy operation and the hash-session-create
operation is ZERO.
\end{note}

\begin{itemize*}
\item If the opcode (in \field{header}) is VIRTIO_CRYPTO_CIPHER_CREATE_SESSION
    then \field{op_flf} is struct virtio_crypto_sym_create_session_flf if
    VIRTIO_CRYPTO_F_REVISION_1 is negotiated and struct virtio_crypto_sym_create_session_flf is
    padded to 56 bytes if NOT negotiated, and \field{op_vlf} is struct
    virtio_crypto_sym_create_session_vlf.
\item If the opcode (in \field{header}) is VIRTIO_CRYPTO_HASH_CREATE_SESSION
    then \field{op_flf} is struct virtio_crypto_hash_create_session_flf if
    VIRTIO_CRYPTO_F_REVISION_1 is negotiated and struct virtio_crypto_hash_create_session_flf is
    padded to 56 bytes if NOT negotiated.
\item If the opcode (in \field{header}) is VIRTIO_CRYPTO_MAC_CREATE_SESSION
    then \field{op_flf} is struct virtio_crypto_mac_create_session_flf if
    VIRTIO_CRYPTO_F_REVISION_1 is negotiated and struct virtio_crypto_mac_create_session_flf is
    padded to 56 bytes if NOT negotiated, and \field{op_vlf} is struct
    virtio_crypto_mac_create_session_vlf.
\item If the opcode (in \field{header}) is VIRTIO_CRYPTO_AEAD_CREATE_SESSION
    then \field{op_flf} is struct virtio_crypto_aead_create_session_flf if
    VIRTIO_CRYPTO_F_REVISION_1 is negotiated and struct virtio_crypto_aead_create_session_flf is
    padded to 56 bytes if NOT negotiated, and \field{op_vlf} is struct
    virtio_crypto_aead_create_session_vlf.
\item If the opcode (in \field{header}) is VIRTIO_CRYPTO_AKCIPHER_CREATE_SESSION
    then \field{op_flf} is struct virtio_crypto_akcipher_create_session_flf if
    VIRTIO_CRYPTO_F_REVISION_1 is negotiated and struct virtio_crypto_akcipher_create_session_flf is
    padded to 56 bytes if NOT negotiated, and \field{op_vlf} is struct
    virtio_crypto_akcipher_create_session_vlf.
\item If the opcode (in \field{header}) is VIRTIO_CRYPTO_CIPHER_DESTROY_SESSION
    or VIRTIO_CRYPTO_HASH_DESTROY_SESSION or VIRTIO_CRYPTO_MAC_DESTROY_SESSION or
    VIRTIO_CRYPTO_AEAD_DESTROY_SESSION then \field{op_flf} is struct
    virtio_crypto_destroy_session_flf if VIRTIO_CRYPTO_F_REVISION_1 is negotiated and
    struct virtio_crypto_destroy_session_flf is padded to 56 bytes if NOT negotiated.
\end{itemize*}

\field{op_outcome} stores the result of operation and must be struct
virtio_crypto_destroy_session_input for destroy session or
struct virtio_crypto_create_session_input for create session.

\field{outcome_len} is the size of the structure used.


\paragraph{Session operation}\label{sec:Device Types / Crypto Device / Device
Operation / Control Virtqueue / Session operation}

The session is a handle which describes the cryptographic parameters to be
applied to a number of buffers.

The following structure stores the result of session creation set by the device:

\begin{lstlisting}
struct virtio_crypto_create_session_input {
    le64 session_id;
    le32 status;
    le32 padding;
};
\end{lstlisting}

A request to destroy a session includes the following information:

\begin{lstlisting}
struct virtio_crypto_destroy_session_flf {
    /* Device read only portion */
    le64  session_id;
};

struct virtio_crypto_destroy_session_input {
    /* Device write only portion */
    u8  status;
};
\end{lstlisting}


\subparagraph{Session operation: HASH session}\label{sec:Device Types / Crypto Device / Device
Operation / Control Virtqueue / Session operation / Session operation: HASH session}

The fixed-length parameters of HASH session requests is as follows:

\begin{lstlisting}
struct virtio_crypto_hash_create_session_flf {
    /* Device read only portion */

    /* See VIRTIO_CRYPTO_HASH_* above */
    le32 algo;
    /* hash result length */
    le32 hash_result_len;
};
\end{lstlisting}


\subparagraph{Session operation: MAC session}\label{sec:Device Types / Crypto Device / Device
Operation / Control Virtqueue / Session operation / Session operation: MAC session}

The fixed-length and the variable-length parameters of MAC session requests are as follows:

\begin{lstlisting}
struct virtio_crypto_mac_create_session_flf {
    /* Device read only portion */

    /* See VIRTIO_CRYPTO_MAC_* above */
    le32 algo;
    /* hash result length */
    le32 hash_result_len;
    /* length of authenticated key */
    le32 auth_key_len;
    le32 padding;
};

struct virtio_crypto_mac_create_session_vlf {
    /* Device read only portion */

    /* The authenticated key */
    u8 auth_key[auth_key_len];
};
\end{lstlisting}

The length of \field{auth_key} is specified in \field{auth_key_len} in the struct
virtio_crypto_mac_create_session_flf.


\subparagraph{Session operation: Symmetric algorithms session}\label{sec:Device Types / Crypto Device / Device
Operation / Control Virtqueue / Session operation / Session operation: Symmetric algorithms session}

The request of symmetric session could be the CIPHER algorithms request
or the chain algorithms (chaining CIPHER and HASH/MAC) request.

The fixed-length and the variable-length parameters of CIPHER session requests are as follows:

\begin{lstlisting}
struct virtio_crypto_cipher_session_flf {
    /* Device read only portion */

    /* See VIRTIO_CRYPTO_CIPHER* above */
    le32 algo;
    /* length of key */
    le32 key_len;
#define VIRTIO_CRYPTO_OP_ENCRYPT  1
#define VIRTIO_CRYPTO_OP_DECRYPT  2
    /* encryption or decryption */
    le32 op;
    le32 padding;
};

struct virtio_crypto_cipher_session_vlf {
    /* Device read only portion */

    /* The cipher key */
    u8 cipher_key[key_len];
};
\end{lstlisting}

The length of \field{cipher_key} is specified in \field{key_len} in the struct
virtio_crypto_cipher_session_flf.

The fixed-length and the variable-length parameters of Chain session requests are as follows:

\begin{lstlisting}
struct virtio_crypto_alg_chain_session_flf {
    /* Device read only portion */

#define VIRTIO_CRYPTO_SYM_ALG_CHAIN_ORDER_HASH_THEN_CIPHER  1
#define VIRTIO_CRYPTO_SYM_ALG_CHAIN_ORDER_CIPHER_THEN_HASH  2
    le32 alg_chain_order;
/* Plain hash */
#define VIRTIO_CRYPTO_SYM_HASH_MODE_PLAIN    1
/* Authenticated hash (mac) */
#define VIRTIO_CRYPTO_SYM_HASH_MODE_AUTH     2
/* Nested hash */
#define VIRTIO_CRYPTO_SYM_HASH_MODE_NESTED   3
    le32 hash_mode;
    struct virtio_crypto_cipher_session_flf cipher_hdr;

#define VIRTIO_CRYPTO_ALG_CHAIN_SESS_OP_SPEC_HDR_SIZE  16
    /* fixed length fields, algo specific */
    u8 algo_flf[VIRTIO_CRYPTO_ALG_CHAIN_SESS_OP_SPEC_HDR_SIZE];

    /* length of the additional authenticated data (AAD) in bytes */
    le32 aad_len;
    le32 padding;
};

struct virtio_crypto_alg_chain_session_vlf {
    /* Device read only portion */

    /* The cipher key */
    u8 cipher_key[key_len];
    /* The authenticated key */
    u8 auth_key[auth_key_len];
};
\end{lstlisting}

\field{hash_mode} decides the type used by \field{algo_flf}.

\field{algo_flf} is fixed to 16 bytes and MUST contains or be one of
the following types:
\begin{itemize*}
\item struct virtio_crypto_hash_create_session_flf
\item struct virtio_crypto_mac_create_session_flf
\end{itemize*}
The data of unused part (if has) in \field{algo_flf} will be ignored.

The length of \field{cipher_key} is specified in \field{key_len} in \field{cipher_hdr}.

The length of \field{auth_key} is specified in \field{auth_key_len} in struct
virtio_crypto_mac_create_session_flf.

The fixed-length parameters of Symmetric session requests are as follows:

\begin{lstlisting}
struct virtio_crypto_sym_create_session_flf {
    /* Device read only portion */

#define VIRTIO_CRYPTO_SYM_SESS_OP_SPEC_HDR_SIZE  48
    /* fixed length fields, opcode specific */
    u8 op_flf[VIRTIO_CRYPTO_SYM_SESS_OP_SPEC_HDR_SIZE];

/* No operation */
#define VIRTIO_CRYPTO_SYM_OP_NONE  0
/* Cipher only operation on the data */
#define VIRTIO_CRYPTO_SYM_OP_CIPHER  1
/* Chain any cipher with any hash or mac operation. The order
   depends on the value of alg_chain_order param */
#define VIRTIO_CRYPTO_SYM_OP_ALGORITHM_CHAINING  2
    le32 op_type;
    le32 padding;
};
\end{lstlisting}

\field{op_flf} is fixed to 48 bytes, MUST contains or be one of
the following types:
\begin{itemize*}
\item struct virtio_crypto_cipher_session_flf
\item struct virtio_crypto_alg_chain_session_flf
\end{itemize*}
The data of unused part (if has) in \field{op_flf} will be ignored.

\field{op_type} decides the type used by \field{op_flf}.

The variable-length parameters of Symmetric session requests are as follows:

\begin{lstlisting}
struct virtio_crypto_sym_create_session_vlf {
    /* Device read only portion */
    /* variable length fields, opcode specific */
    u8 op_vlf[vlf_len];
};
\end{lstlisting}

\field{op_vlf} MUST contains or be one of the following types:
\begin{itemize*}
\item struct virtio_crypto_cipher_session_vlf
\item struct virtio_crypto_alg_chain_session_vlf
\end{itemize*}

\field{op_type} in struct virtio_crypto_sym_create_session_flf decides the
type used by \field{op_vlf}.

\field{vlf_len} is the size of the specific structure used.


\subparagraph{Session operation: AEAD session}\label{sec:Device Types / Crypto Device / Device
Operation / Control Virtqueue / Session operation / Session operation: AEAD session}

The fixed-length and the variable-length parameters of AEAD session requests are as follows:

\begin{lstlisting}
struct virtio_crypto_aead_create_session_flf {
    /* Device read only portion */

    /* See VIRTIO_CRYPTO_AEAD_* above */
    le32 algo;
    /* length of key */
    le32 key_len;
    /* Authentication tag length */
    le32 tag_len;
    /* The length of the additional authenticated data (AAD) in bytes */
    le32 aad_len;
    /* encryption or decryption, See above VIRTIO_CRYPTO_OP_* */
    le32 op;
    le32 padding;
};

struct virtio_crypto_aead_create_session_vlf {
    /* Device read only portion */
    u8 key[key_len];
};
\end{lstlisting}

The length of \field{key} is specified in \field{key_len} in struct
virtio_crypto_aead_create_session_flf.

\subparagraph{Session operation: AKCIPHER session}\label{sec:Device Types / Crypto Device / Device
Operation / Control Virtqueue / Session operation / Session operation: AKCIPHER session}

Due to the complexity of asymmetric key algorithms, different algorithms
require different parameters. The following data structures are used as
supplementary parameters to describe the asymmetric algorithm sessions.

For the RSA algorithm, the extra parameters are as follows:
\begin{lstlisting}
struct virtio_crypto_rsa_session_para {
#define VIRTIO_CRYPTO_RSA_RAW_PADDING   0
#define VIRTIO_CRYPTO_RSA_PKCS1_PADDING 1
    le32 padding_algo;

#define VIRTIO_CRYPTO_RSA_NO_HASH   0
#define VIRTIO_CRYPTO_RSA_MD2       1
#define VIRTIO_CRYPTO_RSA_MD3       2
#define VIRTIO_CRYPTO_RSA_MD4       3
#define VIRTIO_CRYPTO_RSA_MD5       4
#define VIRTIO_CRYPTO_RSA_SHA1      5
#define VIRTIO_CRYPTO_RSA_SHA256    6
#define VIRTIO_CRYPTO_RSA_SHA384    7
#define VIRTIO_CRYPTO_RSA_SHA512    8
#define VIRTIO_CRYPTO_RSA_SHA224    9
    le32 hash_algo;
};
\end{lstlisting}

\field{padding_algo} specifies the padding method used by RSA sessions.
\begin{itemize*}
\item If VIRTIO_CRYPTO_RSA_RAW_PADDING is specified, 1) \field{hash_algo}
is ignored, 2) ciphertext and plaintext MUST be padded with leading zeros,
3) and RSA sessions with VIRTIO_CRYPTO_RSA_RAW_PADDING MUST not be used
for verification and signing operations.
\item If VIRTIO_CRYPTO_RSA_PKCS1_PADDING is specified, EMSA-PKCS1-v1_5 padding method
is used (see \hyperref[intro:rfc3447]{PKCS\#1}), \field{hash_algo} specifies how the
digest of the data passed to RSA sessions is calculated when verifying and signing.
It only affects the padding algorithm and is ignored during encryption and decryption.
\end{itemize*}

The ECC algorithms such as the ECDSA algorithm, cannot use custom curves, only the
following known curves can be used (see \hyperref[intro:NIST]{NIST-recommended curves}).

\begin{lstlisting}
#define VIRTIO_CRYPTO_CURVE_UNKNOWN   0
#define VIRTIO_CRYPTO_CURVE_NIST_P192 1
#define VIRTIO_CRYPTO_CURVE_NIST_P224 2
#define VIRTIO_CRYPTO_CURVE_NIST_P256 3
#define VIRTIO_CRYPTO_CURVE_NIST_P384 4
#define VIRTIO_CRYPTO_CURVE_NIST_P521 5
\end{lstlisting}

For the ECDSA algorithm, the extra parameters are as follows:
\begin{lstlisting}
struct virtio_crypto_ecdsa_session_para {
    /* See VIRTIO_CRYPTO_CURVE_* above */
    le32 curve_id;
};
\end{lstlisting}

The fixed-length and the variable-length parameters of AKCIPHER session requests are as follows:
\begin{lstlisting}
struct virtio_crypto_akcipher_create_session_flf {
    /* Device read only portion */

    /* See VIRTIO_CRYPTO_AKCIPHER_* above */
    le32 algo;
#define VIRTIO_CRYPTO_AKCIPHER_KEY_TYPE_PUBLIC 1
#define VIRTIO_CRYPTO_AKCIPHER_KEY_TYPE_PRIVATE 2
    le32 key_type;
    /* length of key */
    le32 key_len;

#define VIRTIO_CRYPTO_AKCIPHER_SESS_ALGO_SPEC_HDR_SIZE 44
    u8 algo_flf[VIRTIO_CRYPTO_AKCIPHER_SESS_ALGO_SPEC_HDR_SIZE];
};

struct virtio_crypto_akcipher_create_session_vlf {
    /* Device read only portion */
    u8 key[key_len];
};
\end{lstlisting}

\field{algo} decides the type used by \field{algo_flf}.
\field{algo_flf} is fixed to 44 bytes and MUST contains of be one the
following structures:
\begin{itemize*}
\item struct virtio_crypto_rsa_session_para
\item struct virtio_crypto_ecdsa_session_para
\end{itemize*}

The length of \field{key} is specified in \field{key_len} in the struct
virtio_crypto_akcipher_create_session_flf.

For the RSA algorithm, the key needs to be encoded according to
\hyperref[intro:rfc3447]{PKCS\#1}. The private key is described with the
RSAPrivateKey structure, and the public key is described with the RSAPublicKey
structure. These ASN.1 structures are encoded in DER encoding rules (see
\hyperref[intro:rfc6025]{rfc6025}).

\begin{lstlisting}
RSAPrivateKey ::= SEQUENCE {
    version          INTEGER,
    modulus          INTEGER,
    publicExponent   INTEGER,
    privateExponent  INTEGER,
    prime1           INTEGER,
    prime2           INTEGER,
    exponent1        INTEGER,
    exponent1        INTEGER,
    coefficient      INTEGER,
    otherPrimeInfos  OtherPrimeInfos OPTIONAL
}

OtherPrimeInfos ::= SEQUENCE SIZE(1...MAX) OF OtherPrimeInfo

OtherPrimeINfo ::= SEQUENCE {
    prime           INTEGER,
    exponent        INTEGER,
    coefficient     INTEGER
}

RSAPublicKey ::= SEQUENCE {
    modulus         INTEGER,
    publicExponent  INTEGER
}
\end{lstlisting}

For the ECDSA algorithm, the private key is encoded according to
\hyperref[intro:rfc5915]{RFC5915}, the private key of the ECDSA algorithm
is described by the ASN.1 structure ECPrivateKey and encoded with DER
encoding rules (see \hyperref[intro:rfc6025]{rfc6025}).

\begin{lstlisting}
ECPrivateKey ::= SEQUNCE {
    version         INTEGER,
    privateKey      OCTET STRING,
    parameters [0]  ECParameters {{ NamedCurve }} OPTIONAL,
    publicKey  [1]  BIT STRING OPTIONAL
}
\end{lstlisting}

The public key of the ECDSA algorithm is encoded according to \hyperref[intro:SEC1]{SEC1},
and the public key of ECDSA is described by the ASN.1 structure ECPoint.
When initializing a session with ECDSA public key, the ECPoint is DER encoded and the
\field{key} only contains the value part of ECPoint, that is, the header part of the
OCTET STRING will be omitted (see \hyperref[intro:rfc6025]{rfc6025}).

\begin{lstlisting}
ECPoint ::= OCTET STRING
\end{lstlisting}

The length of \field{key} is specified in \field{key_len} in
struct virtio_crypto_akcipher_create_session_flf.

\drivernormative{\subparagraph}{Session operation: create session}{Device Types / Crypto Device / Device
Operation / Control Virtqueue / Session operation / Session operation: create session}

\begin{itemize*}
\item The driver MUST set the \field{opcode} field based on service type: CIPHER, HASH, MAC, AEAD or AKCIPHER.
\item The driver MUST set the control general header, the opcode specific header,
    the opcode specific extra parameters and the opcode specific outcome buffer in turn.
    See \ref{sec:Device Types / Crypto Device / Device Operation / Control Virtqueue}.
\item The driver MUST set the \field{reversed} field to zero.
\end{itemize*}

\devicenormative{\subparagraph}{Session operation: create session}{Device Types / Crypto Device / Device
Operation / Control Virtqueue / Session operation / Session operation: create session}

\begin{itemize*}
\item The device MUST use the corresponding opcode specific structure according to the
    \field{opcode} in the control general header.
\item The device MUST extract extra parameters according to the structures used.
\item The device MUST set the \field{status} field to one of the following values of enum
    VIRTIO_CRYPTO_STATUS after finish a session creation:
\begin{itemize*}
\item VIRTIO_CRYPTO_OK if a session is created successfully.
\item VIRTIO_CRYPTO_NOTSUPP if the requested algorithm or operation is unsupported.
\item VIRTIO_CRYPTO_NOSPC if no free session ID (only when the VIRTIO_CRYPTO_F_REVISION_1
    feature bit is negotiated).
\item VIRTIO_CRYPTO_ERR if failure not mentioned above occurs.
\end{itemize*}
\item The device MUST set the \field{session_id} field to a unique session identifier only
    if the status is set to VIRTIO_CRYPTO_OK.
\end{itemize*}

\drivernormative{\subparagraph}{Session operation: destroy session}{Device Types / Crypto Device / Device
Operation / Control Virtqueue / Session operation / Session operation: destroy session}

\begin{itemize*}
\item The driver MUST set the \field{opcode} field based on service type: CIPHER, HASH, MAC, AEAD or AKCIPHER.
\item The driver MUST set the \field{session_id} to a valid value assigned by the device
    when the session was created.
\end{itemize*}

\devicenormative{\subparagraph}{Session operation: destroy session}{Device Types / Crypto Device / Device
Operation / Control Virtqueue / Session operation / Session operation: destroy session}

\begin{itemize*}
\item The device MUST set the \field{status} field to one of the following values of enum VIRTIO_CRYPTO_STATUS.
\begin{itemize*}
\item VIRTIO_CRYPTO_OK if a session is created successfully.
\item VIRTIO_CRYPTO_ERR if any failure occurs.
\end{itemize*}
\end{itemize*}


\subsubsection{Data Virtqueue}\label{sec:Device Types / Crypto Device / Device Operation / Data Virtqueue}

The driver uses the data virtqueues to transmit crypto operation requests to the device,
and completes the crypto operations.

The header for dataq is as follows:

\begin{lstlisting}
struct virtio_crypto_op_header {
#define VIRTIO_CRYPTO_CIPHER_ENCRYPT \
    VIRTIO_CRYPTO_OPCODE(VIRTIO_CRYPTO_SERVICE_CIPHER, 0x00)
#define VIRTIO_CRYPTO_CIPHER_DECRYPT \
    VIRTIO_CRYPTO_OPCODE(VIRTIO_CRYPTO_SERVICE_CIPHER, 0x01)
#define VIRTIO_CRYPTO_HASH \
    VIRTIO_CRYPTO_OPCODE(VIRTIO_CRYPTO_SERVICE_HASH, 0x00)
#define VIRTIO_CRYPTO_MAC \
    VIRTIO_CRYPTO_OPCODE(VIRTIO_CRYPTO_SERVICE_MAC, 0x00)
#define VIRTIO_CRYPTO_AEAD_ENCRYPT \
    VIRTIO_CRYPTO_OPCODE(VIRTIO_CRYPTO_SERVICE_AEAD, 0x00)
#define VIRTIO_CRYPTO_AEAD_DECRYPT \
    VIRTIO_CRYPTO_OPCODE(VIRTIO_CRYPTO_SERVICE_AEAD, 0x01)
#define VIRTIO_CRYPTO_AKCIPHER_ENCRYPT \
    VIRTIO_CRYPTO_OPCODE(VIRTIO_CRYPTO_SERVICE_AKCIPHER, 0x00)
#define VIRTIO_CRYPTO_AKCIPHER_DECRYPT \
    VIRTIO_CRYPTO_OPCODE(VIRTIO_CRYPTO_SERVICE_AKCIPHER, 0x01)
#define VIRTIO_CRYPTO_AKCIPHER_SIGN \
    VIRTIO_CRYPTO_OPCODE(VIRTIO_CRYPTO_SERVICE_AKCIPHER, 0x02)
#define VIRTIO_CRYPTO_AKCIPHER_VERIFY \
    VIRTIO_CRYPTO_OPCODE(VIRTIO_CRYPTO_SERVICE_AKCIPHER, 0x03)
    le32 opcode;
    /* algo should be service-specific algorithms */
    le32 algo;
    le64 session_id;
#define VIRTIO_CRYPTO_FLAG_SESSION_MODE 1
    /* control flag to control the request */
    le32 flag;
    le32 padding;
};
\end{lstlisting}

\begin{note}
If VIRTIO_CRYPTO_F_REVISION_1 is not negotiated the \field{flag} is ignored.

If VIRTIO_CRYPTO_F_REVISION_1 is negotiated but VIRTIO_CRYPTO_F_<SERVICE>_STATELESS_MODE
is not negotiated, then the device SHOULD reject <SERVICE> requests if
VIRTIO_CRYPTO_FLAG_SESSION_MODE is not set (in \field{flag}).
\end{note}

The dataq request is composed of four parts:
\begin{lstlisting}
struct virtio_crypto_op_data_req {
    /* Device read only portion */

    struct virtio_crypto_op_header header;

#define VIRTIO_CRYPTO_DATAQ_OP_SPEC_HDR_LEGACY 48
    /* fixed length fields, opcode specific */
    u8 op_flf[flf_len];

    /* Device read && write portion */
    /* variable length fields, opcode specific */
    u8 op_vlf[vlf_len];

    /* Device write only portion */
    struct virtio_crypto_inhdr inhdr;
};
\end{lstlisting}

\field{header} is a general header (see above).

\field{op_flf} is the opcode (in \field{header}) specific header.

\field{flf_len} depends on the VIRTIO_CRYPTO_F_REVISION_1 feature bit
(see below).

\field{op_vlf} is the opcode (in \field{header}) specific parameters.

\field{vlf_len} is the size of the specific structure used.

\begin{itemize*}
\item If the the opcode (in \field{header}) is VIRTIO_CRYPTO_CIPHER_ENCRYPT
    or VIRTIO_CRYPTO_CIPHER_DECRYPT then:
    \begin{itemize*}
    \item If VIRTIO_CRYPTO_F_CIPHER_STATELESS_MODE is negotiated, \field{op_flf} is
        struct virtio_crypto_sym_data_flf_stateless, and \field{op_vlf} is struct
        virtio_crypto_sym_data_vlf_stateless.
    \item If VIRTIO_CRYPTO_F_CIPHER_STATELESS_MODE is NOT negotiated, \field{op_flf}
        is struct virtio_crypto_sym_data_flf if VIRTIO_CRYPTO_F_REVISION_1 is negotiated
        and struct virtio_crypto_sym_data_flf is padded to 48 bytes if NOT negotiated,
        and \field{op_vlf} is struct virtio_crypto_sym_data_vlf.
    \end{itemize*}
\item If the the opcode (in \field{header}) is VIRTIO_CRYPTO_HASH:
    \begin{itemize*}
    \item If VIRTIO_CRYPTO_F_HASH_STATELESS_MODE is negotiated, \field{op_flf} is
        struct virtio_crypto_hash_data_flf_stateless, and \field{op_vlf} is struct
        virtio_crypto_hash_data_vlf_stateless.
    \item If VIRTIO_CRYPTO_F_HASH_STATELESS_MODE is NOT negotiated, \field{op_flf}
        is struct virtio_crypto_hash_data_flf if VIRTIO_CRYPTO_F_REVISION_1 is negotiated
        and struct virtio_crypto_hash_data_flf is padded to 48 bytes if NOT negotiated,
        and \field{op_vlf} is struct virtio_crypto_hash_data_vlf.
    \end{itemize*}
\item If the the opcode (in \field{header}) is VIRTIO_CRYPTO_MAC:
    \begin{itemize*}
    \item If VIRTIO_CRYPTO_F_MAC_STATELESS_MODE is negotiated, \field{op_flf} is
        struct virtio_crypto_mac_data_flf_stateless, and \field{op_vlf} is struct
        virtio_crypto_mac_data_vlf_stateless.
    \item If VIRTIO_CRYPTO_F_MAC_STATELESS_MODE is NOT negotiated, \field{op_flf}
        is struct virtio_crypto_mac_data_flf if VIRTIO_CRYPTO_F_REVISION_1 is negotiated
        and struct virtio_crypto_mac_data_flf is padded to 48 bytes if NOT negotiated,
        and \field{op_vlf} is struct virtio_crypto_mac_data_vlf.
    \end{itemize*}
\item If the the opcode (in \field{header}) is VIRTIO_CRYPTO_AEAD_ENCRYPT
    or VIRTIO_CRYPTO_AEAD_DECRYPT then:
    \begin{itemize*}
    \item If VIRTIO_CRYPTO_F_AEAD_STATELESS_MODE is negotiated, \field{op_flf} is
        struct virtio_crypto_aead_data_flf_stateless, and \field{op_vlf} is struct
        virtio_crypto_aead_data_vlf_stateless.
    \item If VIRTIO_CRYPTO_F_AEAD_STATELESS_MODE is NOT negotiated, \field{op_flf}
        is struct virtio_crypto_aead_data_flf if VIRTIO_CRYPTO_F_REVISION_1 is negotiated
        and struct virtio_crypto_aead_data_flf is padded to 48 bytes if NOT negotiated,
        and \field{op_vlf} is struct virtio_crypto_aead_data_vlf.
    \end{itemize*}
\item If the opcode (in \field{header}) is VIRTIO_CRYPTO_AKCIPHER_ENCRYPT, VIRTIO_CRYPTO_AKCIPHER_DECRYPT,
    VIRTIO_CRYPTO_AKCIPHER_SIGN or VIRTIO_CRYPTO_AKCIPHER_VERIFY then:
    \begin{itemize*}
    \item If VIRTIO_CRYPTO_F_AKCIPHER_STATELESS_MODE is negotiated, \field{op_flf} is
        struct virtio_crypto_akcipher_data_flf_statless, and \field{op_vlf} is struct
        virtio_crypto_akcipher_data_vlf_stateless.
    \item If VIRTIO_CRYPTO_F_AKCIPHER_STATELESS_MODE is NOT negotiated, \field{op_flf}
        is struct virtio_crypto_akcipher_data_flf if VIRTIO_CRYPTO_F_REVISION_1 is negotiated
        and struct virtio_crypto_akcipher_data_flf is padded to 48 bytes if NOT negotiated,
        and \field{op_vlf} is struct virtio_crypto_akcipher_data_vlf.
    \end{itemize*}
\end{itemize*}

\field{inhdr} is a unified input header that used to return the status of
the operations, is defined as follows:

\begin{lstlisting}
struct virtio_crypto_inhdr {
    u8 status;
};
\end{lstlisting}

\subsubsection{HASH Service Operation}\label{sec:Device Types / Crypto Device / Device Operation / HASH Service Operation}

Session mode HASH service requests are as follows:

\begin{lstlisting}
struct virtio_crypto_hash_data_flf {
    /* length of source data */
    le32 src_data_len;
    /* hash result length */
    le32 hash_result_len;
};

struct virtio_crypto_hash_data_vlf {
    /* Device read only portion */
    /* Source data */
    u8 src_data[src_data_len];

    /* Device write only portion */
    /* Hash result data */
    u8 hash_result[hash_result_len];
};
\end{lstlisting}

Each data request uses the virtio_crypto_hash_data_flf structure and the
virtio_crypto_hash_data_vlf structure to store information used to run the
HASH operations.

\field{src_data} is the source data that will be processed.
\field{src_data_len} is the length of source data.
\field{hash_result} is the result data and \field{hash_result_len} is the length
of it.

Stateless mode HASH service requests are as follows:

\begin{lstlisting}
struct virtio_crypto_hash_data_flf_stateless {
    struct {
        /* See VIRTIO_CRYPTO_HASH_* above */
        le32 algo;
    } sess_para;

    /* length of source data */
    le32 src_data_len;
    /* hash result length */
    le32 hash_result_len;
    le32 reserved;
};
struct virtio_crypto_hash_data_vlf_stateless {
    /* Device read only portion */
    /* Source data */
    u8 src_data[src_data_len];

    /* Device write only portion */
    /* Hash result data */
    u8 hash_result[hash_result_len];
};
\end{lstlisting}

\drivernormative{\paragraph}{HASH Service Operation}{Device Types / Crypto Device / Device Operation / HASH Service Operation}

\begin{itemize*}
\item If the driver uses the session mode, then the driver MUST set \field{session_id}
    in struct virtio_crypto_op_header to a valid value assigned by the device when the
    session was created.
\item If the VIRTIO_CRYPTO_F_HASH_STATELESS_MODE feature bit is negotiated, 1) if the
    driver uses the stateless mode, then the driver MUST set the \field{flag} field in
    struct virtio_crypto_op_header to ZERO and MUST set the fields in struct
    virtio_crypto_hash_data_flf_stateless.sess_para, 2) if the driver uses the session
    mode, then the driver MUST set the \field{flag} field in struct virtio_crypto_op_header
    to VIRTIO_CRYPTO_FLAG_SESSION_MODE.
\item The driver MUST set \field{opcode} in struct virtio_crypto_op_header to VIRTIO_CRYPTO_HASH.
\end{itemize*}

\devicenormative{\paragraph}{HASH Service Operation}{Device Types / Crypto Device / Device Operation / HASH Service Operation}

\begin{itemize*}
\item The device MUST use the corresponding structure according to the \field{opcode}
    in the data general header.
\item If the VIRTIO_CRYPTO_F_HASH_STATELESS_MODE feature bit is negotiated, the device
    MUST parse \field{flag} field in struct virtio_crypto_op_header in order to decide
    which mode the driver uses.
\item The device MUST copy the results of HASH operations in the hash_result[] if HASH
    operations success.
\item The device MUST set \field{status} in struct virtio_crypto_inhdr to one of the
    following values of enum VIRTIO_CRYPTO_STATUS:
\begin{itemize*}
\item VIRTIO_CRYPTO_OK if the operation success.
\item VIRTIO_CRYPTO_NOTSUPP if the requested algorithm or operation is unsupported.
\item VIRTIO_CRYPTO_INVSESS if the session ID invalid when in session mode.
\item VIRTIO_CRYPTO_ERR if any failure not mentioned above occurs.
\end{itemize*}
\end{itemize*}


\subsubsection{MAC Service Operation}\label{sec:Device Types / Crypto Device / Device Operation / MAC Service Operation}

Session mode MAC service requests are as follows:

\begin{lstlisting}
struct virtio_crypto_mac_data_flf {
    struct virtio_crypto_hash_data_flf hdr;
};

struct virtio_crypto_mac_data_vlf {
    /* Device read only portion */
    /* Source data */
    u8 src_data[src_data_len];

    /* Device write only portion */
    /* Hash result data */
    u8 hash_result[hash_result_len];
};
\end{lstlisting}

Each request uses the virtio_crypto_mac_data_flf structure and the
virtio_crypto_mac_data_vlf structure to store information used to run the
MAC operations.

\field{src_data} is the source data that will be processed.
\field{src_data_len} is the length of source data.
\field{hash_result} is the result data and \field{hash_result_len} is the length
of it.

Stateless mode MAC service requests are as follows:

\begin{lstlisting}
struct virtio_crypto_mac_data_flf_stateless {
    struct {
        /* See VIRTIO_CRYPTO_MAC_* above */
        le32 algo;
        /* length of authenticated key */
        le32 auth_key_len;
    } sess_para;

    /* length of source data */
    le32 src_data_len;
    /* hash result length */
    le32 hash_result_len;
};

struct virtio_crypto_mac_data_vlf_stateless {
    /* Device read only portion */
    /* The authenticated key */
    u8 auth_key[auth_key_len];
    /* Source data */
    u8 src_data[src_data_len];

    /* Device write only portion */
    /* Hash result data */
    u8 hash_result[hash_result_len];
};
\end{lstlisting}

\field{auth_key} is the authenticated key that will be used during the process.
\field{auth_key_len} is the length of the key.

\drivernormative{\paragraph}{MAC Service Operation}{Device Types / Crypto Device / Device Operation / MAC Service Operation}

\begin{itemize*}
\item If the driver uses the session mode, then the driver MUST set \field{session_id}
    in struct virtio_crypto_op_header to a valid value assigned by the device when the
    session was created.
\item If the VIRTIO_CRYPTO_F_MAC_STATELESS_MODE feature bit is negotiated, 1) if the
    driver uses the stateless mode, then the driver MUST set the \field{flag} field
    in struct virtio_crypto_op_header to ZERO and MUST set the fields in struct
    virtio_crypto_mac_data_flf_stateless.sess_para, 2) if the driver uses the session
    mode, then the driver MUST set the \field{flag} field in struct virtio_crypto_op_header
    to VIRTIO_CRYPTO_FLAG_SESSION_MODE.
\item The driver MUST set \field{opcode} in struct virtio_crypto_op_header to VIRTIO_CRYPTO_MAC.
\end{itemize*}

\devicenormative{\paragraph}{MAC Service Operation}{Device Types / Crypto Device / Device Operation / MAC Service Operation}

\begin{itemize*}
\item If the VIRTIO_CRYPTO_F_MAC_STATELESS_MODE feature bit is negotiated, the device
    MUST parse \field{flag} field in struct virtio_crypto_op_header in order to decide
	which mode the driver uses.
\item The device MUST copy the results of MAC operations in the hash_result[] if HASH
    operations success.
\item The device MUST set \field{status} in struct virtio_crypto_inhdr to one of the
    following values of enum VIRTIO_CRYPTO_STATUS:
\begin{itemize*}
\item VIRTIO_CRYPTO_OK if the operation success.
\item VIRTIO_CRYPTO_NOTSUPP if the requested algorithm or operation is unsupported.
\item VIRTIO_CRYPTO_INVSESS if the session ID invalid when in session mode.
\item VIRTIO_CRYPTO_ERR if any failure not mentioned above occurs.
\end{itemize*}
\end{itemize*}

\subsubsection{Symmetric algorithms Operation}\label{sec:Device Types / Crypto Device / Device Operation / Symmetric algorithms Operation}

Session mode CIPHER service requests are as follows:

\begin{lstlisting}
struct virtio_crypto_cipher_data_flf {
    /*
     * Byte Length of valid IV/Counter data pointed to by the below iv data.
     *
     * For block ciphers in CBC or F8 mode, or for Kasumi in F8 mode, or for
     *   SNOW3G in UEA2 mode, this is the length of the IV (which
     *   must be the same as the block length of the cipher).
     * For block ciphers in CTR mode, this is the length of the counter
     *   (which must be the same as the block length of the cipher).
     */
    le32 iv_len;
    /* length of source data */
    le32 src_data_len;
    /* length of destination data */
    le32 dst_data_len;
    le32 padding;
};

struct virtio_crypto_cipher_data_vlf {
    /* Device read only portion */

    /*
     * Initialization Vector or Counter data.
     *
     * For block ciphers in CBC or F8 mode, or for Kasumi in F8 mode, or for
     *   SNOW3G in UEA2 mode, this is the Initialization Vector (IV)
     *   value.
     * For block ciphers in CTR mode, this is the counter.
     * For AES-XTS, this is the 128bit tweak, i, from IEEE Std 1619-2007.
     *
     * The IV/Counter will be updated after every partial cryptographic
     * operation.
     */
    u8 iv[iv_len];
    /* Source data */
    u8 src_data[src_data_len];

    /* Device write only portion */
    /* Destination data */
    u8 dst_data[dst_data_len];
};
\end{lstlisting}

Session mode requests of algorithm chaining are as follows:

\begin{lstlisting}
struct virtio_crypto_alg_chain_data_flf {
    le32 iv_len;
    /* Length of source data */
    le32 src_data_len;
    /* Length of destination data */
    le32 dst_data_len;
    /* Starting point for cipher processing in source data */
    le32 cipher_start_src_offset;
    /* Length of the source data that the cipher will be computed on */
    le32 len_to_cipher;
    /* Starting point for hash processing in source data */
    le32 hash_start_src_offset;
    /* Length of the source data that the hash will be computed on */
    le32 len_to_hash;
    /* Length of the additional auth data */
    le32 aad_len;
    /* Length of the hash result */
    le32 hash_result_len;
    le32 reserved;
};

struct virtio_crypto_alg_chain_data_vlf {
    /* Device read only portion */

    /* Initialization Vector or Counter data */
    u8 iv[iv_len];
    /* Source data */
    u8 src_data[src_data_len];
    /* Additional authenticated data if exists */
    u8 aad[aad_len];

    /* Device write only portion */

    /* Destination data */
    u8 dst_data[dst_data_len];
    /* Hash result data */
    u8 hash_result[hash_result_len];
};
\end{lstlisting}

Session mode requests of symmetric algorithm are as follows:

\begin{lstlisting}
struct virtio_crypto_sym_data_flf {
    /* Device read only portion */

#define VIRTIO_CRYPTO_SYM_DATA_REQ_HDR_SIZE    40
    u8 op_type_flf[VIRTIO_CRYPTO_SYM_DATA_REQ_HDR_SIZE];

    /* See above VIRTIO_CRYPTO_SYM_OP_* */
    le32 op_type;
    le32 padding;
};

struct virtio_crypto_sym_data_vlf {
    u8 op_type_vlf[sym_para_len];
};
\end{lstlisting}

Each request uses the virtio_crypto_sym_data_flf structure and the
virtio_crypto_sym_data_flf structure to store information used to run the
CIPHER operations.

\field{op_type_flf} is the \field{op_type} specific header, it MUST starts
with or be one of the following structures:
\begin{itemize*}
\item struct virtio_crypto_cipher_data_flf
\item struct virtio_crypto_alg_chain_data_flf
\end{itemize*}

The length of \field{op_type_flf} is fixed to 40 bytes, the data of unused
part (if has) will be ignored.

\field{op_type_vlf} is the \field{op_type} specific parameters, it MUST starts
with or be one of the following structures:
\begin{itemize*}
\item struct virtio_crypto_cipher_data_vlf
\item struct virtio_crypto_alg_chain_data_vlf
\end{itemize*}

\field{sym_para_len} is the size of the specific structure used.

Stateless mode CIPHER service requests are as follows:

\begin{lstlisting}
struct virtio_crypto_cipher_data_flf_stateless {
    struct {
        /* See VIRTIO_CRYPTO_CIPHER* above */
        le32 algo;
        /* length of key */
        le32 key_len;

        /* See VIRTIO_CRYPTO_OP_* above */
        le32 op;
    } sess_para;

    /*
     * Byte Length of valid IV/Counter data pointed to by the below iv data.
     */
    le32 iv_len;
    /* length of source data */
    le32 src_data_len;
    /* length of destination data */
    le32 dst_data_len;
};

struct virtio_crypto_cipher_data_vlf_stateless {
    /* Device read only portion */

    /* The cipher key */
    u8 cipher_key[key_len];

    /* Initialization Vector or Counter data. */
    u8 iv[iv_len];
    /* Source data */
    u8 src_data[src_data_len];

    /* Device write only portion */
    /* Destination data */
    u8 dst_data[dst_data_len];
};
\end{lstlisting}

Stateless mode requests of algorithm chaining are as follows:

\begin{lstlisting}
struct virtio_crypto_alg_chain_data_flf_stateless {
    struct {
        /* See VIRTIO_CRYPTO_SYM_ALG_CHAIN_ORDER_* above */
        le32 alg_chain_order;
        /* length of the additional authenticated data in bytes */
        le32 aad_len;

        struct {
            /* See VIRTIO_CRYPTO_CIPHER* above */
            le32 algo;
            /* length of key */
            le32 key_len;
            /* See VIRTIO_CRYPTO_OP_* above */
            le32 op;
        } cipher;

        struct {
            /* See VIRTIO_CRYPTO_HASH_* or VIRTIO_CRYPTO_MAC_* above */
            le32 algo;
            /* length of authenticated key */
            le32 auth_key_len;
            /* See VIRTIO_CRYPTO_SYM_HASH_MODE_* above */
            le32 hash_mode;
        } hash;
    } sess_para;

    le32 iv_len;
    /* Length of source data */
    le32 src_data_len;
    /* Length of destination data */
    le32 dst_data_len;
    /* Starting point for cipher processing in source data */
    le32 cipher_start_src_offset;
    /* Length of the source data that the cipher will be computed on */
    le32 len_to_cipher;
    /* Starting point for hash processing in source data */
    le32 hash_start_src_offset;
    /* Length of the source data that the hash will be computed on */
    le32 len_to_hash;
    /* Length of the additional auth data */
    le32 aad_len;
    /* Length of the hash result */
    le32 hash_result_len;
    le32 reserved;
};

struct virtio_crypto_alg_chain_data_vlf_stateless {
    /* Device read only portion */

    /* The cipher key */
    u8 cipher_key[key_len];
    /* The auth key */
    u8 auth_key[auth_key_len];
    /* Initialization Vector or Counter data */
    u8 iv[iv_len];
    /* Additional authenticated data if exists */
    u8 aad[aad_len];
    /* Source data */
    u8 src_data[src_data_len];

    /* Device write only portion */

    /* Destination data */
    u8 dst_data[dst_data_len];
    /* Hash result data */
    u8 hash_result[hash_result_len];
};
\end{lstlisting}

Stateless mode requests of symmetric algorithm are as follows:

\begin{lstlisting}
struct virtio_crypto_sym_data_flf_stateless {
    /* Device read only portion */
#define VIRTIO_CRYPTO_SYM_DATE_REQ_HDR_STATELESS_SIZE    72
    u8 op_type_flf[VIRTIO_CRYPTO_SYM_DATE_REQ_HDR_STATELESS_SIZE];

    /* Device write only portion */
    /* See above VIRTIO_CRYPTO_SYM_OP_* */
    le32 op_type;
};

struct virtio_crypto_sym_data_vlf_stateless {
    u8 op_type_vlf[sym_para_len];
};
\end{lstlisting}

\field{op_type_flf} is the \field{op_type} specific header, it MUST starts
with or be one of the following structures:
\begin{itemize*}
\item struct virtio_crypto_cipher_data_flf_stateless
\item struct virtio_crypto_alg_chain_data_flf_stateless
\end{itemize*}

The length of \field{op_type_flf} is fixed to 72 bytes, the data of unused
part (if has) will be ignored.

\field{op_type_vlf} is the \field{op_type} specific parameters, it MUST starts
with or be one of the following structures:
\begin{itemize*}
\item struct virtio_crypto_cipher_data_vlf_stateless
\item struct virtio_crypto_alg_chain_data_vlf_stateless
\end{itemize*}

\field{sym_para_len} is the size of the specific structure used.

\drivernormative{\paragraph}{Symmetric algorithms Operation}{Device Types / Crypto Device / Device Operation / Symmetric algorithms Operation}

\begin{itemize*}
\item If the driver uses the session mode, then the driver MUST set \field{session_id}
    in struct virtio_crypto_op_header to a valid value assigned by the device when the
    session was created.
\item If the VIRTIO_CRYPTO_F_CIPHER_STATELESS_MODE feature bit is negotiated, 1) if the
    driver uses the stateless mode, then the driver MUST set the \field{flag} field in
    struct virtio_crypto_op_header to ZERO and MUST set the fields in struct
    virtio_crypto_cipher_data_flf_stateless.sess_para or struct
    virtio_crypto_alg_chain_data_flf_stateless.sess_para, 2) if the driver uses the
    session mode, then the driver MUST set the \field{flag} field in struct
    virtio_crypto_op_header to VIRTIO_CRYPTO_FLAG_SESSION_MODE.
\item The driver MUST set the \field{opcode} field in struct virtio_crypto_op_header
    to VIRTIO_CRYPTO_CIPHER_ENCRYPT or VIRTIO_CRYPTO_CIPHER_DECRYPT.
\item The driver MUST specify the fields of struct virtio_crypto_cipher_data_flf in
    struct virtio_crypto_sym_data_flf and struct virtio_crypto_cipher_data_vlf in
    struct virtio_crypto_sym_data_vlf if the request is based on VIRTIO_CRYPTO_SYM_OP_CIPHER.
\item The driver MUST specify the fields of struct virtio_crypto_alg_chain_data_flf
    in struct virtio_crypto_sym_data_flf and struct virtio_crypto_alg_chain_data_vlf
    in struct virtio_crypto_sym_data_vlf if the request is of the VIRTIO_CRYPTO_SYM_OP_ALGORITHM_CHAINING
    type.
\end{itemize*}

\devicenormative{\paragraph}{Symmetric algorithms Operation}{Device Types / Crypto Device / Device Operation / Symmetric algorithms Operation}

\begin{itemize*}
\item If the VIRTIO_CRYPTO_F_CIPHER_STATELESS_MODE feature bit is negotiated, the device
    MUST parse \field{flag} field in struct virtio_crypto_op_header in order to decide
	which mode the driver uses.
\item The device MUST parse the virtio_crypto_sym_data_req based on the \field{opcode}
    field in general header.
\item The device MUST parse the fields of struct virtio_crypto_cipher_data_flf in
    struct virtio_crypto_sym_data_flf and struct virtio_crypto_cipher_data_vlf in
    struct virtio_crypto_sym_data_vlf if the request is based on VIRTIO_CRYPTO_SYM_OP_CIPHER.
\item The device MUST parse the fields of struct virtio_crypto_alg_chain_data_flf
    in struct virtio_crypto_sym_data_flf and struct virtio_crypto_alg_chain_data_vlf
    in struct virtio_crypto_sym_data_vlf if the request is of the VIRTIO_CRYPTO_SYM_OP_ALGORITHM_CHAINING
    type.
\item The device MUST copy the result of cryptographic operation in the dst_data[] in
    both plain CIPHER mode and algorithms chain mode.
\item The device MUST check the \field{para}.\field{add_len} is bigger than 0 before
    parse the additional authenticated data in plain algorithms chain mode.
\item The device MUST copy the result of HASH/MAC operation in the hash_result[] is
    of the VIRTIO_CRYPTO_SYM_OP_ALGORITHM_CHAINING type.
\item The device MUST set the \field{status} field in struct virtio_crypto_inhdr to
    one of the following values of enum VIRTIO_CRYPTO_STATUS:
\begin{itemize*}
\item VIRTIO_CRYPTO_OK if the operation success.
\item VIRTIO_CRYPTO_NOTSUPP if the requested algorithm or operation is unsupported.
\item VIRTIO_CRYPTO_INVSESS if the session ID is invalid in session mode.
\item VIRTIO_CRYPTO_ERR if failure not mentioned above occurs.
\end{itemize*}
\end{itemize*}

\subsubsection{AEAD Service Operation}\label{sec:Device Types / Crypto Device / Device Operation / AEAD Service Operation}

Session mode requests of symmetric algorithm are as follows:

\begin{lstlisting}
struct virtio_crypto_aead_data_flf {
    /*
     * Byte Length of valid IV data.
     *
     * For GCM mode, this is either 12 (for 96-bit IVs) or 16, in which
     *   case iv points to J0.
     * For CCM mode, this is the length of the nonce, which can be in the
     *   range 7 to 13 inclusive.
     */
    le32 iv_len;
    /* length of additional auth data */
    le32 aad_len;
    /* length of source data */
    le32 src_data_len;
    /* length of dst data, this should be at least src_data_len + tag_len */
    le32 dst_data_len;
    /* Authentication tag length */
    le32 tag_len;
    le32 reserved;
};

struct virtio_crypto_aead_data_vlf {
    /* Device read only portion */

    /*
     * Initialization Vector data.
     *
     * For GCM mode, this is either the IV (if the length is 96 bits) or J0
     *   (for other sizes), where J0 is as defined by NIST SP800-38D.
     *   Regardless of the IV length, a full 16 bytes needs to be allocated.
     * For CCM mode, the first byte is reserved, and the nonce should be
     *   written starting at &iv[1] (to allow space for the implementation
     *   to write in the flags in the first byte).  Note that a full 16 bytes
     *   should be allocated, even though the iv_len field will have
     *   a value less than this.
     *
     * The IV will be updated after every partial cryptographic operation.
     */
    u8 iv[iv_len];
    /* Source data */
    u8 src_data[src_data_len];
    /* Additional authenticated data if exists */
    u8 aad[aad_len];

    /* Device write only portion */
    /* Pointer to output data */
    u8 dst_data[dst_data_len];
};
\end{lstlisting}

Each request uses the virtio_crypto_aead_data_flf structure and the
virtio_crypto_aead_data_flf structure to store information used to run the
AEAD operations.

Stateless mode AEAD service requests are as follows:

\begin{lstlisting}
struct virtio_crypto_aead_data_flf_stateless {
    struct {
        /* See VIRTIO_CRYPTO_AEAD_* above */
        le32 algo;
        /* length of key */
        le32 key_len;
        /* encrypt or decrypt, See above VIRTIO_CRYPTO_OP_* */
        le32 op;
    } sess_para;

    /* Byte Length of valid IV data. */
    le32 iv_len;
    /* Authentication tag length */
    le32 tag_len;
    /* length of additional auth data */
    le32 aad_len;
    /* length of source data */
    le32 src_data_len;
    /* length of dst data, this should be at least src_data_len + tag_len */
    le32 dst_data_len;
};

struct virtio_crypto_aead_data_vlf_stateless {
    /* Device read only portion */

    /* The cipher key */
    u8 key[key_len];
    /* Initialization Vector data. */
    u8 iv[iv_len];
    /* Source data */
    u8 src_data[src_data_len];
    /* Additional authenticated data if exists */
    u8 aad[aad_len];

    /* Device write only portion */
    /* Pointer to output data */
    u8 dst_data[dst_data_len];
};
\end{lstlisting}

\drivernormative{\paragraph}{AEAD Service Operation}{Device Types / Crypto Device / Device Operation / AEAD Service Operation}

\begin{itemize*}
\item If the driver uses the session mode, then the driver MUST set
    \field{session_id} in struct virtio_crypto_op_header to a valid value assigned
    by the device when the session was created.
\item If the VIRTIO_CRYPTO_F_AEAD_STATELESS_MODE feature bit is negotiated, 1) if
    the driver uses the stateless mode, then the driver MUST set the \field{flag}
    field in struct virtio_crypto_op_header to ZERO and MUST set the fields in
    struct virtio_crypto_aead_data_flf_stateless.sess_para, 2) if the driver uses
    the session mode, then the driver MUST set the \field{flag} field in struct
    virtio_crypto_op_header to VIRTIO_CRYPTO_FLAG_SESSION_MODE.
\item The driver MUST set the \field{opcode} field in struct virtio_crypto_op_header
    to VIRTIO_CRYPTO_AEAD_ENCRYPT or VIRTIO_CRYPTO_AEAD_DECRYPT.
\end{itemize*}

\devicenormative{\paragraph}{AEAD Service Operation}{Device Types / Crypto Device / Device Operation / AEAD Service Operation}

\begin{itemize*}
\item If the VIRTIO_CRYPTO_F_AEAD_STATELESS_MODE feature bit is negotiated, the
    device MUST parse the virtio_crypto_aead_data_vlf_stateless based on the \field{opcode}
	field in general header.
\item The device MUST copy the result of cryptographic operation in the dst_data[].
\item The device MUST copy the authentication tag in the dst_data[] offset the cipher result.
\item The device MUST set the \field{status} field in struct virtio_crypto_inhdr to
    one of the following values of enum VIRTIO_CRYPTO_STATUS:
\item When the \field{opcode} field is VIRTIO_CRYPTO_AEAD_DECRYPT, the device MUST
    verify and return the verification result to the driver.
\begin{itemize*}
\item VIRTIO_CRYPTO_OK if the operation success.
\item VIRTIO_CRYPTO_NOTSUPP if the requested algorithm or operation is unsupported.
\item VIRTIO_CRYPTO_BADMSG if the verification result is incorrect.
\item VIRTIO_CRYPTO_INVSESS if the session ID invalid when in session mode.
\item VIRTIO_CRYPTO_ERR if any failure not mentioned above occurs.
\end{itemize*}
\end{itemize*}

\subsubsection{AKCIPHER Service Operation}\label{sec:Device Types / Crypto Device / Device Operation / AKCIPHER Service Operation}

Session mode AKCIPHER requests are as follows:

\begin{lstlisting}
struct virtio_crypto_akcipher_data_flf {
    /* length of source data */
    le32 src_data_len;
    /* length of dst data */
    le32 dst_data_len;
};

struct virtio_crypto_akcipher_data_vlf {
    /* Device read only portion */
    /* Source data */
    u8 src_data[src_data_len];

    /* Device write only portion */
    /* Pointer to output data */
    u8 dst_data[dst_data_len];
};
\end{lstlisting}

Each data request uses the virtio_crypto_akcipher_flf structure and the virtio_crypto_akcipher_data_vlf
structure to store information used to run the AKCIPHER operations.

For encryption, decryption, and signing:
\field{src_data} is the source data that will be processed, note that for signing operations,
src_data stores the data to be signed, which usually is the digest of some data rather than the
data itself.
\field{src_data_len} is the length of source data.
\field{dst_result} is the result data and \field{dst_data_len} is the length of it. Note that the
length of the result is not always exactly equal to dst_data_len, the driver needs to check how
many bytes the device has written and calculate the actual length of the result.

For verification:
\field{src_data_len} refers to the length of the signature, and \field{dst_data_len} refers to
the length of signed data, where the signed data is usually the digest of some data.
\field{src_data} is spliced by the signature and the signed data, the src_data with the lower
address stores the signature, and the higher address stores the signed data.
\field{dst_data} is always empty for verification.

Different algorithms have different signature formats.
For the RSA algorithm, the result is determined by the padding algorithm specified by
\field{padding_algo} in structure virtio_crypto_rsa_session_para.

For the ECDSA algorithm, the signature is composed of the following
ASN.1 structure (see \hyperref[intro:rfc3279]{RFC3279})
and MUST be DER encoded (see \hyperref[intro:rfc6025]{rfc6025}).

\begin{lstlisting}
Ecdsa-Sig-Value ::= SEQUENCE {
    r INTEGER,
    s INTEGER
}
\end{lstlisting}

Stateless mode AKCIPHER service requests are as follows:
\begin{lstlisting}
struct virtio_crypto_akcipher_data_flf_stateless {
    struct {
        /* See VIRTIO_CYRPTO_AKCIPHER* above */
        le32 algo;
        /* See VIRTIO_CRYPTO_AKCIPHER_KEY_TYPE_* above */
        le32 key_type;
        /* length of key */
        le32 key_len;

        /* algothrim specific parameters described above */
        union {
            struct virtio_crypto_rsa_session_para rsa;
            struct virtio_crypto_ecdsa_session_para ecdsa;
        } u;
    } sess_para;

    /* length of source data */
    le32 src_data_len;
    /* length of destination data */
    le32 dst_data_len;
};

struct virtio_crypto_akcipher_data_vlf_stateless {
    /* Device read only portion */
    u8 akcipher_key[key_len];

    /* Source data */
    u8 src_data[src_data_len];

    /* Device write only portion */
    u8 dst_data[dst_data_len];
};
\end{lstlisting}

In stateless mode, the format of key and signature, the meaning of src_data and dst_data, are all the same
with session mode.

\drivernormative{\paragraph}{AKCIPHER Service Operation}{Device Types / Crypto Device / Device Operation / AKCIPHER Service Operation}

\begin{itemize*}
\item If the driver uses the session mode, then the driver MUST set
    \field{session_id} in struct virtio_crypto_op_header to a valid
    value assigned by the device when the session was created.
\item If the VIRTIO_CRYPTO_F_AKCIPHER_STATELESS_MODE feature bit is negotiated, 1) if the
    driver uses the stateless mode, then the driver MUST set the \field{flag} field in
    struct virtio_crypto_op_header to ZERO and MUST set the fields in struct
    virtio_crypto_akcipher_flf_stateless.sess_para, 2) if the driver uses the session
    mode, then the driver MUST set the \field{flag} field in struct virtio_crypto_op_header
    to VIRTIO_CRYPTO_FLAG_SESSION_MODE.
\item The driver MUST set the \field{opcode} field in struct virtio_crypto_op_header
    to one of VIRTIO_CRYPTO_AKCIPHER_ENCRYPT, VIRTIO_CRYPTO_AKCIPHER_DESTROY_SESSION,
    VIRTIO_CRYPTO_AKCIPHER_SIGN, and VIRTIO_CRYPTO_AKCIPHER_VERIFY.
\end{itemize*}

\devicenormative{\paragraph}{AKCIPHER Service Operation}{Device Types / Crypto Device / Device Operation / AKCIPHER Service Operation}

\begin{itemize*}
\item If the VIRTIO_CRYPTO_F_AKCIPHER_STATELESS_MODE feature bit is negotiated, the
    device MUST parse the virtio_crypto_akcipher_data_vlf_stateless based on the \field{opcode}
    field in general header.
\item The device MUST copy the result of cryptographic operation in the dst_data[].
\item The device MUST set the \field{status} field in struct virtio_crypto_inhdr to
    one of the following values of enum VIRTIO_CRYPTO_STATUS:
\begin{itemize*}
\item VIRTIO_CRYPTO_OK if the operation success.
\item VIRTIO_CRYPTO_NOTSUPP if the requested algorithm or operation is unsupported.
\item VIRTIO_CRYPTO_BADMSG if the verification result is incorrect.
\item VIRTIO_CRYPTO_INVSESS if the session ID invalid when in session mode.
\item VIRTIO_CRYPTO_KEY_REJECTED if the signature verification failed.
\item VIRTIO_CRYPTO_ERR if any failure not mentioned above occurs.
\end{itemize*}
\end{itemize*}

\section{Crypto Device}\label{sec:Device Types / Crypto Device}

The virtio crypto device is a virtual cryptography device as well as a
virtual cryptographic accelerator. The virtio crypto device provides the
following crypto services: CIPHER, MAC, HASH, AEAD and AKCIPHER. Virtio crypto
devices have a single control queue and at least one data queue. Crypto
operation requests are placed into a data queue, and serviced by the
device. Some crypto operation requests are only valid in the context of a
session. The role of the control queue is facilitating control operation
requests. Sessions management is realized with control operation
requests.

\subsection{Device ID}\label{sec:Device Types / Crypto Device / Device ID}

20

\subsection{Virtqueues}\label{sec:Device Types / Crypto Device / Virtqueues}

\begin{description}
\item[0] dataq1
\item[\ldots]
\item[N-1] dataqN
\item[N] controlq
\end{description}

N is set by \field{max_dataqueues}.

\subsection{Feature bits}\label{sec:Device Types / Crypto Device / Feature bits}

\begin{description}
\item VIRTIO_CRYPTO_F_REVISION_1 (0) revision 1. Revision 1 has a specific
    request format and other enhancements (which result in some additional
    requirements).
\item VIRTIO_CRYPTO_F_CIPHER_STATELESS_MODE (1) stateless mode requests are
    supported by the CIPHER service.
\item VIRTIO_CRYPTO_F_HASH_STATELESS_MODE (2) stateless mode requests are
    supported by the HASH service.
\item VIRTIO_CRYPTO_F_MAC_STATELESS_MODE (3) stateless mode requests are
    supported by the MAC service.
\item VIRTIO_CRYPTO_F_AEAD_STATELESS_MODE (4) stateless mode requests are
    supported by the AEAD service.
\item VIRTIO_CRYPTO_F_AKCIPHER_STATELESS_MODE (5) stateless mode requests are
    supported by the AKCIPHER service.
\end{description}


\subsubsection{Feature bit requirements}\label{sec:Device Types / Crypto Device / Feature bit requirements}

Some crypto feature bits require other crypto feature bits
(see \ref{drivernormative:Basic Facilities of a Virtio Device / Feature Bits}):

\begin{description}
\item[VIRTIO_CRYPTO_F_CIPHER_STATELESS_MODE] Requires VIRTIO_CRYPTO_F_REVISION_1.
\item[VIRTIO_CRYPTO_F_HASH_STATELESS_MODE] Requires VIRTIO_CRYPTO_F_REVISION_1.
\item[VIRTIO_CRYPTO_F_MAC_STATELESS_MODE] Requires VIRTIO_CRYPTO_F_REVISION_1.
\item[VIRTIO_CRYPTO_F_AEAD_STATELESS_MODE] Requires VIRTIO_CRYPTO_F_REVISION_1.
\item[VIRTIO_CRYPTO_F_AKCIPHER_STATELESS_MODE] Requires VIRTIO_CRYPTO_F_REVISION_1.
\end{description}

\subsection{Supported crypto services}\label{sec:Device Types / Crypto Device / Supported crypto services}

The following crypto services are defined:

\begin{lstlisting}
/* CIPHER (Symmetric Key Cipher) service */
#define VIRTIO_CRYPTO_SERVICE_CIPHER 0
/* HASH service */
#define VIRTIO_CRYPTO_SERVICE_HASH   1
/* MAC (Message Authentication Codes) service */
#define VIRTIO_CRYPTO_SERVICE_MAC    2
/* AEAD (Authenticated Encryption with Associated Data) service */
#define VIRTIO_CRYPTO_SERVICE_AEAD   3
/* AKCIPHER (Asymmetric Key Cipher) service */
#define VIRTIO_CRYPTO_SERVICE_AKCIPHER 4
\end{lstlisting}

The above constants designate bits used to indicate the which of crypto services are
offered by the device as described in, see \ref{sec:Device Types / Crypto Device / Device configuration layout}.

\subsubsection{CIPHER services}\label{sec:Device Types / Crypto Device / Supported crypto services / CIPHER services}

The following CIPHER algorithms are defined:

\begin{lstlisting}
#define VIRTIO_CRYPTO_NO_CIPHER                 0
#define VIRTIO_CRYPTO_CIPHER_ARC4               1
#define VIRTIO_CRYPTO_CIPHER_AES_ECB            2
#define VIRTIO_CRYPTO_CIPHER_AES_CBC            3
#define VIRTIO_CRYPTO_CIPHER_AES_CTR            4
#define VIRTIO_CRYPTO_CIPHER_DES_ECB            5
#define VIRTIO_CRYPTO_CIPHER_DES_CBC            6
#define VIRTIO_CRYPTO_CIPHER_3DES_ECB           7
#define VIRTIO_CRYPTO_CIPHER_3DES_CBC           8
#define VIRTIO_CRYPTO_CIPHER_3DES_CTR           9
#define VIRTIO_CRYPTO_CIPHER_KASUMI_F8          10
#define VIRTIO_CRYPTO_CIPHER_SNOW3G_UEA2        11
#define VIRTIO_CRYPTO_CIPHER_AES_F8             12
#define VIRTIO_CRYPTO_CIPHER_AES_XTS            13
#define VIRTIO_CRYPTO_CIPHER_ZUC_EEA3           14
\end{lstlisting}

The above constants have two usages:
\begin{enumerate}
\item As bit numbers, used to tell the driver which CIPHER algorithms
are supported by the device, see \ref{sec:Device Types / Crypto Device / Device configuration layout}.
\item As values, used to designate the algorithm in (CIPHER type) crypto
operation requests, see \ref{sec:Device Types / Crypto Device / Device Operation / Control Virtqueue / Session operation}.
\end{enumerate}

\subsubsection{HASH services}\label{sec:Device Types / Crypto Device / Supported crypto services / HASH services}

The following HASH algorithms are defined:

\begin{lstlisting}
#define VIRTIO_CRYPTO_NO_HASH            0
#define VIRTIO_CRYPTO_HASH_MD5           1
#define VIRTIO_CRYPTO_HASH_SHA1          2
#define VIRTIO_CRYPTO_HASH_SHA_224       3
#define VIRTIO_CRYPTO_HASH_SHA_256       4
#define VIRTIO_CRYPTO_HASH_SHA_384       5
#define VIRTIO_CRYPTO_HASH_SHA_512       6
#define VIRTIO_CRYPTO_HASH_SHA3_224      7
#define VIRTIO_CRYPTO_HASH_SHA3_256      8
#define VIRTIO_CRYPTO_HASH_SHA3_384      9
#define VIRTIO_CRYPTO_HASH_SHA3_512      10
#define VIRTIO_CRYPTO_HASH_SHA3_SHAKE128      11
#define VIRTIO_CRYPTO_HASH_SHA3_SHAKE256      12
\end{lstlisting}

The above constants have two usages:
\begin{enumerate}
\item As bit numbers, used to tell the driver which HASH algorithms
are supported by the device, see \ref{sec:Device Types / Crypto Device / Device configuration layout}.
\item As values, used to designate the algorithm in (HASH type) crypto
operation requires, see \ref{sec:Device Types / Crypto Device / Device Operation / Control Virtqueue / Session operation}.
\end{enumerate}

\subsubsection{MAC services}\label{sec:Device Types / Crypto Device / Supported crypto services / MAC services}

The following MAC algorithms are defined:

\begin{lstlisting}
#define VIRTIO_CRYPTO_NO_MAC                       0
#define VIRTIO_CRYPTO_MAC_HMAC_MD5                 1
#define VIRTIO_CRYPTO_MAC_HMAC_SHA1                2
#define VIRTIO_CRYPTO_MAC_HMAC_SHA_224             3
#define VIRTIO_CRYPTO_MAC_HMAC_SHA_256             4
#define VIRTIO_CRYPTO_MAC_HMAC_SHA_384             5
#define VIRTIO_CRYPTO_MAC_HMAC_SHA_512             6
#define VIRTIO_CRYPTO_MAC_CMAC_3DES                25
#define VIRTIO_CRYPTO_MAC_CMAC_AES                 26
#define VIRTIO_CRYPTO_MAC_KASUMI_F9                27
#define VIRTIO_CRYPTO_MAC_SNOW3G_UIA2              28
#define VIRTIO_CRYPTO_MAC_GMAC_AES                 41
#define VIRTIO_CRYPTO_MAC_GMAC_TWOFISH             42
#define VIRTIO_CRYPTO_MAC_CBCMAC_AES               49
#define VIRTIO_CRYPTO_MAC_CBCMAC_KASUMI_F9         50
#define VIRTIO_CRYPTO_MAC_XCBC_AES                 53
#define VIRTIO_CRYPTO_MAC_ZUC_EIA3                 54
\end{lstlisting}

The above constants have two usages:
\begin{enumerate}
\item As bit numbers, used to tell the driver which MAC algorithms
are supported by the device, see \ref{sec:Device Types / Crypto Device / Device configuration layout}.
\item As values, used to designate the algorithm in (MAC type) crypto
operation requests, see \ref{sec:Device Types / Crypto Device / Device Operation / Control Virtqueue / Session operation}.
\end{enumerate}

\subsubsection{AEAD services}\label{sec:Device Types / Crypto Device / Supported crypto services / AEAD services}

The following AEAD algorithms are defined:

\begin{lstlisting}
#define VIRTIO_CRYPTO_NO_AEAD     0
#define VIRTIO_CRYPTO_AEAD_GCM    1
#define VIRTIO_CRYPTO_AEAD_CCM    2
#define VIRTIO_CRYPTO_AEAD_CHACHA20_POLY1305  3
\end{lstlisting}

The above constants have two usages:
\begin{enumerate}
\item As bit numbers, used to tell the driver which AEAD algorithms
are supported by the device, see \ref{sec:Device Types / Crypto Device / Device configuration layout}.
\item As values, used to designate the algorithm in (DEAD type) crypto
operation requests, see \ref{sec:Device Types / Crypto Device / Device Operation / Control Virtqueue / Session operation}.
\end{enumerate}

\subsubsection{AKCIPHER services}\label{sec: Device Types / Crypto Device / Supported crypto services / AKCIPHER services}

The following AKCIPHER algorithms are defined:
\begin{lstlisting}
#define VIRTIO_CRYPTO_NO_AKCIPHER 0
#define VIRTIO_CRYPTO_AKCIPHER_RSA   1
#define VIRTIO_CRYPTO_AKCIPHER_ECDSA 2
\end{lstlisting}

The above constants have two usages:
\begin{enumerate}
\item As bit numbers, used to tell the driver which AKCIPHER algorithms
are supported by the device, see \ref{sec:Device Types / Crypto Device / Device configuration layout}.
\item As values, used to designate the algorithm in asymmetric crypto operation requests,
see \ref{sec:Device Types / Crypto Device / Device Operation / Control Virtqueue / Session operation}.
\end{enumerate}


\subsection{Device configuration layout}\label{sec:Device Types / Crypto Device / Device configuration layout}

Crypto device configuration uses the following layout structure:

\begin{lstlisting}
struct virtio_crypto_config {
    le32 status;
    le32 max_dataqueues;
    le32 crypto_services;
    /* Detailed algorithms mask */
    le32 cipher_algo_l;
    le32 cipher_algo_h;
    le32 hash_algo;
    le32 mac_algo_l;
    le32 mac_algo_h;
    le32 aead_algo;
    /* Maximum length of cipher key in bytes */
    le32 max_cipher_key_len;
    /* Maximum length of authenticated key in bytes */
    le32 max_auth_key_len;
    le32 akcipher_algo;
    /* Maximum size of each crypto request's content in bytes */
    le64 max_size;
};
\end{lstlisting}

\begin{description}
\item Currently, only one \field{status} bit is defined: VIRTIO_CRYPTO_S_HW_READY
    set indicates that the device is ready to process requests, this bit is read-only
    for the driver
\begin{lstlisting}
#define VIRTIO_CRYPTO_S_HW_READY  (1 << 0)
\end{lstlisting}

\item [\field{max_dataqueues}] is the maximum number of data virtqueues that can
    be configured by the device. The driver MAY use only one data queue, or it
    can use more to achieve better performance.

\item [\field{crypto_services}] crypto service offered, see \ref{sec:Device Types / Crypto Device / Supported crypto services}.

\item [\field{cipher_algo_l}] CIPHER algorithms bits 0-31, see \ref{sec:Device Types / Crypto Device / Supported crypto services  / CIPHER services}.

\item [\field{cipher_algo_h}] CIPHER algorithms bits 32-63, see \ref{sec:Device Types / Crypto Device / Supported crypto services  / CIPHER services}.

\item [\field{hash_algo}] HASH algorithms bits, see \ref{sec:Device Types / Crypto Device / Supported crypto services  / HASH services}.

\item [\field{mac_algo_l}] MAC algorithms bits 0-31, see \ref{sec:Device Types / Crypto Device / Supported crypto services  / MAC services}.

\item [\field{mac_algo_h}] MAC algorithms bits 32-63, see \ref{sec:Device Types / Crypto Device / Supported crypto services  / MAC services}.

\item [\field{aead_algo}] AEAD algorithms bits, see \ref{sec:Device Types / Crypto Device / Supported crypto services  / AEAD services}.

\item [\field{max_cipher_key_len}] is the maximum length of cipher key supported by the device.

\item [\field{max_auth_key_len}] is the maximum length of authenticated key supported by the device.

\item [\field{akcipher_algo}] AKCIPHER algorithms bit 0-31, see \ref{sec: Device Types / Crypto Device / Supported crypto services / AKCIPHER services}.

\item [\field{max_size}] is the maximum size of the variable-length parameters of
    data operation of each crypto request's content supported by the device.
\end{description}

\begin{note}
Unless explicitly stated otherwise all lengths and sizes are in bytes.
\end{note}

\devicenormative{\subsubsection}{Device configuration layout}{Device Types / Crypto Device / Device configuration layout}

\begin{itemize*}
\item The device MUST set \field{max_dataqueues} to between 1 and 65535 inclusive.
\item The device MUST set the \field{status} with valid flags, undefined flags MUST NOT be set.
\item The device MUST accept and handle requests after \field{status} is set to VIRTIO_CRYPTO_S_HW_READY.
\item The device MUST set \field{crypto_services} based on the crypto services the device offers.
\item The device MUST set detailed algorithms masks for each service advertised by \field{crypto_services}.
    The device MUST NOT set the not defined algorithms bits.
\item The device MUST set \field{max_size} to show the maximum size of crypto request the device supports.
\item The device MUST set \field{max_cipher_key_len} to show the maximum length of cipher key if the
    device supports CIPHER service.
\item The device MUST set \field{max_auth_key_len} to show the maximum length of authenticated key if
    the device supports MAC service.
\end{itemize*}

\drivernormative{\subsubsection}{Device configuration layout}{Device Types / Crypto Device / Device configuration layout}

\begin{itemize*}
\item The driver MUST read the \field{status} from the bottom bit of status to check whether the
    VIRTIO_CRYPTO_S_HW_READY is set, and the driver MUST reread it after device reset.
\item The driver MUST NOT transmit any requests to the device if the VIRTIO_CRYPTO_S_HW_READY is not set.
\item The driver MUST read \field{max_dataqueues} field to discover the number of data queues the device supports.
\item The driver MUST read \field{crypto_services} field to discover which services the device is able to offer.
\item The driver SHOULD ignore the not defined algorithms bits.
\item The driver MUST read the detailed algorithms fields based on \field{crypto_services} field.
\item The driver SHOULD read \field{max_size} to discover the maximum size of the variable-length
    parameters of data operation of the crypto request's content the device supports and MUST
    guarantee that the size of each crypto request's content is within the \field{max_size}, otherwise
    the request will fail and the driver MUST reset the device.
\item The driver SHOULD read \field{max_cipher_key_len} to discover the maximum length of cipher key
    the device supports and MUST guarantee that the \field{key_len} (CIPHER service or AEAD service) is within
    the \field{max_cipher_key_len} of the device configuration, otherwise the request will fail.
\item The driver SHOULD read \field{max_auth_key_len} to discover the maximum length of authenticated
    key the device supports and MUST guarantee that the \field{auth_key_len} (MAC service) is within the
    \field{max_auth_key_len} of the device configuration, otherwise the request will fail.
\end{itemize*}

\subsection{Device Initialization}\label{sec:Device Types / Crypto Device / Device Initialization}

\drivernormative{\subsubsection}{Device Initialization}{Device Types / Crypto Device / Device Initialization}

\begin{itemize*}
\item The driver MUST configure and initialize all virtqueues.
\item The driver MUST read the supported crypto services from bits of \field{crypto_services}.
\item The driver MUST read the supported algorithms based on \field{crypto_services} field.
\end{itemize*}

\subsection{Device Operation}\label{sec:Device Types / Crypto Device / Device Operation}

The operation of a virtio crypto device is driven by requests placed on the virtqueues.
Requests consist of a queue-type specific header (specifying among others the operation)
and an operation specific payload.

If VIRTIO_CRYPTO_F_REVISION_1 is negotiated the device may support both session mode
(See \ref{sec:Device Types / Crypto Device / Device Operation / Control Virtqueue / Session operation})
and stateless mode operation requests.
In stateless mode all operation parameters are supplied as a part of each request,
while in session mode, some or all operation parameters are managed within the
session. Stateless mode is guarded by feature bits 0-4 on a service level. If
stateless mode is negotiated for a service, the service accepts both session
mode and stateless requests; otherwise stateless mode requests are rejected
(via operation status).

\subsubsection{Operation Status}\label{sec:Device Types / Crypto Device / Device Operation / Operation status}
The device MUST return a status code as part of the operation (both session
operation and service operation) result. The valid operation status as follows:

\begin{lstlisting}
enum VIRTIO_CRYPTO_STATUS {
    VIRTIO_CRYPTO_OK = 0,
    VIRTIO_CRYPTO_ERR = 1,
    VIRTIO_CRYPTO_BADMSG = 2,
    VIRTIO_CRYPTO_NOTSUPP = 3,
    VIRTIO_CRYPTO_INVSESS = 4,
    VIRTIO_CRYPTO_NOSPC = 5,
    VIRTIO_CRYPTO_KEY_REJECTED = 6,
    VIRTIO_CRYPTO_MAX
};
\end{lstlisting}

\begin{itemize*}
\item VIRTIO_CRYPTO_OK: success.
\item VIRTIO_CRYPTO_BADMSG: authentication failed (only when AEAD decryption).
\item VIRTIO_CRYPTO_NOTSUPP: operation or algorithm is unsupported.
\item VIRTIO_CRYPTO_INVSESS: invalid session ID when executing crypto operations.
\item VIRTIO_CRYPTO_NOSPC: no free session ID (only when the VIRTIO_CRYPTO_F_REVISION_1
    feature bit is negotiated).
\item VIRTIO_CRYPTO_KEY_REJECTED: signature verification failed (only when AKCIPHER verification).
\item VIRTIO_CRYPTO_ERR: any failure not mentioned above occurs.
\end{itemize*}

\subsubsection{Control Virtqueue}\label{sec:Device Types / Crypto Device / Device Operation / Control Virtqueue}

The driver uses the control virtqueue to send control commands to the
device, such as session operations (See \ref{sec:Device Types / Crypto Device / Device
Operation / Control Virtqueue / Session operation}).

The header for controlq is of the following form:
\begin{lstlisting}
#define VIRTIO_CRYPTO_OPCODE(service, op)   (((service) << 8) | (op))

struct virtio_crypto_ctrl_header {
#define VIRTIO_CRYPTO_CIPHER_CREATE_SESSION \
       VIRTIO_CRYPTO_OPCODE(VIRTIO_CRYPTO_SERVICE_CIPHER, 0x02)
#define VIRTIO_CRYPTO_CIPHER_DESTROY_SESSION \
       VIRTIO_CRYPTO_OPCODE(VIRTIO_CRYPTO_SERVICE_CIPHER, 0x03)
#define VIRTIO_CRYPTO_HASH_CREATE_SESSION \
       VIRTIO_CRYPTO_OPCODE(VIRTIO_CRYPTO_SERVICE_HASH, 0x02)
#define VIRTIO_CRYPTO_HASH_DESTROY_SESSION \
       VIRTIO_CRYPTO_OPCODE(VIRTIO_CRYPTO_SERVICE_HASH, 0x03)
#define VIRTIO_CRYPTO_MAC_CREATE_SESSION \
       VIRTIO_CRYPTO_OPCODE(VIRTIO_CRYPTO_SERVICE_MAC, 0x02)
#define VIRTIO_CRYPTO_MAC_DESTROY_SESSION \
       VIRTIO_CRYPTO_OPCODE(VIRTIO_CRYPTO_SERVICE_MAC, 0x03)
#define VIRTIO_CRYPTO_AEAD_CREATE_SESSION \
       VIRTIO_CRYPTO_OPCODE(VIRTIO_CRYPTO_SERVICE_AEAD, 0x02)
#define VIRTIO_CRYPTO_AEAD_DESTROY_SESSION \
       VIRTIO_CRYPTO_OPCODE(VIRTIO_CRYPTO_SERVICE_AEAD, 0x03)
#define VIRTIO_CRYPTO_AKCIPHER_CREATE_SESSION \
       VIRTIO_CRYPTO_OPCODE(VIRTIO_CRYPTO_SERVICE_AKCIPHER, 0x04)
#define VIRTIO_CRYPTO_AKCIPHER_DESTROY_SESSION \
       VIRTIO_CRYPTO_OPCDE(VIRTIO_CRYPTO_SERVICE_AKCIPHER, 0x05)
    le32 opcode;
    /* algo should be service-specific algorithms */
    le32 algo;
    le32 flag;
    le32 reserved;
};
\end{lstlisting}

The controlq request is composed of four parts:
\begin{lstlisting}
struct virtio_crypto_op_ctrl_req {
    /* Device read only portion */

    struct virtio_crypto_ctrl_header header;

#define VIRTIO_CRYPTO_CTRLQ_OP_SPEC_HDR_LEGACY 56
    /* fixed length fields, opcode specific */
    u8 op_flf[flf_len];

    /* variable length fields, opcode specific */
    u8 op_vlf[vlf_len];

    /* Device write only portion */

    /* op result or completion status */
    u8 op_outcome[outcome_len];
};
\end{lstlisting}

\field{header} is a general header (see above).

\field{op_flf} is the opcode (in \field{header}) specific fixed-length parameters.

\field{flf_len} depends on the VIRTIO_CRYPTO_F_REVISION_1 feature bit (see below).

\field{op_vlf} is the opcode (in \field{header}) specific variable-length parameters.

\field{vlf_len} is the size of the specific structure used.
\begin{note}
The \field{vlf_len} of session-destroy operation and the hash-session-create
operation is ZERO.
\end{note}

\begin{itemize*}
\item If the opcode (in \field{header}) is VIRTIO_CRYPTO_CIPHER_CREATE_SESSION
    then \field{op_flf} is struct virtio_crypto_sym_create_session_flf if
    VIRTIO_CRYPTO_F_REVISION_1 is negotiated and struct virtio_crypto_sym_create_session_flf is
    padded to 56 bytes if NOT negotiated, and \field{op_vlf} is struct
    virtio_crypto_sym_create_session_vlf.
\item If the opcode (in \field{header}) is VIRTIO_CRYPTO_HASH_CREATE_SESSION
    then \field{op_flf} is struct virtio_crypto_hash_create_session_flf if
    VIRTIO_CRYPTO_F_REVISION_1 is negotiated and struct virtio_crypto_hash_create_session_flf is
    padded to 56 bytes if NOT negotiated.
\item If the opcode (in \field{header}) is VIRTIO_CRYPTO_MAC_CREATE_SESSION
    then \field{op_flf} is struct virtio_crypto_mac_create_session_flf if
    VIRTIO_CRYPTO_F_REVISION_1 is negotiated and struct virtio_crypto_mac_create_session_flf is
    padded to 56 bytes if NOT negotiated, and \field{op_vlf} is struct
    virtio_crypto_mac_create_session_vlf.
\item If the opcode (in \field{header}) is VIRTIO_CRYPTO_AEAD_CREATE_SESSION
    then \field{op_flf} is struct virtio_crypto_aead_create_session_flf if
    VIRTIO_CRYPTO_F_REVISION_1 is negotiated and struct virtio_crypto_aead_create_session_flf is
    padded to 56 bytes if NOT negotiated, and \field{op_vlf} is struct
    virtio_crypto_aead_create_session_vlf.
\item If the opcode (in \field{header}) is VIRTIO_CRYPTO_AKCIPHER_CREATE_SESSION
    then \field{op_flf} is struct virtio_crypto_akcipher_create_session_flf if
    VIRTIO_CRYPTO_F_REVISION_1 is negotiated and struct virtio_crypto_akcipher_create_session_flf is
    padded to 56 bytes if NOT negotiated, and \field{op_vlf} is struct
    virtio_crypto_akcipher_create_session_vlf.
\item If the opcode (in \field{header}) is VIRTIO_CRYPTO_CIPHER_DESTROY_SESSION
    or VIRTIO_CRYPTO_HASH_DESTROY_SESSION or VIRTIO_CRYPTO_MAC_DESTROY_SESSION or
    VIRTIO_CRYPTO_AEAD_DESTROY_SESSION then \field{op_flf} is struct
    virtio_crypto_destroy_session_flf if VIRTIO_CRYPTO_F_REVISION_1 is negotiated and
    struct virtio_crypto_destroy_session_flf is padded to 56 bytes if NOT negotiated.
\end{itemize*}

\field{op_outcome} stores the result of operation and must be struct
virtio_crypto_destroy_session_input for destroy session or
struct virtio_crypto_create_session_input for create session.

\field{outcome_len} is the size of the structure used.


\paragraph{Session operation}\label{sec:Device Types / Crypto Device / Device
Operation / Control Virtqueue / Session operation}

The session is a handle which describes the cryptographic parameters to be
applied to a number of buffers.

The following structure stores the result of session creation set by the device:

\begin{lstlisting}
struct virtio_crypto_create_session_input {
    le64 session_id;
    le32 status;
    le32 padding;
};
\end{lstlisting}

A request to destroy a session includes the following information:

\begin{lstlisting}
struct virtio_crypto_destroy_session_flf {
    /* Device read only portion */
    le64  session_id;
};

struct virtio_crypto_destroy_session_input {
    /* Device write only portion */
    u8  status;
};
\end{lstlisting}


\subparagraph{Session operation: HASH session}\label{sec:Device Types / Crypto Device / Device
Operation / Control Virtqueue / Session operation / Session operation: HASH session}

The fixed-length parameters of HASH session requests is as follows:

\begin{lstlisting}
struct virtio_crypto_hash_create_session_flf {
    /* Device read only portion */

    /* See VIRTIO_CRYPTO_HASH_* above */
    le32 algo;
    /* hash result length */
    le32 hash_result_len;
};
\end{lstlisting}


\subparagraph{Session operation: MAC session}\label{sec:Device Types / Crypto Device / Device
Operation / Control Virtqueue / Session operation / Session operation: MAC session}

The fixed-length and the variable-length parameters of MAC session requests are as follows:

\begin{lstlisting}
struct virtio_crypto_mac_create_session_flf {
    /* Device read only portion */

    /* See VIRTIO_CRYPTO_MAC_* above */
    le32 algo;
    /* hash result length */
    le32 hash_result_len;
    /* length of authenticated key */
    le32 auth_key_len;
    le32 padding;
};

struct virtio_crypto_mac_create_session_vlf {
    /* Device read only portion */

    /* The authenticated key */
    u8 auth_key[auth_key_len];
};
\end{lstlisting}

The length of \field{auth_key} is specified in \field{auth_key_len} in the struct
virtio_crypto_mac_create_session_flf.


\subparagraph{Session operation: Symmetric algorithms session}\label{sec:Device Types / Crypto Device / Device
Operation / Control Virtqueue / Session operation / Session operation: Symmetric algorithms session}

The request of symmetric session could be the CIPHER algorithms request
or the chain algorithms (chaining CIPHER and HASH/MAC) request.

The fixed-length and the variable-length parameters of CIPHER session requests are as follows:

\begin{lstlisting}
struct virtio_crypto_cipher_session_flf {
    /* Device read only portion */

    /* See VIRTIO_CRYPTO_CIPHER* above */
    le32 algo;
    /* length of key */
    le32 key_len;
#define VIRTIO_CRYPTO_OP_ENCRYPT  1
#define VIRTIO_CRYPTO_OP_DECRYPT  2
    /* encryption or decryption */
    le32 op;
    le32 padding;
};

struct virtio_crypto_cipher_session_vlf {
    /* Device read only portion */

    /* The cipher key */
    u8 cipher_key[key_len];
};
\end{lstlisting}

The length of \field{cipher_key} is specified in \field{key_len} in the struct
virtio_crypto_cipher_session_flf.

The fixed-length and the variable-length parameters of Chain session requests are as follows:

\begin{lstlisting}
struct virtio_crypto_alg_chain_session_flf {
    /* Device read only portion */

#define VIRTIO_CRYPTO_SYM_ALG_CHAIN_ORDER_HASH_THEN_CIPHER  1
#define VIRTIO_CRYPTO_SYM_ALG_CHAIN_ORDER_CIPHER_THEN_HASH  2
    le32 alg_chain_order;
/* Plain hash */
#define VIRTIO_CRYPTO_SYM_HASH_MODE_PLAIN    1
/* Authenticated hash (mac) */
#define VIRTIO_CRYPTO_SYM_HASH_MODE_AUTH     2
/* Nested hash */
#define VIRTIO_CRYPTO_SYM_HASH_MODE_NESTED   3
    le32 hash_mode;
    struct virtio_crypto_cipher_session_flf cipher_hdr;

#define VIRTIO_CRYPTO_ALG_CHAIN_SESS_OP_SPEC_HDR_SIZE  16
    /* fixed length fields, algo specific */
    u8 algo_flf[VIRTIO_CRYPTO_ALG_CHAIN_SESS_OP_SPEC_HDR_SIZE];

    /* length of the additional authenticated data (AAD) in bytes */
    le32 aad_len;
    le32 padding;
};

struct virtio_crypto_alg_chain_session_vlf {
    /* Device read only portion */

    /* The cipher key */
    u8 cipher_key[key_len];
    /* The authenticated key */
    u8 auth_key[auth_key_len];
};
\end{lstlisting}

\field{hash_mode} decides the type used by \field{algo_flf}.

\field{algo_flf} is fixed to 16 bytes and MUST contains or be one of
the following types:
\begin{itemize*}
\item struct virtio_crypto_hash_create_session_flf
\item struct virtio_crypto_mac_create_session_flf
\end{itemize*}
The data of unused part (if has) in \field{algo_flf} will be ignored.

The length of \field{cipher_key} is specified in \field{key_len} in \field{cipher_hdr}.

The length of \field{auth_key} is specified in \field{auth_key_len} in struct
virtio_crypto_mac_create_session_flf.

The fixed-length parameters of Symmetric session requests are as follows:

\begin{lstlisting}
struct virtio_crypto_sym_create_session_flf {
    /* Device read only portion */

#define VIRTIO_CRYPTO_SYM_SESS_OP_SPEC_HDR_SIZE  48
    /* fixed length fields, opcode specific */
    u8 op_flf[VIRTIO_CRYPTO_SYM_SESS_OP_SPEC_HDR_SIZE];

/* No operation */
#define VIRTIO_CRYPTO_SYM_OP_NONE  0
/* Cipher only operation on the data */
#define VIRTIO_CRYPTO_SYM_OP_CIPHER  1
/* Chain any cipher with any hash or mac operation. The order
   depends on the value of alg_chain_order param */
#define VIRTIO_CRYPTO_SYM_OP_ALGORITHM_CHAINING  2
    le32 op_type;
    le32 padding;
};
\end{lstlisting}

\field{op_flf} is fixed to 48 bytes, MUST contains or be one of
the following types:
\begin{itemize*}
\item struct virtio_crypto_cipher_session_flf
\item struct virtio_crypto_alg_chain_session_flf
\end{itemize*}
The data of unused part (if has) in \field{op_flf} will be ignored.

\field{op_type} decides the type used by \field{op_flf}.

The variable-length parameters of Symmetric session requests are as follows:

\begin{lstlisting}
struct virtio_crypto_sym_create_session_vlf {
    /* Device read only portion */
    /* variable length fields, opcode specific */
    u8 op_vlf[vlf_len];
};
\end{lstlisting}

\field{op_vlf} MUST contains or be one of the following types:
\begin{itemize*}
\item struct virtio_crypto_cipher_session_vlf
\item struct virtio_crypto_alg_chain_session_vlf
\end{itemize*}

\field{op_type} in struct virtio_crypto_sym_create_session_flf decides the
type used by \field{op_vlf}.

\field{vlf_len} is the size of the specific structure used.


\subparagraph{Session operation: AEAD session}\label{sec:Device Types / Crypto Device / Device
Operation / Control Virtqueue / Session operation / Session operation: AEAD session}

The fixed-length and the variable-length parameters of AEAD session requests are as follows:

\begin{lstlisting}
struct virtio_crypto_aead_create_session_flf {
    /* Device read only portion */

    /* See VIRTIO_CRYPTO_AEAD_* above */
    le32 algo;
    /* length of key */
    le32 key_len;
    /* Authentication tag length */
    le32 tag_len;
    /* The length of the additional authenticated data (AAD) in bytes */
    le32 aad_len;
    /* encryption or decryption, See above VIRTIO_CRYPTO_OP_* */
    le32 op;
    le32 padding;
};

struct virtio_crypto_aead_create_session_vlf {
    /* Device read only portion */
    u8 key[key_len];
};
\end{lstlisting}

The length of \field{key} is specified in \field{key_len} in struct
virtio_crypto_aead_create_session_flf.

\subparagraph{Session operation: AKCIPHER session}\label{sec:Device Types / Crypto Device / Device
Operation / Control Virtqueue / Session operation / Session operation: AKCIPHER session}

Due to the complexity of asymmetric key algorithms, different algorithms
require different parameters. The following data structures are used as
supplementary parameters to describe the asymmetric algorithm sessions.

For the RSA algorithm, the extra parameters are as follows:
\begin{lstlisting}
struct virtio_crypto_rsa_session_para {
#define VIRTIO_CRYPTO_RSA_RAW_PADDING   0
#define VIRTIO_CRYPTO_RSA_PKCS1_PADDING 1
    le32 padding_algo;

#define VIRTIO_CRYPTO_RSA_NO_HASH   0
#define VIRTIO_CRYPTO_RSA_MD2       1
#define VIRTIO_CRYPTO_RSA_MD3       2
#define VIRTIO_CRYPTO_RSA_MD4       3
#define VIRTIO_CRYPTO_RSA_MD5       4
#define VIRTIO_CRYPTO_RSA_SHA1      5
#define VIRTIO_CRYPTO_RSA_SHA256    6
#define VIRTIO_CRYPTO_RSA_SHA384    7
#define VIRTIO_CRYPTO_RSA_SHA512    8
#define VIRTIO_CRYPTO_RSA_SHA224    9
    le32 hash_algo;
};
\end{lstlisting}

\field{padding_algo} specifies the padding method used by RSA sessions.
\begin{itemize*}
\item If VIRTIO_CRYPTO_RSA_RAW_PADDING is specified, 1) \field{hash_algo}
is ignored, 2) ciphertext and plaintext MUST be padded with leading zeros,
3) and RSA sessions with VIRTIO_CRYPTO_RSA_RAW_PADDING MUST not be used
for verification and signing operations.
\item If VIRTIO_CRYPTO_RSA_PKCS1_PADDING is specified, EMSA-PKCS1-v1_5 padding method
is used (see \hyperref[intro:rfc3447]{PKCS\#1}), \field{hash_algo} specifies how the
digest of the data passed to RSA sessions is calculated when verifying and signing.
It only affects the padding algorithm and is ignored during encryption and decryption.
\end{itemize*}

The ECC algorithms such as the ECDSA algorithm, cannot use custom curves, only the
following known curves can be used (see \hyperref[intro:NIST]{NIST-recommended curves}).

\begin{lstlisting}
#define VIRTIO_CRYPTO_CURVE_UNKNOWN   0
#define VIRTIO_CRYPTO_CURVE_NIST_P192 1
#define VIRTIO_CRYPTO_CURVE_NIST_P224 2
#define VIRTIO_CRYPTO_CURVE_NIST_P256 3
#define VIRTIO_CRYPTO_CURVE_NIST_P384 4
#define VIRTIO_CRYPTO_CURVE_NIST_P521 5
\end{lstlisting}

For the ECDSA algorithm, the extra parameters are as follows:
\begin{lstlisting}
struct virtio_crypto_ecdsa_session_para {
    /* See VIRTIO_CRYPTO_CURVE_* above */
    le32 curve_id;
};
\end{lstlisting}

The fixed-length and the variable-length parameters of AKCIPHER session requests are as follows:
\begin{lstlisting}
struct virtio_crypto_akcipher_create_session_flf {
    /* Device read only portion */

    /* See VIRTIO_CRYPTO_AKCIPHER_* above */
    le32 algo;
#define VIRTIO_CRYPTO_AKCIPHER_KEY_TYPE_PUBLIC 1
#define VIRTIO_CRYPTO_AKCIPHER_KEY_TYPE_PRIVATE 2
    le32 key_type;
    /* length of key */
    le32 key_len;

#define VIRTIO_CRYPTO_AKCIPHER_SESS_ALGO_SPEC_HDR_SIZE 44
    u8 algo_flf[VIRTIO_CRYPTO_AKCIPHER_SESS_ALGO_SPEC_HDR_SIZE];
};

struct virtio_crypto_akcipher_create_session_vlf {
    /* Device read only portion */
    u8 key[key_len];
};
\end{lstlisting}

\field{algo} decides the type used by \field{algo_flf}.
\field{algo_flf} is fixed to 44 bytes and MUST contains of be one the
following structures:
\begin{itemize*}
\item struct virtio_crypto_rsa_session_para
\item struct virtio_crypto_ecdsa_session_para
\end{itemize*}

The length of \field{key} is specified in \field{key_len} in the struct
virtio_crypto_akcipher_create_session_flf.

For the RSA algorithm, the key needs to be encoded according to
\hyperref[intro:rfc3447]{PKCS\#1}. The private key is described with the
RSAPrivateKey structure, and the public key is described with the RSAPublicKey
structure. These ASN.1 structures are encoded in DER encoding rules (see
\hyperref[intro:rfc6025]{rfc6025}).

\begin{lstlisting}
RSAPrivateKey ::= SEQUENCE {
    version          INTEGER,
    modulus          INTEGER,
    publicExponent   INTEGER,
    privateExponent  INTEGER,
    prime1           INTEGER,
    prime2           INTEGER,
    exponent1        INTEGER,
    exponent1        INTEGER,
    coefficient      INTEGER,
    otherPrimeInfos  OtherPrimeInfos OPTIONAL
}

OtherPrimeInfos ::= SEQUENCE SIZE(1...MAX) OF OtherPrimeInfo

OtherPrimeINfo ::= SEQUENCE {
    prime           INTEGER,
    exponent        INTEGER,
    coefficient     INTEGER
}

RSAPublicKey ::= SEQUENCE {
    modulus         INTEGER,
    publicExponent  INTEGER
}
\end{lstlisting}

For the ECDSA algorithm, the private key is encoded according to
\hyperref[intro:rfc5915]{RFC5915}, the private key of the ECDSA algorithm
is described by the ASN.1 structure ECPrivateKey and encoded with DER
encoding rules (see \hyperref[intro:rfc6025]{rfc6025}).

\begin{lstlisting}
ECPrivateKey ::= SEQUNCE {
    version         INTEGER,
    privateKey      OCTET STRING,
    parameters [0]  ECParameters {{ NamedCurve }} OPTIONAL,
    publicKey  [1]  BIT STRING OPTIONAL
}
\end{lstlisting}

The public key of the ECDSA algorithm is encoded according to \hyperref[intro:SEC1]{SEC1},
and the public key of ECDSA is described by the ASN.1 structure ECPoint.
When initializing a session with ECDSA public key, the ECPoint is DER encoded and the
\field{key} only contains the value part of ECPoint, that is, the header part of the
OCTET STRING will be omitted (see \hyperref[intro:rfc6025]{rfc6025}).

\begin{lstlisting}
ECPoint ::= OCTET STRING
\end{lstlisting}

The length of \field{key} is specified in \field{key_len} in
struct virtio_crypto_akcipher_create_session_flf.

\drivernormative{\subparagraph}{Session operation: create session}{Device Types / Crypto Device / Device
Operation / Control Virtqueue / Session operation / Session operation: create session}

\begin{itemize*}
\item The driver MUST set the \field{opcode} field based on service type: CIPHER, HASH, MAC, AEAD or AKCIPHER.
\item The driver MUST set the control general header, the opcode specific header,
    the opcode specific extra parameters and the opcode specific outcome buffer in turn.
    See \ref{sec:Device Types / Crypto Device / Device Operation / Control Virtqueue}.
\item The driver MUST set the \field{reversed} field to zero.
\end{itemize*}

\devicenormative{\subparagraph}{Session operation: create session}{Device Types / Crypto Device / Device
Operation / Control Virtqueue / Session operation / Session operation: create session}

\begin{itemize*}
\item The device MUST use the corresponding opcode specific structure according to the
    \field{opcode} in the control general header.
\item The device MUST extract extra parameters according to the structures used.
\item The device MUST set the \field{status} field to one of the following values of enum
    VIRTIO_CRYPTO_STATUS after finish a session creation:
\begin{itemize*}
\item VIRTIO_CRYPTO_OK if a session is created successfully.
\item VIRTIO_CRYPTO_NOTSUPP if the requested algorithm or operation is unsupported.
\item VIRTIO_CRYPTO_NOSPC if no free session ID (only when the VIRTIO_CRYPTO_F_REVISION_1
    feature bit is negotiated).
\item VIRTIO_CRYPTO_ERR if failure not mentioned above occurs.
\end{itemize*}
\item The device MUST set the \field{session_id} field to a unique session identifier only
    if the status is set to VIRTIO_CRYPTO_OK.
\end{itemize*}

\drivernormative{\subparagraph}{Session operation: destroy session}{Device Types / Crypto Device / Device
Operation / Control Virtqueue / Session operation / Session operation: destroy session}

\begin{itemize*}
\item The driver MUST set the \field{opcode} field based on service type: CIPHER, HASH, MAC, AEAD or AKCIPHER.
\item The driver MUST set the \field{session_id} to a valid value assigned by the device
    when the session was created.
\end{itemize*}

\devicenormative{\subparagraph}{Session operation: destroy session}{Device Types / Crypto Device / Device
Operation / Control Virtqueue / Session operation / Session operation: destroy session}

\begin{itemize*}
\item The device MUST set the \field{status} field to one of the following values of enum VIRTIO_CRYPTO_STATUS.
\begin{itemize*}
\item VIRTIO_CRYPTO_OK if a session is created successfully.
\item VIRTIO_CRYPTO_ERR if any failure occurs.
\end{itemize*}
\end{itemize*}


\subsubsection{Data Virtqueue}\label{sec:Device Types / Crypto Device / Device Operation / Data Virtqueue}

The driver uses the data virtqueues to transmit crypto operation requests to the device,
and completes the crypto operations.

The header for dataq is as follows:

\begin{lstlisting}
struct virtio_crypto_op_header {
#define VIRTIO_CRYPTO_CIPHER_ENCRYPT \
    VIRTIO_CRYPTO_OPCODE(VIRTIO_CRYPTO_SERVICE_CIPHER, 0x00)
#define VIRTIO_CRYPTO_CIPHER_DECRYPT \
    VIRTIO_CRYPTO_OPCODE(VIRTIO_CRYPTO_SERVICE_CIPHER, 0x01)
#define VIRTIO_CRYPTO_HASH \
    VIRTIO_CRYPTO_OPCODE(VIRTIO_CRYPTO_SERVICE_HASH, 0x00)
#define VIRTIO_CRYPTO_MAC \
    VIRTIO_CRYPTO_OPCODE(VIRTIO_CRYPTO_SERVICE_MAC, 0x00)
#define VIRTIO_CRYPTO_AEAD_ENCRYPT \
    VIRTIO_CRYPTO_OPCODE(VIRTIO_CRYPTO_SERVICE_AEAD, 0x00)
#define VIRTIO_CRYPTO_AEAD_DECRYPT \
    VIRTIO_CRYPTO_OPCODE(VIRTIO_CRYPTO_SERVICE_AEAD, 0x01)
#define VIRTIO_CRYPTO_AKCIPHER_ENCRYPT \
    VIRTIO_CRYPTO_OPCODE(VIRTIO_CRYPTO_SERVICE_AKCIPHER, 0x00)
#define VIRTIO_CRYPTO_AKCIPHER_DECRYPT \
    VIRTIO_CRYPTO_OPCODE(VIRTIO_CRYPTO_SERVICE_AKCIPHER, 0x01)
#define VIRTIO_CRYPTO_AKCIPHER_SIGN \
    VIRTIO_CRYPTO_OPCODE(VIRTIO_CRYPTO_SERVICE_AKCIPHER, 0x02)
#define VIRTIO_CRYPTO_AKCIPHER_VERIFY \
    VIRTIO_CRYPTO_OPCODE(VIRTIO_CRYPTO_SERVICE_AKCIPHER, 0x03)
    le32 opcode;
    /* algo should be service-specific algorithms */
    le32 algo;
    le64 session_id;
#define VIRTIO_CRYPTO_FLAG_SESSION_MODE 1
    /* control flag to control the request */
    le32 flag;
    le32 padding;
};
\end{lstlisting}

\begin{note}
If VIRTIO_CRYPTO_F_REVISION_1 is not negotiated the \field{flag} is ignored.

If VIRTIO_CRYPTO_F_REVISION_1 is negotiated but VIRTIO_CRYPTO_F_<SERVICE>_STATELESS_MODE
is not negotiated, then the device SHOULD reject <SERVICE> requests if
VIRTIO_CRYPTO_FLAG_SESSION_MODE is not set (in \field{flag}).
\end{note}

The dataq request is composed of four parts:
\begin{lstlisting}
struct virtio_crypto_op_data_req {
    /* Device read only portion */

    struct virtio_crypto_op_header header;

#define VIRTIO_CRYPTO_DATAQ_OP_SPEC_HDR_LEGACY 48
    /* fixed length fields, opcode specific */
    u8 op_flf[flf_len];

    /* Device read && write portion */
    /* variable length fields, opcode specific */
    u8 op_vlf[vlf_len];

    /* Device write only portion */
    struct virtio_crypto_inhdr inhdr;
};
\end{lstlisting}

\field{header} is a general header (see above).

\field{op_flf} is the opcode (in \field{header}) specific header.

\field{flf_len} depends on the VIRTIO_CRYPTO_F_REVISION_1 feature bit
(see below).

\field{op_vlf} is the opcode (in \field{header}) specific parameters.

\field{vlf_len} is the size of the specific structure used.

\begin{itemize*}
\item If the the opcode (in \field{header}) is VIRTIO_CRYPTO_CIPHER_ENCRYPT
    or VIRTIO_CRYPTO_CIPHER_DECRYPT then:
    \begin{itemize*}
    \item If VIRTIO_CRYPTO_F_CIPHER_STATELESS_MODE is negotiated, \field{op_flf} is
        struct virtio_crypto_sym_data_flf_stateless, and \field{op_vlf} is struct
        virtio_crypto_sym_data_vlf_stateless.
    \item If VIRTIO_CRYPTO_F_CIPHER_STATELESS_MODE is NOT negotiated, \field{op_flf}
        is struct virtio_crypto_sym_data_flf if VIRTIO_CRYPTO_F_REVISION_1 is negotiated
        and struct virtio_crypto_sym_data_flf is padded to 48 bytes if NOT negotiated,
        and \field{op_vlf} is struct virtio_crypto_sym_data_vlf.
    \end{itemize*}
\item If the the opcode (in \field{header}) is VIRTIO_CRYPTO_HASH:
    \begin{itemize*}
    \item If VIRTIO_CRYPTO_F_HASH_STATELESS_MODE is negotiated, \field{op_flf} is
        struct virtio_crypto_hash_data_flf_stateless, and \field{op_vlf} is struct
        virtio_crypto_hash_data_vlf_stateless.
    \item If VIRTIO_CRYPTO_F_HASH_STATELESS_MODE is NOT negotiated, \field{op_flf}
        is struct virtio_crypto_hash_data_flf if VIRTIO_CRYPTO_F_REVISION_1 is negotiated
        and struct virtio_crypto_hash_data_flf is padded to 48 bytes if NOT negotiated,
        and \field{op_vlf} is struct virtio_crypto_hash_data_vlf.
    \end{itemize*}
\item If the the opcode (in \field{header}) is VIRTIO_CRYPTO_MAC:
    \begin{itemize*}
    \item If VIRTIO_CRYPTO_F_MAC_STATELESS_MODE is negotiated, \field{op_flf} is
        struct virtio_crypto_mac_data_flf_stateless, and \field{op_vlf} is struct
        virtio_crypto_mac_data_vlf_stateless.
    \item If VIRTIO_CRYPTO_F_MAC_STATELESS_MODE is NOT negotiated, \field{op_flf}
        is struct virtio_crypto_mac_data_flf if VIRTIO_CRYPTO_F_REVISION_1 is negotiated
        and struct virtio_crypto_mac_data_flf is padded to 48 bytes if NOT negotiated,
        and \field{op_vlf} is struct virtio_crypto_mac_data_vlf.
    \end{itemize*}
\item If the the opcode (in \field{header}) is VIRTIO_CRYPTO_AEAD_ENCRYPT
    or VIRTIO_CRYPTO_AEAD_DECRYPT then:
    \begin{itemize*}
    \item If VIRTIO_CRYPTO_F_AEAD_STATELESS_MODE is negotiated, \field{op_flf} is
        struct virtio_crypto_aead_data_flf_stateless, and \field{op_vlf} is struct
        virtio_crypto_aead_data_vlf_stateless.
    \item If VIRTIO_CRYPTO_F_AEAD_STATELESS_MODE is NOT negotiated, \field{op_flf}
        is struct virtio_crypto_aead_data_flf if VIRTIO_CRYPTO_F_REVISION_1 is negotiated
        and struct virtio_crypto_aead_data_flf is padded to 48 bytes if NOT negotiated,
        and \field{op_vlf} is struct virtio_crypto_aead_data_vlf.
    \end{itemize*}
\item If the opcode (in \field{header}) is VIRTIO_CRYPTO_AKCIPHER_ENCRYPT, VIRTIO_CRYPTO_AKCIPHER_DECRYPT,
    VIRTIO_CRYPTO_AKCIPHER_SIGN or VIRTIO_CRYPTO_AKCIPHER_VERIFY then:
    \begin{itemize*}
    \item If VIRTIO_CRYPTO_F_AKCIPHER_STATELESS_MODE is negotiated, \field{op_flf} is
        struct virtio_crypto_akcipher_data_flf_statless, and \field{op_vlf} is struct
        virtio_crypto_akcipher_data_vlf_stateless.
    \item If VIRTIO_CRYPTO_F_AKCIPHER_STATELESS_MODE is NOT negotiated, \field{op_flf}
        is struct virtio_crypto_akcipher_data_flf if VIRTIO_CRYPTO_F_REVISION_1 is negotiated
        and struct virtio_crypto_akcipher_data_flf is padded to 48 bytes if NOT negotiated,
        and \field{op_vlf} is struct virtio_crypto_akcipher_data_vlf.
    \end{itemize*}
\end{itemize*}

\field{inhdr} is a unified input header that used to return the status of
the operations, is defined as follows:

\begin{lstlisting}
struct virtio_crypto_inhdr {
    u8 status;
};
\end{lstlisting}

\subsubsection{HASH Service Operation}\label{sec:Device Types / Crypto Device / Device Operation / HASH Service Operation}

Session mode HASH service requests are as follows:

\begin{lstlisting}
struct virtio_crypto_hash_data_flf {
    /* length of source data */
    le32 src_data_len;
    /* hash result length */
    le32 hash_result_len;
};

struct virtio_crypto_hash_data_vlf {
    /* Device read only portion */
    /* Source data */
    u8 src_data[src_data_len];

    /* Device write only portion */
    /* Hash result data */
    u8 hash_result[hash_result_len];
};
\end{lstlisting}

Each data request uses the virtio_crypto_hash_data_flf structure and the
virtio_crypto_hash_data_vlf structure to store information used to run the
HASH operations.

\field{src_data} is the source data that will be processed.
\field{src_data_len} is the length of source data.
\field{hash_result} is the result data and \field{hash_result_len} is the length
of it.

Stateless mode HASH service requests are as follows:

\begin{lstlisting}
struct virtio_crypto_hash_data_flf_stateless {
    struct {
        /* See VIRTIO_CRYPTO_HASH_* above */
        le32 algo;
    } sess_para;

    /* length of source data */
    le32 src_data_len;
    /* hash result length */
    le32 hash_result_len;
    le32 reserved;
};
struct virtio_crypto_hash_data_vlf_stateless {
    /* Device read only portion */
    /* Source data */
    u8 src_data[src_data_len];

    /* Device write only portion */
    /* Hash result data */
    u8 hash_result[hash_result_len];
};
\end{lstlisting}

\drivernormative{\paragraph}{HASH Service Operation}{Device Types / Crypto Device / Device Operation / HASH Service Operation}

\begin{itemize*}
\item If the driver uses the session mode, then the driver MUST set \field{session_id}
    in struct virtio_crypto_op_header to a valid value assigned by the device when the
    session was created.
\item If the VIRTIO_CRYPTO_F_HASH_STATELESS_MODE feature bit is negotiated, 1) if the
    driver uses the stateless mode, then the driver MUST set the \field{flag} field in
    struct virtio_crypto_op_header to ZERO and MUST set the fields in struct
    virtio_crypto_hash_data_flf_stateless.sess_para, 2) if the driver uses the session
    mode, then the driver MUST set the \field{flag} field in struct virtio_crypto_op_header
    to VIRTIO_CRYPTO_FLAG_SESSION_MODE.
\item The driver MUST set \field{opcode} in struct virtio_crypto_op_header to VIRTIO_CRYPTO_HASH.
\end{itemize*}

\devicenormative{\paragraph}{HASH Service Operation}{Device Types / Crypto Device / Device Operation / HASH Service Operation}

\begin{itemize*}
\item The device MUST use the corresponding structure according to the \field{opcode}
    in the data general header.
\item If the VIRTIO_CRYPTO_F_HASH_STATELESS_MODE feature bit is negotiated, the device
    MUST parse \field{flag} field in struct virtio_crypto_op_header in order to decide
    which mode the driver uses.
\item The device MUST copy the results of HASH operations in the hash_result[] if HASH
    operations success.
\item The device MUST set \field{status} in struct virtio_crypto_inhdr to one of the
    following values of enum VIRTIO_CRYPTO_STATUS:
\begin{itemize*}
\item VIRTIO_CRYPTO_OK if the operation success.
\item VIRTIO_CRYPTO_NOTSUPP if the requested algorithm or operation is unsupported.
\item VIRTIO_CRYPTO_INVSESS if the session ID invalid when in session mode.
\item VIRTIO_CRYPTO_ERR if any failure not mentioned above occurs.
\end{itemize*}
\end{itemize*}


\subsubsection{MAC Service Operation}\label{sec:Device Types / Crypto Device / Device Operation / MAC Service Operation}

Session mode MAC service requests are as follows:

\begin{lstlisting}
struct virtio_crypto_mac_data_flf {
    struct virtio_crypto_hash_data_flf hdr;
};

struct virtio_crypto_mac_data_vlf {
    /* Device read only portion */
    /* Source data */
    u8 src_data[src_data_len];

    /* Device write only portion */
    /* Hash result data */
    u8 hash_result[hash_result_len];
};
\end{lstlisting}

Each request uses the virtio_crypto_mac_data_flf structure and the
virtio_crypto_mac_data_vlf structure to store information used to run the
MAC operations.

\field{src_data} is the source data that will be processed.
\field{src_data_len} is the length of source data.
\field{hash_result} is the result data and \field{hash_result_len} is the length
of it.

Stateless mode MAC service requests are as follows:

\begin{lstlisting}
struct virtio_crypto_mac_data_flf_stateless {
    struct {
        /* See VIRTIO_CRYPTO_MAC_* above */
        le32 algo;
        /* length of authenticated key */
        le32 auth_key_len;
    } sess_para;

    /* length of source data */
    le32 src_data_len;
    /* hash result length */
    le32 hash_result_len;
};

struct virtio_crypto_mac_data_vlf_stateless {
    /* Device read only portion */
    /* The authenticated key */
    u8 auth_key[auth_key_len];
    /* Source data */
    u8 src_data[src_data_len];

    /* Device write only portion */
    /* Hash result data */
    u8 hash_result[hash_result_len];
};
\end{lstlisting}

\field{auth_key} is the authenticated key that will be used during the process.
\field{auth_key_len} is the length of the key.

\drivernormative{\paragraph}{MAC Service Operation}{Device Types / Crypto Device / Device Operation / MAC Service Operation}

\begin{itemize*}
\item If the driver uses the session mode, then the driver MUST set \field{session_id}
    in struct virtio_crypto_op_header to a valid value assigned by the device when the
    session was created.
\item If the VIRTIO_CRYPTO_F_MAC_STATELESS_MODE feature bit is negotiated, 1) if the
    driver uses the stateless mode, then the driver MUST set the \field{flag} field
    in struct virtio_crypto_op_header to ZERO and MUST set the fields in struct
    virtio_crypto_mac_data_flf_stateless.sess_para, 2) if the driver uses the session
    mode, then the driver MUST set the \field{flag} field in struct virtio_crypto_op_header
    to VIRTIO_CRYPTO_FLAG_SESSION_MODE.
\item The driver MUST set \field{opcode} in struct virtio_crypto_op_header to VIRTIO_CRYPTO_MAC.
\end{itemize*}

\devicenormative{\paragraph}{MAC Service Operation}{Device Types / Crypto Device / Device Operation / MAC Service Operation}

\begin{itemize*}
\item If the VIRTIO_CRYPTO_F_MAC_STATELESS_MODE feature bit is negotiated, the device
    MUST parse \field{flag} field in struct virtio_crypto_op_header in order to decide
	which mode the driver uses.
\item The device MUST copy the results of MAC operations in the hash_result[] if HASH
    operations success.
\item The device MUST set \field{status} in struct virtio_crypto_inhdr to one of the
    following values of enum VIRTIO_CRYPTO_STATUS:
\begin{itemize*}
\item VIRTIO_CRYPTO_OK if the operation success.
\item VIRTIO_CRYPTO_NOTSUPP if the requested algorithm or operation is unsupported.
\item VIRTIO_CRYPTO_INVSESS if the session ID invalid when in session mode.
\item VIRTIO_CRYPTO_ERR if any failure not mentioned above occurs.
\end{itemize*}
\end{itemize*}

\subsubsection{Symmetric algorithms Operation}\label{sec:Device Types / Crypto Device / Device Operation / Symmetric algorithms Operation}

Session mode CIPHER service requests are as follows:

\begin{lstlisting}
struct virtio_crypto_cipher_data_flf {
    /*
     * Byte Length of valid IV/Counter data pointed to by the below iv data.
     *
     * For block ciphers in CBC or F8 mode, or for Kasumi in F8 mode, or for
     *   SNOW3G in UEA2 mode, this is the length of the IV (which
     *   must be the same as the block length of the cipher).
     * For block ciphers in CTR mode, this is the length of the counter
     *   (which must be the same as the block length of the cipher).
     */
    le32 iv_len;
    /* length of source data */
    le32 src_data_len;
    /* length of destination data */
    le32 dst_data_len;
    le32 padding;
};

struct virtio_crypto_cipher_data_vlf {
    /* Device read only portion */

    /*
     * Initialization Vector or Counter data.
     *
     * For block ciphers in CBC or F8 mode, or for Kasumi in F8 mode, or for
     *   SNOW3G in UEA2 mode, this is the Initialization Vector (IV)
     *   value.
     * For block ciphers in CTR mode, this is the counter.
     * For AES-XTS, this is the 128bit tweak, i, from IEEE Std 1619-2007.
     *
     * The IV/Counter will be updated after every partial cryptographic
     * operation.
     */
    u8 iv[iv_len];
    /* Source data */
    u8 src_data[src_data_len];

    /* Device write only portion */
    /* Destination data */
    u8 dst_data[dst_data_len];
};
\end{lstlisting}

Session mode requests of algorithm chaining are as follows:

\begin{lstlisting}
struct virtio_crypto_alg_chain_data_flf {
    le32 iv_len;
    /* Length of source data */
    le32 src_data_len;
    /* Length of destination data */
    le32 dst_data_len;
    /* Starting point for cipher processing in source data */
    le32 cipher_start_src_offset;
    /* Length of the source data that the cipher will be computed on */
    le32 len_to_cipher;
    /* Starting point for hash processing in source data */
    le32 hash_start_src_offset;
    /* Length of the source data that the hash will be computed on */
    le32 len_to_hash;
    /* Length of the additional auth data */
    le32 aad_len;
    /* Length of the hash result */
    le32 hash_result_len;
    le32 reserved;
};

struct virtio_crypto_alg_chain_data_vlf {
    /* Device read only portion */

    /* Initialization Vector or Counter data */
    u8 iv[iv_len];
    /* Source data */
    u8 src_data[src_data_len];
    /* Additional authenticated data if exists */
    u8 aad[aad_len];

    /* Device write only portion */

    /* Destination data */
    u8 dst_data[dst_data_len];
    /* Hash result data */
    u8 hash_result[hash_result_len];
};
\end{lstlisting}

Session mode requests of symmetric algorithm are as follows:

\begin{lstlisting}
struct virtio_crypto_sym_data_flf {
    /* Device read only portion */

#define VIRTIO_CRYPTO_SYM_DATA_REQ_HDR_SIZE    40
    u8 op_type_flf[VIRTIO_CRYPTO_SYM_DATA_REQ_HDR_SIZE];

    /* See above VIRTIO_CRYPTO_SYM_OP_* */
    le32 op_type;
    le32 padding;
};

struct virtio_crypto_sym_data_vlf {
    u8 op_type_vlf[sym_para_len];
};
\end{lstlisting}

Each request uses the virtio_crypto_sym_data_flf structure and the
virtio_crypto_sym_data_flf structure to store information used to run the
CIPHER operations.

\field{op_type_flf} is the \field{op_type} specific header, it MUST starts
with or be one of the following structures:
\begin{itemize*}
\item struct virtio_crypto_cipher_data_flf
\item struct virtio_crypto_alg_chain_data_flf
\end{itemize*}

The length of \field{op_type_flf} is fixed to 40 bytes, the data of unused
part (if has) will be ignored.

\field{op_type_vlf} is the \field{op_type} specific parameters, it MUST starts
with or be one of the following structures:
\begin{itemize*}
\item struct virtio_crypto_cipher_data_vlf
\item struct virtio_crypto_alg_chain_data_vlf
\end{itemize*}

\field{sym_para_len} is the size of the specific structure used.

Stateless mode CIPHER service requests are as follows:

\begin{lstlisting}
struct virtio_crypto_cipher_data_flf_stateless {
    struct {
        /* See VIRTIO_CRYPTO_CIPHER* above */
        le32 algo;
        /* length of key */
        le32 key_len;

        /* See VIRTIO_CRYPTO_OP_* above */
        le32 op;
    } sess_para;

    /*
     * Byte Length of valid IV/Counter data pointed to by the below iv data.
     */
    le32 iv_len;
    /* length of source data */
    le32 src_data_len;
    /* length of destination data */
    le32 dst_data_len;
};

struct virtio_crypto_cipher_data_vlf_stateless {
    /* Device read only portion */

    /* The cipher key */
    u8 cipher_key[key_len];

    /* Initialization Vector or Counter data. */
    u8 iv[iv_len];
    /* Source data */
    u8 src_data[src_data_len];

    /* Device write only portion */
    /* Destination data */
    u8 dst_data[dst_data_len];
};
\end{lstlisting}

Stateless mode requests of algorithm chaining are as follows:

\begin{lstlisting}
struct virtio_crypto_alg_chain_data_flf_stateless {
    struct {
        /* See VIRTIO_CRYPTO_SYM_ALG_CHAIN_ORDER_* above */
        le32 alg_chain_order;
        /* length of the additional authenticated data in bytes */
        le32 aad_len;

        struct {
            /* See VIRTIO_CRYPTO_CIPHER* above */
            le32 algo;
            /* length of key */
            le32 key_len;
            /* See VIRTIO_CRYPTO_OP_* above */
            le32 op;
        } cipher;

        struct {
            /* See VIRTIO_CRYPTO_HASH_* or VIRTIO_CRYPTO_MAC_* above */
            le32 algo;
            /* length of authenticated key */
            le32 auth_key_len;
            /* See VIRTIO_CRYPTO_SYM_HASH_MODE_* above */
            le32 hash_mode;
        } hash;
    } sess_para;

    le32 iv_len;
    /* Length of source data */
    le32 src_data_len;
    /* Length of destination data */
    le32 dst_data_len;
    /* Starting point for cipher processing in source data */
    le32 cipher_start_src_offset;
    /* Length of the source data that the cipher will be computed on */
    le32 len_to_cipher;
    /* Starting point for hash processing in source data */
    le32 hash_start_src_offset;
    /* Length of the source data that the hash will be computed on */
    le32 len_to_hash;
    /* Length of the additional auth data */
    le32 aad_len;
    /* Length of the hash result */
    le32 hash_result_len;
    le32 reserved;
};

struct virtio_crypto_alg_chain_data_vlf_stateless {
    /* Device read only portion */

    /* The cipher key */
    u8 cipher_key[key_len];
    /* The auth key */
    u8 auth_key[auth_key_len];
    /* Initialization Vector or Counter data */
    u8 iv[iv_len];
    /* Additional authenticated data if exists */
    u8 aad[aad_len];
    /* Source data */
    u8 src_data[src_data_len];

    /* Device write only portion */

    /* Destination data */
    u8 dst_data[dst_data_len];
    /* Hash result data */
    u8 hash_result[hash_result_len];
};
\end{lstlisting}

Stateless mode requests of symmetric algorithm are as follows:

\begin{lstlisting}
struct virtio_crypto_sym_data_flf_stateless {
    /* Device read only portion */
#define VIRTIO_CRYPTO_SYM_DATE_REQ_HDR_STATELESS_SIZE    72
    u8 op_type_flf[VIRTIO_CRYPTO_SYM_DATE_REQ_HDR_STATELESS_SIZE];

    /* Device write only portion */
    /* See above VIRTIO_CRYPTO_SYM_OP_* */
    le32 op_type;
};

struct virtio_crypto_sym_data_vlf_stateless {
    u8 op_type_vlf[sym_para_len];
};
\end{lstlisting}

\field{op_type_flf} is the \field{op_type} specific header, it MUST starts
with or be one of the following structures:
\begin{itemize*}
\item struct virtio_crypto_cipher_data_flf_stateless
\item struct virtio_crypto_alg_chain_data_flf_stateless
\end{itemize*}

The length of \field{op_type_flf} is fixed to 72 bytes, the data of unused
part (if has) will be ignored.

\field{op_type_vlf} is the \field{op_type} specific parameters, it MUST starts
with or be one of the following structures:
\begin{itemize*}
\item struct virtio_crypto_cipher_data_vlf_stateless
\item struct virtio_crypto_alg_chain_data_vlf_stateless
\end{itemize*}

\field{sym_para_len} is the size of the specific structure used.

\drivernormative{\paragraph}{Symmetric algorithms Operation}{Device Types / Crypto Device / Device Operation / Symmetric algorithms Operation}

\begin{itemize*}
\item If the driver uses the session mode, then the driver MUST set \field{session_id}
    in struct virtio_crypto_op_header to a valid value assigned by the device when the
    session was created.
\item If the VIRTIO_CRYPTO_F_CIPHER_STATELESS_MODE feature bit is negotiated, 1) if the
    driver uses the stateless mode, then the driver MUST set the \field{flag} field in
    struct virtio_crypto_op_header to ZERO and MUST set the fields in struct
    virtio_crypto_cipher_data_flf_stateless.sess_para or struct
    virtio_crypto_alg_chain_data_flf_stateless.sess_para, 2) if the driver uses the
    session mode, then the driver MUST set the \field{flag} field in struct
    virtio_crypto_op_header to VIRTIO_CRYPTO_FLAG_SESSION_MODE.
\item The driver MUST set the \field{opcode} field in struct virtio_crypto_op_header
    to VIRTIO_CRYPTO_CIPHER_ENCRYPT or VIRTIO_CRYPTO_CIPHER_DECRYPT.
\item The driver MUST specify the fields of struct virtio_crypto_cipher_data_flf in
    struct virtio_crypto_sym_data_flf and struct virtio_crypto_cipher_data_vlf in
    struct virtio_crypto_sym_data_vlf if the request is based on VIRTIO_CRYPTO_SYM_OP_CIPHER.
\item The driver MUST specify the fields of struct virtio_crypto_alg_chain_data_flf
    in struct virtio_crypto_sym_data_flf and struct virtio_crypto_alg_chain_data_vlf
    in struct virtio_crypto_sym_data_vlf if the request is of the VIRTIO_CRYPTO_SYM_OP_ALGORITHM_CHAINING
    type.
\end{itemize*}

\devicenormative{\paragraph}{Symmetric algorithms Operation}{Device Types / Crypto Device / Device Operation / Symmetric algorithms Operation}

\begin{itemize*}
\item If the VIRTIO_CRYPTO_F_CIPHER_STATELESS_MODE feature bit is negotiated, the device
    MUST parse \field{flag} field in struct virtio_crypto_op_header in order to decide
	which mode the driver uses.
\item The device MUST parse the virtio_crypto_sym_data_req based on the \field{opcode}
    field in general header.
\item The device MUST parse the fields of struct virtio_crypto_cipher_data_flf in
    struct virtio_crypto_sym_data_flf and struct virtio_crypto_cipher_data_vlf in
    struct virtio_crypto_sym_data_vlf if the request is based on VIRTIO_CRYPTO_SYM_OP_CIPHER.
\item The device MUST parse the fields of struct virtio_crypto_alg_chain_data_flf
    in struct virtio_crypto_sym_data_flf and struct virtio_crypto_alg_chain_data_vlf
    in struct virtio_crypto_sym_data_vlf if the request is of the VIRTIO_CRYPTO_SYM_OP_ALGORITHM_CHAINING
    type.
\item The device MUST copy the result of cryptographic operation in the dst_data[] in
    both plain CIPHER mode and algorithms chain mode.
\item The device MUST check the \field{para}.\field{add_len} is bigger than 0 before
    parse the additional authenticated data in plain algorithms chain mode.
\item The device MUST copy the result of HASH/MAC operation in the hash_result[] is
    of the VIRTIO_CRYPTO_SYM_OP_ALGORITHM_CHAINING type.
\item The device MUST set the \field{status} field in struct virtio_crypto_inhdr to
    one of the following values of enum VIRTIO_CRYPTO_STATUS:
\begin{itemize*}
\item VIRTIO_CRYPTO_OK if the operation success.
\item VIRTIO_CRYPTO_NOTSUPP if the requested algorithm or operation is unsupported.
\item VIRTIO_CRYPTO_INVSESS if the session ID is invalid in session mode.
\item VIRTIO_CRYPTO_ERR if failure not mentioned above occurs.
\end{itemize*}
\end{itemize*}

\subsubsection{AEAD Service Operation}\label{sec:Device Types / Crypto Device / Device Operation / AEAD Service Operation}

Session mode requests of symmetric algorithm are as follows:

\begin{lstlisting}
struct virtio_crypto_aead_data_flf {
    /*
     * Byte Length of valid IV data.
     *
     * For GCM mode, this is either 12 (for 96-bit IVs) or 16, in which
     *   case iv points to J0.
     * For CCM mode, this is the length of the nonce, which can be in the
     *   range 7 to 13 inclusive.
     */
    le32 iv_len;
    /* length of additional auth data */
    le32 aad_len;
    /* length of source data */
    le32 src_data_len;
    /* length of dst data, this should be at least src_data_len + tag_len */
    le32 dst_data_len;
    /* Authentication tag length */
    le32 tag_len;
    le32 reserved;
};

struct virtio_crypto_aead_data_vlf {
    /* Device read only portion */

    /*
     * Initialization Vector data.
     *
     * For GCM mode, this is either the IV (if the length is 96 bits) or J0
     *   (for other sizes), where J0 is as defined by NIST SP800-38D.
     *   Regardless of the IV length, a full 16 bytes needs to be allocated.
     * For CCM mode, the first byte is reserved, and the nonce should be
     *   written starting at &iv[1] (to allow space for the implementation
     *   to write in the flags in the first byte).  Note that a full 16 bytes
     *   should be allocated, even though the iv_len field will have
     *   a value less than this.
     *
     * The IV will be updated after every partial cryptographic operation.
     */
    u8 iv[iv_len];
    /* Source data */
    u8 src_data[src_data_len];
    /* Additional authenticated data if exists */
    u8 aad[aad_len];

    /* Device write only portion */
    /* Pointer to output data */
    u8 dst_data[dst_data_len];
};
\end{lstlisting}

Each request uses the virtio_crypto_aead_data_flf structure and the
virtio_crypto_aead_data_flf structure to store information used to run the
AEAD operations.

Stateless mode AEAD service requests are as follows:

\begin{lstlisting}
struct virtio_crypto_aead_data_flf_stateless {
    struct {
        /* See VIRTIO_CRYPTO_AEAD_* above */
        le32 algo;
        /* length of key */
        le32 key_len;
        /* encrypt or decrypt, See above VIRTIO_CRYPTO_OP_* */
        le32 op;
    } sess_para;

    /* Byte Length of valid IV data. */
    le32 iv_len;
    /* Authentication tag length */
    le32 tag_len;
    /* length of additional auth data */
    le32 aad_len;
    /* length of source data */
    le32 src_data_len;
    /* length of dst data, this should be at least src_data_len + tag_len */
    le32 dst_data_len;
};

struct virtio_crypto_aead_data_vlf_stateless {
    /* Device read only portion */

    /* The cipher key */
    u8 key[key_len];
    /* Initialization Vector data. */
    u8 iv[iv_len];
    /* Source data */
    u8 src_data[src_data_len];
    /* Additional authenticated data if exists */
    u8 aad[aad_len];

    /* Device write only portion */
    /* Pointer to output data */
    u8 dst_data[dst_data_len];
};
\end{lstlisting}

\drivernormative{\paragraph}{AEAD Service Operation}{Device Types / Crypto Device / Device Operation / AEAD Service Operation}

\begin{itemize*}
\item If the driver uses the session mode, then the driver MUST set
    \field{session_id} in struct virtio_crypto_op_header to a valid value assigned
    by the device when the session was created.
\item If the VIRTIO_CRYPTO_F_AEAD_STATELESS_MODE feature bit is negotiated, 1) if
    the driver uses the stateless mode, then the driver MUST set the \field{flag}
    field in struct virtio_crypto_op_header to ZERO and MUST set the fields in
    struct virtio_crypto_aead_data_flf_stateless.sess_para, 2) if the driver uses
    the session mode, then the driver MUST set the \field{flag} field in struct
    virtio_crypto_op_header to VIRTIO_CRYPTO_FLAG_SESSION_MODE.
\item The driver MUST set the \field{opcode} field in struct virtio_crypto_op_header
    to VIRTIO_CRYPTO_AEAD_ENCRYPT or VIRTIO_CRYPTO_AEAD_DECRYPT.
\end{itemize*}

\devicenormative{\paragraph}{AEAD Service Operation}{Device Types / Crypto Device / Device Operation / AEAD Service Operation}

\begin{itemize*}
\item If the VIRTIO_CRYPTO_F_AEAD_STATELESS_MODE feature bit is negotiated, the
    device MUST parse the virtio_crypto_aead_data_vlf_stateless based on the \field{opcode}
	field in general header.
\item The device MUST copy the result of cryptographic operation in the dst_data[].
\item The device MUST copy the authentication tag in the dst_data[] offset the cipher result.
\item The device MUST set the \field{status} field in struct virtio_crypto_inhdr to
    one of the following values of enum VIRTIO_CRYPTO_STATUS:
\item When the \field{opcode} field is VIRTIO_CRYPTO_AEAD_DECRYPT, the device MUST
    verify and return the verification result to the driver.
\begin{itemize*}
\item VIRTIO_CRYPTO_OK if the operation success.
\item VIRTIO_CRYPTO_NOTSUPP if the requested algorithm or operation is unsupported.
\item VIRTIO_CRYPTO_BADMSG if the verification result is incorrect.
\item VIRTIO_CRYPTO_INVSESS if the session ID invalid when in session mode.
\item VIRTIO_CRYPTO_ERR if any failure not mentioned above occurs.
\end{itemize*}
\end{itemize*}

\subsubsection{AKCIPHER Service Operation}\label{sec:Device Types / Crypto Device / Device Operation / AKCIPHER Service Operation}

Session mode AKCIPHER requests are as follows:

\begin{lstlisting}
struct virtio_crypto_akcipher_data_flf {
    /* length of source data */
    le32 src_data_len;
    /* length of dst data */
    le32 dst_data_len;
};

struct virtio_crypto_akcipher_data_vlf {
    /* Device read only portion */
    /* Source data */
    u8 src_data[src_data_len];

    /* Device write only portion */
    /* Pointer to output data */
    u8 dst_data[dst_data_len];
};
\end{lstlisting}

Each data request uses the virtio_crypto_akcipher_flf structure and the virtio_crypto_akcipher_data_vlf
structure to store information used to run the AKCIPHER operations.

For encryption, decryption, and signing:
\field{src_data} is the source data that will be processed, note that for signing operations,
src_data stores the data to be signed, which usually is the digest of some data rather than the
data itself.
\field{src_data_len} is the length of source data.
\field{dst_result} is the result data and \field{dst_data_len} is the length of it. Note that the
length of the result is not always exactly equal to dst_data_len, the driver needs to check how
many bytes the device has written and calculate the actual length of the result.

For verification:
\field{src_data_len} refers to the length of the signature, and \field{dst_data_len} refers to
the length of signed data, where the signed data is usually the digest of some data.
\field{src_data} is spliced by the signature and the signed data, the src_data with the lower
address stores the signature, and the higher address stores the signed data.
\field{dst_data} is always empty for verification.

Different algorithms have different signature formats.
For the RSA algorithm, the result is determined by the padding algorithm specified by
\field{padding_algo} in structure virtio_crypto_rsa_session_para.

For the ECDSA algorithm, the signature is composed of the following
ASN.1 structure (see \hyperref[intro:rfc3279]{RFC3279})
and MUST be DER encoded (see \hyperref[intro:rfc6025]{rfc6025}).

\begin{lstlisting}
Ecdsa-Sig-Value ::= SEQUENCE {
    r INTEGER,
    s INTEGER
}
\end{lstlisting}

Stateless mode AKCIPHER service requests are as follows:
\begin{lstlisting}
struct virtio_crypto_akcipher_data_flf_stateless {
    struct {
        /* See VIRTIO_CYRPTO_AKCIPHER* above */
        le32 algo;
        /* See VIRTIO_CRYPTO_AKCIPHER_KEY_TYPE_* above */
        le32 key_type;
        /* length of key */
        le32 key_len;

        /* algothrim specific parameters described above */
        union {
            struct virtio_crypto_rsa_session_para rsa;
            struct virtio_crypto_ecdsa_session_para ecdsa;
        } u;
    } sess_para;

    /* length of source data */
    le32 src_data_len;
    /* length of destination data */
    le32 dst_data_len;
};

struct virtio_crypto_akcipher_data_vlf_stateless {
    /* Device read only portion */
    u8 akcipher_key[key_len];

    /* Source data */
    u8 src_data[src_data_len];

    /* Device write only portion */
    u8 dst_data[dst_data_len];
};
\end{lstlisting}

In stateless mode, the format of key and signature, the meaning of src_data and dst_data, are all the same
with session mode.

\drivernormative{\paragraph}{AKCIPHER Service Operation}{Device Types / Crypto Device / Device Operation / AKCIPHER Service Operation}

\begin{itemize*}
\item If the driver uses the session mode, then the driver MUST set
    \field{session_id} in struct virtio_crypto_op_header to a valid
    value assigned by the device when the session was created.
\item If the VIRTIO_CRYPTO_F_AKCIPHER_STATELESS_MODE feature bit is negotiated, 1) if the
    driver uses the stateless mode, then the driver MUST set the \field{flag} field in
    struct virtio_crypto_op_header to ZERO and MUST set the fields in struct
    virtio_crypto_akcipher_flf_stateless.sess_para, 2) if the driver uses the session
    mode, then the driver MUST set the \field{flag} field in struct virtio_crypto_op_header
    to VIRTIO_CRYPTO_FLAG_SESSION_MODE.
\item The driver MUST set the \field{opcode} field in struct virtio_crypto_op_header
    to one of VIRTIO_CRYPTO_AKCIPHER_ENCRYPT, VIRTIO_CRYPTO_AKCIPHER_DESTROY_SESSION,
    VIRTIO_CRYPTO_AKCIPHER_SIGN, and VIRTIO_CRYPTO_AKCIPHER_VERIFY.
\end{itemize*}

\devicenormative{\paragraph}{AKCIPHER Service Operation}{Device Types / Crypto Device / Device Operation / AKCIPHER Service Operation}

\begin{itemize*}
\item If the VIRTIO_CRYPTO_F_AKCIPHER_STATELESS_MODE feature bit is negotiated, the
    device MUST parse the virtio_crypto_akcipher_data_vlf_stateless based on the \field{opcode}
    field in general header.
\item The device MUST copy the result of cryptographic operation in the dst_data[].
\item The device MUST set the \field{status} field in struct virtio_crypto_inhdr to
    one of the following values of enum VIRTIO_CRYPTO_STATUS:
\begin{itemize*}
\item VIRTIO_CRYPTO_OK if the operation success.
\item VIRTIO_CRYPTO_NOTSUPP if the requested algorithm or operation is unsupported.
\item VIRTIO_CRYPTO_BADMSG if the verification result is incorrect.
\item VIRTIO_CRYPTO_INVSESS if the session ID invalid when in session mode.
\item VIRTIO_CRYPTO_KEY_REJECTED if the signature verification failed.
\item VIRTIO_CRYPTO_ERR if any failure not mentioned above occurs.
\end{itemize*}
\end{itemize*}

\section{Crypto Device}\label{sec:Device Types / Crypto Device}

The virtio crypto device is a virtual cryptography device as well as a
virtual cryptographic accelerator. The virtio crypto device provides the
following crypto services: CIPHER, MAC, HASH, AEAD and AKCIPHER. Virtio crypto
devices have a single control queue and at least one data queue. Crypto
operation requests are placed into a data queue, and serviced by the
device. Some crypto operation requests are only valid in the context of a
session. The role of the control queue is facilitating control operation
requests. Sessions management is realized with control operation
requests.

\subsection{Device ID}\label{sec:Device Types / Crypto Device / Device ID}

20

\subsection{Virtqueues}\label{sec:Device Types / Crypto Device / Virtqueues}

\begin{description}
\item[0] dataq1
\item[\ldots]
\item[N-1] dataqN
\item[N] controlq
\end{description}

N is set by \field{max_dataqueues}.

\subsection{Feature bits}\label{sec:Device Types / Crypto Device / Feature bits}

\begin{description}
\item VIRTIO_CRYPTO_F_REVISION_1 (0) revision 1. Revision 1 has a specific
    request format and other enhancements (which result in some additional
    requirements).
\item VIRTIO_CRYPTO_F_CIPHER_STATELESS_MODE (1) stateless mode requests are
    supported by the CIPHER service.
\item VIRTIO_CRYPTO_F_HASH_STATELESS_MODE (2) stateless mode requests are
    supported by the HASH service.
\item VIRTIO_CRYPTO_F_MAC_STATELESS_MODE (3) stateless mode requests are
    supported by the MAC service.
\item VIRTIO_CRYPTO_F_AEAD_STATELESS_MODE (4) stateless mode requests are
    supported by the AEAD service.
\item VIRTIO_CRYPTO_F_AKCIPHER_STATELESS_MODE (5) stateless mode requests are
    supported by the AKCIPHER service.
\end{description}


\subsubsection{Feature bit requirements}\label{sec:Device Types / Crypto Device / Feature bit requirements}

Some crypto feature bits require other crypto feature bits
(see \ref{drivernormative:Basic Facilities of a Virtio Device / Feature Bits}):

\begin{description}
\item[VIRTIO_CRYPTO_F_CIPHER_STATELESS_MODE] Requires VIRTIO_CRYPTO_F_REVISION_1.
\item[VIRTIO_CRYPTO_F_HASH_STATELESS_MODE] Requires VIRTIO_CRYPTO_F_REVISION_1.
\item[VIRTIO_CRYPTO_F_MAC_STATELESS_MODE] Requires VIRTIO_CRYPTO_F_REVISION_1.
\item[VIRTIO_CRYPTO_F_AEAD_STATELESS_MODE] Requires VIRTIO_CRYPTO_F_REVISION_1.
\item[VIRTIO_CRYPTO_F_AKCIPHER_STATELESS_MODE] Requires VIRTIO_CRYPTO_F_REVISION_1.
\end{description}

\subsection{Supported crypto services}\label{sec:Device Types / Crypto Device / Supported crypto services}

The following crypto services are defined:

\begin{lstlisting}
/* CIPHER (Symmetric Key Cipher) service */
#define VIRTIO_CRYPTO_SERVICE_CIPHER 0
/* HASH service */
#define VIRTIO_CRYPTO_SERVICE_HASH   1
/* MAC (Message Authentication Codes) service */
#define VIRTIO_CRYPTO_SERVICE_MAC    2
/* AEAD (Authenticated Encryption with Associated Data) service */
#define VIRTIO_CRYPTO_SERVICE_AEAD   3
/* AKCIPHER (Asymmetric Key Cipher) service */
#define VIRTIO_CRYPTO_SERVICE_AKCIPHER 4
\end{lstlisting}

The above constants designate bits used to indicate the which of crypto services are
offered by the device as described in, see \ref{sec:Device Types / Crypto Device / Device configuration layout}.

\subsubsection{CIPHER services}\label{sec:Device Types / Crypto Device / Supported crypto services / CIPHER services}

The following CIPHER algorithms are defined:

\begin{lstlisting}
#define VIRTIO_CRYPTO_NO_CIPHER                 0
#define VIRTIO_CRYPTO_CIPHER_ARC4               1
#define VIRTIO_CRYPTO_CIPHER_AES_ECB            2
#define VIRTIO_CRYPTO_CIPHER_AES_CBC            3
#define VIRTIO_CRYPTO_CIPHER_AES_CTR            4
#define VIRTIO_CRYPTO_CIPHER_DES_ECB            5
#define VIRTIO_CRYPTO_CIPHER_DES_CBC            6
#define VIRTIO_CRYPTO_CIPHER_3DES_ECB           7
#define VIRTIO_CRYPTO_CIPHER_3DES_CBC           8
#define VIRTIO_CRYPTO_CIPHER_3DES_CTR           9
#define VIRTIO_CRYPTO_CIPHER_KASUMI_F8          10
#define VIRTIO_CRYPTO_CIPHER_SNOW3G_UEA2        11
#define VIRTIO_CRYPTO_CIPHER_AES_F8             12
#define VIRTIO_CRYPTO_CIPHER_AES_XTS            13
#define VIRTIO_CRYPTO_CIPHER_ZUC_EEA3           14
\end{lstlisting}

The above constants have two usages:
\begin{enumerate}
\item As bit numbers, used to tell the driver which CIPHER algorithms
are supported by the device, see \ref{sec:Device Types / Crypto Device / Device configuration layout}.
\item As values, used to designate the algorithm in (CIPHER type) crypto
operation requests, see \ref{sec:Device Types / Crypto Device / Device Operation / Control Virtqueue / Session operation}.
\end{enumerate}

\subsubsection{HASH services}\label{sec:Device Types / Crypto Device / Supported crypto services / HASH services}

The following HASH algorithms are defined:

\begin{lstlisting}
#define VIRTIO_CRYPTO_NO_HASH            0
#define VIRTIO_CRYPTO_HASH_MD5           1
#define VIRTIO_CRYPTO_HASH_SHA1          2
#define VIRTIO_CRYPTO_HASH_SHA_224       3
#define VIRTIO_CRYPTO_HASH_SHA_256       4
#define VIRTIO_CRYPTO_HASH_SHA_384       5
#define VIRTIO_CRYPTO_HASH_SHA_512       6
#define VIRTIO_CRYPTO_HASH_SHA3_224      7
#define VIRTIO_CRYPTO_HASH_SHA3_256      8
#define VIRTIO_CRYPTO_HASH_SHA3_384      9
#define VIRTIO_CRYPTO_HASH_SHA3_512      10
#define VIRTIO_CRYPTO_HASH_SHA3_SHAKE128      11
#define VIRTIO_CRYPTO_HASH_SHA3_SHAKE256      12
\end{lstlisting}

The above constants have two usages:
\begin{enumerate}
\item As bit numbers, used to tell the driver which HASH algorithms
are supported by the device, see \ref{sec:Device Types / Crypto Device / Device configuration layout}.
\item As values, used to designate the algorithm in (HASH type) crypto
operation requires, see \ref{sec:Device Types / Crypto Device / Device Operation / Control Virtqueue / Session operation}.
\end{enumerate}

\subsubsection{MAC services}\label{sec:Device Types / Crypto Device / Supported crypto services / MAC services}

The following MAC algorithms are defined:

\begin{lstlisting}
#define VIRTIO_CRYPTO_NO_MAC                       0
#define VIRTIO_CRYPTO_MAC_HMAC_MD5                 1
#define VIRTIO_CRYPTO_MAC_HMAC_SHA1                2
#define VIRTIO_CRYPTO_MAC_HMAC_SHA_224             3
#define VIRTIO_CRYPTO_MAC_HMAC_SHA_256             4
#define VIRTIO_CRYPTO_MAC_HMAC_SHA_384             5
#define VIRTIO_CRYPTO_MAC_HMAC_SHA_512             6
#define VIRTIO_CRYPTO_MAC_CMAC_3DES                25
#define VIRTIO_CRYPTO_MAC_CMAC_AES                 26
#define VIRTIO_CRYPTO_MAC_KASUMI_F9                27
#define VIRTIO_CRYPTO_MAC_SNOW3G_UIA2              28
#define VIRTIO_CRYPTO_MAC_GMAC_AES                 41
#define VIRTIO_CRYPTO_MAC_GMAC_TWOFISH             42
#define VIRTIO_CRYPTO_MAC_CBCMAC_AES               49
#define VIRTIO_CRYPTO_MAC_CBCMAC_KASUMI_F9         50
#define VIRTIO_CRYPTO_MAC_XCBC_AES                 53
#define VIRTIO_CRYPTO_MAC_ZUC_EIA3                 54
\end{lstlisting}

The above constants have two usages:
\begin{enumerate}
\item As bit numbers, used to tell the driver which MAC algorithms
are supported by the device, see \ref{sec:Device Types / Crypto Device / Device configuration layout}.
\item As values, used to designate the algorithm in (MAC type) crypto
operation requests, see \ref{sec:Device Types / Crypto Device / Device Operation / Control Virtqueue / Session operation}.
\end{enumerate}

\subsubsection{AEAD services}\label{sec:Device Types / Crypto Device / Supported crypto services / AEAD services}

The following AEAD algorithms are defined:

\begin{lstlisting}
#define VIRTIO_CRYPTO_NO_AEAD     0
#define VIRTIO_CRYPTO_AEAD_GCM    1
#define VIRTIO_CRYPTO_AEAD_CCM    2
#define VIRTIO_CRYPTO_AEAD_CHACHA20_POLY1305  3
\end{lstlisting}

The above constants have two usages:
\begin{enumerate}
\item As bit numbers, used to tell the driver which AEAD algorithms
are supported by the device, see \ref{sec:Device Types / Crypto Device / Device configuration layout}.
\item As values, used to designate the algorithm in (DEAD type) crypto
operation requests, see \ref{sec:Device Types / Crypto Device / Device Operation / Control Virtqueue / Session operation}.
\end{enumerate}

\subsubsection{AKCIPHER services}\label{sec: Device Types / Crypto Device / Supported crypto services / AKCIPHER services}

The following AKCIPHER algorithms are defined:
\begin{lstlisting}
#define VIRTIO_CRYPTO_NO_AKCIPHER 0
#define VIRTIO_CRYPTO_AKCIPHER_RSA   1
#define VIRTIO_CRYPTO_AKCIPHER_ECDSA 2
\end{lstlisting}

The above constants have two usages:
\begin{enumerate}
\item As bit numbers, used to tell the driver which AKCIPHER algorithms
are supported by the device, see \ref{sec:Device Types / Crypto Device / Device configuration layout}.
\item As values, used to designate the algorithm in asymmetric crypto operation requests,
see \ref{sec:Device Types / Crypto Device / Device Operation / Control Virtqueue / Session operation}.
\end{enumerate}


\subsection{Device configuration layout}\label{sec:Device Types / Crypto Device / Device configuration layout}

Crypto device configuration uses the following layout structure:

\begin{lstlisting}
struct virtio_crypto_config {
    le32 status;
    le32 max_dataqueues;
    le32 crypto_services;
    /* Detailed algorithms mask */
    le32 cipher_algo_l;
    le32 cipher_algo_h;
    le32 hash_algo;
    le32 mac_algo_l;
    le32 mac_algo_h;
    le32 aead_algo;
    /* Maximum length of cipher key in bytes */
    le32 max_cipher_key_len;
    /* Maximum length of authenticated key in bytes */
    le32 max_auth_key_len;
    le32 akcipher_algo;
    /* Maximum size of each crypto request's content in bytes */
    le64 max_size;
};
\end{lstlisting}

\begin{description}
\item Currently, only one \field{status} bit is defined: VIRTIO_CRYPTO_S_HW_READY
    set indicates that the device is ready to process requests, this bit is read-only
    for the driver
\begin{lstlisting}
#define VIRTIO_CRYPTO_S_HW_READY  (1 << 0)
\end{lstlisting}

\item [\field{max_dataqueues}] is the maximum number of data virtqueues that can
    be configured by the device. The driver MAY use only one data queue, or it
    can use more to achieve better performance.

\item [\field{crypto_services}] crypto service offered, see \ref{sec:Device Types / Crypto Device / Supported crypto services}.

\item [\field{cipher_algo_l}] CIPHER algorithms bits 0-31, see \ref{sec:Device Types / Crypto Device / Supported crypto services  / CIPHER services}.

\item [\field{cipher_algo_h}] CIPHER algorithms bits 32-63, see \ref{sec:Device Types / Crypto Device / Supported crypto services  / CIPHER services}.

\item [\field{hash_algo}] HASH algorithms bits, see \ref{sec:Device Types / Crypto Device / Supported crypto services  / HASH services}.

\item [\field{mac_algo_l}] MAC algorithms bits 0-31, see \ref{sec:Device Types / Crypto Device / Supported crypto services  / MAC services}.

\item [\field{mac_algo_h}] MAC algorithms bits 32-63, see \ref{sec:Device Types / Crypto Device / Supported crypto services  / MAC services}.

\item [\field{aead_algo}] AEAD algorithms bits, see \ref{sec:Device Types / Crypto Device / Supported crypto services  / AEAD services}.

\item [\field{max_cipher_key_len}] is the maximum length of cipher key supported by the device.

\item [\field{max_auth_key_len}] is the maximum length of authenticated key supported by the device.

\item [\field{akcipher_algo}] AKCIPHER algorithms bit 0-31, see \ref{sec: Device Types / Crypto Device / Supported crypto services / AKCIPHER services}.

\item [\field{max_size}] is the maximum size of the variable-length parameters of
    data operation of each crypto request's content supported by the device.
\end{description}

\begin{note}
Unless explicitly stated otherwise all lengths and sizes are in bytes.
\end{note}

\devicenormative{\subsubsection}{Device configuration layout}{Device Types / Crypto Device / Device configuration layout}

\begin{itemize*}
\item The device MUST set \field{max_dataqueues} to between 1 and 65535 inclusive.
\item The device MUST set the \field{status} with valid flags, undefined flags MUST NOT be set.
\item The device MUST accept and handle requests after \field{status} is set to VIRTIO_CRYPTO_S_HW_READY.
\item The device MUST set \field{crypto_services} based on the crypto services the device offers.
\item The device MUST set detailed algorithms masks for each service advertised by \field{crypto_services}.
    The device MUST NOT set the not defined algorithms bits.
\item The device MUST set \field{max_size} to show the maximum size of crypto request the device supports.
\item The device MUST set \field{max_cipher_key_len} to show the maximum length of cipher key if the
    device supports CIPHER service.
\item The device MUST set \field{max_auth_key_len} to show the maximum length of authenticated key if
    the device supports MAC service.
\end{itemize*}

\drivernormative{\subsubsection}{Device configuration layout}{Device Types / Crypto Device / Device configuration layout}

\begin{itemize*}
\item The driver MUST read the \field{status} from the bottom bit of status to check whether the
    VIRTIO_CRYPTO_S_HW_READY is set, and the driver MUST reread it after device reset.
\item The driver MUST NOT transmit any requests to the device if the VIRTIO_CRYPTO_S_HW_READY is not set.
\item The driver MUST read \field{max_dataqueues} field to discover the number of data queues the device supports.
\item The driver MUST read \field{crypto_services} field to discover which services the device is able to offer.
\item The driver SHOULD ignore the not defined algorithms bits.
\item The driver MUST read the detailed algorithms fields based on \field{crypto_services} field.
\item The driver SHOULD read \field{max_size} to discover the maximum size of the variable-length
    parameters of data operation of the crypto request's content the device supports and MUST
    guarantee that the size of each crypto request's content is within the \field{max_size}, otherwise
    the request will fail and the driver MUST reset the device.
\item The driver SHOULD read \field{max_cipher_key_len} to discover the maximum length of cipher key
    the device supports and MUST guarantee that the \field{key_len} (CIPHER service or AEAD service) is within
    the \field{max_cipher_key_len} of the device configuration, otherwise the request will fail.
\item The driver SHOULD read \field{max_auth_key_len} to discover the maximum length of authenticated
    key the device supports and MUST guarantee that the \field{auth_key_len} (MAC service) is within the
    \field{max_auth_key_len} of the device configuration, otherwise the request will fail.
\end{itemize*}

\subsection{Device Initialization}\label{sec:Device Types / Crypto Device / Device Initialization}

\drivernormative{\subsubsection}{Device Initialization}{Device Types / Crypto Device / Device Initialization}

\begin{itemize*}
\item The driver MUST configure and initialize all virtqueues.
\item The driver MUST read the supported crypto services from bits of \field{crypto_services}.
\item The driver MUST read the supported algorithms based on \field{crypto_services} field.
\end{itemize*}

\subsection{Device Operation}\label{sec:Device Types / Crypto Device / Device Operation}

The operation of a virtio crypto device is driven by requests placed on the virtqueues.
Requests consist of a queue-type specific header (specifying among others the operation)
and an operation specific payload.

If VIRTIO_CRYPTO_F_REVISION_1 is negotiated the device may support both session mode
(See \ref{sec:Device Types / Crypto Device / Device Operation / Control Virtqueue / Session operation})
and stateless mode operation requests.
In stateless mode all operation parameters are supplied as a part of each request,
while in session mode, some or all operation parameters are managed within the
session. Stateless mode is guarded by feature bits 0-4 on a service level. If
stateless mode is negotiated for a service, the service accepts both session
mode and stateless requests; otherwise stateless mode requests are rejected
(via operation status).

\subsubsection{Operation Status}\label{sec:Device Types / Crypto Device / Device Operation / Operation status}
The device MUST return a status code as part of the operation (both session
operation and service operation) result. The valid operation status as follows:

\begin{lstlisting}
enum VIRTIO_CRYPTO_STATUS {
    VIRTIO_CRYPTO_OK = 0,
    VIRTIO_CRYPTO_ERR = 1,
    VIRTIO_CRYPTO_BADMSG = 2,
    VIRTIO_CRYPTO_NOTSUPP = 3,
    VIRTIO_CRYPTO_INVSESS = 4,
    VIRTIO_CRYPTO_NOSPC = 5,
    VIRTIO_CRYPTO_KEY_REJECTED = 6,
    VIRTIO_CRYPTO_MAX
};
\end{lstlisting}

\begin{itemize*}
\item VIRTIO_CRYPTO_OK: success.
\item VIRTIO_CRYPTO_BADMSG: authentication failed (only when AEAD decryption).
\item VIRTIO_CRYPTO_NOTSUPP: operation or algorithm is unsupported.
\item VIRTIO_CRYPTO_INVSESS: invalid session ID when executing crypto operations.
\item VIRTIO_CRYPTO_NOSPC: no free session ID (only when the VIRTIO_CRYPTO_F_REVISION_1
    feature bit is negotiated).
\item VIRTIO_CRYPTO_KEY_REJECTED: signature verification failed (only when AKCIPHER verification).
\item VIRTIO_CRYPTO_ERR: any failure not mentioned above occurs.
\end{itemize*}

\subsubsection{Control Virtqueue}\label{sec:Device Types / Crypto Device / Device Operation / Control Virtqueue}

The driver uses the control virtqueue to send control commands to the
device, such as session operations (See \ref{sec:Device Types / Crypto Device / Device
Operation / Control Virtqueue / Session operation}).

The header for controlq is of the following form:
\begin{lstlisting}
#define VIRTIO_CRYPTO_OPCODE(service, op)   (((service) << 8) | (op))

struct virtio_crypto_ctrl_header {
#define VIRTIO_CRYPTO_CIPHER_CREATE_SESSION \
       VIRTIO_CRYPTO_OPCODE(VIRTIO_CRYPTO_SERVICE_CIPHER, 0x02)
#define VIRTIO_CRYPTO_CIPHER_DESTROY_SESSION \
       VIRTIO_CRYPTO_OPCODE(VIRTIO_CRYPTO_SERVICE_CIPHER, 0x03)
#define VIRTIO_CRYPTO_HASH_CREATE_SESSION \
       VIRTIO_CRYPTO_OPCODE(VIRTIO_CRYPTO_SERVICE_HASH, 0x02)
#define VIRTIO_CRYPTO_HASH_DESTROY_SESSION \
       VIRTIO_CRYPTO_OPCODE(VIRTIO_CRYPTO_SERVICE_HASH, 0x03)
#define VIRTIO_CRYPTO_MAC_CREATE_SESSION \
       VIRTIO_CRYPTO_OPCODE(VIRTIO_CRYPTO_SERVICE_MAC, 0x02)
#define VIRTIO_CRYPTO_MAC_DESTROY_SESSION \
       VIRTIO_CRYPTO_OPCODE(VIRTIO_CRYPTO_SERVICE_MAC, 0x03)
#define VIRTIO_CRYPTO_AEAD_CREATE_SESSION \
       VIRTIO_CRYPTO_OPCODE(VIRTIO_CRYPTO_SERVICE_AEAD, 0x02)
#define VIRTIO_CRYPTO_AEAD_DESTROY_SESSION \
       VIRTIO_CRYPTO_OPCODE(VIRTIO_CRYPTO_SERVICE_AEAD, 0x03)
#define VIRTIO_CRYPTO_AKCIPHER_CREATE_SESSION \
       VIRTIO_CRYPTO_OPCODE(VIRTIO_CRYPTO_SERVICE_AKCIPHER, 0x04)
#define VIRTIO_CRYPTO_AKCIPHER_DESTROY_SESSION \
       VIRTIO_CRYPTO_OPCDE(VIRTIO_CRYPTO_SERVICE_AKCIPHER, 0x05)
    le32 opcode;
    /* algo should be service-specific algorithms */
    le32 algo;
    le32 flag;
    le32 reserved;
};
\end{lstlisting}

The controlq request is composed of four parts:
\begin{lstlisting}
struct virtio_crypto_op_ctrl_req {
    /* Device read only portion */

    struct virtio_crypto_ctrl_header header;

#define VIRTIO_CRYPTO_CTRLQ_OP_SPEC_HDR_LEGACY 56
    /* fixed length fields, opcode specific */
    u8 op_flf[flf_len];

    /* variable length fields, opcode specific */
    u8 op_vlf[vlf_len];

    /* Device write only portion */

    /* op result or completion status */
    u8 op_outcome[outcome_len];
};
\end{lstlisting}

\field{header} is a general header (see above).

\field{op_flf} is the opcode (in \field{header}) specific fixed-length parameters.

\field{flf_len} depends on the VIRTIO_CRYPTO_F_REVISION_1 feature bit (see below).

\field{op_vlf} is the opcode (in \field{header}) specific variable-length parameters.

\field{vlf_len} is the size of the specific structure used.
\begin{note}
The \field{vlf_len} of session-destroy operation and the hash-session-create
operation is ZERO.
\end{note}

\begin{itemize*}
\item If the opcode (in \field{header}) is VIRTIO_CRYPTO_CIPHER_CREATE_SESSION
    then \field{op_flf} is struct virtio_crypto_sym_create_session_flf if
    VIRTIO_CRYPTO_F_REVISION_1 is negotiated and struct virtio_crypto_sym_create_session_flf is
    padded to 56 bytes if NOT negotiated, and \field{op_vlf} is struct
    virtio_crypto_sym_create_session_vlf.
\item If the opcode (in \field{header}) is VIRTIO_CRYPTO_HASH_CREATE_SESSION
    then \field{op_flf} is struct virtio_crypto_hash_create_session_flf if
    VIRTIO_CRYPTO_F_REVISION_1 is negotiated and struct virtio_crypto_hash_create_session_flf is
    padded to 56 bytes if NOT negotiated.
\item If the opcode (in \field{header}) is VIRTIO_CRYPTO_MAC_CREATE_SESSION
    then \field{op_flf} is struct virtio_crypto_mac_create_session_flf if
    VIRTIO_CRYPTO_F_REVISION_1 is negotiated and struct virtio_crypto_mac_create_session_flf is
    padded to 56 bytes if NOT negotiated, and \field{op_vlf} is struct
    virtio_crypto_mac_create_session_vlf.
\item If the opcode (in \field{header}) is VIRTIO_CRYPTO_AEAD_CREATE_SESSION
    then \field{op_flf} is struct virtio_crypto_aead_create_session_flf if
    VIRTIO_CRYPTO_F_REVISION_1 is negotiated and struct virtio_crypto_aead_create_session_flf is
    padded to 56 bytes if NOT negotiated, and \field{op_vlf} is struct
    virtio_crypto_aead_create_session_vlf.
\item If the opcode (in \field{header}) is VIRTIO_CRYPTO_AKCIPHER_CREATE_SESSION
    then \field{op_flf} is struct virtio_crypto_akcipher_create_session_flf if
    VIRTIO_CRYPTO_F_REVISION_1 is negotiated and struct virtio_crypto_akcipher_create_session_flf is
    padded to 56 bytes if NOT negotiated, and \field{op_vlf} is struct
    virtio_crypto_akcipher_create_session_vlf.
\item If the opcode (in \field{header}) is VIRTIO_CRYPTO_CIPHER_DESTROY_SESSION
    or VIRTIO_CRYPTO_HASH_DESTROY_SESSION or VIRTIO_CRYPTO_MAC_DESTROY_SESSION or
    VIRTIO_CRYPTO_AEAD_DESTROY_SESSION then \field{op_flf} is struct
    virtio_crypto_destroy_session_flf if VIRTIO_CRYPTO_F_REVISION_1 is negotiated and
    struct virtio_crypto_destroy_session_flf is padded to 56 bytes if NOT negotiated.
\end{itemize*}

\field{op_outcome} stores the result of operation and must be struct
virtio_crypto_destroy_session_input for destroy session or
struct virtio_crypto_create_session_input for create session.

\field{outcome_len} is the size of the structure used.


\paragraph{Session operation}\label{sec:Device Types / Crypto Device / Device
Operation / Control Virtqueue / Session operation}

The session is a handle which describes the cryptographic parameters to be
applied to a number of buffers.

The following structure stores the result of session creation set by the device:

\begin{lstlisting}
struct virtio_crypto_create_session_input {
    le64 session_id;
    le32 status;
    le32 padding;
};
\end{lstlisting}

A request to destroy a session includes the following information:

\begin{lstlisting}
struct virtio_crypto_destroy_session_flf {
    /* Device read only portion */
    le64  session_id;
};

struct virtio_crypto_destroy_session_input {
    /* Device write only portion */
    u8  status;
};
\end{lstlisting}


\subparagraph{Session operation: HASH session}\label{sec:Device Types / Crypto Device / Device
Operation / Control Virtqueue / Session operation / Session operation: HASH session}

The fixed-length parameters of HASH session requests is as follows:

\begin{lstlisting}
struct virtio_crypto_hash_create_session_flf {
    /* Device read only portion */

    /* See VIRTIO_CRYPTO_HASH_* above */
    le32 algo;
    /* hash result length */
    le32 hash_result_len;
};
\end{lstlisting}


\subparagraph{Session operation: MAC session}\label{sec:Device Types / Crypto Device / Device
Operation / Control Virtqueue / Session operation / Session operation: MAC session}

The fixed-length and the variable-length parameters of MAC session requests are as follows:

\begin{lstlisting}
struct virtio_crypto_mac_create_session_flf {
    /* Device read only portion */

    /* See VIRTIO_CRYPTO_MAC_* above */
    le32 algo;
    /* hash result length */
    le32 hash_result_len;
    /* length of authenticated key */
    le32 auth_key_len;
    le32 padding;
};

struct virtio_crypto_mac_create_session_vlf {
    /* Device read only portion */

    /* The authenticated key */
    u8 auth_key[auth_key_len];
};
\end{lstlisting}

The length of \field{auth_key} is specified in \field{auth_key_len} in the struct
virtio_crypto_mac_create_session_flf.


\subparagraph{Session operation: Symmetric algorithms session}\label{sec:Device Types / Crypto Device / Device
Operation / Control Virtqueue / Session operation / Session operation: Symmetric algorithms session}

The request of symmetric session could be the CIPHER algorithms request
or the chain algorithms (chaining CIPHER and HASH/MAC) request.

The fixed-length and the variable-length parameters of CIPHER session requests are as follows:

\begin{lstlisting}
struct virtio_crypto_cipher_session_flf {
    /* Device read only portion */

    /* See VIRTIO_CRYPTO_CIPHER* above */
    le32 algo;
    /* length of key */
    le32 key_len;
#define VIRTIO_CRYPTO_OP_ENCRYPT  1
#define VIRTIO_CRYPTO_OP_DECRYPT  2
    /* encryption or decryption */
    le32 op;
    le32 padding;
};

struct virtio_crypto_cipher_session_vlf {
    /* Device read only portion */

    /* The cipher key */
    u8 cipher_key[key_len];
};
\end{lstlisting}

The length of \field{cipher_key} is specified in \field{key_len} in the struct
virtio_crypto_cipher_session_flf.

The fixed-length and the variable-length parameters of Chain session requests are as follows:

\begin{lstlisting}
struct virtio_crypto_alg_chain_session_flf {
    /* Device read only portion */

#define VIRTIO_CRYPTO_SYM_ALG_CHAIN_ORDER_HASH_THEN_CIPHER  1
#define VIRTIO_CRYPTO_SYM_ALG_CHAIN_ORDER_CIPHER_THEN_HASH  2
    le32 alg_chain_order;
/* Plain hash */
#define VIRTIO_CRYPTO_SYM_HASH_MODE_PLAIN    1
/* Authenticated hash (mac) */
#define VIRTIO_CRYPTO_SYM_HASH_MODE_AUTH     2
/* Nested hash */
#define VIRTIO_CRYPTO_SYM_HASH_MODE_NESTED   3
    le32 hash_mode;
    struct virtio_crypto_cipher_session_flf cipher_hdr;

#define VIRTIO_CRYPTO_ALG_CHAIN_SESS_OP_SPEC_HDR_SIZE  16
    /* fixed length fields, algo specific */
    u8 algo_flf[VIRTIO_CRYPTO_ALG_CHAIN_SESS_OP_SPEC_HDR_SIZE];

    /* length of the additional authenticated data (AAD) in bytes */
    le32 aad_len;
    le32 padding;
};

struct virtio_crypto_alg_chain_session_vlf {
    /* Device read only portion */

    /* The cipher key */
    u8 cipher_key[key_len];
    /* The authenticated key */
    u8 auth_key[auth_key_len];
};
\end{lstlisting}

\field{hash_mode} decides the type used by \field{algo_flf}.

\field{algo_flf} is fixed to 16 bytes and MUST contains or be one of
the following types:
\begin{itemize*}
\item struct virtio_crypto_hash_create_session_flf
\item struct virtio_crypto_mac_create_session_flf
\end{itemize*}
The data of unused part (if has) in \field{algo_flf} will be ignored.

The length of \field{cipher_key} is specified in \field{key_len} in \field{cipher_hdr}.

The length of \field{auth_key} is specified in \field{auth_key_len} in struct
virtio_crypto_mac_create_session_flf.

The fixed-length parameters of Symmetric session requests are as follows:

\begin{lstlisting}
struct virtio_crypto_sym_create_session_flf {
    /* Device read only portion */

#define VIRTIO_CRYPTO_SYM_SESS_OP_SPEC_HDR_SIZE  48
    /* fixed length fields, opcode specific */
    u8 op_flf[VIRTIO_CRYPTO_SYM_SESS_OP_SPEC_HDR_SIZE];

/* No operation */
#define VIRTIO_CRYPTO_SYM_OP_NONE  0
/* Cipher only operation on the data */
#define VIRTIO_CRYPTO_SYM_OP_CIPHER  1
/* Chain any cipher with any hash or mac operation. The order
   depends on the value of alg_chain_order param */
#define VIRTIO_CRYPTO_SYM_OP_ALGORITHM_CHAINING  2
    le32 op_type;
    le32 padding;
};
\end{lstlisting}

\field{op_flf} is fixed to 48 bytes, MUST contains or be one of
the following types:
\begin{itemize*}
\item struct virtio_crypto_cipher_session_flf
\item struct virtio_crypto_alg_chain_session_flf
\end{itemize*}
The data of unused part (if has) in \field{op_flf} will be ignored.

\field{op_type} decides the type used by \field{op_flf}.

The variable-length parameters of Symmetric session requests are as follows:

\begin{lstlisting}
struct virtio_crypto_sym_create_session_vlf {
    /* Device read only portion */
    /* variable length fields, opcode specific */
    u8 op_vlf[vlf_len];
};
\end{lstlisting}

\field{op_vlf} MUST contains or be one of the following types:
\begin{itemize*}
\item struct virtio_crypto_cipher_session_vlf
\item struct virtio_crypto_alg_chain_session_vlf
\end{itemize*}

\field{op_type} in struct virtio_crypto_sym_create_session_flf decides the
type used by \field{op_vlf}.

\field{vlf_len} is the size of the specific structure used.


\subparagraph{Session operation: AEAD session}\label{sec:Device Types / Crypto Device / Device
Operation / Control Virtqueue / Session operation / Session operation: AEAD session}

The fixed-length and the variable-length parameters of AEAD session requests are as follows:

\begin{lstlisting}
struct virtio_crypto_aead_create_session_flf {
    /* Device read only portion */

    /* See VIRTIO_CRYPTO_AEAD_* above */
    le32 algo;
    /* length of key */
    le32 key_len;
    /* Authentication tag length */
    le32 tag_len;
    /* The length of the additional authenticated data (AAD) in bytes */
    le32 aad_len;
    /* encryption or decryption, See above VIRTIO_CRYPTO_OP_* */
    le32 op;
    le32 padding;
};

struct virtio_crypto_aead_create_session_vlf {
    /* Device read only portion */
    u8 key[key_len];
};
\end{lstlisting}

The length of \field{key} is specified in \field{key_len} in struct
virtio_crypto_aead_create_session_flf.

\subparagraph{Session operation: AKCIPHER session}\label{sec:Device Types / Crypto Device / Device
Operation / Control Virtqueue / Session operation / Session operation: AKCIPHER session}

Due to the complexity of asymmetric key algorithms, different algorithms
require different parameters. The following data structures are used as
supplementary parameters to describe the asymmetric algorithm sessions.

For the RSA algorithm, the extra parameters are as follows:
\begin{lstlisting}
struct virtio_crypto_rsa_session_para {
#define VIRTIO_CRYPTO_RSA_RAW_PADDING   0
#define VIRTIO_CRYPTO_RSA_PKCS1_PADDING 1
    le32 padding_algo;

#define VIRTIO_CRYPTO_RSA_NO_HASH   0
#define VIRTIO_CRYPTO_RSA_MD2       1
#define VIRTIO_CRYPTO_RSA_MD3       2
#define VIRTIO_CRYPTO_RSA_MD4       3
#define VIRTIO_CRYPTO_RSA_MD5       4
#define VIRTIO_CRYPTO_RSA_SHA1      5
#define VIRTIO_CRYPTO_RSA_SHA256    6
#define VIRTIO_CRYPTO_RSA_SHA384    7
#define VIRTIO_CRYPTO_RSA_SHA512    8
#define VIRTIO_CRYPTO_RSA_SHA224    9
    le32 hash_algo;
};
\end{lstlisting}

\field{padding_algo} specifies the padding method used by RSA sessions.
\begin{itemize*}
\item If VIRTIO_CRYPTO_RSA_RAW_PADDING is specified, 1) \field{hash_algo}
is ignored, 2) ciphertext and plaintext MUST be padded with leading zeros,
3) and RSA sessions with VIRTIO_CRYPTO_RSA_RAW_PADDING MUST not be used
for verification and signing operations.
\item If VIRTIO_CRYPTO_RSA_PKCS1_PADDING is specified, EMSA-PKCS1-v1_5 padding method
is used (see \hyperref[intro:rfc3447]{PKCS\#1}), \field{hash_algo} specifies how the
digest of the data passed to RSA sessions is calculated when verifying and signing.
It only affects the padding algorithm and is ignored during encryption and decryption.
\end{itemize*}

The ECC algorithms such as the ECDSA algorithm, cannot use custom curves, only the
following known curves can be used (see \hyperref[intro:NIST]{NIST-recommended curves}).

\begin{lstlisting}
#define VIRTIO_CRYPTO_CURVE_UNKNOWN   0
#define VIRTIO_CRYPTO_CURVE_NIST_P192 1
#define VIRTIO_CRYPTO_CURVE_NIST_P224 2
#define VIRTIO_CRYPTO_CURVE_NIST_P256 3
#define VIRTIO_CRYPTO_CURVE_NIST_P384 4
#define VIRTIO_CRYPTO_CURVE_NIST_P521 5
\end{lstlisting}

For the ECDSA algorithm, the extra parameters are as follows:
\begin{lstlisting}
struct virtio_crypto_ecdsa_session_para {
    /* See VIRTIO_CRYPTO_CURVE_* above */
    le32 curve_id;
};
\end{lstlisting}

The fixed-length and the variable-length parameters of AKCIPHER session requests are as follows:
\begin{lstlisting}
struct virtio_crypto_akcipher_create_session_flf {
    /* Device read only portion */

    /* See VIRTIO_CRYPTO_AKCIPHER_* above */
    le32 algo;
#define VIRTIO_CRYPTO_AKCIPHER_KEY_TYPE_PUBLIC 1
#define VIRTIO_CRYPTO_AKCIPHER_KEY_TYPE_PRIVATE 2
    le32 key_type;
    /* length of key */
    le32 key_len;

#define VIRTIO_CRYPTO_AKCIPHER_SESS_ALGO_SPEC_HDR_SIZE 44
    u8 algo_flf[VIRTIO_CRYPTO_AKCIPHER_SESS_ALGO_SPEC_HDR_SIZE];
};

struct virtio_crypto_akcipher_create_session_vlf {
    /* Device read only portion */
    u8 key[key_len];
};
\end{lstlisting}

\field{algo} decides the type used by \field{algo_flf}.
\field{algo_flf} is fixed to 44 bytes and MUST contains of be one the
following structures:
\begin{itemize*}
\item struct virtio_crypto_rsa_session_para
\item struct virtio_crypto_ecdsa_session_para
\end{itemize*}

The length of \field{key} is specified in \field{key_len} in the struct
virtio_crypto_akcipher_create_session_flf.

For the RSA algorithm, the key needs to be encoded according to
\hyperref[intro:rfc3447]{PKCS\#1}. The private key is described with the
RSAPrivateKey structure, and the public key is described with the RSAPublicKey
structure. These ASN.1 structures are encoded in DER encoding rules (see
\hyperref[intro:rfc6025]{rfc6025}).

\begin{lstlisting}
RSAPrivateKey ::= SEQUENCE {
    version          INTEGER,
    modulus          INTEGER,
    publicExponent   INTEGER,
    privateExponent  INTEGER,
    prime1           INTEGER,
    prime2           INTEGER,
    exponent1        INTEGER,
    exponent1        INTEGER,
    coefficient      INTEGER,
    otherPrimeInfos  OtherPrimeInfos OPTIONAL
}

OtherPrimeInfos ::= SEQUENCE SIZE(1...MAX) OF OtherPrimeInfo

OtherPrimeINfo ::= SEQUENCE {
    prime           INTEGER,
    exponent        INTEGER,
    coefficient     INTEGER
}

RSAPublicKey ::= SEQUENCE {
    modulus         INTEGER,
    publicExponent  INTEGER
}
\end{lstlisting}

For the ECDSA algorithm, the private key is encoded according to
\hyperref[intro:rfc5915]{RFC5915}, the private key of the ECDSA algorithm
is described by the ASN.1 structure ECPrivateKey and encoded with DER
encoding rules (see \hyperref[intro:rfc6025]{rfc6025}).

\begin{lstlisting}
ECPrivateKey ::= SEQUNCE {
    version         INTEGER,
    privateKey      OCTET STRING,
    parameters [0]  ECParameters {{ NamedCurve }} OPTIONAL,
    publicKey  [1]  BIT STRING OPTIONAL
}
\end{lstlisting}

The public key of the ECDSA algorithm is encoded according to \hyperref[intro:SEC1]{SEC1},
and the public key of ECDSA is described by the ASN.1 structure ECPoint.
When initializing a session with ECDSA public key, the ECPoint is DER encoded and the
\field{key} only contains the value part of ECPoint, that is, the header part of the
OCTET STRING will be omitted (see \hyperref[intro:rfc6025]{rfc6025}).

\begin{lstlisting}
ECPoint ::= OCTET STRING
\end{lstlisting}

The length of \field{key} is specified in \field{key_len} in
struct virtio_crypto_akcipher_create_session_flf.

\drivernormative{\subparagraph}{Session operation: create session}{Device Types / Crypto Device / Device
Operation / Control Virtqueue / Session operation / Session operation: create session}

\begin{itemize*}
\item The driver MUST set the \field{opcode} field based on service type: CIPHER, HASH, MAC, AEAD or AKCIPHER.
\item The driver MUST set the control general header, the opcode specific header,
    the opcode specific extra parameters and the opcode specific outcome buffer in turn.
    See \ref{sec:Device Types / Crypto Device / Device Operation / Control Virtqueue}.
\item The driver MUST set the \field{reversed} field to zero.
\end{itemize*}

\devicenormative{\subparagraph}{Session operation: create session}{Device Types / Crypto Device / Device
Operation / Control Virtqueue / Session operation / Session operation: create session}

\begin{itemize*}
\item The device MUST use the corresponding opcode specific structure according to the
    \field{opcode} in the control general header.
\item The device MUST extract extra parameters according to the structures used.
\item The device MUST set the \field{status} field to one of the following values of enum
    VIRTIO_CRYPTO_STATUS after finish a session creation:
\begin{itemize*}
\item VIRTIO_CRYPTO_OK if a session is created successfully.
\item VIRTIO_CRYPTO_NOTSUPP if the requested algorithm or operation is unsupported.
\item VIRTIO_CRYPTO_NOSPC if no free session ID (only when the VIRTIO_CRYPTO_F_REVISION_1
    feature bit is negotiated).
\item VIRTIO_CRYPTO_ERR if failure not mentioned above occurs.
\end{itemize*}
\item The device MUST set the \field{session_id} field to a unique session identifier only
    if the status is set to VIRTIO_CRYPTO_OK.
\end{itemize*}

\drivernormative{\subparagraph}{Session operation: destroy session}{Device Types / Crypto Device / Device
Operation / Control Virtqueue / Session operation / Session operation: destroy session}

\begin{itemize*}
\item The driver MUST set the \field{opcode} field based on service type: CIPHER, HASH, MAC, AEAD or AKCIPHER.
\item The driver MUST set the \field{session_id} to a valid value assigned by the device
    when the session was created.
\end{itemize*}

\devicenormative{\subparagraph}{Session operation: destroy session}{Device Types / Crypto Device / Device
Operation / Control Virtqueue / Session operation / Session operation: destroy session}

\begin{itemize*}
\item The device MUST set the \field{status} field to one of the following values of enum VIRTIO_CRYPTO_STATUS.
\begin{itemize*}
\item VIRTIO_CRYPTO_OK if a session is created successfully.
\item VIRTIO_CRYPTO_ERR if any failure occurs.
\end{itemize*}
\end{itemize*}


\subsubsection{Data Virtqueue}\label{sec:Device Types / Crypto Device / Device Operation / Data Virtqueue}

The driver uses the data virtqueues to transmit crypto operation requests to the device,
and completes the crypto operations.

The header for dataq is as follows:

\begin{lstlisting}
struct virtio_crypto_op_header {
#define VIRTIO_CRYPTO_CIPHER_ENCRYPT \
    VIRTIO_CRYPTO_OPCODE(VIRTIO_CRYPTO_SERVICE_CIPHER, 0x00)
#define VIRTIO_CRYPTO_CIPHER_DECRYPT \
    VIRTIO_CRYPTO_OPCODE(VIRTIO_CRYPTO_SERVICE_CIPHER, 0x01)
#define VIRTIO_CRYPTO_HASH \
    VIRTIO_CRYPTO_OPCODE(VIRTIO_CRYPTO_SERVICE_HASH, 0x00)
#define VIRTIO_CRYPTO_MAC \
    VIRTIO_CRYPTO_OPCODE(VIRTIO_CRYPTO_SERVICE_MAC, 0x00)
#define VIRTIO_CRYPTO_AEAD_ENCRYPT \
    VIRTIO_CRYPTO_OPCODE(VIRTIO_CRYPTO_SERVICE_AEAD, 0x00)
#define VIRTIO_CRYPTO_AEAD_DECRYPT \
    VIRTIO_CRYPTO_OPCODE(VIRTIO_CRYPTO_SERVICE_AEAD, 0x01)
#define VIRTIO_CRYPTO_AKCIPHER_ENCRYPT \
    VIRTIO_CRYPTO_OPCODE(VIRTIO_CRYPTO_SERVICE_AKCIPHER, 0x00)
#define VIRTIO_CRYPTO_AKCIPHER_DECRYPT \
    VIRTIO_CRYPTO_OPCODE(VIRTIO_CRYPTO_SERVICE_AKCIPHER, 0x01)
#define VIRTIO_CRYPTO_AKCIPHER_SIGN \
    VIRTIO_CRYPTO_OPCODE(VIRTIO_CRYPTO_SERVICE_AKCIPHER, 0x02)
#define VIRTIO_CRYPTO_AKCIPHER_VERIFY \
    VIRTIO_CRYPTO_OPCODE(VIRTIO_CRYPTO_SERVICE_AKCIPHER, 0x03)
    le32 opcode;
    /* algo should be service-specific algorithms */
    le32 algo;
    le64 session_id;
#define VIRTIO_CRYPTO_FLAG_SESSION_MODE 1
    /* control flag to control the request */
    le32 flag;
    le32 padding;
};
\end{lstlisting}

\begin{note}
If VIRTIO_CRYPTO_F_REVISION_1 is not negotiated the \field{flag} is ignored.

If VIRTIO_CRYPTO_F_REVISION_1 is negotiated but VIRTIO_CRYPTO_F_<SERVICE>_STATELESS_MODE
is not negotiated, then the device SHOULD reject <SERVICE> requests if
VIRTIO_CRYPTO_FLAG_SESSION_MODE is not set (in \field{flag}).
\end{note}

The dataq request is composed of four parts:
\begin{lstlisting}
struct virtio_crypto_op_data_req {
    /* Device read only portion */

    struct virtio_crypto_op_header header;

#define VIRTIO_CRYPTO_DATAQ_OP_SPEC_HDR_LEGACY 48
    /* fixed length fields, opcode specific */
    u8 op_flf[flf_len];

    /* Device read && write portion */
    /* variable length fields, opcode specific */
    u8 op_vlf[vlf_len];

    /* Device write only portion */
    struct virtio_crypto_inhdr inhdr;
};
\end{lstlisting}

\field{header} is a general header (see above).

\field{op_flf} is the opcode (in \field{header}) specific header.

\field{flf_len} depends on the VIRTIO_CRYPTO_F_REVISION_1 feature bit
(see below).

\field{op_vlf} is the opcode (in \field{header}) specific parameters.

\field{vlf_len} is the size of the specific structure used.

\begin{itemize*}
\item If the the opcode (in \field{header}) is VIRTIO_CRYPTO_CIPHER_ENCRYPT
    or VIRTIO_CRYPTO_CIPHER_DECRYPT then:
    \begin{itemize*}
    \item If VIRTIO_CRYPTO_F_CIPHER_STATELESS_MODE is negotiated, \field{op_flf} is
        struct virtio_crypto_sym_data_flf_stateless, and \field{op_vlf} is struct
        virtio_crypto_sym_data_vlf_stateless.
    \item If VIRTIO_CRYPTO_F_CIPHER_STATELESS_MODE is NOT negotiated, \field{op_flf}
        is struct virtio_crypto_sym_data_flf if VIRTIO_CRYPTO_F_REVISION_1 is negotiated
        and struct virtio_crypto_sym_data_flf is padded to 48 bytes if NOT negotiated,
        and \field{op_vlf} is struct virtio_crypto_sym_data_vlf.
    \end{itemize*}
\item If the the opcode (in \field{header}) is VIRTIO_CRYPTO_HASH:
    \begin{itemize*}
    \item If VIRTIO_CRYPTO_F_HASH_STATELESS_MODE is negotiated, \field{op_flf} is
        struct virtio_crypto_hash_data_flf_stateless, and \field{op_vlf} is struct
        virtio_crypto_hash_data_vlf_stateless.
    \item If VIRTIO_CRYPTO_F_HASH_STATELESS_MODE is NOT negotiated, \field{op_flf}
        is struct virtio_crypto_hash_data_flf if VIRTIO_CRYPTO_F_REVISION_1 is negotiated
        and struct virtio_crypto_hash_data_flf is padded to 48 bytes if NOT negotiated,
        and \field{op_vlf} is struct virtio_crypto_hash_data_vlf.
    \end{itemize*}
\item If the the opcode (in \field{header}) is VIRTIO_CRYPTO_MAC:
    \begin{itemize*}
    \item If VIRTIO_CRYPTO_F_MAC_STATELESS_MODE is negotiated, \field{op_flf} is
        struct virtio_crypto_mac_data_flf_stateless, and \field{op_vlf} is struct
        virtio_crypto_mac_data_vlf_stateless.
    \item If VIRTIO_CRYPTO_F_MAC_STATELESS_MODE is NOT negotiated, \field{op_flf}
        is struct virtio_crypto_mac_data_flf if VIRTIO_CRYPTO_F_REVISION_1 is negotiated
        and struct virtio_crypto_mac_data_flf is padded to 48 bytes if NOT negotiated,
        and \field{op_vlf} is struct virtio_crypto_mac_data_vlf.
    \end{itemize*}
\item If the the opcode (in \field{header}) is VIRTIO_CRYPTO_AEAD_ENCRYPT
    or VIRTIO_CRYPTO_AEAD_DECRYPT then:
    \begin{itemize*}
    \item If VIRTIO_CRYPTO_F_AEAD_STATELESS_MODE is negotiated, \field{op_flf} is
        struct virtio_crypto_aead_data_flf_stateless, and \field{op_vlf} is struct
        virtio_crypto_aead_data_vlf_stateless.
    \item If VIRTIO_CRYPTO_F_AEAD_STATELESS_MODE is NOT negotiated, \field{op_flf}
        is struct virtio_crypto_aead_data_flf if VIRTIO_CRYPTO_F_REVISION_1 is negotiated
        and struct virtio_crypto_aead_data_flf is padded to 48 bytes if NOT negotiated,
        and \field{op_vlf} is struct virtio_crypto_aead_data_vlf.
    \end{itemize*}
\item If the opcode (in \field{header}) is VIRTIO_CRYPTO_AKCIPHER_ENCRYPT, VIRTIO_CRYPTO_AKCIPHER_DECRYPT,
    VIRTIO_CRYPTO_AKCIPHER_SIGN or VIRTIO_CRYPTO_AKCIPHER_VERIFY then:
    \begin{itemize*}
    \item If VIRTIO_CRYPTO_F_AKCIPHER_STATELESS_MODE is negotiated, \field{op_flf} is
        struct virtio_crypto_akcipher_data_flf_statless, and \field{op_vlf} is struct
        virtio_crypto_akcipher_data_vlf_stateless.
    \item If VIRTIO_CRYPTO_F_AKCIPHER_STATELESS_MODE is NOT negotiated, \field{op_flf}
        is struct virtio_crypto_akcipher_data_flf if VIRTIO_CRYPTO_F_REVISION_1 is negotiated
        and struct virtio_crypto_akcipher_data_flf is padded to 48 bytes if NOT negotiated,
        and \field{op_vlf} is struct virtio_crypto_akcipher_data_vlf.
    \end{itemize*}
\end{itemize*}

\field{inhdr} is a unified input header that used to return the status of
the operations, is defined as follows:

\begin{lstlisting}
struct virtio_crypto_inhdr {
    u8 status;
};
\end{lstlisting}

\subsubsection{HASH Service Operation}\label{sec:Device Types / Crypto Device / Device Operation / HASH Service Operation}

Session mode HASH service requests are as follows:

\begin{lstlisting}
struct virtio_crypto_hash_data_flf {
    /* length of source data */
    le32 src_data_len;
    /* hash result length */
    le32 hash_result_len;
};

struct virtio_crypto_hash_data_vlf {
    /* Device read only portion */
    /* Source data */
    u8 src_data[src_data_len];

    /* Device write only portion */
    /* Hash result data */
    u8 hash_result[hash_result_len];
};
\end{lstlisting}

Each data request uses the virtio_crypto_hash_data_flf structure and the
virtio_crypto_hash_data_vlf structure to store information used to run the
HASH operations.

\field{src_data} is the source data that will be processed.
\field{src_data_len} is the length of source data.
\field{hash_result} is the result data and \field{hash_result_len} is the length
of it.

Stateless mode HASH service requests are as follows:

\begin{lstlisting}
struct virtio_crypto_hash_data_flf_stateless {
    struct {
        /* See VIRTIO_CRYPTO_HASH_* above */
        le32 algo;
    } sess_para;

    /* length of source data */
    le32 src_data_len;
    /* hash result length */
    le32 hash_result_len;
    le32 reserved;
};
struct virtio_crypto_hash_data_vlf_stateless {
    /* Device read only portion */
    /* Source data */
    u8 src_data[src_data_len];

    /* Device write only portion */
    /* Hash result data */
    u8 hash_result[hash_result_len];
};
\end{lstlisting}

\drivernormative{\paragraph}{HASH Service Operation}{Device Types / Crypto Device / Device Operation / HASH Service Operation}

\begin{itemize*}
\item If the driver uses the session mode, then the driver MUST set \field{session_id}
    in struct virtio_crypto_op_header to a valid value assigned by the device when the
    session was created.
\item If the VIRTIO_CRYPTO_F_HASH_STATELESS_MODE feature bit is negotiated, 1) if the
    driver uses the stateless mode, then the driver MUST set the \field{flag} field in
    struct virtio_crypto_op_header to ZERO and MUST set the fields in struct
    virtio_crypto_hash_data_flf_stateless.sess_para, 2) if the driver uses the session
    mode, then the driver MUST set the \field{flag} field in struct virtio_crypto_op_header
    to VIRTIO_CRYPTO_FLAG_SESSION_MODE.
\item The driver MUST set \field{opcode} in struct virtio_crypto_op_header to VIRTIO_CRYPTO_HASH.
\end{itemize*}

\devicenormative{\paragraph}{HASH Service Operation}{Device Types / Crypto Device / Device Operation / HASH Service Operation}

\begin{itemize*}
\item The device MUST use the corresponding structure according to the \field{opcode}
    in the data general header.
\item If the VIRTIO_CRYPTO_F_HASH_STATELESS_MODE feature bit is negotiated, the device
    MUST parse \field{flag} field in struct virtio_crypto_op_header in order to decide
    which mode the driver uses.
\item The device MUST copy the results of HASH operations in the hash_result[] if HASH
    operations success.
\item The device MUST set \field{status} in struct virtio_crypto_inhdr to one of the
    following values of enum VIRTIO_CRYPTO_STATUS:
\begin{itemize*}
\item VIRTIO_CRYPTO_OK if the operation success.
\item VIRTIO_CRYPTO_NOTSUPP if the requested algorithm or operation is unsupported.
\item VIRTIO_CRYPTO_INVSESS if the session ID invalid when in session mode.
\item VIRTIO_CRYPTO_ERR if any failure not mentioned above occurs.
\end{itemize*}
\end{itemize*}


\subsubsection{MAC Service Operation}\label{sec:Device Types / Crypto Device / Device Operation / MAC Service Operation}

Session mode MAC service requests are as follows:

\begin{lstlisting}
struct virtio_crypto_mac_data_flf {
    struct virtio_crypto_hash_data_flf hdr;
};

struct virtio_crypto_mac_data_vlf {
    /* Device read only portion */
    /* Source data */
    u8 src_data[src_data_len];

    /* Device write only portion */
    /* Hash result data */
    u8 hash_result[hash_result_len];
};
\end{lstlisting}

Each request uses the virtio_crypto_mac_data_flf structure and the
virtio_crypto_mac_data_vlf structure to store information used to run the
MAC operations.

\field{src_data} is the source data that will be processed.
\field{src_data_len} is the length of source data.
\field{hash_result} is the result data and \field{hash_result_len} is the length
of it.

Stateless mode MAC service requests are as follows:

\begin{lstlisting}
struct virtio_crypto_mac_data_flf_stateless {
    struct {
        /* See VIRTIO_CRYPTO_MAC_* above */
        le32 algo;
        /* length of authenticated key */
        le32 auth_key_len;
    } sess_para;

    /* length of source data */
    le32 src_data_len;
    /* hash result length */
    le32 hash_result_len;
};

struct virtio_crypto_mac_data_vlf_stateless {
    /* Device read only portion */
    /* The authenticated key */
    u8 auth_key[auth_key_len];
    /* Source data */
    u8 src_data[src_data_len];

    /* Device write only portion */
    /* Hash result data */
    u8 hash_result[hash_result_len];
};
\end{lstlisting}

\field{auth_key} is the authenticated key that will be used during the process.
\field{auth_key_len} is the length of the key.

\drivernormative{\paragraph}{MAC Service Operation}{Device Types / Crypto Device / Device Operation / MAC Service Operation}

\begin{itemize*}
\item If the driver uses the session mode, then the driver MUST set \field{session_id}
    in struct virtio_crypto_op_header to a valid value assigned by the device when the
    session was created.
\item If the VIRTIO_CRYPTO_F_MAC_STATELESS_MODE feature bit is negotiated, 1) if the
    driver uses the stateless mode, then the driver MUST set the \field{flag} field
    in struct virtio_crypto_op_header to ZERO and MUST set the fields in struct
    virtio_crypto_mac_data_flf_stateless.sess_para, 2) if the driver uses the session
    mode, then the driver MUST set the \field{flag} field in struct virtio_crypto_op_header
    to VIRTIO_CRYPTO_FLAG_SESSION_MODE.
\item The driver MUST set \field{opcode} in struct virtio_crypto_op_header to VIRTIO_CRYPTO_MAC.
\end{itemize*}

\devicenormative{\paragraph}{MAC Service Operation}{Device Types / Crypto Device / Device Operation / MAC Service Operation}

\begin{itemize*}
\item If the VIRTIO_CRYPTO_F_MAC_STATELESS_MODE feature bit is negotiated, the device
    MUST parse \field{flag} field in struct virtio_crypto_op_header in order to decide
	which mode the driver uses.
\item The device MUST copy the results of MAC operations in the hash_result[] if HASH
    operations success.
\item The device MUST set \field{status} in struct virtio_crypto_inhdr to one of the
    following values of enum VIRTIO_CRYPTO_STATUS:
\begin{itemize*}
\item VIRTIO_CRYPTO_OK if the operation success.
\item VIRTIO_CRYPTO_NOTSUPP if the requested algorithm or operation is unsupported.
\item VIRTIO_CRYPTO_INVSESS if the session ID invalid when in session mode.
\item VIRTIO_CRYPTO_ERR if any failure not mentioned above occurs.
\end{itemize*}
\end{itemize*}

\subsubsection{Symmetric algorithms Operation}\label{sec:Device Types / Crypto Device / Device Operation / Symmetric algorithms Operation}

Session mode CIPHER service requests are as follows:

\begin{lstlisting}
struct virtio_crypto_cipher_data_flf {
    /*
     * Byte Length of valid IV/Counter data pointed to by the below iv data.
     *
     * For block ciphers in CBC or F8 mode, or for Kasumi in F8 mode, or for
     *   SNOW3G in UEA2 mode, this is the length of the IV (which
     *   must be the same as the block length of the cipher).
     * For block ciphers in CTR mode, this is the length of the counter
     *   (which must be the same as the block length of the cipher).
     */
    le32 iv_len;
    /* length of source data */
    le32 src_data_len;
    /* length of destination data */
    le32 dst_data_len;
    le32 padding;
};

struct virtio_crypto_cipher_data_vlf {
    /* Device read only portion */

    /*
     * Initialization Vector or Counter data.
     *
     * For block ciphers in CBC or F8 mode, or for Kasumi in F8 mode, or for
     *   SNOW3G in UEA2 mode, this is the Initialization Vector (IV)
     *   value.
     * For block ciphers in CTR mode, this is the counter.
     * For AES-XTS, this is the 128bit tweak, i, from IEEE Std 1619-2007.
     *
     * The IV/Counter will be updated after every partial cryptographic
     * operation.
     */
    u8 iv[iv_len];
    /* Source data */
    u8 src_data[src_data_len];

    /* Device write only portion */
    /* Destination data */
    u8 dst_data[dst_data_len];
};
\end{lstlisting}

Session mode requests of algorithm chaining are as follows:

\begin{lstlisting}
struct virtio_crypto_alg_chain_data_flf {
    le32 iv_len;
    /* Length of source data */
    le32 src_data_len;
    /* Length of destination data */
    le32 dst_data_len;
    /* Starting point for cipher processing in source data */
    le32 cipher_start_src_offset;
    /* Length of the source data that the cipher will be computed on */
    le32 len_to_cipher;
    /* Starting point for hash processing in source data */
    le32 hash_start_src_offset;
    /* Length of the source data that the hash will be computed on */
    le32 len_to_hash;
    /* Length of the additional auth data */
    le32 aad_len;
    /* Length of the hash result */
    le32 hash_result_len;
    le32 reserved;
};

struct virtio_crypto_alg_chain_data_vlf {
    /* Device read only portion */

    /* Initialization Vector or Counter data */
    u8 iv[iv_len];
    /* Source data */
    u8 src_data[src_data_len];
    /* Additional authenticated data if exists */
    u8 aad[aad_len];

    /* Device write only portion */

    /* Destination data */
    u8 dst_data[dst_data_len];
    /* Hash result data */
    u8 hash_result[hash_result_len];
};
\end{lstlisting}

Session mode requests of symmetric algorithm are as follows:

\begin{lstlisting}
struct virtio_crypto_sym_data_flf {
    /* Device read only portion */

#define VIRTIO_CRYPTO_SYM_DATA_REQ_HDR_SIZE    40
    u8 op_type_flf[VIRTIO_CRYPTO_SYM_DATA_REQ_HDR_SIZE];

    /* See above VIRTIO_CRYPTO_SYM_OP_* */
    le32 op_type;
    le32 padding;
};

struct virtio_crypto_sym_data_vlf {
    u8 op_type_vlf[sym_para_len];
};
\end{lstlisting}

Each request uses the virtio_crypto_sym_data_flf structure and the
virtio_crypto_sym_data_flf structure to store information used to run the
CIPHER operations.

\field{op_type_flf} is the \field{op_type} specific header, it MUST starts
with or be one of the following structures:
\begin{itemize*}
\item struct virtio_crypto_cipher_data_flf
\item struct virtio_crypto_alg_chain_data_flf
\end{itemize*}

The length of \field{op_type_flf} is fixed to 40 bytes, the data of unused
part (if has) will be ignored.

\field{op_type_vlf} is the \field{op_type} specific parameters, it MUST starts
with or be one of the following structures:
\begin{itemize*}
\item struct virtio_crypto_cipher_data_vlf
\item struct virtio_crypto_alg_chain_data_vlf
\end{itemize*}

\field{sym_para_len} is the size of the specific structure used.

Stateless mode CIPHER service requests are as follows:

\begin{lstlisting}
struct virtio_crypto_cipher_data_flf_stateless {
    struct {
        /* See VIRTIO_CRYPTO_CIPHER* above */
        le32 algo;
        /* length of key */
        le32 key_len;

        /* See VIRTIO_CRYPTO_OP_* above */
        le32 op;
    } sess_para;

    /*
     * Byte Length of valid IV/Counter data pointed to by the below iv data.
     */
    le32 iv_len;
    /* length of source data */
    le32 src_data_len;
    /* length of destination data */
    le32 dst_data_len;
};

struct virtio_crypto_cipher_data_vlf_stateless {
    /* Device read only portion */

    /* The cipher key */
    u8 cipher_key[key_len];

    /* Initialization Vector or Counter data. */
    u8 iv[iv_len];
    /* Source data */
    u8 src_data[src_data_len];

    /* Device write only portion */
    /* Destination data */
    u8 dst_data[dst_data_len];
};
\end{lstlisting}

Stateless mode requests of algorithm chaining are as follows:

\begin{lstlisting}
struct virtio_crypto_alg_chain_data_flf_stateless {
    struct {
        /* See VIRTIO_CRYPTO_SYM_ALG_CHAIN_ORDER_* above */
        le32 alg_chain_order;
        /* length of the additional authenticated data in bytes */
        le32 aad_len;

        struct {
            /* See VIRTIO_CRYPTO_CIPHER* above */
            le32 algo;
            /* length of key */
            le32 key_len;
            /* See VIRTIO_CRYPTO_OP_* above */
            le32 op;
        } cipher;

        struct {
            /* See VIRTIO_CRYPTO_HASH_* or VIRTIO_CRYPTO_MAC_* above */
            le32 algo;
            /* length of authenticated key */
            le32 auth_key_len;
            /* See VIRTIO_CRYPTO_SYM_HASH_MODE_* above */
            le32 hash_mode;
        } hash;
    } sess_para;

    le32 iv_len;
    /* Length of source data */
    le32 src_data_len;
    /* Length of destination data */
    le32 dst_data_len;
    /* Starting point for cipher processing in source data */
    le32 cipher_start_src_offset;
    /* Length of the source data that the cipher will be computed on */
    le32 len_to_cipher;
    /* Starting point for hash processing in source data */
    le32 hash_start_src_offset;
    /* Length of the source data that the hash will be computed on */
    le32 len_to_hash;
    /* Length of the additional auth data */
    le32 aad_len;
    /* Length of the hash result */
    le32 hash_result_len;
    le32 reserved;
};

struct virtio_crypto_alg_chain_data_vlf_stateless {
    /* Device read only portion */

    /* The cipher key */
    u8 cipher_key[key_len];
    /* The auth key */
    u8 auth_key[auth_key_len];
    /* Initialization Vector or Counter data */
    u8 iv[iv_len];
    /* Additional authenticated data if exists */
    u8 aad[aad_len];
    /* Source data */
    u8 src_data[src_data_len];

    /* Device write only portion */

    /* Destination data */
    u8 dst_data[dst_data_len];
    /* Hash result data */
    u8 hash_result[hash_result_len];
};
\end{lstlisting}

Stateless mode requests of symmetric algorithm are as follows:

\begin{lstlisting}
struct virtio_crypto_sym_data_flf_stateless {
    /* Device read only portion */
#define VIRTIO_CRYPTO_SYM_DATE_REQ_HDR_STATELESS_SIZE    72
    u8 op_type_flf[VIRTIO_CRYPTO_SYM_DATE_REQ_HDR_STATELESS_SIZE];

    /* Device write only portion */
    /* See above VIRTIO_CRYPTO_SYM_OP_* */
    le32 op_type;
};

struct virtio_crypto_sym_data_vlf_stateless {
    u8 op_type_vlf[sym_para_len];
};
\end{lstlisting}

\field{op_type_flf} is the \field{op_type} specific header, it MUST starts
with or be one of the following structures:
\begin{itemize*}
\item struct virtio_crypto_cipher_data_flf_stateless
\item struct virtio_crypto_alg_chain_data_flf_stateless
\end{itemize*}

The length of \field{op_type_flf} is fixed to 72 bytes, the data of unused
part (if has) will be ignored.

\field{op_type_vlf} is the \field{op_type} specific parameters, it MUST starts
with or be one of the following structures:
\begin{itemize*}
\item struct virtio_crypto_cipher_data_vlf_stateless
\item struct virtio_crypto_alg_chain_data_vlf_stateless
\end{itemize*}

\field{sym_para_len} is the size of the specific structure used.

\drivernormative{\paragraph}{Symmetric algorithms Operation}{Device Types / Crypto Device / Device Operation / Symmetric algorithms Operation}

\begin{itemize*}
\item If the driver uses the session mode, then the driver MUST set \field{session_id}
    in struct virtio_crypto_op_header to a valid value assigned by the device when the
    session was created.
\item If the VIRTIO_CRYPTO_F_CIPHER_STATELESS_MODE feature bit is negotiated, 1) if the
    driver uses the stateless mode, then the driver MUST set the \field{flag} field in
    struct virtio_crypto_op_header to ZERO and MUST set the fields in struct
    virtio_crypto_cipher_data_flf_stateless.sess_para or struct
    virtio_crypto_alg_chain_data_flf_stateless.sess_para, 2) if the driver uses the
    session mode, then the driver MUST set the \field{flag} field in struct
    virtio_crypto_op_header to VIRTIO_CRYPTO_FLAG_SESSION_MODE.
\item The driver MUST set the \field{opcode} field in struct virtio_crypto_op_header
    to VIRTIO_CRYPTO_CIPHER_ENCRYPT or VIRTIO_CRYPTO_CIPHER_DECRYPT.
\item The driver MUST specify the fields of struct virtio_crypto_cipher_data_flf in
    struct virtio_crypto_sym_data_flf and struct virtio_crypto_cipher_data_vlf in
    struct virtio_crypto_sym_data_vlf if the request is based on VIRTIO_CRYPTO_SYM_OP_CIPHER.
\item The driver MUST specify the fields of struct virtio_crypto_alg_chain_data_flf
    in struct virtio_crypto_sym_data_flf and struct virtio_crypto_alg_chain_data_vlf
    in struct virtio_crypto_sym_data_vlf if the request is of the VIRTIO_CRYPTO_SYM_OP_ALGORITHM_CHAINING
    type.
\end{itemize*}

\devicenormative{\paragraph}{Symmetric algorithms Operation}{Device Types / Crypto Device / Device Operation / Symmetric algorithms Operation}

\begin{itemize*}
\item If the VIRTIO_CRYPTO_F_CIPHER_STATELESS_MODE feature bit is negotiated, the device
    MUST parse \field{flag} field in struct virtio_crypto_op_header in order to decide
	which mode the driver uses.
\item The device MUST parse the virtio_crypto_sym_data_req based on the \field{opcode}
    field in general header.
\item The device MUST parse the fields of struct virtio_crypto_cipher_data_flf in
    struct virtio_crypto_sym_data_flf and struct virtio_crypto_cipher_data_vlf in
    struct virtio_crypto_sym_data_vlf if the request is based on VIRTIO_CRYPTO_SYM_OP_CIPHER.
\item The device MUST parse the fields of struct virtio_crypto_alg_chain_data_flf
    in struct virtio_crypto_sym_data_flf and struct virtio_crypto_alg_chain_data_vlf
    in struct virtio_crypto_sym_data_vlf if the request is of the VIRTIO_CRYPTO_SYM_OP_ALGORITHM_CHAINING
    type.
\item The device MUST copy the result of cryptographic operation in the dst_data[] in
    both plain CIPHER mode and algorithms chain mode.
\item The device MUST check the \field{para}.\field{add_len} is bigger than 0 before
    parse the additional authenticated data in plain algorithms chain mode.
\item The device MUST copy the result of HASH/MAC operation in the hash_result[] is
    of the VIRTIO_CRYPTO_SYM_OP_ALGORITHM_CHAINING type.
\item The device MUST set the \field{status} field in struct virtio_crypto_inhdr to
    one of the following values of enum VIRTIO_CRYPTO_STATUS:
\begin{itemize*}
\item VIRTIO_CRYPTO_OK if the operation success.
\item VIRTIO_CRYPTO_NOTSUPP if the requested algorithm or operation is unsupported.
\item VIRTIO_CRYPTO_INVSESS if the session ID is invalid in session mode.
\item VIRTIO_CRYPTO_ERR if failure not mentioned above occurs.
\end{itemize*}
\end{itemize*}

\subsubsection{AEAD Service Operation}\label{sec:Device Types / Crypto Device / Device Operation / AEAD Service Operation}

Session mode requests of symmetric algorithm are as follows:

\begin{lstlisting}
struct virtio_crypto_aead_data_flf {
    /*
     * Byte Length of valid IV data.
     *
     * For GCM mode, this is either 12 (for 96-bit IVs) or 16, in which
     *   case iv points to J0.
     * For CCM mode, this is the length of the nonce, which can be in the
     *   range 7 to 13 inclusive.
     */
    le32 iv_len;
    /* length of additional auth data */
    le32 aad_len;
    /* length of source data */
    le32 src_data_len;
    /* length of dst data, this should be at least src_data_len + tag_len */
    le32 dst_data_len;
    /* Authentication tag length */
    le32 tag_len;
    le32 reserved;
};

struct virtio_crypto_aead_data_vlf {
    /* Device read only portion */

    /*
     * Initialization Vector data.
     *
     * For GCM mode, this is either the IV (if the length is 96 bits) or J0
     *   (for other sizes), where J0 is as defined by NIST SP800-38D.
     *   Regardless of the IV length, a full 16 bytes needs to be allocated.
     * For CCM mode, the first byte is reserved, and the nonce should be
     *   written starting at &iv[1] (to allow space for the implementation
     *   to write in the flags in the first byte).  Note that a full 16 bytes
     *   should be allocated, even though the iv_len field will have
     *   a value less than this.
     *
     * The IV will be updated after every partial cryptographic operation.
     */
    u8 iv[iv_len];
    /* Source data */
    u8 src_data[src_data_len];
    /* Additional authenticated data if exists */
    u8 aad[aad_len];

    /* Device write only portion */
    /* Pointer to output data */
    u8 dst_data[dst_data_len];
};
\end{lstlisting}

Each request uses the virtio_crypto_aead_data_flf structure and the
virtio_crypto_aead_data_flf structure to store information used to run the
AEAD operations.

Stateless mode AEAD service requests are as follows:

\begin{lstlisting}
struct virtio_crypto_aead_data_flf_stateless {
    struct {
        /* See VIRTIO_CRYPTO_AEAD_* above */
        le32 algo;
        /* length of key */
        le32 key_len;
        /* encrypt or decrypt, See above VIRTIO_CRYPTO_OP_* */
        le32 op;
    } sess_para;

    /* Byte Length of valid IV data. */
    le32 iv_len;
    /* Authentication tag length */
    le32 tag_len;
    /* length of additional auth data */
    le32 aad_len;
    /* length of source data */
    le32 src_data_len;
    /* length of dst data, this should be at least src_data_len + tag_len */
    le32 dst_data_len;
};

struct virtio_crypto_aead_data_vlf_stateless {
    /* Device read only portion */

    /* The cipher key */
    u8 key[key_len];
    /* Initialization Vector data. */
    u8 iv[iv_len];
    /* Source data */
    u8 src_data[src_data_len];
    /* Additional authenticated data if exists */
    u8 aad[aad_len];

    /* Device write only portion */
    /* Pointer to output data */
    u8 dst_data[dst_data_len];
};
\end{lstlisting}

\drivernormative{\paragraph}{AEAD Service Operation}{Device Types / Crypto Device / Device Operation / AEAD Service Operation}

\begin{itemize*}
\item If the driver uses the session mode, then the driver MUST set
    \field{session_id} in struct virtio_crypto_op_header to a valid value assigned
    by the device when the session was created.
\item If the VIRTIO_CRYPTO_F_AEAD_STATELESS_MODE feature bit is negotiated, 1) if
    the driver uses the stateless mode, then the driver MUST set the \field{flag}
    field in struct virtio_crypto_op_header to ZERO and MUST set the fields in
    struct virtio_crypto_aead_data_flf_stateless.sess_para, 2) if the driver uses
    the session mode, then the driver MUST set the \field{flag} field in struct
    virtio_crypto_op_header to VIRTIO_CRYPTO_FLAG_SESSION_MODE.
\item The driver MUST set the \field{opcode} field in struct virtio_crypto_op_header
    to VIRTIO_CRYPTO_AEAD_ENCRYPT or VIRTIO_CRYPTO_AEAD_DECRYPT.
\end{itemize*}

\devicenormative{\paragraph}{AEAD Service Operation}{Device Types / Crypto Device / Device Operation / AEAD Service Operation}

\begin{itemize*}
\item If the VIRTIO_CRYPTO_F_AEAD_STATELESS_MODE feature bit is negotiated, the
    device MUST parse the virtio_crypto_aead_data_vlf_stateless based on the \field{opcode}
	field in general header.
\item The device MUST copy the result of cryptographic operation in the dst_data[].
\item The device MUST copy the authentication tag in the dst_data[] offset the cipher result.
\item The device MUST set the \field{status} field in struct virtio_crypto_inhdr to
    one of the following values of enum VIRTIO_CRYPTO_STATUS:
\item When the \field{opcode} field is VIRTIO_CRYPTO_AEAD_DECRYPT, the device MUST
    verify and return the verification result to the driver.
\begin{itemize*}
\item VIRTIO_CRYPTO_OK if the operation success.
\item VIRTIO_CRYPTO_NOTSUPP if the requested algorithm or operation is unsupported.
\item VIRTIO_CRYPTO_BADMSG if the verification result is incorrect.
\item VIRTIO_CRYPTO_INVSESS if the session ID invalid when in session mode.
\item VIRTIO_CRYPTO_ERR if any failure not mentioned above occurs.
\end{itemize*}
\end{itemize*}

\subsubsection{AKCIPHER Service Operation}\label{sec:Device Types / Crypto Device / Device Operation / AKCIPHER Service Operation}

Session mode AKCIPHER requests are as follows:

\begin{lstlisting}
struct virtio_crypto_akcipher_data_flf {
    /* length of source data */
    le32 src_data_len;
    /* length of dst data */
    le32 dst_data_len;
};

struct virtio_crypto_akcipher_data_vlf {
    /* Device read only portion */
    /* Source data */
    u8 src_data[src_data_len];

    /* Device write only portion */
    /* Pointer to output data */
    u8 dst_data[dst_data_len];
};
\end{lstlisting}

Each data request uses the virtio_crypto_akcipher_flf structure and the virtio_crypto_akcipher_data_vlf
structure to store information used to run the AKCIPHER operations.

For encryption, decryption, and signing:
\field{src_data} is the source data that will be processed, note that for signing operations,
src_data stores the data to be signed, which usually is the digest of some data rather than the
data itself.
\field{src_data_len} is the length of source data.
\field{dst_result} is the result data and \field{dst_data_len} is the length of it. Note that the
length of the result is not always exactly equal to dst_data_len, the driver needs to check how
many bytes the device has written and calculate the actual length of the result.

For verification:
\field{src_data_len} refers to the length of the signature, and \field{dst_data_len} refers to
the length of signed data, where the signed data is usually the digest of some data.
\field{src_data} is spliced by the signature and the signed data, the src_data with the lower
address stores the signature, and the higher address stores the signed data.
\field{dst_data} is always empty for verification.

Different algorithms have different signature formats.
For the RSA algorithm, the result is determined by the padding algorithm specified by
\field{padding_algo} in structure virtio_crypto_rsa_session_para.

For the ECDSA algorithm, the signature is composed of the following
ASN.1 structure (see \hyperref[intro:rfc3279]{RFC3279})
and MUST be DER encoded (see \hyperref[intro:rfc6025]{rfc6025}).

\begin{lstlisting}
Ecdsa-Sig-Value ::= SEQUENCE {
    r INTEGER,
    s INTEGER
}
\end{lstlisting}

Stateless mode AKCIPHER service requests are as follows:
\begin{lstlisting}
struct virtio_crypto_akcipher_data_flf_stateless {
    struct {
        /* See VIRTIO_CYRPTO_AKCIPHER* above */
        le32 algo;
        /* See VIRTIO_CRYPTO_AKCIPHER_KEY_TYPE_* above */
        le32 key_type;
        /* length of key */
        le32 key_len;

        /* algothrim specific parameters described above */
        union {
            struct virtio_crypto_rsa_session_para rsa;
            struct virtio_crypto_ecdsa_session_para ecdsa;
        } u;
    } sess_para;

    /* length of source data */
    le32 src_data_len;
    /* length of destination data */
    le32 dst_data_len;
};

struct virtio_crypto_akcipher_data_vlf_stateless {
    /* Device read only portion */
    u8 akcipher_key[key_len];

    /* Source data */
    u8 src_data[src_data_len];

    /* Device write only portion */
    u8 dst_data[dst_data_len];
};
\end{lstlisting}

In stateless mode, the format of key and signature, the meaning of src_data and dst_data, are all the same
with session mode.

\drivernormative{\paragraph}{AKCIPHER Service Operation}{Device Types / Crypto Device / Device Operation / AKCIPHER Service Operation}

\begin{itemize*}
\item If the driver uses the session mode, then the driver MUST set
    \field{session_id} in struct virtio_crypto_op_header to a valid
    value assigned by the device when the session was created.
\item If the VIRTIO_CRYPTO_F_AKCIPHER_STATELESS_MODE feature bit is negotiated, 1) if the
    driver uses the stateless mode, then the driver MUST set the \field{flag} field in
    struct virtio_crypto_op_header to ZERO and MUST set the fields in struct
    virtio_crypto_akcipher_flf_stateless.sess_para, 2) if the driver uses the session
    mode, then the driver MUST set the \field{flag} field in struct virtio_crypto_op_header
    to VIRTIO_CRYPTO_FLAG_SESSION_MODE.
\item The driver MUST set the \field{opcode} field in struct virtio_crypto_op_header
    to one of VIRTIO_CRYPTO_AKCIPHER_ENCRYPT, VIRTIO_CRYPTO_AKCIPHER_DESTROY_SESSION,
    VIRTIO_CRYPTO_AKCIPHER_SIGN, and VIRTIO_CRYPTO_AKCIPHER_VERIFY.
\end{itemize*}

\devicenormative{\paragraph}{AKCIPHER Service Operation}{Device Types / Crypto Device / Device Operation / AKCIPHER Service Operation}

\begin{itemize*}
\item If the VIRTIO_CRYPTO_F_AKCIPHER_STATELESS_MODE feature bit is negotiated, the
    device MUST parse the virtio_crypto_akcipher_data_vlf_stateless based on the \field{opcode}
    field in general header.
\item The device MUST copy the result of cryptographic operation in the dst_data[].
\item The device MUST set the \field{status} field in struct virtio_crypto_inhdr to
    one of the following values of enum VIRTIO_CRYPTO_STATUS:
\begin{itemize*}
\item VIRTIO_CRYPTO_OK if the operation success.
\item VIRTIO_CRYPTO_NOTSUPP if the requested algorithm or operation is unsupported.
\item VIRTIO_CRYPTO_BADMSG if the verification result is incorrect.
\item VIRTIO_CRYPTO_INVSESS if the session ID invalid when in session mode.
\item VIRTIO_CRYPTO_KEY_REJECTED if the signature verification failed.
\item VIRTIO_CRYPTO_ERR if any failure not mentioned above occurs.
\end{itemize*}
\end{itemize*}

\section{Crypto Device}\label{sec:Device Types / Crypto Device}

The virtio crypto device is a virtual cryptography device as well as a
virtual cryptographic accelerator. The virtio crypto device provides the
following crypto services: CIPHER, MAC, HASH, AEAD and AKCIPHER. Virtio crypto
devices have a single control queue and at least one data queue. Crypto
operation requests are placed into a data queue, and serviced by the
device. Some crypto operation requests are only valid in the context of a
session. The role of the control queue is facilitating control operation
requests. Sessions management is realized with control operation
requests.

\subsection{Device ID}\label{sec:Device Types / Crypto Device / Device ID}

20

\subsection{Virtqueues}\label{sec:Device Types / Crypto Device / Virtqueues}

\begin{description}
\item[0] dataq1
\item[\ldots]
\item[N-1] dataqN
\item[N] controlq
\end{description}

N is set by \field{max_dataqueues}.

\subsection{Feature bits}\label{sec:Device Types / Crypto Device / Feature bits}

\begin{description}
\item VIRTIO_CRYPTO_F_REVISION_1 (0) revision 1. Revision 1 has a specific
    request format and other enhancements (which result in some additional
    requirements).
\item VIRTIO_CRYPTO_F_CIPHER_STATELESS_MODE (1) stateless mode requests are
    supported by the CIPHER service.
\item VIRTIO_CRYPTO_F_HASH_STATELESS_MODE (2) stateless mode requests are
    supported by the HASH service.
\item VIRTIO_CRYPTO_F_MAC_STATELESS_MODE (3) stateless mode requests are
    supported by the MAC service.
\item VIRTIO_CRYPTO_F_AEAD_STATELESS_MODE (4) stateless mode requests are
    supported by the AEAD service.
\item VIRTIO_CRYPTO_F_AKCIPHER_STATELESS_MODE (5) stateless mode requests are
    supported by the AKCIPHER service.
\end{description}


\subsubsection{Feature bit requirements}\label{sec:Device Types / Crypto Device / Feature bit requirements}

Some crypto feature bits require other crypto feature bits
(see \ref{drivernormative:Basic Facilities of a Virtio Device / Feature Bits}):

\begin{description}
\item[VIRTIO_CRYPTO_F_CIPHER_STATELESS_MODE] Requires VIRTIO_CRYPTO_F_REVISION_1.
\item[VIRTIO_CRYPTO_F_HASH_STATELESS_MODE] Requires VIRTIO_CRYPTO_F_REVISION_1.
\item[VIRTIO_CRYPTO_F_MAC_STATELESS_MODE] Requires VIRTIO_CRYPTO_F_REVISION_1.
\item[VIRTIO_CRYPTO_F_AEAD_STATELESS_MODE] Requires VIRTIO_CRYPTO_F_REVISION_1.
\item[VIRTIO_CRYPTO_F_AKCIPHER_STATELESS_MODE] Requires VIRTIO_CRYPTO_F_REVISION_1.
\end{description}

\subsection{Supported crypto services}\label{sec:Device Types / Crypto Device / Supported crypto services}

The following crypto services are defined:

\begin{lstlisting}
/* CIPHER (Symmetric Key Cipher) service */
#define VIRTIO_CRYPTO_SERVICE_CIPHER 0
/* HASH service */
#define VIRTIO_CRYPTO_SERVICE_HASH   1
/* MAC (Message Authentication Codes) service */
#define VIRTIO_CRYPTO_SERVICE_MAC    2
/* AEAD (Authenticated Encryption with Associated Data) service */
#define VIRTIO_CRYPTO_SERVICE_AEAD   3
/* AKCIPHER (Asymmetric Key Cipher) service */
#define VIRTIO_CRYPTO_SERVICE_AKCIPHER 4
\end{lstlisting}

The above constants designate bits used to indicate the which of crypto services are
offered by the device as described in, see \ref{sec:Device Types / Crypto Device / Device configuration layout}.

\subsubsection{CIPHER services}\label{sec:Device Types / Crypto Device / Supported crypto services / CIPHER services}

The following CIPHER algorithms are defined:

\begin{lstlisting}
#define VIRTIO_CRYPTO_NO_CIPHER                 0
#define VIRTIO_CRYPTO_CIPHER_ARC4               1
#define VIRTIO_CRYPTO_CIPHER_AES_ECB            2
#define VIRTIO_CRYPTO_CIPHER_AES_CBC            3
#define VIRTIO_CRYPTO_CIPHER_AES_CTR            4
#define VIRTIO_CRYPTO_CIPHER_DES_ECB            5
#define VIRTIO_CRYPTO_CIPHER_DES_CBC            6
#define VIRTIO_CRYPTO_CIPHER_3DES_ECB           7
#define VIRTIO_CRYPTO_CIPHER_3DES_CBC           8
#define VIRTIO_CRYPTO_CIPHER_3DES_CTR           9
#define VIRTIO_CRYPTO_CIPHER_KASUMI_F8          10
#define VIRTIO_CRYPTO_CIPHER_SNOW3G_UEA2        11
#define VIRTIO_CRYPTO_CIPHER_AES_F8             12
#define VIRTIO_CRYPTO_CIPHER_AES_XTS            13
#define VIRTIO_CRYPTO_CIPHER_ZUC_EEA3           14
\end{lstlisting}

The above constants have two usages:
\begin{enumerate}
\item As bit numbers, used to tell the driver which CIPHER algorithms
are supported by the device, see \ref{sec:Device Types / Crypto Device / Device configuration layout}.
\item As values, used to designate the algorithm in (CIPHER type) crypto
operation requests, see \ref{sec:Device Types / Crypto Device / Device Operation / Control Virtqueue / Session operation}.
\end{enumerate}

\subsubsection{HASH services}\label{sec:Device Types / Crypto Device / Supported crypto services / HASH services}

The following HASH algorithms are defined:

\begin{lstlisting}
#define VIRTIO_CRYPTO_NO_HASH            0
#define VIRTIO_CRYPTO_HASH_MD5           1
#define VIRTIO_CRYPTO_HASH_SHA1          2
#define VIRTIO_CRYPTO_HASH_SHA_224       3
#define VIRTIO_CRYPTO_HASH_SHA_256       4
#define VIRTIO_CRYPTO_HASH_SHA_384       5
#define VIRTIO_CRYPTO_HASH_SHA_512       6
#define VIRTIO_CRYPTO_HASH_SHA3_224      7
#define VIRTIO_CRYPTO_HASH_SHA3_256      8
#define VIRTIO_CRYPTO_HASH_SHA3_384      9
#define VIRTIO_CRYPTO_HASH_SHA3_512      10
#define VIRTIO_CRYPTO_HASH_SHA3_SHAKE128      11
#define VIRTIO_CRYPTO_HASH_SHA3_SHAKE256      12
\end{lstlisting}

The above constants have two usages:
\begin{enumerate}
\item As bit numbers, used to tell the driver which HASH algorithms
are supported by the device, see \ref{sec:Device Types / Crypto Device / Device configuration layout}.
\item As values, used to designate the algorithm in (HASH type) crypto
operation requires, see \ref{sec:Device Types / Crypto Device / Device Operation / Control Virtqueue / Session operation}.
\end{enumerate}

\subsubsection{MAC services}\label{sec:Device Types / Crypto Device / Supported crypto services / MAC services}

The following MAC algorithms are defined:

\begin{lstlisting}
#define VIRTIO_CRYPTO_NO_MAC                       0
#define VIRTIO_CRYPTO_MAC_HMAC_MD5                 1
#define VIRTIO_CRYPTO_MAC_HMAC_SHA1                2
#define VIRTIO_CRYPTO_MAC_HMAC_SHA_224             3
#define VIRTIO_CRYPTO_MAC_HMAC_SHA_256             4
#define VIRTIO_CRYPTO_MAC_HMAC_SHA_384             5
#define VIRTIO_CRYPTO_MAC_HMAC_SHA_512             6
#define VIRTIO_CRYPTO_MAC_CMAC_3DES                25
#define VIRTIO_CRYPTO_MAC_CMAC_AES                 26
#define VIRTIO_CRYPTO_MAC_KASUMI_F9                27
#define VIRTIO_CRYPTO_MAC_SNOW3G_UIA2              28
#define VIRTIO_CRYPTO_MAC_GMAC_AES                 41
#define VIRTIO_CRYPTO_MAC_GMAC_TWOFISH             42
#define VIRTIO_CRYPTO_MAC_CBCMAC_AES               49
#define VIRTIO_CRYPTO_MAC_CBCMAC_KASUMI_F9         50
#define VIRTIO_CRYPTO_MAC_XCBC_AES                 53
#define VIRTIO_CRYPTO_MAC_ZUC_EIA3                 54
\end{lstlisting}

The above constants have two usages:
\begin{enumerate}
\item As bit numbers, used to tell the driver which MAC algorithms
are supported by the device, see \ref{sec:Device Types / Crypto Device / Device configuration layout}.
\item As values, used to designate the algorithm in (MAC type) crypto
operation requests, see \ref{sec:Device Types / Crypto Device / Device Operation / Control Virtqueue / Session operation}.
\end{enumerate}

\subsubsection{AEAD services}\label{sec:Device Types / Crypto Device / Supported crypto services / AEAD services}

The following AEAD algorithms are defined:

\begin{lstlisting}
#define VIRTIO_CRYPTO_NO_AEAD     0
#define VIRTIO_CRYPTO_AEAD_GCM    1
#define VIRTIO_CRYPTO_AEAD_CCM    2
#define VIRTIO_CRYPTO_AEAD_CHACHA20_POLY1305  3
\end{lstlisting}

The above constants have two usages:
\begin{enumerate}
\item As bit numbers, used to tell the driver which AEAD algorithms
are supported by the device, see \ref{sec:Device Types / Crypto Device / Device configuration layout}.
\item As values, used to designate the algorithm in (DEAD type) crypto
operation requests, see \ref{sec:Device Types / Crypto Device / Device Operation / Control Virtqueue / Session operation}.
\end{enumerate}

\subsubsection{AKCIPHER services}\label{sec: Device Types / Crypto Device / Supported crypto services / AKCIPHER services}

The following AKCIPHER algorithms are defined:
\begin{lstlisting}
#define VIRTIO_CRYPTO_NO_AKCIPHER 0
#define VIRTIO_CRYPTO_AKCIPHER_RSA   1
#define VIRTIO_CRYPTO_AKCIPHER_ECDSA 2
\end{lstlisting}

The above constants have two usages:
\begin{enumerate}
\item As bit numbers, used to tell the driver which AKCIPHER algorithms
are supported by the device, see \ref{sec:Device Types / Crypto Device / Device configuration layout}.
\item As values, used to designate the algorithm in asymmetric crypto operation requests,
see \ref{sec:Device Types / Crypto Device / Device Operation / Control Virtqueue / Session operation}.
\end{enumerate}


\subsection{Device configuration layout}\label{sec:Device Types / Crypto Device / Device configuration layout}

Crypto device configuration uses the following layout structure:

\begin{lstlisting}
struct virtio_crypto_config {
    le32 status;
    le32 max_dataqueues;
    le32 crypto_services;
    /* Detailed algorithms mask */
    le32 cipher_algo_l;
    le32 cipher_algo_h;
    le32 hash_algo;
    le32 mac_algo_l;
    le32 mac_algo_h;
    le32 aead_algo;
    /* Maximum length of cipher key in bytes */
    le32 max_cipher_key_len;
    /* Maximum length of authenticated key in bytes */
    le32 max_auth_key_len;
    le32 akcipher_algo;
    /* Maximum size of each crypto request's content in bytes */
    le64 max_size;
};
\end{lstlisting}

\begin{description}
\item Currently, only one \field{status} bit is defined: VIRTIO_CRYPTO_S_HW_READY
    set indicates that the device is ready to process requests, this bit is read-only
    for the driver
\begin{lstlisting}
#define VIRTIO_CRYPTO_S_HW_READY  (1 << 0)
\end{lstlisting}

\item [\field{max_dataqueues}] is the maximum number of data virtqueues that can
    be configured by the device. The driver MAY use only one data queue, or it
    can use more to achieve better performance.

\item [\field{crypto_services}] crypto service offered, see \ref{sec:Device Types / Crypto Device / Supported crypto services}.

\item [\field{cipher_algo_l}] CIPHER algorithms bits 0-31, see \ref{sec:Device Types / Crypto Device / Supported crypto services  / CIPHER services}.

\item [\field{cipher_algo_h}] CIPHER algorithms bits 32-63, see \ref{sec:Device Types / Crypto Device / Supported crypto services  / CIPHER services}.

\item [\field{hash_algo}] HASH algorithms bits, see \ref{sec:Device Types / Crypto Device / Supported crypto services  / HASH services}.

\item [\field{mac_algo_l}] MAC algorithms bits 0-31, see \ref{sec:Device Types / Crypto Device / Supported crypto services  / MAC services}.

\item [\field{mac_algo_h}] MAC algorithms bits 32-63, see \ref{sec:Device Types / Crypto Device / Supported crypto services  / MAC services}.

\item [\field{aead_algo}] AEAD algorithms bits, see \ref{sec:Device Types / Crypto Device / Supported crypto services  / AEAD services}.

\item [\field{max_cipher_key_len}] is the maximum length of cipher key supported by the device.

\item [\field{max_auth_key_len}] is the maximum length of authenticated key supported by the device.

\item [\field{akcipher_algo}] AKCIPHER algorithms bit 0-31, see \ref{sec: Device Types / Crypto Device / Supported crypto services / AKCIPHER services}.

\item [\field{max_size}] is the maximum size of the variable-length parameters of
    data operation of each crypto request's content supported by the device.
\end{description}

\begin{note}
Unless explicitly stated otherwise all lengths and sizes are in bytes.
\end{note}

\devicenormative{\subsubsection}{Device configuration layout}{Device Types / Crypto Device / Device configuration layout}

\begin{itemize*}
\item The device MUST set \field{max_dataqueues} to between 1 and 65535 inclusive.
\item The device MUST set the \field{status} with valid flags, undefined flags MUST NOT be set.
\item The device MUST accept and handle requests after \field{status} is set to VIRTIO_CRYPTO_S_HW_READY.
\item The device MUST set \field{crypto_services} based on the crypto services the device offers.
\item The device MUST set detailed algorithms masks for each service advertised by \field{crypto_services}.
    The device MUST NOT set the not defined algorithms bits.
\item The device MUST set \field{max_size} to show the maximum size of crypto request the device supports.
\item The device MUST set \field{max_cipher_key_len} to show the maximum length of cipher key if the
    device supports CIPHER service.
\item The device MUST set \field{max_auth_key_len} to show the maximum length of authenticated key if
    the device supports MAC service.
\end{itemize*}

\drivernormative{\subsubsection}{Device configuration layout}{Device Types / Crypto Device / Device configuration layout}

\begin{itemize*}
\item The driver MUST read the \field{status} from the bottom bit of status to check whether the
    VIRTIO_CRYPTO_S_HW_READY is set, and the driver MUST reread it after device reset.
\item The driver MUST NOT transmit any requests to the device if the VIRTIO_CRYPTO_S_HW_READY is not set.
\item The driver MUST read \field{max_dataqueues} field to discover the number of data queues the device supports.
\item The driver MUST read \field{crypto_services} field to discover which services the device is able to offer.
\item The driver SHOULD ignore the not defined algorithms bits.
\item The driver MUST read the detailed algorithms fields based on \field{crypto_services} field.
\item The driver SHOULD read \field{max_size} to discover the maximum size of the variable-length
    parameters of data operation of the crypto request's content the device supports and MUST
    guarantee that the size of each crypto request's content is within the \field{max_size}, otherwise
    the request will fail and the driver MUST reset the device.
\item The driver SHOULD read \field{max_cipher_key_len} to discover the maximum length of cipher key
    the device supports and MUST guarantee that the \field{key_len} (CIPHER service or AEAD service) is within
    the \field{max_cipher_key_len} of the device configuration, otherwise the request will fail.
\item The driver SHOULD read \field{max_auth_key_len} to discover the maximum length of authenticated
    key the device supports and MUST guarantee that the \field{auth_key_len} (MAC service) is within the
    \field{max_auth_key_len} of the device configuration, otherwise the request will fail.
\end{itemize*}

\subsection{Device Initialization}\label{sec:Device Types / Crypto Device / Device Initialization}

\drivernormative{\subsubsection}{Device Initialization}{Device Types / Crypto Device / Device Initialization}

\begin{itemize*}
\item The driver MUST configure and initialize all virtqueues.
\item The driver MUST read the supported crypto services from bits of \field{crypto_services}.
\item The driver MUST read the supported algorithms based on \field{crypto_services} field.
\end{itemize*}

\subsection{Device Operation}\label{sec:Device Types / Crypto Device / Device Operation}

The operation of a virtio crypto device is driven by requests placed on the virtqueues.
Requests consist of a queue-type specific header (specifying among others the operation)
and an operation specific payload.

If VIRTIO_CRYPTO_F_REVISION_1 is negotiated the device may support both session mode
(See \ref{sec:Device Types / Crypto Device / Device Operation / Control Virtqueue / Session operation})
and stateless mode operation requests.
In stateless mode all operation parameters are supplied as a part of each request,
while in session mode, some or all operation parameters are managed within the
session. Stateless mode is guarded by feature bits 0-4 on a service level. If
stateless mode is negotiated for a service, the service accepts both session
mode and stateless requests; otherwise stateless mode requests are rejected
(via operation status).

\subsubsection{Operation Status}\label{sec:Device Types / Crypto Device / Device Operation / Operation status}
The device MUST return a status code as part of the operation (both session
operation and service operation) result. The valid operation status as follows:

\begin{lstlisting}
enum VIRTIO_CRYPTO_STATUS {
    VIRTIO_CRYPTO_OK = 0,
    VIRTIO_CRYPTO_ERR = 1,
    VIRTIO_CRYPTO_BADMSG = 2,
    VIRTIO_CRYPTO_NOTSUPP = 3,
    VIRTIO_CRYPTO_INVSESS = 4,
    VIRTIO_CRYPTO_NOSPC = 5,
    VIRTIO_CRYPTO_KEY_REJECTED = 6,
    VIRTIO_CRYPTO_MAX
};
\end{lstlisting}

\begin{itemize*}
\item VIRTIO_CRYPTO_OK: success.
\item VIRTIO_CRYPTO_BADMSG: authentication failed (only when AEAD decryption).
\item VIRTIO_CRYPTO_NOTSUPP: operation or algorithm is unsupported.
\item VIRTIO_CRYPTO_INVSESS: invalid session ID when executing crypto operations.
\item VIRTIO_CRYPTO_NOSPC: no free session ID (only when the VIRTIO_CRYPTO_F_REVISION_1
    feature bit is negotiated).
\item VIRTIO_CRYPTO_KEY_REJECTED: signature verification failed (only when AKCIPHER verification).
\item VIRTIO_CRYPTO_ERR: any failure not mentioned above occurs.
\end{itemize*}

\subsubsection{Control Virtqueue}\label{sec:Device Types / Crypto Device / Device Operation / Control Virtqueue}

The driver uses the control virtqueue to send control commands to the
device, such as session operations (See \ref{sec:Device Types / Crypto Device / Device
Operation / Control Virtqueue / Session operation}).

The header for controlq is of the following form:
\begin{lstlisting}
#define VIRTIO_CRYPTO_OPCODE(service, op)   (((service) << 8) | (op))

struct virtio_crypto_ctrl_header {
#define VIRTIO_CRYPTO_CIPHER_CREATE_SESSION \
       VIRTIO_CRYPTO_OPCODE(VIRTIO_CRYPTO_SERVICE_CIPHER, 0x02)
#define VIRTIO_CRYPTO_CIPHER_DESTROY_SESSION \
       VIRTIO_CRYPTO_OPCODE(VIRTIO_CRYPTO_SERVICE_CIPHER, 0x03)
#define VIRTIO_CRYPTO_HASH_CREATE_SESSION \
       VIRTIO_CRYPTO_OPCODE(VIRTIO_CRYPTO_SERVICE_HASH, 0x02)
#define VIRTIO_CRYPTO_HASH_DESTROY_SESSION \
       VIRTIO_CRYPTO_OPCODE(VIRTIO_CRYPTO_SERVICE_HASH, 0x03)
#define VIRTIO_CRYPTO_MAC_CREATE_SESSION \
       VIRTIO_CRYPTO_OPCODE(VIRTIO_CRYPTO_SERVICE_MAC, 0x02)
#define VIRTIO_CRYPTO_MAC_DESTROY_SESSION \
       VIRTIO_CRYPTO_OPCODE(VIRTIO_CRYPTO_SERVICE_MAC, 0x03)
#define VIRTIO_CRYPTO_AEAD_CREATE_SESSION \
       VIRTIO_CRYPTO_OPCODE(VIRTIO_CRYPTO_SERVICE_AEAD, 0x02)
#define VIRTIO_CRYPTO_AEAD_DESTROY_SESSION \
       VIRTIO_CRYPTO_OPCODE(VIRTIO_CRYPTO_SERVICE_AEAD, 0x03)
#define VIRTIO_CRYPTO_AKCIPHER_CREATE_SESSION \
       VIRTIO_CRYPTO_OPCODE(VIRTIO_CRYPTO_SERVICE_AKCIPHER, 0x04)
#define VIRTIO_CRYPTO_AKCIPHER_DESTROY_SESSION \
       VIRTIO_CRYPTO_OPCDE(VIRTIO_CRYPTO_SERVICE_AKCIPHER, 0x05)
    le32 opcode;
    /* algo should be service-specific algorithms */
    le32 algo;
    le32 flag;
    le32 reserved;
};
\end{lstlisting}

The controlq request is composed of four parts:
\begin{lstlisting}
struct virtio_crypto_op_ctrl_req {
    /* Device read only portion */

    struct virtio_crypto_ctrl_header header;

#define VIRTIO_CRYPTO_CTRLQ_OP_SPEC_HDR_LEGACY 56
    /* fixed length fields, opcode specific */
    u8 op_flf[flf_len];

    /* variable length fields, opcode specific */
    u8 op_vlf[vlf_len];

    /* Device write only portion */

    /* op result or completion status */
    u8 op_outcome[outcome_len];
};
\end{lstlisting}

\field{header} is a general header (see above).

\field{op_flf} is the opcode (in \field{header}) specific fixed-length parameters.

\field{flf_len} depends on the VIRTIO_CRYPTO_F_REVISION_1 feature bit (see below).

\field{op_vlf} is the opcode (in \field{header}) specific variable-length parameters.

\field{vlf_len} is the size of the specific structure used.
\begin{note}
The \field{vlf_len} of session-destroy operation and the hash-session-create
operation is ZERO.
\end{note}

\begin{itemize*}
\item If the opcode (in \field{header}) is VIRTIO_CRYPTO_CIPHER_CREATE_SESSION
    then \field{op_flf} is struct virtio_crypto_sym_create_session_flf if
    VIRTIO_CRYPTO_F_REVISION_1 is negotiated and struct virtio_crypto_sym_create_session_flf is
    padded to 56 bytes if NOT negotiated, and \field{op_vlf} is struct
    virtio_crypto_sym_create_session_vlf.
\item If the opcode (in \field{header}) is VIRTIO_CRYPTO_HASH_CREATE_SESSION
    then \field{op_flf} is struct virtio_crypto_hash_create_session_flf if
    VIRTIO_CRYPTO_F_REVISION_1 is negotiated and struct virtio_crypto_hash_create_session_flf is
    padded to 56 bytes if NOT negotiated.
\item If the opcode (in \field{header}) is VIRTIO_CRYPTO_MAC_CREATE_SESSION
    then \field{op_flf} is struct virtio_crypto_mac_create_session_flf if
    VIRTIO_CRYPTO_F_REVISION_1 is negotiated and struct virtio_crypto_mac_create_session_flf is
    padded to 56 bytes if NOT negotiated, and \field{op_vlf} is struct
    virtio_crypto_mac_create_session_vlf.
\item If the opcode (in \field{header}) is VIRTIO_CRYPTO_AEAD_CREATE_SESSION
    then \field{op_flf} is struct virtio_crypto_aead_create_session_flf if
    VIRTIO_CRYPTO_F_REVISION_1 is negotiated and struct virtio_crypto_aead_create_session_flf is
    padded to 56 bytes if NOT negotiated, and \field{op_vlf} is struct
    virtio_crypto_aead_create_session_vlf.
\item If the opcode (in \field{header}) is VIRTIO_CRYPTO_AKCIPHER_CREATE_SESSION
    then \field{op_flf} is struct virtio_crypto_akcipher_create_session_flf if
    VIRTIO_CRYPTO_F_REVISION_1 is negotiated and struct virtio_crypto_akcipher_create_session_flf is
    padded to 56 bytes if NOT negotiated, and \field{op_vlf} is struct
    virtio_crypto_akcipher_create_session_vlf.
\item If the opcode (in \field{header}) is VIRTIO_CRYPTO_CIPHER_DESTROY_SESSION
    or VIRTIO_CRYPTO_HASH_DESTROY_SESSION or VIRTIO_CRYPTO_MAC_DESTROY_SESSION or
    VIRTIO_CRYPTO_AEAD_DESTROY_SESSION then \field{op_flf} is struct
    virtio_crypto_destroy_session_flf if VIRTIO_CRYPTO_F_REVISION_1 is negotiated and
    struct virtio_crypto_destroy_session_flf is padded to 56 bytes if NOT negotiated.
\end{itemize*}

\field{op_outcome} stores the result of operation and must be struct
virtio_crypto_destroy_session_input for destroy session or
struct virtio_crypto_create_session_input for create session.

\field{outcome_len} is the size of the structure used.


\paragraph{Session operation}\label{sec:Device Types / Crypto Device / Device
Operation / Control Virtqueue / Session operation}

The session is a handle which describes the cryptographic parameters to be
applied to a number of buffers.

The following structure stores the result of session creation set by the device:

\begin{lstlisting}
struct virtio_crypto_create_session_input {
    le64 session_id;
    le32 status;
    le32 padding;
};
\end{lstlisting}

A request to destroy a session includes the following information:

\begin{lstlisting}
struct virtio_crypto_destroy_session_flf {
    /* Device read only portion */
    le64  session_id;
};

struct virtio_crypto_destroy_session_input {
    /* Device write only portion */
    u8  status;
};
\end{lstlisting}


\subparagraph{Session operation: HASH session}\label{sec:Device Types / Crypto Device / Device
Operation / Control Virtqueue / Session operation / Session operation: HASH session}

The fixed-length parameters of HASH session requests is as follows:

\begin{lstlisting}
struct virtio_crypto_hash_create_session_flf {
    /* Device read only portion */

    /* See VIRTIO_CRYPTO_HASH_* above */
    le32 algo;
    /* hash result length */
    le32 hash_result_len;
};
\end{lstlisting}


\subparagraph{Session operation: MAC session}\label{sec:Device Types / Crypto Device / Device
Operation / Control Virtqueue / Session operation / Session operation: MAC session}

The fixed-length and the variable-length parameters of MAC session requests are as follows:

\begin{lstlisting}
struct virtio_crypto_mac_create_session_flf {
    /* Device read only portion */

    /* See VIRTIO_CRYPTO_MAC_* above */
    le32 algo;
    /* hash result length */
    le32 hash_result_len;
    /* length of authenticated key */
    le32 auth_key_len;
    le32 padding;
};

struct virtio_crypto_mac_create_session_vlf {
    /* Device read only portion */

    /* The authenticated key */
    u8 auth_key[auth_key_len];
};
\end{lstlisting}

The length of \field{auth_key} is specified in \field{auth_key_len} in the struct
virtio_crypto_mac_create_session_flf.


\subparagraph{Session operation: Symmetric algorithms session}\label{sec:Device Types / Crypto Device / Device
Operation / Control Virtqueue / Session operation / Session operation: Symmetric algorithms session}

The request of symmetric session could be the CIPHER algorithms request
or the chain algorithms (chaining CIPHER and HASH/MAC) request.

The fixed-length and the variable-length parameters of CIPHER session requests are as follows:

\begin{lstlisting}
struct virtio_crypto_cipher_session_flf {
    /* Device read only portion */

    /* See VIRTIO_CRYPTO_CIPHER* above */
    le32 algo;
    /* length of key */
    le32 key_len;
#define VIRTIO_CRYPTO_OP_ENCRYPT  1
#define VIRTIO_CRYPTO_OP_DECRYPT  2
    /* encryption or decryption */
    le32 op;
    le32 padding;
};

struct virtio_crypto_cipher_session_vlf {
    /* Device read only portion */

    /* The cipher key */
    u8 cipher_key[key_len];
};
\end{lstlisting}

The length of \field{cipher_key} is specified in \field{key_len} in the struct
virtio_crypto_cipher_session_flf.

The fixed-length and the variable-length parameters of Chain session requests are as follows:

\begin{lstlisting}
struct virtio_crypto_alg_chain_session_flf {
    /* Device read only portion */

#define VIRTIO_CRYPTO_SYM_ALG_CHAIN_ORDER_HASH_THEN_CIPHER  1
#define VIRTIO_CRYPTO_SYM_ALG_CHAIN_ORDER_CIPHER_THEN_HASH  2
    le32 alg_chain_order;
/* Plain hash */
#define VIRTIO_CRYPTO_SYM_HASH_MODE_PLAIN    1
/* Authenticated hash (mac) */
#define VIRTIO_CRYPTO_SYM_HASH_MODE_AUTH     2
/* Nested hash */
#define VIRTIO_CRYPTO_SYM_HASH_MODE_NESTED   3
    le32 hash_mode;
    struct virtio_crypto_cipher_session_flf cipher_hdr;

#define VIRTIO_CRYPTO_ALG_CHAIN_SESS_OP_SPEC_HDR_SIZE  16
    /* fixed length fields, algo specific */
    u8 algo_flf[VIRTIO_CRYPTO_ALG_CHAIN_SESS_OP_SPEC_HDR_SIZE];

    /* length of the additional authenticated data (AAD) in bytes */
    le32 aad_len;
    le32 padding;
};

struct virtio_crypto_alg_chain_session_vlf {
    /* Device read only portion */

    /* The cipher key */
    u8 cipher_key[key_len];
    /* The authenticated key */
    u8 auth_key[auth_key_len];
};
\end{lstlisting}

\field{hash_mode} decides the type used by \field{algo_flf}.

\field{algo_flf} is fixed to 16 bytes and MUST contains or be one of
the following types:
\begin{itemize*}
\item struct virtio_crypto_hash_create_session_flf
\item struct virtio_crypto_mac_create_session_flf
\end{itemize*}
The data of unused part (if has) in \field{algo_flf} will be ignored.

The length of \field{cipher_key} is specified in \field{key_len} in \field{cipher_hdr}.

The length of \field{auth_key} is specified in \field{auth_key_len} in struct
virtio_crypto_mac_create_session_flf.

The fixed-length parameters of Symmetric session requests are as follows:

\begin{lstlisting}
struct virtio_crypto_sym_create_session_flf {
    /* Device read only portion */

#define VIRTIO_CRYPTO_SYM_SESS_OP_SPEC_HDR_SIZE  48
    /* fixed length fields, opcode specific */
    u8 op_flf[VIRTIO_CRYPTO_SYM_SESS_OP_SPEC_HDR_SIZE];

/* No operation */
#define VIRTIO_CRYPTO_SYM_OP_NONE  0
/* Cipher only operation on the data */
#define VIRTIO_CRYPTO_SYM_OP_CIPHER  1
/* Chain any cipher with any hash or mac operation. The order
   depends on the value of alg_chain_order param */
#define VIRTIO_CRYPTO_SYM_OP_ALGORITHM_CHAINING  2
    le32 op_type;
    le32 padding;
};
\end{lstlisting}

\field{op_flf} is fixed to 48 bytes, MUST contains or be one of
the following types:
\begin{itemize*}
\item struct virtio_crypto_cipher_session_flf
\item struct virtio_crypto_alg_chain_session_flf
\end{itemize*}
The data of unused part (if has) in \field{op_flf} will be ignored.

\field{op_type} decides the type used by \field{op_flf}.

The variable-length parameters of Symmetric session requests are as follows:

\begin{lstlisting}
struct virtio_crypto_sym_create_session_vlf {
    /* Device read only portion */
    /* variable length fields, opcode specific */
    u8 op_vlf[vlf_len];
};
\end{lstlisting}

\field{op_vlf} MUST contains or be one of the following types:
\begin{itemize*}
\item struct virtio_crypto_cipher_session_vlf
\item struct virtio_crypto_alg_chain_session_vlf
\end{itemize*}

\field{op_type} in struct virtio_crypto_sym_create_session_flf decides the
type used by \field{op_vlf}.

\field{vlf_len} is the size of the specific structure used.


\subparagraph{Session operation: AEAD session}\label{sec:Device Types / Crypto Device / Device
Operation / Control Virtqueue / Session operation / Session operation: AEAD session}

The fixed-length and the variable-length parameters of AEAD session requests are as follows:

\begin{lstlisting}
struct virtio_crypto_aead_create_session_flf {
    /* Device read only portion */

    /* See VIRTIO_CRYPTO_AEAD_* above */
    le32 algo;
    /* length of key */
    le32 key_len;
    /* Authentication tag length */
    le32 tag_len;
    /* The length of the additional authenticated data (AAD) in bytes */
    le32 aad_len;
    /* encryption or decryption, See above VIRTIO_CRYPTO_OP_* */
    le32 op;
    le32 padding;
};

struct virtio_crypto_aead_create_session_vlf {
    /* Device read only portion */
    u8 key[key_len];
};
\end{lstlisting}

The length of \field{key} is specified in \field{key_len} in struct
virtio_crypto_aead_create_session_flf.

\subparagraph{Session operation: AKCIPHER session}\label{sec:Device Types / Crypto Device / Device
Operation / Control Virtqueue / Session operation / Session operation: AKCIPHER session}

Due to the complexity of asymmetric key algorithms, different algorithms
require different parameters. The following data structures are used as
supplementary parameters to describe the asymmetric algorithm sessions.

For the RSA algorithm, the extra parameters are as follows:
\begin{lstlisting}
struct virtio_crypto_rsa_session_para {
#define VIRTIO_CRYPTO_RSA_RAW_PADDING   0
#define VIRTIO_CRYPTO_RSA_PKCS1_PADDING 1
    le32 padding_algo;

#define VIRTIO_CRYPTO_RSA_NO_HASH   0
#define VIRTIO_CRYPTO_RSA_MD2       1
#define VIRTIO_CRYPTO_RSA_MD3       2
#define VIRTIO_CRYPTO_RSA_MD4       3
#define VIRTIO_CRYPTO_RSA_MD5       4
#define VIRTIO_CRYPTO_RSA_SHA1      5
#define VIRTIO_CRYPTO_RSA_SHA256    6
#define VIRTIO_CRYPTO_RSA_SHA384    7
#define VIRTIO_CRYPTO_RSA_SHA512    8
#define VIRTIO_CRYPTO_RSA_SHA224    9
    le32 hash_algo;
};
\end{lstlisting}

\field{padding_algo} specifies the padding method used by RSA sessions.
\begin{itemize*}
\item If VIRTIO_CRYPTO_RSA_RAW_PADDING is specified, 1) \field{hash_algo}
is ignored, 2) ciphertext and plaintext MUST be padded with leading zeros,
3) and RSA sessions with VIRTIO_CRYPTO_RSA_RAW_PADDING MUST not be used
for verification and signing operations.
\item If VIRTIO_CRYPTO_RSA_PKCS1_PADDING is specified, EMSA-PKCS1-v1_5 padding method
is used (see \hyperref[intro:rfc3447]{PKCS\#1}), \field{hash_algo} specifies how the
digest of the data passed to RSA sessions is calculated when verifying and signing.
It only affects the padding algorithm and is ignored during encryption and decryption.
\end{itemize*}

The ECC algorithms such as the ECDSA algorithm, cannot use custom curves, only the
following known curves can be used (see \hyperref[intro:NIST]{NIST-recommended curves}).

\begin{lstlisting}
#define VIRTIO_CRYPTO_CURVE_UNKNOWN   0
#define VIRTIO_CRYPTO_CURVE_NIST_P192 1
#define VIRTIO_CRYPTO_CURVE_NIST_P224 2
#define VIRTIO_CRYPTO_CURVE_NIST_P256 3
#define VIRTIO_CRYPTO_CURVE_NIST_P384 4
#define VIRTIO_CRYPTO_CURVE_NIST_P521 5
\end{lstlisting}

For the ECDSA algorithm, the extra parameters are as follows:
\begin{lstlisting}
struct virtio_crypto_ecdsa_session_para {
    /* See VIRTIO_CRYPTO_CURVE_* above */
    le32 curve_id;
};
\end{lstlisting}

The fixed-length and the variable-length parameters of AKCIPHER session requests are as follows:
\begin{lstlisting}
struct virtio_crypto_akcipher_create_session_flf {
    /* Device read only portion */

    /* See VIRTIO_CRYPTO_AKCIPHER_* above */
    le32 algo;
#define VIRTIO_CRYPTO_AKCIPHER_KEY_TYPE_PUBLIC 1
#define VIRTIO_CRYPTO_AKCIPHER_KEY_TYPE_PRIVATE 2
    le32 key_type;
    /* length of key */
    le32 key_len;

#define VIRTIO_CRYPTO_AKCIPHER_SESS_ALGO_SPEC_HDR_SIZE 44
    u8 algo_flf[VIRTIO_CRYPTO_AKCIPHER_SESS_ALGO_SPEC_HDR_SIZE];
};

struct virtio_crypto_akcipher_create_session_vlf {
    /* Device read only portion */
    u8 key[key_len];
};
\end{lstlisting}

\field{algo} decides the type used by \field{algo_flf}.
\field{algo_flf} is fixed to 44 bytes and MUST contains of be one the
following structures:
\begin{itemize*}
\item struct virtio_crypto_rsa_session_para
\item struct virtio_crypto_ecdsa_session_para
\end{itemize*}

The length of \field{key} is specified in \field{key_len} in the struct
virtio_crypto_akcipher_create_session_flf.

For the RSA algorithm, the key needs to be encoded according to
\hyperref[intro:rfc3447]{PKCS\#1}. The private key is described with the
RSAPrivateKey structure, and the public key is described with the RSAPublicKey
structure. These ASN.1 structures are encoded in DER encoding rules (see
\hyperref[intro:rfc6025]{rfc6025}).

\begin{lstlisting}
RSAPrivateKey ::= SEQUENCE {
    version          INTEGER,
    modulus          INTEGER,
    publicExponent   INTEGER,
    privateExponent  INTEGER,
    prime1           INTEGER,
    prime2           INTEGER,
    exponent1        INTEGER,
    exponent1        INTEGER,
    coefficient      INTEGER,
    otherPrimeInfos  OtherPrimeInfos OPTIONAL
}

OtherPrimeInfos ::= SEQUENCE SIZE(1...MAX) OF OtherPrimeInfo

OtherPrimeINfo ::= SEQUENCE {
    prime           INTEGER,
    exponent        INTEGER,
    coefficient     INTEGER
}

RSAPublicKey ::= SEQUENCE {
    modulus         INTEGER,
    publicExponent  INTEGER
}
\end{lstlisting}

For the ECDSA algorithm, the private key is encoded according to
\hyperref[intro:rfc5915]{RFC5915}, the private key of the ECDSA algorithm
is described by the ASN.1 structure ECPrivateKey and encoded with DER
encoding rules (see \hyperref[intro:rfc6025]{rfc6025}).

\begin{lstlisting}
ECPrivateKey ::= SEQUNCE {
    version         INTEGER,
    privateKey      OCTET STRING,
    parameters [0]  ECParameters {{ NamedCurve }} OPTIONAL,
    publicKey  [1]  BIT STRING OPTIONAL
}
\end{lstlisting}

The public key of the ECDSA algorithm is encoded according to \hyperref[intro:SEC1]{SEC1},
and the public key of ECDSA is described by the ASN.1 structure ECPoint.
When initializing a session with ECDSA public key, the ECPoint is DER encoded and the
\field{key} only contains the value part of ECPoint, that is, the header part of the
OCTET STRING will be omitted (see \hyperref[intro:rfc6025]{rfc6025}).

\begin{lstlisting}
ECPoint ::= OCTET STRING
\end{lstlisting}

The length of \field{key} is specified in \field{key_len} in
struct virtio_crypto_akcipher_create_session_flf.

\drivernormative{\subparagraph}{Session operation: create session}{Device Types / Crypto Device / Device
Operation / Control Virtqueue / Session operation / Session operation: create session}

\begin{itemize*}
\item The driver MUST set the \field{opcode} field based on service type: CIPHER, HASH, MAC, AEAD or AKCIPHER.
\item The driver MUST set the control general header, the opcode specific header,
    the opcode specific extra parameters and the opcode specific outcome buffer in turn.
    See \ref{sec:Device Types / Crypto Device / Device Operation / Control Virtqueue}.
\item The driver MUST set the \field{reversed} field to zero.
\end{itemize*}

\devicenormative{\subparagraph}{Session operation: create session}{Device Types / Crypto Device / Device
Operation / Control Virtqueue / Session operation / Session operation: create session}

\begin{itemize*}
\item The device MUST use the corresponding opcode specific structure according to the
    \field{opcode} in the control general header.
\item The device MUST extract extra parameters according to the structures used.
\item The device MUST set the \field{status} field to one of the following values of enum
    VIRTIO_CRYPTO_STATUS after finish a session creation:
\begin{itemize*}
\item VIRTIO_CRYPTO_OK if a session is created successfully.
\item VIRTIO_CRYPTO_NOTSUPP if the requested algorithm or operation is unsupported.
\item VIRTIO_CRYPTO_NOSPC if no free session ID (only when the VIRTIO_CRYPTO_F_REVISION_1
    feature bit is negotiated).
\item VIRTIO_CRYPTO_ERR if failure not mentioned above occurs.
\end{itemize*}
\item The device MUST set the \field{session_id} field to a unique session identifier only
    if the status is set to VIRTIO_CRYPTO_OK.
\end{itemize*}

\drivernormative{\subparagraph}{Session operation: destroy session}{Device Types / Crypto Device / Device
Operation / Control Virtqueue / Session operation / Session operation: destroy session}

\begin{itemize*}
\item The driver MUST set the \field{opcode} field based on service type: CIPHER, HASH, MAC, AEAD or AKCIPHER.
\item The driver MUST set the \field{session_id} to a valid value assigned by the device
    when the session was created.
\end{itemize*}

\devicenormative{\subparagraph}{Session operation: destroy session}{Device Types / Crypto Device / Device
Operation / Control Virtqueue / Session operation / Session operation: destroy session}

\begin{itemize*}
\item The device MUST set the \field{status} field to one of the following values of enum VIRTIO_CRYPTO_STATUS.
\begin{itemize*}
\item VIRTIO_CRYPTO_OK if a session is created successfully.
\item VIRTIO_CRYPTO_ERR if any failure occurs.
\end{itemize*}
\end{itemize*}


\subsubsection{Data Virtqueue}\label{sec:Device Types / Crypto Device / Device Operation / Data Virtqueue}

The driver uses the data virtqueues to transmit crypto operation requests to the device,
and completes the crypto operations.

The header for dataq is as follows:

\begin{lstlisting}
struct virtio_crypto_op_header {
#define VIRTIO_CRYPTO_CIPHER_ENCRYPT \
    VIRTIO_CRYPTO_OPCODE(VIRTIO_CRYPTO_SERVICE_CIPHER, 0x00)
#define VIRTIO_CRYPTO_CIPHER_DECRYPT \
    VIRTIO_CRYPTO_OPCODE(VIRTIO_CRYPTO_SERVICE_CIPHER, 0x01)
#define VIRTIO_CRYPTO_HASH \
    VIRTIO_CRYPTO_OPCODE(VIRTIO_CRYPTO_SERVICE_HASH, 0x00)
#define VIRTIO_CRYPTO_MAC \
    VIRTIO_CRYPTO_OPCODE(VIRTIO_CRYPTO_SERVICE_MAC, 0x00)
#define VIRTIO_CRYPTO_AEAD_ENCRYPT \
    VIRTIO_CRYPTO_OPCODE(VIRTIO_CRYPTO_SERVICE_AEAD, 0x00)
#define VIRTIO_CRYPTO_AEAD_DECRYPT \
    VIRTIO_CRYPTO_OPCODE(VIRTIO_CRYPTO_SERVICE_AEAD, 0x01)
#define VIRTIO_CRYPTO_AKCIPHER_ENCRYPT \
    VIRTIO_CRYPTO_OPCODE(VIRTIO_CRYPTO_SERVICE_AKCIPHER, 0x00)
#define VIRTIO_CRYPTO_AKCIPHER_DECRYPT \
    VIRTIO_CRYPTO_OPCODE(VIRTIO_CRYPTO_SERVICE_AKCIPHER, 0x01)
#define VIRTIO_CRYPTO_AKCIPHER_SIGN \
    VIRTIO_CRYPTO_OPCODE(VIRTIO_CRYPTO_SERVICE_AKCIPHER, 0x02)
#define VIRTIO_CRYPTO_AKCIPHER_VERIFY \
    VIRTIO_CRYPTO_OPCODE(VIRTIO_CRYPTO_SERVICE_AKCIPHER, 0x03)
    le32 opcode;
    /* algo should be service-specific algorithms */
    le32 algo;
    le64 session_id;
#define VIRTIO_CRYPTO_FLAG_SESSION_MODE 1
    /* control flag to control the request */
    le32 flag;
    le32 padding;
};
\end{lstlisting}

\begin{note}
If VIRTIO_CRYPTO_F_REVISION_1 is not negotiated the \field{flag} is ignored.

If VIRTIO_CRYPTO_F_REVISION_1 is negotiated but VIRTIO_CRYPTO_F_<SERVICE>_STATELESS_MODE
is not negotiated, then the device SHOULD reject <SERVICE> requests if
VIRTIO_CRYPTO_FLAG_SESSION_MODE is not set (in \field{flag}).
\end{note}

The dataq request is composed of four parts:
\begin{lstlisting}
struct virtio_crypto_op_data_req {
    /* Device read only portion */

    struct virtio_crypto_op_header header;

#define VIRTIO_CRYPTO_DATAQ_OP_SPEC_HDR_LEGACY 48
    /* fixed length fields, opcode specific */
    u8 op_flf[flf_len];

    /* Device read && write portion */
    /* variable length fields, opcode specific */
    u8 op_vlf[vlf_len];

    /* Device write only portion */
    struct virtio_crypto_inhdr inhdr;
};
\end{lstlisting}

\field{header} is a general header (see above).

\field{op_flf} is the opcode (in \field{header}) specific header.

\field{flf_len} depends on the VIRTIO_CRYPTO_F_REVISION_1 feature bit
(see below).

\field{op_vlf} is the opcode (in \field{header}) specific parameters.

\field{vlf_len} is the size of the specific structure used.

\begin{itemize*}
\item If the the opcode (in \field{header}) is VIRTIO_CRYPTO_CIPHER_ENCRYPT
    or VIRTIO_CRYPTO_CIPHER_DECRYPT then:
    \begin{itemize*}
    \item If VIRTIO_CRYPTO_F_CIPHER_STATELESS_MODE is negotiated, \field{op_flf} is
        struct virtio_crypto_sym_data_flf_stateless, and \field{op_vlf} is struct
        virtio_crypto_sym_data_vlf_stateless.
    \item If VIRTIO_CRYPTO_F_CIPHER_STATELESS_MODE is NOT negotiated, \field{op_flf}
        is struct virtio_crypto_sym_data_flf if VIRTIO_CRYPTO_F_REVISION_1 is negotiated
        and struct virtio_crypto_sym_data_flf is padded to 48 bytes if NOT negotiated,
        and \field{op_vlf} is struct virtio_crypto_sym_data_vlf.
    \end{itemize*}
\item If the the opcode (in \field{header}) is VIRTIO_CRYPTO_HASH:
    \begin{itemize*}
    \item If VIRTIO_CRYPTO_F_HASH_STATELESS_MODE is negotiated, \field{op_flf} is
        struct virtio_crypto_hash_data_flf_stateless, and \field{op_vlf} is struct
        virtio_crypto_hash_data_vlf_stateless.
    \item If VIRTIO_CRYPTO_F_HASH_STATELESS_MODE is NOT negotiated, \field{op_flf}
        is struct virtio_crypto_hash_data_flf if VIRTIO_CRYPTO_F_REVISION_1 is negotiated
        and struct virtio_crypto_hash_data_flf is padded to 48 bytes if NOT negotiated,
        and \field{op_vlf} is struct virtio_crypto_hash_data_vlf.
    \end{itemize*}
\item If the the opcode (in \field{header}) is VIRTIO_CRYPTO_MAC:
    \begin{itemize*}
    \item If VIRTIO_CRYPTO_F_MAC_STATELESS_MODE is negotiated, \field{op_flf} is
        struct virtio_crypto_mac_data_flf_stateless, and \field{op_vlf} is struct
        virtio_crypto_mac_data_vlf_stateless.
    \item If VIRTIO_CRYPTO_F_MAC_STATELESS_MODE is NOT negotiated, \field{op_flf}
        is struct virtio_crypto_mac_data_flf if VIRTIO_CRYPTO_F_REVISION_1 is negotiated
        and struct virtio_crypto_mac_data_flf is padded to 48 bytes if NOT negotiated,
        and \field{op_vlf} is struct virtio_crypto_mac_data_vlf.
    \end{itemize*}
\item If the the opcode (in \field{header}) is VIRTIO_CRYPTO_AEAD_ENCRYPT
    or VIRTIO_CRYPTO_AEAD_DECRYPT then:
    \begin{itemize*}
    \item If VIRTIO_CRYPTO_F_AEAD_STATELESS_MODE is negotiated, \field{op_flf} is
        struct virtio_crypto_aead_data_flf_stateless, and \field{op_vlf} is struct
        virtio_crypto_aead_data_vlf_stateless.
    \item If VIRTIO_CRYPTO_F_AEAD_STATELESS_MODE is NOT negotiated, \field{op_flf}
        is struct virtio_crypto_aead_data_flf if VIRTIO_CRYPTO_F_REVISION_1 is negotiated
        and struct virtio_crypto_aead_data_flf is padded to 48 bytes if NOT negotiated,
        and \field{op_vlf} is struct virtio_crypto_aead_data_vlf.
    \end{itemize*}
\item If the opcode (in \field{header}) is VIRTIO_CRYPTO_AKCIPHER_ENCRYPT, VIRTIO_CRYPTO_AKCIPHER_DECRYPT,
    VIRTIO_CRYPTO_AKCIPHER_SIGN or VIRTIO_CRYPTO_AKCIPHER_VERIFY then:
    \begin{itemize*}
    \item If VIRTIO_CRYPTO_F_AKCIPHER_STATELESS_MODE is negotiated, \field{op_flf} is
        struct virtio_crypto_akcipher_data_flf_statless, and \field{op_vlf} is struct
        virtio_crypto_akcipher_data_vlf_stateless.
    \item If VIRTIO_CRYPTO_F_AKCIPHER_STATELESS_MODE is NOT negotiated, \field{op_flf}
        is struct virtio_crypto_akcipher_data_flf if VIRTIO_CRYPTO_F_REVISION_1 is negotiated
        and struct virtio_crypto_akcipher_data_flf is padded to 48 bytes if NOT negotiated,
        and \field{op_vlf} is struct virtio_crypto_akcipher_data_vlf.
    \end{itemize*}
\end{itemize*}

\field{inhdr} is a unified input header that used to return the status of
the operations, is defined as follows:

\begin{lstlisting}
struct virtio_crypto_inhdr {
    u8 status;
};
\end{lstlisting}

\subsubsection{HASH Service Operation}\label{sec:Device Types / Crypto Device / Device Operation / HASH Service Operation}

Session mode HASH service requests are as follows:

\begin{lstlisting}
struct virtio_crypto_hash_data_flf {
    /* length of source data */
    le32 src_data_len;
    /* hash result length */
    le32 hash_result_len;
};

struct virtio_crypto_hash_data_vlf {
    /* Device read only portion */
    /* Source data */
    u8 src_data[src_data_len];

    /* Device write only portion */
    /* Hash result data */
    u8 hash_result[hash_result_len];
};
\end{lstlisting}

Each data request uses the virtio_crypto_hash_data_flf structure and the
virtio_crypto_hash_data_vlf structure to store information used to run the
HASH operations.

\field{src_data} is the source data that will be processed.
\field{src_data_len} is the length of source data.
\field{hash_result} is the result data and \field{hash_result_len} is the length
of it.

Stateless mode HASH service requests are as follows:

\begin{lstlisting}
struct virtio_crypto_hash_data_flf_stateless {
    struct {
        /* See VIRTIO_CRYPTO_HASH_* above */
        le32 algo;
    } sess_para;

    /* length of source data */
    le32 src_data_len;
    /* hash result length */
    le32 hash_result_len;
    le32 reserved;
};
struct virtio_crypto_hash_data_vlf_stateless {
    /* Device read only portion */
    /* Source data */
    u8 src_data[src_data_len];

    /* Device write only portion */
    /* Hash result data */
    u8 hash_result[hash_result_len];
};
\end{lstlisting}

\drivernormative{\paragraph}{HASH Service Operation}{Device Types / Crypto Device / Device Operation / HASH Service Operation}

\begin{itemize*}
\item If the driver uses the session mode, then the driver MUST set \field{session_id}
    in struct virtio_crypto_op_header to a valid value assigned by the device when the
    session was created.
\item If the VIRTIO_CRYPTO_F_HASH_STATELESS_MODE feature bit is negotiated, 1) if the
    driver uses the stateless mode, then the driver MUST set the \field{flag} field in
    struct virtio_crypto_op_header to ZERO and MUST set the fields in struct
    virtio_crypto_hash_data_flf_stateless.sess_para, 2) if the driver uses the session
    mode, then the driver MUST set the \field{flag} field in struct virtio_crypto_op_header
    to VIRTIO_CRYPTO_FLAG_SESSION_MODE.
\item The driver MUST set \field{opcode} in struct virtio_crypto_op_header to VIRTIO_CRYPTO_HASH.
\end{itemize*}

\devicenormative{\paragraph}{HASH Service Operation}{Device Types / Crypto Device / Device Operation / HASH Service Operation}

\begin{itemize*}
\item The device MUST use the corresponding structure according to the \field{opcode}
    in the data general header.
\item If the VIRTIO_CRYPTO_F_HASH_STATELESS_MODE feature bit is negotiated, the device
    MUST parse \field{flag} field in struct virtio_crypto_op_header in order to decide
    which mode the driver uses.
\item The device MUST copy the results of HASH operations in the hash_result[] if HASH
    operations success.
\item The device MUST set \field{status} in struct virtio_crypto_inhdr to one of the
    following values of enum VIRTIO_CRYPTO_STATUS:
\begin{itemize*}
\item VIRTIO_CRYPTO_OK if the operation success.
\item VIRTIO_CRYPTO_NOTSUPP if the requested algorithm or operation is unsupported.
\item VIRTIO_CRYPTO_INVSESS if the session ID invalid when in session mode.
\item VIRTIO_CRYPTO_ERR if any failure not mentioned above occurs.
\end{itemize*}
\end{itemize*}


\subsubsection{MAC Service Operation}\label{sec:Device Types / Crypto Device / Device Operation / MAC Service Operation}

Session mode MAC service requests are as follows:

\begin{lstlisting}
struct virtio_crypto_mac_data_flf {
    struct virtio_crypto_hash_data_flf hdr;
};

struct virtio_crypto_mac_data_vlf {
    /* Device read only portion */
    /* Source data */
    u8 src_data[src_data_len];

    /* Device write only portion */
    /* Hash result data */
    u8 hash_result[hash_result_len];
};
\end{lstlisting}

Each request uses the virtio_crypto_mac_data_flf structure and the
virtio_crypto_mac_data_vlf structure to store information used to run the
MAC operations.

\field{src_data} is the source data that will be processed.
\field{src_data_len} is the length of source data.
\field{hash_result} is the result data and \field{hash_result_len} is the length
of it.

Stateless mode MAC service requests are as follows:

\begin{lstlisting}
struct virtio_crypto_mac_data_flf_stateless {
    struct {
        /* See VIRTIO_CRYPTO_MAC_* above */
        le32 algo;
        /* length of authenticated key */
        le32 auth_key_len;
    } sess_para;

    /* length of source data */
    le32 src_data_len;
    /* hash result length */
    le32 hash_result_len;
};

struct virtio_crypto_mac_data_vlf_stateless {
    /* Device read only portion */
    /* The authenticated key */
    u8 auth_key[auth_key_len];
    /* Source data */
    u8 src_data[src_data_len];

    /* Device write only portion */
    /* Hash result data */
    u8 hash_result[hash_result_len];
};
\end{lstlisting}

\field{auth_key} is the authenticated key that will be used during the process.
\field{auth_key_len} is the length of the key.

\drivernormative{\paragraph}{MAC Service Operation}{Device Types / Crypto Device / Device Operation / MAC Service Operation}

\begin{itemize*}
\item If the driver uses the session mode, then the driver MUST set \field{session_id}
    in struct virtio_crypto_op_header to a valid value assigned by the device when the
    session was created.
\item If the VIRTIO_CRYPTO_F_MAC_STATELESS_MODE feature bit is negotiated, 1) if the
    driver uses the stateless mode, then the driver MUST set the \field{flag} field
    in struct virtio_crypto_op_header to ZERO and MUST set the fields in struct
    virtio_crypto_mac_data_flf_stateless.sess_para, 2) if the driver uses the session
    mode, then the driver MUST set the \field{flag} field in struct virtio_crypto_op_header
    to VIRTIO_CRYPTO_FLAG_SESSION_MODE.
\item The driver MUST set \field{opcode} in struct virtio_crypto_op_header to VIRTIO_CRYPTO_MAC.
\end{itemize*}

\devicenormative{\paragraph}{MAC Service Operation}{Device Types / Crypto Device / Device Operation / MAC Service Operation}

\begin{itemize*}
\item If the VIRTIO_CRYPTO_F_MAC_STATELESS_MODE feature bit is negotiated, the device
    MUST parse \field{flag} field in struct virtio_crypto_op_header in order to decide
	which mode the driver uses.
\item The device MUST copy the results of MAC operations in the hash_result[] if HASH
    operations success.
\item The device MUST set \field{status} in struct virtio_crypto_inhdr to one of the
    following values of enum VIRTIO_CRYPTO_STATUS:
\begin{itemize*}
\item VIRTIO_CRYPTO_OK if the operation success.
\item VIRTIO_CRYPTO_NOTSUPP if the requested algorithm or operation is unsupported.
\item VIRTIO_CRYPTO_INVSESS if the session ID invalid when in session mode.
\item VIRTIO_CRYPTO_ERR if any failure not mentioned above occurs.
\end{itemize*}
\end{itemize*}

\subsubsection{Symmetric algorithms Operation}\label{sec:Device Types / Crypto Device / Device Operation / Symmetric algorithms Operation}

Session mode CIPHER service requests are as follows:

\begin{lstlisting}
struct virtio_crypto_cipher_data_flf {
    /*
     * Byte Length of valid IV/Counter data pointed to by the below iv data.
     *
     * For block ciphers in CBC or F8 mode, or for Kasumi in F8 mode, or for
     *   SNOW3G in UEA2 mode, this is the length of the IV (which
     *   must be the same as the block length of the cipher).
     * For block ciphers in CTR mode, this is the length of the counter
     *   (which must be the same as the block length of the cipher).
     */
    le32 iv_len;
    /* length of source data */
    le32 src_data_len;
    /* length of destination data */
    le32 dst_data_len;
    le32 padding;
};

struct virtio_crypto_cipher_data_vlf {
    /* Device read only portion */

    /*
     * Initialization Vector or Counter data.
     *
     * For block ciphers in CBC or F8 mode, or for Kasumi in F8 mode, or for
     *   SNOW3G in UEA2 mode, this is the Initialization Vector (IV)
     *   value.
     * For block ciphers in CTR mode, this is the counter.
     * For AES-XTS, this is the 128bit tweak, i, from IEEE Std 1619-2007.
     *
     * The IV/Counter will be updated after every partial cryptographic
     * operation.
     */
    u8 iv[iv_len];
    /* Source data */
    u8 src_data[src_data_len];

    /* Device write only portion */
    /* Destination data */
    u8 dst_data[dst_data_len];
};
\end{lstlisting}

Session mode requests of algorithm chaining are as follows:

\begin{lstlisting}
struct virtio_crypto_alg_chain_data_flf {
    le32 iv_len;
    /* Length of source data */
    le32 src_data_len;
    /* Length of destination data */
    le32 dst_data_len;
    /* Starting point for cipher processing in source data */
    le32 cipher_start_src_offset;
    /* Length of the source data that the cipher will be computed on */
    le32 len_to_cipher;
    /* Starting point for hash processing in source data */
    le32 hash_start_src_offset;
    /* Length of the source data that the hash will be computed on */
    le32 len_to_hash;
    /* Length of the additional auth data */
    le32 aad_len;
    /* Length of the hash result */
    le32 hash_result_len;
    le32 reserved;
};

struct virtio_crypto_alg_chain_data_vlf {
    /* Device read only portion */

    /* Initialization Vector or Counter data */
    u8 iv[iv_len];
    /* Source data */
    u8 src_data[src_data_len];
    /* Additional authenticated data if exists */
    u8 aad[aad_len];

    /* Device write only portion */

    /* Destination data */
    u8 dst_data[dst_data_len];
    /* Hash result data */
    u8 hash_result[hash_result_len];
};
\end{lstlisting}

Session mode requests of symmetric algorithm are as follows:

\begin{lstlisting}
struct virtio_crypto_sym_data_flf {
    /* Device read only portion */

#define VIRTIO_CRYPTO_SYM_DATA_REQ_HDR_SIZE    40
    u8 op_type_flf[VIRTIO_CRYPTO_SYM_DATA_REQ_HDR_SIZE];

    /* See above VIRTIO_CRYPTO_SYM_OP_* */
    le32 op_type;
    le32 padding;
};

struct virtio_crypto_sym_data_vlf {
    u8 op_type_vlf[sym_para_len];
};
\end{lstlisting}

Each request uses the virtio_crypto_sym_data_flf structure and the
virtio_crypto_sym_data_flf structure to store information used to run the
CIPHER operations.

\field{op_type_flf} is the \field{op_type} specific header, it MUST starts
with or be one of the following structures:
\begin{itemize*}
\item struct virtio_crypto_cipher_data_flf
\item struct virtio_crypto_alg_chain_data_flf
\end{itemize*}

The length of \field{op_type_flf} is fixed to 40 bytes, the data of unused
part (if has) will be ignored.

\field{op_type_vlf} is the \field{op_type} specific parameters, it MUST starts
with or be one of the following structures:
\begin{itemize*}
\item struct virtio_crypto_cipher_data_vlf
\item struct virtio_crypto_alg_chain_data_vlf
\end{itemize*}

\field{sym_para_len} is the size of the specific structure used.

Stateless mode CIPHER service requests are as follows:

\begin{lstlisting}
struct virtio_crypto_cipher_data_flf_stateless {
    struct {
        /* See VIRTIO_CRYPTO_CIPHER* above */
        le32 algo;
        /* length of key */
        le32 key_len;

        /* See VIRTIO_CRYPTO_OP_* above */
        le32 op;
    } sess_para;

    /*
     * Byte Length of valid IV/Counter data pointed to by the below iv data.
     */
    le32 iv_len;
    /* length of source data */
    le32 src_data_len;
    /* length of destination data */
    le32 dst_data_len;
};

struct virtio_crypto_cipher_data_vlf_stateless {
    /* Device read only portion */

    /* The cipher key */
    u8 cipher_key[key_len];

    /* Initialization Vector or Counter data. */
    u8 iv[iv_len];
    /* Source data */
    u8 src_data[src_data_len];

    /* Device write only portion */
    /* Destination data */
    u8 dst_data[dst_data_len];
};
\end{lstlisting}

Stateless mode requests of algorithm chaining are as follows:

\begin{lstlisting}
struct virtio_crypto_alg_chain_data_flf_stateless {
    struct {
        /* See VIRTIO_CRYPTO_SYM_ALG_CHAIN_ORDER_* above */
        le32 alg_chain_order;
        /* length of the additional authenticated data in bytes */
        le32 aad_len;

        struct {
            /* See VIRTIO_CRYPTO_CIPHER* above */
            le32 algo;
            /* length of key */
            le32 key_len;
            /* See VIRTIO_CRYPTO_OP_* above */
            le32 op;
        } cipher;

        struct {
            /* See VIRTIO_CRYPTO_HASH_* or VIRTIO_CRYPTO_MAC_* above */
            le32 algo;
            /* length of authenticated key */
            le32 auth_key_len;
            /* See VIRTIO_CRYPTO_SYM_HASH_MODE_* above */
            le32 hash_mode;
        } hash;
    } sess_para;

    le32 iv_len;
    /* Length of source data */
    le32 src_data_len;
    /* Length of destination data */
    le32 dst_data_len;
    /* Starting point for cipher processing in source data */
    le32 cipher_start_src_offset;
    /* Length of the source data that the cipher will be computed on */
    le32 len_to_cipher;
    /* Starting point for hash processing in source data */
    le32 hash_start_src_offset;
    /* Length of the source data that the hash will be computed on */
    le32 len_to_hash;
    /* Length of the additional auth data */
    le32 aad_len;
    /* Length of the hash result */
    le32 hash_result_len;
    le32 reserved;
};

struct virtio_crypto_alg_chain_data_vlf_stateless {
    /* Device read only portion */

    /* The cipher key */
    u8 cipher_key[key_len];
    /* The auth key */
    u8 auth_key[auth_key_len];
    /* Initialization Vector or Counter data */
    u8 iv[iv_len];
    /* Additional authenticated data if exists */
    u8 aad[aad_len];
    /* Source data */
    u8 src_data[src_data_len];

    /* Device write only portion */

    /* Destination data */
    u8 dst_data[dst_data_len];
    /* Hash result data */
    u8 hash_result[hash_result_len];
};
\end{lstlisting}

Stateless mode requests of symmetric algorithm are as follows:

\begin{lstlisting}
struct virtio_crypto_sym_data_flf_stateless {
    /* Device read only portion */
#define VIRTIO_CRYPTO_SYM_DATE_REQ_HDR_STATELESS_SIZE    72
    u8 op_type_flf[VIRTIO_CRYPTO_SYM_DATE_REQ_HDR_STATELESS_SIZE];

    /* Device write only portion */
    /* See above VIRTIO_CRYPTO_SYM_OP_* */
    le32 op_type;
};

struct virtio_crypto_sym_data_vlf_stateless {
    u8 op_type_vlf[sym_para_len];
};
\end{lstlisting}

\field{op_type_flf} is the \field{op_type} specific header, it MUST starts
with or be one of the following structures:
\begin{itemize*}
\item struct virtio_crypto_cipher_data_flf_stateless
\item struct virtio_crypto_alg_chain_data_flf_stateless
\end{itemize*}

The length of \field{op_type_flf} is fixed to 72 bytes, the data of unused
part (if has) will be ignored.

\field{op_type_vlf} is the \field{op_type} specific parameters, it MUST starts
with or be one of the following structures:
\begin{itemize*}
\item struct virtio_crypto_cipher_data_vlf_stateless
\item struct virtio_crypto_alg_chain_data_vlf_stateless
\end{itemize*}

\field{sym_para_len} is the size of the specific structure used.

\drivernormative{\paragraph}{Symmetric algorithms Operation}{Device Types / Crypto Device / Device Operation / Symmetric algorithms Operation}

\begin{itemize*}
\item If the driver uses the session mode, then the driver MUST set \field{session_id}
    in struct virtio_crypto_op_header to a valid value assigned by the device when the
    session was created.
\item If the VIRTIO_CRYPTO_F_CIPHER_STATELESS_MODE feature bit is negotiated, 1) if the
    driver uses the stateless mode, then the driver MUST set the \field{flag} field in
    struct virtio_crypto_op_header to ZERO and MUST set the fields in struct
    virtio_crypto_cipher_data_flf_stateless.sess_para or struct
    virtio_crypto_alg_chain_data_flf_stateless.sess_para, 2) if the driver uses the
    session mode, then the driver MUST set the \field{flag} field in struct
    virtio_crypto_op_header to VIRTIO_CRYPTO_FLAG_SESSION_MODE.
\item The driver MUST set the \field{opcode} field in struct virtio_crypto_op_header
    to VIRTIO_CRYPTO_CIPHER_ENCRYPT or VIRTIO_CRYPTO_CIPHER_DECRYPT.
\item The driver MUST specify the fields of struct virtio_crypto_cipher_data_flf in
    struct virtio_crypto_sym_data_flf and struct virtio_crypto_cipher_data_vlf in
    struct virtio_crypto_sym_data_vlf if the request is based on VIRTIO_CRYPTO_SYM_OP_CIPHER.
\item The driver MUST specify the fields of struct virtio_crypto_alg_chain_data_flf
    in struct virtio_crypto_sym_data_flf and struct virtio_crypto_alg_chain_data_vlf
    in struct virtio_crypto_sym_data_vlf if the request is of the VIRTIO_CRYPTO_SYM_OP_ALGORITHM_CHAINING
    type.
\end{itemize*}

\devicenormative{\paragraph}{Symmetric algorithms Operation}{Device Types / Crypto Device / Device Operation / Symmetric algorithms Operation}

\begin{itemize*}
\item If the VIRTIO_CRYPTO_F_CIPHER_STATELESS_MODE feature bit is negotiated, the device
    MUST parse \field{flag} field in struct virtio_crypto_op_header in order to decide
	which mode the driver uses.
\item The device MUST parse the virtio_crypto_sym_data_req based on the \field{opcode}
    field in general header.
\item The device MUST parse the fields of struct virtio_crypto_cipher_data_flf in
    struct virtio_crypto_sym_data_flf and struct virtio_crypto_cipher_data_vlf in
    struct virtio_crypto_sym_data_vlf if the request is based on VIRTIO_CRYPTO_SYM_OP_CIPHER.
\item The device MUST parse the fields of struct virtio_crypto_alg_chain_data_flf
    in struct virtio_crypto_sym_data_flf and struct virtio_crypto_alg_chain_data_vlf
    in struct virtio_crypto_sym_data_vlf if the request is of the VIRTIO_CRYPTO_SYM_OP_ALGORITHM_CHAINING
    type.
\item The device MUST copy the result of cryptographic operation in the dst_data[] in
    both plain CIPHER mode and algorithms chain mode.
\item The device MUST check the \field{para}.\field{add_len} is bigger than 0 before
    parse the additional authenticated data in plain algorithms chain mode.
\item The device MUST copy the result of HASH/MAC operation in the hash_result[] is
    of the VIRTIO_CRYPTO_SYM_OP_ALGORITHM_CHAINING type.
\item The device MUST set the \field{status} field in struct virtio_crypto_inhdr to
    one of the following values of enum VIRTIO_CRYPTO_STATUS:
\begin{itemize*}
\item VIRTIO_CRYPTO_OK if the operation success.
\item VIRTIO_CRYPTO_NOTSUPP if the requested algorithm or operation is unsupported.
\item VIRTIO_CRYPTO_INVSESS if the session ID is invalid in session mode.
\item VIRTIO_CRYPTO_ERR if failure not mentioned above occurs.
\end{itemize*}
\end{itemize*}

\subsubsection{AEAD Service Operation}\label{sec:Device Types / Crypto Device / Device Operation / AEAD Service Operation}

Session mode requests of symmetric algorithm are as follows:

\begin{lstlisting}
struct virtio_crypto_aead_data_flf {
    /*
     * Byte Length of valid IV data.
     *
     * For GCM mode, this is either 12 (for 96-bit IVs) or 16, in which
     *   case iv points to J0.
     * For CCM mode, this is the length of the nonce, which can be in the
     *   range 7 to 13 inclusive.
     */
    le32 iv_len;
    /* length of additional auth data */
    le32 aad_len;
    /* length of source data */
    le32 src_data_len;
    /* length of dst data, this should be at least src_data_len + tag_len */
    le32 dst_data_len;
    /* Authentication tag length */
    le32 tag_len;
    le32 reserved;
};

struct virtio_crypto_aead_data_vlf {
    /* Device read only portion */

    /*
     * Initialization Vector data.
     *
     * For GCM mode, this is either the IV (if the length is 96 bits) or J0
     *   (for other sizes), where J0 is as defined by NIST SP800-38D.
     *   Regardless of the IV length, a full 16 bytes needs to be allocated.
     * For CCM mode, the first byte is reserved, and the nonce should be
     *   written starting at &iv[1] (to allow space for the implementation
     *   to write in the flags in the first byte).  Note that a full 16 bytes
     *   should be allocated, even though the iv_len field will have
     *   a value less than this.
     *
     * The IV will be updated after every partial cryptographic operation.
     */
    u8 iv[iv_len];
    /* Source data */
    u8 src_data[src_data_len];
    /* Additional authenticated data if exists */
    u8 aad[aad_len];

    /* Device write only portion */
    /* Pointer to output data */
    u8 dst_data[dst_data_len];
};
\end{lstlisting}

Each request uses the virtio_crypto_aead_data_flf structure and the
virtio_crypto_aead_data_flf structure to store information used to run the
AEAD operations.

Stateless mode AEAD service requests are as follows:

\begin{lstlisting}
struct virtio_crypto_aead_data_flf_stateless {
    struct {
        /* See VIRTIO_CRYPTO_AEAD_* above */
        le32 algo;
        /* length of key */
        le32 key_len;
        /* encrypt or decrypt, See above VIRTIO_CRYPTO_OP_* */
        le32 op;
    } sess_para;

    /* Byte Length of valid IV data. */
    le32 iv_len;
    /* Authentication tag length */
    le32 tag_len;
    /* length of additional auth data */
    le32 aad_len;
    /* length of source data */
    le32 src_data_len;
    /* length of dst data, this should be at least src_data_len + tag_len */
    le32 dst_data_len;
};

struct virtio_crypto_aead_data_vlf_stateless {
    /* Device read only portion */

    /* The cipher key */
    u8 key[key_len];
    /* Initialization Vector data. */
    u8 iv[iv_len];
    /* Source data */
    u8 src_data[src_data_len];
    /* Additional authenticated data if exists */
    u8 aad[aad_len];

    /* Device write only portion */
    /* Pointer to output data */
    u8 dst_data[dst_data_len];
};
\end{lstlisting}

\drivernormative{\paragraph}{AEAD Service Operation}{Device Types / Crypto Device / Device Operation / AEAD Service Operation}

\begin{itemize*}
\item If the driver uses the session mode, then the driver MUST set
    \field{session_id} in struct virtio_crypto_op_header to a valid value assigned
    by the device when the session was created.
\item If the VIRTIO_CRYPTO_F_AEAD_STATELESS_MODE feature bit is negotiated, 1) if
    the driver uses the stateless mode, then the driver MUST set the \field{flag}
    field in struct virtio_crypto_op_header to ZERO and MUST set the fields in
    struct virtio_crypto_aead_data_flf_stateless.sess_para, 2) if the driver uses
    the session mode, then the driver MUST set the \field{flag} field in struct
    virtio_crypto_op_header to VIRTIO_CRYPTO_FLAG_SESSION_MODE.
\item The driver MUST set the \field{opcode} field in struct virtio_crypto_op_header
    to VIRTIO_CRYPTO_AEAD_ENCRYPT or VIRTIO_CRYPTO_AEAD_DECRYPT.
\end{itemize*}

\devicenormative{\paragraph}{AEAD Service Operation}{Device Types / Crypto Device / Device Operation / AEAD Service Operation}

\begin{itemize*}
\item If the VIRTIO_CRYPTO_F_AEAD_STATELESS_MODE feature bit is negotiated, the
    device MUST parse the virtio_crypto_aead_data_vlf_stateless based on the \field{opcode}
	field in general header.
\item The device MUST copy the result of cryptographic operation in the dst_data[].
\item The device MUST copy the authentication tag in the dst_data[] offset the cipher result.
\item The device MUST set the \field{status} field in struct virtio_crypto_inhdr to
    one of the following values of enum VIRTIO_CRYPTO_STATUS:
\item When the \field{opcode} field is VIRTIO_CRYPTO_AEAD_DECRYPT, the device MUST
    verify and return the verification result to the driver.
\begin{itemize*}
\item VIRTIO_CRYPTO_OK if the operation success.
\item VIRTIO_CRYPTO_NOTSUPP if the requested algorithm or operation is unsupported.
\item VIRTIO_CRYPTO_BADMSG if the verification result is incorrect.
\item VIRTIO_CRYPTO_INVSESS if the session ID invalid when in session mode.
\item VIRTIO_CRYPTO_ERR if any failure not mentioned above occurs.
\end{itemize*}
\end{itemize*}

\subsubsection{AKCIPHER Service Operation}\label{sec:Device Types / Crypto Device / Device Operation / AKCIPHER Service Operation}

Session mode AKCIPHER requests are as follows:

\begin{lstlisting}
struct virtio_crypto_akcipher_data_flf {
    /* length of source data */
    le32 src_data_len;
    /* length of dst data */
    le32 dst_data_len;
};

struct virtio_crypto_akcipher_data_vlf {
    /* Device read only portion */
    /* Source data */
    u8 src_data[src_data_len];

    /* Device write only portion */
    /* Pointer to output data */
    u8 dst_data[dst_data_len];
};
\end{lstlisting}

Each data request uses the virtio_crypto_akcipher_flf structure and the virtio_crypto_akcipher_data_vlf
structure to store information used to run the AKCIPHER operations.

For encryption, decryption, and signing:
\field{src_data} is the source data that will be processed, note that for signing operations,
src_data stores the data to be signed, which usually is the digest of some data rather than the
data itself.
\field{src_data_len} is the length of source data.
\field{dst_result} is the result data and \field{dst_data_len} is the length of it. Note that the
length of the result is not always exactly equal to dst_data_len, the driver needs to check how
many bytes the device has written and calculate the actual length of the result.

For verification:
\field{src_data_len} refers to the length of the signature, and \field{dst_data_len} refers to
the length of signed data, where the signed data is usually the digest of some data.
\field{src_data} is spliced by the signature and the signed data, the src_data with the lower
address stores the signature, and the higher address stores the signed data.
\field{dst_data} is always empty for verification.

Different algorithms have different signature formats.
For the RSA algorithm, the result is determined by the padding algorithm specified by
\field{padding_algo} in structure virtio_crypto_rsa_session_para.

For the ECDSA algorithm, the signature is composed of the following
ASN.1 structure (see \hyperref[intro:rfc3279]{RFC3279})
and MUST be DER encoded (see \hyperref[intro:rfc6025]{rfc6025}).

\begin{lstlisting}
Ecdsa-Sig-Value ::= SEQUENCE {
    r INTEGER,
    s INTEGER
}
\end{lstlisting}

Stateless mode AKCIPHER service requests are as follows:
\begin{lstlisting}
struct virtio_crypto_akcipher_data_flf_stateless {
    struct {
        /* See VIRTIO_CYRPTO_AKCIPHER* above */
        le32 algo;
        /* See VIRTIO_CRYPTO_AKCIPHER_KEY_TYPE_* above */
        le32 key_type;
        /* length of key */
        le32 key_len;

        /* algothrim specific parameters described above */
        union {
            struct virtio_crypto_rsa_session_para rsa;
            struct virtio_crypto_ecdsa_session_para ecdsa;
        } u;
    } sess_para;

    /* length of source data */
    le32 src_data_len;
    /* length of destination data */
    le32 dst_data_len;
};

struct virtio_crypto_akcipher_data_vlf_stateless {
    /* Device read only portion */
    u8 akcipher_key[key_len];

    /* Source data */
    u8 src_data[src_data_len];

    /* Device write only portion */
    u8 dst_data[dst_data_len];
};
\end{lstlisting}

In stateless mode, the format of key and signature, the meaning of src_data and dst_data, are all the same
with session mode.

\drivernormative{\paragraph}{AKCIPHER Service Operation}{Device Types / Crypto Device / Device Operation / AKCIPHER Service Operation}

\begin{itemize*}
\item If the driver uses the session mode, then the driver MUST set
    \field{session_id} in struct virtio_crypto_op_header to a valid
    value assigned by the device when the session was created.
\item If the VIRTIO_CRYPTO_F_AKCIPHER_STATELESS_MODE feature bit is negotiated, 1) if the
    driver uses the stateless mode, then the driver MUST set the \field{flag} field in
    struct virtio_crypto_op_header to ZERO and MUST set the fields in struct
    virtio_crypto_akcipher_flf_stateless.sess_para, 2) if the driver uses the session
    mode, then the driver MUST set the \field{flag} field in struct virtio_crypto_op_header
    to VIRTIO_CRYPTO_FLAG_SESSION_MODE.
\item The driver MUST set the \field{opcode} field in struct virtio_crypto_op_header
    to one of VIRTIO_CRYPTO_AKCIPHER_ENCRYPT, VIRTIO_CRYPTO_AKCIPHER_DESTROY_SESSION,
    VIRTIO_CRYPTO_AKCIPHER_SIGN, and VIRTIO_CRYPTO_AKCIPHER_VERIFY.
\end{itemize*}

\devicenormative{\paragraph}{AKCIPHER Service Operation}{Device Types / Crypto Device / Device Operation / AKCIPHER Service Operation}

\begin{itemize*}
\item If the VIRTIO_CRYPTO_F_AKCIPHER_STATELESS_MODE feature bit is negotiated, the
    device MUST parse the virtio_crypto_akcipher_data_vlf_stateless based on the \field{opcode}
    field in general header.
\item The device MUST copy the result of cryptographic operation in the dst_data[].
\item The device MUST set the \field{status} field in struct virtio_crypto_inhdr to
    one of the following values of enum VIRTIO_CRYPTO_STATUS:
\begin{itemize*}
\item VIRTIO_CRYPTO_OK if the operation success.
\item VIRTIO_CRYPTO_NOTSUPP if the requested algorithm or operation is unsupported.
\item VIRTIO_CRYPTO_BADMSG if the verification result is incorrect.
\item VIRTIO_CRYPTO_INVSESS if the session ID invalid when in session mode.
\item VIRTIO_CRYPTO_KEY_REJECTED if the signature verification failed.
\item VIRTIO_CRYPTO_ERR if any failure not mentioned above occurs.
\end{itemize*}
\end{itemize*}

\section{Crypto Device}\label{sec:Device Types / Crypto Device}

The virtio crypto device is a virtual cryptography device as well as a
virtual cryptographic accelerator. The virtio crypto device provides the
following crypto services: CIPHER, MAC, HASH, AEAD and AKCIPHER. Virtio crypto
devices have a single control queue and at least one data queue. Crypto
operation requests are placed into a data queue, and serviced by the
device. Some crypto operation requests are only valid in the context of a
session. The role of the control queue is facilitating control operation
requests. Sessions management is realized with control operation
requests.

\subsection{Device ID}\label{sec:Device Types / Crypto Device / Device ID}

20

\subsection{Virtqueues}\label{sec:Device Types / Crypto Device / Virtqueues}

\begin{description}
\item[0] dataq1
\item[\ldots]
\item[N-1] dataqN
\item[N] controlq
\end{description}

N is set by \field{max_dataqueues}.

\subsection{Feature bits}\label{sec:Device Types / Crypto Device / Feature bits}

\begin{description}
\item VIRTIO_CRYPTO_F_REVISION_1 (0) revision 1. Revision 1 has a specific
    request format and other enhancements (which result in some additional
    requirements).
\item VIRTIO_CRYPTO_F_CIPHER_STATELESS_MODE (1) stateless mode requests are
    supported by the CIPHER service.
\item VIRTIO_CRYPTO_F_HASH_STATELESS_MODE (2) stateless mode requests are
    supported by the HASH service.
\item VIRTIO_CRYPTO_F_MAC_STATELESS_MODE (3) stateless mode requests are
    supported by the MAC service.
\item VIRTIO_CRYPTO_F_AEAD_STATELESS_MODE (4) stateless mode requests are
    supported by the AEAD service.
\item VIRTIO_CRYPTO_F_AKCIPHER_STATELESS_MODE (5) stateless mode requests are
    supported by the AKCIPHER service.
\end{description}


\subsubsection{Feature bit requirements}\label{sec:Device Types / Crypto Device / Feature bit requirements}

Some crypto feature bits require other crypto feature bits
(see \ref{drivernormative:Basic Facilities of a Virtio Device / Feature Bits}):

\begin{description}
\item[VIRTIO_CRYPTO_F_CIPHER_STATELESS_MODE] Requires VIRTIO_CRYPTO_F_REVISION_1.
\item[VIRTIO_CRYPTO_F_HASH_STATELESS_MODE] Requires VIRTIO_CRYPTO_F_REVISION_1.
\item[VIRTIO_CRYPTO_F_MAC_STATELESS_MODE] Requires VIRTIO_CRYPTO_F_REVISION_1.
\item[VIRTIO_CRYPTO_F_AEAD_STATELESS_MODE] Requires VIRTIO_CRYPTO_F_REVISION_1.
\item[VIRTIO_CRYPTO_F_AKCIPHER_STATELESS_MODE] Requires VIRTIO_CRYPTO_F_REVISION_1.
\end{description}

\subsection{Supported crypto services}\label{sec:Device Types / Crypto Device / Supported crypto services}

The following crypto services are defined:

\begin{lstlisting}
/* CIPHER (Symmetric Key Cipher) service */
#define VIRTIO_CRYPTO_SERVICE_CIPHER 0
/* HASH service */
#define VIRTIO_CRYPTO_SERVICE_HASH   1
/* MAC (Message Authentication Codes) service */
#define VIRTIO_CRYPTO_SERVICE_MAC    2
/* AEAD (Authenticated Encryption with Associated Data) service */
#define VIRTIO_CRYPTO_SERVICE_AEAD   3
/* AKCIPHER (Asymmetric Key Cipher) service */
#define VIRTIO_CRYPTO_SERVICE_AKCIPHER 4
\end{lstlisting}

The above constants designate bits used to indicate the which of crypto services are
offered by the device as described in, see \ref{sec:Device Types / Crypto Device / Device configuration layout}.

\subsubsection{CIPHER services}\label{sec:Device Types / Crypto Device / Supported crypto services / CIPHER services}

The following CIPHER algorithms are defined:

\begin{lstlisting}
#define VIRTIO_CRYPTO_NO_CIPHER                 0
#define VIRTIO_CRYPTO_CIPHER_ARC4               1
#define VIRTIO_CRYPTO_CIPHER_AES_ECB            2
#define VIRTIO_CRYPTO_CIPHER_AES_CBC            3
#define VIRTIO_CRYPTO_CIPHER_AES_CTR            4
#define VIRTIO_CRYPTO_CIPHER_DES_ECB            5
#define VIRTIO_CRYPTO_CIPHER_DES_CBC            6
#define VIRTIO_CRYPTO_CIPHER_3DES_ECB           7
#define VIRTIO_CRYPTO_CIPHER_3DES_CBC           8
#define VIRTIO_CRYPTO_CIPHER_3DES_CTR           9
#define VIRTIO_CRYPTO_CIPHER_KASUMI_F8          10
#define VIRTIO_CRYPTO_CIPHER_SNOW3G_UEA2        11
#define VIRTIO_CRYPTO_CIPHER_AES_F8             12
#define VIRTIO_CRYPTO_CIPHER_AES_XTS            13
#define VIRTIO_CRYPTO_CIPHER_ZUC_EEA3           14
\end{lstlisting}

The above constants have two usages:
\begin{enumerate}
\item As bit numbers, used to tell the driver which CIPHER algorithms
are supported by the device, see \ref{sec:Device Types / Crypto Device / Device configuration layout}.
\item As values, used to designate the algorithm in (CIPHER type) crypto
operation requests, see \ref{sec:Device Types / Crypto Device / Device Operation / Control Virtqueue / Session operation}.
\end{enumerate}

\subsubsection{HASH services}\label{sec:Device Types / Crypto Device / Supported crypto services / HASH services}

The following HASH algorithms are defined:

\begin{lstlisting}
#define VIRTIO_CRYPTO_NO_HASH            0
#define VIRTIO_CRYPTO_HASH_MD5           1
#define VIRTIO_CRYPTO_HASH_SHA1          2
#define VIRTIO_CRYPTO_HASH_SHA_224       3
#define VIRTIO_CRYPTO_HASH_SHA_256       4
#define VIRTIO_CRYPTO_HASH_SHA_384       5
#define VIRTIO_CRYPTO_HASH_SHA_512       6
#define VIRTIO_CRYPTO_HASH_SHA3_224      7
#define VIRTIO_CRYPTO_HASH_SHA3_256      8
#define VIRTIO_CRYPTO_HASH_SHA3_384      9
#define VIRTIO_CRYPTO_HASH_SHA3_512      10
#define VIRTIO_CRYPTO_HASH_SHA3_SHAKE128      11
#define VIRTIO_CRYPTO_HASH_SHA3_SHAKE256      12
\end{lstlisting}

The above constants have two usages:
\begin{enumerate}
\item As bit numbers, used to tell the driver which HASH algorithms
are supported by the device, see \ref{sec:Device Types / Crypto Device / Device configuration layout}.
\item As values, used to designate the algorithm in (HASH type) crypto
operation requires, see \ref{sec:Device Types / Crypto Device / Device Operation / Control Virtqueue / Session operation}.
\end{enumerate}

\subsubsection{MAC services}\label{sec:Device Types / Crypto Device / Supported crypto services / MAC services}

The following MAC algorithms are defined:

\begin{lstlisting}
#define VIRTIO_CRYPTO_NO_MAC                       0
#define VIRTIO_CRYPTO_MAC_HMAC_MD5                 1
#define VIRTIO_CRYPTO_MAC_HMAC_SHA1                2
#define VIRTIO_CRYPTO_MAC_HMAC_SHA_224             3
#define VIRTIO_CRYPTO_MAC_HMAC_SHA_256             4
#define VIRTIO_CRYPTO_MAC_HMAC_SHA_384             5
#define VIRTIO_CRYPTO_MAC_HMAC_SHA_512             6
#define VIRTIO_CRYPTO_MAC_CMAC_3DES                25
#define VIRTIO_CRYPTO_MAC_CMAC_AES                 26
#define VIRTIO_CRYPTO_MAC_KASUMI_F9                27
#define VIRTIO_CRYPTO_MAC_SNOW3G_UIA2              28
#define VIRTIO_CRYPTO_MAC_GMAC_AES                 41
#define VIRTIO_CRYPTO_MAC_GMAC_TWOFISH             42
#define VIRTIO_CRYPTO_MAC_CBCMAC_AES               49
#define VIRTIO_CRYPTO_MAC_CBCMAC_KASUMI_F9         50
#define VIRTIO_CRYPTO_MAC_XCBC_AES                 53
#define VIRTIO_CRYPTO_MAC_ZUC_EIA3                 54
\end{lstlisting}

The above constants have two usages:
\begin{enumerate}
\item As bit numbers, used to tell the driver which MAC algorithms
are supported by the device, see \ref{sec:Device Types / Crypto Device / Device configuration layout}.
\item As values, used to designate the algorithm in (MAC type) crypto
operation requests, see \ref{sec:Device Types / Crypto Device / Device Operation / Control Virtqueue / Session operation}.
\end{enumerate}

\subsubsection{AEAD services}\label{sec:Device Types / Crypto Device / Supported crypto services / AEAD services}

The following AEAD algorithms are defined:

\begin{lstlisting}
#define VIRTIO_CRYPTO_NO_AEAD     0
#define VIRTIO_CRYPTO_AEAD_GCM    1
#define VIRTIO_CRYPTO_AEAD_CCM    2
#define VIRTIO_CRYPTO_AEAD_CHACHA20_POLY1305  3
\end{lstlisting}

The above constants have two usages:
\begin{enumerate}
\item As bit numbers, used to tell the driver which AEAD algorithms
are supported by the device, see \ref{sec:Device Types / Crypto Device / Device configuration layout}.
\item As values, used to designate the algorithm in (DEAD type) crypto
operation requests, see \ref{sec:Device Types / Crypto Device / Device Operation / Control Virtqueue / Session operation}.
\end{enumerate}

\subsubsection{AKCIPHER services}\label{sec: Device Types / Crypto Device / Supported crypto services / AKCIPHER services}

The following AKCIPHER algorithms are defined:
\begin{lstlisting}
#define VIRTIO_CRYPTO_NO_AKCIPHER 0
#define VIRTIO_CRYPTO_AKCIPHER_RSA   1
#define VIRTIO_CRYPTO_AKCIPHER_ECDSA 2
\end{lstlisting}

The above constants have two usages:
\begin{enumerate}
\item As bit numbers, used to tell the driver which AKCIPHER algorithms
are supported by the device, see \ref{sec:Device Types / Crypto Device / Device configuration layout}.
\item As values, used to designate the algorithm in asymmetric crypto operation requests,
see \ref{sec:Device Types / Crypto Device / Device Operation / Control Virtqueue / Session operation}.
\end{enumerate}


\subsection{Device configuration layout}\label{sec:Device Types / Crypto Device / Device configuration layout}

Crypto device configuration uses the following layout structure:

\begin{lstlisting}
struct virtio_crypto_config {
    le32 status;
    le32 max_dataqueues;
    le32 crypto_services;
    /* Detailed algorithms mask */
    le32 cipher_algo_l;
    le32 cipher_algo_h;
    le32 hash_algo;
    le32 mac_algo_l;
    le32 mac_algo_h;
    le32 aead_algo;
    /* Maximum length of cipher key in bytes */
    le32 max_cipher_key_len;
    /* Maximum length of authenticated key in bytes */
    le32 max_auth_key_len;
    le32 akcipher_algo;
    /* Maximum size of each crypto request's content in bytes */
    le64 max_size;
};
\end{lstlisting}

\begin{description}
\item Currently, only one \field{status} bit is defined: VIRTIO_CRYPTO_S_HW_READY
    set indicates that the device is ready to process requests, this bit is read-only
    for the driver
\begin{lstlisting}
#define VIRTIO_CRYPTO_S_HW_READY  (1 << 0)
\end{lstlisting}

\item [\field{max_dataqueues}] is the maximum number of data virtqueues that can
    be configured by the device. The driver MAY use only one data queue, or it
    can use more to achieve better performance.

\item [\field{crypto_services}] crypto service offered, see \ref{sec:Device Types / Crypto Device / Supported crypto services}.

\item [\field{cipher_algo_l}] CIPHER algorithms bits 0-31, see \ref{sec:Device Types / Crypto Device / Supported crypto services  / CIPHER services}.

\item [\field{cipher_algo_h}] CIPHER algorithms bits 32-63, see \ref{sec:Device Types / Crypto Device / Supported crypto services  / CIPHER services}.

\item [\field{hash_algo}] HASH algorithms bits, see \ref{sec:Device Types / Crypto Device / Supported crypto services  / HASH services}.

\item [\field{mac_algo_l}] MAC algorithms bits 0-31, see \ref{sec:Device Types / Crypto Device / Supported crypto services  / MAC services}.

\item [\field{mac_algo_h}] MAC algorithms bits 32-63, see \ref{sec:Device Types / Crypto Device / Supported crypto services  / MAC services}.

\item [\field{aead_algo}] AEAD algorithms bits, see \ref{sec:Device Types / Crypto Device / Supported crypto services  / AEAD services}.

\item [\field{max_cipher_key_len}] is the maximum length of cipher key supported by the device.

\item [\field{max_auth_key_len}] is the maximum length of authenticated key supported by the device.

\item [\field{akcipher_algo}] AKCIPHER algorithms bit 0-31, see \ref{sec: Device Types / Crypto Device / Supported crypto services / AKCIPHER services}.

\item [\field{max_size}] is the maximum size of the variable-length parameters of
    data operation of each crypto request's content supported by the device.
\end{description}

\begin{note}
Unless explicitly stated otherwise all lengths and sizes are in bytes.
\end{note}

\devicenormative{\subsubsection}{Device configuration layout}{Device Types / Crypto Device / Device configuration layout}

\begin{itemize*}
\item The device MUST set \field{max_dataqueues} to between 1 and 65535 inclusive.
\item The device MUST set the \field{status} with valid flags, undefined flags MUST NOT be set.
\item The device MUST accept and handle requests after \field{status} is set to VIRTIO_CRYPTO_S_HW_READY.
\item The device MUST set \field{crypto_services} based on the crypto services the device offers.
\item The device MUST set detailed algorithms masks for each service advertised by \field{crypto_services}.
    The device MUST NOT set the not defined algorithms bits.
\item The device MUST set \field{max_size} to show the maximum size of crypto request the device supports.
\item The device MUST set \field{max_cipher_key_len} to show the maximum length of cipher key if the
    device supports CIPHER service.
\item The device MUST set \field{max_auth_key_len} to show the maximum length of authenticated key if
    the device supports MAC service.
\end{itemize*}

\drivernormative{\subsubsection}{Device configuration layout}{Device Types / Crypto Device / Device configuration layout}

\begin{itemize*}
\item The driver MUST read the \field{status} from the bottom bit of status to check whether the
    VIRTIO_CRYPTO_S_HW_READY is set, and the driver MUST reread it after device reset.
\item The driver MUST NOT transmit any requests to the device if the VIRTIO_CRYPTO_S_HW_READY is not set.
\item The driver MUST read \field{max_dataqueues} field to discover the number of data queues the device supports.
\item The driver MUST read \field{crypto_services} field to discover which services the device is able to offer.
\item The driver SHOULD ignore the not defined algorithms bits.
\item The driver MUST read the detailed algorithms fields based on \field{crypto_services} field.
\item The driver SHOULD read \field{max_size} to discover the maximum size of the variable-length
    parameters of data operation of the crypto request's content the device supports and MUST
    guarantee that the size of each crypto request's content is within the \field{max_size}, otherwise
    the request will fail and the driver MUST reset the device.
\item The driver SHOULD read \field{max_cipher_key_len} to discover the maximum length of cipher key
    the device supports and MUST guarantee that the \field{key_len} (CIPHER service or AEAD service) is within
    the \field{max_cipher_key_len} of the device configuration, otherwise the request will fail.
\item The driver SHOULD read \field{max_auth_key_len} to discover the maximum length of authenticated
    key the device supports and MUST guarantee that the \field{auth_key_len} (MAC service) is within the
    \field{max_auth_key_len} of the device configuration, otherwise the request will fail.
\end{itemize*}

\subsection{Device Initialization}\label{sec:Device Types / Crypto Device / Device Initialization}

\drivernormative{\subsubsection}{Device Initialization}{Device Types / Crypto Device / Device Initialization}

\begin{itemize*}
\item The driver MUST configure and initialize all virtqueues.
\item The driver MUST read the supported crypto services from bits of \field{crypto_services}.
\item The driver MUST read the supported algorithms based on \field{crypto_services} field.
\end{itemize*}

\subsection{Device Operation}\label{sec:Device Types / Crypto Device / Device Operation}

The operation of a virtio crypto device is driven by requests placed on the virtqueues.
Requests consist of a queue-type specific header (specifying among others the operation)
and an operation specific payload.

If VIRTIO_CRYPTO_F_REVISION_1 is negotiated the device may support both session mode
(See \ref{sec:Device Types / Crypto Device / Device Operation / Control Virtqueue / Session operation})
and stateless mode operation requests.
In stateless mode all operation parameters are supplied as a part of each request,
while in session mode, some or all operation parameters are managed within the
session. Stateless mode is guarded by feature bits 0-4 on a service level. If
stateless mode is negotiated for a service, the service accepts both session
mode and stateless requests; otherwise stateless mode requests are rejected
(via operation status).

\subsubsection{Operation Status}\label{sec:Device Types / Crypto Device / Device Operation / Operation status}
The device MUST return a status code as part of the operation (both session
operation and service operation) result. The valid operation status as follows:

\begin{lstlisting}
enum VIRTIO_CRYPTO_STATUS {
    VIRTIO_CRYPTO_OK = 0,
    VIRTIO_CRYPTO_ERR = 1,
    VIRTIO_CRYPTO_BADMSG = 2,
    VIRTIO_CRYPTO_NOTSUPP = 3,
    VIRTIO_CRYPTO_INVSESS = 4,
    VIRTIO_CRYPTO_NOSPC = 5,
    VIRTIO_CRYPTO_KEY_REJECTED = 6,
    VIRTIO_CRYPTO_MAX
};
\end{lstlisting}

\begin{itemize*}
\item VIRTIO_CRYPTO_OK: success.
\item VIRTIO_CRYPTO_BADMSG: authentication failed (only when AEAD decryption).
\item VIRTIO_CRYPTO_NOTSUPP: operation or algorithm is unsupported.
\item VIRTIO_CRYPTO_INVSESS: invalid session ID when executing crypto operations.
\item VIRTIO_CRYPTO_NOSPC: no free session ID (only when the VIRTIO_CRYPTO_F_REVISION_1
    feature bit is negotiated).
\item VIRTIO_CRYPTO_KEY_REJECTED: signature verification failed (only when AKCIPHER verification).
\item VIRTIO_CRYPTO_ERR: any failure not mentioned above occurs.
\end{itemize*}

\subsubsection{Control Virtqueue}\label{sec:Device Types / Crypto Device / Device Operation / Control Virtqueue}

The driver uses the control virtqueue to send control commands to the
device, such as session operations (See \ref{sec:Device Types / Crypto Device / Device
Operation / Control Virtqueue / Session operation}).

The header for controlq is of the following form:
\begin{lstlisting}
#define VIRTIO_CRYPTO_OPCODE(service, op)   (((service) << 8) | (op))

struct virtio_crypto_ctrl_header {
#define VIRTIO_CRYPTO_CIPHER_CREATE_SESSION \
       VIRTIO_CRYPTO_OPCODE(VIRTIO_CRYPTO_SERVICE_CIPHER, 0x02)
#define VIRTIO_CRYPTO_CIPHER_DESTROY_SESSION \
       VIRTIO_CRYPTO_OPCODE(VIRTIO_CRYPTO_SERVICE_CIPHER, 0x03)
#define VIRTIO_CRYPTO_HASH_CREATE_SESSION \
       VIRTIO_CRYPTO_OPCODE(VIRTIO_CRYPTO_SERVICE_HASH, 0x02)
#define VIRTIO_CRYPTO_HASH_DESTROY_SESSION \
       VIRTIO_CRYPTO_OPCODE(VIRTIO_CRYPTO_SERVICE_HASH, 0x03)
#define VIRTIO_CRYPTO_MAC_CREATE_SESSION \
       VIRTIO_CRYPTO_OPCODE(VIRTIO_CRYPTO_SERVICE_MAC, 0x02)
#define VIRTIO_CRYPTO_MAC_DESTROY_SESSION \
       VIRTIO_CRYPTO_OPCODE(VIRTIO_CRYPTO_SERVICE_MAC, 0x03)
#define VIRTIO_CRYPTO_AEAD_CREATE_SESSION \
       VIRTIO_CRYPTO_OPCODE(VIRTIO_CRYPTO_SERVICE_AEAD, 0x02)
#define VIRTIO_CRYPTO_AEAD_DESTROY_SESSION \
       VIRTIO_CRYPTO_OPCODE(VIRTIO_CRYPTO_SERVICE_AEAD, 0x03)
#define VIRTIO_CRYPTO_AKCIPHER_CREATE_SESSION \
       VIRTIO_CRYPTO_OPCODE(VIRTIO_CRYPTO_SERVICE_AKCIPHER, 0x04)
#define VIRTIO_CRYPTO_AKCIPHER_DESTROY_SESSION \
       VIRTIO_CRYPTO_OPCDE(VIRTIO_CRYPTO_SERVICE_AKCIPHER, 0x05)
    le32 opcode;
    /* algo should be service-specific algorithms */
    le32 algo;
    le32 flag;
    le32 reserved;
};
\end{lstlisting}

The controlq request is composed of four parts:
\begin{lstlisting}
struct virtio_crypto_op_ctrl_req {
    /* Device read only portion */

    struct virtio_crypto_ctrl_header header;

#define VIRTIO_CRYPTO_CTRLQ_OP_SPEC_HDR_LEGACY 56
    /* fixed length fields, opcode specific */
    u8 op_flf[flf_len];

    /* variable length fields, opcode specific */
    u8 op_vlf[vlf_len];

    /* Device write only portion */

    /* op result or completion status */
    u8 op_outcome[outcome_len];
};
\end{lstlisting}

\field{header} is a general header (see above).

\field{op_flf} is the opcode (in \field{header}) specific fixed-length parameters.

\field{flf_len} depends on the VIRTIO_CRYPTO_F_REVISION_1 feature bit (see below).

\field{op_vlf} is the opcode (in \field{header}) specific variable-length parameters.

\field{vlf_len} is the size of the specific structure used.
\begin{note}
The \field{vlf_len} of session-destroy operation and the hash-session-create
operation is ZERO.
\end{note}

\begin{itemize*}
\item If the opcode (in \field{header}) is VIRTIO_CRYPTO_CIPHER_CREATE_SESSION
    then \field{op_flf} is struct virtio_crypto_sym_create_session_flf if
    VIRTIO_CRYPTO_F_REVISION_1 is negotiated and struct virtio_crypto_sym_create_session_flf is
    padded to 56 bytes if NOT negotiated, and \field{op_vlf} is struct
    virtio_crypto_sym_create_session_vlf.
\item If the opcode (in \field{header}) is VIRTIO_CRYPTO_HASH_CREATE_SESSION
    then \field{op_flf} is struct virtio_crypto_hash_create_session_flf if
    VIRTIO_CRYPTO_F_REVISION_1 is negotiated and struct virtio_crypto_hash_create_session_flf is
    padded to 56 bytes if NOT negotiated.
\item If the opcode (in \field{header}) is VIRTIO_CRYPTO_MAC_CREATE_SESSION
    then \field{op_flf} is struct virtio_crypto_mac_create_session_flf if
    VIRTIO_CRYPTO_F_REVISION_1 is negotiated and struct virtio_crypto_mac_create_session_flf is
    padded to 56 bytes if NOT negotiated, and \field{op_vlf} is struct
    virtio_crypto_mac_create_session_vlf.
\item If the opcode (in \field{header}) is VIRTIO_CRYPTO_AEAD_CREATE_SESSION
    then \field{op_flf} is struct virtio_crypto_aead_create_session_flf if
    VIRTIO_CRYPTO_F_REVISION_1 is negotiated and struct virtio_crypto_aead_create_session_flf is
    padded to 56 bytes if NOT negotiated, and \field{op_vlf} is struct
    virtio_crypto_aead_create_session_vlf.
\item If the opcode (in \field{header}) is VIRTIO_CRYPTO_AKCIPHER_CREATE_SESSION
    then \field{op_flf} is struct virtio_crypto_akcipher_create_session_flf if
    VIRTIO_CRYPTO_F_REVISION_1 is negotiated and struct virtio_crypto_akcipher_create_session_flf is
    padded to 56 bytes if NOT negotiated, and \field{op_vlf} is struct
    virtio_crypto_akcipher_create_session_vlf.
\item If the opcode (in \field{header}) is VIRTIO_CRYPTO_CIPHER_DESTROY_SESSION
    or VIRTIO_CRYPTO_HASH_DESTROY_SESSION or VIRTIO_CRYPTO_MAC_DESTROY_SESSION or
    VIRTIO_CRYPTO_AEAD_DESTROY_SESSION then \field{op_flf} is struct
    virtio_crypto_destroy_session_flf if VIRTIO_CRYPTO_F_REVISION_1 is negotiated and
    struct virtio_crypto_destroy_session_flf is padded to 56 bytes if NOT negotiated.
\end{itemize*}

\field{op_outcome} stores the result of operation and must be struct
virtio_crypto_destroy_session_input for destroy session or
struct virtio_crypto_create_session_input for create session.

\field{outcome_len} is the size of the structure used.


\paragraph{Session operation}\label{sec:Device Types / Crypto Device / Device
Operation / Control Virtqueue / Session operation}

The session is a handle which describes the cryptographic parameters to be
applied to a number of buffers.

The following structure stores the result of session creation set by the device:

\begin{lstlisting}
struct virtio_crypto_create_session_input {
    le64 session_id;
    le32 status;
    le32 padding;
};
\end{lstlisting}

A request to destroy a session includes the following information:

\begin{lstlisting}
struct virtio_crypto_destroy_session_flf {
    /* Device read only portion */
    le64  session_id;
};

struct virtio_crypto_destroy_session_input {
    /* Device write only portion */
    u8  status;
};
\end{lstlisting}


\subparagraph{Session operation: HASH session}\label{sec:Device Types / Crypto Device / Device
Operation / Control Virtqueue / Session operation / Session operation: HASH session}

The fixed-length parameters of HASH session requests is as follows:

\begin{lstlisting}
struct virtio_crypto_hash_create_session_flf {
    /* Device read only portion */

    /* See VIRTIO_CRYPTO_HASH_* above */
    le32 algo;
    /* hash result length */
    le32 hash_result_len;
};
\end{lstlisting}


\subparagraph{Session operation: MAC session}\label{sec:Device Types / Crypto Device / Device
Operation / Control Virtqueue / Session operation / Session operation: MAC session}

The fixed-length and the variable-length parameters of MAC session requests are as follows:

\begin{lstlisting}
struct virtio_crypto_mac_create_session_flf {
    /* Device read only portion */

    /* See VIRTIO_CRYPTO_MAC_* above */
    le32 algo;
    /* hash result length */
    le32 hash_result_len;
    /* length of authenticated key */
    le32 auth_key_len;
    le32 padding;
};

struct virtio_crypto_mac_create_session_vlf {
    /* Device read only portion */

    /* The authenticated key */
    u8 auth_key[auth_key_len];
};
\end{lstlisting}

The length of \field{auth_key} is specified in \field{auth_key_len} in the struct
virtio_crypto_mac_create_session_flf.


\subparagraph{Session operation: Symmetric algorithms session}\label{sec:Device Types / Crypto Device / Device
Operation / Control Virtqueue / Session operation / Session operation: Symmetric algorithms session}

The request of symmetric session could be the CIPHER algorithms request
or the chain algorithms (chaining CIPHER and HASH/MAC) request.

The fixed-length and the variable-length parameters of CIPHER session requests are as follows:

\begin{lstlisting}
struct virtio_crypto_cipher_session_flf {
    /* Device read only portion */

    /* See VIRTIO_CRYPTO_CIPHER* above */
    le32 algo;
    /* length of key */
    le32 key_len;
#define VIRTIO_CRYPTO_OP_ENCRYPT  1
#define VIRTIO_CRYPTO_OP_DECRYPT  2
    /* encryption or decryption */
    le32 op;
    le32 padding;
};

struct virtio_crypto_cipher_session_vlf {
    /* Device read only portion */

    /* The cipher key */
    u8 cipher_key[key_len];
};
\end{lstlisting}

The length of \field{cipher_key} is specified in \field{key_len} in the struct
virtio_crypto_cipher_session_flf.

The fixed-length and the variable-length parameters of Chain session requests are as follows:

\begin{lstlisting}
struct virtio_crypto_alg_chain_session_flf {
    /* Device read only portion */

#define VIRTIO_CRYPTO_SYM_ALG_CHAIN_ORDER_HASH_THEN_CIPHER  1
#define VIRTIO_CRYPTO_SYM_ALG_CHAIN_ORDER_CIPHER_THEN_HASH  2
    le32 alg_chain_order;
/* Plain hash */
#define VIRTIO_CRYPTO_SYM_HASH_MODE_PLAIN    1
/* Authenticated hash (mac) */
#define VIRTIO_CRYPTO_SYM_HASH_MODE_AUTH     2
/* Nested hash */
#define VIRTIO_CRYPTO_SYM_HASH_MODE_NESTED   3
    le32 hash_mode;
    struct virtio_crypto_cipher_session_flf cipher_hdr;

#define VIRTIO_CRYPTO_ALG_CHAIN_SESS_OP_SPEC_HDR_SIZE  16
    /* fixed length fields, algo specific */
    u8 algo_flf[VIRTIO_CRYPTO_ALG_CHAIN_SESS_OP_SPEC_HDR_SIZE];

    /* length of the additional authenticated data (AAD) in bytes */
    le32 aad_len;
    le32 padding;
};

struct virtio_crypto_alg_chain_session_vlf {
    /* Device read only portion */

    /* The cipher key */
    u8 cipher_key[key_len];
    /* The authenticated key */
    u8 auth_key[auth_key_len];
};
\end{lstlisting}

\field{hash_mode} decides the type used by \field{algo_flf}.

\field{algo_flf} is fixed to 16 bytes and MUST contains or be one of
the following types:
\begin{itemize*}
\item struct virtio_crypto_hash_create_session_flf
\item struct virtio_crypto_mac_create_session_flf
\end{itemize*}
The data of unused part (if has) in \field{algo_flf} will be ignored.

The length of \field{cipher_key} is specified in \field{key_len} in \field{cipher_hdr}.

The length of \field{auth_key} is specified in \field{auth_key_len} in struct
virtio_crypto_mac_create_session_flf.

The fixed-length parameters of Symmetric session requests are as follows:

\begin{lstlisting}
struct virtio_crypto_sym_create_session_flf {
    /* Device read only portion */

#define VIRTIO_CRYPTO_SYM_SESS_OP_SPEC_HDR_SIZE  48
    /* fixed length fields, opcode specific */
    u8 op_flf[VIRTIO_CRYPTO_SYM_SESS_OP_SPEC_HDR_SIZE];

/* No operation */
#define VIRTIO_CRYPTO_SYM_OP_NONE  0
/* Cipher only operation on the data */
#define VIRTIO_CRYPTO_SYM_OP_CIPHER  1
/* Chain any cipher with any hash or mac operation. The order
   depends on the value of alg_chain_order param */
#define VIRTIO_CRYPTO_SYM_OP_ALGORITHM_CHAINING  2
    le32 op_type;
    le32 padding;
};
\end{lstlisting}

\field{op_flf} is fixed to 48 bytes, MUST contains or be one of
the following types:
\begin{itemize*}
\item struct virtio_crypto_cipher_session_flf
\item struct virtio_crypto_alg_chain_session_flf
\end{itemize*}
The data of unused part (if has) in \field{op_flf} will be ignored.

\field{op_type} decides the type used by \field{op_flf}.

The variable-length parameters of Symmetric session requests are as follows:

\begin{lstlisting}
struct virtio_crypto_sym_create_session_vlf {
    /* Device read only portion */
    /* variable length fields, opcode specific */
    u8 op_vlf[vlf_len];
};
\end{lstlisting}

\field{op_vlf} MUST contains or be one of the following types:
\begin{itemize*}
\item struct virtio_crypto_cipher_session_vlf
\item struct virtio_crypto_alg_chain_session_vlf
\end{itemize*}

\field{op_type} in struct virtio_crypto_sym_create_session_flf decides the
type used by \field{op_vlf}.

\field{vlf_len} is the size of the specific structure used.


\subparagraph{Session operation: AEAD session}\label{sec:Device Types / Crypto Device / Device
Operation / Control Virtqueue / Session operation / Session operation: AEAD session}

The fixed-length and the variable-length parameters of AEAD session requests are as follows:

\begin{lstlisting}
struct virtio_crypto_aead_create_session_flf {
    /* Device read only portion */

    /* See VIRTIO_CRYPTO_AEAD_* above */
    le32 algo;
    /* length of key */
    le32 key_len;
    /* Authentication tag length */
    le32 tag_len;
    /* The length of the additional authenticated data (AAD) in bytes */
    le32 aad_len;
    /* encryption or decryption, See above VIRTIO_CRYPTO_OP_* */
    le32 op;
    le32 padding;
};

struct virtio_crypto_aead_create_session_vlf {
    /* Device read only portion */
    u8 key[key_len];
};
\end{lstlisting}

The length of \field{key} is specified in \field{key_len} in struct
virtio_crypto_aead_create_session_flf.

\subparagraph{Session operation: AKCIPHER session}\label{sec:Device Types / Crypto Device / Device
Operation / Control Virtqueue / Session operation / Session operation: AKCIPHER session}

Due to the complexity of asymmetric key algorithms, different algorithms
require different parameters. The following data structures are used as
supplementary parameters to describe the asymmetric algorithm sessions.

For the RSA algorithm, the extra parameters are as follows:
\begin{lstlisting}
struct virtio_crypto_rsa_session_para {
#define VIRTIO_CRYPTO_RSA_RAW_PADDING   0
#define VIRTIO_CRYPTO_RSA_PKCS1_PADDING 1
    le32 padding_algo;

#define VIRTIO_CRYPTO_RSA_NO_HASH   0
#define VIRTIO_CRYPTO_RSA_MD2       1
#define VIRTIO_CRYPTO_RSA_MD3       2
#define VIRTIO_CRYPTO_RSA_MD4       3
#define VIRTIO_CRYPTO_RSA_MD5       4
#define VIRTIO_CRYPTO_RSA_SHA1      5
#define VIRTIO_CRYPTO_RSA_SHA256    6
#define VIRTIO_CRYPTO_RSA_SHA384    7
#define VIRTIO_CRYPTO_RSA_SHA512    8
#define VIRTIO_CRYPTO_RSA_SHA224    9
    le32 hash_algo;
};
\end{lstlisting}

\field{padding_algo} specifies the padding method used by RSA sessions.
\begin{itemize*}
\item If VIRTIO_CRYPTO_RSA_RAW_PADDING is specified, 1) \field{hash_algo}
is ignored, 2) ciphertext and plaintext MUST be padded with leading zeros,
3) and RSA sessions with VIRTIO_CRYPTO_RSA_RAW_PADDING MUST not be used
for verification and signing operations.
\item If VIRTIO_CRYPTO_RSA_PKCS1_PADDING is specified, EMSA-PKCS1-v1_5 padding method
is used (see \hyperref[intro:rfc3447]{PKCS\#1}), \field{hash_algo} specifies how the
digest of the data passed to RSA sessions is calculated when verifying and signing.
It only affects the padding algorithm and is ignored during encryption and decryption.
\end{itemize*}

The ECC algorithms such as the ECDSA algorithm, cannot use custom curves, only the
following known curves can be used (see \hyperref[intro:NIST]{NIST-recommended curves}).

\begin{lstlisting}
#define VIRTIO_CRYPTO_CURVE_UNKNOWN   0
#define VIRTIO_CRYPTO_CURVE_NIST_P192 1
#define VIRTIO_CRYPTO_CURVE_NIST_P224 2
#define VIRTIO_CRYPTO_CURVE_NIST_P256 3
#define VIRTIO_CRYPTO_CURVE_NIST_P384 4
#define VIRTIO_CRYPTO_CURVE_NIST_P521 5
\end{lstlisting}

For the ECDSA algorithm, the extra parameters are as follows:
\begin{lstlisting}
struct virtio_crypto_ecdsa_session_para {
    /* See VIRTIO_CRYPTO_CURVE_* above */
    le32 curve_id;
};
\end{lstlisting}

The fixed-length and the variable-length parameters of AKCIPHER session requests are as follows:
\begin{lstlisting}
struct virtio_crypto_akcipher_create_session_flf {
    /* Device read only portion */

    /* See VIRTIO_CRYPTO_AKCIPHER_* above */
    le32 algo;
#define VIRTIO_CRYPTO_AKCIPHER_KEY_TYPE_PUBLIC 1
#define VIRTIO_CRYPTO_AKCIPHER_KEY_TYPE_PRIVATE 2
    le32 key_type;
    /* length of key */
    le32 key_len;

#define VIRTIO_CRYPTO_AKCIPHER_SESS_ALGO_SPEC_HDR_SIZE 44
    u8 algo_flf[VIRTIO_CRYPTO_AKCIPHER_SESS_ALGO_SPEC_HDR_SIZE];
};

struct virtio_crypto_akcipher_create_session_vlf {
    /* Device read only portion */
    u8 key[key_len];
};
\end{lstlisting}

\field{algo} decides the type used by \field{algo_flf}.
\field{algo_flf} is fixed to 44 bytes and MUST contains of be one the
following structures:
\begin{itemize*}
\item struct virtio_crypto_rsa_session_para
\item struct virtio_crypto_ecdsa_session_para
\end{itemize*}

The length of \field{key} is specified in \field{key_len} in the struct
virtio_crypto_akcipher_create_session_flf.

For the RSA algorithm, the key needs to be encoded according to
\hyperref[intro:rfc3447]{PKCS\#1}. The private key is described with the
RSAPrivateKey structure, and the public key is described with the RSAPublicKey
structure. These ASN.1 structures are encoded in DER encoding rules (see
\hyperref[intro:rfc6025]{rfc6025}).

\begin{lstlisting}
RSAPrivateKey ::= SEQUENCE {
    version          INTEGER,
    modulus          INTEGER,
    publicExponent   INTEGER,
    privateExponent  INTEGER,
    prime1           INTEGER,
    prime2           INTEGER,
    exponent1        INTEGER,
    exponent1        INTEGER,
    coefficient      INTEGER,
    otherPrimeInfos  OtherPrimeInfos OPTIONAL
}

OtherPrimeInfos ::= SEQUENCE SIZE(1...MAX) OF OtherPrimeInfo

OtherPrimeINfo ::= SEQUENCE {
    prime           INTEGER,
    exponent        INTEGER,
    coefficient     INTEGER
}

RSAPublicKey ::= SEQUENCE {
    modulus         INTEGER,
    publicExponent  INTEGER
}
\end{lstlisting}

For the ECDSA algorithm, the private key is encoded according to
\hyperref[intro:rfc5915]{RFC5915}, the private key of the ECDSA algorithm
is described by the ASN.1 structure ECPrivateKey and encoded with DER
encoding rules (see \hyperref[intro:rfc6025]{rfc6025}).

\begin{lstlisting}
ECPrivateKey ::= SEQUNCE {
    version         INTEGER,
    privateKey      OCTET STRING,
    parameters [0]  ECParameters {{ NamedCurve }} OPTIONAL,
    publicKey  [1]  BIT STRING OPTIONAL
}
\end{lstlisting}

The public key of the ECDSA algorithm is encoded according to \hyperref[intro:SEC1]{SEC1},
and the public key of ECDSA is described by the ASN.1 structure ECPoint.
When initializing a session with ECDSA public key, the ECPoint is DER encoded and the
\field{key} only contains the value part of ECPoint, that is, the header part of the
OCTET STRING will be omitted (see \hyperref[intro:rfc6025]{rfc6025}).

\begin{lstlisting}
ECPoint ::= OCTET STRING
\end{lstlisting}

The length of \field{key} is specified in \field{key_len} in
struct virtio_crypto_akcipher_create_session_flf.

\drivernormative{\subparagraph}{Session operation: create session}{Device Types / Crypto Device / Device
Operation / Control Virtqueue / Session operation / Session operation: create session}

\begin{itemize*}
\item The driver MUST set the \field{opcode} field based on service type: CIPHER, HASH, MAC, AEAD or AKCIPHER.
\item The driver MUST set the control general header, the opcode specific header,
    the opcode specific extra parameters and the opcode specific outcome buffer in turn.
    See \ref{sec:Device Types / Crypto Device / Device Operation / Control Virtqueue}.
\item The driver MUST set the \field{reversed} field to zero.
\end{itemize*}

\devicenormative{\subparagraph}{Session operation: create session}{Device Types / Crypto Device / Device
Operation / Control Virtqueue / Session operation / Session operation: create session}

\begin{itemize*}
\item The device MUST use the corresponding opcode specific structure according to the
    \field{opcode} in the control general header.
\item The device MUST extract extra parameters according to the structures used.
\item The device MUST set the \field{status} field to one of the following values of enum
    VIRTIO_CRYPTO_STATUS after finish a session creation:
\begin{itemize*}
\item VIRTIO_CRYPTO_OK if a session is created successfully.
\item VIRTIO_CRYPTO_NOTSUPP if the requested algorithm or operation is unsupported.
\item VIRTIO_CRYPTO_NOSPC if no free session ID (only when the VIRTIO_CRYPTO_F_REVISION_1
    feature bit is negotiated).
\item VIRTIO_CRYPTO_ERR if failure not mentioned above occurs.
\end{itemize*}
\item The device MUST set the \field{session_id} field to a unique session identifier only
    if the status is set to VIRTIO_CRYPTO_OK.
\end{itemize*}

\drivernormative{\subparagraph}{Session operation: destroy session}{Device Types / Crypto Device / Device
Operation / Control Virtqueue / Session operation / Session operation: destroy session}

\begin{itemize*}
\item The driver MUST set the \field{opcode} field based on service type: CIPHER, HASH, MAC, AEAD or AKCIPHER.
\item The driver MUST set the \field{session_id} to a valid value assigned by the device
    when the session was created.
\end{itemize*}

\devicenormative{\subparagraph}{Session operation: destroy session}{Device Types / Crypto Device / Device
Operation / Control Virtqueue / Session operation / Session operation: destroy session}

\begin{itemize*}
\item The device MUST set the \field{status} field to one of the following values of enum VIRTIO_CRYPTO_STATUS.
\begin{itemize*}
\item VIRTIO_CRYPTO_OK if a session is created successfully.
\item VIRTIO_CRYPTO_ERR if any failure occurs.
\end{itemize*}
\end{itemize*}


\subsubsection{Data Virtqueue}\label{sec:Device Types / Crypto Device / Device Operation / Data Virtqueue}

The driver uses the data virtqueues to transmit crypto operation requests to the device,
and completes the crypto operations.

The header for dataq is as follows:

\begin{lstlisting}
struct virtio_crypto_op_header {
#define VIRTIO_CRYPTO_CIPHER_ENCRYPT \
    VIRTIO_CRYPTO_OPCODE(VIRTIO_CRYPTO_SERVICE_CIPHER, 0x00)
#define VIRTIO_CRYPTO_CIPHER_DECRYPT \
    VIRTIO_CRYPTO_OPCODE(VIRTIO_CRYPTO_SERVICE_CIPHER, 0x01)
#define VIRTIO_CRYPTO_HASH \
    VIRTIO_CRYPTO_OPCODE(VIRTIO_CRYPTO_SERVICE_HASH, 0x00)
#define VIRTIO_CRYPTO_MAC \
    VIRTIO_CRYPTO_OPCODE(VIRTIO_CRYPTO_SERVICE_MAC, 0x00)
#define VIRTIO_CRYPTO_AEAD_ENCRYPT \
    VIRTIO_CRYPTO_OPCODE(VIRTIO_CRYPTO_SERVICE_AEAD, 0x00)
#define VIRTIO_CRYPTO_AEAD_DECRYPT \
    VIRTIO_CRYPTO_OPCODE(VIRTIO_CRYPTO_SERVICE_AEAD, 0x01)
#define VIRTIO_CRYPTO_AKCIPHER_ENCRYPT \
    VIRTIO_CRYPTO_OPCODE(VIRTIO_CRYPTO_SERVICE_AKCIPHER, 0x00)
#define VIRTIO_CRYPTO_AKCIPHER_DECRYPT \
    VIRTIO_CRYPTO_OPCODE(VIRTIO_CRYPTO_SERVICE_AKCIPHER, 0x01)
#define VIRTIO_CRYPTO_AKCIPHER_SIGN \
    VIRTIO_CRYPTO_OPCODE(VIRTIO_CRYPTO_SERVICE_AKCIPHER, 0x02)
#define VIRTIO_CRYPTO_AKCIPHER_VERIFY \
    VIRTIO_CRYPTO_OPCODE(VIRTIO_CRYPTO_SERVICE_AKCIPHER, 0x03)
    le32 opcode;
    /* algo should be service-specific algorithms */
    le32 algo;
    le64 session_id;
#define VIRTIO_CRYPTO_FLAG_SESSION_MODE 1
    /* control flag to control the request */
    le32 flag;
    le32 padding;
};
\end{lstlisting}

\begin{note}
If VIRTIO_CRYPTO_F_REVISION_1 is not negotiated the \field{flag} is ignored.

If VIRTIO_CRYPTO_F_REVISION_1 is negotiated but VIRTIO_CRYPTO_F_<SERVICE>_STATELESS_MODE
is not negotiated, then the device SHOULD reject <SERVICE> requests if
VIRTIO_CRYPTO_FLAG_SESSION_MODE is not set (in \field{flag}).
\end{note}

The dataq request is composed of four parts:
\begin{lstlisting}
struct virtio_crypto_op_data_req {
    /* Device read only portion */

    struct virtio_crypto_op_header header;

#define VIRTIO_CRYPTO_DATAQ_OP_SPEC_HDR_LEGACY 48
    /* fixed length fields, opcode specific */
    u8 op_flf[flf_len];

    /* Device read && write portion */
    /* variable length fields, opcode specific */
    u8 op_vlf[vlf_len];

    /* Device write only portion */
    struct virtio_crypto_inhdr inhdr;
};
\end{lstlisting}

\field{header} is a general header (see above).

\field{op_flf} is the opcode (in \field{header}) specific header.

\field{flf_len} depends on the VIRTIO_CRYPTO_F_REVISION_1 feature bit
(see below).

\field{op_vlf} is the opcode (in \field{header}) specific parameters.

\field{vlf_len} is the size of the specific structure used.

\begin{itemize*}
\item If the the opcode (in \field{header}) is VIRTIO_CRYPTO_CIPHER_ENCRYPT
    or VIRTIO_CRYPTO_CIPHER_DECRYPT then:
    \begin{itemize*}
    \item If VIRTIO_CRYPTO_F_CIPHER_STATELESS_MODE is negotiated, \field{op_flf} is
        struct virtio_crypto_sym_data_flf_stateless, and \field{op_vlf} is struct
        virtio_crypto_sym_data_vlf_stateless.
    \item If VIRTIO_CRYPTO_F_CIPHER_STATELESS_MODE is NOT negotiated, \field{op_flf}
        is struct virtio_crypto_sym_data_flf if VIRTIO_CRYPTO_F_REVISION_1 is negotiated
        and struct virtio_crypto_sym_data_flf is padded to 48 bytes if NOT negotiated,
        and \field{op_vlf} is struct virtio_crypto_sym_data_vlf.
    \end{itemize*}
\item If the the opcode (in \field{header}) is VIRTIO_CRYPTO_HASH:
    \begin{itemize*}
    \item If VIRTIO_CRYPTO_F_HASH_STATELESS_MODE is negotiated, \field{op_flf} is
        struct virtio_crypto_hash_data_flf_stateless, and \field{op_vlf} is struct
        virtio_crypto_hash_data_vlf_stateless.
    \item If VIRTIO_CRYPTO_F_HASH_STATELESS_MODE is NOT negotiated, \field{op_flf}
        is struct virtio_crypto_hash_data_flf if VIRTIO_CRYPTO_F_REVISION_1 is negotiated
        and struct virtio_crypto_hash_data_flf is padded to 48 bytes if NOT negotiated,
        and \field{op_vlf} is struct virtio_crypto_hash_data_vlf.
    \end{itemize*}
\item If the the opcode (in \field{header}) is VIRTIO_CRYPTO_MAC:
    \begin{itemize*}
    \item If VIRTIO_CRYPTO_F_MAC_STATELESS_MODE is negotiated, \field{op_flf} is
        struct virtio_crypto_mac_data_flf_stateless, and \field{op_vlf} is struct
        virtio_crypto_mac_data_vlf_stateless.
    \item If VIRTIO_CRYPTO_F_MAC_STATELESS_MODE is NOT negotiated, \field{op_flf}
        is struct virtio_crypto_mac_data_flf if VIRTIO_CRYPTO_F_REVISION_1 is negotiated
        and struct virtio_crypto_mac_data_flf is padded to 48 bytes if NOT negotiated,
        and \field{op_vlf} is struct virtio_crypto_mac_data_vlf.
    \end{itemize*}
\item If the the opcode (in \field{header}) is VIRTIO_CRYPTO_AEAD_ENCRYPT
    or VIRTIO_CRYPTO_AEAD_DECRYPT then:
    \begin{itemize*}
    \item If VIRTIO_CRYPTO_F_AEAD_STATELESS_MODE is negotiated, \field{op_flf} is
        struct virtio_crypto_aead_data_flf_stateless, and \field{op_vlf} is struct
        virtio_crypto_aead_data_vlf_stateless.
    \item If VIRTIO_CRYPTO_F_AEAD_STATELESS_MODE is NOT negotiated, \field{op_flf}
        is struct virtio_crypto_aead_data_flf if VIRTIO_CRYPTO_F_REVISION_1 is negotiated
        and struct virtio_crypto_aead_data_flf is padded to 48 bytes if NOT negotiated,
        and \field{op_vlf} is struct virtio_crypto_aead_data_vlf.
    \end{itemize*}
\item If the opcode (in \field{header}) is VIRTIO_CRYPTO_AKCIPHER_ENCRYPT, VIRTIO_CRYPTO_AKCIPHER_DECRYPT,
    VIRTIO_CRYPTO_AKCIPHER_SIGN or VIRTIO_CRYPTO_AKCIPHER_VERIFY then:
    \begin{itemize*}
    \item If VIRTIO_CRYPTO_F_AKCIPHER_STATELESS_MODE is negotiated, \field{op_flf} is
        struct virtio_crypto_akcipher_data_flf_statless, and \field{op_vlf} is struct
        virtio_crypto_akcipher_data_vlf_stateless.
    \item If VIRTIO_CRYPTO_F_AKCIPHER_STATELESS_MODE is NOT negotiated, \field{op_flf}
        is struct virtio_crypto_akcipher_data_flf if VIRTIO_CRYPTO_F_REVISION_1 is negotiated
        and struct virtio_crypto_akcipher_data_flf is padded to 48 bytes if NOT negotiated,
        and \field{op_vlf} is struct virtio_crypto_akcipher_data_vlf.
    \end{itemize*}
\end{itemize*}

\field{inhdr} is a unified input header that used to return the status of
the operations, is defined as follows:

\begin{lstlisting}
struct virtio_crypto_inhdr {
    u8 status;
};
\end{lstlisting}

\subsubsection{HASH Service Operation}\label{sec:Device Types / Crypto Device / Device Operation / HASH Service Operation}

Session mode HASH service requests are as follows:

\begin{lstlisting}
struct virtio_crypto_hash_data_flf {
    /* length of source data */
    le32 src_data_len;
    /* hash result length */
    le32 hash_result_len;
};

struct virtio_crypto_hash_data_vlf {
    /* Device read only portion */
    /* Source data */
    u8 src_data[src_data_len];

    /* Device write only portion */
    /* Hash result data */
    u8 hash_result[hash_result_len];
};
\end{lstlisting}

Each data request uses the virtio_crypto_hash_data_flf structure and the
virtio_crypto_hash_data_vlf structure to store information used to run the
HASH operations.

\field{src_data} is the source data that will be processed.
\field{src_data_len} is the length of source data.
\field{hash_result} is the result data and \field{hash_result_len} is the length
of it.

Stateless mode HASH service requests are as follows:

\begin{lstlisting}
struct virtio_crypto_hash_data_flf_stateless {
    struct {
        /* See VIRTIO_CRYPTO_HASH_* above */
        le32 algo;
    } sess_para;

    /* length of source data */
    le32 src_data_len;
    /* hash result length */
    le32 hash_result_len;
    le32 reserved;
};
struct virtio_crypto_hash_data_vlf_stateless {
    /* Device read only portion */
    /* Source data */
    u8 src_data[src_data_len];

    /* Device write only portion */
    /* Hash result data */
    u8 hash_result[hash_result_len];
};
\end{lstlisting}

\drivernormative{\paragraph}{HASH Service Operation}{Device Types / Crypto Device / Device Operation / HASH Service Operation}

\begin{itemize*}
\item If the driver uses the session mode, then the driver MUST set \field{session_id}
    in struct virtio_crypto_op_header to a valid value assigned by the device when the
    session was created.
\item If the VIRTIO_CRYPTO_F_HASH_STATELESS_MODE feature bit is negotiated, 1) if the
    driver uses the stateless mode, then the driver MUST set the \field{flag} field in
    struct virtio_crypto_op_header to ZERO and MUST set the fields in struct
    virtio_crypto_hash_data_flf_stateless.sess_para, 2) if the driver uses the session
    mode, then the driver MUST set the \field{flag} field in struct virtio_crypto_op_header
    to VIRTIO_CRYPTO_FLAG_SESSION_MODE.
\item The driver MUST set \field{opcode} in struct virtio_crypto_op_header to VIRTIO_CRYPTO_HASH.
\end{itemize*}

\devicenormative{\paragraph}{HASH Service Operation}{Device Types / Crypto Device / Device Operation / HASH Service Operation}

\begin{itemize*}
\item The device MUST use the corresponding structure according to the \field{opcode}
    in the data general header.
\item If the VIRTIO_CRYPTO_F_HASH_STATELESS_MODE feature bit is negotiated, the device
    MUST parse \field{flag} field in struct virtio_crypto_op_header in order to decide
    which mode the driver uses.
\item The device MUST copy the results of HASH operations in the hash_result[] if HASH
    operations success.
\item The device MUST set \field{status} in struct virtio_crypto_inhdr to one of the
    following values of enum VIRTIO_CRYPTO_STATUS:
\begin{itemize*}
\item VIRTIO_CRYPTO_OK if the operation success.
\item VIRTIO_CRYPTO_NOTSUPP if the requested algorithm or operation is unsupported.
\item VIRTIO_CRYPTO_INVSESS if the session ID invalid when in session mode.
\item VIRTIO_CRYPTO_ERR if any failure not mentioned above occurs.
\end{itemize*}
\end{itemize*}


\subsubsection{MAC Service Operation}\label{sec:Device Types / Crypto Device / Device Operation / MAC Service Operation}

Session mode MAC service requests are as follows:

\begin{lstlisting}
struct virtio_crypto_mac_data_flf {
    struct virtio_crypto_hash_data_flf hdr;
};

struct virtio_crypto_mac_data_vlf {
    /* Device read only portion */
    /* Source data */
    u8 src_data[src_data_len];

    /* Device write only portion */
    /* Hash result data */
    u8 hash_result[hash_result_len];
};
\end{lstlisting}

Each request uses the virtio_crypto_mac_data_flf structure and the
virtio_crypto_mac_data_vlf structure to store information used to run the
MAC operations.

\field{src_data} is the source data that will be processed.
\field{src_data_len} is the length of source data.
\field{hash_result} is the result data and \field{hash_result_len} is the length
of it.

Stateless mode MAC service requests are as follows:

\begin{lstlisting}
struct virtio_crypto_mac_data_flf_stateless {
    struct {
        /* See VIRTIO_CRYPTO_MAC_* above */
        le32 algo;
        /* length of authenticated key */
        le32 auth_key_len;
    } sess_para;

    /* length of source data */
    le32 src_data_len;
    /* hash result length */
    le32 hash_result_len;
};

struct virtio_crypto_mac_data_vlf_stateless {
    /* Device read only portion */
    /* The authenticated key */
    u8 auth_key[auth_key_len];
    /* Source data */
    u8 src_data[src_data_len];

    /* Device write only portion */
    /* Hash result data */
    u8 hash_result[hash_result_len];
};
\end{lstlisting}

\field{auth_key} is the authenticated key that will be used during the process.
\field{auth_key_len} is the length of the key.

\drivernormative{\paragraph}{MAC Service Operation}{Device Types / Crypto Device / Device Operation / MAC Service Operation}

\begin{itemize*}
\item If the driver uses the session mode, then the driver MUST set \field{session_id}
    in struct virtio_crypto_op_header to a valid value assigned by the device when the
    session was created.
\item If the VIRTIO_CRYPTO_F_MAC_STATELESS_MODE feature bit is negotiated, 1) if the
    driver uses the stateless mode, then the driver MUST set the \field{flag} field
    in struct virtio_crypto_op_header to ZERO and MUST set the fields in struct
    virtio_crypto_mac_data_flf_stateless.sess_para, 2) if the driver uses the session
    mode, then the driver MUST set the \field{flag} field in struct virtio_crypto_op_header
    to VIRTIO_CRYPTO_FLAG_SESSION_MODE.
\item The driver MUST set \field{opcode} in struct virtio_crypto_op_header to VIRTIO_CRYPTO_MAC.
\end{itemize*}

\devicenormative{\paragraph}{MAC Service Operation}{Device Types / Crypto Device / Device Operation / MAC Service Operation}

\begin{itemize*}
\item If the VIRTIO_CRYPTO_F_MAC_STATELESS_MODE feature bit is negotiated, the device
    MUST parse \field{flag} field in struct virtio_crypto_op_header in order to decide
	which mode the driver uses.
\item The device MUST copy the results of MAC operations in the hash_result[] if HASH
    operations success.
\item The device MUST set \field{status} in struct virtio_crypto_inhdr to one of the
    following values of enum VIRTIO_CRYPTO_STATUS:
\begin{itemize*}
\item VIRTIO_CRYPTO_OK if the operation success.
\item VIRTIO_CRYPTO_NOTSUPP if the requested algorithm or operation is unsupported.
\item VIRTIO_CRYPTO_INVSESS if the session ID invalid when in session mode.
\item VIRTIO_CRYPTO_ERR if any failure not mentioned above occurs.
\end{itemize*}
\end{itemize*}

\subsubsection{Symmetric algorithms Operation}\label{sec:Device Types / Crypto Device / Device Operation / Symmetric algorithms Operation}

Session mode CIPHER service requests are as follows:

\begin{lstlisting}
struct virtio_crypto_cipher_data_flf {
    /*
     * Byte Length of valid IV/Counter data pointed to by the below iv data.
     *
     * For block ciphers in CBC or F8 mode, or for Kasumi in F8 mode, or for
     *   SNOW3G in UEA2 mode, this is the length of the IV (which
     *   must be the same as the block length of the cipher).
     * For block ciphers in CTR mode, this is the length of the counter
     *   (which must be the same as the block length of the cipher).
     */
    le32 iv_len;
    /* length of source data */
    le32 src_data_len;
    /* length of destination data */
    le32 dst_data_len;
    le32 padding;
};

struct virtio_crypto_cipher_data_vlf {
    /* Device read only portion */

    /*
     * Initialization Vector or Counter data.
     *
     * For block ciphers in CBC or F8 mode, or for Kasumi in F8 mode, or for
     *   SNOW3G in UEA2 mode, this is the Initialization Vector (IV)
     *   value.
     * For block ciphers in CTR mode, this is the counter.
     * For AES-XTS, this is the 128bit tweak, i, from IEEE Std 1619-2007.
     *
     * The IV/Counter will be updated after every partial cryptographic
     * operation.
     */
    u8 iv[iv_len];
    /* Source data */
    u8 src_data[src_data_len];

    /* Device write only portion */
    /* Destination data */
    u8 dst_data[dst_data_len];
};
\end{lstlisting}

Session mode requests of algorithm chaining are as follows:

\begin{lstlisting}
struct virtio_crypto_alg_chain_data_flf {
    le32 iv_len;
    /* Length of source data */
    le32 src_data_len;
    /* Length of destination data */
    le32 dst_data_len;
    /* Starting point for cipher processing in source data */
    le32 cipher_start_src_offset;
    /* Length of the source data that the cipher will be computed on */
    le32 len_to_cipher;
    /* Starting point for hash processing in source data */
    le32 hash_start_src_offset;
    /* Length of the source data that the hash will be computed on */
    le32 len_to_hash;
    /* Length of the additional auth data */
    le32 aad_len;
    /* Length of the hash result */
    le32 hash_result_len;
    le32 reserved;
};

struct virtio_crypto_alg_chain_data_vlf {
    /* Device read only portion */

    /* Initialization Vector or Counter data */
    u8 iv[iv_len];
    /* Source data */
    u8 src_data[src_data_len];
    /* Additional authenticated data if exists */
    u8 aad[aad_len];

    /* Device write only portion */

    /* Destination data */
    u8 dst_data[dst_data_len];
    /* Hash result data */
    u8 hash_result[hash_result_len];
};
\end{lstlisting}

Session mode requests of symmetric algorithm are as follows:

\begin{lstlisting}
struct virtio_crypto_sym_data_flf {
    /* Device read only portion */

#define VIRTIO_CRYPTO_SYM_DATA_REQ_HDR_SIZE    40
    u8 op_type_flf[VIRTIO_CRYPTO_SYM_DATA_REQ_HDR_SIZE];

    /* See above VIRTIO_CRYPTO_SYM_OP_* */
    le32 op_type;
    le32 padding;
};

struct virtio_crypto_sym_data_vlf {
    u8 op_type_vlf[sym_para_len];
};
\end{lstlisting}

Each request uses the virtio_crypto_sym_data_flf structure and the
virtio_crypto_sym_data_flf structure to store information used to run the
CIPHER operations.

\field{op_type_flf} is the \field{op_type} specific header, it MUST starts
with or be one of the following structures:
\begin{itemize*}
\item struct virtio_crypto_cipher_data_flf
\item struct virtio_crypto_alg_chain_data_flf
\end{itemize*}

The length of \field{op_type_flf} is fixed to 40 bytes, the data of unused
part (if has) will be ignored.

\field{op_type_vlf} is the \field{op_type} specific parameters, it MUST starts
with or be one of the following structures:
\begin{itemize*}
\item struct virtio_crypto_cipher_data_vlf
\item struct virtio_crypto_alg_chain_data_vlf
\end{itemize*}

\field{sym_para_len} is the size of the specific structure used.

Stateless mode CIPHER service requests are as follows:

\begin{lstlisting}
struct virtio_crypto_cipher_data_flf_stateless {
    struct {
        /* See VIRTIO_CRYPTO_CIPHER* above */
        le32 algo;
        /* length of key */
        le32 key_len;

        /* See VIRTIO_CRYPTO_OP_* above */
        le32 op;
    } sess_para;

    /*
     * Byte Length of valid IV/Counter data pointed to by the below iv data.
     */
    le32 iv_len;
    /* length of source data */
    le32 src_data_len;
    /* length of destination data */
    le32 dst_data_len;
};

struct virtio_crypto_cipher_data_vlf_stateless {
    /* Device read only portion */

    /* The cipher key */
    u8 cipher_key[key_len];

    /* Initialization Vector or Counter data. */
    u8 iv[iv_len];
    /* Source data */
    u8 src_data[src_data_len];

    /* Device write only portion */
    /* Destination data */
    u8 dst_data[dst_data_len];
};
\end{lstlisting}

Stateless mode requests of algorithm chaining are as follows:

\begin{lstlisting}
struct virtio_crypto_alg_chain_data_flf_stateless {
    struct {
        /* See VIRTIO_CRYPTO_SYM_ALG_CHAIN_ORDER_* above */
        le32 alg_chain_order;
        /* length of the additional authenticated data in bytes */
        le32 aad_len;

        struct {
            /* See VIRTIO_CRYPTO_CIPHER* above */
            le32 algo;
            /* length of key */
            le32 key_len;
            /* See VIRTIO_CRYPTO_OP_* above */
            le32 op;
        } cipher;

        struct {
            /* See VIRTIO_CRYPTO_HASH_* or VIRTIO_CRYPTO_MAC_* above */
            le32 algo;
            /* length of authenticated key */
            le32 auth_key_len;
            /* See VIRTIO_CRYPTO_SYM_HASH_MODE_* above */
            le32 hash_mode;
        } hash;
    } sess_para;

    le32 iv_len;
    /* Length of source data */
    le32 src_data_len;
    /* Length of destination data */
    le32 dst_data_len;
    /* Starting point for cipher processing in source data */
    le32 cipher_start_src_offset;
    /* Length of the source data that the cipher will be computed on */
    le32 len_to_cipher;
    /* Starting point for hash processing in source data */
    le32 hash_start_src_offset;
    /* Length of the source data that the hash will be computed on */
    le32 len_to_hash;
    /* Length of the additional auth data */
    le32 aad_len;
    /* Length of the hash result */
    le32 hash_result_len;
    le32 reserved;
};

struct virtio_crypto_alg_chain_data_vlf_stateless {
    /* Device read only portion */

    /* The cipher key */
    u8 cipher_key[key_len];
    /* The auth key */
    u8 auth_key[auth_key_len];
    /* Initialization Vector or Counter data */
    u8 iv[iv_len];
    /* Additional authenticated data if exists */
    u8 aad[aad_len];
    /* Source data */
    u8 src_data[src_data_len];

    /* Device write only portion */

    /* Destination data */
    u8 dst_data[dst_data_len];
    /* Hash result data */
    u8 hash_result[hash_result_len];
};
\end{lstlisting}

Stateless mode requests of symmetric algorithm are as follows:

\begin{lstlisting}
struct virtio_crypto_sym_data_flf_stateless {
    /* Device read only portion */
#define VIRTIO_CRYPTO_SYM_DATE_REQ_HDR_STATELESS_SIZE    72
    u8 op_type_flf[VIRTIO_CRYPTO_SYM_DATE_REQ_HDR_STATELESS_SIZE];

    /* Device write only portion */
    /* See above VIRTIO_CRYPTO_SYM_OP_* */
    le32 op_type;
};

struct virtio_crypto_sym_data_vlf_stateless {
    u8 op_type_vlf[sym_para_len];
};
\end{lstlisting}

\field{op_type_flf} is the \field{op_type} specific header, it MUST starts
with or be one of the following structures:
\begin{itemize*}
\item struct virtio_crypto_cipher_data_flf_stateless
\item struct virtio_crypto_alg_chain_data_flf_stateless
\end{itemize*}

The length of \field{op_type_flf} is fixed to 72 bytes, the data of unused
part (if has) will be ignored.

\field{op_type_vlf} is the \field{op_type} specific parameters, it MUST starts
with or be one of the following structures:
\begin{itemize*}
\item struct virtio_crypto_cipher_data_vlf_stateless
\item struct virtio_crypto_alg_chain_data_vlf_stateless
\end{itemize*}

\field{sym_para_len} is the size of the specific structure used.

\drivernormative{\paragraph}{Symmetric algorithms Operation}{Device Types / Crypto Device / Device Operation / Symmetric algorithms Operation}

\begin{itemize*}
\item If the driver uses the session mode, then the driver MUST set \field{session_id}
    in struct virtio_crypto_op_header to a valid value assigned by the device when the
    session was created.
\item If the VIRTIO_CRYPTO_F_CIPHER_STATELESS_MODE feature bit is negotiated, 1) if the
    driver uses the stateless mode, then the driver MUST set the \field{flag} field in
    struct virtio_crypto_op_header to ZERO and MUST set the fields in struct
    virtio_crypto_cipher_data_flf_stateless.sess_para or struct
    virtio_crypto_alg_chain_data_flf_stateless.sess_para, 2) if the driver uses the
    session mode, then the driver MUST set the \field{flag} field in struct
    virtio_crypto_op_header to VIRTIO_CRYPTO_FLAG_SESSION_MODE.
\item The driver MUST set the \field{opcode} field in struct virtio_crypto_op_header
    to VIRTIO_CRYPTO_CIPHER_ENCRYPT or VIRTIO_CRYPTO_CIPHER_DECRYPT.
\item The driver MUST specify the fields of struct virtio_crypto_cipher_data_flf in
    struct virtio_crypto_sym_data_flf and struct virtio_crypto_cipher_data_vlf in
    struct virtio_crypto_sym_data_vlf if the request is based on VIRTIO_CRYPTO_SYM_OP_CIPHER.
\item The driver MUST specify the fields of struct virtio_crypto_alg_chain_data_flf
    in struct virtio_crypto_sym_data_flf and struct virtio_crypto_alg_chain_data_vlf
    in struct virtio_crypto_sym_data_vlf if the request is of the VIRTIO_CRYPTO_SYM_OP_ALGORITHM_CHAINING
    type.
\end{itemize*}

\devicenormative{\paragraph}{Symmetric algorithms Operation}{Device Types / Crypto Device / Device Operation / Symmetric algorithms Operation}

\begin{itemize*}
\item If the VIRTIO_CRYPTO_F_CIPHER_STATELESS_MODE feature bit is negotiated, the device
    MUST parse \field{flag} field in struct virtio_crypto_op_header in order to decide
	which mode the driver uses.
\item The device MUST parse the virtio_crypto_sym_data_req based on the \field{opcode}
    field in general header.
\item The device MUST parse the fields of struct virtio_crypto_cipher_data_flf in
    struct virtio_crypto_sym_data_flf and struct virtio_crypto_cipher_data_vlf in
    struct virtio_crypto_sym_data_vlf if the request is based on VIRTIO_CRYPTO_SYM_OP_CIPHER.
\item The device MUST parse the fields of struct virtio_crypto_alg_chain_data_flf
    in struct virtio_crypto_sym_data_flf and struct virtio_crypto_alg_chain_data_vlf
    in struct virtio_crypto_sym_data_vlf if the request is of the VIRTIO_CRYPTO_SYM_OP_ALGORITHM_CHAINING
    type.
\item The device MUST copy the result of cryptographic operation in the dst_data[] in
    both plain CIPHER mode and algorithms chain mode.
\item The device MUST check the \field{para}.\field{add_len} is bigger than 0 before
    parse the additional authenticated data in plain algorithms chain mode.
\item The device MUST copy the result of HASH/MAC operation in the hash_result[] is
    of the VIRTIO_CRYPTO_SYM_OP_ALGORITHM_CHAINING type.
\item The device MUST set the \field{status} field in struct virtio_crypto_inhdr to
    one of the following values of enum VIRTIO_CRYPTO_STATUS:
\begin{itemize*}
\item VIRTIO_CRYPTO_OK if the operation success.
\item VIRTIO_CRYPTO_NOTSUPP if the requested algorithm or operation is unsupported.
\item VIRTIO_CRYPTO_INVSESS if the session ID is invalid in session mode.
\item VIRTIO_CRYPTO_ERR if failure not mentioned above occurs.
\end{itemize*}
\end{itemize*}

\subsubsection{AEAD Service Operation}\label{sec:Device Types / Crypto Device / Device Operation / AEAD Service Operation}

Session mode requests of symmetric algorithm are as follows:

\begin{lstlisting}
struct virtio_crypto_aead_data_flf {
    /*
     * Byte Length of valid IV data.
     *
     * For GCM mode, this is either 12 (for 96-bit IVs) or 16, in which
     *   case iv points to J0.
     * For CCM mode, this is the length of the nonce, which can be in the
     *   range 7 to 13 inclusive.
     */
    le32 iv_len;
    /* length of additional auth data */
    le32 aad_len;
    /* length of source data */
    le32 src_data_len;
    /* length of dst data, this should be at least src_data_len + tag_len */
    le32 dst_data_len;
    /* Authentication tag length */
    le32 tag_len;
    le32 reserved;
};

struct virtio_crypto_aead_data_vlf {
    /* Device read only portion */

    /*
     * Initialization Vector data.
     *
     * For GCM mode, this is either the IV (if the length is 96 bits) or J0
     *   (for other sizes), where J0 is as defined by NIST SP800-38D.
     *   Regardless of the IV length, a full 16 bytes needs to be allocated.
     * For CCM mode, the first byte is reserved, and the nonce should be
     *   written starting at &iv[1] (to allow space for the implementation
     *   to write in the flags in the first byte).  Note that a full 16 bytes
     *   should be allocated, even though the iv_len field will have
     *   a value less than this.
     *
     * The IV will be updated after every partial cryptographic operation.
     */
    u8 iv[iv_len];
    /* Source data */
    u8 src_data[src_data_len];
    /* Additional authenticated data if exists */
    u8 aad[aad_len];

    /* Device write only portion */
    /* Pointer to output data */
    u8 dst_data[dst_data_len];
};
\end{lstlisting}

Each request uses the virtio_crypto_aead_data_flf structure and the
virtio_crypto_aead_data_flf structure to store information used to run the
AEAD operations.

Stateless mode AEAD service requests are as follows:

\begin{lstlisting}
struct virtio_crypto_aead_data_flf_stateless {
    struct {
        /* See VIRTIO_CRYPTO_AEAD_* above */
        le32 algo;
        /* length of key */
        le32 key_len;
        /* encrypt or decrypt, See above VIRTIO_CRYPTO_OP_* */
        le32 op;
    } sess_para;

    /* Byte Length of valid IV data. */
    le32 iv_len;
    /* Authentication tag length */
    le32 tag_len;
    /* length of additional auth data */
    le32 aad_len;
    /* length of source data */
    le32 src_data_len;
    /* length of dst data, this should be at least src_data_len + tag_len */
    le32 dst_data_len;
};

struct virtio_crypto_aead_data_vlf_stateless {
    /* Device read only portion */

    /* The cipher key */
    u8 key[key_len];
    /* Initialization Vector data. */
    u8 iv[iv_len];
    /* Source data */
    u8 src_data[src_data_len];
    /* Additional authenticated data if exists */
    u8 aad[aad_len];

    /* Device write only portion */
    /* Pointer to output data */
    u8 dst_data[dst_data_len];
};
\end{lstlisting}

\drivernormative{\paragraph}{AEAD Service Operation}{Device Types / Crypto Device / Device Operation / AEAD Service Operation}

\begin{itemize*}
\item If the driver uses the session mode, then the driver MUST set
    \field{session_id} in struct virtio_crypto_op_header to a valid value assigned
    by the device when the session was created.
\item If the VIRTIO_CRYPTO_F_AEAD_STATELESS_MODE feature bit is negotiated, 1) if
    the driver uses the stateless mode, then the driver MUST set the \field{flag}
    field in struct virtio_crypto_op_header to ZERO and MUST set the fields in
    struct virtio_crypto_aead_data_flf_stateless.sess_para, 2) if the driver uses
    the session mode, then the driver MUST set the \field{flag} field in struct
    virtio_crypto_op_header to VIRTIO_CRYPTO_FLAG_SESSION_MODE.
\item The driver MUST set the \field{opcode} field in struct virtio_crypto_op_header
    to VIRTIO_CRYPTO_AEAD_ENCRYPT or VIRTIO_CRYPTO_AEAD_DECRYPT.
\end{itemize*}

\devicenormative{\paragraph}{AEAD Service Operation}{Device Types / Crypto Device / Device Operation / AEAD Service Operation}

\begin{itemize*}
\item If the VIRTIO_CRYPTO_F_AEAD_STATELESS_MODE feature bit is negotiated, the
    device MUST parse the virtio_crypto_aead_data_vlf_stateless based on the \field{opcode}
	field in general header.
\item The device MUST copy the result of cryptographic operation in the dst_data[].
\item The device MUST copy the authentication tag in the dst_data[] offset the cipher result.
\item The device MUST set the \field{status} field in struct virtio_crypto_inhdr to
    one of the following values of enum VIRTIO_CRYPTO_STATUS:
\item When the \field{opcode} field is VIRTIO_CRYPTO_AEAD_DECRYPT, the device MUST
    verify and return the verification result to the driver.
\begin{itemize*}
\item VIRTIO_CRYPTO_OK if the operation success.
\item VIRTIO_CRYPTO_NOTSUPP if the requested algorithm or operation is unsupported.
\item VIRTIO_CRYPTO_BADMSG if the verification result is incorrect.
\item VIRTIO_CRYPTO_INVSESS if the session ID invalid when in session mode.
\item VIRTIO_CRYPTO_ERR if any failure not mentioned above occurs.
\end{itemize*}
\end{itemize*}

\subsubsection{AKCIPHER Service Operation}\label{sec:Device Types / Crypto Device / Device Operation / AKCIPHER Service Operation}

Session mode AKCIPHER requests are as follows:

\begin{lstlisting}
struct virtio_crypto_akcipher_data_flf {
    /* length of source data */
    le32 src_data_len;
    /* length of dst data */
    le32 dst_data_len;
};

struct virtio_crypto_akcipher_data_vlf {
    /* Device read only portion */
    /* Source data */
    u8 src_data[src_data_len];

    /* Device write only portion */
    /* Pointer to output data */
    u8 dst_data[dst_data_len];
};
\end{lstlisting}

Each data request uses the virtio_crypto_akcipher_flf structure and the virtio_crypto_akcipher_data_vlf
structure to store information used to run the AKCIPHER operations.

For encryption, decryption, and signing:
\field{src_data} is the source data that will be processed, note that for signing operations,
src_data stores the data to be signed, which usually is the digest of some data rather than the
data itself.
\field{src_data_len} is the length of source data.
\field{dst_result} is the result data and \field{dst_data_len} is the length of it. Note that the
length of the result is not always exactly equal to dst_data_len, the driver needs to check how
many bytes the device has written and calculate the actual length of the result.

For verification:
\field{src_data_len} refers to the length of the signature, and \field{dst_data_len} refers to
the length of signed data, where the signed data is usually the digest of some data.
\field{src_data} is spliced by the signature and the signed data, the src_data with the lower
address stores the signature, and the higher address stores the signed data.
\field{dst_data} is always empty for verification.

Different algorithms have different signature formats.
For the RSA algorithm, the result is determined by the padding algorithm specified by
\field{padding_algo} in structure virtio_crypto_rsa_session_para.

For the ECDSA algorithm, the signature is composed of the following
ASN.1 structure (see \hyperref[intro:rfc3279]{RFC3279})
and MUST be DER encoded (see \hyperref[intro:rfc6025]{rfc6025}).

\begin{lstlisting}
Ecdsa-Sig-Value ::= SEQUENCE {
    r INTEGER,
    s INTEGER
}
\end{lstlisting}

Stateless mode AKCIPHER service requests are as follows:
\begin{lstlisting}
struct virtio_crypto_akcipher_data_flf_stateless {
    struct {
        /* See VIRTIO_CYRPTO_AKCIPHER* above */
        le32 algo;
        /* See VIRTIO_CRYPTO_AKCIPHER_KEY_TYPE_* above */
        le32 key_type;
        /* length of key */
        le32 key_len;

        /* algothrim specific parameters described above */
        union {
            struct virtio_crypto_rsa_session_para rsa;
            struct virtio_crypto_ecdsa_session_para ecdsa;
        } u;
    } sess_para;

    /* length of source data */
    le32 src_data_len;
    /* length of destination data */
    le32 dst_data_len;
};

struct virtio_crypto_akcipher_data_vlf_stateless {
    /* Device read only portion */
    u8 akcipher_key[key_len];

    /* Source data */
    u8 src_data[src_data_len];

    /* Device write only portion */
    u8 dst_data[dst_data_len];
};
\end{lstlisting}

In stateless mode, the format of key and signature, the meaning of src_data and dst_data, are all the same
with session mode.

\drivernormative{\paragraph}{AKCIPHER Service Operation}{Device Types / Crypto Device / Device Operation / AKCIPHER Service Operation}

\begin{itemize*}
\item If the driver uses the session mode, then the driver MUST set
    \field{session_id} in struct virtio_crypto_op_header to a valid
    value assigned by the device when the session was created.
\item If the VIRTIO_CRYPTO_F_AKCIPHER_STATELESS_MODE feature bit is negotiated, 1) if the
    driver uses the stateless mode, then the driver MUST set the \field{flag} field in
    struct virtio_crypto_op_header to ZERO and MUST set the fields in struct
    virtio_crypto_akcipher_flf_stateless.sess_para, 2) if the driver uses the session
    mode, then the driver MUST set the \field{flag} field in struct virtio_crypto_op_header
    to VIRTIO_CRYPTO_FLAG_SESSION_MODE.
\item The driver MUST set the \field{opcode} field in struct virtio_crypto_op_header
    to one of VIRTIO_CRYPTO_AKCIPHER_ENCRYPT, VIRTIO_CRYPTO_AKCIPHER_DESTROY_SESSION,
    VIRTIO_CRYPTO_AKCIPHER_SIGN, and VIRTIO_CRYPTO_AKCIPHER_VERIFY.
\end{itemize*}

\devicenormative{\paragraph}{AKCIPHER Service Operation}{Device Types / Crypto Device / Device Operation / AKCIPHER Service Operation}

\begin{itemize*}
\item If the VIRTIO_CRYPTO_F_AKCIPHER_STATELESS_MODE feature bit is negotiated, the
    device MUST parse the virtio_crypto_akcipher_data_vlf_stateless based on the \field{opcode}
    field in general header.
\item The device MUST copy the result of cryptographic operation in the dst_data[].
\item The device MUST set the \field{status} field in struct virtio_crypto_inhdr to
    one of the following values of enum VIRTIO_CRYPTO_STATUS:
\begin{itemize*}
\item VIRTIO_CRYPTO_OK if the operation success.
\item VIRTIO_CRYPTO_NOTSUPP if the requested algorithm or operation is unsupported.
\item VIRTIO_CRYPTO_BADMSG if the verification result is incorrect.
\item VIRTIO_CRYPTO_INVSESS if the session ID invalid when in session mode.
\item VIRTIO_CRYPTO_KEY_REJECTED if the signature verification failed.
\item VIRTIO_CRYPTO_ERR if any failure not mentioned above occurs.
\end{itemize*}
\end{itemize*}

\section{Crypto Device}\label{sec:Device Types / Crypto Device}

The virtio crypto device is a virtual cryptography device as well as a
virtual cryptographic accelerator. The virtio crypto device provides the
following crypto services: CIPHER, MAC, HASH, AEAD and AKCIPHER. Virtio crypto
devices have a single control queue and at least one data queue. Crypto
operation requests are placed into a data queue, and serviced by the
device. Some crypto operation requests are only valid in the context of a
session. The role of the control queue is facilitating control operation
requests. Sessions management is realized with control operation
requests.

\subsection{Device ID}\label{sec:Device Types / Crypto Device / Device ID}

20

\subsection{Virtqueues}\label{sec:Device Types / Crypto Device / Virtqueues}

\begin{description}
\item[0] dataq1
\item[\ldots]
\item[N-1] dataqN
\item[N] controlq
\end{description}

N is set by \field{max_dataqueues}.

\subsection{Feature bits}\label{sec:Device Types / Crypto Device / Feature bits}

\begin{description}
\item VIRTIO_CRYPTO_F_REVISION_1 (0) revision 1. Revision 1 has a specific
    request format and other enhancements (which result in some additional
    requirements).
\item VIRTIO_CRYPTO_F_CIPHER_STATELESS_MODE (1) stateless mode requests are
    supported by the CIPHER service.
\item VIRTIO_CRYPTO_F_HASH_STATELESS_MODE (2) stateless mode requests are
    supported by the HASH service.
\item VIRTIO_CRYPTO_F_MAC_STATELESS_MODE (3) stateless mode requests are
    supported by the MAC service.
\item VIRTIO_CRYPTO_F_AEAD_STATELESS_MODE (4) stateless mode requests are
    supported by the AEAD service.
\item VIRTIO_CRYPTO_F_AKCIPHER_STATELESS_MODE (5) stateless mode requests are
    supported by the AKCIPHER service.
\end{description}


\subsubsection{Feature bit requirements}\label{sec:Device Types / Crypto Device / Feature bit requirements}

Some crypto feature bits require other crypto feature bits
(see \ref{drivernormative:Basic Facilities of a Virtio Device / Feature Bits}):

\begin{description}
\item[VIRTIO_CRYPTO_F_CIPHER_STATELESS_MODE] Requires VIRTIO_CRYPTO_F_REVISION_1.
\item[VIRTIO_CRYPTO_F_HASH_STATELESS_MODE] Requires VIRTIO_CRYPTO_F_REVISION_1.
\item[VIRTIO_CRYPTO_F_MAC_STATELESS_MODE] Requires VIRTIO_CRYPTO_F_REVISION_1.
\item[VIRTIO_CRYPTO_F_AEAD_STATELESS_MODE] Requires VIRTIO_CRYPTO_F_REVISION_1.
\item[VIRTIO_CRYPTO_F_AKCIPHER_STATELESS_MODE] Requires VIRTIO_CRYPTO_F_REVISION_1.
\end{description}

\subsection{Supported crypto services}\label{sec:Device Types / Crypto Device / Supported crypto services}

The following crypto services are defined:

\begin{lstlisting}
/* CIPHER (Symmetric Key Cipher) service */
#define VIRTIO_CRYPTO_SERVICE_CIPHER 0
/* HASH service */
#define VIRTIO_CRYPTO_SERVICE_HASH   1
/* MAC (Message Authentication Codes) service */
#define VIRTIO_CRYPTO_SERVICE_MAC    2
/* AEAD (Authenticated Encryption with Associated Data) service */
#define VIRTIO_CRYPTO_SERVICE_AEAD   3
/* AKCIPHER (Asymmetric Key Cipher) service */
#define VIRTIO_CRYPTO_SERVICE_AKCIPHER 4
\end{lstlisting}

The above constants designate bits used to indicate the which of crypto services are
offered by the device as described in, see \ref{sec:Device Types / Crypto Device / Device configuration layout}.

\subsubsection{CIPHER services}\label{sec:Device Types / Crypto Device / Supported crypto services / CIPHER services}

The following CIPHER algorithms are defined:

\begin{lstlisting}
#define VIRTIO_CRYPTO_NO_CIPHER                 0
#define VIRTIO_CRYPTO_CIPHER_ARC4               1
#define VIRTIO_CRYPTO_CIPHER_AES_ECB            2
#define VIRTIO_CRYPTO_CIPHER_AES_CBC            3
#define VIRTIO_CRYPTO_CIPHER_AES_CTR            4
#define VIRTIO_CRYPTO_CIPHER_DES_ECB            5
#define VIRTIO_CRYPTO_CIPHER_DES_CBC            6
#define VIRTIO_CRYPTO_CIPHER_3DES_ECB           7
#define VIRTIO_CRYPTO_CIPHER_3DES_CBC           8
#define VIRTIO_CRYPTO_CIPHER_3DES_CTR           9
#define VIRTIO_CRYPTO_CIPHER_KASUMI_F8          10
#define VIRTIO_CRYPTO_CIPHER_SNOW3G_UEA2        11
#define VIRTIO_CRYPTO_CIPHER_AES_F8             12
#define VIRTIO_CRYPTO_CIPHER_AES_XTS            13
#define VIRTIO_CRYPTO_CIPHER_ZUC_EEA3           14
\end{lstlisting}

The above constants have two usages:
\begin{enumerate}
\item As bit numbers, used to tell the driver which CIPHER algorithms
are supported by the device, see \ref{sec:Device Types / Crypto Device / Device configuration layout}.
\item As values, used to designate the algorithm in (CIPHER type) crypto
operation requests, see \ref{sec:Device Types / Crypto Device / Device Operation / Control Virtqueue / Session operation}.
\end{enumerate}

\subsubsection{HASH services}\label{sec:Device Types / Crypto Device / Supported crypto services / HASH services}

The following HASH algorithms are defined:

\begin{lstlisting}
#define VIRTIO_CRYPTO_NO_HASH            0
#define VIRTIO_CRYPTO_HASH_MD5           1
#define VIRTIO_CRYPTO_HASH_SHA1          2
#define VIRTIO_CRYPTO_HASH_SHA_224       3
#define VIRTIO_CRYPTO_HASH_SHA_256       4
#define VIRTIO_CRYPTO_HASH_SHA_384       5
#define VIRTIO_CRYPTO_HASH_SHA_512       6
#define VIRTIO_CRYPTO_HASH_SHA3_224      7
#define VIRTIO_CRYPTO_HASH_SHA3_256      8
#define VIRTIO_CRYPTO_HASH_SHA3_384      9
#define VIRTIO_CRYPTO_HASH_SHA3_512      10
#define VIRTIO_CRYPTO_HASH_SHA3_SHAKE128      11
#define VIRTIO_CRYPTO_HASH_SHA3_SHAKE256      12
\end{lstlisting}

The above constants have two usages:
\begin{enumerate}
\item As bit numbers, used to tell the driver which HASH algorithms
are supported by the device, see \ref{sec:Device Types / Crypto Device / Device configuration layout}.
\item As values, used to designate the algorithm in (HASH type) crypto
operation requires, see \ref{sec:Device Types / Crypto Device / Device Operation / Control Virtqueue / Session operation}.
\end{enumerate}

\subsubsection{MAC services}\label{sec:Device Types / Crypto Device / Supported crypto services / MAC services}

The following MAC algorithms are defined:

\begin{lstlisting}
#define VIRTIO_CRYPTO_NO_MAC                       0
#define VIRTIO_CRYPTO_MAC_HMAC_MD5                 1
#define VIRTIO_CRYPTO_MAC_HMAC_SHA1                2
#define VIRTIO_CRYPTO_MAC_HMAC_SHA_224             3
#define VIRTIO_CRYPTO_MAC_HMAC_SHA_256             4
#define VIRTIO_CRYPTO_MAC_HMAC_SHA_384             5
#define VIRTIO_CRYPTO_MAC_HMAC_SHA_512             6
#define VIRTIO_CRYPTO_MAC_CMAC_3DES                25
#define VIRTIO_CRYPTO_MAC_CMAC_AES                 26
#define VIRTIO_CRYPTO_MAC_KASUMI_F9                27
#define VIRTIO_CRYPTO_MAC_SNOW3G_UIA2              28
#define VIRTIO_CRYPTO_MAC_GMAC_AES                 41
#define VIRTIO_CRYPTO_MAC_GMAC_TWOFISH             42
#define VIRTIO_CRYPTO_MAC_CBCMAC_AES               49
#define VIRTIO_CRYPTO_MAC_CBCMAC_KASUMI_F9         50
#define VIRTIO_CRYPTO_MAC_XCBC_AES                 53
#define VIRTIO_CRYPTO_MAC_ZUC_EIA3                 54
\end{lstlisting}

The above constants have two usages:
\begin{enumerate}
\item As bit numbers, used to tell the driver which MAC algorithms
are supported by the device, see \ref{sec:Device Types / Crypto Device / Device configuration layout}.
\item As values, used to designate the algorithm in (MAC type) crypto
operation requests, see \ref{sec:Device Types / Crypto Device / Device Operation / Control Virtqueue / Session operation}.
\end{enumerate}

\subsubsection{AEAD services}\label{sec:Device Types / Crypto Device / Supported crypto services / AEAD services}

The following AEAD algorithms are defined:

\begin{lstlisting}
#define VIRTIO_CRYPTO_NO_AEAD     0
#define VIRTIO_CRYPTO_AEAD_GCM    1
#define VIRTIO_CRYPTO_AEAD_CCM    2
#define VIRTIO_CRYPTO_AEAD_CHACHA20_POLY1305  3
\end{lstlisting}

The above constants have two usages:
\begin{enumerate}
\item As bit numbers, used to tell the driver which AEAD algorithms
are supported by the device, see \ref{sec:Device Types / Crypto Device / Device configuration layout}.
\item As values, used to designate the algorithm in (DEAD type) crypto
operation requests, see \ref{sec:Device Types / Crypto Device / Device Operation / Control Virtqueue / Session operation}.
\end{enumerate}

\subsubsection{AKCIPHER services}\label{sec: Device Types / Crypto Device / Supported crypto services / AKCIPHER services}

The following AKCIPHER algorithms are defined:
\begin{lstlisting}
#define VIRTIO_CRYPTO_NO_AKCIPHER 0
#define VIRTIO_CRYPTO_AKCIPHER_RSA   1
#define VIRTIO_CRYPTO_AKCIPHER_ECDSA 2
\end{lstlisting}

The above constants have two usages:
\begin{enumerate}
\item As bit numbers, used to tell the driver which AKCIPHER algorithms
are supported by the device, see \ref{sec:Device Types / Crypto Device / Device configuration layout}.
\item As values, used to designate the algorithm in asymmetric crypto operation requests,
see \ref{sec:Device Types / Crypto Device / Device Operation / Control Virtqueue / Session operation}.
\end{enumerate}


\subsection{Device configuration layout}\label{sec:Device Types / Crypto Device / Device configuration layout}

Crypto device configuration uses the following layout structure:

\begin{lstlisting}
struct virtio_crypto_config {
    le32 status;
    le32 max_dataqueues;
    le32 crypto_services;
    /* Detailed algorithms mask */
    le32 cipher_algo_l;
    le32 cipher_algo_h;
    le32 hash_algo;
    le32 mac_algo_l;
    le32 mac_algo_h;
    le32 aead_algo;
    /* Maximum length of cipher key in bytes */
    le32 max_cipher_key_len;
    /* Maximum length of authenticated key in bytes */
    le32 max_auth_key_len;
    le32 akcipher_algo;
    /* Maximum size of each crypto request's content in bytes */
    le64 max_size;
};
\end{lstlisting}

\begin{description}
\item Currently, only one \field{status} bit is defined: VIRTIO_CRYPTO_S_HW_READY
    set indicates that the device is ready to process requests, this bit is read-only
    for the driver
\begin{lstlisting}
#define VIRTIO_CRYPTO_S_HW_READY  (1 << 0)
\end{lstlisting}

\item [\field{max_dataqueues}] is the maximum number of data virtqueues that can
    be configured by the device. The driver MAY use only one data queue, or it
    can use more to achieve better performance.

\item [\field{crypto_services}] crypto service offered, see \ref{sec:Device Types / Crypto Device / Supported crypto services}.

\item [\field{cipher_algo_l}] CIPHER algorithms bits 0-31, see \ref{sec:Device Types / Crypto Device / Supported crypto services  / CIPHER services}.

\item [\field{cipher_algo_h}] CIPHER algorithms bits 32-63, see \ref{sec:Device Types / Crypto Device / Supported crypto services  / CIPHER services}.

\item [\field{hash_algo}] HASH algorithms bits, see \ref{sec:Device Types / Crypto Device / Supported crypto services  / HASH services}.

\item [\field{mac_algo_l}] MAC algorithms bits 0-31, see \ref{sec:Device Types / Crypto Device / Supported crypto services  / MAC services}.

\item [\field{mac_algo_h}] MAC algorithms bits 32-63, see \ref{sec:Device Types / Crypto Device / Supported crypto services  / MAC services}.

\item [\field{aead_algo}] AEAD algorithms bits, see \ref{sec:Device Types / Crypto Device / Supported crypto services  / AEAD services}.

\item [\field{max_cipher_key_len}] is the maximum length of cipher key supported by the device.

\item [\field{max_auth_key_len}] is the maximum length of authenticated key supported by the device.

\item [\field{akcipher_algo}] AKCIPHER algorithms bit 0-31, see \ref{sec: Device Types / Crypto Device / Supported crypto services / AKCIPHER services}.

\item [\field{max_size}] is the maximum size of the variable-length parameters of
    data operation of each crypto request's content supported by the device.
\end{description}

\begin{note}
Unless explicitly stated otherwise all lengths and sizes are in bytes.
\end{note}

\devicenormative{\subsubsection}{Device configuration layout}{Device Types / Crypto Device / Device configuration layout}

\begin{itemize*}
\item The device MUST set \field{max_dataqueues} to between 1 and 65535 inclusive.
\item The device MUST set the \field{status} with valid flags, undefined flags MUST NOT be set.
\item The device MUST accept and handle requests after \field{status} is set to VIRTIO_CRYPTO_S_HW_READY.
\item The device MUST set \field{crypto_services} based on the crypto services the device offers.
\item The device MUST set detailed algorithms masks for each service advertised by \field{crypto_services}.
    The device MUST NOT set the not defined algorithms bits.
\item The device MUST set \field{max_size} to show the maximum size of crypto request the device supports.
\item The device MUST set \field{max_cipher_key_len} to show the maximum length of cipher key if the
    device supports CIPHER service.
\item The device MUST set \field{max_auth_key_len} to show the maximum length of authenticated key if
    the device supports MAC service.
\end{itemize*}

\drivernormative{\subsubsection}{Device configuration layout}{Device Types / Crypto Device / Device configuration layout}

\begin{itemize*}
\item The driver MUST read the \field{status} from the bottom bit of status to check whether the
    VIRTIO_CRYPTO_S_HW_READY is set, and the driver MUST reread it after device reset.
\item The driver MUST NOT transmit any requests to the device if the VIRTIO_CRYPTO_S_HW_READY is not set.
\item The driver MUST read \field{max_dataqueues} field to discover the number of data queues the device supports.
\item The driver MUST read \field{crypto_services} field to discover which services the device is able to offer.
\item The driver SHOULD ignore the not defined algorithms bits.
\item The driver MUST read the detailed algorithms fields based on \field{crypto_services} field.
\item The driver SHOULD read \field{max_size} to discover the maximum size of the variable-length
    parameters of data operation of the crypto request's content the device supports and MUST
    guarantee that the size of each crypto request's content is within the \field{max_size}, otherwise
    the request will fail and the driver MUST reset the device.
\item The driver SHOULD read \field{max_cipher_key_len} to discover the maximum length of cipher key
    the device supports and MUST guarantee that the \field{key_len} (CIPHER service or AEAD service) is within
    the \field{max_cipher_key_len} of the device configuration, otherwise the request will fail.
\item The driver SHOULD read \field{max_auth_key_len} to discover the maximum length of authenticated
    key the device supports and MUST guarantee that the \field{auth_key_len} (MAC service) is within the
    \field{max_auth_key_len} of the device configuration, otherwise the request will fail.
\end{itemize*}

\subsection{Device Initialization}\label{sec:Device Types / Crypto Device / Device Initialization}

\drivernormative{\subsubsection}{Device Initialization}{Device Types / Crypto Device / Device Initialization}

\begin{itemize*}
\item The driver MUST configure and initialize all virtqueues.
\item The driver MUST read the supported crypto services from bits of \field{crypto_services}.
\item The driver MUST read the supported algorithms based on \field{crypto_services} field.
\end{itemize*}

\subsection{Device Operation}\label{sec:Device Types / Crypto Device / Device Operation}

The operation of a virtio crypto device is driven by requests placed on the virtqueues.
Requests consist of a queue-type specific header (specifying among others the operation)
and an operation specific payload.

If VIRTIO_CRYPTO_F_REVISION_1 is negotiated the device may support both session mode
(See \ref{sec:Device Types / Crypto Device / Device Operation / Control Virtqueue / Session operation})
and stateless mode operation requests.
In stateless mode all operation parameters are supplied as a part of each request,
while in session mode, some or all operation parameters are managed within the
session. Stateless mode is guarded by feature bits 0-4 on a service level. If
stateless mode is negotiated for a service, the service accepts both session
mode and stateless requests; otherwise stateless mode requests are rejected
(via operation status).

\subsubsection{Operation Status}\label{sec:Device Types / Crypto Device / Device Operation / Operation status}
The device MUST return a status code as part of the operation (both session
operation and service operation) result. The valid operation status as follows:

\begin{lstlisting}
enum VIRTIO_CRYPTO_STATUS {
    VIRTIO_CRYPTO_OK = 0,
    VIRTIO_CRYPTO_ERR = 1,
    VIRTIO_CRYPTO_BADMSG = 2,
    VIRTIO_CRYPTO_NOTSUPP = 3,
    VIRTIO_CRYPTO_INVSESS = 4,
    VIRTIO_CRYPTO_NOSPC = 5,
    VIRTIO_CRYPTO_KEY_REJECTED = 6,
    VIRTIO_CRYPTO_MAX
};
\end{lstlisting}

\begin{itemize*}
\item VIRTIO_CRYPTO_OK: success.
\item VIRTIO_CRYPTO_BADMSG: authentication failed (only when AEAD decryption).
\item VIRTIO_CRYPTO_NOTSUPP: operation or algorithm is unsupported.
\item VIRTIO_CRYPTO_INVSESS: invalid session ID when executing crypto operations.
\item VIRTIO_CRYPTO_NOSPC: no free session ID (only when the VIRTIO_CRYPTO_F_REVISION_1
    feature bit is negotiated).
\item VIRTIO_CRYPTO_KEY_REJECTED: signature verification failed (only when AKCIPHER verification).
\item VIRTIO_CRYPTO_ERR: any failure not mentioned above occurs.
\end{itemize*}

\subsubsection{Control Virtqueue}\label{sec:Device Types / Crypto Device / Device Operation / Control Virtqueue}

The driver uses the control virtqueue to send control commands to the
device, such as session operations (See \ref{sec:Device Types / Crypto Device / Device
Operation / Control Virtqueue / Session operation}).

The header for controlq is of the following form:
\begin{lstlisting}
#define VIRTIO_CRYPTO_OPCODE(service, op)   (((service) << 8) | (op))

struct virtio_crypto_ctrl_header {
#define VIRTIO_CRYPTO_CIPHER_CREATE_SESSION \
       VIRTIO_CRYPTO_OPCODE(VIRTIO_CRYPTO_SERVICE_CIPHER, 0x02)
#define VIRTIO_CRYPTO_CIPHER_DESTROY_SESSION \
       VIRTIO_CRYPTO_OPCODE(VIRTIO_CRYPTO_SERVICE_CIPHER, 0x03)
#define VIRTIO_CRYPTO_HASH_CREATE_SESSION \
       VIRTIO_CRYPTO_OPCODE(VIRTIO_CRYPTO_SERVICE_HASH, 0x02)
#define VIRTIO_CRYPTO_HASH_DESTROY_SESSION \
       VIRTIO_CRYPTO_OPCODE(VIRTIO_CRYPTO_SERVICE_HASH, 0x03)
#define VIRTIO_CRYPTO_MAC_CREATE_SESSION \
       VIRTIO_CRYPTO_OPCODE(VIRTIO_CRYPTO_SERVICE_MAC, 0x02)
#define VIRTIO_CRYPTO_MAC_DESTROY_SESSION \
       VIRTIO_CRYPTO_OPCODE(VIRTIO_CRYPTO_SERVICE_MAC, 0x03)
#define VIRTIO_CRYPTO_AEAD_CREATE_SESSION \
       VIRTIO_CRYPTO_OPCODE(VIRTIO_CRYPTO_SERVICE_AEAD, 0x02)
#define VIRTIO_CRYPTO_AEAD_DESTROY_SESSION \
       VIRTIO_CRYPTO_OPCODE(VIRTIO_CRYPTO_SERVICE_AEAD, 0x03)
#define VIRTIO_CRYPTO_AKCIPHER_CREATE_SESSION \
       VIRTIO_CRYPTO_OPCODE(VIRTIO_CRYPTO_SERVICE_AKCIPHER, 0x04)
#define VIRTIO_CRYPTO_AKCIPHER_DESTROY_SESSION \
       VIRTIO_CRYPTO_OPCDE(VIRTIO_CRYPTO_SERVICE_AKCIPHER, 0x05)
    le32 opcode;
    /* algo should be service-specific algorithms */
    le32 algo;
    le32 flag;
    le32 reserved;
};
\end{lstlisting}

The controlq request is composed of four parts:
\begin{lstlisting}
struct virtio_crypto_op_ctrl_req {
    /* Device read only portion */

    struct virtio_crypto_ctrl_header header;

#define VIRTIO_CRYPTO_CTRLQ_OP_SPEC_HDR_LEGACY 56
    /* fixed length fields, opcode specific */
    u8 op_flf[flf_len];

    /* variable length fields, opcode specific */
    u8 op_vlf[vlf_len];

    /* Device write only portion */

    /* op result or completion status */
    u8 op_outcome[outcome_len];
};
\end{lstlisting}

\field{header} is a general header (see above).

\field{op_flf} is the opcode (in \field{header}) specific fixed-length parameters.

\field{flf_len} depends on the VIRTIO_CRYPTO_F_REVISION_1 feature bit (see below).

\field{op_vlf} is the opcode (in \field{header}) specific variable-length parameters.

\field{vlf_len} is the size of the specific structure used.
\begin{note}
The \field{vlf_len} of session-destroy operation and the hash-session-create
operation is ZERO.
\end{note}

\begin{itemize*}
\item If the opcode (in \field{header}) is VIRTIO_CRYPTO_CIPHER_CREATE_SESSION
    then \field{op_flf} is struct virtio_crypto_sym_create_session_flf if
    VIRTIO_CRYPTO_F_REVISION_1 is negotiated and struct virtio_crypto_sym_create_session_flf is
    padded to 56 bytes if NOT negotiated, and \field{op_vlf} is struct
    virtio_crypto_sym_create_session_vlf.
\item If the opcode (in \field{header}) is VIRTIO_CRYPTO_HASH_CREATE_SESSION
    then \field{op_flf} is struct virtio_crypto_hash_create_session_flf if
    VIRTIO_CRYPTO_F_REVISION_1 is negotiated and struct virtio_crypto_hash_create_session_flf is
    padded to 56 bytes if NOT negotiated.
\item If the opcode (in \field{header}) is VIRTIO_CRYPTO_MAC_CREATE_SESSION
    then \field{op_flf} is struct virtio_crypto_mac_create_session_flf if
    VIRTIO_CRYPTO_F_REVISION_1 is negotiated and struct virtio_crypto_mac_create_session_flf is
    padded to 56 bytes if NOT negotiated, and \field{op_vlf} is struct
    virtio_crypto_mac_create_session_vlf.
\item If the opcode (in \field{header}) is VIRTIO_CRYPTO_AEAD_CREATE_SESSION
    then \field{op_flf} is struct virtio_crypto_aead_create_session_flf if
    VIRTIO_CRYPTO_F_REVISION_1 is negotiated and struct virtio_crypto_aead_create_session_flf is
    padded to 56 bytes if NOT negotiated, and \field{op_vlf} is struct
    virtio_crypto_aead_create_session_vlf.
\item If the opcode (in \field{header}) is VIRTIO_CRYPTO_AKCIPHER_CREATE_SESSION
    then \field{op_flf} is struct virtio_crypto_akcipher_create_session_flf if
    VIRTIO_CRYPTO_F_REVISION_1 is negotiated and struct virtio_crypto_akcipher_create_session_flf is
    padded to 56 bytes if NOT negotiated, and \field{op_vlf} is struct
    virtio_crypto_akcipher_create_session_vlf.
\item If the opcode (in \field{header}) is VIRTIO_CRYPTO_CIPHER_DESTROY_SESSION
    or VIRTIO_CRYPTO_HASH_DESTROY_SESSION or VIRTIO_CRYPTO_MAC_DESTROY_SESSION or
    VIRTIO_CRYPTO_AEAD_DESTROY_SESSION then \field{op_flf} is struct
    virtio_crypto_destroy_session_flf if VIRTIO_CRYPTO_F_REVISION_1 is negotiated and
    struct virtio_crypto_destroy_session_flf is padded to 56 bytes if NOT negotiated.
\end{itemize*}

\field{op_outcome} stores the result of operation and must be struct
virtio_crypto_destroy_session_input for destroy session or
struct virtio_crypto_create_session_input for create session.

\field{outcome_len} is the size of the structure used.


\paragraph{Session operation}\label{sec:Device Types / Crypto Device / Device
Operation / Control Virtqueue / Session operation}

The session is a handle which describes the cryptographic parameters to be
applied to a number of buffers.

The following structure stores the result of session creation set by the device:

\begin{lstlisting}
struct virtio_crypto_create_session_input {
    le64 session_id;
    le32 status;
    le32 padding;
};
\end{lstlisting}

A request to destroy a session includes the following information:

\begin{lstlisting}
struct virtio_crypto_destroy_session_flf {
    /* Device read only portion */
    le64  session_id;
};

struct virtio_crypto_destroy_session_input {
    /* Device write only portion */
    u8  status;
};
\end{lstlisting}


\subparagraph{Session operation: HASH session}\label{sec:Device Types / Crypto Device / Device
Operation / Control Virtqueue / Session operation / Session operation: HASH session}

The fixed-length parameters of HASH session requests is as follows:

\begin{lstlisting}
struct virtio_crypto_hash_create_session_flf {
    /* Device read only portion */

    /* See VIRTIO_CRYPTO_HASH_* above */
    le32 algo;
    /* hash result length */
    le32 hash_result_len;
};
\end{lstlisting}


\subparagraph{Session operation: MAC session}\label{sec:Device Types / Crypto Device / Device
Operation / Control Virtqueue / Session operation / Session operation: MAC session}

The fixed-length and the variable-length parameters of MAC session requests are as follows:

\begin{lstlisting}
struct virtio_crypto_mac_create_session_flf {
    /* Device read only portion */

    /* See VIRTIO_CRYPTO_MAC_* above */
    le32 algo;
    /* hash result length */
    le32 hash_result_len;
    /* length of authenticated key */
    le32 auth_key_len;
    le32 padding;
};

struct virtio_crypto_mac_create_session_vlf {
    /* Device read only portion */

    /* The authenticated key */
    u8 auth_key[auth_key_len];
};
\end{lstlisting}

The length of \field{auth_key} is specified in \field{auth_key_len} in the struct
virtio_crypto_mac_create_session_flf.


\subparagraph{Session operation: Symmetric algorithms session}\label{sec:Device Types / Crypto Device / Device
Operation / Control Virtqueue / Session operation / Session operation: Symmetric algorithms session}

The request of symmetric session could be the CIPHER algorithms request
or the chain algorithms (chaining CIPHER and HASH/MAC) request.

The fixed-length and the variable-length parameters of CIPHER session requests are as follows:

\begin{lstlisting}
struct virtio_crypto_cipher_session_flf {
    /* Device read only portion */

    /* See VIRTIO_CRYPTO_CIPHER* above */
    le32 algo;
    /* length of key */
    le32 key_len;
#define VIRTIO_CRYPTO_OP_ENCRYPT  1
#define VIRTIO_CRYPTO_OP_DECRYPT  2
    /* encryption or decryption */
    le32 op;
    le32 padding;
};

struct virtio_crypto_cipher_session_vlf {
    /* Device read only portion */

    /* The cipher key */
    u8 cipher_key[key_len];
};
\end{lstlisting}

The length of \field{cipher_key} is specified in \field{key_len} in the struct
virtio_crypto_cipher_session_flf.

The fixed-length and the variable-length parameters of Chain session requests are as follows:

\begin{lstlisting}
struct virtio_crypto_alg_chain_session_flf {
    /* Device read only portion */

#define VIRTIO_CRYPTO_SYM_ALG_CHAIN_ORDER_HASH_THEN_CIPHER  1
#define VIRTIO_CRYPTO_SYM_ALG_CHAIN_ORDER_CIPHER_THEN_HASH  2
    le32 alg_chain_order;
/* Plain hash */
#define VIRTIO_CRYPTO_SYM_HASH_MODE_PLAIN    1
/* Authenticated hash (mac) */
#define VIRTIO_CRYPTO_SYM_HASH_MODE_AUTH     2
/* Nested hash */
#define VIRTIO_CRYPTO_SYM_HASH_MODE_NESTED   3
    le32 hash_mode;
    struct virtio_crypto_cipher_session_flf cipher_hdr;

#define VIRTIO_CRYPTO_ALG_CHAIN_SESS_OP_SPEC_HDR_SIZE  16
    /* fixed length fields, algo specific */
    u8 algo_flf[VIRTIO_CRYPTO_ALG_CHAIN_SESS_OP_SPEC_HDR_SIZE];

    /* length of the additional authenticated data (AAD) in bytes */
    le32 aad_len;
    le32 padding;
};

struct virtio_crypto_alg_chain_session_vlf {
    /* Device read only portion */

    /* The cipher key */
    u8 cipher_key[key_len];
    /* The authenticated key */
    u8 auth_key[auth_key_len];
};
\end{lstlisting}

\field{hash_mode} decides the type used by \field{algo_flf}.

\field{algo_flf} is fixed to 16 bytes and MUST contains or be one of
the following types:
\begin{itemize*}
\item struct virtio_crypto_hash_create_session_flf
\item struct virtio_crypto_mac_create_session_flf
\end{itemize*}
The data of unused part (if has) in \field{algo_flf} will be ignored.

The length of \field{cipher_key} is specified in \field{key_len} in \field{cipher_hdr}.

The length of \field{auth_key} is specified in \field{auth_key_len} in struct
virtio_crypto_mac_create_session_flf.

The fixed-length parameters of Symmetric session requests are as follows:

\begin{lstlisting}
struct virtio_crypto_sym_create_session_flf {
    /* Device read only portion */

#define VIRTIO_CRYPTO_SYM_SESS_OP_SPEC_HDR_SIZE  48
    /* fixed length fields, opcode specific */
    u8 op_flf[VIRTIO_CRYPTO_SYM_SESS_OP_SPEC_HDR_SIZE];

/* No operation */
#define VIRTIO_CRYPTO_SYM_OP_NONE  0
/* Cipher only operation on the data */
#define VIRTIO_CRYPTO_SYM_OP_CIPHER  1
/* Chain any cipher with any hash or mac operation. The order
   depends on the value of alg_chain_order param */
#define VIRTIO_CRYPTO_SYM_OP_ALGORITHM_CHAINING  2
    le32 op_type;
    le32 padding;
};
\end{lstlisting}

\field{op_flf} is fixed to 48 bytes, MUST contains or be one of
the following types:
\begin{itemize*}
\item struct virtio_crypto_cipher_session_flf
\item struct virtio_crypto_alg_chain_session_flf
\end{itemize*}
The data of unused part (if has) in \field{op_flf} will be ignored.

\field{op_type} decides the type used by \field{op_flf}.

The variable-length parameters of Symmetric session requests are as follows:

\begin{lstlisting}
struct virtio_crypto_sym_create_session_vlf {
    /* Device read only portion */
    /* variable length fields, opcode specific */
    u8 op_vlf[vlf_len];
};
\end{lstlisting}

\field{op_vlf} MUST contains or be one of the following types:
\begin{itemize*}
\item struct virtio_crypto_cipher_session_vlf
\item struct virtio_crypto_alg_chain_session_vlf
\end{itemize*}

\field{op_type} in struct virtio_crypto_sym_create_session_flf decides the
type used by \field{op_vlf}.

\field{vlf_len} is the size of the specific structure used.


\subparagraph{Session operation: AEAD session}\label{sec:Device Types / Crypto Device / Device
Operation / Control Virtqueue / Session operation / Session operation: AEAD session}

The fixed-length and the variable-length parameters of AEAD session requests are as follows:

\begin{lstlisting}
struct virtio_crypto_aead_create_session_flf {
    /* Device read only portion */

    /* See VIRTIO_CRYPTO_AEAD_* above */
    le32 algo;
    /* length of key */
    le32 key_len;
    /* Authentication tag length */
    le32 tag_len;
    /* The length of the additional authenticated data (AAD) in bytes */
    le32 aad_len;
    /* encryption or decryption, See above VIRTIO_CRYPTO_OP_* */
    le32 op;
    le32 padding;
};

struct virtio_crypto_aead_create_session_vlf {
    /* Device read only portion */
    u8 key[key_len];
};
\end{lstlisting}

The length of \field{key} is specified in \field{key_len} in struct
virtio_crypto_aead_create_session_flf.

\subparagraph{Session operation: AKCIPHER session}\label{sec:Device Types / Crypto Device / Device
Operation / Control Virtqueue / Session operation / Session operation: AKCIPHER session}

Due to the complexity of asymmetric key algorithms, different algorithms
require different parameters. The following data structures are used as
supplementary parameters to describe the asymmetric algorithm sessions.

For the RSA algorithm, the extra parameters are as follows:
\begin{lstlisting}
struct virtio_crypto_rsa_session_para {
#define VIRTIO_CRYPTO_RSA_RAW_PADDING   0
#define VIRTIO_CRYPTO_RSA_PKCS1_PADDING 1
    le32 padding_algo;

#define VIRTIO_CRYPTO_RSA_NO_HASH   0
#define VIRTIO_CRYPTO_RSA_MD2       1
#define VIRTIO_CRYPTO_RSA_MD3       2
#define VIRTIO_CRYPTO_RSA_MD4       3
#define VIRTIO_CRYPTO_RSA_MD5       4
#define VIRTIO_CRYPTO_RSA_SHA1      5
#define VIRTIO_CRYPTO_RSA_SHA256    6
#define VIRTIO_CRYPTO_RSA_SHA384    7
#define VIRTIO_CRYPTO_RSA_SHA512    8
#define VIRTIO_CRYPTO_RSA_SHA224    9
    le32 hash_algo;
};
\end{lstlisting}

\field{padding_algo} specifies the padding method used by RSA sessions.
\begin{itemize*}
\item If VIRTIO_CRYPTO_RSA_RAW_PADDING is specified, 1) \field{hash_algo}
is ignored, 2) ciphertext and plaintext MUST be padded with leading zeros,
3) and RSA sessions with VIRTIO_CRYPTO_RSA_RAW_PADDING MUST not be used
for verification and signing operations.
\item If VIRTIO_CRYPTO_RSA_PKCS1_PADDING is specified, EMSA-PKCS1-v1_5 padding method
is used (see \hyperref[intro:rfc3447]{PKCS\#1}), \field{hash_algo} specifies how the
digest of the data passed to RSA sessions is calculated when verifying and signing.
It only affects the padding algorithm and is ignored during encryption and decryption.
\end{itemize*}

The ECC algorithms such as the ECDSA algorithm, cannot use custom curves, only the
following known curves can be used (see \hyperref[intro:NIST]{NIST-recommended curves}).

\begin{lstlisting}
#define VIRTIO_CRYPTO_CURVE_UNKNOWN   0
#define VIRTIO_CRYPTO_CURVE_NIST_P192 1
#define VIRTIO_CRYPTO_CURVE_NIST_P224 2
#define VIRTIO_CRYPTO_CURVE_NIST_P256 3
#define VIRTIO_CRYPTO_CURVE_NIST_P384 4
#define VIRTIO_CRYPTO_CURVE_NIST_P521 5
\end{lstlisting}

For the ECDSA algorithm, the extra parameters are as follows:
\begin{lstlisting}
struct virtio_crypto_ecdsa_session_para {
    /* See VIRTIO_CRYPTO_CURVE_* above */
    le32 curve_id;
};
\end{lstlisting}

The fixed-length and the variable-length parameters of AKCIPHER session requests are as follows:
\begin{lstlisting}
struct virtio_crypto_akcipher_create_session_flf {
    /* Device read only portion */

    /* See VIRTIO_CRYPTO_AKCIPHER_* above */
    le32 algo;
#define VIRTIO_CRYPTO_AKCIPHER_KEY_TYPE_PUBLIC 1
#define VIRTIO_CRYPTO_AKCIPHER_KEY_TYPE_PRIVATE 2
    le32 key_type;
    /* length of key */
    le32 key_len;

#define VIRTIO_CRYPTO_AKCIPHER_SESS_ALGO_SPEC_HDR_SIZE 44
    u8 algo_flf[VIRTIO_CRYPTO_AKCIPHER_SESS_ALGO_SPEC_HDR_SIZE];
};

struct virtio_crypto_akcipher_create_session_vlf {
    /* Device read only portion */
    u8 key[key_len];
};
\end{lstlisting}

\field{algo} decides the type used by \field{algo_flf}.
\field{algo_flf} is fixed to 44 bytes and MUST contains of be one the
following structures:
\begin{itemize*}
\item struct virtio_crypto_rsa_session_para
\item struct virtio_crypto_ecdsa_session_para
\end{itemize*}

The length of \field{key} is specified in \field{key_len} in the struct
virtio_crypto_akcipher_create_session_flf.

For the RSA algorithm, the key needs to be encoded according to
\hyperref[intro:rfc3447]{PKCS\#1}. The private key is described with the
RSAPrivateKey structure, and the public key is described with the RSAPublicKey
structure. These ASN.1 structures are encoded in DER encoding rules (see
\hyperref[intro:rfc6025]{rfc6025}).

\begin{lstlisting}
RSAPrivateKey ::= SEQUENCE {
    version          INTEGER,
    modulus          INTEGER,
    publicExponent   INTEGER,
    privateExponent  INTEGER,
    prime1           INTEGER,
    prime2           INTEGER,
    exponent1        INTEGER,
    exponent1        INTEGER,
    coefficient      INTEGER,
    otherPrimeInfos  OtherPrimeInfos OPTIONAL
}

OtherPrimeInfos ::= SEQUENCE SIZE(1...MAX) OF OtherPrimeInfo

OtherPrimeINfo ::= SEQUENCE {
    prime           INTEGER,
    exponent        INTEGER,
    coefficient     INTEGER
}

RSAPublicKey ::= SEQUENCE {
    modulus         INTEGER,
    publicExponent  INTEGER
}
\end{lstlisting}

For the ECDSA algorithm, the private key is encoded according to
\hyperref[intro:rfc5915]{RFC5915}, the private key of the ECDSA algorithm
is described by the ASN.1 structure ECPrivateKey and encoded with DER
encoding rules (see \hyperref[intro:rfc6025]{rfc6025}).

\begin{lstlisting}
ECPrivateKey ::= SEQUNCE {
    version         INTEGER,
    privateKey      OCTET STRING,
    parameters [0]  ECParameters {{ NamedCurve }} OPTIONAL,
    publicKey  [1]  BIT STRING OPTIONAL
}
\end{lstlisting}

The public key of the ECDSA algorithm is encoded according to \hyperref[intro:SEC1]{SEC1},
and the public key of ECDSA is described by the ASN.1 structure ECPoint.
When initializing a session with ECDSA public key, the ECPoint is DER encoded and the
\field{key} only contains the value part of ECPoint, that is, the header part of the
OCTET STRING will be omitted (see \hyperref[intro:rfc6025]{rfc6025}).

\begin{lstlisting}
ECPoint ::= OCTET STRING
\end{lstlisting}

The length of \field{key} is specified in \field{key_len} in
struct virtio_crypto_akcipher_create_session_flf.

\drivernormative{\subparagraph}{Session operation: create session}{Device Types / Crypto Device / Device
Operation / Control Virtqueue / Session operation / Session operation: create session}

\begin{itemize*}
\item The driver MUST set the \field{opcode} field based on service type: CIPHER, HASH, MAC, AEAD or AKCIPHER.
\item The driver MUST set the control general header, the opcode specific header,
    the opcode specific extra parameters and the opcode specific outcome buffer in turn.
    See \ref{sec:Device Types / Crypto Device / Device Operation / Control Virtqueue}.
\item The driver MUST set the \field{reversed} field to zero.
\end{itemize*}

\devicenormative{\subparagraph}{Session operation: create session}{Device Types / Crypto Device / Device
Operation / Control Virtqueue / Session operation / Session operation: create session}

\begin{itemize*}
\item The device MUST use the corresponding opcode specific structure according to the
    \field{opcode} in the control general header.
\item The device MUST extract extra parameters according to the structures used.
\item The device MUST set the \field{status} field to one of the following values of enum
    VIRTIO_CRYPTO_STATUS after finish a session creation:
\begin{itemize*}
\item VIRTIO_CRYPTO_OK if a session is created successfully.
\item VIRTIO_CRYPTO_NOTSUPP if the requested algorithm or operation is unsupported.
\item VIRTIO_CRYPTO_NOSPC if no free session ID (only when the VIRTIO_CRYPTO_F_REVISION_1
    feature bit is negotiated).
\item VIRTIO_CRYPTO_ERR if failure not mentioned above occurs.
\end{itemize*}
\item The device MUST set the \field{session_id} field to a unique session identifier only
    if the status is set to VIRTIO_CRYPTO_OK.
\end{itemize*}

\drivernormative{\subparagraph}{Session operation: destroy session}{Device Types / Crypto Device / Device
Operation / Control Virtqueue / Session operation / Session operation: destroy session}

\begin{itemize*}
\item The driver MUST set the \field{opcode} field based on service type: CIPHER, HASH, MAC, AEAD or AKCIPHER.
\item The driver MUST set the \field{session_id} to a valid value assigned by the device
    when the session was created.
\end{itemize*}

\devicenormative{\subparagraph}{Session operation: destroy session}{Device Types / Crypto Device / Device
Operation / Control Virtqueue / Session operation / Session operation: destroy session}

\begin{itemize*}
\item The device MUST set the \field{status} field to one of the following values of enum VIRTIO_CRYPTO_STATUS.
\begin{itemize*}
\item VIRTIO_CRYPTO_OK if a session is created successfully.
\item VIRTIO_CRYPTO_ERR if any failure occurs.
\end{itemize*}
\end{itemize*}


\subsubsection{Data Virtqueue}\label{sec:Device Types / Crypto Device / Device Operation / Data Virtqueue}

The driver uses the data virtqueues to transmit crypto operation requests to the device,
and completes the crypto operations.

The header for dataq is as follows:

\begin{lstlisting}
struct virtio_crypto_op_header {
#define VIRTIO_CRYPTO_CIPHER_ENCRYPT \
    VIRTIO_CRYPTO_OPCODE(VIRTIO_CRYPTO_SERVICE_CIPHER, 0x00)
#define VIRTIO_CRYPTO_CIPHER_DECRYPT \
    VIRTIO_CRYPTO_OPCODE(VIRTIO_CRYPTO_SERVICE_CIPHER, 0x01)
#define VIRTIO_CRYPTO_HASH \
    VIRTIO_CRYPTO_OPCODE(VIRTIO_CRYPTO_SERVICE_HASH, 0x00)
#define VIRTIO_CRYPTO_MAC \
    VIRTIO_CRYPTO_OPCODE(VIRTIO_CRYPTO_SERVICE_MAC, 0x00)
#define VIRTIO_CRYPTO_AEAD_ENCRYPT \
    VIRTIO_CRYPTO_OPCODE(VIRTIO_CRYPTO_SERVICE_AEAD, 0x00)
#define VIRTIO_CRYPTO_AEAD_DECRYPT \
    VIRTIO_CRYPTO_OPCODE(VIRTIO_CRYPTO_SERVICE_AEAD, 0x01)
#define VIRTIO_CRYPTO_AKCIPHER_ENCRYPT \
    VIRTIO_CRYPTO_OPCODE(VIRTIO_CRYPTO_SERVICE_AKCIPHER, 0x00)
#define VIRTIO_CRYPTO_AKCIPHER_DECRYPT \
    VIRTIO_CRYPTO_OPCODE(VIRTIO_CRYPTO_SERVICE_AKCIPHER, 0x01)
#define VIRTIO_CRYPTO_AKCIPHER_SIGN \
    VIRTIO_CRYPTO_OPCODE(VIRTIO_CRYPTO_SERVICE_AKCIPHER, 0x02)
#define VIRTIO_CRYPTO_AKCIPHER_VERIFY \
    VIRTIO_CRYPTO_OPCODE(VIRTIO_CRYPTO_SERVICE_AKCIPHER, 0x03)
    le32 opcode;
    /* algo should be service-specific algorithms */
    le32 algo;
    le64 session_id;
#define VIRTIO_CRYPTO_FLAG_SESSION_MODE 1
    /* control flag to control the request */
    le32 flag;
    le32 padding;
};
\end{lstlisting}

\begin{note}
If VIRTIO_CRYPTO_F_REVISION_1 is not negotiated the \field{flag} is ignored.

If VIRTIO_CRYPTO_F_REVISION_1 is negotiated but VIRTIO_CRYPTO_F_<SERVICE>_STATELESS_MODE
is not negotiated, then the device SHOULD reject <SERVICE> requests if
VIRTIO_CRYPTO_FLAG_SESSION_MODE is not set (in \field{flag}).
\end{note}

The dataq request is composed of four parts:
\begin{lstlisting}
struct virtio_crypto_op_data_req {
    /* Device read only portion */

    struct virtio_crypto_op_header header;

#define VIRTIO_CRYPTO_DATAQ_OP_SPEC_HDR_LEGACY 48
    /* fixed length fields, opcode specific */
    u8 op_flf[flf_len];

    /* Device read && write portion */
    /* variable length fields, opcode specific */
    u8 op_vlf[vlf_len];

    /* Device write only portion */
    struct virtio_crypto_inhdr inhdr;
};
\end{lstlisting}

\field{header} is a general header (see above).

\field{op_flf} is the opcode (in \field{header}) specific header.

\field{flf_len} depends on the VIRTIO_CRYPTO_F_REVISION_1 feature bit
(see below).

\field{op_vlf} is the opcode (in \field{header}) specific parameters.

\field{vlf_len} is the size of the specific structure used.

\begin{itemize*}
\item If the the opcode (in \field{header}) is VIRTIO_CRYPTO_CIPHER_ENCRYPT
    or VIRTIO_CRYPTO_CIPHER_DECRYPT then:
    \begin{itemize*}
    \item If VIRTIO_CRYPTO_F_CIPHER_STATELESS_MODE is negotiated, \field{op_flf} is
        struct virtio_crypto_sym_data_flf_stateless, and \field{op_vlf} is struct
        virtio_crypto_sym_data_vlf_stateless.
    \item If VIRTIO_CRYPTO_F_CIPHER_STATELESS_MODE is NOT negotiated, \field{op_flf}
        is struct virtio_crypto_sym_data_flf if VIRTIO_CRYPTO_F_REVISION_1 is negotiated
        and struct virtio_crypto_sym_data_flf is padded to 48 bytes if NOT negotiated,
        and \field{op_vlf} is struct virtio_crypto_sym_data_vlf.
    \end{itemize*}
\item If the the opcode (in \field{header}) is VIRTIO_CRYPTO_HASH:
    \begin{itemize*}
    \item If VIRTIO_CRYPTO_F_HASH_STATELESS_MODE is negotiated, \field{op_flf} is
        struct virtio_crypto_hash_data_flf_stateless, and \field{op_vlf} is struct
        virtio_crypto_hash_data_vlf_stateless.
    \item If VIRTIO_CRYPTO_F_HASH_STATELESS_MODE is NOT negotiated, \field{op_flf}
        is struct virtio_crypto_hash_data_flf if VIRTIO_CRYPTO_F_REVISION_1 is negotiated
        and struct virtio_crypto_hash_data_flf is padded to 48 bytes if NOT negotiated,
        and \field{op_vlf} is struct virtio_crypto_hash_data_vlf.
    \end{itemize*}
\item If the the opcode (in \field{header}) is VIRTIO_CRYPTO_MAC:
    \begin{itemize*}
    \item If VIRTIO_CRYPTO_F_MAC_STATELESS_MODE is negotiated, \field{op_flf} is
        struct virtio_crypto_mac_data_flf_stateless, and \field{op_vlf} is struct
        virtio_crypto_mac_data_vlf_stateless.
    \item If VIRTIO_CRYPTO_F_MAC_STATELESS_MODE is NOT negotiated, \field{op_flf}
        is struct virtio_crypto_mac_data_flf if VIRTIO_CRYPTO_F_REVISION_1 is negotiated
        and struct virtio_crypto_mac_data_flf is padded to 48 bytes if NOT negotiated,
        and \field{op_vlf} is struct virtio_crypto_mac_data_vlf.
    \end{itemize*}
\item If the the opcode (in \field{header}) is VIRTIO_CRYPTO_AEAD_ENCRYPT
    or VIRTIO_CRYPTO_AEAD_DECRYPT then:
    \begin{itemize*}
    \item If VIRTIO_CRYPTO_F_AEAD_STATELESS_MODE is negotiated, \field{op_flf} is
        struct virtio_crypto_aead_data_flf_stateless, and \field{op_vlf} is struct
        virtio_crypto_aead_data_vlf_stateless.
    \item If VIRTIO_CRYPTO_F_AEAD_STATELESS_MODE is NOT negotiated, \field{op_flf}
        is struct virtio_crypto_aead_data_flf if VIRTIO_CRYPTO_F_REVISION_1 is negotiated
        and struct virtio_crypto_aead_data_flf is padded to 48 bytes if NOT negotiated,
        and \field{op_vlf} is struct virtio_crypto_aead_data_vlf.
    \end{itemize*}
\item If the opcode (in \field{header}) is VIRTIO_CRYPTO_AKCIPHER_ENCRYPT, VIRTIO_CRYPTO_AKCIPHER_DECRYPT,
    VIRTIO_CRYPTO_AKCIPHER_SIGN or VIRTIO_CRYPTO_AKCIPHER_VERIFY then:
    \begin{itemize*}
    \item If VIRTIO_CRYPTO_F_AKCIPHER_STATELESS_MODE is negotiated, \field{op_flf} is
        struct virtio_crypto_akcipher_data_flf_statless, and \field{op_vlf} is struct
        virtio_crypto_akcipher_data_vlf_stateless.
    \item If VIRTIO_CRYPTO_F_AKCIPHER_STATELESS_MODE is NOT negotiated, \field{op_flf}
        is struct virtio_crypto_akcipher_data_flf if VIRTIO_CRYPTO_F_REVISION_1 is negotiated
        and struct virtio_crypto_akcipher_data_flf is padded to 48 bytes if NOT negotiated,
        and \field{op_vlf} is struct virtio_crypto_akcipher_data_vlf.
    \end{itemize*}
\end{itemize*}

\field{inhdr} is a unified input header that used to return the status of
the operations, is defined as follows:

\begin{lstlisting}
struct virtio_crypto_inhdr {
    u8 status;
};
\end{lstlisting}

\subsubsection{HASH Service Operation}\label{sec:Device Types / Crypto Device / Device Operation / HASH Service Operation}

Session mode HASH service requests are as follows:

\begin{lstlisting}
struct virtio_crypto_hash_data_flf {
    /* length of source data */
    le32 src_data_len;
    /* hash result length */
    le32 hash_result_len;
};

struct virtio_crypto_hash_data_vlf {
    /* Device read only portion */
    /* Source data */
    u8 src_data[src_data_len];

    /* Device write only portion */
    /* Hash result data */
    u8 hash_result[hash_result_len];
};
\end{lstlisting}

Each data request uses the virtio_crypto_hash_data_flf structure and the
virtio_crypto_hash_data_vlf structure to store information used to run the
HASH operations.

\field{src_data} is the source data that will be processed.
\field{src_data_len} is the length of source data.
\field{hash_result} is the result data and \field{hash_result_len} is the length
of it.

Stateless mode HASH service requests are as follows:

\begin{lstlisting}
struct virtio_crypto_hash_data_flf_stateless {
    struct {
        /* See VIRTIO_CRYPTO_HASH_* above */
        le32 algo;
    } sess_para;

    /* length of source data */
    le32 src_data_len;
    /* hash result length */
    le32 hash_result_len;
    le32 reserved;
};
struct virtio_crypto_hash_data_vlf_stateless {
    /* Device read only portion */
    /* Source data */
    u8 src_data[src_data_len];

    /* Device write only portion */
    /* Hash result data */
    u8 hash_result[hash_result_len];
};
\end{lstlisting}

\drivernormative{\paragraph}{HASH Service Operation}{Device Types / Crypto Device / Device Operation / HASH Service Operation}

\begin{itemize*}
\item If the driver uses the session mode, then the driver MUST set \field{session_id}
    in struct virtio_crypto_op_header to a valid value assigned by the device when the
    session was created.
\item If the VIRTIO_CRYPTO_F_HASH_STATELESS_MODE feature bit is negotiated, 1) if the
    driver uses the stateless mode, then the driver MUST set the \field{flag} field in
    struct virtio_crypto_op_header to ZERO and MUST set the fields in struct
    virtio_crypto_hash_data_flf_stateless.sess_para, 2) if the driver uses the session
    mode, then the driver MUST set the \field{flag} field in struct virtio_crypto_op_header
    to VIRTIO_CRYPTO_FLAG_SESSION_MODE.
\item The driver MUST set \field{opcode} in struct virtio_crypto_op_header to VIRTIO_CRYPTO_HASH.
\end{itemize*}

\devicenormative{\paragraph}{HASH Service Operation}{Device Types / Crypto Device / Device Operation / HASH Service Operation}

\begin{itemize*}
\item The device MUST use the corresponding structure according to the \field{opcode}
    in the data general header.
\item If the VIRTIO_CRYPTO_F_HASH_STATELESS_MODE feature bit is negotiated, the device
    MUST parse \field{flag} field in struct virtio_crypto_op_header in order to decide
    which mode the driver uses.
\item The device MUST copy the results of HASH operations in the hash_result[] if HASH
    operations success.
\item The device MUST set \field{status} in struct virtio_crypto_inhdr to one of the
    following values of enum VIRTIO_CRYPTO_STATUS:
\begin{itemize*}
\item VIRTIO_CRYPTO_OK if the operation success.
\item VIRTIO_CRYPTO_NOTSUPP if the requested algorithm or operation is unsupported.
\item VIRTIO_CRYPTO_INVSESS if the session ID invalid when in session mode.
\item VIRTIO_CRYPTO_ERR if any failure not mentioned above occurs.
\end{itemize*}
\end{itemize*}


\subsubsection{MAC Service Operation}\label{sec:Device Types / Crypto Device / Device Operation / MAC Service Operation}

Session mode MAC service requests are as follows:

\begin{lstlisting}
struct virtio_crypto_mac_data_flf {
    struct virtio_crypto_hash_data_flf hdr;
};

struct virtio_crypto_mac_data_vlf {
    /* Device read only portion */
    /* Source data */
    u8 src_data[src_data_len];

    /* Device write only portion */
    /* Hash result data */
    u8 hash_result[hash_result_len];
};
\end{lstlisting}

Each request uses the virtio_crypto_mac_data_flf structure and the
virtio_crypto_mac_data_vlf structure to store information used to run the
MAC operations.

\field{src_data} is the source data that will be processed.
\field{src_data_len} is the length of source data.
\field{hash_result} is the result data and \field{hash_result_len} is the length
of it.

Stateless mode MAC service requests are as follows:

\begin{lstlisting}
struct virtio_crypto_mac_data_flf_stateless {
    struct {
        /* See VIRTIO_CRYPTO_MAC_* above */
        le32 algo;
        /* length of authenticated key */
        le32 auth_key_len;
    } sess_para;

    /* length of source data */
    le32 src_data_len;
    /* hash result length */
    le32 hash_result_len;
};

struct virtio_crypto_mac_data_vlf_stateless {
    /* Device read only portion */
    /* The authenticated key */
    u8 auth_key[auth_key_len];
    /* Source data */
    u8 src_data[src_data_len];

    /* Device write only portion */
    /* Hash result data */
    u8 hash_result[hash_result_len];
};
\end{lstlisting}

\field{auth_key} is the authenticated key that will be used during the process.
\field{auth_key_len} is the length of the key.

\drivernormative{\paragraph}{MAC Service Operation}{Device Types / Crypto Device / Device Operation / MAC Service Operation}

\begin{itemize*}
\item If the driver uses the session mode, then the driver MUST set \field{session_id}
    in struct virtio_crypto_op_header to a valid value assigned by the device when the
    session was created.
\item If the VIRTIO_CRYPTO_F_MAC_STATELESS_MODE feature bit is negotiated, 1) if the
    driver uses the stateless mode, then the driver MUST set the \field{flag} field
    in struct virtio_crypto_op_header to ZERO and MUST set the fields in struct
    virtio_crypto_mac_data_flf_stateless.sess_para, 2) if the driver uses the session
    mode, then the driver MUST set the \field{flag} field in struct virtio_crypto_op_header
    to VIRTIO_CRYPTO_FLAG_SESSION_MODE.
\item The driver MUST set \field{opcode} in struct virtio_crypto_op_header to VIRTIO_CRYPTO_MAC.
\end{itemize*}

\devicenormative{\paragraph}{MAC Service Operation}{Device Types / Crypto Device / Device Operation / MAC Service Operation}

\begin{itemize*}
\item If the VIRTIO_CRYPTO_F_MAC_STATELESS_MODE feature bit is negotiated, the device
    MUST parse \field{flag} field in struct virtio_crypto_op_header in order to decide
	which mode the driver uses.
\item The device MUST copy the results of MAC operations in the hash_result[] if HASH
    operations success.
\item The device MUST set \field{status} in struct virtio_crypto_inhdr to one of the
    following values of enum VIRTIO_CRYPTO_STATUS:
\begin{itemize*}
\item VIRTIO_CRYPTO_OK if the operation success.
\item VIRTIO_CRYPTO_NOTSUPP if the requested algorithm or operation is unsupported.
\item VIRTIO_CRYPTO_INVSESS if the session ID invalid when in session mode.
\item VIRTIO_CRYPTO_ERR if any failure not mentioned above occurs.
\end{itemize*}
\end{itemize*}

\subsubsection{Symmetric algorithms Operation}\label{sec:Device Types / Crypto Device / Device Operation / Symmetric algorithms Operation}

Session mode CIPHER service requests are as follows:

\begin{lstlisting}
struct virtio_crypto_cipher_data_flf {
    /*
     * Byte Length of valid IV/Counter data pointed to by the below iv data.
     *
     * For block ciphers in CBC or F8 mode, or for Kasumi in F8 mode, or for
     *   SNOW3G in UEA2 mode, this is the length of the IV (which
     *   must be the same as the block length of the cipher).
     * For block ciphers in CTR mode, this is the length of the counter
     *   (which must be the same as the block length of the cipher).
     */
    le32 iv_len;
    /* length of source data */
    le32 src_data_len;
    /* length of destination data */
    le32 dst_data_len;
    le32 padding;
};

struct virtio_crypto_cipher_data_vlf {
    /* Device read only portion */

    /*
     * Initialization Vector or Counter data.
     *
     * For block ciphers in CBC or F8 mode, or for Kasumi in F8 mode, or for
     *   SNOW3G in UEA2 mode, this is the Initialization Vector (IV)
     *   value.
     * For block ciphers in CTR mode, this is the counter.
     * For AES-XTS, this is the 128bit tweak, i, from IEEE Std 1619-2007.
     *
     * The IV/Counter will be updated after every partial cryptographic
     * operation.
     */
    u8 iv[iv_len];
    /* Source data */
    u8 src_data[src_data_len];

    /* Device write only portion */
    /* Destination data */
    u8 dst_data[dst_data_len];
};
\end{lstlisting}

Session mode requests of algorithm chaining are as follows:

\begin{lstlisting}
struct virtio_crypto_alg_chain_data_flf {
    le32 iv_len;
    /* Length of source data */
    le32 src_data_len;
    /* Length of destination data */
    le32 dst_data_len;
    /* Starting point for cipher processing in source data */
    le32 cipher_start_src_offset;
    /* Length of the source data that the cipher will be computed on */
    le32 len_to_cipher;
    /* Starting point for hash processing in source data */
    le32 hash_start_src_offset;
    /* Length of the source data that the hash will be computed on */
    le32 len_to_hash;
    /* Length of the additional auth data */
    le32 aad_len;
    /* Length of the hash result */
    le32 hash_result_len;
    le32 reserved;
};

struct virtio_crypto_alg_chain_data_vlf {
    /* Device read only portion */

    /* Initialization Vector or Counter data */
    u8 iv[iv_len];
    /* Source data */
    u8 src_data[src_data_len];
    /* Additional authenticated data if exists */
    u8 aad[aad_len];

    /* Device write only portion */

    /* Destination data */
    u8 dst_data[dst_data_len];
    /* Hash result data */
    u8 hash_result[hash_result_len];
};
\end{lstlisting}

Session mode requests of symmetric algorithm are as follows:

\begin{lstlisting}
struct virtio_crypto_sym_data_flf {
    /* Device read only portion */

#define VIRTIO_CRYPTO_SYM_DATA_REQ_HDR_SIZE    40
    u8 op_type_flf[VIRTIO_CRYPTO_SYM_DATA_REQ_HDR_SIZE];

    /* See above VIRTIO_CRYPTO_SYM_OP_* */
    le32 op_type;
    le32 padding;
};

struct virtio_crypto_sym_data_vlf {
    u8 op_type_vlf[sym_para_len];
};
\end{lstlisting}

Each request uses the virtio_crypto_sym_data_flf structure and the
virtio_crypto_sym_data_flf structure to store information used to run the
CIPHER operations.

\field{op_type_flf} is the \field{op_type} specific header, it MUST starts
with or be one of the following structures:
\begin{itemize*}
\item struct virtio_crypto_cipher_data_flf
\item struct virtio_crypto_alg_chain_data_flf
\end{itemize*}

The length of \field{op_type_flf} is fixed to 40 bytes, the data of unused
part (if has) will be ignored.

\field{op_type_vlf} is the \field{op_type} specific parameters, it MUST starts
with or be one of the following structures:
\begin{itemize*}
\item struct virtio_crypto_cipher_data_vlf
\item struct virtio_crypto_alg_chain_data_vlf
\end{itemize*}

\field{sym_para_len} is the size of the specific structure used.

Stateless mode CIPHER service requests are as follows:

\begin{lstlisting}
struct virtio_crypto_cipher_data_flf_stateless {
    struct {
        /* See VIRTIO_CRYPTO_CIPHER* above */
        le32 algo;
        /* length of key */
        le32 key_len;

        /* See VIRTIO_CRYPTO_OP_* above */
        le32 op;
    } sess_para;

    /*
     * Byte Length of valid IV/Counter data pointed to by the below iv data.
     */
    le32 iv_len;
    /* length of source data */
    le32 src_data_len;
    /* length of destination data */
    le32 dst_data_len;
};

struct virtio_crypto_cipher_data_vlf_stateless {
    /* Device read only portion */

    /* The cipher key */
    u8 cipher_key[key_len];

    /* Initialization Vector or Counter data. */
    u8 iv[iv_len];
    /* Source data */
    u8 src_data[src_data_len];

    /* Device write only portion */
    /* Destination data */
    u8 dst_data[dst_data_len];
};
\end{lstlisting}

Stateless mode requests of algorithm chaining are as follows:

\begin{lstlisting}
struct virtio_crypto_alg_chain_data_flf_stateless {
    struct {
        /* See VIRTIO_CRYPTO_SYM_ALG_CHAIN_ORDER_* above */
        le32 alg_chain_order;
        /* length of the additional authenticated data in bytes */
        le32 aad_len;

        struct {
            /* See VIRTIO_CRYPTO_CIPHER* above */
            le32 algo;
            /* length of key */
            le32 key_len;
            /* See VIRTIO_CRYPTO_OP_* above */
            le32 op;
        } cipher;

        struct {
            /* See VIRTIO_CRYPTO_HASH_* or VIRTIO_CRYPTO_MAC_* above */
            le32 algo;
            /* length of authenticated key */
            le32 auth_key_len;
            /* See VIRTIO_CRYPTO_SYM_HASH_MODE_* above */
            le32 hash_mode;
        } hash;
    } sess_para;

    le32 iv_len;
    /* Length of source data */
    le32 src_data_len;
    /* Length of destination data */
    le32 dst_data_len;
    /* Starting point for cipher processing in source data */
    le32 cipher_start_src_offset;
    /* Length of the source data that the cipher will be computed on */
    le32 len_to_cipher;
    /* Starting point for hash processing in source data */
    le32 hash_start_src_offset;
    /* Length of the source data that the hash will be computed on */
    le32 len_to_hash;
    /* Length of the additional auth data */
    le32 aad_len;
    /* Length of the hash result */
    le32 hash_result_len;
    le32 reserved;
};

struct virtio_crypto_alg_chain_data_vlf_stateless {
    /* Device read only portion */

    /* The cipher key */
    u8 cipher_key[key_len];
    /* The auth key */
    u8 auth_key[auth_key_len];
    /* Initialization Vector or Counter data */
    u8 iv[iv_len];
    /* Additional authenticated data if exists */
    u8 aad[aad_len];
    /* Source data */
    u8 src_data[src_data_len];

    /* Device write only portion */

    /* Destination data */
    u8 dst_data[dst_data_len];
    /* Hash result data */
    u8 hash_result[hash_result_len];
};
\end{lstlisting}

Stateless mode requests of symmetric algorithm are as follows:

\begin{lstlisting}
struct virtio_crypto_sym_data_flf_stateless {
    /* Device read only portion */
#define VIRTIO_CRYPTO_SYM_DATE_REQ_HDR_STATELESS_SIZE    72
    u8 op_type_flf[VIRTIO_CRYPTO_SYM_DATE_REQ_HDR_STATELESS_SIZE];

    /* Device write only portion */
    /* See above VIRTIO_CRYPTO_SYM_OP_* */
    le32 op_type;
};

struct virtio_crypto_sym_data_vlf_stateless {
    u8 op_type_vlf[sym_para_len];
};
\end{lstlisting}

\field{op_type_flf} is the \field{op_type} specific header, it MUST starts
with or be one of the following structures:
\begin{itemize*}
\item struct virtio_crypto_cipher_data_flf_stateless
\item struct virtio_crypto_alg_chain_data_flf_stateless
\end{itemize*}

The length of \field{op_type_flf} is fixed to 72 bytes, the data of unused
part (if has) will be ignored.

\field{op_type_vlf} is the \field{op_type} specific parameters, it MUST starts
with or be one of the following structures:
\begin{itemize*}
\item struct virtio_crypto_cipher_data_vlf_stateless
\item struct virtio_crypto_alg_chain_data_vlf_stateless
\end{itemize*}

\field{sym_para_len} is the size of the specific structure used.

\drivernormative{\paragraph}{Symmetric algorithms Operation}{Device Types / Crypto Device / Device Operation / Symmetric algorithms Operation}

\begin{itemize*}
\item If the driver uses the session mode, then the driver MUST set \field{session_id}
    in struct virtio_crypto_op_header to a valid value assigned by the device when the
    session was created.
\item If the VIRTIO_CRYPTO_F_CIPHER_STATELESS_MODE feature bit is negotiated, 1) if the
    driver uses the stateless mode, then the driver MUST set the \field{flag} field in
    struct virtio_crypto_op_header to ZERO and MUST set the fields in struct
    virtio_crypto_cipher_data_flf_stateless.sess_para or struct
    virtio_crypto_alg_chain_data_flf_stateless.sess_para, 2) if the driver uses the
    session mode, then the driver MUST set the \field{flag} field in struct
    virtio_crypto_op_header to VIRTIO_CRYPTO_FLAG_SESSION_MODE.
\item The driver MUST set the \field{opcode} field in struct virtio_crypto_op_header
    to VIRTIO_CRYPTO_CIPHER_ENCRYPT or VIRTIO_CRYPTO_CIPHER_DECRYPT.
\item The driver MUST specify the fields of struct virtio_crypto_cipher_data_flf in
    struct virtio_crypto_sym_data_flf and struct virtio_crypto_cipher_data_vlf in
    struct virtio_crypto_sym_data_vlf if the request is based on VIRTIO_CRYPTO_SYM_OP_CIPHER.
\item The driver MUST specify the fields of struct virtio_crypto_alg_chain_data_flf
    in struct virtio_crypto_sym_data_flf and struct virtio_crypto_alg_chain_data_vlf
    in struct virtio_crypto_sym_data_vlf if the request is of the VIRTIO_CRYPTO_SYM_OP_ALGORITHM_CHAINING
    type.
\end{itemize*}

\devicenormative{\paragraph}{Symmetric algorithms Operation}{Device Types / Crypto Device / Device Operation / Symmetric algorithms Operation}

\begin{itemize*}
\item If the VIRTIO_CRYPTO_F_CIPHER_STATELESS_MODE feature bit is negotiated, the device
    MUST parse \field{flag} field in struct virtio_crypto_op_header in order to decide
	which mode the driver uses.
\item The device MUST parse the virtio_crypto_sym_data_req based on the \field{opcode}
    field in general header.
\item The device MUST parse the fields of struct virtio_crypto_cipher_data_flf in
    struct virtio_crypto_sym_data_flf and struct virtio_crypto_cipher_data_vlf in
    struct virtio_crypto_sym_data_vlf if the request is based on VIRTIO_CRYPTO_SYM_OP_CIPHER.
\item The device MUST parse the fields of struct virtio_crypto_alg_chain_data_flf
    in struct virtio_crypto_sym_data_flf and struct virtio_crypto_alg_chain_data_vlf
    in struct virtio_crypto_sym_data_vlf if the request is of the VIRTIO_CRYPTO_SYM_OP_ALGORITHM_CHAINING
    type.
\item The device MUST copy the result of cryptographic operation in the dst_data[] in
    both plain CIPHER mode and algorithms chain mode.
\item The device MUST check the \field{para}.\field{add_len} is bigger than 0 before
    parse the additional authenticated data in plain algorithms chain mode.
\item The device MUST copy the result of HASH/MAC operation in the hash_result[] is
    of the VIRTIO_CRYPTO_SYM_OP_ALGORITHM_CHAINING type.
\item The device MUST set the \field{status} field in struct virtio_crypto_inhdr to
    one of the following values of enum VIRTIO_CRYPTO_STATUS:
\begin{itemize*}
\item VIRTIO_CRYPTO_OK if the operation success.
\item VIRTIO_CRYPTO_NOTSUPP if the requested algorithm or operation is unsupported.
\item VIRTIO_CRYPTO_INVSESS if the session ID is invalid in session mode.
\item VIRTIO_CRYPTO_ERR if failure not mentioned above occurs.
\end{itemize*}
\end{itemize*}

\subsubsection{AEAD Service Operation}\label{sec:Device Types / Crypto Device / Device Operation / AEAD Service Operation}

Session mode requests of symmetric algorithm are as follows:

\begin{lstlisting}
struct virtio_crypto_aead_data_flf {
    /*
     * Byte Length of valid IV data.
     *
     * For GCM mode, this is either 12 (for 96-bit IVs) or 16, in which
     *   case iv points to J0.
     * For CCM mode, this is the length of the nonce, which can be in the
     *   range 7 to 13 inclusive.
     */
    le32 iv_len;
    /* length of additional auth data */
    le32 aad_len;
    /* length of source data */
    le32 src_data_len;
    /* length of dst data, this should be at least src_data_len + tag_len */
    le32 dst_data_len;
    /* Authentication tag length */
    le32 tag_len;
    le32 reserved;
};

struct virtio_crypto_aead_data_vlf {
    /* Device read only portion */

    /*
     * Initialization Vector data.
     *
     * For GCM mode, this is either the IV (if the length is 96 bits) or J0
     *   (for other sizes), where J0 is as defined by NIST SP800-38D.
     *   Regardless of the IV length, a full 16 bytes needs to be allocated.
     * For CCM mode, the first byte is reserved, and the nonce should be
     *   written starting at &iv[1] (to allow space for the implementation
     *   to write in the flags in the first byte).  Note that a full 16 bytes
     *   should be allocated, even though the iv_len field will have
     *   a value less than this.
     *
     * The IV will be updated after every partial cryptographic operation.
     */
    u8 iv[iv_len];
    /* Source data */
    u8 src_data[src_data_len];
    /* Additional authenticated data if exists */
    u8 aad[aad_len];

    /* Device write only portion */
    /* Pointer to output data */
    u8 dst_data[dst_data_len];
};
\end{lstlisting}

Each request uses the virtio_crypto_aead_data_flf structure and the
virtio_crypto_aead_data_flf structure to store information used to run the
AEAD operations.

Stateless mode AEAD service requests are as follows:

\begin{lstlisting}
struct virtio_crypto_aead_data_flf_stateless {
    struct {
        /* See VIRTIO_CRYPTO_AEAD_* above */
        le32 algo;
        /* length of key */
        le32 key_len;
        /* encrypt or decrypt, See above VIRTIO_CRYPTO_OP_* */
        le32 op;
    } sess_para;

    /* Byte Length of valid IV data. */
    le32 iv_len;
    /* Authentication tag length */
    le32 tag_len;
    /* length of additional auth data */
    le32 aad_len;
    /* length of source data */
    le32 src_data_len;
    /* length of dst data, this should be at least src_data_len + tag_len */
    le32 dst_data_len;
};

struct virtio_crypto_aead_data_vlf_stateless {
    /* Device read only portion */

    /* The cipher key */
    u8 key[key_len];
    /* Initialization Vector data. */
    u8 iv[iv_len];
    /* Source data */
    u8 src_data[src_data_len];
    /* Additional authenticated data if exists */
    u8 aad[aad_len];

    /* Device write only portion */
    /* Pointer to output data */
    u8 dst_data[dst_data_len];
};
\end{lstlisting}

\drivernormative{\paragraph}{AEAD Service Operation}{Device Types / Crypto Device / Device Operation / AEAD Service Operation}

\begin{itemize*}
\item If the driver uses the session mode, then the driver MUST set
    \field{session_id} in struct virtio_crypto_op_header to a valid value assigned
    by the device when the session was created.
\item If the VIRTIO_CRYPTO_F_AEAD_STATELESS_MODE feature bit is negotiated, 1) if
    the driver uses the stateless mode, then the driver MUST set the \field{flag}
    field in struct virtio_crypto_op_header to ZERO and MUST set the fields in
    struct virtio_crypto_aead_data_flf_stateless.sess_para, 2) if the driver uses
    the session mode, then the driver MUST set the \field{flag} field in struct
    virtio_crypto_op_header to VIRTIO_CRYPTO_FLAG_SESSION_MODE.
\item The driver MUST set the \field{opcode} field in struct virtio_crypto_op_header
    to VIRTIO_CRYPTO_AEAD_ENCRYPT or VIRTIO_CRYPTO_AEAD_DECRYPT.
\end{itemize*}

\devicenormative{\paragraph}{AEAD Service Operation}{Device Types / Crypto Device / Device Operation / AEAD Service Operation}

\begin{itemize*}
\item If the VIRTIO_CRYPTO_F_AEAD_STATELESS_MODE feature bit is negotiated, the
    device MUST parse the virtio_crypto_aead_data_vlf_stateless based on the \field{opcode}
	field in general header.
\item The device MUST copy the result of cryptographic operation in the dst_data[].
\item The device MUST copy the authentication tag in the dst_data[] offset the cipher result.
\item The device MUST set the \field{status} field in struct virtio_crypto_inhdr to
    one of the following values of enum VIRTIO_CRYPTO_STATUS:
\item When the \field{opcode} field is VIRTIO_CRYPTO_AEAD_DECRYPT, the device MUST
    verify and return the verification result to the driver.
\begin{itemize*}
\item VIRTIO_CRYPTO_OK if the operation success.
\item VIRTIO_CRYPTO_NOTSUPP if the requested algorithm or operation is unsupported.
\item VIRTIO_CRYPTO_BADMSG if the verification result is incorrect.
\item VIRTIO_CRYPTO_INVSESS if the session ID invalid when in session mode.
\item VIRTIO_CRYPTO_ERR if any failure not mentioned above occurs.
\end{itemize*}
\end{itemize*}

\subsubsection{AKCIPHER Service Operation}\label{sec:Device Types / Crypto Device / Device Operation / AKCIPHER Service Operation}

Session mode AKCIPHER requests are as follows:

\begin{lstlisting}
struct virtio_crypto_akcipher_data_flf {
    /* length of source data */
    le32 src_data_len;
    /* length of dst data */
    le32 dst_data_len;
};

struct virtio_crypto_akcipher_data_vlf {
    /* Device read only portion */
    /* Source data */
    u8 src_data[src_data_len];

    /* Device write only portion */
    /* Pointer to output data */
    u8 dst_data[dst_data_len];
};
\end{lstlisting}

Each data request uses the virtio_crypto_akcipher_flf structure and the virtio_crypto_akcipher_data_vlf
structure to store information used to run the AKCIPHER operations.

For encryption, decryption, and signing:
\field{src_data} is the source data that will be processed, note that for signing operations,
src_data stores the data to be signed, which usually is the digest of some data rather than the
data itself.
\field{src_data_len} is the length of source data.
\field{dst_result} is the result data and \field{dst_data_len} is the length of it. Note that the
length of the result is not always exactly equal to dst_data_len, the driver needs to check how
many bytes the device has written and calculate the actual length of the result.

For verification:
\field{src_data_len} refers to the length of the signature, and \field{dst_data_len} refers to
the length of signed data, where the signed data is usually the digest of some data.
\field{src_data} is spliced by the signature and the signed data, the src_data with the lower
address stores the signature, and the higher address stores the signed data.
\field{dst_data} is always empty for verification.

Different algorithms have different signature formats.
For the RSA algorithm, the result is determined by the padding algorithm specified by
\field{padding_algo} in structure virtio_crypto_rsa_session_para.

For the ECDSA algorithm, the signature is composed of the following
ASN.1 structure (see \hyperref[intro:rfc3279]{RFC3279})
and MUST be DER encoded (see \hyperref[intro:rfc6025]{rfc6025}).

\begin{lstlisting}
Ecdsa-Sig-Value ::= SEQUENCE {
    r INTEGER,
    s INTEGER
}
\end{lstlisting}

Stateless mode AKCIPHER service requests are as follows:
\begin{lstlisting}
struct virtio_crypto_akcipher_data_flf_stateless {
    struct {
        /* See VIRTIO_CYRPTO_AKCIPHER* above */
        le32 algo;
        /* See VIRTIO_CRYPTO_AKCIPHER_KEY_TYPE_* above */
        le32 key_type;
        /* length of key */
        le32 key_len;

        /* algothrim specific parameters described above */
        union {
            struct virtio_crypto_rsa_session_para rsa;
            struct virtio_crypto_ecdsa_session_para ecdsa;
        } u;
    } sess_para;

    /* length of source data */
    le32 src_data_len;
    /* length of destination data */
    le32 dst_data_len;
};

struct virtio_crypto_akcipher_data_vlf_stateless {
    /* Device read only portion */
    u8 akcipher_key[key_len];

    /* Source data */
    u8 src_data[src_data_len];

    /* Device write only portion */
    u8 dst_data[dst_data_len];
};
\end{lstlisting}

In stateless mode, the format of key and signature, the meaning of src_data and dst_data, are all the same
with session mode.

\drivernormative{\paragraph}{AKCIPHER Service Operation}{Device Types / Crypto Device / Device Operation / AKCIPHER Service Operation}

\begin{itemize*}
\item If the driver uses the session mode, then the driver MUST set
    \field{session_id} in struct virtio_crypto_op_header to a valid
    value assigned by the device when the session was created.
\item If the VIRTIO_CRYPTO_F_AKCIPHER_STATELESS_MODE feature bit is negotiated, 1) if the
    driver uses the stateless mode, then the driver MUST set the \field{flag} field in
    struct virtio_crypto_op_header to ZERO and MUST set the fields in struct
    virtio_crypto_akcipher_flf_stateless.sess_para, 2) if the driver uses the session
    mode, then the driver MUST set the \field{flag} field in struct virtio_crypto_op_header
    to VIRTIO_CRYPTO_FLAG_SESSION_MODE.
\item The driver MUST set the \field{opcode} field in struct virtio_crypto_op_header
    to one of VIRTIO_CRYPTO_AKCIPHER_ENCRYPT, VIRTIO_CRYPTO_AKCIPHER_DESTROY_SESSION,
    VIRTIO_CRYPTO_AKCIPHER_SIGN, and VIRTIO_CRYPTO_AKCIPHER_VERIFY.
\end{itemize*}

\devicenormative{\paragraph}{AKCIPHER Service Operation}{Device Types / Crypto Device / Device Operation / AKCIPHER Service Operation}

\begin{itemize*}
\item If the VIRTIO_CRYPTO_F_AKCIPHER_STATELESS_MODE feature bit is negotiated, the
    device MUST parse the virtio_crypto_akcipher_data_vlf_stateless based on the \field{opcode}
    field in general header.
\item The device MUST copy the result of cryptographic operation in the dst_data[].
\item The device MUST set the \field{status} field in struct virtio_crypto_inhdr to
    one of the following values of enum VIRTIO_CRYPTO_STATUS:
\begin{itemize*}
\item VIRTIO_CRYPTO_OK if the operation success.
\item VIRTIO_CRYPTO_NOTSUPP if the requested algorithm or operation is unsupported.
\item VIRTIO_CRYPTO_BADMSG if the verification result is incorrect.
\item VIRTIO_CRYPTO_INVSESS if the session ID invalid when in session mode.
\item VIRTIO_CRYPTO_KEY_REJECTED if the signature verification failed.
\item VIRTIO_CRYPTO_ERR if any failure not mentioned above occurs.
\end{itemize*}
\end{itemize*}

\section{Crypto Device}\label{sec:Device Types / Crypto Device}

The virtio crypto device is a virtual cryptography device as well as a
virtual cryptographic accelerator. The virtio crypto device provides the
following crypto services: CIPHER, MAC, HASH, AEAD and AKCIPHER. Virtio crypto
devices have a single control queue and at least one data queue. Crypto
operation requests are placed into a data queue, and serviced by the
device. Some crypto operation requests are only valid in the context of a
session. The role of the control queue is facilitating control operation
requests. Sessions management is realized with control operation
requests.

\subsection{Device ID}\label{sec:Device Types / Crypto Device / Device ID}

20

\subsection{Virtqueues}\label{sec:Device Types / Crypto Device / Virtqueues}

\begin{description}
\item[0] dataq1
\item[\ldots]
\item[N-1] dataqN
\item[N] controlq
\end{description}

N is set by \field{max_dataqueues}.

\subsection{Feature bits}\label{sec:Device Types / Crypto Device / Feature bits}

\begin{description}
\item VIRTIO_CRYPTO_F_REVISION_1 (0) revision 1. Revision 1 has a specific
    request format and other enhancements (which result in some additional
    requirements).
\item VIRTIO_CRYPTO_F_CIPHER_STATELESS_MODE (1) stateless mode requests are
    supported by the CIPHER service.
\item VIRTIO_CRYPTO_F_HASH_STATELESS_MODE (2) stateless mode requests are
    supported by the HASH service.
\item VIRTIO_CRYPTO_F_MAC_STATELESS_MODE (3) stateless mode requests are
    supported by the MAC service.
\item VIRTIO_CRYPTO_F_AEAD_STATELESS_MODE (4) stateless mode requests are
    supported by the AEAD service.
\item VIRTIO_CRYPTO_F_AKCIPHER_STATELESS_MODE (5) stateless mode requests are
    supported by the AKCIPHER service.
\end{description}


\subsubsection{Feature bit requirements}\label{sec:Device Types / Crypto Device / Feature bit requirements}

Some crypto feature bits require other crypto feature bits
(see \ref{drivernormative:Basic Facilities of a Virtio Device / Feature Bits}):

\begin{description}
\item[VIRTIO_CRYPTO_F_CIPHER_STATELESS_MODE] Requires VIRTIO_CRYPTO_F_REVISION_1.
\item[VIRTIO_CRYPTO_F_HASH_STATELESS_MODE] Requires VIRTIO_CRYPTO_F_REVISION_1.
\item[VIRTIO_CRYPTO_F_MAC_STATELESS_MODE] Requires VIRTIO_CRYPTO_F_REVISION_1.
\item[VIRTIO_CRYPTO_F_AEAD_STATELESS_MODE] Requires VIRTIO_CRYPTO_F_REVISION_1.
\item[VIRTIO_CRYPTO_F_AKCIPHER_STATELESS_MODE] Requires VIRTIO_CRYPTO_F_REVISION_1.
\end{description}

\subsection{Supported crypto services}\label{sec:Device Types / Crypto Device / Supported crypto services}

The following crypto services are defined:

\begin{lstlisting}
/* CIPHER (Symmetric Key Cipher) service */
#define VIRTIO_CRYPTO_SERVICE_CIPHER 0
/* HASH service */
#define VIRTIO_CRYPTO_SERVICE_HASH   1
/* MAC (Message Authentication Codes) service */
#define VIRTIO_CRYPTO_SERVICE_MAC    2
/* AEAD (Authenticated Encryption with Associated Data) service */
#define VIRTIO_CRYPTO_SERVICE_AEAD   3
/* AKCIPHER (Asymmetric Key Cipher) service */
#define VIRTIO_CRYPTO_SERVICE_AKCIPHER 4
\end{lstlisting}

The above constants designate bits used to indicate the which of crypto services are
offered by the device as described in, see \ref{sec:Device Types / Crypto Device / Device configuration layout}.

\subsubsection{CIPHER services}\label{sec:Device Types / Crypto Device / Supported crypto services / CIPHER services}

The following CIPHER algorithms are defined:

\begin{lstlisting}
#define VIRTIO_CRYPTO_NO_CIPHER                 0
#define VIRTIO_CRYPTO_CIPHER_ARC4               1
#define VIRTIO_CRYPTO_CIPHER_AES_ECB            2
#define VIRTIO_CRYPTO_CIPHER_AES_CBC            3
#define VIRTIO_CRYPTO_CIPHER_AES_CTR            4
#define VIRTIO_CRYPTO_CIPHER_DES_ECB            5
#define VIRTIO_CRYPTO_CIPHER_DES_CBC            6
#define VIRTIO_CRYPTO_CIPHER_3DES_ECB           7
#define VIRTIO_CRYPTO_CIPHER_3DES_CBC           8
#define VIRTIO_CRYPTO_CIPHER_3DES_CTR           9
#define VIRTIO_CRYPTO_CIPHER_KASUMI_F8          10
#define VIRTIO_CRYPTO_CIPHER_SNOW3G_UEA2        11
#define VIRTIO_CRYPTO_CIPHER_AES_F8             12
#define VIRTIO_CRYPTO_CIPHER_AES_XTS            13
#define VIRTIO_CRYPTO_CIPHER_ZUC_EEA3           14
\end{lstlisting}

The above constants have two usages:
\begin{enumerate}
\item As bit numbers, used to tell the driver which CIPHER algorithms
are supported by the device, see \ref{sec:Device Types / Crypto Device / Device configuration layout}.
\item As values, used to designate the algorithm in (CIPHER type) crypto
operation requests, see \ref{sec:Device Types / Crypto Device / Device Operation / Control Virtqueue / Session operation}.
\end{enumerate}

\subsubsection{HASH services}\label{sec:Device Types / Crypto Device / Supported crypto services / HASH services}

The following HASH algorithms are defined:

\begin{lstlisting}
#define VIRTIO_CRYPTO_NO_HASH            0
#define VIRTIO_CRYPTO_HASH_MD5           1
#define VIRTIO_CRYPTO_HASH_SHA1          2
#define VIRTIO_CRYPTO_HASH_SHA_224       3
#define VIRTIO_CRYPTO_HASH_SHA_256       4
#define VIRTIO_CRYPTO_HASH_SHA_384       5
#define VIRTIO_CRYPTO_HASH_SHA_512       6
#define VIRTIO_CRYPTO_HASH_SHA3_224      7
#define VIRTIO_CRYPTO_HASH_SHA3_256      8
#define VIRTIO_CRYPTO_HASH_SHA3_384      9
#define VIRTIO_CRYPTO_HASH_SHA3_512      10
#define VIRTIO_CRYPTO_HASH_SHA3_SHAKE128      11
#define VIRTIO_CRYPTO_HASH_SHA3_SHAKE256      12
\end{lstlisting}

The above constants have two usages:
\begin{enumerate}
\item As bit numbers, used to tell the driver which HASH algorithms
are supported by the device, see \ref{sec:Device Types / Crypto Device / Device configuration layout}.
\item As values, used to designate the algorithm in (HASH type) crypto
operation requires, see \ref{sec:Device Types / Crypto Device / Device Operation / Control Virtqueue / Session operation}.
\end{enumerate}

\subsubsection{MAC services}\label{sec:Device Types / Crypto Device / Supported crypto services / MAC services}

The following MAC algorithms are defined:

\begin{lstlisting}
#define VIRTIO_CRYPTO_NO_MAC                       0
#define VIRTIO_CRYPTO_MAC_HMAC_MD5                 1
#define VIRTIO_CRYPTO_MAC_HMAC_SHA1                2
#define VIRTIO_CRYPTO_MAC_HMAC_SHA_224             3
#define VIRTIO_CRYPTO_MAC_HMAC_SHA_256             4
#define VIRTIO_CRYPTO_MAC_HMAC_SHA_384             5
#define VIRTIO_CRYPTO_MAC_HMAC_SHA_512             6
#define VIRTIO_CRYPTO_MAC_CMAC_3DES                25
#define VIRTIO_CRYPTO_MAC_CMAC_AES                 26
#define VIRTIO_CRYPTO_MAC_KASUMI_F9                27
#define VIRTIO_CRYPTO_MAC_SNOW3G_UIA2              28
#define VIRTIO_CRYPTO_MAC_GMAC_AES                 41
#define VIRTIO_CRYPTO_MAC_GMAC_TWOFISH             42
#define VIRTIO_CRYPTO_MAC_CBCMAC_AES               49
#define VIRTIO_CRYPTO_MAC_CBCMAC_KASUMI_F9         50
#define VIRTIO_CRYPTO_MAC_XCBC_AES                 53
#define VIRTIO_CRYPTO_MAC_ZUC_EIA3                 54
\end{lstlisting}

The above constants have two usages:
\begin{enumerate}
\item As bit numbers, used to tell the driver which MAC algorithms
are supported by the device, see \ref{sec:Device Types / Crypto Device / Device configuration layout}.
\item As values, used to designate the algorithm in (MAC type) crypto
operation requests, see \ref{sec:Device Types / Crypto Device / Device Operation / Control Virtqueue / Session operation}.
\end{enumerate}

\subsubsection{AEAD services}\label{sec:Device Types / Crypto Device / Supported crypto services / AEAD services}

The following AEAD algorithms are defined:

\begin{lstlisting}
#define VIRTIO_CRYPTO_NO_AEAD     0
#define VIRTIO_CRYPTO_AEAD_GCM    1
#define VIRTIO_CRYPTO_AEAD_CCM    2
#define VIRTIO_CRYPTO_AEAD_CHACHA20_POLY1305  3
\end{lstlisting}

The above constants have two usages:
\begin{enumerate}
\item As bit numbers, used to tell the driver which AEAD algorithms
are supported by the device, see \ref{sec:Device Types / Crypto Device / Device configuration layout}.
\item As values, used to designate the algorithm in (DEAD type) crypto
operation requests, see \ref{sec:Device Types / Crypto Device / Device Operation / Control Virtqueue / Session operation}.
\end{enumerate}

\subsubsection{AKCIPHER services}\label{sec: Device Types / Crypto Device / Supported crypto services / AKCIPHER services}

The following AKCIPHER algorithms are defined:
\begin{lstlisting}
#define VIRTIO_CRYPTO_NO_AKCIPHER 0
#define VIRTIO_CRYPTO_AKCIPHER_RSA   1
#define VIRTIO_CRYPTO_AKCIPHER_ECDSA 2
\end{lstlisting}

The above constants have two usages:
\begin{enumerate}
\item As bit numbers, used to tell the driver which AKCIPHER algorithms
are supported by the device, see \ref{sec:Device Types / Crypto Device / Device configuration layout}.
\item As values, used to designate the algorithm in asymmetric crypto operation requests,
see \ref{sec:Device Types / Crypto Device / Device Operation / Control Virtqueue / Session operation}.
\end{enumerate}


\subsection{Device configuration layout}\label{sec:Device Types / Crypto Device / Device configuration layout}

Crypto device configuration uses the following layout structure:

\begin{lstlisting}
struct virtio_crypto_config {
    le32 status;
    le32 max_dataqueues;
    le32 crypto_services;
    /* Detailed algorithms mask */
    le32 cipher_algo_l;
    le32 cipher_algo_h;
    le32 hash_algo;
    le32 mac_algo_l;
    le32 mac_algo_h;
    le32 aead_algo;
    /* Maximum length of cipher key in bytes */
    le32 max_cipher_key_len;
    /* Maximum length of authenticated key in bytes */
    le32 max_auth_key_len;
    le32 akcipher_algo;
    /* Maximum size of each crypto request's content in bytes */
    le64 max_size;
};
\end{lstlisting}

\begin{description}
\item Currently, only one \field{status} bit is defined: VIRTIO_CRYPTO_S_HW_READY
    set indicates that the device is ready to process requests, this bit is read-only
    for the driver
\begin{lstlisting}
#define VIRTIO_CRYPTO_S_HW_READY  (1 << 0)
\end{lstlisting}

\item [\field{max_dataqueues}] is the maximum number of data virtqueues that can
    be configured by the device. The driver MAY use only one data queue, or it
    can use more to achieve better performance.

\item [\field{crypto_services}] crypto service offered, see \ref{sec:Device Types / Crypto Device / Supported crypto services}.

\item [\field{cipher_algo_l}] CIPHER algorithms bits 0-31, see \ref{sec:Device Types / Crypto Device / Supported crypto services  / CIPHER services}.

\item [\field{cipher_algo_h}] CIPHER algorithms bits 32-63, see \ref{sec:Device Types / Crypto Device / Supported crypto services  / CIPHER services}.

\item [\field{hash_algo}] HASH algorithms bits, see \ref{sec:Device Types / Crypto Device / Supported crypto services  / HASH services}.

\item [\field{mac_algo_l}] MAC algorithms bits 0-31, see \ref{sec:Device Types / Crypto Device / Supported crypto services  / MAC services}.

\item [\field{mac_algo_h}] MAC algorithms bits 32-63, see \ref{sec:Device Types / Crypto Device / Supported crypto services  / MAC services}.

\item [\field{aead_algo}] AEAD algorithms bits, see \ref{sec:Device Types / Crypto Device / Supported crypto services  / AEAD services}.

\item [\field{max_cipher_key_len}] is the maximum length of cipher key supported by the device.

\item [\field{max_auth_key_len}] is the maximum length of authenticated key supported by the device.

\item [\field{akcipher_algo}] AKCIPHER algorithms bit 0-31, see \ref{sec: Device Types / Crypto Device / Supported crypto services / AKCIPHER services}.

\item [\field{max_size}] is the maximum size of the variable-length parameters of
    data operation of each crypto request's content supported by the device.
\end{description}

\begin{note}
Unless explicitly stated otherwise all lengths and sizes are in bytes.
\end{note}

\devicenormative{\subsubsection}{Device configuration layout}{Device Types / Crypto Device / Device configuration layout}

\begin{itemize*}
\item The device MUST set \field{max_dataqueues} to between 1 and 65535 inclusive.
\item The device MUST set the \field{status} with valid flags, undefined flags MUST NOT be set.
\item The device MUST accept and handle requests after \field{status} is set to VIRTIO_CRYPTO_S_HW_READY.
\item The device MUST set \field{crypto_services} based on the crypto services the device offers.
\item The device MUST set detailed algorithms masks for each service advertised by \field{crypto_services}.
    The device MUST NOT set the not defined algorithms bits.
\item The device MUST set \field{max_size} to show the maximum size of crypto request the device supports.
\item The device MUST set \field{max_cipher_key_len} to show the maximum length of cipher key if the
    device supports CIPHER service.
\item The device MUST set \field{max_auth_key_len} to show the maximum length of authenticated key if
    the device supports MAC service.
\end{itemize*}

\drivernormative{\subsubsection}{Device configuration layout}{Device Types / Crypto Device / Device configuration layout}

\begin{itemize*}
\item The driver MUST read the \field{status} from the bottom bit of status to check whether the
    VIRTIO_CRYPTO_S_HW_READY is set, and the driver MUST reread it after device reset.
\item The driver MUST NOT transmit any requests to the device if the VIRTIO_CRYPTO_S_HW_READY is not set.
\item The driver MUST read \field{max_dataqueues} field to discover the number of data queues the device supports.
\item The driver MUST read \field{crypto_services} field to discover which services the device is able to offer.
\item The driver SHOULD ignore the not defined algorithms bits.
\item The driver MUST read the detailed algorithms fields based on \field{crypto_services} field.
\item The driver SHOULD read \field{max_size} to discover the maximum size of the variable-length
    parameters of data operation of the crypto request's content the device supports and MUST
    guarantee that the size of each crypto request's content is within the \field{max_size}, otherwise
    the request will fail and the driver MUST reset the device.
\item The driver SHOULD read \field{max_cipher_key_len} to discover the maximum length of cipher key
    the device supports and MUST guarantee that the \field{key_len} (CIPHER service or AEAD service) is within
    the \field{max_cipher_key_len} of the device configuration, otherwise the request will fail.
\item The driver SHOULD read \field{max_auth_key_len} to discover the maximum length of authenticated
    key the device supports and MUST guarantee that the \field{auth_key_len} (MAC service) is within the
    \field{max_auth_key_len} of the device configuration, otherwise the request will fail.
\end{itemize*}

\subsection{Device Initialization}\label{sec:Device Types / Crypto Device / Device Initialization}

\drivernormative{\subsubsection}{Device Initialization}{Device Types / Crypto Device / Device Initialization}

\begin{itemize*}
\item The driver MUST configure and initialize all virtqueues.
\item The driver MUST read the supported crypto services from bits of \field{crypto_services}.
\item The driver MUST read the supported algorithms based on \field{crypto_services} field.
\end{itemize*}

\subsection{Device Operation}\label{sec:Device Types / Crypto Device / Device Operation}

The operation of a virtio crypto device is driven by requests placed on the virtqueues.
Requests consist of a queue-type specific header (specifying among others the operation)
and an operation specific payload.

If VIRTIO_CRYPTO_F_REVISION_1 is negotiated the device may support both session mode
(See \ref{sec:Device Types / Crypto Device / Device Operation / Control Virtqueue / Session operation})
and stateless mode operation requests.
In stateless mode all operation parameters are supplied as a part of each request,
while in session mode, some or all operation parameters are managed within the
session. Stateless mode is guarded by feature bits 0-4 on a service level. If
stateless mode is negotiated for a service, the service accepts both session
mode and stateless requests; otherwise stateless mode requests are rejected
(via operation status).

\subsubsection{Operation Status}\label{sec:Device Types / Crypto Device / Device Operation / Operation status}
The device MUST return a status code as part of the operation (both session
operation and service operation) result. The valid operation status as follows:

\begin{lstlisting}
enum VIRTIO_CRYPTO_STATUS {
    VIRTIO_CRYPTO_OK = 0,
    VIRTIO_CRYPTO_ERR = 1,
    VIRTIO_CRYPTO_BADMSG = 2,
    VIRTIO_CRYPTO_NOTSUPP = 3,
    VIRTIO_CRYPTO_INVSESS = 4,
    VIRTIO_CRYPTO_NOSPC = 5,
    VIRTIO_CRYPTO_KEY_REJECTED = 6,
    VIRTIO_CRYPTO_MAX
};
\end{lstlisting}

\begin{itemize*}
\item VIRTIO_CRYPTO_OK: success.
\item VIRTIO_CRYPTO_BADMSG: authentication failed (only when AEAD decryption).
\item VIRTIO_CRYPTO_NOTSUPP: operation or algorithm is unsupported.
\item VIRTIO_CRYPTO_INVSESS: invalid session ID when executing crypto operations.
\item VIRTIO_CRYPTO_NOSPC: no free session ID (only when the VIRTIO_CRYPTO_F_REVISION_1
    feature bit is negotiated).
\item VIRTIO_CRYPTO_KEY_REJECTED: signature verification failed (only when AKCIPHER verification).
\item VIRTIO_CRYPTO_ERR: any failure not mentioned above occurs.
\end{itemize*}

\subsubsection{Control Virtqueue}\label{sec:Device Types / Crypto Device / Device Operation / Control Virtqueue}

The driver uses the control virtqueue to send control commands to the
device, such as session operations (See \ref{sec:Device Types / Crypto Device / Device
Operation / Control Virtqueue / Session operation}).

The header for controlq is of the following form:
\begin{lstlisting}
#define VIRTIO_CRYPTO_OPCODE(service, op)   (((service) << 8) | (op))

struct virtio_crypto_ctrl_header {
#define VIRTIO_CRYPTO_CIPHER_CREATE_SESSION \
       VIRTIO_CRYPTO_OPCODE(VIRTIO_CRYPTO_SERVICE_CIPHER, 0x02)
#define VIRTIO_CRYPTO_CIPHER_DESTROY_SESSION \
       VIRTIO_CRYPTO_OPCODE(VIRTIO_CRYPTO_SERVICE_CIPHER, 0x03)
#define VIRTIO_CRYPTO_HASH_CREATE_SESSION \
       VIRTIO_CRYPTO_OPCODE(VIRTIO_CRYPTO_SERVICE_HASH, 0x02)
#define VIRTIO_CRYPTO_HASH_DESTROY_SESSION \
       VIRTIO_CRYPTO_OPCODE(VIRTIO_CRYPTO_SERVICE_HASH, 0x03)
#define VIRTIO_CRYPTO_MAC_CREATE_SESSION \
       VIRTIO_CRYPTO_OPCODE(VIRTIO_CRYPTO_SERVICE_MAC, 0x02)
#define VIRTIO_CRYPTO_MAC_DESTROY_SESSION \
       VIRTIO_CRYPTO_OPCODE(VIRTIO_CRYPTO_SERVICE_MAC, 0x03)
#define VIRTIO_CRYPTO_AEAD_CREATE_SESSION \
       VIRTIO_CRYPTO_OPCODE(VIRTIO_CRYPTO_SERVICE_AEAD, 0x02)
#define VIRTIO_CRYPTO_AEAD_DESTROY_SESSION \
       VIRTIO_CRYPTO_OPCODE(VIRTIO_CRYPTO_SERVICE_AEAD, 0x03)
#define VIRTIO_CRYPTO_AKCIPHER_CREATE_SESSION \
       VIRTIO_CRYPTO_OPCODE(VIRTIO_CRYPTO_SERVICE_AKCIPHER, 0x04)
#define VIRTIO_CRYPTO_AKCIPHER_DESTROY_SESSION \
       VIRTIO_CRYPTO_OPCDE(VIRTIO_CRYPTO_SERVICE_AKCIPHER, 0x05)
    le32 opcode;
    /* algo should be service-specific algorithms */
    le32 algo;
    le32 flag;
    le32 reserved;
};
\end{lstlisting}

The controlq request is composed of four parts:
\begin{lstlisting}
struct virtio_crypto_op_ctrl_req {
    /* Device read only portion */

    struct virtio_crypto_ctrl_header header;

#define VIRTIO_CRYPTO_CTRLQ_OP_SPEC_HDR_LEGACY 56
    /* fixed length fields, opcode specific */
    u8 op_flf[flf_len];

    /* variable length fields, opcode specific */
    u8 op_vlf[vlf_len];

    /* Device write only portion */

    /* op result or completion status */
    u8 op_outcome[outcome_len];
};
\end{lstlisting}

\field{header} is a general header (see above).

\field{op_flf} is the opcode (in \field{header}) specific fixed-length parameters.

\field{flf_len} depends on the VIRTIO_CRYPTO_F_REVISION_1 feature bit (see below).

\field{op_vlf} is the opcode (in \field{header}) specific variable-length parameters.

\field{vlf_len} is the size of the specific structure used.
\begin{note}
The \field{vlf_len} of session-destroy operation and the hash-session-create
operation is ZERO.
\end{note}

\begin{itemize*}
\item If the opcode (in \field{header}) is VIRTIO_CRYPTO_CIPHER_CREATE_SESSION
    then \field{op_flf} is struct virtio_crypto_sym_create_session_flf if
    VIRTIO_CRYPTO_F_REVISION_1 is negotiated and struct virtio_crypto_sym_create_session_flf is
    padded to 56 bytes if NOT negotiated, and \field{op_vlf} is struct
    virtio_crypto_sym_create_session_vlf.
\item If the opcode (in \field{header}) is VIRTIO_CRYPTO_HASH_CREATE_SESSION
    then \field{op_flf} is struct virtio_crypto_hash_create_session_flf if
    VIRTIO_CRYPTO_F_REVISION_1 is negotiated and struct virtio_crypto_hash_create_session_flf is
    padded to 56 bytes if NOT negotiated.
\item If the opcode (in \field{header}) is VIRTIO_CRYPTO_MAC_CREATE_SESSION
    then \field{op_flf} is struct virtio_crypto_mac_create_session_flf if
    VIRTIO_CRYPTO_F_REVISION_1 is negotiated and struct virtio_crypto_mac_create_session_flf is
    padded to 56 bytes if NOT negotiated, and \field{op_vlf} is struct
    virtio_crypto_mac_create_session_vlf.
\item If the opcode (in \field{header}) is VIRTIO_CRYPTO_AEAD_CREATE_SESSION
    then \field{op_flf} is struct virtio_crypto_aead_create_session_flf if
    VIRTIO_CRYPTO_F_REVISION_1 is negotiated and struct virtio_crypto_aead_create_session_flf is
    padded to 56 bytes if NOT negotiated, and \field{op_vlf} is struct
    virtio_crypto_aead_create_session_vlf.
\item If the opcode (in \field{header}) is VIRTIO_CRYPTO_AKCIPHER_CREATE_SESSION
    then \field{op_flf} is struct virtio_crypto_akcipher_create_session_flf if
    VIRTIO_CRYPTO_F_REVISION_1 is negotiated and struct virtio_crypto_akcipher_create_session_flf is
    padded to 56 bytes if NOT negotiated, and \field{op_vlf} is struct
    virtio_crypto_akcipher_create_session_vlf.
\item If the opcode (in \field{header}) is VIRTIO_CRYPTO_CIPHER_DESTROY_SESSION
    or VIRTIO_CRYPTO_HASH_DESTROY_SESSION or VIRTIO_CRYPTO_MAC_DESTROY_SESSION or
    VIRTIO_CRYPTO_AEAD_DESTROY_SESSION then \field{op_flf} is struct
    virtio_crypto_destroy_session_flf if VIRTIO_CRYPTO_F_REVISION_1 is negotiated and
    struct virtio_crypto_destroy_session_flf is padded to 56 bytes if NOT negotiated.
\end{itemize*}

\field{op_outcome} stores the result of operation and must be struct
virtio_crypto_destroy_session_input for destroy session or
struct virtio_crypto_create_session_input for create session.

\field{outcome_len} is the size of the structure used.


\paragraph{Session operation}\label{sec:Device Types / Crypto Device / Device
Operation / Control Virtqueue / Session operation}

The session is a handle which describes the cryptographic parameters to be
applied to a number of buffers.

The following structure stores the result of session creation set by the device:

\begin{lstlisting}
struct virtio_crypto_create_session_input {
    le64 session_id;
    le32 status;
    le32 padding;
};
\end{lstlisting}

A request to destroy a session includes the following information:

\begin{lstlisting}
struct virtio_crypto_destroy_session_flf {
    /* Device read only portion */
    le64  session_id;
};

struct virtio_crypto_destroy_session_input {
    /* Device write only portion */
    u8  status;
};
\end{lstlisting}


\subparagraph{Session operation: HASH session}\label{sec:Device Types / Crypto Device / Device
Operation / Control Virtqueue / Session operation / Session operation: HASH session}

The fixed-length parameters of HASH session requests is as follows:

\begin{lstlisting}
struct virtio_crypto_hash_create_session_flf {
    /* Device read only portion */

    /* See VIRTIO_CRYPTO_HASH_* above */
    le32 algo;
    /* hash result length */
    le32 hash_result_len;
};
\end{lstlisting}


\subparagraph{Session operation: MAC session}\label{sec:Device Types / Crypto Device / Device
Operation / Control Virtqueue / Session operation / Session operation: MAC session}

The fixed-length and the variable-length parameters of MAC session requests are as follows:

\begin{lstlisting}
struct virtio_crypto_mac_create_session_flf {
    /* Device read only portion */

    /* See VIRTIO_CRYPTO_MAC_* above */
    le32 algo;
    /* hash result length */
    le32 hash_result_len;
    /* length of authenticated key */
    le32 auth_key_len;
    le32 padding;
};

struct virtio_crypto_mac_create_session_vlf {
    /* Device read only portion */

    /* The authenticated key */
    u8 auth_key[auth_key_len];
};
\end{lstlisting}

The length of \field{auth_key} is specified in \field{auth_key_len} in the struct
virtio_crypto_mac_create_session_flf.


\subparagraph{Session operation: Symmetric algorithms session}\label{sec:Device Types / Crypto Device / Device
Operation / Control Virtqueue / Session operation / Session operation: Symmetric algorithms session}

The request of symmetric session could be the CIPHER algorithms request
or the chain algorithms (chaining CIPHER and HASH/MAC) request.

The fixed-length and the variable-length parameters of CIPHER session requests are as follows:

\begin{lstlisting}
struct virtio_crypto_cipher_session_flf {
    /* Device read only portion */

    /* See VIRTIO_CRYPTO_CIPHER* above */
    le32 algo;
    /* length of key */
    le32 key_len;
#define VIRTIO_CRYPTO_OP_ENCRYPT  1
#define VIRTIO_CRYPTO_OP_DECRYPT  2
    /* encryption or decryption */
    le32 op;
    le32 padding;
};

struct virtio_crypto_cipher_session_vlf {
    /* Device read only portion */

    /* The cipher key */
    u8 cipher_key[key_len];
};
\end{lstlisting}

The length of \field{cipher_key} is specified in \field{key_len} in the struct
virtio_crypto_cipher_session_flf.

The fixed-length and the variable-length parameters of Chain session requests are as follows:

\begin{lstlisting}
struct virtio_crypto_alg_chain_session_flf {
    /* Device read only portion */

#define VIRTIO_CRYPTO_SYM_ALG_CHAIN_ORDER_HASH_THEN_CIPHER  1
#define VIRTIO_CRYPTO_SYM_ALG_CHAIN_ORDER_CIPHER_THEN_HASH  2
    le32 alg_chain_order;
/* Plain hash */
#define VIRTIO_CRYPTO_SYM_HASH_MODE_PLAIN    1
/* Authenticated hash (mac) */
#define VIRTIO_CRYPTO_SYM_HASH_MODE_AUTH     2
/* Nested hash */
#define VIRTIO_CRYPTO_SYM_HASH_MODE_NESTED   3
    le32 hash_mode;
    struct virtio_crypto_cipher_session_flf cipher_hdr;

#define VIRTIO_CRYPTO_ALG_CHAIN_SESS_OP_SPEC_HDR_SIZE  16
    /* fixed length fields, algo specific */
    u8 algo_flf[VIRTIO_CRYPTO_ALG_CHAIN_SESS_OP_SPEC_HDR_SIZE];

    /* length of the additional authenticated data (AAD) in bytes */
    le32 aad_len;
    le32 padding;
};

struct virtio_crypto_alg_chain_session_vlf {
    /* Device read only portion */

    /* The cipher key */
    u8 cipher_key[key_len];
    /* The authenticated key */
    u8 auth_key[auth_key_len];
};
\end{lstlisting}

\field{hash_mode} decides the type used by \field{algo_flf}.

\field{algo_flf} is fixed to 16 bytes and MUST contains or be one of
the following types:
\begin{itemize*}
\item struct virtio_crypto_hash_create_session_flf
\item struct virtio_crypto_mac_create_session_flf
\end{itemize*}
The data of unused part (if has) in \field{algo_flf} will be ignored.

The length of \field{cipher_key} is specified in \field{key_len} in \field{cipher_hdr}.

The length of \field{auth_key} is specified in \field{auth_key_len} in struct
virtio_crypto_mac_create_session_flf.

The fixed-length parameters of Symmetric session requests are as follows:

\begin{lstlisting}
struct virtio_crypto_sym_create_session_flf {
    /* Device read only portion */

#define VIRTIO_CRYPTO_SYM_SESS_OP_SPEC_HDR_SIZE  48
    /* fixed length fields, opcode specific */
    u8 op_flf[VIRTIO_CRYPTO_SYM_SESS_OP_SPEC_HDR_SIZE];

/* No operation */
#define VIRTIO_CRYPTO_SYM_OP_NONE  0
/* Cipher only operation on the data */
#define VIRTIO_CRYPTO_SYM_OP_CIPHER  1
/* Chain any cipher with any hash or mac operation. The order
   depends on the value of alg_chain_order param */
#define VIRTIO_CRYPTO_SYM_OP_ALGORITHM_CHAINING  2
    le32 op_type;
    le32 padding;
};
\end{lstlisting}

\field{op_flf} is fixed to 48 bytes, MUST contains or be one of
the following types:
\begin{itemize*}
\item struct virtio_crypto_cipher_session_flf
\item struct virtio_crypto_alg_chain_session_flf
\end{itemize*}
The data of unused part (if has) in \field{op_flf} will be ignored.

\field{op_type} decides the type used by \field{op_flf}.

The variable-length parameters of Symmetric session requests are as follows:

\begin{lstlisting}
struct virtio_crypto_sym_create_session_vlf {
    /* Device read only portion */
    /* variable length fields, opcode specific */
    u8 op_vlf[vlf_len];
};
\end{lstlisting}

\field{op_vlf} MUST contains or be one of the following types:
\begin{itemize*}
\item struct virtio_crypto_cipher_session_vlf
\item struct virtio_crypto_alg_chain_session_vlf
\end{itemize*}

\field{op_type} in struct virtio_crypto_sym_create_session_flf decides the
type used by \field{op_vlf}.

\field{vlf_len} is the size of the specific structure used.


\subparagraph{Session operation: AEAD session}\label{sec:Device Types / Crypto Device / Device
Operation / Control Virtqueue / Session operation / Session operation: AEAD session}

The fixed-length and the variable-length parameters of AEAD session requests are as follows:

\begin{lstlisting}
struct virtio_crypto_aead_create_session_flf {
    /* Device read only portion */

    /* See VIRTIO_CRYPTO_AEAD_* above */
    le32 algo;
    /* length of key */
    le32 key_len;
    /* Authentication tag length */
    le32 tag_len;
    /* The length of the additional authenticated data (AAD) in bytes */
    le32 aad_len;
    /* encryption or decryption, See above VIRTIO_CRYPTO_OP_* */
    le32 op;
    le32 padding;
};

struct virtio_crypto_aead_create_session_vlf {
    /* Device read only portion */
    u8 key[key_len];
};
\end{lstlisting}

The length of \field{key} is specified in \field{key_len} in struct
virtio_crypto_aead_create_session_flf.

\subparagraph{Session operation: AKCIPHER session}\label{sec:Device Types / Crypto Device / Device
Operation / Control Virtqueue / Session operation / Session operation: AKCIPHER session}

Due to the complexity of asymmetric key algorithms, different algorithms
require different parameters. The following data structures are used as
supplementary parameters to describe the asymmetric algorithm sessions.

For the RSA algorithm, the extra parameters are as follows:
\begin{lstlisting}
struct virtio_crypto_rsa_session_para {
#define VIRTIO_CRYPTO_RSA_RAW_PADDING   0
#define VIRTIO_CRYPTO_RSA_PKCS1_PADDING 1
    le32 padding_algo;

#define VIRTIO_CRYPTO_RSA_NO_HASH   0
#define VIRTIO_CRYPTO_RSA_MD2       1
#define VIRTIO_CRYPTO_RSA_MD3       2
#define VIRTIO_CRYPTO_RSA_MD4       3
#define VIRTIO_CRYPTO_RSA_MD5       4
#define VIRTIO_CRYPTO_RSA_SHA1      5
#define VIRTIO_CRYPTO_RSA_SHA256    6
#define VIRTIO_CRYPTO_RSA_SHA384    7
#define VIRTIO_CRYPTO_RSA_SHA512    8
#define VIRTIO_CRYPTO_RSA_SHA224    9
    le32 hash_algo;
};
\end{lstlisting}

\field{padding_algo} specifies the padding method used by RSA sessions.
\begin{itemize*}
\item If VIRTIO_CRYPTO_RSA_RAW_PADDING is specified, 1) \field{hash_algo}
is ignored, 2) ciphertext and plaintext MUST be padded with leading zeros,
3) and RSA sessions with VIRTIO_CRYPTO_RSA_RAW_PADDING MUST not be used
for verification and signing operations.
\item If VIRTIO_CRYPTO_RSA_PKCS1_PADDING is specified, EMSA-PKCS1-v1_5 padding method
is used (see \hyperref[intro:rfc3447]{PKCS\#1}), \field{hash_algo} specifies how the
digest of the data passed to RSA sessions is calculated when verifying and signing.
It only affects the padding algorithm and is ignored during encryption and decryption.
\end{itemize*}

The ECC algorithms such as the ECDSA algorithm, cannot use custom curves, only the
following known curves can be used (see \hyperref[intro:NIST]{NIST-recommended curves}).

\begin{lstlisting}
#define VIRTIO_CRYPTO_CURVE_UNKNOWN   0
#define VIRTIO_CRYPTO_CURVE_NIST_P192 1
#define VIRTIO_CRYPTO_CURVE_NIST_P224 2
#define VIRTIO_CRYPTO_CURVE_NIST_P256 3
#define VIRTIO_CRYPTO_CURVE_NIST_P384 4
#define VIRTIO_CRYPTO_CURVE_NIST_P521 5
\end{lstlisting}

For the ECDSA algorithm, the extra parameters are as follows:
\begin{lstlisting}
struct virtio_crypto_ecdsa_session_para {
    /* See VIRTIO_CRYPTO_CURVE_* above */
    le32 curve_id;
};
\end{lstlisting}

The fixed-length and the variable-length parameters of AKCIPHER session requests are as follows:
\begin{lstlisting}
struct virtio_crypto_akcipher_create_session_flf {
    /* Device read only portion */

    /* See VIRTIO_CRYPTO_AKCIPHER_* above */
    le32 algo;
#define VIRTIO_CRYPTO_AKCIPHER_KEY_TYPE_PUBLIC 1
#define VIRTIO_CRYPTO_AKCIPHER_KEY_TYPE_PRIVATE 2
    le32 key_type;
    /* length of key */
    le32 key_len;

#define VIRTIO_CRYPTO_AKCIPHER_SESS_ALGO_SPEC_HDR_SIZE 44
    u8 algo_flf[VIRTIO_CRYPTO_AKCIPHER_SESS_ALGO_SPEC_HDR_SIZE];
};

struct virtio_crypto_akcipher_create_session_vlf {
    /* Device read only portion */
    u8 key[key_len];
};
\end{lstlisting}

\field{algo} decides the type used by \field{algo_flf}.
\field{algo_flf} is fixed to 44 bytes and MUST contains of be one the
following structures:
\begin{itemize*}
\item struct virtio_crypto_rsa_session_para
\item struct virtio_crypto_ecdsa_session_para
\end{itemize*}

The length of \field{key} is specified in \field{key_len} in the struct
virtio_crypto_akcipher_create_session_flf.

For the RSA algorithm, the key needs to be encoded according to
\hyperref[intro:rfc3447]{PKCS\#1}. The private key is described with the
RSAPrivateKey structure, and the public key is described with the RSAPublicKey
structure. These ASN.1 structures are encoded in DER encoding rules (see
\hyperref[intro:rfc6025]{rfc6025}).

\begin{lstlisting}
RSAPrivateKey ::= SEQUENCE {
    version          INTEGER,
    modulus          INTEGER,
    publicExponent   INTEGER,
    privateExponent  INTEGER,
    prime1           INTEGER,
    prime2           INTEGER,
    exponent1        INTEGER,
    exponent1        INTEGER,
    coefficient      INTEGER,
    otherPrimeInfos  OtherPrimeInfos OPTIONAL
}

OtherPrimeInfos ::= SEQUENCE SIZE(1...MAX) OF OtherPrimeInfo

OtherPrimeINfo ::= SEQUENCE {
    prime           INTEGER,
    exponent        INTEGER,
    coefficient     INTEGER
}

RSAPublicKey ::= SEQUENCE {
    modulus         INTEGER,
    publicExponent  INTEGER
}
\end{lstlisting}

For the ECDSA algorithm, the private key is encoded according to
\hyperref[intro:rfc5915]{RFC5915}, the private key of the ECDSA algorithm
is described by the ASN.1 structure ECPrivateKey and encoded with DER
encoding rules (see \hyperref[intro:rfc6025]{rfc6025}).

\begin{lstlisting}
ECPrivateKey ::= SEQUNCE {
    version         INTEGER,
    privateKey      OCTET STRING,
    parameters [0]  ECParameters {{ NamedCurve }} OPTIONAL,
    publicKey  [1]  BIT STRING OPTIONAL
}
\end{lstlisting}

The public key of the ECDSA algorithm is encoded according to \hyperref[intro:SEC1]{SEC1},
and the public key of ECDSA is described by the ASN.1 structure ECPoint.
When initializing a session with ECDSA public key, the ECPoint is DER encoded and the
\field{key} only contains the value part of ECPoint, that is, the header part of the
OCTET STRING will be omitted (see \hyperref[intro:rfc6025]{rfc6025}).

\begin{lstlisting}
ECPoint ::= OCTET STRING
\end{lstlisting}

The length of \field{key} is specified in \field{key_len} in
struct virtio_crypto_akcipher_create_session_flf.

\drivernormative{\subparagraph}{Session operation: create session}{Device Types / Crypto Device / Device
Operation / Control Virtqueue / Session operation / Session operation: create session}

\begin{itemize*}
\item The driver MUST set the \field{opcode} field based on service type: CIPHER, HASH, MAC, AEAD or AKCIPHER.
\item The driver MUST set the control general header, the opcode specific header,
    the opcode specific extra parameters and the opcode specific outcome buffer in turn.
    See \ref{sec:Device Types / Crypto Device / Device Operation / Control Virtqueue}.
\item The driver MUST set the \field{reversed} field to zero.
\end{itemize*}

\devicenormative{\subparagraph}{Session operation: create session}{Device Types / Crypto Device / Device
Operation / Control Virtqueue / Session operation / Session operation: create session}

\begin{itemize*}
\item The device MUST use the corresponding opcode specific structure according to the
    \field{opcode} in the control general header.
\item The device MUST extract extra parameters according to the structures used.
\item The device MUST set the \field{status} field to one of the following values of enum
    VIRTIO_CRYPTO_STATUS after finish a session creation:
\begin{itemize*}
\item VIRTIO_CRYPTO_OK if a session is created successfully.
\item VIRTIO_CRYPTO_NOTSUPP if the requested algorithm or operation is unsupported.
\item VIRTIO_CRYPTO_NOSPC if no free session ID (only when the VIRTIO_CRYPTO_F_REVISION_1
    feature bit is negotiated).
\item VIRTIO_CRYPTO_ERR if failure not mentioned above occurs.
\end{itemize*}
\item The device MUST set the \field{session_id} field to a unique session identifier only
    if the status is set to VIRTIO_CRYPTO_OK.
\end{itemize*}

\drivernormative{\subparagraph}{Session operation: destroy session}{Device Types / Crypto Device / Device
Operation / Control Virtqueue / Session operation / Session operation: destroy session}

\begin{itemize*}
\item The driver MUST set the \field{opcode} field based on service type: CIPHER, HASH, MAC, AEAD or AKCIPHER.
\item The driver MUST set the \field{session_id} to a valid value assigned by the device
    when the session was created.
\end{itemize*}

\devicenormative{\subparagraph}{Session operation: destroy session}{Device Types / Crypto Device / Device
Operation / Control Virtqueue / Session operation / Session operation: destroy session}

\begin{itemize*}
\item The device MUST set the \field{status} field to one of the following values of enum VIRTIO_CRYPTO_STATUS.
\begin{itemize*}
\item VIRTIO_CRYPTO_OK if a session is created successfully.
\item VIRTIO_CRYPTO_ERR if any failure occurs.
\end{itemize*}
\end{itemize*}


\subsubsection{Data Virtqueue}\label{sec:Device Types / Crypto Device / Device Operation / Data Virtqueue}

The driver uses the data virtqueues to transmit crypto operation requests to the device,
and completes the crypto operations.

The header for dataq is as follows:

\begin{lstlisting}
struct virtio_crypto_op_header {
#define VIRTIO_CRYPTO_CIPHER_ENCRYPT \
    VIRTIO_CRYPTO_OPCODE(VIRTIO_CRYPTO_SERVICE_CIPHER, 0x00)
#define VIRTIO_CRYPTO_CIPHER_DECRYPT \
    VIRTIO_CRYPTO_OPCODE(VIRTIO_CRYPTO_SERVICE_CIPHER, 0x01)
#define VIRTIO_CRYPTO_HASH \
    VIRTIO_CRYPTO_OPCODE(VIRTIO_CRYPTO_SERVICE_HASH, 0x00)
#define VIRTIO_CRYPTO_MAC \
    VIRTIO_CRYPTO_OPCODE(VIRTIO_CRYPTO_SERVICE_MAC, 0x00)
#define VIRTIO_CRYPTO_AEAD_ENCRYPT \
    VIRTIO_CRYPTO_OPCODE(VIRTIO_CRYPTO_SERVICE_AEAD, 0x00)
#define VIRTIO_CRYPTO_AEAD_DECRYPT \
    VIRTIO_CRYPTO_OPCODE(VIRTIO_CRYPTO_SERVICE_AEAD, 0x01)
#define VIRTIO_CRYPTO_AKCIPHER_ENCRYPT \
    VIRTIO_CRYPTO_OPCODE(VIRTIO_CRYPTO_SERVICE_AKCIPHER, 0x00)
#define VIRTIO_CRYPTO_AKCIPHER_DECRYPT \
    VIRTIO_CRYPTO_OPCODE(VIRTIO_CRYPTO_SERVICE_AKCIPHER, 0x01)
#define VIRTIO_CRYPTO_AKCIPHER_SIGN \
    VIRTIO_CRYPTO_OPCODE(VIRTIO_CRYPTO_SERVICE_AKCIPHER, 0x02)
#define VIRTIO_CRYPTO_AKCIPHER_VERIFY \
    VIRTIO_CRYPTO_OPCODE(VIRTIO_CRYPTO_SERVICE_AKCIPHER, 0x03)
    le32 opcode;
    /* algo should be service-specific algorithms */
    le32 algo;
    le64 session_id;
#define VIRTIO_CRYPTO_FLAG_SESSION_MODE 1
    /* control flag to control the request */
    le32 flag;
    le32 padding;
};
\end{lstlisting}

\begin{note}
If VIRTIO_CRYPTO_F_REVISION_1 is not negotiated the \field{flag} is ignored.

If VIRTIO_CRYPTO_F_REVISION_1 is negotiated but VIRTIO_CRYPTO_F_<SERVICE>_STATELESS_MODE
is not negotiated, then the device SHOULD reject <SERVICE> requests if
VIRTIO_CRYPTO_FLAG_SESSION_MODE is not set (in \field{flag}).
\end{note}

The dataq request is composed of four parts:
\begin{lstlisting}
struct virtio_crypto_op_data_req {
    /* Device read only portion */

    struct virtio_crypto_op_header header;

#define VIRTIO_CRYPTO_DATAQ_OP_SPEC_HDR_LEGACY 48
    /* fixed length fields, opcode specific */
    u8 op_flf[flf_len];

    /* Device read && write portion */
    /* variable length fields, opcode specific */
    u8 op_vlf[vlf_len];

    /* Device write only portion */
    struct virtio_crypto_inhdr inhdr;
};
\end{lstlisting}

\field{header} is a general header (see above).

\field{op_flf} is the opcode (in \field{header}) specific header.

\field{flf_len} depends on the VIRTIO_CRYPTO_F_REVISION_1 feature bit
(see below).

\field{op_vlf} is the opcode (in \field{header}) specific parameters.

\field{vlf_len} is the size of the specific structure used.

\begin{itemize*}
\item If the the opcode (in \field{header}) is VIRTIO_CRYPTO_CIPHER_ENCRYPT
    or VIRTIO_CRYPTO_CIPHER_DECRYPT then:
    \begin{itemize*}
    \item If VIRTIO_CRYPTO_F_CIPHER_STATELESS_MODE is negotiated, \field{op_flf} is
        struct virtio_crypto_sym_data_flf_stateless, and \field{op_vlf} is struct
        virtio_crypto_sym_data_vlf_stateless.
    \item If VIRTIO_CRYPTO_F_CIPHER_STATELESS_MODE is NOT negotiated, \field{op_flf}
        is struct virtio_crypto_sym_data_flf if VIRTIO_CRYPTO_F_REVISION_1 is negotiated
        and struct virtio_crypto_sym_data_flf is padded to 48 bytes if NOT negotiated,
        and \field{op_vlf} is struct virtio_crypto_sym_data_vlf.
    \end{itemize*}
\item If the the opcode (in \field{header}) is VIRTIO_CRYPTO_HASH:
    \begin{itemize*}
    \item If VIRTIO_CRYPTO_F_HASH_STATELESS_MODE is negotiated, \field{op_flf} is
        struct virtio_crypto_hash_data_flf_stateless, and \field{op_vlf} is struct
        virtio_crypto_hash_data_vlf_stateless.
    \item If VIRTIO_CRYPTO_F_HASH_STATELESS_MODE is NOT negotiated, \field{op_flf}
        is struct virtio_crypto_hash_data_flf if VIRTIO_CRYPTO_F_REVISION_1 is negotiated
        and struct virtio_crypto_hash_data_flf is padded to 48 bytes if NOT negotiated,
        and \field{op_vlf} is struct virtio_crypto_hash_data_vlf.
    \end{itemize*}
\item If the the opcode (in \field{header}) is VIRTIO_CRYPTO_MAC:
    \begin{itemize*}
    \item If VIRTIO_CRYPTO_F_MAC_STATELESS_MODE is negotiated, \field{op_flf} is
        struct virtio_crypto_mac_data_flf_stateless, and \field{op_vlf} is struct
        virtio_crypto_mac_data_vlf_stateless.
    \item If VIRTIO_CRYPTO_F_MAC_STATELESS_MODE is NOT negotiated, \field{op_flf}
        is struct virtio_crypto_mac_data_flf if VIRTIO_CRYPTO_F_REVISION_1 is negotiated
        and struct virtio_crypto_mac_data_flf is padded to 48 bytes if NOT negotiated,
        and \field{op_vlf} is struct virtio_crypto_mac_data_vlf.
    \end{itemize*}
\item If the the opcode (in \field{header}) is VIRTIO_CRYPTO_AEAD_ENCRYPT
    or VIRTIO_CRYPTO_AEAD_DECRYPT then:
    \begin{itemize*}
    \item If VIRTIO_CRYPTO_F_AEAD_STATELESS_MODE is negotiated, \field{op_flf} is
        struct virtio_crypto_aead_data_flf_stateless, and \field{op_vlf} is struct
        virtio_crypto_aead_data_vlf_stateless.
    \item If VIRTIO_CRYPTO_F_AEAD_STATELESS_MODE is NOT negotiated, \field{op_flf}
        is struct virtio_crypto_aead_data_flf if VIRTIO_CRYPTO_F_REVISION_1 is negotiated
        and struct virtio_crypto_aead_data_flf is padded to 48 bytes if NOT negotiated,
        and \field{op_vlf} is struct virtio_crypto_aead_data_vlf.
    \end{itemize*}
\item If the opcode (in \field{header}) is VIRTIO_CRYPTO_AKCIPHER_ENCRYPT, VIRTIO_CRYPTO_AKCIPHER_DECRYPT,
    VIRTIO_CRYPTO_AKCIPHER_SIGN or VIRTIO_CRYPTO_AKCIPHER_VERIFY then:
    \begin{itemize*}
    \item If VIRTIO_CRYPTO_F_AKCIPHER_STATELESS_MODE is negotiated, \field{op_flf} is
        struct virtio_crypto_akcipher_data_flf_statless, and \field{op_vlf} is struct
        virtio_crypto_akcipher_data_vlf_stateless.
    \item If VIRTIO_CRYPTO_F_AKCIPHER_STATELESS_MODE is NOT negotiated, \field{op_flf}
        is struct virtio_crypto_akcipher_data_flf if VIRTIO_CRYPTO_F_REVISION_1 is negotiated
        and struct virtio_crypto_akcipher_data_flf is padded to 48 bytes if NOT negotiated,
        and \field{op_vlf} is struct virtio_crypto_akcipher_data_vlf.
    \end{itemize*}
\end{itemize*}

\field{inhdr} is a unified input header that used to return the status of
the operations, is defined as follows:

\begin{lstlisting}
struct virtio_crypto_inhdr {
    u8 status;
};
\end{lstlisting}

\subsubsection{HASH Service Operation}\label{sec:Device Types / Crypto Device / Device Operation / HASH Service Operation}

Session mode HASH service requests are as follows:

\begin{lstlisting}
struct virtio_crypto_hash_data_flf {
    /* length of source data */
    le32 src_data_len;
    /* hash result length */
    le32 hash_result_len;
};

struct virtio_crypto_hash_data_vlf {
    /* Device read only portion */
    /* Source data */
    u8 src_data[src_data_len];

    /* Device write only portion */
    /* Hash result data */
    u8 hash_result[hash_result_len];
};
\end{lstlisting}

Each data request uses the virtio_crypto_hash_data_flf structure and the
virtio_crypto_hash_data_vlf structure to store information used to run the
HASH operations.

\field{src_data} is the source data that will be processed.
\field{src_data_len} is the length of source data.
\field{hash_result} is the result data and \field{hash_result_len} is the length
of it.

Stateless mode HASH service requests are as follows:

\begin{lstlisting}
struct virtio_crypto_hash_data_flf_stateless {
    struct {
        /* See VIRTIO_CRYPTO_HASH_* above */
        le32 algo;
    } sess_para;

    /* length of source data */
    le32 src_data_len;
    /* hash result length */
    le32 hash_result_len;
    le32 reserved;
};
struct virtio_crypto_hash_data_vlf_stateless {
    /* Device read only portion */
    /* Source data */
    u8 src_data[src_data_len];

    /* Device write only portion */
    /* Hash result data */
    u8 hash_result[hash_result_len];
};
\end{lstlisting}

\drivernormative{\paragraph}{HASH Service Operation}{Device Types / Crypto Device / Device Operation / HASH Service Operation}

\begin{itemize*}
\item If the driver uses the session mode, then the driver MUST set \field{session_id}
    in struct virtio_crypto_op_header to a valid value assigned by the device when the
    session was created.
\item If the VIRTIO_CRYPTO_F_HASH_STATELESS_MODE feature bit is negotiated, 1) if the
    driver uses the stateless mode, then the driver MUST set the \field{flag} field in
    struct virtio_crypto_op_header to ZERO and MUST set the fields in struct
    virtio_crypto_hash_data_flf_stateless.sess_para, 2) if the driver uses the session
    mode, then the driver MUST set the \field{flag} field in struct virtio_crypto_op_header
    to VIRTIO_CRYPTO_FLAG_SESSION_MODE.
\item The driver MUST set \field{opcode} in struct virtio_crypto_op_header to VIRTIO_CRYPTO_HASH.
\end{itemize*}

\devicenormative{\paragraph}{HASH Service Operation}{Device Types / Crypto Device / Device Operation / HASH Service Operation}

\begin{itemize*}
\item The device MUST use the corresponding structure according to the \field{opcode}
    in the data general header.
\item If the VIRTIO_CRYPTO_F_HASH_STATELESS_MODE feature bit is negotiated, the device
    MUST parse \field{flag} field in struct virtio_crypto_op_header in order to decide
    which mode the driver uses.
\item The device MUST copy the results of HASH operations in the hash_result[] if HASH
    operations success.
\item The device MUST set \field{status} in struct virtio_crypto_inhdr to one of the
    following values of enum VIRTIO_CRYPTO_STATUS:
\begin{itemize*}
\item VIRTIO_CRYPTO_OK if the operation success.
\item VIRTIO_CRYPTO_NOTSUPP if the requested algorithm or operation is unsupported.
\item VIRTIO_CRYPTO_INVSESS if the session ID invalid when in session mode.
\item VIRTIO_CRYPTO_ERR if any failure not mentioned above occurs.
\end{itemize*}
\end{itemize*}


\subsubsection{MAC Service Operation}\label{sec:Device Types / Crypto Device / Device Operation / MAC Service Operation}

Session mode MAC service requests are as follows:

\begin{lstlisting}
struct virtio_crypto_mac_data_flf {
    struct virtio_crypto_hash_data_flf hdr;
};

struct virtio_crypto_mac_data_vlf {
    /* Device read only portion */
    /* Source data */
    u8 src_data[src_data_len];

    /* Device write only portion */
    /* Hash result data */
    u8 hash_result[hash_result_len];
};
\end{lstlisting}

Each request uses the virtio_crypto_mac_data_flf structure and the
virtio_crypto_mac_data_vlf structure to store information used to run the
MAC operations.

\field{src_data} is the source data that will be processed.
\field{src_data_len} is the length of source data.
\field{hash_result} is the result data and \field{hash_result_len} is the length
of it.

Stateless mode MAC service requests are as follows:

\begin{lstlisting}
struct virtio_crypto_mac_data_flf_stateless {
    struct {
        /* See VIRTIO_CRYPTO_MAC_* above */
        le32 algo;
        /* length of authenticated key */
        le32 auth_key_len;
    } sess_para;

    /* length of source data */
    le32 src_data_len;
    /* hash result length */
    le32 hash_result_len;
};

struct virtio_crypto_mac_data_vlf_stateless {
    /* Device read only portion */
    /* The authenticated key */
    u8 auth_key[auth_key_len];
    /* Source data */
    u8 src_data[src_data_len];

    /* Device write only portion */
    /* Hash result data */
    u8 hash_result[hash_result_len];
};
\end{lstlisting}

\field{auth_key} is the authenticated key that will be used during the process.
\field{auth_key_len} is the length of the key.

\drivernormative{\paragraph}{MAC Service Operation}{Device Types / Crypto Device / Device Operation / MAC Service Operation}

\begin{itemize*}
\item If the driver uses the session mode, then the driver MUST set \field{session_id}
    in struct virtio_crypto_op_header to a valid value assigned by the device when the
    session was created.
\item If the VIRTIO_CRYPTO_F_MAC_STATELESS_MODE feature bit is negotiated, 1) if the
    driver uses the stateless mode, then the driver MUST set the \field{flag} field
    in struct virtio_crypto_op_header to ZERO and MUST set the fields in struct
    virtio_crypto_mac_data_flf_stateless.sess_para, 2) if the driver uses the session
    mode, then the driver MUST set the \field{flag} field in struct virtio_crypto_op_header
    to VIRTIO_CRYPTO_FLAG_SESSION_MODE.
\item The driver MUST set \field{opcode} in struct virtio_crypto_op_header to VIRTIO_CRYPTO_MAC.
\end{itemize*}

\devicenormative{\paragraph}{MAC Service Operation}{Device Types / Crypto Device / Device Operation / MAC Service Operation}

\begin{itemize*}
\item If the VIRTIO_CRYPTO_F_MAC_STATELESS_MODE feature bit is negotiated, the device
    MUST parse \field{flag} field in struct virtio_crypto_op_header in order to decide
	which mode the driver uses.
\item The device MUST copy the results of MAC operations in the hash_result[] if HASH
    operations success.
\item The device MUST set \field{status} in struct virtio_crypto_inhdr to one of the
    following values of enum VIRTIO_CRYPTO_STATUS:
\begin{itemize*}
\item VIRTIO_CRYPTO_OK if the operation success.
\item VIRTIO_CRYPTO_NOTSUPP if the requested algorithm or operation is unsupported.
\item VIRTIO_CRYPTO_INVSESS if the session ID invalid when in session mode.
\item VIRTIO_CRYPTO_ERR if any failure not mentioned above occurs.
\end{itemize*}
\end{itemize*}

\subsubsection{Symmetric algorithms Operation}\label{sec:Device Types / Crypto Device / Device Operation / Symmetric algorithms Operation}

Session mode CIPHER service requests are as follows:

\begin{lstlisting}
struct virtio_crypto_cipher_data_flf {
    /*
     * Byte Length of valid IV/Counter data pointed to by the below iv data.
     *
     * For block ciphers in CBC or F8 mode, or for Kasumi in F8 mode, or for
     *   SNOW3G in UEA2 mode, this is the length of the IV (which
     *   must be the same as the block length of the cipher).
     * For block ciphers in CTR mode, this is the length of the counter
     *   (which must be the same as the block length of the cipher).
     */
    le32 iv_len;
    /* length of source data */
    le32 src_data_len;
    /* length of destination data */
    le32 dst_data_len;
    le32 padding;
};

struct virtio_crypto_cipher_data_vlf {
    /* Device read only portion */

    /*
     * Initialization Vector or Counter data.
     *
     * For block ciphers in CBC or F8 mode, or for Kasumi in F8 mode, or for
     *   SNOW3G in UEA2 mode, this is the Initialization Vector (IV)
     *   value.
     * For block ciphers in CTR mode, this is the counter.
     * For AES-XTS, this is the 128bit tweak, i, from IEEE Std 1619-2007.
     *
     * The IV/Counter will be updated after every partial cryptographic
     * operation.
     */
    u8 iv[iv_len];
    /* Source data */
    u8 src_data[src_data_len];

    /* Device write only portion */
    /* Destination data */
    u8 dst_data[dst_data_len];
};
\end{lstlisting}

Session mode requests of algorithm chaining are as follows:

\begin{lstlisting}
struct virtio_crypto_alg_chain_data_flf {
    le32 iv_len;
    /* Length of source data */
    le32 src_data_len;
    /* Length of destination data */
    le32 dst_data_len;
    /* Starting point for cipher processing in source data */
    le32 cipher_start_src_offset;
    /* Length of the source data that the cipher will be computed on */
    le32 len_to_cipher;
    /* Starting point for hash processing in source data */
    le32 hash_start_src_offset;
    /* Length of the source data that the hash will be computed on */
    le32 len_to_hash;
    /* Length of the additional auth data */
    le32 aad_len;
    /* Length of the hash result */
    le32 hash_result_len;
    le32 reserved;
};

struct virtio_crypto_alg_chain_data_vlf {
    /* Device read only portion */

    /* Initialization Vector or Counter data */
    u8 iv[iv_len];
    /* Source data */
    u8 src_data[src_data_len];
    /* Additional authenticated data if exists */
    u8 aad[aad_len];

    /* Device write only portion */

    /* Destination data */
    u8 dst_data[dst_data_len];
    /* Hash result data */
    u8 hash_result[hash_result_len];
};
\end{lstlisting}

Session mode requests of symmetric algorithm are as follows:

\begin{lstlisting}
struct virtio_crypto_sym_data_flf {
    /* Device read only portion */

#define VIRTIO_CRYPTO_SYM_DATA_REQ_HDR_SIZE    40
    u8 op_type_flf[VIRTIO_CRYPTO_SYM_DATA_REQ_HDR_SIZE];

    /* See above VIRTIO_CRYPTO_SYM_OP_* */
    le32 op_type;
    le32 padding;
};

struct virtio_crypto_sym_data_vlf {
    u8 op_type_vlf[sym_para_len];
};
\end{lstlisting}

Each request uses the virtio_crypto_sym_data_flf structure and the
virtio_crypto_sym_data_flf structure to store information used to run the
CIPHER operations.

\field{op_type_flf} is the \field{op_type} specific header, it MUST starts
with or be one of the following structures:
\begin{itemize*}
\item struct virtio_crypto_cipher_data_flf
\item struct virtio_crypto_alg_chain_data_flf
\end{itemize*}

The length of \field{op_type_flf} is fixed to 40 bytes, the data of unused
part (if has) will be ignored.

\field{op_type_vlf} is the \field{op_type} specific parameters, it MUST starts
with or be one of the following structures:
\begin{itemize*}
\item struct virtio_crypto_cipher_data_vlf
\item struct virtio_crypto_alg_chain_data_vlf
\end{itemize*}

\field{sym_para_len} is the size of the specific structure used.

Stateless mode CIPHER service requests are as follows:

\begin{lstlisting}
struct virtio_crypto_cipher_data_flf_stateless {
    struct {
        /* See VIRTIO_CRYPTO_CIPHER* above */
        le32 algo;
        /* length of key */
        le32 key_len;

        /* See VIRTIO_CRYPTO_OP_* above */
        le32 op;
    } sess_para;

    /*
     * Byte Length of valid IV/Counter data pointed to by the below iv data.
     */
    le32 iv_len;
    /* length of source data */
    le32 src_data_len;
    /* length of destination data */
    le32 dst_data_len;
};

struct virtio_crypto_cipher_data_vlf_stateless {
    /* Device read only portion */

    /* The cipher key */
    u8 cipher_key[key_len];

    /* Initialization Vector or Counter data. */
    u8 iv[iv_len];
    /* Source data */
    u8 src_data[src_data_len];

    /* Device write only portion */
    /* Destination data */
    u8 dst_data[dst_data_len];
};
\end{lstlisting}

Stateless mode requests of algorithm chaining are as follows:

\begin{lstlisting}
struct virtio_crypto_alg_chain_data_flf_stateless {
    struct {
        /* See VIRTIO_CRYPTO_SYM_ALG_CHAIN_ORDER_* above */
        le32 alg_chain_order;
        /* length of the additional authenticated data in bytes */
        le32 aad_len;

        struct {
            /* See VIRTIO_CRYPTO_CIPHER* above */
            le32 algo;
            /* length of key */
            le32 key_len;
            /* See VIRTIO_CRYPTO_OP_* above */
            le32 op;
        } cipher;

        struct {
            /* See VIRTIO_CRYPTO_HASH_* or VIRTIO_CRYPTO_MAC_* above */
            le32 algo;
            /* length of authenticated key */
            le32 auth_key_len;
            /* See VIRTIO_CRYPTO_SYM_HASH_MODE_* above */
            le32 hash_mode;
        } hash;
    } sess_para;

    le32 iv_len;
    /* Length of source data */
    le32 src_data_len;
    /* Length of destination data */
    le32 dst_data_len;
    /* Starting point for cipher processing in source data */
    le32 cipher_start_src_offset;
    /* Length of the source data that the cipher will be computed on */
    le32 len_to_cipher;
    /* Starting point for hash processing in source data */
    le32 hash_start_src_offset;
    /* Length of the source data that the hash will be computed on */
    le32 len_to_hash;
    /* Length of the additional auth data */
    le32 aad_len;
    /* Length of the hash result */
    le32 hash_result_len;
    le32 reserved;
};

struct virtio_crypto_alg_chain_data_vlf_stateless {
    /* Device read only portion */

    /* The cipher key */
    u8 cipher_key[key_len];
    /* The auth key */
    u8 auth_key[auth_key_len];
    /* Initialization Vector or Counter data */
    u8 iv[iv_len];
    /* Additional authenticated data if exists */
    u8 aad[aad_len];
    /* Source data */
    u8 src_data[src_data_len];

    /* Device write only portion */

    /* Destination data */
    u8 dst_data[dst_data_len];
    /* Hash result data */
    u8 hash_result[hash_result_len];
};
\end{lstlisting}

Stateless mode requests of symmetric algorithm are as follows:

\begin{lstlisting}
struct virtio_crypto_sym_data_flf_stateless {
    /* Device read only portion */
#define VIRTIO_CRYPTO_SYM_DATE_REQ_HDR_STATELESS_SIZE    72
    u8 op_type_flf[VIRTIO_CRYPTO_SYM_DATE_REQ_HDR_STATELESS_SIZE];

    /* Device write only portion */
    /* See above VIRTIO_CRYPTO_SYM_OP_* */
    le32 op_type;
};

struct virtio_crypto_sym_data_vlf_stateless {
    u8 op_type_vlf[sym_para_len];
};
\end{lstlisting}

\field{op_type_flf} is the \field{op_type} specific header, it MUST starts
with or be one of the following structures:
\begin{itemize*}
\item struct virtio_crypto_cipher_data_flf_stateless
\item struct virtio_crypto_alg_chain_data_flf_stateless
\end{itemize*}

The length of \field{op_type_flf} is fixed to 72 bytes, the data of unused
part (if has) will be ignored.

\field{op_type_vlf} is the \field{op_type} specific parameters, it MUST starts
with or be one of the following structures:
\begin{itemize*}
\item struct virtio_crypto_cipher_data_vlf_stateless
\item struct virtio_crypto_alg_chain_data_vlf_stateless
\end{itemize*}

\field{sym_para_len} is the size of the specific structure used.

\drivernormative{\paragraph}{Symmetric algorithms Operation}{Device Types / Crypto Device / Device Operation / Symmetric algorithms Operation}

\begin{itemize*}
\item If the driver uses the session mode, then the driver MUST set \field{session_id}
    in struct virtio_crypto_op_header to a valid value assigned by the device when the
    session was created.
\item If the VIRTIO_CRYPTO_F_CIPHER_STATELESS_MODE feature bit is negotiated, 1) if the
    driver uses the stateless mode, then the driver MUST set the \field{flag} field in
    struct virtio_crypto_op_header to ZERO and MUST set the fields in struct
    virtio_crypto_cipher_data_flf_stateless.sess_para or struct
    virtio_crypto_alg_chain_data_flf_stateless.sess_para, 2) if the driver uses the
    session mode, then the driver MUST set the \field{flag} field in struct
    virtio_crypto_op_header to VIRTIO_CRYPTO_FLAG_SESSION_MODE.
\item The driver MUST set the \field{opcode} field in struct virtio_crypto_op_header
    to VIRTIO_CRYPTO_CIPHER_ENCRYPT or VIRTIO_CRYPTO_CIPHER_DECRYPT.
\item The driver MUST specify the fields of struct virtio_crypto_cipher_data_flf in
    struct virtio_crypto_sym_data_flf and struct virtio_crypto_cipher_data_vlf in
    struct virtio_crypto_sym_data_vlf if the request is based on VIRTIO_CRYPTO_SYM_OP_CIPHER.
\item The driver MUST specify the fields of struct virtio_crypto_alg_chain_data_flf
    in struct virtio_crypto_sym_data_flf and struct virtio_crypto_alg_chain_data_vlf
    in struct virtio_crypto_sym_data_vlf if the request is of the VIRTIO_CRYPTO_SYM_OP_ALGORITHM_CHAINING
    type.
\end{itemize*}

\devicenormative{\paragraph}{Symmetric algorithms Operation}{Device Types / Crypto Device / Device Operation / Symmetric algorithms Operation}

\begin{itemize*}
\item If the VIRTIO_CRYPTO_F_CIPHER_STATELESS_MODE feature bit is negotiated, the device
    MUST parse \field{flag} field in struct virtio_crypto_op_header in order to decide
	which mode the driver uses.
\item The device MUST parse the virtio_crypto_sym_data_req based on the \field{opcode}
    field in general header.
\item The device MUST parse the fields of struct virtio_crypto_cipher_data_flf in
    struct virtio_crypto_sym_data_flf and struct virtio_crypto_cipher_data_vlf in
    struct virtio_crypto_sym_data_vlf if the request is based on VIRTIO_CRYPTO_SYM_OP_CIPHER.
\item The device MUST parse the fields of struct virtio_crypto_alg_chain_data_flf
    in struct virtio_crypto_sym_data_flf and struct virtio_crypto_alg_chain_data_vlf
    in struct virtio_crypto_sym_data_vlf if the request is of the VIRTIO_CRYPTO_SYM_OP_ALGORITHM_CHAINING
    type.
\item The device MUST copy the result of cryptographic operation in the dst_data[] in
    both plain CIPHER mode and algorithms chain mode.
\item The device MUST check the \field{para}.\field{add_len} is bigger than 0 before
    parse the additional authenticated data in plain algorithms chain mode.
\item The device MUST copy the result of HASH/MAC operation in the hash_result[] is
    of the VIRTIO_CRYPTO_SYM_OP_ALGORITHM_CHAINING type.
\item The device MUST set the \field{status} field in struct virtio_crypto_inhdr to
    one of the following values of enum VIRTIO_CRYPTO_STATUS:
\begin{itemize*}
\item VIRTIO_CRYPTO_OK if the operation success.
\item VIRTIO_CRYPTO_NOTSUPP if the requested algorithm or operation is unsupported.
\item VIRTIO_CRYPTO_INVSESS if the session ID is invalid in session mode.
\item VIRTIO_CRYPTO_ERR if failure not mentioned above occurs.
\end{itemize*}
\end{itemize*}

\subsubsection{AEAD Service Operation}\label{sec:Device Types / Crypto Device / Device Operation / AEAD Service Operation}

Session mode requests of symmetric algorithm are as follows:

\begin{lstlisting}
struct virtio_crypto_aead_data_flf {
    /*
     * Byte Length of valid IV data.
     *
     * For GCM mode, this is either 12 (for 96-bit IVs) or 16, in which
     *   case iv points to J0.
     * For CCM mode, this is the length of the nonce, which can be in the
     *   range 7 to 13 inclusive.
     */
    le32 iv_len;
    /* length of additional auth data */
    le32 aad_len;
    /* length of source data */
    le32 src_data_len;
    /* length of dst data, this should be at least src_data_len + tag_len */
    le32 dst_data_len;
    /* Authentication tag length */
    le32 tag_len;
    le32 reserved;
};

struct virtio_crypto_aead_data_vlf {
    /* Device read only portion */

    /*
     * Initialization Vector data.
     *
     * For GCM mode, this is either the IV (if the length is 96 bits) or J0
     *   (for other sizes), where J0 is as defined by NIST SP800-38D.
     *   Regardless of the IV length, a full 16 bytes needs to be allocated.
     * For CCM mode, the first byte is reserved, and the nonce should be
     *   written starting at &iv[1] (to allow space for the implementation
     *   to write in the flags in the first byte).  Note that a full 16 bytes
     *   should be allocated, even though the iv_len field will have
     *   a value less than this.
     *
     * The IV will be updated after every partial cryptographic operation.
     */
    u8 iv[iv_len];
    /* Source data */
    u8 src_data[src_data_len];
    /* Additional authenticated data if exists */
    u8 aad[aad_len];

    /* Device write only portion */
    /* Pointer to output data */
    u8 dst_data[dst_data_len];
};
\end{lstlisting}

Each request uses the virtio_crypto_aead_data_flf structure and the
virtio_crypto_aead_data_flf structure to store information used to run the
AEAD operations.

Stateless mode AEAD service requests are as follows:

\begin{lstlisting}
struct virtio_crypto_aead_data_flf_stateless {
    struct {
        /* See VIRTIO_CRYPTO_AEAD_* above */
        le32 algo;
        /* length of key */
        le32 key_len;
        /* encrypt or decrypt, See above VIRTIO_CRYPTO_OP_* */
        le32 op;
    } sess_para;

    /* Byte Length of valid IV data. */
    le32 iv_len;
    /* Authentication tag length */
    le32 tag_len;
    /* length of additional auth data */
    le32 aad_len;
    /* length of source data */
    le32 src_data_len;
    /* length of dst data, this should be at least src_data_len + tag_len */
    le32 dst_data_len;
};

struct virtio_crypto_aead_data_vlf_stateless {
    /* Device read only portion */

    /* The cipher key */
    u8 key[key_len];
    /* Initialization Vector data. */
    u8 iv[iv_len];
    /* Source data */
    u8 src_data[src_data_len];
    /* Additional authenticated data if exists */
    u8 aad[aad_len];

    /* Device write only portion */
    /* Pointer to output data */
    u8 dst_data[dst_data_len];
};
\end{lstlisting}

\drivernormative{\paragraph}{AEAD Service Operation}{Device Types / Crypto Device / Device Operation / AEAD Service Operation}

\begin{itemize*}
\item If the driver uses the session mode, then the driver MUST set
    \field{session_id} in struct virtio_crypto_op_header to a valid value assigned
    by the device when the session was created.
\item If the VIRTIO_CRYPTO_F_AEAD_STATELESS_MODE feature bit is negotiated, 1) if
    the driver uses the stateless mode, then the driver MUST set the \field{flag}
    field in struct virtio_crypto_op_header to ZERO and MUST set the fields in
    struct virtio_crypto_aead_data_flf_stateless.sess_para, 2) if the driver uses
    the session mode, then the driver MUST set the \field{flag} field in struct
    virtio_crypto_op_header to VIRTIO_CRYPTO_FLAG_SESSION_MODE.
\item The driver MUST set the \field{opcode} field in struct virtio_crypto_op_header
    to VIRTIO_CRYPTO_AEAD_ENCRYPT or VIRTIO_CRYPTO_AEAD_DECRYPT.
\end{itemize*}

\devicenormative{\paragraph}{AEAD Service Operation}{Device Types / Crypto Device / Device Operation / AEAD Service Operation}

\begin{itemize*}
\item If the VIRTIO_CRYPTO_F_AEAD_STATELESS_MODE feature bit is negotiated, the
    device MUST parse the virtio_crypto_aead_data_vlf_stateless based on the \field{opcode}
	field in general header.
\item The device MUST copy the result of cryptographic operation in the dst_data[].
\item The device MUST copy the authentication tag in the dst_data[] offset the cipher result.
\item The device MUST set the \field{status} field in struct virtio_crypto_inhdr to
    one of the following values of enum VIRTIO_CRYPTO_STATUS:
\item When the \field{opcode} field is VIRTIO_CRYPTO_AEAD_DECRYPT, the device MUST
    verify and return the verification result to the driver.
\begin{itemize*}
\item VIRTIO_CRYPTO_OK if the operation success.
\item VIRTIO_CRYPTO_NOTSUPP if the requested algorithm or operation is unsupported.
\item VIRTIO_CRYPTO_BADMSG if the verification result is incorrect.
\item VIRTIO_CRYPTO_INVSESS if the session ID invalid when in session mode.
\item VIRTIO_CRYPTO_ERR if any failure not mentioned above occurs.
\end{itemize*}
\end{itemize*}

\subsubsection{AKCIPHER Service Operation}\label{sec:Device Types / Crypto Device / Device Operation / AKCIPHER Service Operation}

Session mode AKCIPHER requests are as follows:

\begin{lstlisting}
struct virtio_crypto_akcipher_data_flf {
    /* length of source data */
    le32 src_data_len;
    /* length of dst data */
    le32 dst_data_len;
};

struct virtio_crypto_akcipher_data_vlf {
    /* Device read only portion */
    /* Source data */
    u8 src_data[src_data_len];

    /* Device write only portion */
    /* Pointer to output data */
    u8 dst_data[dst_data_len];
};
\end{lstlisting}

Each data request uses the virtio_crypto_akcipher_flf structure and the virtio_crypto_akcipher_data_vlf
structure to store information used to run the AKCIPHER operations.

For encryption, decryption, and signing:
\field{src_data} is the source data that will be processed, note that for signing operations,
src_data stores the data to be signed, which usually is the digest of some data rather than the
data itself.
\field{src_data_len} is the length of source data.
\field{dst_result} is the result data and \field{dst_data_len} is the length of it. Note that the
length of the result is not always exactly equal to dst_data_len, the driver needs to check how
many bytes the device has written and calculate the actual length of the result.

For verification:
\field{src_data_len} refers to the length of the signature, and \field{dst_data_len} refers to
the length of signed data, where the signed data is usually the digest of some data.
\field{src_data} is spliced by the signature and the signed data, the src_data with the lower
address stores the signature, and the higher address stores the signed data.
\field{dst_data} is always empty for verification.

Different algorithms have different signature formats.
For the RSA algorithm, the result is determined by the padding algorithm specified by
\field{padding_algo} in structure virtio_crypto_rsa_session_para.

For the ECDSA algorithm, the signature is composed of the following
ASN.1 structure (see \hyperref[intro:rfc3279]{RFC3279})
and MUST be DER encoded (see \hyperref[intro:rfc6025]{rfc6025}).

\begin{lstlisting}
Ecdsa-Sig-Value ::= SEQUENCE {
    r INTEGER,
    s INTEGER
}
\end{lstlisting}

Stateless mode AKCIPHER service requests are as follows:
\begin{lstlisting}
struct virtio_crypto_akcipher_data_flf_stateless {
    struct {
        /* See VIRTIO_CYRPTO_AKCIPHER* above */
        le32 algo;
        /* See VIRTIO_CRYPTO_AKCIPHER_KEY_TYPE_* above */
        le32 key_type;
        /* length of key */
        le32 key_len;

        /* algothrim specific parameters described above */
        union {
            struct virtio_crypto_rsa_session_para rsa;
            struct virtio_crypto_ecdsa_session_para ecdsa;
        } u;
    } sess_para;

    /* length of source data */
    le32 src_data_len;
    /* length of destination data */
    le32 dst_data_len;
};

struct virtio_crypto_akcipher_data_vlf_stateless {
    /* Device read only portion */
    u8 akcipher_key[key_len];

    /* Source data */
    u8 src_data[src_data_len];

    /* Device write only portion */
    u8 dst_data[dst_data_len];
};
\end{lstlisting}

In stateless mode, the format of key and signature, the meaning of src_data and dst_data, are all the same
with session mode.

\drivernormative{\paragraph}{AKCIPHER Service Operation}{Device Types / Crypto Device / Device Operation / AKCIPHER Service Operation}

\begin{itemize*}
\item If the driver uses the session mode, then the driver MUST set
    \field{session_id} in struct virtio_crypto_op_header to a valid
    value assigned by the device when the session was created.
\item If the VIRTIO_CRYPTO_F_AKCIPHER_STATELESS_MODE feature bit is negotiated, 1) if the
    driver uses the stateless mode, then the driver MUST set the \field{flag} field in
    struct virtio_crypto_op_header to ZERO and MUST set the fields in struct
    virtio_crypto_akcipher_flf_stateless.sess_para, 2) if the driver uses the session
    mode, then the driver MUST set the \field{flag} field in struct virtio_crypto_op_header
    to VIRTIO_CRYPTO_FLAG_SESSION_MODE.
\item The driver MUST set the \field{opcode} field in struct virtio_crypto_op_header
    to one of VIRTIO_CRYPTO_AKCIPHER_ENCRYPT, VIRTIO_CRYPTO_AKCIPHER_DESTROY_SESSION,
    VIRTIO_CRYPTO_AKCIPHER_SIGN, and VIRTIO_CRYPTO_AKCIPHER_VERIFY.
\end{itemize*}

\devicenormative{\paragraph}{AKCIPHER Service Operation}{Device Types / Crypto Device / Device Operation / AKCIPHER Service Operation}

\begin{itemize*}
\item If the VIRTIO_CRYPTO_F_AKCIPHER_STATELESS_MODE feature bit is negotiated, the
    device MUST parse the virtio_crypto_akcipher_data_vlf_stateless based on the \field{opcode}
    field in general header.
\item The device MUST copy the result of cryptographic operation in the dst_data[].
\item The device MUST set the \field{status} field in struct virtio_crypto_inhdr to
    one of the following values of enum VIRTIO_CRYPTO_STATUS:
\begin{itemize*}
\item VIRTIO_CRYPTO_OK if the operation success.
\item VIRTIO_CRYPTO_NOTSUPP if the requested algorithm or operation is unsupported.
\item VIRTIO_CRYPTO_BADMSG if the verification result is incorrect.
\item VIRTIO_CRYPTO_INVSESS if the session ID invalid when in session mode.
\item VIRTIO_CRYPTO_KEY_REJECTED if the signature verification failed.
\item VIRTIO_CRYPTO_ERR if any failure not mentioned above occurs.
\end{itemize*}
\end{itemize*}

\section{Crypto Device}\label{sec:Device Types / Crypto Device}

The virtio crypto device is a virtual cryptography device as well as a
virtual cryptographic accelerator. The virtio crypto device provides the
following crypto services: CIPHER, MAC, HASH, AEAD and AKCIPHER. Virtio crypto
devices have a single control queue and at least one data queue. Crypto
operation requests are placed into a data queue, and serviced by the
device. Some crypto operation requests are only valid in the context of a
session. The role of the control queue is facilitating control operation
requests. Sessions management is realized with control operation
requests.

\subsection{Device ID}\label{sec:Device Types / Crypto Device / Device ID}

20

\subsection{Virtqueues}\label{sec:Device Types / Crypto Device / Virtqueues}

\begin{description}
\item[0] dataq1
\item[\ldots]
\item[N-1] dataqN
\item[N] controlq
\end{description}

N is set by \field{max_dataqueues}.

\subsection{Feature bits}\label{sec:Device Types / Crypto Device / Feature bits}

\begin{description}
\item VIRTIO_CRYPTO_F_REVISION_1 (0) revision 1. Revision 1 has a specific
    request format and other enhancements (which result in some additional
    requirements).
\item VIRTIO_CRYPTO_F_CIPHER_STATELESS_MODE (1) stateless mode requests are
    supported by the CIPHER service.
\item VIRTIO_CRYPTO_F_HASH_STATELESS_MODE (2) stateless mode requests are
    supported by the HASH service.
\item VIRTIO_CRYPTO_F_MAC_STATELESS_MODE (3) stateless mode requests are
    supported by the MAC service.
\item VIRTIO_CRYPTO_F_AEAD_STATELESS_MODE (4) stateless mode requests are
    supported by the AEAD service.
\item VIRTIO_CRYPTO_F_AKCIPHER_STATELESS_MODE (5) stateless mode requests are
    supported by the AKCIPHER service.
\end{description}


\subsubsection{Feature bit requirements}\label{sec:Device Types / Crypto Device / Feature bit requirements}

Some crypto feature bits require other crypto feature bits
(see \ref{drivernormative:Basic Facilities of a Virtio Device / Feature Bits}):

\begin{description}
\item[VIRTIO_CRYPTO_F_CIPHER_STATELESS_MODE] Requires VIRTIO_CRYPTO_F_REVISION_1.
\item[VIRTIO_CRYPTO_F_HASH_STATELESS_MODE] Requires VIRTIO_CRYPTO_F_REVISION_1.
\item[VIRTIO_CRYPTO_F_MAC_STATELESS_MODE] Requires VIRTIO_CRYPTO_F_REVISION_1.
\item[VIRTIO_CRYPTO_F_AEAD_STATELESS_MODE] Requires VIRTIO_CRYPTO_F_REVISION_1.
\item[VIRTIO_CRYPTO_F_AKCIPHER_STATELESS_MODE] Requires VIRTIO_CRYPTO_F_REVISION_1.
\end{description}

\subsection{Supported crypto services}\label{sec:Device Types / Crypto Device / Supported crypto services}

The following crypto services are defined:

\begin{lstlisting}
/* CIPHER (Symmetric Key Cipher) service */
#define VIRTIO_CRYPTO_SERVICE_CIPHER 0
/* HASH service */
#define VIRTIO_CRYPTO_SERVICE_HASH   1
/* MAC (Message Authentication Codes) service */
#define VIRTIO_CRYPTO_SERVICE_MAC    2
/* AEAD (Authenticated Encryption with Associated Data) service */
#define VIRTIO_CRYPTO_SERVICE_AEAD   3
/* AKCIPHER (Asymmetric Key Cipher) service */
#define VIRTIO_CRYPTO_SERVICE_AKCIPHER 4
\end{lstlisting}

The above constants designate bits used to indicate the which of crypto services are
offered by the device as described in, see \ref{sec:Device Types / Crypto Device / Device configuration layout}.

\subsubsection{CIPHER services}\label{sec:Device Types / Crypto Device / Supported crypto services / CIPHER services}

The following CIPHER algorithms are defined:

\begin{lstlisting}
#define VIRTIO_CRYPTO_NO_CIPHER                 0
#define VIRTIO_CRYPTO_CIPHER_ARC4               1
#define VIRTIO_CRYPTO_CIPHER_AES_ECB            2
#define VIRTIO_CRYPTO_CIPHER_AES_CBC            3
#define VIRTIO_CRYPTO_CIPHER_AES_CTR            4
#define VIRTIO_CRYPTO_CIPHER_DES_ECB            5
#define VIRTIO_CRYPTO_CIPHER_DES_CBC            6
#define VIRTIO_CRYPTO_CIPHER_3DES_ECB           7
#define VIRTIO_CRYPTO_CIPHER_3DES_CBC           8
#define VIRTIO_CRYPTO_CIPHER_3DES_CTR           9
#define VIRTIO_CRYPTO_CIPHER_KASUMI_F8          10
#define VIRTIO_CRYPTO_CIPHER_SNOW3G_UEA2        11
#define VIRTIO_CRYPTO_CIPHER_AES_F8             12
#define VIRTIO_CRYPTO_CIPHER_AES_XTS            13
#define VIRTIO_CRYPTO_CIPHER_ZUC_EEA3           14
\end{lstlisting}

The above constants have two usages:
\begin{enumerate}
\item As bit numbers, used to tell the driver which CIPHER algorithms
are supported by the device, see \ref{sec:Device Types / Crypto Device / Device configuration layout}.
\item As values, used to designate the algorithm in (CIPHER type) crypto
operation requests, see \ref{sec:Device Types / Crypto Device / Device Operation / Control Virtqueue / Session operation}.
\end{enumerate}

\subsubsection{HASH services}\label{sec:Device Types / Crypto Device / Supported crypto services / HASH services}

The following HASH algorithms are defined:

\begin{lstlisting}
#define VIRTIO_CRYPTO_NO_HASH            0
#define VIRTIO_CRYPTO_HASH_MD5           1
#define VIRTIO_CRYPTO_HASH_SHA1          2
#define VIRTIO_CRYPTO_HASH_SHA_224       3
#define VIRTIO_CRYPTO_HASH_SHA_256       4
#define VIRTIO_CRYPTO_HASH_SHA_384       5
#define VIRTIO_CRYPTO_HASH_SHA_512       6
#define VIRTIO_CRYPTO_HASH_SHA3_224      7
#define VIRTIO_CRYPTO_HASH_SHA3_256      8
#define VIRTIO_CRYPTO_HASH_SHA3_384      9
#define VIRTIO_CRYPTO_HASH_SHA3_512      10
#define VIRTIO_CRYPTO_HASH_SHA3_SHAKE128      11
#define VIRTIO_CRYPTO_HASH_SHA3_SHAKE256      12
\end{lstlisting}

The above constants have two usages:
\begin{enumerate}
\item As bit numbers, used to tell the driver which HASH algorithms
are supported by the device, see \ref{sec:Device Types / Crypto Device / Device configuration layout}.
\item As values, used to designate the algorithm in (HASH type) crypto
operation requires, see \ref{sec:Device Types / Crypto Device / Device Operation / Control Virtqueue / Session operation}.
\end{enumerate}

\subsubsection{MAC services}\label{sec:Device Types / Crypto Device / Supported crypto services / MAC services}

The following MAC algorithms are defined:

\begin{lstlisting}
#define VIRTIO_CRYPTO_NO_MAC                       0
#define VIRTIO_CRYPTO_MAC_HMAC_MD5                 1
#define VIRTIO_CRYPTO_MAC_HMAC_SHA1                2
#define VIRTIO_CRYPTO_MAC_HMAC_SHA_224             3
#define VIRTIO_CRYPTO_MAC_HMAC_SHA_256             4
#define VIRTIO_CRYPTO_MAC_HMAC_SHA_384             5
#define VIRTIO_CRYPTO_MAC_HMAC_SHA_512             6
#define VIRTIO_CRYPTO_MAC_CMAC_3DES                25
#define VIRTIO_CRYPTO_MAC_CMAC_AES                 26
#define VIRTIO_CRYPTO_MAC_KASUMI_F9                27
#define VIRTIO_CRYPTO_MAC_SNOW3G_UIA2              28
#define VIRTIO_CRYPTO_MAC_GMAC_AES                 41
#define VIRTIO_CRYPTO_MAC_GMAC_TWOFISH             42
#define VIRTIO_CRYPTO_MAC_CBCMAC_AES               49
#define VIRTIO_CRYPTO_MAC_CBCMAC_KASUMI_F9         50
#define VIRTIO_CRYPTO_MAC_XCBC_AES                 53
#define VIRTIO_CRYPTO_MAC_ZUC_EIA3                 54
\end{lstlisting}

The above constants have two usages:
\begin{enumerate}
\item As bit numbers, used to tell the driver which MAC algorithms
are supported by the device, see \ref{sec:Device Types / Crypto Device / Device configuration layout}.
\item As values, used to designate the algorithm in (MAC type) crypto
operation requests, see \ref{sec:Device Types / Crypto Device / Device Operation / Control Virtqueue / Session operation}.
\end{enumerate}

\subsubsection{AEAD services}\label{sec:Device Types / Crypto Device / Supported crypto services / AEAD services}

The following AEAD algorithms are defined:

\begin{lstlisting}
#define VIRTIO_CRYPTO_NO_AEAD     0
#define VIRTIO_CRYPTO_AEAD_GCM    1
#define VIRTIO_CRYPTO_AEAD_CCM    2
#define VIRTIO_CRYPTO_AEAD_CHACHA20_POLY1305  3
\end{lstlisting}

The above constants have two usages:
\begin{enumerate}
\item As bit numbers, used to tell the driver which AEAD algorithms
are supported by the device, see \ref{sec:Device Types / Crypto Device / Device configuration layout}.
\item As values, used to designate the algorithm in (DEAD type) crypto
operation requests, see \ref{sec:Device Types / Crypto Device / Device Operation / Control Virtqueue / Session operation}.
\end{enumerate}

\subsubsection{AKCIPHER services}\label{sec: Device Types / Crypto Device / Supported crypto services / AKCIPHER services}

The following AKCIPHER algorithms are defined:
\begin{lstlisting}
#define VIRTIO_CRYPTO_NO_AKCIPHER 0
#define VIRTIO_CRYPTO_AKCIPHER_RSA   1
#define VIRTIO_CRYPTO_AKCIPHER_ECDSA 2
\end{lstlisting}

The above constants have two usages:
\begin{enumerate}
\item As bit numbers, used to tell the driver which AKCIPHER algorithms
are supported by the device, see \ref{sec:Device Types / Crypto Device / Device configuration layout}.
\item As values, used to designate the algorithm in asymmetric crypto operation requests,
see \ref{sec:Device Types / Crypto Device / Device Operation / Control Virtqueue / Session operation}.
\end{enumerate}


\subsection{Device configuration layout}\label{sec:Device Types / Crypto Device / Device configuration layout}

Crypto device configuration uses the following layout structure:

\begin{lstlisting}
struct virtio_crypto_config {
    le32 status;
    le32 max_dataqueues;
    le32 crypto_services;
    /* Detailed algorithms mask */
    le32 cipher_algo_l;
    le32 cipher_algo_h;
    le32 hash_algo;
    le32 mac_algo_l;
    le32 mac_algo_h;
    le32 aead_algo;
    /* Maximum length of cipher key in bytes */
    le32 max_cipher_key_len;
    /* Maximum length of authenticated key in bytes */
    le32 max_auth_key_len;
    le32 akcipher_algo;
    /* Maximum size of each crypto request's content in bytes */
    le64 max_size;
};
\end{lstlisting}

\begin{description}
\item Currently, only one \field{status} bit is defined: VIRTIO_CRYPTO_S_HW_READY
    set indicates that the device is ready to process requests, this bit is read-only
    for the driver
\begin{lstlisting}
#define VIRTIO_CRYPTO_S_HW_READY  (1 << 0)
\end{lstlisting}

\item [\field{max_dataqueues}] is the maximum number of data virtqueues that can
    be configured by the device. The driver MAY use only one data queue, or it
    can use more to achieve better performance.

\item [\field{crypto_services}] crypto service offered, see \ref{sec:Device Types / Crypto Device / Supported crypto services}.

\item [\field{cipher_algo_l}] CIPHER algorithms bits 0-31, see \ref{sec:Device Types / Crypto Device / Supported crypto services  / CIPHER services}.

\item [\field{cipher_algo_h}] CIPHER algorithms bits 32-63, see \ref{sec:Device Types / Crypto Device / Supported crypto services  / CIPHER services}.

\item [\field{hash_algo}] HASH algorithms bits, see \ref{sec:Device Types / Crypto Device / Supported crypto services  / HASH services}.

\item [\field{mac_algo_l}] MAC algorithms bits 0-31, see \ref{sec:Device Types / Crypto Device / Supported crypto services  / MAC services}.

\item [\field{mac_algo_h}] MAC algorithms bits 32-63, see \ref{sec:Device Types / Crypto Device / Supported crypto services  / MAC services}.

\item [\field{aead_algo}] AEAD algorithms bits, see \ref{sec:Device Types / Crypto Device / Supported crypto services  / AEAD services}.

\item [\field{max_cipher_key_len}] is the maximum length of cipher key supported by the device.

\item [\field{max_auth_key_len}] is the maximum length of authenticated key supported by the device.

\item [\field{akcipher_algo}] AKCIPHER algorithms bit 0-31, see \ref{sec: Device Types / Crypto Device / Supported crypto services / AKCIPHER services}.

\item [\field{max_size}] is the maximum size of the variable-length parameters of
    data operation of each crypto request's content supported by the device.
\end{description}

\begin{note}
Unless explicitly stated otherwise all lengths and sizes are in bytes.
\end{note}

\devicenormative{\subsubsection}{Device configuration layout}{Device Types / Crypto Device / Device configuration layout}

\begin{itemize*}
\item The device MUST set \field{max_dataqueues} to between 1 and 65535 inclusive.
\item The device MUST set the \field{status} with valid flags, undefined flags MUST NOT be set.
\item The device MUST accept and handle requests after \field{status} is set to VIRTIO_CRYPTO_S_HW_READY.
\item The device MUST set \field{crypto_services} based on the crypto services the device offers.
\item The device MUST set detailed algorithms masks for each service advertised by \field{crypto_services}.
    The device MUST NOT set the not defined algorithms bits.
\item The device MUST set \field{max_size} to show the maximum size of crypto request the device supports.
\item The device MUST set \field{max_cipher_key_len} to show the maximum length of cipher key if the
    device supports CIPHER service.
\item The device MUST set \field{max_auth_key_len} to show the maximum length of authenticated key if
    the device supports MAC service.
\end{itemize*}

\drivernormative{\subsubsection}{Device configuration layout}{Device Types / Crypto Device / Device configuration layout}

\begin{itemize*}
\item The driver MUST read the \field{status} from the bottom bit of status to check whether the
    VIRTIO_CRYPTO_S_HW_READY is set, and the driver MUST reread it after device reset.
\item The driver MUST NOT transmit any requests to the device if the VIRTIO_CRYPTO_S_HW_READY is not set.
\item The driver MUST read \field{max_dataqueues} field to discover the number of data queues the device supports.
\item The driver MUST read \field{crypto_services} field to discover which services the device is able to offer.
\item The driver SHOULD ignore the not defined algorithms bits.
\item The driver MUST read the detailed algorithms fields based on \field{crypto_services} field.
\item The driver SHOULD read \field{max_size} to discover the maximum size of the variable-length
    parameters of data operation of the crypto request's content the device supports and MUST
    guarantee that the size of each crypto request's content is within the \field{max_size}, otherwise
    the request will fail and the driver MUST reset the device.
\item The driver SHOULD read \field{max_cipher_key_len} to discover the maximum length of cipher key
    the device supports and MUST guarantee that the \field{key_len} (CIPHER service or AEAD service) is within
    the \field{max_cipher_key_len} of the device configuration, otherwise the request will fail.
\item The driver SHOULD read \field{max_auth_key_len} to discover the maximum length of authenticated
    key the device supports and MUST guarantee that the \field{auth_key_len} (MAC service) is within the
    \field{max_auth_key_len} of the device configuration, otherwise the request will fail.
\end{itemize*}

\subsection{Device Initialization}\label{sec:Device Types / Crypto Device / Device Initialization}

\drivernormative{\subsubsection}{Device Initialization}{Device Types / Crypto Device / Device Initialization}

\begin{itemize*}
\item The driver MUST configure and initialize all virtqueues.
\item The driver MUST read the supported crypto services from bits of \field{crypto_services}.
\item The driver MUST read the supported algorithms based on \field{crypto_services} field.
\end{itemize*}

\subsection{Device Operation}\label{sec:Device Types / Crypto Device / Device Operation}

The operation of a virtio crypto device is driven by requests placed on the virtqueues.
Requests consist of a queue-type specific header (specifying among others the operation)
and an operation specific payload.

If VIRTIO_CRYPTO_F_REVISION_1 is negotiated the device may support both session mode
(See \ref{sec:Device Types / Crypto Device / Device Operation / Control Virtqueue / Session operation})
and stateless mode operation requests.
In stateless mode all operation parameters are supplied as a part of each request,
while in session mode, some or all operation parameters are managed within the
session. Stateless mode is guarded by feature bits 0-4 on a service level. If
stateless mode is negotiated for a service, the service accepts both session
mode and stateless requests; otherwise stateless mode requests are rejected
(via operation status).

\subsubsection{Operation Status}\label{sec:Device Types / Crypto Device / Device Operation / Operation status}
The device MUST return a status code as part of the operation (both session
operation and service operation) result. The valid operation status as follows:

\begin{lstlisting}
enum VIRTIO_CRYPTO_STATUS {
    VIRTIO_CRYPTO_OK = 0,
    VIRTIO_CRYPTO_ERR = 1,
    VIRTIO_CRYPTO_BADMSG = 2,
    VIRTIO_CRYPTO_NOTSUPP = 3,
    VIRTIO_CRYPTO_INVSESS = 4,
    VIRTIO_CRYPTO_NOSPC = 5,
    VIRTIO_CRYPTO_KEY_REJECTED = 6,
    VIRTIO_CRYPTO_MAX
};
\end{lstlisting}

\begin{itemize*}
\item VIRTIO_CRYPTO_OK: success.
\item VIRTIO_CRYPTO_BADMSG: authentication failed (only when AEAD decryption).
\item VIRTIO_CRYPTO_NOTSUPP: operation or algorithm is unsupported.
\item VIRTIO_CRYPTO_INVSESS: invalid session ID when executing crypto operations.
\item VIRTIO_CRYPTO_NOSPC: no free session ID (only when the VIRTIO_CRYPTO_F_REVISION_1
    feature bit is negotiated).
\item VIRTIO_CRYPTO_KEY_REJECTED: signature verification failed (only when AKCIPHER verification).
\item VIRTIO_CRYPTO_ERR: any failure not mentioned above occurs.
\end{itemize*}

\subsubsection{Control Virtqueue}\label{sec:Device Types / Crypto Device / Device Operation / Control Virtqueue}

The driver uses the control virtqueue to send control commands to the
device, such as session operations (See \ref{sec:Device Types / Crypto Device / Device
Operation / Control Virtqueue / Session operation}).

The header for controlq is of the following form:
\begin{lstlisting}
#define VIRTIO_CRYPTO_OPCODE(service, op)   (((service) << 8) | (op))

struct virtio_crypto_ctrl_header {
#define VIRTIO_CRYPTO_CIPHER_CREATE_SESSION \
       VIRTIO_CRYPTO_OPCODE(VIRTIO_CRYPTO_SERVICE_CIPHER, 0x02)
#define VIRTIO_CRYPTO_CIPHER_DESTROY_SESSION \
       VIRTIO_CRYPTO_OPCODE(VIRTIO_CRYPTO_SERVICE_CIPHER, 0x03)
#define VIRTIO_CRYPTO_HASH_CREATE_SESSION \
       VIRTIO_CRYPTO_OPCODE(VIRTIO_CRYPTO_SERVICE_HASH, 0x02)
#define VIRTIO_CRYPTO_HASH_DESTROY_SESSION \
       VIRTIO_CRYPTO_OPCODE(VIRTIO_CRYPTO_SERVICE_HASH, 0x03)
#define VIRTIO_CRYPTO_MAC_CREATE_SESSION \
       VIRTIO_CRYPTO_OPCODE(VIRTIO_CRYPTO_SERVICE_MAC, 0x02)
#define VIRTIO_CRYPTO_MAC_DESTROY_SESSION \
       VIRTIO_CRYPTO_OPCODE(VIRTIO_CRYPTO_SERVICE_MAC, 0x03)
#define VIRTIO_CRYPTO_AEAD_CREATE_SESSION \
       VIRTIO_CRYPTO_OPCODE(VIRTIO_CRYPTO_SERVICE_AEAD, 0x02)
#define VIRTIO_CRYPTO_AEAD_DESTROY_SESSION \
       VIRTIO_CRYPTO_OPCODE(VIRTIO_CRYPTO_SERVICE_AEAD, 0x03)
#define VIRTIO_CRYPTO_AKCIPHER_CREATE_SESSION \
       VIRTIO_CRYPTO_OPCODE(VIRTIO_CRYPTO_SERVICE_AKCIPHER, 0x04)
#define VIRTIO_CRYPTO_AKCIPHER_DESTROY_SESSION \
       VIRTIO_CRYPTO_OPCDE(VIRTIO_CRYPTO_SERVICE_AKCIPHER, 0x05)
    le32 opcode;
    /* algo should be service-specific algorithms */
    le32 algo;
    le32 flag;
    le32 reserved;
};
\end{lstlisting}

The controlq request is composed of four parts:
\begin{lstlisting}
struct virtio_crypto_op_ctrl_req {
    /* Device read only portion */

    struct virtio_crypto_ctrl_header header;

#define VIRTIO_CRYPTO_CTRLQ_OP_SPEC_HDR_LEGACY 56
    /* fixed length fields, opcode specific */
    u8 op_flf[flf_len];

    /* variable length fields, opcode specific */
    u8 op_vlf[vlf_len];

    /* Device write only portion */

    /* op result or completion status */
    u8 op_outcome[outcome_len];
};
\end{lstlisting}

\field{header} is a general header (see above).

\field{op_flf} is the opcode (in \field{header}) specific fixed-length parameters.

\field{flf_len} depends on the VIRTIO_CRYPTO_F_REVISION_1 feature bit (see below).

\field{op_vlf} is the opcode (in \field{header}) specific variable-length parameters.

\field{vlf_len} is the size of the specific structure used.
\begin{note}
The \field{vlf_len} of session-destroy operation and the hash-session-create
operation is ZERO.
\end{note}

\begin{itemize*}
\item If the opcode (in \field{header}) is VIRTIO_CRYPTO_CIPHER_CREATE_SESSION
    then \field{op_flf} is struct virtio_crypto_sym_create_session_flf if
    VIRTIO_CRYPTO_F_REVISION_1 is negotiated and struct virtio_crypto_sym_create_session_flf is
    padded to 56 bytes if NOT negotiated, and \field{op_vlf} is struct
    virtio_crypto_sym_create_session_vlf.
\item If the opcode (in \field{header}) is VIRTIO_CRYPTO_HASH_CREATE_SESSION
    then \field{op_flf} is struct virtio_crypto_hash_create_session_flf if
    VIRTIO_CRYPTO_F_REVISION_1 is negotiated and struct virtio_crypto_hash_create_session_flf is
    padded to 56 bytes if NOT negotiated.
\item If the opcode (in \field{header}) is VIRTIO_CRYPTO_MAC_CREATE_SESSION
    then \field{op_flf} is struct virtio_crypto_mac_create_session_flf if
    VIRTIO_CRYPTO_F_REVISION_1 is negotiated and struct virtio_crypto_mac_create_session_flf is
    padded to 56 bytes if NOT negotiated, and \field{op_vlf} is struct
    virtio_crypto_mac_create_session_vlf.
\item If the opcode (in \field{header}) is VIRTIO_CRYPTO_AEAD_CREATE_SESSION
    then \field{op_flf} is struct virtio_crypto_aead_create_session_flf if
    VIRTIO_CRYPTO_F_REVISION_1 is negotiated and struct virtio_crypto_aead_create_session_flf is
    padded to 56 bytes if NOT negotiated, and \field{op_vlf} is struct
    virtio_crypto_aead_create_session_vlf.
\item If the opcode (in \field{header}) is VIRTIO_CRYPTO_AKCIPHER_CREATE_SESSION
    then \field{op_flf} is struct virtio_crypto_akcipher_create_session_flf if
    VIRTIO_CRYPTO_F_REVISION_1 is negotiated and struct virtio_crypto_akcipher_create_session_flf is
    padded to 56 bytes if NOT negotiated, and \field{op_vlf} is struct
    virtio_crypto_akcipher_create_session_vlf.
\item If the opcode (in \field{header}) is VIRTIO_CRYPTO_CIPHER_DESTROY_SESSION
    or VIRTIO_CRYPTO_HASH_DESTROY_SESSION or VIRTIO_CRYPTO_MAC_DESTROY_SESSION or
    VIRTIO_CRYPTO_AEAD_DESTROY_SESSION then \field{op_flf} is struct
    virtio_crypto_destroy_session_flf if VIRTIO_CRYPTO_F_REVISION_1 is negotiated and
    struct virtio_crypto_destroy_session_flf is padded to 56 bytes if NOT negotiated.
\end{itemize*}

\field{op_outcome} stores the result of operation and must be struct
virtio_crypto_destroy_session_input for destroy session or
struct virtio_crypto_create_session_input for create session.

\field{outcome_len} is the size of the structure used.


\paragraph{Session operation}\label{sec:Device Types / Crypto Device / Device
Operation / Control Virtqueue / Session operation}

The session is a handle which describes the cryptographic parameters to be
applied to a number of buffers.

The following structure stores the result of session creation set by the device:

\begin{lstlisting}
struct virtio_crypto_create_session_input {
    le64 session_id;
    le32 status;
    le32 padding;
};
\end{lstlisting}

A request to destroy a session includes the following information:

\begin{lstlisting}
struct virtio_crypto_destroy_session_flf {
    /* Device read only portion */
    le64  session_id;
};

struct virtio_crypto_destroy_session_input {
    /* Device write only portion */
    u8  status;
};
\end{lstlisting}


\subparagraph{Session operation: HASH session}\label{sec:Device Types / Crypto Device / Device
Operation / Control Virtqueue / Session operation / Session operation: HASH session}

The fixed-length parameters of HASH session requests is as follows:

\begin{lstlisting}
struct virtio_crypto_hash_create_session_flf {
    /* Device read only portion */

    /* See VIRTIO_CRYPTO_HASH_* above */
    le32 algo;
    /* hash result length */
    le32 hash_result_len;
};
\end{lstlisting}


\subparagraph{Session operation: MAC session}\label{sec:Device Types / Crypto Device / Device
Operation / Control Virtqueue / Session operation / Session operation: MAC session}

The fixed-length and the variable-length parameters of MAC session requests are as follows:

\begin{lstlisting}
struct virtio_crypto_mac_create_session_flf {
    /* Device read only portion */

    /* See VIRTIO_CRYPTO_MAC_* above */
    le32 algo;
    /* hash result length */
    le32 hash_result_len;
    /* length of authenticated key */
    le32 auth_key_len;
    le32 padding;
};

struct virtio_crypto_mac_create_session_vlf {
    /* Device read only portion */

    /* The authenticated key */
    u8 auth_key[auth_key_len];
};
\end{lstlisting}

The length of \field{auth_key} is specified in \field{auth_key_len} in the struct
virtio_crypto_mac_create_session_flf.


\subparagraph{Session operation: Symmetric algorithms session}\label{sec:Device Types / Crypto Device / Device
Operation / Control Virtqueue / Session operation / Session operation: Symmetric algorithms session}

The request of symmetric session could be the CIPHER algorithms request
or the chain algorithms (chaining CIPHER and HASH/MAC) request.

The fixed-length and the variable-length parameters of CIPHER session requests are as follows:

\begin{lstlisting}
struct virtio_crypto_cipher_session_flf {
    /* Device read only portion */

    /* See VIRTIO_CRYPTO_CIPHER* above */
    le32 algo;
    /* length of key */
    le32 key_len;
#define VIRTIO_CRYPTO_OP_ENCRYPT  1
#define VIRTIO_CRYPTO_OP_DECRYPT  2
    /* encryption or decryption */
    le32 op;
    le32 padding;
};

struct virtio_crypto_cipher_session_vlf {
    /* Device read only portion */

    /* The cipher key */
    u8 cipher_key[key_len];
};
\end{lstlisting}

The length of \field{cipher_key} is specified in \field{key_len} in the struct
virtio_crypto_cipher_session_flf.

The fixed-length and the variable-length parameters of Chain session requests are as follows:

\begin{lstlisting}
struct virtio_crypto_alg_chain_session_flf {
    /* Device read only portion */

#define VIRTIO_CRYPTO_SYM_ALG_CHAIN_ORDER_HASH_THEN_CIPHER  1
#define VIRTIO_CRYPTO_SYM_ALG_CHAIN_ORDER_CIPHER_THEN_HASH  2
    le32 alg_chain_order;
/* Plain hash */
#define VIRTIO_CRYPTO_SYM_HASH_MODE_PLAIN    1
/* Authenticated hash (mac) */
#define VIRTIO_CRYPTO_SYM_HASH_MODE_AUTH     2
/* Nested hash */
#define VIRTIO_CRYPTO_SYM_HASH_MODE_NESTED   3
    le32 hash_mode;
    struct virtio_crypto_cipher_session_flf cipher_hdr;

#define VIRTIO_CRYPTO_ALG_CHAIN_SESS_OP_SPEC_HDR_SIZE  16
    /* fixed length fields, algo specific */
    u8 algo_flf[VIRTIO_CRYPTO_ALG_CHAIN_SESS_OP_SPEC_HDR_SIZE];

    /* length of the additional authenticated data (AAD) in bytes */
    le32 aad_len;
    le32 padding;
};

struct virtio_crypto_alg_chain_session_vlf {
    /* Device read only portion */

    /* The cipher key */
    u8 cipher_key[key_len];
    /* The authenticated key */
    u8 auth_key[auth_key_len];
};
\end{lstlisting}

\field{hash_mode} decides the type used by \field{algo_flf}.

\field{algo_flf} is fixed to 16 bytes and MUST contains or be one of
the following types:
\begin{itemize*}
\item struct virtio_crypto_hash_create_session_flf
\item struct virtio_crypto_mac_create_session_flf
\end{itemize*}
The data of unused part (if has) in \field{algo_flf} will be ignored.

The length of \field{cipher_key} is specified in \field{key_len} in \field{cipher_hdr}.

The length of \field{auth_key} is specified in \field{auth_key_len} in struct
virtio_crypto_mac_create_session_flf.

The fixed-length parameters of Symmetric session requests are as follows:

\begin{lstlisting}
struct virtio_crypto_sym_create_session_flf {
    /* Device read only portion */

#define VIRTIO_CRYPTO_SYM_SESS_OP_SPEC_HDR_SIZE  48
    /* fixed length fields, opcode specific */
    u8 op_flf[VIRTIO_CRYPTO_SYM_SESS_OP_SPEC_HDR_SIZE];

/* No operation */
#define VIRTIO_CRYPTO_SYM_OP_NONE  0
/* Cipher only operation on the data */
#define VIRTIO_CRYPTO_SYM_OP_CIPHER  1
/* Chain any cipher with any hash or mac operation. The order
   depends on the value of alg_chain_order param */
#define VIRTIO_CRYPTO_SYM_OP_ALGORITHM_CHAINING  2
    le32 op_type;
    le32 padding;
};
\end{lstlisting}

\field{op_flf} is fixed to 48 bytes, MUST contains or be one of
the following types:
\begin{itemize*}
\item struct virtio_crypto_cipher_session_flf
\item struct virtio_crypto_alg_chain_session_flf
\end{itemize*}
The data of unused part (if has) in \field{op_flf} will be ignored.

\field{op_type} decides the type used by \field{op_flf}.

The variable-length parameters of Symmetric session requests are as follows:

\begin{lstlisting}
struct virtio_crypto_sym_create_session_vlf {
    /* Device read only portion */
    /* variable length fields, opcode specific */
    u8 op_vlf[vlf_len];
};
\end{lstlisting}

\field{op_vlf} MUST contains or be one of the following types:
\begin{itemize*}
\item struct virtio_crypto_cipher_session_vlf
\item struct virtio_crypto_alg_chain_session_vlf
\end{itemize*}

\field{op_type} in struct virtio_crypto_sym_create_session_flf decides the
type used by \field{op_vlf}.

\field{vlf_len} is the size of the specific structure used.


\subparagraph{Session operation: AEAD session}\label{sec:Device Types / Crypto Device / Device
Operation / Control Virtqueue / Session operation / Session operation: AEAD session}

The fixed-length and the variable-length parameters of AEAD session requests are as follows:

\begin{lstlisting}
struct virtio_crypto_aead_create_session_flf {
    /* Device read only portion */

    /* See VIRTIO_CRYPTO_AEAD_* above */
    le32 algo;
    /* length of key */
    le32 key_len;
    /* Authentication tag length */
    le32 tag_len;
    /* The length of the additional authenticated data (AAD) in bytes */
    le32 aad_len;
    /* encryption or decryption, See above VIRTIO_CRYPTO_OP_* */
    le32 op;
    le32 padding;
};

struct virtio_crypto_aead_create_session_vlf {
    /* Device read only portion */
    u8 key[key_len];
};
\end{lstlisting}

The length of \field{key} is specified in \field{key_len} in struct
virtio_crypto_aead_create_session_flf.

\subparagraph{Session operation: AKCIPHER session}\label{sec:Device Types / Crypto Device / Device
Operation / Control Virtqueue / Session operation / Session operation: AKCIPHER session}

Due to the complexity of asymmetric key algorithms, different algorithms
require different parameters. The following data structures are used as
supplementary parameters to describe the asymmetric algorithm sessions.

For the RSA algorithm, the extra parameters are as follows:
\begin{lstlisting}
struct virtio_crypto_rsa_session_para {
#define VIRTIO_CRYPTO_RSA_RAW_PADDING   0
#define VIRTIO_CRYPTO_RSA_PKCS1_PADDING 1
    le32 padding_algo;

#define VIRTIO_CRYPTO_RSA_NO_HASH   0
#define VIRTIO_CRYPTO_RSA_MD2       1
#define VIRTIO_CRYPTO_RSA_MD3       2
#define VIRTIO_CRYPTO_RSA_MD4       3
#define VIRTIO_CRYPTO_RSA_MD5       4
#define VIRTIO_CRYPTO_RSA_SHA1      5
#define VIRTIO_CRYPTO_RSA_SHA256    6
#define VIRTIO_CRYPTO_RSA_SHA384    7
#define VIRTIO_CRYPTO_RSA_SHA512    8
#define VIRTIO_CRYPTO_RSA_SHA224    9
    le32 hash_algo;
};
\end{lstlisting}

\field{padding_algo} specifies the padding method used by RSA sessions.
\begin{itemize*}
\item If VIRTIO_CRYPTO_RSA_RAW_PADDING is specified, 1) \field{hash_algo}
is ignored, 2) ciphertext and plaintext MUST be padded with leading zeros,
3) and RSA sessions with VIRTIO_CRYPTO_RSA_RAW_PADDING MUST not be used
for verification and signing operations.
\item If VIRTIO_CRYPTO_RSA_PKCS1_PADDING is specified, EMSA-PKCS1-v1_5 padding method
is used (see \hyperref[intro:rfc3447]{PKCS\#1}), \field{hash_algo} specifies how the
digest of the data passed to RSA sessions is calculated when verifying and signing.
It only affects the padding algorithm and is ignored during encryption and decryption.
\end{itemize*}

The ECC algorithms such as the ECDSA algorithm, cannot use custom curves, only the
following known curves can be used (see \hyperref[intro:NIST]{NIST-recommended curves}).

\begin{lstlisting}
#define VIRTIO_CRYPTO_CURVE_UNKNOWN   0
#define VIRTIO_CRYPTO_CURVE_NIST_P192 1
#define VIRTIO_CRYPTO_CURVE_NIST_P224 2
#define VIRTIO_CRYPTO_CURVE_NIST_P256 3
#define VIRTIO_CRYPTO_CURVE_NIST_P384 4
#define VIRTIO_CRYPTO_CURVE_NIST_P521 5
\end{lstlisting}

For the ECDSA algorithm, the extra parameters are as follows:
\begin{lstlisting}
struct virtio_crypto_ecdsa_session_para {
    /* See VIRTIO_CRYPTO_CURVE_* above */
    le32 curve_id;
};
\end{lstlisting}

The fixed-length and the variable-length parameters of AKCIPHER session requests are as follows:
\begin{lstlisting}
struct virtio_crypto_akcipher_create_session_flf {
    /* Device read only portion */

    /* See VIRTIO_CRYPTO_AKCIPHER_* above */
    le32 algo;
#define VIRTIO_CRYPTO_AKCIPHER_KEY_TYPE_PUBLIC 1
#define VIRTIO_CRYPTO_AKCIPHER_KEY_TYPE_PRIVATE 2
    le32 key_type;
    /* length of key */
    le32 key_len;

#define VIRTIO_CRYPTO_AKCIPHER_SESS_ALGO_SPEC_HDR_SIZE 44
    u8 algo_flf[VIRTIO_CRYPTO_AKCIPHER_SESS_ALGO_SPEC_HDR_SIZE];
};

struct virtio_crypto_akcipher_create_session_vlf {
    /* Device read only portion */
    u8 key[key_len];
};
\end{lstlisting}

\field{algo} decides the type used by \field{algo_flf}.
\field{algo_flf} is fixed to 44 bytes and MUST contains of be one the
following structures:
\begin{itemize*}
\item struct virtio_crypto_rsa_session_para
\item struct virtio_crypto_ecdsa_session_para
\end{itemize*}

The length of \field{key} is specified in \field{key_len} in the struct
virtio_crypto_akcipher_create_session_flf.

For the RSA algorithm, the key needs to be encoded according to
\hyperref[intro:rfc3447]{PKCS\#1}. The private key is described with the
RSAPrivateKey structure, and the public key is described with the RSAPublicKey
structure. These ASN.1 structures are encoded in DER encoding rules (see
\hyperref[intro:rfc6025]{rfc6025}).

\begin{lstlisting}
RSAPrivateKey ::= SEQUENCE {
    version          INTEGER,
    modulus          INTEGER,
    publicExponent   INTEGER,
    privateExponent  INTEGER,
    prime1           INTEGER,
    prime2           INTEGER,
    exponent1        INTEGER,
    exponent1        INTEGER,
    coefficient      INTEGER,
    otherPrimeInfos  OtherPrimeInfos OPTIONAL
}

OtherPrimeInfos ::= SEQUENCE SIZE(1...MAX) OF OtherPrimeInfo

OtherPrimeINfo ::= SEQUENCE {
    prime           INTEGER,
    exponent        INTEGER,
    coefficient     INTEGER
}

RSAPublicKey ::= SEQUENCE {
    modulus         INTEGER,
    publicExponent  INTEGER
}
\end{lstlisting}

For the ECDSA algorithm, the private key is encoded according to
\hyperref[intro:rfc5915]{RFC5915}, the private key of the ECDSA algorithm
is described by the ASN.1 structure ECPrivateKey and encoded with DER
encoding rules (see \hyperref[intro:rfc6025]{rfc6025}).

\begin{lstlisting}
ECPrivateKey ::= SEQUNCE {
    version         INTEGER,
    privateKey      OCTET STRING,
    parameters [0]  ECParameters {{ NamedCurve }} OPTIONAL,
    publicKey  [1]  BIT STRING OPTIONAL
}
\end{lstlisting}

The public key of the ECDSA algorithm is encoded according to \hyperref[intro:SEC1]{SEC1},
and the public key of ECDSA is described by the ASN.1 structure ECPoint.
When initializing a session with ECDSA public key, the ECPoint is DER encoded and the
\field{key} only contains the value part of ECPoint, that is, the header part of the
OCTET STRING will be omitted (see \hyperref[intro:rfc6025]{rfc6025}).

\begin{lstlisting}
ECPoint ::= OCTET STRING
\end{lstlisting}

The length of \field{key} is specified in \field{key_len} in
struct virtio_crypto_akcipher_create_session_flf.

\drivernormative{\subparagraph}{Session operation: create session}{Device Types / Crypto Device / Device
Operation / Control Virtqueue / Session operation / Session operation: create session}

\begin{itemize*}
\item The driver MUST set the \field{opcode} field based on service type: CIPHER, HASH, MAC, AEAD or AKCIPHER.
\item The driver MUST set the control general header, the opcode specific header,
    the opcode specific extra parameters and the opcode specific outcome buffer in turn.
    See \ref{sec:Device Types / Crypto Device / Device Operation / Control Virtqueue}.
\item The driver MUST set the \field{reversed} field to zero.
\end{itemize*}

\devicenormative{\subparagraph}{Session operation: create session}{Device Types / Crypto Device / Device
Operation / Control Virtqueue / Session operation / Session operation: create session}

\begin{itemize*}
\item The device MUST use the corresponding opcode specific structure according to the
    \field{opcode} in the control general header.
\item The device MUST extract extra parameters according to the structures used.
\item The device MUST set the \field{status} field to one of the following values of enum
    VIRTIO_CRYPTO_STATUS after finish a session creation:
\begin{itemize*}
\item VIRTIO_CRYPTO_OK if a session is created successfully.
\item VIRTIO_CRYPTO_NOTSUPP if the requested algorithm or operation is unsupported.
\item VIRTIO_CRYPTO_NOSPC if no free session ID (only when the VIRTIO_CRYPTO_F_REVISION_1
    feature bit is negotiated).
\item VIRTIO_CRYPTO_ERR if failure not mentioned above occurs.
\end{itemize*}
\item The device MUST set the \field{session_id} field to a unique session identifier only
    if the status is set to VIRTIO_CRYPTO_OK.
\end{itemize*}

\drivernormative{\subparagraph}{Session operation: destroy session}{Device Types / Crypto Device / Device
Operation / Control Virtqueue / Session operation / Session operation: destroy session}

\begin{itemize*}
\item The driver MUST set the \field{opcode} field based on service type: CIPHER, HASH, MAC, AEAD or AKCIPHER.
\item The driver MUST set the \field{session_id} to a valid value assigned by the device
    when the session was created.
\end{itemize*}

\devicenormative{\subparagraph}{Session operation: destroy session}{Device Types / Crypto Device / Device
Operation / Control Virtqueue / Session operation / Session operation: destroy session}

\begin{itemize*}
\item The device MUST set the \field{status} field to one of the following values of enum VIRTIO_CRYPTO_STATUS.
\begin{itemize*}
\item VIRTIO_CRYPTO_OK if a session is created successfully.
\item VIRTIO_CRYPTO_ERR if any failure occurs.
\end{itemize*}
\end{itemize*}


\subsubsection{Data Virtqueue}\label{sec:Device Types / Crypto Device / Device Operation / Data Virtqueue}

The driver uses the data virtqueues to transmit crypto operation requests to the device,
and completes the crypto operations.

The header for dataq is as follows:

\begin{lstlisting}
struct virtio_crypto_op_header {
#define VIRTIO_CRYPTO_CIPHER_ENCRYPT \
    VIRTIO_CRYPTO_OPCODE(VIRTIO_CRYPTO_SERVICE_CIPHER, 0x00)
#define VIRTIO_CRYPTO_CIPHER_DECRYPT \
    VIRTIO_CRYPTO_OPCODE(VIRTIO_CRYPTO_SERVICE_CIPHER, 0x01)
#define VIRTIO_CRYPTO_HASH \
    VIRTIO_CRYPTO_OPCODE(VIRTIO_CRYPTO_SERVICE_HASH, 0x00)
#define VIRTIO_CRYPTO_MAC \
    VIRTIO_CRYPTO_OPCODE(VIRTIO_CRYPTO_SERVICE_MAC, 0x00)
#define VIRTIO_CRYPTO_AEAD_ENCRYPT \
    VIRTIO_CRYPTO_OPCODE(VIRTIO_CRYPTO_SERVICE_AEAD, 0x00)
#define VIRTIO_CRYPTO_AEAD_DECRYPT \
    VIRTIO_CRYPTO_OPCODE(VIRTIO_CRYPTO_SERVICE_AEAD, 0x01)
#define VIRTIO_CRYPTO_AKCIPHER_ENCRYPT \
    VIRTIO_CRYPTO_OPCODE(VIRTIO_CRYPTO_SERVICE_AKCIPHER, 0x00)
#define VIRTIO_CRYPTO_AKCIPHER_DECRYPT \
    VIRTIO_CRYPTO_OPCODE(VIRTIO_CRYPTO_SERVICE_AKCIPHER, 0x01)
#define VIRTIO_CRYPTO_AKCIPHER_SIGN \
    VIRTIO_CRYPTO_OPCODE(VIRTIO_CRYPTO_SERVICE_AKCIPHER, 0x02)
#define VIRTIO_CRYPTO_AKCIPHER_VERIFY \
    VIRTIO_CRYPTO_OPCODE(VIRTIO_CRYPTO_SERVICE_AKCIPHER, 0x03)
    le32 opcode;
    /* algo should be service-specific algorithms */
    le32 algo;
    le64 session_id;
#define VIRTIO_CRYPTO_FLAG_SESSION_MODE 1
    /* control flag to control the request */
    le32 flag;
    le32 padding;
};
\end{lstlisting}

\begin{note}
If VIRTIO_CRYPTO_F_REVISION_1 is not negotiated the \field{flag} is ignored.

If VIRTIO_CRYPTO_F_REVISION_1 is negotiated but VIRTIO_CRYPTO_F_<SERVICE>_STATELESS_MODE
is not negotiated, then the device SHOULD reject <SERVICE> requests if
VIRTIO_CRYPTO_FLAG_SESSION_MODE is not set (in \field{flag}).
\end{note}

The dataq request is composed of four parts:
\begin{lstlisting}
struct virtio_crypto_op_data_req {
    /* Device read only portion */

    struct virtio_crypto_op_header header;

#define VIRTIO_CRYPTO_DATAQ_OP_SPEC_HDR_LEGACY 48
    /* fixed length fields, opcode specific */
    u8 op_flf[flf_len];

    /* Device read && write portion */
    /* variable length fields, opcode specific */
    u8 op_vlf[vlf_len];

    /* Device write only portion */
    struct virtio_crypto_inhdr inhdr;
};
\end{lstlisting}

\field{header} is a general header (see above).

\field{op_flf} is the opcode (in \field{header}) specific header.

\field{flf_len} depends on the VIRTIO_CRYPTO_F_REVISION_1 feature bit
(see below).

\field{op_vlf} is the opcode (in \field{header}) specific parameters.

\field{vlf_len} is the size of the specific structure used.

\begin{itemize*}
\item If the the opcode (in \field{header}) is VIRTIO_CRYPTO_CIPHER_ENCRYPT
    or VIRTIO_CRYPTO_CIPHER_DECRYPT then:
    \begin{itemize*}
    \item If VIRTIO_CRYPTO_F_CIPHER_STATELESS_MODE is negotiated, \field{op_flf} is
        struct virtio_crypto_sym_data_flf_stateless, and \field{op_vlf} is struct
        virtio_crypto_sym_data_vlf_stateless.
    \item If VIRTIO_CRYPTO_F_CIPHER_STATELESS_MODE is NOT negotiated, \field{op_flf}
        is struct virtio_crypto_sym_data_flf if VIRTIO_CRYPTO_F_REVISION_1 is negotiated
        and struct virtio_crypto_sym_data_flf is padded to 48 bytes if NOT negotiated,
        and \field{op_vlf} is struct virtio_crypto_sym_data_vlf.
    \end{itemize*}
\item If the the opcode (in \field{header}) is VIRTIO_CRYPTO_HASH:
    \begin{itemize*}
    \item If VIRTIO_CRYPTO_F_HASH_STATELESS_MODE is negotiated, \field{op_flf} is
        struct virtio_crypto_hash_data_flf_stateless, and \field{op_vlf} is struct
        virtio_crypto_hash_data_vlf_stateless.
    \item If VIRTIO_CRYPTO_F_HASH_STATELESS_MODE is NOT negotiated, \field{op_flf}
        is struct virtio_crypto_hash_data_flf if VIRTIO_CRYPTO_F_REVISION_1 is negotiated
        and struct virtio_crypto_hash_data_flf is padded to 48 bytes if NOT negotiated,
        and \field{op_vlf} is struct virtio_crypto_hash_data_vlf.
    \end{itemize*}
\item If the the opcode (in \field{header}) is VIRTIO_CRYPTO_MAC:
    \begin{itemize*}
    \item If VIRTIO_CRYPTO_F_MAC_STATELESS_MODE is negotiated, \field{op_flf} is
        struct virtio_crypto_mac_data_flf_stateless, and \field{op_vlf} is struct
        virtio_crypto_mac_data_vlf_stateless.
    \item If VIRTIO_CRYPTO_F_MAC_STATELESS_MODE is NOT negotiated, \field{op_flf}
        is struct virtio_crypto_mac_data_flf if VIRTIO_CRYPTO_F_REVISION_1 is negotiated
        and struct virtio_crypto_mac_data_flf is padded to 48 bytes if NOT negotiated,
        and \field{op_vlf} is struct virtio_crypto_mac_data_vlf.
    \end{itemize*}
\item If the the opcode (in \field{header}) is VIRTIO_CRYPTO_AEAD_ENCRYPT
    or VIRTIO_CRYPTO_AEAD_DECRYPT then:
    \begin{itemize*}
    \item If VIRTIO_CRYPTO_F_AEAD_STATELESS_MODE is negotiated, \field{op_flf} is
        struct virtio_crypto_aead_data_flf_stateless, and \field{op_vlf} is struct
        virtio_crypto_aead_data_vlf_stateless.
    \item If VIRTIO_CRYPTO_F_AEAD_STATELESS_MODE is NOT negotiated, \field{op_flf}
        is struct virtio_crypto_aead_data_flf if VIRTIO_CRYPTO_F_REVISION_1 is negotiated
        and struct virtio_crypto_aead_data_flf is padded to 48 bytes if NOT negotiated,
        and \field{op_vlf} is struct virtio_crypto_aead_data_vlf.
    \end{itemize*}
\item If the opcode (in \field{header}) is VIRTIO_CRYPTO_AKCIPHER_ENCRYPT, VIRTIO_CRYPTO_AKCIPHER_DECRYPT,
    VIRTIO_CRYPTO_AKCIPHER_SIGN or VIRTIO_CRYPTO_AKCIPHER_VERIFY then:
    \begin{itemize*}
    \item If VIRTIO_CRYPTO_F_AKCIPHER_STATELESS_MODE is negotiated, \field{op_flf} is
        struct virtio_crypto_akcipher_data_flf_statless, and \field{op_vlf} is struct
        virtio_crypto_akcipher_data_vlf_stateless.
    \item If VIRTIO_CRYPTO_F_AKCIPHER_STATELESS_MODE is NOT negotiated, \field{op_flf}
        is struct virtio_crypto_akcipher_data_flf if VIRTIO_CRYPTO_F_REVISION_1 is negotiated
        and struct virtio_crypto_akcipher_data_flf is padded to 48 bytes if NOT negotiated,
        and \field{op_vlf} is struct virtio_crypto_akcipher_data_vlf.
    \end{itemize*}
\end{itemize*}

\field{inhdr} is a unified input header that used to return the status of
the operations, is defined as follows:

\begin{lstlisting}
struct virtio_crypto_inhdr {
    u8 status;
};
\end{lstlisting}

\subsubsection{HASH Service Operation}\label{sec:Device Types / Crypto Device / Device Operation / HASH Service Operation}

Session mode HASH service requests are as follows:

\begin{lstlisting}
struct virtio_crypto_hash_data_flf {
    /* length of source data */
    le32 src_data_len;
    /* hash result length */
    le32 hash_result_len;
};

struct virtio_crypto_hash_data_vlf {
    /* Device read only portion */
    /* Source data */
    u8 src_data[src_data_len];

    /* Device write only portion */
    /* Hash result data */
    u8 hash_result[hash_result_len];
};
\end{lstlisting}

Each data request uses the virtio_crypto_hash_data_flf structure and the
virtio_crypto_hash_data_vlf structure to store information used to run the
HASH operations.

\field{src_data} is the source data that will be processed.
\field{src_data_len} is the length of source data.
\field{hash_result} is the result data and \field{hash_result_len} is the length
of it.

Stateless mode HASH service requests are as follows:

\begin{lstlisting}
struct virtio_crypto_hash_data_flf_stateless {
    struct {
        /* See VIRTIO_CRYPTO_HASH_* above */
        le32 algo;
    } sess_para;

    /* length of source data */
    le32 src_data_len;
    /* hash result length */
    le32 hash_result_len;
    le32 reserved;
};
struct virtio_crypto_hash_data_vlf_stateless {
    /* Device read only portion */
    /* Source data */
    u8 src_data[src_data_len];

    /* Device write only portion */
    /* Hash result data */
    u8 hash_result[hash_result_len];
};
\end{lstlisting}

\drivernormative{\paragraph}{HASH Service Operation}{Device Types / Crypto Device / Device Operation / HASH Service Operation}

\begin{itemize*}
\item If the driver uses the session mode, then the driver MUST set \field{session_id}
    in struct virtio_crypto_op_header to a valid value assigned by the device when the
    session was created.
\item If the VIRTIO_CRYPTO_F_HASH_STATELESS_MODE feature bit is negotiated, 1) if the
    driver uses the stateless mode, then the driver MUST set the \field{flag} field in
    struct virtio_crypto_op_header to ZERO and MUST set the fields in struct
    virtio_crypto_hash_data_flf_stateless.sess_para, 2) if the driver uses the session
    mode, then the driver MUST set the \field{flag} field in struct virtio_crypto_op_header
    to VIRTIO_CRYPTO_FLAG_SESSION_MODE.
\item The driver MUST set \field{opcode} in struct virtio_crypto_op_header to VIRTIO_CRYPTO_HASH.
\end{itemize*}

\devicenormative{\paragraph}{HASH Service Operation}{Device Types / Crypto Device / Device Operation / HASH Service Operation}

\begin{itemize*}
\item The device MUST use the corresponding structure according to the \field{opcode}
    in the data general header.
\item If the VIRTIO_CRYPTO_F_HASH_STATELESS_MODE feature bit is negotiated, the device
    MUST parse \field{flag} field in struct virtio_crypto_op_header in order to decide
    which mode the driver uses.
\item The device MUST copy the results of HASH operations in the hash_result[] if HASH
    operations success.
\item The device MUST set \field{status} in struct virtio_crypto_inhdr to one of the
    following values of enum VIRTIO_CRYPTO_STATUS:
\begin{itemize*}
\item VIRTIO_CRYPTO_OK if the operation success.
\item VIRTIO_CRYPTO_NOTSUPP if the requested algorithm or operation is unsupported.
\item VIRTIO_CRYPTO_INVSESS if the session ID invalid when in session mode.
\item VIRTIO_CRYPTO_ERR if any failure not mentioned above occurs.
\end{itemize*}
\end{itemize*}


\subsubsection{MAC Service Operation}\label{sec:Device Types / Crypto Device / Device Operation / MAC Service Operation}

Session mode MAC service requests are as follows:

\begin{lstlisting}
struct virtio_crypto_mac_data_flf {
    struct virtio_crypto_hash_data_flf hdr;
};

struct virtio_crypto_mac_data_vlf {
    /* Device read only portion */
    /* Source data */
    u8 src_data[src_data_len];

    /* Device write only portion */
    /* Hash result data */
    u8 hash_result[hash_result_len];
};
\end{lstlisting}

Each request uses the virtio_crypto_mac_data_flf structure and the
virtio_crypto_mac_data_vlf structure to store information used to run the
MAC operations.

\field{src_data} is the source data that will be processed.
\field{src_data_len} is the length of source data.
\field{hash_result} is the result data and \field{hash_result_len} is the length
of it.

Stateless mode MAC service requests are as follows:

\begin{lstlisting}
struct virtio_crypto_mac_data_flf_stateless {
    struct {
        /* See VIRTIO_CRYPTO_MAC_* above */
        le32 algo;
        /* length of authenticated key */
        le32 auth_key_len;
    } sess_para;

    /* length of source data */
    le32 src_data_len;
    /* hash result length */
    le32 hash_result_len;
};

struct virtio_crypto_mac_data_vlf_stateless {
    /* Device read only portion */
    /* The authenticated key */
    u8 auth_key[auth_key_len];
    /* Source data */
    u8 src_data[src_data_len];

    /* Device write only portion */
    /* Hash result data */
    u8 hash_result[hash_result_len];
};
\end{lstlisting}

\field{auth_key} is the authenticated key that will be used during the process.
\field{auth_key_len} is the length of the key.

\drivernormative{\paragraph}{MAC Service Operation}{Device Types / Crypto Device / Device Operation / MAC Service Operation}

\begin{itemize*}
\item If the driver uses the session mode, then the driver MUST set \field{session_id}
    in struct virtio_crypto_op_header to a valid value assigned by the device when the
    session was created.
\item If the VIRTIO_CRYPTO_F_MAC_STATELESS_MODE feature bit is negotiated, 1) if the
    driver uses the stateless mode, then the driver MUST set the \field{flag} field
    in struct virtio_crypto_op_header to ZERO and MUST set the fields in struct
    virtio_crypto_mac_data_flf_stateless.sess_para, 2) if the driver uses the session
    mode, then the driver MUST set the \field{flag} field in struct virtio_crypto_op_header
    to VIRTIO_CRYPTO_FLAG_SESSION_MODE.
\item The driver MUST set \field{opcode} in struct virtio_crypto_op_header to VIRTIO_CRYPTO_MAC.
\end{itemize*}

\devicenormative{\paragraph}{MAC Service Operation}{Device Types / Crypto Device / Device Operation / MAC Service Operation}

\begin{itemize*}
\item If the VIRTIO_CRYPTO_F_MAC_STATELESS_MODE feature bit is negotiated, the device
    MUST parse \field{flag} field in struct virtio_crypto_op_header in order to decide
	which mode the driver uses.
\item The device MUST copy the results of MAC operations in the hash_result[] if HASH
    operations success.
\item The device MUST set \field{status} in struct virtio_crypto_inhdr to one of the
    following values of enum VIRTIO_CRYPTO_STATUS:
\begin{itemize*}
\item VIRTIO_CRYPTO_OK if the operation success.
\item VIRTIO_CRYPTO_NOTSUPP if the requested algorithm or operation is unsupported.
\item VIRTIO_CRYPTO_INVSESS if the session ID invalid when in session mode.
\item VIRTIO_CRYPTO_ERR if any failure not mentioned above occurs.
\end{itemize*}
\end{itemize*}

\subsubsection{Symmetric algorithms Operation}\label{sec:Device Types / Crypto Device / Device Operation / Symmetric algorithms Operation}

Session mode CIPHER service requests are as follows:

\begin{lstlisting}
struct virtio_crypto_cipher_data_flf {
    /*
     * Byte Length of valid IV/Counter data pointed to by the below iv data.
     *
     * For block ciphers in CBC or F8 mode, or for Kasumi in F8 mode, or for
     *   SNOW3G in UEA2 mode, this is the length of the IV (which
     *   must be the same as the block length of the cipher).
     * For block ciphers in CTR mode, this is the length of the counter
     *   (which must be the same as the block length of the cipher).
     */
    le32 iv_len;
    /* length of source data */
    le32 src_data_len;
    /* length of destination data */
    le32 dst_data_len;
    le32 padding;
};

struct virtio_crypto_cipher_data_vlf {
    /* Device read only portion */

    /*
     * Initialization Vector or Counter data.
     *
     * For block ciphers in CBC or F8 mode, or for Kasumi in F8 mode, or for
     *   SNOW3G in UEA2 mode, this is the Initialization Vector (IV)
     *   value.
     * For block ciphers in CTR mode, this is the counter.
     * For AES-XTS, this is the 128bit tweak, i, from IEEE Std 1619-2007.
     *
     * The IV/Counter will be updated after every partial cryptographic
     * operation.
     */
    u8 iv[iv_len];
    /* Source data */
    u8 src_data[src_data_len];

    /* Device write only portion */
    /* Destination data */
    u8 dst_data[dst_data_len];
};
\end{lstlisting}

Session mode requests of algorithm chaining are as follows:

\begin{lstlisting}
struct virtio_crypto_alg_chain_data_flf {
    le32 iv_len;
    /* Length of source data */
    le32 src_data_len;
    /* Length of destination data */
    le32 dst_data_len;
    /* Starting point for cipher processing in source data */
    le32 cipher_start_src_offset;
    /* Length of the source data that the cipher will be computed on */
    le32 len_to_cipher;
    /* Starting point for hash processing in source data */
    le32 hash_start_src_offset;
    /* Length of the source data that the hash will be computed on */
    le32 len_to_hash;
    /* Length of the additional auth data */
    le32 aad_len;
    /* Length of the hash result */
    le32 hash_result_len;
    le32 reserved;
};

struct virtio_crypto_alg_chain_data_vlf {
    /* Device read only portion */

    /* Initialization Vector or Counter data */
    u8 iv[iv_len];
    /* Source data */
    u8 src_data[src_data_len];
    /* Additional authenticated data if exists */
    u8 aad[aad_len];

    /* Device write only portion */

    /* Destination data */
    u8 dst_data[dst_data_len];
    /* Hash result data */
    u8 hash_result[hash_result_len];
};
\end{lstlisting}

Session mode requests of symmetric algorithm are as follows:

\begin{lstlisting}
struct virtio_crypto_sym_data_flf {
    /* Device read only portion */

#define VIRTIO_CRYPTO_SYM_DATA_REQ_HDR_SIZE    40
    u8 op_type_flf[VIRTIO_CRYPTO_SYM_DATA_REQ_HDR_SIZE];

    /* See above VIRTIO_CRYPTO_SYM_OP_* */
    le32 op_type;
    le32 padding;
};

struct virtio_crypto_sym_data_vlf {
    u8 op_type_vlf[sym_para_len];
};
\end{lstlisting}

Each request uses the virtio_crypto_sym_data_flf structure and the
virtio_crypto_sym_data_flf structure to store information used to run the
CIPHER operations.

\field{op_type_flf} is the \field{op_type} specific header, it MUST starts
with or be one of the following structures:
\begin{itemize*}
\item struct virtio_crypto_cipher_data_flf
\item struct virtio_crypto_alg_chain_data_flf
\end{itemize*}

The length of \field{op_type_flf} is fixed to 40 bytes, the data of unused
part (if has) will be ignored.

\field{op_type_vlf} is the \field{op_type} specific parameters, it MUST starts
with or be one of the following structures:
\begin{itemize*}
\item struct virtio_crypto_cipher_data_vlf
\item struct virtio_crypto_alg_chain_data_vlf
\end{itemize*}

\field{sym_para_len} is the size of the specific structure used.

Stateless mode CIPHER service requests are as follows:

\begin{lstlisting}
struct virtio_crypto_cipher_data_flf_stateless {
    struct {
        /* See VIRTIO_CRYPTO_CIPHER* above */
        le32 algo;
        /* length of key */
        le32 key_len;

        /* See VIRTIO_CRYPTO_OP_* above */
        le32 op;
    } sess_para;

    /*
     * Byte Length of valid IV/Counter data pointed to by the below iv data.
     */
    le32 iv_len;
    /* length of source data */
    le32 src_data_len;
    /* length of destination data */
    le32 dst_data_len;
};

struct virtio_crypto_cipher_data_vlf_stateless {
    /* Device read only portion */

    /* The cipher key */
    u8 cipher_key[key_len];

    /* Initialization Vector or Counter data. */
    u8 iv[iv_len];
    /* Source data */
    u8 src_data[src_data_len];

    /* Device write only portion */
    /* Destination data */
    u8 dst_data[dst_data_len];
};
\end{lstlisting}

Stateless mode requests of algorithm chaining are as follows:

\begin{lstlisting}
struct virtio_crypto_alg_chain_data_flf_stateless {
    struct {
        /* See VIRTIO_CRYPTO_SYM_ALG_CHAIN_ORDER_* above */
        le32 alg_chain_order;
        /* length of the additional authenticated data in bytes */
        le32 aad_len;

        struct {
            /* See VIRTIO_CRYPTO_CIPHER* above */
            le32 algo;
            /* length of key */
            le32 key_len;
            /* See VIRTIO_CRYPTO_OP_* above */
            le32 op;
        } cipher;

        struct {
            /* See VIRTIO_CRYPTO_HASH_* or VIRTIO_CRYPTO_MAC_* above */
            le32 algo;
            /* length of authenticated key */
            le32 auth_key_len;
            /* See VIRTIO_CRYPTO_SYM_HASH_MODE_* above */
            le32 hash_mode;
        } hash;
    } sess_para;

    le32 iv_len;
    /* Length of source data */
    le32 src_data_len;
    /* Length of destination data */
    le32 dst_data_len;
    /* Starting point for cipher processing in source data */
    le32 cipher_start_src_offset;
    /* Length of the source data that the cipher will be computed on */
    le32 len_to_cipher;
    /* Starting point for hash processing in source data */
    le32 hash_start_src_offset;
    /* Length of the source data that the hash will be computed on */
    le32 len_to_hash;
    /* Length of the additional auth data */
    le32 aad_len;
    /* Length of the hash result */
    le32 hash_result_len;
    le32 reserved;
};

struct virtio_crypto_alg_chain_data_vlf_stateless {
    /* Device read only portion */

    /* The cipher key */
    u8 cipher_key[key_len];
    /* The auth key */
    u8 auth_key[auth_key_len];
    /* Initialization Vector or Counter data */
    u8 iv[iv_len];
    /* Additional authenticated data if exists */
    u8 aad[aad_len];
    /* Source data */
    u8 src_data[src_data_len];

    /* Device write only portion */

    /* Destination data */
    u8 dst_data[dst_data_len];
    /* Hash result data */
    u8 hash_result[hash_result_len];
};
\end{lstlisting}

Stateless mode requests of symmetric algorithm are as follows:

\begin{lstlisting}
struct virtio_crypto_sym_data_flf_stateless {
    /* Device read only portion */
#define VIRTIO_CRYPTO_SYM_DATE_REQ_HDR_STATELESS_SIZE    72
    u8 op_type_flf[VIRTIO_CRYPTO_SYM_DATE_REQ_HDR_STATELESS_SIZE];

    /* Device write only portion */
    /* See above VIRTIO_CRYPTO_SYM_OP_* */
    le32 op_type;
};

struct virtio_crypto_sym_data_vlf_stateless {
    u8 op_type_vlf[sym_para_len];
};
\end{lstlisting}

\field{op_type_flf} is the \field{op_type} specific header, it MUST starts
with or be one of the following structures:
\begin{itemize*}
\item struct virtio_crypto_cipher_data_flf_stateless
\item struct virtio_crypto_alg_chain_data_flf_stateless
\end{itemize*}

The length of \field{op_type_flf} is fixed to 72 bytes, the data of unused
part (if has) will be ignored.

\field{op_type_vlf} is the \field{op_type} specific parameters, it MUST starts
with or be one of the following structures:
\begin{itemize*}
\item struct virtio_crypto_cipher_data_vlf_stateless
\item struct virtio_crypto_alg_chain_data_vlf_stateless
\end{itemize*}

\field{sym_para_len} is the size of the specific structure used.

\drivernormative{\paragraph}{Symmetric algorithms Operation}{Device Types / Crypto Device / Device Operation / Symmetric algorithms Operation}

\begin{itemize*}
\item If the driver uses the session mode, then the driver MUST set \field{session_id}
    in struct virtio_crypto_op_header to a valid value assigned by the device when the
    session was created.
\item If the VIRTIO_CRYPTO_F_CIPHER_STATELESS_MODE feature bit is negotiated, 1) if the
    driver uses the stateless mode, then the driver MUST set the \field{flag} field in
    struct virtio_crypto_op_header to ZERO and MUST set the fields in struct
    virtio_crypto_cipher_data_flf_stateless.sess_para or struct
    virtio_crypto_alg_chain_data_flf_stateless.sess_para, 2) if the driver uses the
    session mode, then the driver MUST set the \field{flag} field in struct
    virtio_crypto_op_header to VIRTIO_CRYPTO_FLAG_SESSION_MODE.
\item The driver MUST set the \field{opcode} field in struct virtio_crypto_op_header
    to VIRTIO_CRYPTO_CIPHER_ENCRYPT or VIRTIO_CRYPTO_CIPHER_DECRYPT.
\item The driver MUST specify the fields of struct virtio_crypto_cipher_data_flf in
    struct virtio_crypto_sym_data_flf and struct virtio_crypto_cipher_data_vlf in
    struct virtio_crypto_sym_data_vlf if the request is based on VIRTIO_CRYPTO_SYM_OP_CIPHER.
\item The driver MUST specify the fields of struct virtio_crypto_alg_chain_data_flf
    in struct virtio_crypto_sym_data_flf and struct virtio_crypto_alg_chain_data_vlf
    in struct virtio_crypto_sym_data_vlf if the request is of the VIRTIO_CRYPTO_SYM_OP_ALGORITHM_CHAINING
    type.
\end{itemize*}

\devicenormative{\paragraph}{Symmetric algorithms Operation}{Device Types / Crypto Device / Device Operation / Symmetric algorithms Operation}

\begin{itemize*}
\item If the VIRTIO_CRYPTO_F_CIPHER_STATELESS_MODE feature bit is negotiated, the device
    MUST parse \field{flag} field in struct virtio_crypto_op_header in order to decide
	which mode the driver uses.
\item The device MUST parse the virtio_crypto_sym_data_req based on the \field{opcode}
    field in general header.
\item The device MUST parse the fields of struct virtio_crypto_cipher_data_flf in
    struct virtio_crypto_sym_data_flf and struct virtio_crypto_cipher_data_vlf in
    struct virtio_crypto_sym_data_vlf if the request is based on VIRTIO_CRYPTO_SYM_OP_CIPHER.
\item The device MUST parse the fields of struct virtio_crypto_alg_chain_data_flf
    in struct virtio_crypto_sym_data_flf and struct virtio_crypto_alg_chain_data_vlf
    in struct virtio_crypto_sym_data_vlf if the request is of the VIRTIO_CRYPTO_SYM_OP_ALGORITHM_CHAINING
    type.
\item The device MUST copy the result of cryptographic operation in the dst_data[] in
    both plain CIPHER mode and algorithms chain mode.
\item The device MUST check the \field{para}.\field{add_len} is bigger than 0 before
    parse the additional authenticated data in plain algorithms chain mode.
\item The device MUST copy the result of HASH/MAC operation in the hash_result[] is
    of the VIRTIO_CRYPTO_SYM_OP_ALGORITHM_CHAINING type.
\item The device MUST set the \field{status} field in struct virtio_crypto_inhdr to
    one of the following values of enum VIRTIO_CRYPTO_STATUS:
\begin{itemize*}
\item VIRTIO_CRYPTO_OK if the operation success.
\item VIRTIO_CRYPTO_NOTSUPP if the requested algorithm or operation is unsupported.
\item VIRTIO_CRYPTO_INVSESS if the session ID is invalid in session mode.
\item VIRTIO_CRYPTO_ERR if failure not mentioned above occurs.
\end{itemize*}
\end{itemize*}

\subsubsection{AEAD Service Operation}\label{sec:Device Types / Crypto Device / Device Operation / AEAD Service Operation}

Session mode requests of symmetric algorithm are as follows:

\begin{lstlisting}
struct virtio_crypto_aead_data_flf {
    /*
     * Byte Length of valid IV data.
     *
     * For GCM mode, this is either 12 (for 96-bit IVs) or 16, in which
     *   case iv points to J0.
     * For CCM mode, this is the length of the nonce, which can be in the
     *   range 7 to 13 inclusive.
     */
    le32 iv_len;
    /* length of additional auth data */
    le32 aad_len;
    /* length of source data */
    le32 src_data_len;
    /* length of dst data, this should be at least src_data_len + tag_len */
    le32 dst_data_len;
    /* Authentication tag length */
    le32 tag_len;
    le32 reserved;
};

struct virtio_crypto_aead_data_vlf {
    /* Device read only portion */

    /*
     * Initialization Vector data.
     *
     * For GCM mode, this is either the IV (if the length is 96 bits) or J0
     *   (for other sizes), where J0 is as defined by NIST SP800-38D.
     *   Regardless of the IV length, a full 16 bytes needs to be allocated.
     * For CCM mode, the first byte is reserved, and the nonce should be
     *   written starting at &iv[1] (to allow space for the implementation
     *   to write in the flags in the first byte).  Note that a full 16 bytes
     *   should be allocated, even though the iv_len field will have
     *   a value less than this.
     *
     * The IV will be updated after every partial cryptographic operation.
     */
    u8 iv[iv_len];
    /* Source data */
    u8 src_data[src_data_len];
    /* Additional authenticated data if exists */
    u8 aad[aad_len];

    /* Device write only portion */
    /* Pointer to output data */
    u8 dst_data[dst_data_len];
};
\end{lstlisting}

Each request uses the virtio_crypto_aead_data_flf structure and the
virtio_crypto_aead_data_flf structure to store information used to run the
AEAD operations.

Stateless mode AEAD service requests are as follows:

\begin{lstlisting}
struct virtio_crypto_aead_data_flf_stateless {
    struct {
        /* See VIRTIO_CRYPTO_AEAD_* above */
        le32 algo;
        /* length of key */
        le32 key_len;
        /* encrypt or decrypt, See above VIRTIO_CRYPTO_OP_* */
        le32 op;
    } sess_para;

    /* Byte Length of valid IV data. */
    le32 iv_len;
    /* Authentication tag length */
    le32 tag_len;
    /* length of additional auth data */
    le32 aad_len;
    /* length of source data */
    le32 src_data_len;
    /* length of dst data, this should be at least src_data_len + tag_len */
    le32 dst_data_len;
};

struct virtio_crypto_aead_data_vlf_stateless {
    /* Device read only portion */

    /* The cipher key */
    u8 key[key_len];
    /* Initialization Vector data. */
    u8 iv[iv_len];
    /* Source data */
    u8 src_data[src_data_len];
    /* Additional authenticated data if exists */
    u8 aad[aad_len];

    /* Device write only portion */
    /* Pointer to output data */
    u8 dst_data[dst_data_len];
};
\end{lstlisting}

\drivernormative{\paragraph}{AEAD Service Operation}{Device Types / Crypto Device / Device Operation / AEAD Service Operation}

\begin{itemize*}
\item If the driver uses the session mode, then the driver MUST set
    \field{session_id} in struct virtio_crypto_op_header to a valid value assigned
    by the device when the session was created.
\item If the VIRTIO_CRYPTO_F_AEAD_STATELESS_MODE feature bit is negotiated, 1) if
    the driver uses the stateless mode, then the driver MUST set the \field{flag}
    field in struct virtio_crypto_op_header to ZERO and MUST set the fields in
    struct virtio_crypto_aead_data_flf_stateless.sess_para, 2) if the driver uses
    the session mode, then the driver MUST set the \field{flag} field in struct
    virtio_crypto_op_header to VIRTIO_CRYPTO_FLAG_SESSION_MODE.
\item The driver MUST set the \field{opcode} field in struct virtio_crypto_op_header
    to VIRTIO_CRYPTO_AEAD_ENCRYPT or VIRTIO_CRYPTO_AEAD_DECRYPT.
\end{itemize*}

\devicenormative{\paragraph}{AEAD Service Operation}{Device Types / Crypto Device / Device Operation / AEAD Service Operation}

\begin{itemize*}
\item If the VIRTIO_CRYPTO_F_AEAD_STATELESS_MODE feature bit is negotiated, the
    device MUST parse the virtio_crypto_aead_data_vlf_stateless based on the \field{opcode}
	field in general header.
\item The device MUST copy the result of cryptographic operation in the dst_data[].
\item The device MUST copy the authentication tag in the dst_data[] offset the cipher result.
\item The device MUST set the \field{status} field in struct virtio_crypto_inhdr to
    one of the following values of enum VIRTIO_CRYPTO_STATUS:
\item When the \field{opcode} field is VIRTIO_CRYPTO_AEAD_DECRYPT, the device MUST
    verify and return the verification result to the driver.
\begin{itemize*}
\item VIRTIO_CRYPTO_OK if the operation success.
\item VIRTIO_CRYPTO_NOTSUPP if the requested algorithm or operation is unsupported.
\item VIRTIO_CRYPTO_BADMSG if the verification result is incorrect.
\item VIRTIO_CRYPTO_INVSESS if the session ID invalid when in session mode.
\item VIRTIO_CRYPTO_ERR if any failure not mentioned above occurs.
\end{itemize*}
\end{itemize*}

\subsubsection{AKCIPHER Service Operation}\label{sec:Device Types / Crypto Device / Device Operation / AKCIPHER Service Operation}

Session mode AKCIPHER requests are as follows:

\begin{lstlisting}
struct virtio_crypto_akcipher_data_flf {
    /* length of source data */
    le32 src_data_len;
    /* length of dst data */
    le32 dst_data_len;
};

struct virtio_crypto_akcipher_data_vlf {
    /* Device read only portion */
    /* Source data */
    u8 src_data[src_data_len];

    /* Device write only portion */
    /* Pointer to output data */
    u8 dst_data[dst_data_len];
};
\end{lstlisting}

Each data request uses the virtio_crypto_akcipher_flf structure and the virtio_crypto_akcipher_data_vlf
structure to store information used to run the AKCIPHER operations.

For encryption, decryption, and signing:
\field{src_data} is the source data that will be processed, note that for signing operations,
src_data stores the data to be signed, which usually is the digest of some data rather than the
data itself.
\field{src_data_len} is the length of source data.
\field{dst_result} is the result data and \field{dst_data_len} is the length of it. Note that the
length of the result is not always exactly equal to dst_data_len, the driver needs to check how
many bytes the device has written and calculate the actual length of the result.

For verification:
\field{src_data_len} refers to the length of the signature, and \field{dst_data_len} refers to
the length of signed data, where the signed data is usually the digest of some data.
\field{src_data} is spliced by the signature and the signed data, the src_data with the lower
address stores the signature, and the higher address stores the signed data.
\field{dst_data} is always empty for verification.

Different algorithms have different signature formats.
For the RSA algorithm, the result is determined by the padding algorithm specified by
\field{padding_algo} in structure virtio_crypto_rsa_session_para.

For the ECDSA algorithm, the signature is composed of the following
ASN.1 structure (see \hyperref[intro:rfc3279]{RFC3279})
and MUST be DER encoded (see \hyperref[intro:rfc6025]{rfc6025}).

\begin{lstlisting}
Ecdsa-Sig-Value ::= SEQUENCE {
    r INTEGER,
    s INTEGER
}
\end{lstlisting}

Stateless mode AKCIPHER service requests are as follows:
\begin{lstlisting}
struct virtio_crypto_akcipher_data_flf_stateless {
    struct {
        /* See VIRTIO_CYRPTO_AKCIPHER* above */
        le32 algo;
        /* See VIRTIO_CRYPTO_AKCIPHER_KEY_TYPE_* above */
        le32 key_type;
        /* length of key */
        le32 key_len;

        /* algothrim specific parameters described above */
        union {
            struct virtio_crypto_rsa_session_para rsa;
            struct virtio_crypto_ecdsa_session_para ecdsa;
        } u;
    } sess_para;

    /* length of source data */
    le32 src_data_len;
    /* length of destination data */
    le32 dst_data_len;
};

struct virtio_crypto_akcipher_data_vlf_stateless {
    /* Device read only portion */
    u8 akcipher_key[key_len];

    /* Source data */
    u8 src_data[src_data_len];

    /* Device write only portion */
    u8 dst_data[dst_data_len];
};
\end{lstlisting}

In stateless mode, the format of key and signature, the meaning of src_data and dst_data, are all the same
with session mode.

\drivernormative{\paragraph}{AKCIPHER Service Operation}{Device Types / Crypto Device / Device Operation / AKCIPHER Service Operation}

\begin{itemize*}
\item If the driver uses the session mode, then the driver MUST set
    \field{session_id} in struct virtio_crypto_op_header to a valid
    value assigned by the device when the session was created.
\item If the VIRTIO_CRYPTO_F_AKCIPHER_STATELESS_MODE feature bit is negotiated, 1) if the
    driver uses the stateless mode, then the driver MUST set the \field{flag} field in
    struct virtio_crypto_op_header to ZERO and MUST set the fields in struct
    virtio_crypto_akcipher_flf_stateless.sess_para, 2) if the driver uses the session
    mode, then the driver MUST set the \field{flag} field in struct virtio_crypto_op_header
    to VIRTIO_CRYPTO_FLAG_SESSION_MODE.
\item The driver MUST set the \field{opcode} field in struct virtio_crypto_op_header
    to one of VIRTIO_CRYPTO_AKCIPHER_ENCRYPT, VIRTIO_CRYPTO_AKCIPHER_DESTROY_SESSION,
    VIRTIO_CRYPTO_AKCIPHER_SIGN, and VIRTIO_CRYPTO_AKCIPHER_VERIFY.
\end{itemize*}

\devicenormative{\paragraph}{AKCIPHER Service Operation}{Device Types / Crypto Device / Device Operation / AKCIPHER Service Operation}

\begin{itemize*}
\item If the VIRTIO_CRYPTO_F_AKCIPHER_STATELESS_MODE feature bit is negotiated, the
    device MUST parse the virtio_crypto_akcipher_data_vlf_stateless based on the \field{opcode}
    field in general header.
\item The device MUST copy the result of cryptographic operation in the dst_data[].
\item The device MUST set the \field{status} field in struct virtio_crypto_inhdr to
    one of the following values of enum VIRTIO_CRYPTO_STATUS:
\begin{itemize*}
\item VIRTIO_CRYPTO_OK if the operation success.
\item VIRTIO_CRYPTO_NOTSUPP if the requested algorithm or operation is unsupported.
\item VIRTIO_CRYPTO_BADMSG if the verification result is incorrect.
\item VIRTIO_CRYPTO_INVSESS if the session ID invalid when in session mode.
\item VIRTIO_CRYPTO_KEY_REJECTED if the signature verification failed.
\item VIRTIO_CRYPTO_ERR if any failure not mentioned above occurs.
\end{itemize*}
\end{itemize*}

\section{Crypto Device}\label{sec:Device Types / Crypto Device}

The virtio crypto device is a virtual cryptography device as well as a
virtual cryptographic accelerator. The virtio crypto device provides the
following crypto services: CIPHER, MAC, HASH, AEAD and AKCIPHER. Virtio crypto
devices have a single control queue and at least one data queue. Crypto
operation requests are placed into a data queue, and serviced by the
device. Some crypto operation requests are only valid in the context of a
session. The role of the control queue is facilitating control operation
requests. Sessions management is realized with control operation
requests.

\subsection{Device ID}\label{sec:Device Types / Crypto Device / Device ID}

20

\subsection{Virtqueues}\label{sec:Device Types / Crypto Device / Virtqueues}

\begin{description}
\item[0] dataq1
\item[\ldots]
\item[N-1] dataqN
\item[N] controlq
\end{description}

N is set by \field{max_dataqueues}.

\subsection{Feature bits}\label{sec:Device Types / Crypto Device / Feature bits}

\begin{description}
\item VIRTIO_CRYPTO_F_REVISION_1 (0) revision 1. Revision 1 has a specific
    request format and other enhancements (which result in some additional
    requirements).
\item VIRTIO_CRYPTO_F_CIPHER_STATELESS_MODE (1) stateless mode requests are
    supported by the CIPHER service.
\item VIRTIO_CRYPTO_F_HASH_STATELESS_MODE (2) stateless mode requests are
    supported by the HASH service.
\item VIRTIO_CRYPTO_F_MAC_STATELESS_MODE (3) stateless mode requests are
    supported by the MAC service.
\item VIRTIO_CRYPTO_F_AEAD_STATELESS_MODE (4) stateless mode requests are
    supported by the AEAD service.
\item VIRTIO_CRYPTO_F_AKCIPHER_STATELESS_MODE (5) stateless mode requests are
    supported by the AKCIPHER service.
\end{description}


\subsubsection{Feature bit requirements}\label{sec:Device Types / Crypto Device / Feature bit requirements}

Some crypto feature bits require other crypto feature bits
(see \ref{drivernormative:Basic Facilities of a Virtio Device / Feature Bits}):

\begin{description}
\item[VIRTIO_CRYPTO_F_CIPHER_STATELESS_MODE] Requires VIRTIO_CRYPTO_F_REVISION_1.
\item[VIRTIO_CRYPTO_F_HASH_STATELESS_MODE] Requires VIRTIO_CRYPTO_F_REVISION_1.
\item[VIRTIO_CRYPTO_F_MAC_STATELESS_MODE] Requires VIRTIO_CRYPTO_F_REVISION_1.
\item[VIRTIO_CRYPTO_F_AEAD_STATELESS_MODE] Requires VIRTIO_CRYPTO_F_REVISION_1.
\item[VIRTIO_CRYPTO_F_AKCIPHER_STATELESS_MODE] Requires VIRTIO_CRYPTO_F_REVISION_1.
\end{description}

\subsection{Supported crypto services}\label{sec:Device Types / Crypto Device / Supported crypto services}

The following crypto services are defined:

\begin{lstlisting}
/* CIPHER (Symmetric Key Cipher) service */
#define VIRTIO_CRYPTO_SERVICE_CIPHER 0
/* HASH service */
#define VIRTIO_CRYPTO_SERVICE_HASH   1
/* MAC (Message Authentication Codes) service */
#define VIRTIO_CRYPTO_SERVICE_MAC    2
/* AEAD (Authenticated Encryption with Associated Data) service */
#define VIRTIO_CRYPTO_SERVICE_AEAD   3
/* AKCIPHER (Asymmetric Key Cipher) service */
#define VIRTIO_CRYPTO_SERVICE_AKCIPHER 4
\end{lstlisting}

The above constants designate bits used to indicate the which of crypto services are
offered by the device as described in, see \ref{sec:Device Types / Crypto Device / Device configuration layout}.

\subsubsection{CIPHER services}\label{sec:Device Types / Crypto Device / Supported crypto services / CIPHER services}

The following CIPHER algorithms are defined:

\begin{lstlisting}
#define VIRTIO_CRYPTO_NO_CIPHER                 0
#define VIRTIO_CRYPTO_CIPHER_ARC4               1
#define VIRTIO_CRYPTO_CIPHER_AES_ECB            2
#define VIRTIO_CRYPTO_CIPHER_AES_CBC            3
#define VIRTIO_CRYPTO_CIPHER_AES_CTR            4
#define VIRTIO_CRYPTO_CIPHER_DES_ECB            5
#define VIRTIO_CRYPTO_CIPHER_DES_CBC            6
#define VIRTIO_CRYPTO_CIPHER_3DES_ECB           7
#define VIRTIO_CRYPTO_CIPHER_3DES_CBC           8
#define VIRTIO_CRYPTO_CIPHER_3DES_CTR           9
#define VIRTIO_CRYPTO_CIPHER_KASUMI_F8          10
#define VIRTIO_CRYPTO_CIPHER_SNOW3G_UEA2        11
#define VIRTIO_CRYPTO_CIPHER_AES_F8             12
#define VIRTIO_CRYPTO_CIPHER_AES_XTS            13
#define VIRTIO_CRYPTO_CIPHER_ZUC_EEA3           14
\end{lstlisting}

The above constants have two usages:
\begin{enumerate}
\item As bit numbers, used to tell the driver which CIPHER algorithms
are supported by the device, see \ref{sec:Device Types / Crypto Device / Device configuration layout}.
\item As values, used to designate the algorithm in (CIPHER type) crypto
operation requests, see \ref{sec:Device Types / Crypto Device / Device Operation / Control Virtqueue / Session operation}.
\end{enumerate}

\subsubsection{HASH services}\label{sec:Device Types / Crypto Device / Supported crypto services / HASH services}

The following HASH algorithms are defined:

\begin{lstlisting}
#define VIRTIO_CRYPTO_NO_HASH            0
#define VIRTIO_CRYPTO_HASH_MD5           1
#define VIRTIO_CRYPTO_HASH_SHA1          2
#define VIRTIO_CRYPTO_HASH_SHA_224       3
#define VIRTIO_CRYPTO_HASH_SHA_256       4
#define VIRTIO_CRYPTO_HASH_SHA_384       5
#define VIRTIO_CRYPTO_HASH_SHA_512       6
#define VIRTIO_CRYPTO_HASH_SHA3_224      7
#define VIRTIO_CRYPTO_HASH_SHA3_256      8
#define VIRTIO_CRYPTO_HASH_SHA3_384      9
#define VIRTIO_CRYPTO_HASH_SHA3_512      10
#define VIRTIO_CRYPTO_HASH_SHA3_SHAKE128      11
#define VIRTIO_CRYPTO_HASH_SHA3_SHAKE256      12
\end{lstlisting}

The above constants have two usages:
\begin{enumerate}
\item As bit numbers, used to tell the driver which HASH algorithms
are supported by the device, see \ref{sec:Device Types / Crypto Device / Device configuration layout}.
\item As values, used to designate the algorithm in (HASH type) crypto
operation requires, see \ref{sec:Device Types / Crypto Device / Device Operation / Control Virtqueue / Session operation}.
\end{enumerate}

\subsubsection{MAC services}\label{sec:Device Types / Crypto Device / Supported crypto services / MAC services}

The following MAC algorithms are defined:

\begin{lstlisting}
#define VIRTIO_CRYPTO_NO_MAC                       0
#define VIRTIO_CRYPTO_MAC_HMAC_MD5                 1
#define VIRTIO_CRYPTO_MAC_HMAC_SHA1                2
#define VIRTIO_CRYPTO_MAC_HMAC_SHA_224             3
#define VIRTIO_CRYPTO_MAC_HMAC_SHA_256             4
#define VIRTIO_CRYPTO_MAC_HMAC_SHA_384             5
#define VIRTIO_CRYPTO_MAC_HMAC_SHA_512             6
#define VIRTIO_CRYPTO_MAC_CMAC_3DES                25
#define VIRTIO_CRYPTO_MAC_CMAC_AES                 26
#define VIRTIO_CRYPTO_MAC_KASUMI_F9                27
#define VIRTIO_CRYPTO_MAC_SNOW3G_UIA2              28
#define VIRTIO_CRYPTO_MAC_GMAC_AES                 41
#define VIRTIO_CRYPTO_MAC_GMAC_TWOFISH             42
#define VIRTIO_CRYPTO_MAC_CBCMAC_AES               49
#define VIRTIO_CRYPTO_MAC_CBCMAC_KASUMI_F9         50
#define VIRTIO_CRYPTO_MAC_XCBC_AES                 53
#define VIRTIO_CRYPTO_MAC_ZUC_EIA3                 54
\end{lstlisting}

The above constants have two usages:
\begin{enumerate}
\item As bit numbers, used to tell the driver which MAC algorithms
are supported by the device, see \ref{sec:Device Types / Crypto Device / Device configuration layout}.
\item As values, used to designate the algorithm in (MAC type) crypto
operation requests, see \ref{sec:Device Types / Crypto Device / Device Operation / Control Virtqueue / Session operation}.
\end{enumerate}

\subsubsection{AEAD services}\label{sec:Device Types / Crypto Device / Supported crypto services / AEAD services}

The following AEAD algorithms are defined:

\begin{lstlisting}
#define VIRTIO_CRYPTO_NO_AEAD     0
#define VIRTIO_CRYPTO_AEAD_GCM    1
#define VIRTIO_CRYPTO_AEAD_CCM    2
#define VIRTIO_CRYPTO_AEAD_CHACHA20_POLY1305  3
\end{lstlisting}

The above constants have two usages:
\begin{enumerate}
\item As bit numbers, used to tell the driver which AEAD algorithms
are supported by the device, see \ref{sec:Device Types / Crypto Device / Device configuration layout}.
\item As values, used to designate the algorithm in (DEAD type) crypto
operation requests, see \ref{sec:Device Types / Crypto Device / Device Operation / Control Virtqueue / Session operation}.
\end{enumerate}

\subsubsection{AKCIPHER services}\label{sec: Device Types / Crypto Device / Supported crypto services / AKCIPHER services}

The following AKCIPHER algorithms are defined:
\begin{lstlisting}
#define VIRTIO_CRYPTO_NO_AKCIPHER 0
#define VIRTIO_CRYPTO_AKCIPHER_RSA   1
#define VIRTIO_CRYPTO_AKCIPHER_ECDSA 2
\end{lstlisting}

The above constants have two usages:
\begin{enumerate}
\item As bit numbers, used to tell the driver which AKCIPHER algorithms
are supported by the device, see \ref{sec:Device Types / Crypto Device / Device configuration layout}.
\item As values, used to designate the algorithm in asymmetric crypto operation requests,
see \ref{sec:Device Types / Crypto Device / Device Operation / Control Virtqueue / Session operation}.
\end{enumerate}


\subsection{Device configuration layout}\label{sec:Device Types / Crypto Device / Device configuration layout}

Crypto device configuration uses the following layout structure:

\begin{lstlisting}
struct virtio_crypto_config {
    le32 status;
    le32 max_dataqueues;
    le32 crypto_services;
    /* Detailed algorithms mask */
    le32 cipher_algo_l;
    le32 cipher_algo_h;
    le32 hash_algo;
    le32 mac_algo_l;
    le32 mac_algo_h;
    le32 aead_algo;
    /* Maximum length of cipher key in bytes */
    le32 max_cipher_key_len;
    /* Maximum length of authenticated key in bytes */
    le32 max_auth_key_len;
    le32 akcipher_algo;
    /* Maximum size of each crypto request's content in bytes */
    le64 max_size;
};
\end{lstlisting}

\begin{description}
\item Currently, only one \field{status} bit is defined: VIRTIO_CRYPTO_S_HW_READY
    set indicates that the device is ready to process requests, this bit is read-only
    for the driver
\begin{lstlisting}
#define VIRTIO_CRYPTO_S_HW_READY  (1 << 0)
\end{lstlisting}

\item [\field{max_dataqueues}] is the maximum number of data virtqueues that can
    be configured by the device. The driver MAY use only one data queue, or it
    can use more to achieve better performance.

\item [\field{crypto_services}] crypto service offered, see \ref{sec:Device Types / Crypto Device / Supported crypto services}.

\item [\field{cipher_algo_l}] CIPHER algorithms bits 0-31, see \ref{sec:Device Types / Crypto Device / Supported crypto services  / CIPHER services}.

\item [\field{cipher_algo_h}] CIPHER algorithms bits 32-63, see \ref{sec:Device Types / Crypto Device / Supported crypto services  / CIPHER services}.

\item [\field{hash_algo}] HASH algorithms bits, see \ref{sec:Device Types / Crypto Device / Supported crypto services  / HASH services}.

\item [\field{mac_algo_l}] MAC algorithms bits 0-31, see \ref{sec:Device Types / Crypto Device / Supported crypto services  / MAC services}.

\item [\field{mac_algo_h}] MAC algorithms bits 32-63, see \ref{sec:Device Types / Crypto Device / Supported crypto services  / MAC services}.

\item [\field{aead_algo}] AEAD algorithms bits, see \ref{sec:Device Types / Crypto Device / Supported crypto services  / AEAD services}.

\item [\field{max_cipher_key_len}] is the maximum length of cipher key supported by the device.

\item [\field{max_auth_key_len}] is the maximum length of authenticated key supported by the device.

\item [\field{akcipher_algo}] AKCIPHER algorithms bit 0-31, see \ref{sec: Device Types / Crypto Device / Supported crypto services / AKCIPHER services}.

\item [\field{max_size}] is the maximum size of the variable-length parameters of
    data operation of each crypto request's content supported by the device.
\end{description}

\begin{note}
Unless explicitly stated otherwise all lengths and sizes are in bytes.
\end{note}

\devicenormative{\subsubsection}{Device configuration layout}{Device Types / Crypto Device / Device configuration layout}

\begin{itemize*}
\item The device MUST set \field{max_dataqueues} to between 1 and 65535 inclusive.
\item The device MUST set the \field{status} with valid flags, undefined flags MUST NOT be set.
\item The device MUST accept and handle requests after \field{status} is set to VIRTIO_CRYPTO_S_HW_READY.
\item The device MUST set \field{crypto_services} based on the crypto services the device offers.
\item The device MUST set detailed algorithms masks for each service advertised by \field{crypto_services}.
    The device MUST NOT set the not defined algorithms bits.
\item The device MUST set \field{max_size} to show the maximum size of crypto request the device supports.
\item The device MUST set \field{max_cipher_key_len} to show the maximum length of cipher key if the
    device supports CIPHER service.
\item The device MUST set \field{max_auth_key_len} to show the maximum length of authenticated key if
    the device supports MAC service.
\end{itemize*}

\drivernormative{\subsubsection}{Device configuration layout}{Device Types / Crypto Device / Device configuration layout}

\begin{itemize*}
\item The driver MUST read the \field{status} from the bottom bit of status to check whether the
    VIRTIO_CRYPTO_S_HW_READY is set, and the driver MUST reread it after device reset.
\item The driver MUST NOT transmit any requests to the device if the VIRTIO_CRYPTO_S_HW_READY is not set.
\item The driver MUST read \field{max_dataqueues} field to discover the number of data queues the device supports.
\item The driver MUST read \field{crypto_services} field to discover which services the device is able to offer.
\item The driver SHOULD ignore the not defined algorithms bits.
\item The driver MUST read the detailed algorithms fields based on \field{crypto_services} field.
\item The driver SHOULD read \field{max_size} to discover the maximum size of the variable-length
    parameters of data operation of the crypto request's content the device supports and MUST
    guarantee that the size of each crypto request's content is within the \field{max_size}, otherwise
    the request will fail and the driver MUST reset the device.
\item The driver SHOULD read \field{max_cipher_key_len} to discover the maximum length of cipher key
    the device supports and MUST guarantee that the \field{key_len} (CIPHER service or AEAD service) is within
    the \field{max_cipher_key_len} of the device configuration, otherwise the request will fail.
\item The driver SHOULD read \field{max_auth_key_len} to discover the maximum length of authenticated
    key the device supports and MUST guarantee that the \field{auth_key_len} (MAC service) is within the
    \field{max_auth_key_len} of the device configuration, otherwise the request will fail.
\end{itemize*}

\subsection{Device Initialization}\label{sec:Device Types / Crypto Device / Device Initialization}

\drivernormative{\subsubsection}{Device Initialization}{Device Types / Crypto Device / Device Initialization}

\begin{itemize*}
\item The driver MUST configure and initialize all virtqueues.
\item The driver MUST read the supported crypto services from bits of \field{crypto_services}.
\item The driver MUST read the supported algorithms based on \field{crypto_services} field.
\end{itemize*}

\subsection{Device Operation}\label{sec:Device Types / Crypto Device / Device Operation}

The operation of a virtio crypto device is driven by requests placed on the virtqueues.
Requests consist of a queue-type specific header (specifying among others the operation)
and an operation specific payload.

If VIRTIO_CRYPTO_F_REVISION_1 is negotiated the device may support both session mode
(See \ref{sec:Device Types / Crypto Device / Device Operation / Control Virtqueue / Session operation})
and stateless mode operation requests.
In stateless mode all operation parameters are supplied as a part of each request,
while in session mode, some or all operation parameters are managed within the
session. Stateless mode is guarded by feature bits 0-4 on a service level. If
stateless mode is negotiated for a service, the service accepts both session
mode and stateless requests; otherwise stateless mode requests are rejected
(via operation status).

\subsubsection{Operation Status}\label{sec:Device Types / Crypto Device / Device Operation / Operation status}
The device MUST return a status code as part of the operation (both session
operation and service operation) result. The valid operation status as follows:

\begin{lstlisting}
enum VIRTIO_CRYPTO_STATUS {
    VIRTIO_CRYPTO_OK = 0,
    VIRTIO_CRYPTO_ERR = 1,
    VIRTIO_CRYPTO_BADMSG = 2,
    VIRTIO_CRYPTO_NOTSUPP = 3,
    VIRTIO_CRYPTO_INVSESS = 4,
    VIRTIO_CRYPTO_NOSPC = 5,
    VIRTIO_CRYPTO_KEY_REJECTED = 6,
    VIRTIO_CRYPTO_MAX
};
\end{lstlisting}

\begin{itemize*}
\item VIRTIO_CRYPTO_OK: success.
\item VIRTIO_CRYPTO_BADMSG: authentication failed (only when AEAD decryption).
\item VIRTIO_CRYPTO_NOTSUPP: operation or algorithm is unsupported.
\item VIRTIO_CRYPTO_INVSESS: invalid session ID when executing crypto operations.
\item VIRTIO_CRYPTO_NOSPC: no free session ID (only when the VIRTIO_CRYPTO_F_REVISION_1
    feature bit is negotiated).
\item VIRTIO_CRYPTO_KEY_REJECTED: signature verification failed (only when AKCIPHER verification).
\item VIRTIO_CRYPTO_ERR: any failure not mentioned above occurs.
\end{itemize*}

\subsubsection{Control Virtqueue}\label{sec:Device Types / Crypto Device / Device Operation / Control Virtqueue}

The driver uses the control virtqueue to send control commands to the
device, such as session operations (See \ref{sec:Device Types / Crypto Device / Device
Operation / Control Virtqueue / Session operation}).

The header for controlq is of the following form:
\begin{lstlisting}
#define VIRTIO_CRYPTO_OPCODE(service, op)   (((service) << 8) | (op))

struct virtio_crypto_ctrl_header {
#define VIRTIO_CRYPTO_CIPHER_CREATE_SESSION \
       VIRTIO_CRYPTO_OPCODE(VIRTIO_CRYPTO_SERVICE_CIPHER, 0x02)
#define VIRTIO_CRYPTO_CIPHER_DESTROY_SESSION \
       VIRTIO_CRYPTO_OPCODE(VIRTIO_CRYPTO_SERVICE_CIPHER, 0x03)
#define VIRTIO_CRYPTO_HASH_CREATE_SESSION \
       VIRTIO_CRYPTO_OPCODE(VIRTIO_CRYPTO_SERVICE_HASH, 0x02)
#define VIRTIO_CRYPTO_HASH_DESTROY_SESSION \
       VIRTIO_CRYPTO_OPCODE(VIRTIO_CRYPTO_SERVICE_HASH, 0x03)
#define VIRTIO_CRYPTO_MAC_CREATE_SESSION \
       VIRTIO_CRYPTO_OPCODE(VIRTIO_CRYPTO_SERVICE_MAC, 0x02)
#define VIRTIO_CRYPTO_MAC_DESTROY_SESSION \
       VIRTIO_CRYPTO_OPCODE(VIRTIO_CRYPTO_SERVICE_MAC, 0x03)
#define VIRTIO_CRYPTO_AEAD_CREATE_SESSION \
       VIRTIO_CRYPTO_OPCODE(VIRTIO_CRYPTO_SERVICE_AEAD, 0x02)
#define VIRTIO_CRYPTO_AEAD_DESTROY_SESSION \
       VIRTIO_CRYPTO_OPCODE(VIRTIO_CRYPTO_SERVICE_AEAD, 0x03)
#define VIRTIO_CRYPTO_AKCIPHER_CREATE_SESSION \
       VIRTIO_CRYPTO_OPCODE(VIRTIO_CRYPTO_SERVICE_AKCIPHER, 0x04)
#define VIRTIO_CRYPTO_AKCIPHER_DESTROY_SESSION \
       VIRTIO_CRYPTO_OPCDE(VIRTIO_CRYPTO_SERVICE_AKCIPHER, 0x05)
    le32 opcode;
    /* algo should be service-specific algorithms */
    le32 algo;
    le32 flag;
    le32 reserved;
};
\end{lstlisting}

The controlq request is composed of four parts:
\begin{lstlisting}
struct virtio_crypto_op_ctrl_req {
    /* Device read only portion */

    struct virtio_crypto_ctrl_header header;

#define VIRTIO_CRYPTO_CTRLQ_OP_SPEC_HDR_LEGACY 56
    /* fixed length fields, opcode specific */
    u8 op_flf[flf_len];

    /* variable length fields, opcode specific */
    u8 op_vlf[vlf_len];

    /* Device write only portion */

    /* op result or completion status */
    u8 op_outcome[outcome_len];
};
\end{lstlisting}

\field{header} is a general header (see above).

\field{op_flf} is the opcode (in \field{header}) specific fixed-length parameters.

\field{flf_len} depends on the VIRTIO_CRYPTO_F_REVISION_1 feature bit (see below).

\field{op_vlf} is the opcode (in \field{header}) specific variable-length parameters.

\field{vlf_len} is the size of the specific structure used.
\begin{note}
The \field{vlf_len} of session-destroy operation and the hash-session-create
operation is ZERO.
\end{note}

\begin{itemize*}
\item If the opcode (in \field{header}) is VIRTIO_CRYPTO_CIPHER_CREATE_SESSION
    then \field{op_flf} is struct virtio_crypto_sym_create_session_flf if
    VIRTIO_CRYPTO_F_REVISION_1 is negotiated and struct virtio_crypto_sym_create_session_flf is
    padded to 56 bytes if NOT negotiated, and \field{op_vlf} is struct
    virtio_crypto_sym_create_session_vlf.
\item If the opcode (in \field{header}) is VIRTIO_CRYPTO_HASH_CREATE_SESSION
    then \field{op_flf} is struct virtio_crypto_hash_create_session_flf if
    VIRTIO_CRYPTO_F_REVISION_1 is negotiated and struct virtio_crypto_hash_create_session_flf is
    padded to 56 bytes if NOT negotiated.
\item If the opcode (in \field{header}) is VIRTIO_CRYPTO_MAC_CREATE_SESSION
    then \field{op_flf} is struct virtio_crypto_mac_create_session_flf if
    VIRTIO_CRYPTO_F_REVISION_1 is negotiated and struct virtio_crypto_mac_create_session_flf is
    padded to 56 bytes if NOT negotiated, and \field{op_vlf} is struct
    virtio_crypto_mac_create_session_vlf.
\item If the opcode (in \field{header}) is VIRTIO_CRYPTO_AEAD_CREATE_SESSION
    then \field{op_flf} is struct virtio_crypto_aead_create_session_flf if
    VIRTIO_CRYPTO_F_REVISION_1 is negotiated and struct virtio_crypto_aead_create_session_flf is
    padded to 56 bytes if NOT negotiated, and \field{op_vlf} is struct
    virtio_crypto_aead_create_session_vlf.
\item If the opcode (in \field{header}) is VIRTIO_CRYPTO_AKCIPHER_CREATE_SESSION
    then \field{op_flf} is struct virtio_crypto_akcipher_create_session_flf if
    VIRTIO_CRYPTO_F_REVISION_1 is negotiated and struct virtio_crypto_akcipher_create_session_flf is
    padded to 56 bytes if NOT negotiated, and \field{op_vlf} is struct
    virtio_crypto_akcipher_create_session_vlf.
\item If the opcode (in \field{header}) is VIRTIO_CRYPTO_CIPHER_DESTROY_SESSION
    or VIRTIO_CRYPTO_HASH_DESTROY_SESSION or VIRTIO_CRYPTO_MAC_DESTROY_SESSION or
    VIRTIO_CRYPTO_AEAD_DESTROY_SESSION then \field{op_flf} is struct
    virtio_crypto_destroy_session_flf if VIRTIO_CRYPTO_F_REVISION_1 is negotiated and
    struct virtio_crypto_destroy_session_flf is padded to 56 bytes if NOT negotiated.
\end{itemize*}

\field{op_outcome} stores the result of operation and must be struct
virtio_crypto_destroy_session_input for destroy session or
struct virtio_crypto_create_session_input for create session.

\field{outcome_len} is the size of the structure used.


\paragraph{Session operation}\label{sec:Device Types / Crypto Device / Device
Operation / Control Virtqueue / Session operation}

The session is a handle which describes the cryptographic parameters to be
applied to a number of buffers.

The following structure stores the result of session creation set by the device:

\begin{lstlisting}
struct virtio_crypto_create_session_input {
    le64 session_id;
    le32 status;
    le32 padding;
};
\end{lstlisting}

A request to destroy a session includes the following information:

\begin{lstlisting}
struct virtio_crypto_destroy_session_flf {
    /* Device read only portion */
    le64  session_id;
};

struct virtio_crypto_destroy_session_input {
    /* Device write only portion */
    u8  status;
};
\end{lstlisting}


\subparagraph{Session operation: HASH session}\label{sec:Device Types / Crypto Device / Device
Operation / Control Virtqueue / Session operation / Session operation: HASH session}

The fixed-length parameters of HASH session requests is as follows:

\begin{lstlisting}
struct virtio_crypto_hash_create_session_flf {
    /* Device read only portion */

    /* See VIRTIO_CRYPTO_HASH_* above */
    le32 algo;
    /* hash result length */
    le32 hash_result_len;
};
\end{lstlisting}


\subparagraph{Session operation: MAC session}\label{sec:Device Types / Crypto Device / Device
Operation / Control Virtqueue / Session operation / Session operation: MAC session}

The fixed-length and the variable-length parameters of MAC session requests are as follows:

\begin{lstlisting}
struct virtio_crypto_mac_create_session_flf {
    /* Device read only portion */

    /* See VIRTIO_CRYPTO_MAC_* above */
    le32 algo;
    /* hash result length */
    le32 hash_result_len;
    /* length of authenticated key */
    le32 auth_key_len;
    le32 padding;
};

struct virtio_crypto_mac_create_session_vlf {
    /* Device read only portion */

    /* The authenticated key */
    u8 auth_key[auth_key_len];
};
\end{lstlisting}

The length of \field{auth_key} is specified in \field{auth_key_len} in the struct
virtio_crypto_mac_create_session_flf.


\subparagraph{Session operation: Symmetric algorithms session}\label{sec:Device Types / Crypto Device / Device
Operation / Control Virtqueue / Session operation / Session operation: Symmetric algorithms session}

The request of symmetric session could be the CIPHER algorithms request
or the chain algorithms (chaining CIPHER and HASH/MAC) request.

The fixed-length and the variable-length parameters of CIPHER session requests are as follows:

\begin{lstlisting}
struct virtio_crypto_cipher_session_flf {
    /* Device read only portion */

    /* See VIRTIO_CRYPTO_CIPHER* above */
    le32 algo;
    /* length of key */
    le32 key_len;
#define VIRTIO_CRYPTO_OP_ENCRYPT  1
#define VIRTIO_CRYPTO_OP_DECRYPT  2
    /* encryption or decryption */
    le32 op;
    le32 padding;
};

struct virtio_crypto_cipher_session_vlf {
    /* Device read only portion */

    /* The cipher key */
    u8 cipher_key[key_len];
};
\end{lstlisting}

The length of \field{cipher_key} is specified in \field{key_len} in the struct
virtio_crypto_cipher_session_flf.

The fixed-length and the variable-length parameters of Chain session requests are as follows:

\begin{lstlisting}
struct virtio_crypto_alg_chain_session_flf {
    /* Device read only portion */

#define VIRTIO_CRYPTO_SYM_ALG_CHAIN_ORDER_HASH_THEN_CIPHER  1
#define VIRTIO_CRYPTO_SYM_ALG_CHAIN_ORDER_CIPHER_THEN_HASH  2
    le32 alg_chain_order;
/* Plain hash */
#define VIRTIO_CRYPTO_SYM_HASH_MODE_PLAIN    1
/* Authenticated hash (mac) */
#define VIRTIO_CRYPTO_SYM_HASH_MODE_AUTH     2
/* Nested hash */
#define VIRTIO_CRYPTO_SYM_HASH_MODE_NESTED   3
    le32 hash_mode;
    struct virtio_crypto_cipher_session_flf cipher_hdr;

#define VIRTIO_CRYPTO_ALG_CHAIN_SESS_OP_SPEC_HDR_SIZE  16
    /* fixed length fields, algo specific */
    u8 algo_flf[VIRTIO_CRYPTO_ALG_CHAIN_SESS_OP_SPEC_HDR_SIZE];

    /* length of the additional authenticated data (AAD) in bytes */
    le32 aad_len;
    le32 padding;
};

struct virtio_crypto_alg_chain_session_vlf {
    /* Device read only portion */

    /* The cipher key */
    u8 cipher_key[key_len];
    /* The authenticated key */
    u8 auth_key[auth_key_len];
};
\end{lstlisting}

\field{hash_mode} decides the type used by \field{algo_flf}.

\field{algo_flf} is fixed to 16 bytes and MUST contains or be one of
the following types:
\begin{itemize*}
\item struct virtio_crypto_hash_create_session_flf
\item struct virtio_crypto_mac_create_session_flf
\end{itemize*}
The data of unused part (if has) in \field{algo_flf} will be ignored.

The length of \field{cipher_key} is specified in \field{key_len} in \field{cipher_hdr}.

The length of \field{auth_key} is specified in \field{auth_key_len} in struct
virtio_crypto_mac_create_session_flf.

The fixed-length parameters of Symmetric session requests are as follows:

\begin{lstlisting}
struct virtio_crypto_sym_create_session_flf {
    /* Device read only portion */

#define VIRTIO_CRYPTO_SYM_SESS_OP_SPEC_HDR_SIZE  48
    /* fixed length fields, opcode specific */
    u8 op_flf[VIRTIO_CRYPTO_SYM_SESS_OP_SPEC_HDR_SIZE];

/* No operation */
#define VIRTIO_CRYPTO_SYM_OP_NONE  0
/* Cipher only operation on the data */
#define VIRTIO_CRYPTO_SYM_OP_CIPHER  1
/* Chain any cipher with any hash or mac operation. The order
   depends on the value of alg_chain_order param */
#define VIRTIO_CRYPTO_SYM_OP_ALGORITHM_CHAINING  2
    le32 op_type;
    le32 padding;
};
\end{lstlisting}

\field{op_flf} is fixed to 48 bytes, MUST contains or be one of
the following types:
\begin{itemize*}
\item struct virtio_crypto_cipher_session_flf
\item struct virtio_crypto_alg_chain_session_flf
\end{itemize*}
The data of unused part (if has) in \field{op_flf} will be ignored.

\field{op_type} decides the type used by \field{op_flf}.

The variable-length parameters of Symmetric session requests are as follows:

\begin{lstlisting}
struct virtio_crypto_sym_create_session_vlf {
    /* Device read only portion */
    /* variable length fields, opcode specific */
    u8 op_vlf[vlf_len];
};
\end{lstlisting}

\field{op_vlf} MUST contains or be one of the following types:
\begin{itemize*}
\item struct virtio_crypto_cipher_session_vlf
\item struct virtio_crypto_alg_chain_session_vlf
\end{itemize*}

\field{op_type} in struct virtio_crypto_sym_create_session_flf decides the
type used by \field{op_vlf}.

\field{vlf_len} is the size of the specific structure used.


\subparagraph{Session operation: AEAD session}\label{sec:Device Types / Crypto Device / Device
Operation / Control Virtqueue / Session operation / Session operation: AEAD session}

The fixed-length and the variable-length parameters of AEAD session requests are as follows:

\begin{lstlisting}
struct virtio_crypto_aead_create_session_flf {
    /* Device read only portion */

    /* See VIRTIO_CRYPTO_AEAD_* above */
    le32 algo;
    /* length of key */
    le32 key_len;
    /* Authentication tag length */
    le32 tag_len;
    /* The length of the additional authenticated data (AAD) in bytes */
    le32 aad_len;
    /* encryption or decryption, See above VIRTIO_CRYPTO_OP_* */
    le32 op;
    le32 padding;
};

struct virtio_crypto_aead_create_session_vlf {
    /* Device read only portion */
    u8 key[key_len];
};
\end{lstlisting}

The length of \field{key} is specified in \field{key_len} in struct
virtio_crypto_aead_create_session_flf.

\subparagraph{Session operation: AKCIPHER session}\label{sec:Device Types / Crypto Device / Device
Operation / Control Virtqueue / Session operation / Session operation: AKCIPHER session}

Due to the complexity of asymmetric key algorithms, different algorithms
require different parameters. The following data structures are used as
supplementary parameters to describe the asymmetric algorithm sessions.

For the RSA algorithm, the extra parameters are as follows:
\begin{lstlisting}
struct virtio_crypto_rsa_session_para {
#define VIRTIO_CRYPTO_RSA_RAW_PADDING   0
#define VIRTIO_CRYPTO_RSA_PKCS1_PADDING 1
    le32 padding_algo;

#define VIRTIO_CRYPTO_RSA_NO_HASH   0
#define VIRTIO_CRYPTO_RSA_MD2       1
#define VIRTIO_CRYPTO_RSA_MD3       2
#define VIRTIO_CRYPTO_RSA_MD4       3
#define VIRTIO_CRYPTO_RSA_MD5       4
#define VIRTIO_CRYPTO_RSA_SHA1      5
#define VIRTIO_CRYPTO_RSA_SHA256    6
#define VIRTIO_CRYPTO_RSA_SHA384    7
#define VIRTIO_CRYPTO_RSA_SHA512    8
#define VIRTIO_CRYPTO_RSA_SHA224    9
    le32 hash_algo;
};
\end{lstlisting}

\field{padding_algo} specifies the padding method used by RSA sessions.
\begin{itemize*}
\item If VIRTIO_CRYPTO_RSA_RAW_PADDING is specified, 1) \field{hash_algo}
is ignored, 2) ciphertext and plaintext MUST be padded with leading zeros,
3) and RSA sessions with VIRTIO_CRYPTO_RSA_RAW_PADDING MUST not be used
for verification and signing operations.
\item If VIRTIO_CRYPTO_RSA_PKCS1_PADDING is specified, EMSA-PKCS1-v1_5 padding method
is used (see \hyperref[intro:rfc3447]{PKCS\#1}), \field{hash_algo} specifies how the
digest of the data passed to RSA sessions is calculated when verifying and signing.
It only affects the padding algorithm and is ignored during encryption and decryption.
\end{itemize*}

The ECC algorithms such as the ECDSA algorithm, cannot use custom curves, only the
following known curves can be used (see \hyperref[intro:NIST]{NIST-recommended curves}).

\begin{lstlisting}
#define VIRTIO_CRYPTO_CURVE_UNKNOWN   0
#define VIRTIO_CRYPTO_CURVE_NIST_P192 1
#define VIRTIO_CRYPTO_CURVE_NIST_P224 2
#define VIRTIO_CRYPTO_CURVE_NIST_P256 3
#define VIRTIO_CRYPTO_CURVE_NIST_P384 4
#define VIRTIO_CRYPTO_CURVE_NIST_P521 5
\end{lstlisting}

For the ECDSA algorithm, the extra parameters are as follows:
\begin{lstlisting}
struct virtio_crypto_ecdsa_session_para {
    /* See VIRTIO_CRYPTO_CURVE_* above */
    le32 curve_id;
};
\end{lstlisting}

The fixed-length and the variable-length parameters of AKCIPHER session requests are as follows:
\begin{lstlisting}
struct virtio_crypto_akcipher_create_session_flf {
    /* Device read only portion */

    /* See VIRTIO_CRYPTO_AKCIPHER_* above */
    le32 algo;
#define VIRTIO_CRYPTO_AKCIPHER_KEY_TYPE_PUBLIC 1
#define VIRTIO_CRYPTO_AKCIPHER_KEY_TYPE_PRIVATE 2
    le32 key_type;
    /* length of key */
    le32 key_len;

#define VIRTIO_CRYPTO_AKCIPHER_SESS_ALGO_SPEC_HDR_SIZE 44
    u8 algo_flf[VIRTIO_CRYPTO_AKCIPHER_SESS_ALGO_SPEC_HDR_SIZE];
};

struct virtio_crypto_akcipher_create_session_vlf {
    /* Device read only portion */
    u8 key[key_len];
};
\end{lstlisting}

\field{algo} decides the type used by \field{algo_flf}.
\field{algo_flf} is fixed to 44 bytes and MUST contains of be one the
following structures:
\begin{itemize*}
\item struct virtio_crypto_rsa_session_para
\item struct virtio_crypto_ecdsa_session_para
\end{itemize*}

The length of \field{key} is specified in \field{key_len} in the struct
virtio_crypto_akcipher_create_session_flf.

For the RSA algorithm, the key needs to be encoded according to
\hyperref[intro:rfc3447]{PKCS\#1}. The private key is described with the
RSAPrivateKey structure, and the public key is described with the RSAPublicKey
structure. These ASN.1 structures are encoded in DER encoding rules (see
\hyperref[intro:rfc6025]{rfc6025}).

\begin{lstlisting}
RSAPrivateKey ::= SEQUENCE {
    version          INTEGER,
    modulus          INTEGER,
    publicExponent   INTEGER,
    privateExponent  INTEGER,
    prime1           INTEGER,
    prime2           INTEGER,
    exponent1        INTEGER,
    exponent1        INTEGER,
    coefficient      INTEGER,
    otherPrimeInfos  OtherPrimeInfos OPTIONAL
}

OtherPrimeInfos ::= SEQUENCE SIZE(1...MAX) OF OtherPrimeInfo

OtherPrimeINfo ::= SEQUENCE {
    prime           INTEGER,
    exponent        INTEGER,
    coefficient     INTEGER
}

RSAPublicKey ::= SEQUENCE {
    modulus         INTEGER,
    publicExponent  INTEGER
}
\end{lstlisting}

For the ECDSA algorithm, the private key is encoded according to
\hyperref[intro:rfc5915]{RFC5915}, the private key of the ECDSA algorithm
is described by the ASN.1 structure ECPrivateKey and encoded with DER
encoding rules (see \hyperref[intro:rfc6025]{rfc6025}).

\begin{lstlisting}
ECPrivateKey ::= SEQUNCE {
    version         INTEGER,
    privateKey      OCTET STRING,
    parameters [0]  ECParameters {{ NamedCurve }} OPTIONAL,
    publicKey  [1]  BIT STRING OPTIONAL
}
\end{lstlisting}

The public key of the ECDSA algorithm is encoded according to \hyperref[intro:SEC1]{SEC1},
and the public key of ECDSA is described by the ASN.1 structure ECPoint.
When initializing a session with ECDSA public key, the ECPoint is DER encoded and the
\field{key} only contains the value part of ECPoint, that is, the header part of the
OCTET STRING will be omitted (see \hyperref[intro:rfc6025]{rfc6025}).

\begin{lstlisting}
ECPoint ::= OCTET STRING
\end{lstlisting}

The length of \field{key} is specified in \field{key_len} in
struct virtio_crypto_akcipher_create_session_flf.

\drivernormative{\subparagraph}{Session operation: create session}{Device Types / Crypto Device / Device
Operation / Control Virtqueue / Session operation / Session operation: create session}

\begin{itemize*}
\item The driver MUST set the \field{opcode} field based on service type: CIPHER, HASH, MAC, AEAD or AKCIPHER.
\item The driver MUST set the control general header, the opcode specific header,
    the opcode specific extra parameters and the opcode specific outcome buffer in turn.
    See \ref{sec:Device Types / Crypto Device / Device Operation / Control Virtqueue}.
\item The driver MUST set the \field{reversed} field to zero.
\end{itemize*}

\devicenormative{\subparagraph}{Session operation: create session}{Device Types / Crypto Device / Device
Operation / Control Virtqueue / Session operation / Session operation: create session}

\begin{itemize*}
\item The device MUST use the corresponding opcode specific structure according to the
    \field{opcode} in the control general header.
\item The device MUST extract extra parameters according to the structures used.
\item The device MUST set the \field{status} field to one of the following values of enum
    VIRTIO_CRYPTO_STATUS after finish a session creation:
\begin{itemize*}
\item VIRTIO_CRYPTO_OK if a session is created successfully.
\item VIRTIO_CRYPTO_NOTSUPP if the requested algorithm or operation is unsupported.
\item VIRTIO_CRYPTO_NOSPC if no free session ID (only when the VIRTIO_CRYPTO_F_REVISION_1
    feature bit is negotiated).
\item VIRTIO_CRYPTO_ERR if failure not mentioned above occurs.
\end{itemize*}
\item The device MUST set the \field{session_id} field to a unique session identifier only
    if the status is set to VIRTIO_CRYPTO_OK.
\end{itemize*}

\drivernormative{\subparagraph}{Session operation: destroy session}{Device Types / Crypto Device / Device
Operation / Control Virtqueue / Session operation / Session operation: destroy session}

\begin{itemize*}
\item The driver MUST set the \field{opcode} field based on service type: CIPHER, HASH, MAC, AEAD or AKCIPHER.
\item The driver MUST set the \field{session_id} to a valid value assigned by the device
    when the session was created.
\end{itemize*}

\devicenormative{\subparagraph}{Session operation: destroy session}{Device Types / Crypto Device / Device
Operation / Control Virtqueue / Session operation / Session operation: destroy session}

\begin{itemize*}
\item The device MUST set the \field{status} field to one of the following values of enum VIRTIO_CRYPTO_STATUS.
\begin{itemize*}
\item VIRTIO_CRYPTO_OK if a session is created successfully.
\item VIRTIO_CRYPTO_ERR if any failure occurs.
\end{itemize*}
\end{itemize*}


\subsubsection{Data Virtqueue}\label{sec:Device Types / Crypto Device / Device Operation / Data Virtqueue}

The driver uses the data virtqueues to transmit crypto operation requests to the device,
and completes the crypto operations.

The header for dataq is as follows:

\begin{lstlisting}
struct virtio_crypto_op_header {
#define VIRTIO_CRYPTO_CIPHER_ENCRYPT \
    VIRTIO_CRYPTO_OPCODE(VIRTIO_CRYPTO_SERVICE_CIPHER, 0x00)
#define VIRTIO_CRYPTO_CIPHER_DECRYPT \
    VIRTIO_CRYPTO_OPCODE(VIRTIO_CRYPTO_SERVICE_CIPHER, 0x01)
#define VIRTIO_CRYPTO_HASH \
    VIRTIO_CRYPTO_OPCODE(VIRTIO_CRYPTO_SERVICE_HASH, 0x00)
#define VIRTIO_CRYPTO_MAC \
    VIRTIO_CRYPTO_OPCODE(VIRTIO_CRYPTO_SERVICE_MAC, 0x00)
#define VIRTIO_CRYPTO_AEAD_ENCRYPT \
    VIRTIO_CRYPTO_OPCODE(VIRTIO_CRYPTO_SERVICE_AEAD, 0x00)
#define VIRTIO_CRYPTO_AEAD_DECRYPT \
    VIRTIO_CRYPTO_OPCODE(VIRTIO_CRYPTO_SERVICE_AEAD, 0x01)
#define VIRTIO_CRYPTO_AKCIPHER_ENCRYPT \
    VIRTIO_CRYPTO_OPCODE(VIRTIO_CRYPTO_SERVICE_AKCIPHER, 0x00)
#define VIRTIO_CRYPTO_AKCIPHER_DECRYPT \
    VIRTIO_CRYPTO_OPCODE(VIRTIO_CRYPTO_SERVICE_AKCIPHER, 0x01)
#define VIRTIO_CRYPTO_AKCIPHER_SIGN \
    VIRTIO_CRYPTO_OPCODE(VIRTIO_CRYPTO_SERVICE_AKCIPHER, 0x02)
#define VIRTIO_CRYPTO_AKCIPHER_VERIFY \
    VIRTIO_CRYPTO_OPCODE(VIRTIO_CRYPTO_SERVICE_AKCIPHER, 0x03)
    le32 opcode;
    /* algo should be service-specific algorithms */
    le32 algo;
    le64 session_id;
#define VIRTIO_CRYPTO_FLAG_SESSION_MODE 1
    /* control flag to control the request */
    le32 flag;
    le32 padding;
};
\end{lstlisting}

\begin{note}
If VIRTIO_CRYPTO_F_REVISION_1 is not negotiated the \field{flag} is ignored.

If VIRTIO_CRYPTO_F_REVISION_1 is negotiated but VIRTIO_CRYPTO_F_<SERVICE>_STATELESS_MODE
is not negotiated, then the device SHOULD reject <SERVICE> requests if
VIRTIO_CRYPTO_FLAG_SESSION_MODE is not set (in \field{flag}).
\end{note}

The dataq request is composed of four parts:
\begin{lstlisting}
struct virtio_crypto_op_data_req {
    /* Device read only portion */

    struct virtio_crypto_op_header header;

#define VIRTIO_CRYPTO_DATAQ_OP_SPEC_HDR_LEGACY 48
    /* fixed length fields, opcode specific */
    u8 op_flf[flf_len];

    /* Device read && write portion */
    /* variable length fields, opcode specific */
    u8 op_vlf[vlf_len];

    /* Device write only portion */
    struct virtio_crypto_inhdr inhdr;
};
\end{lstlisting}

\field{header} is a general header (see above).

\field{op_flf} is the opcode (in \field{header}) specific header.

\field{flf_len} depends on the VIRTIO_CRYPTO_F_REVISION_1 feature bit
(see below).

\field{op_vlf} is the opcode (in \field{header}) specific parameters.

\field{vlf_len} is the size of the specific structure used.

\begin{itemize*}
\item If the the opcode (in \field{header}) is VIRTIO_CRYPTO_CIPHER_ENCRYPT
    or VIRTIO_CRYPTO_CIPHER_DECRYPT then:
    \begin{itemize*}
    \item If VIRTIO_CRYPTO_F_CIPHER_STATELESS_MODE is negotiated, \field{op_flf} is
        struct virtio_crypto_sym_data_flf_stateless, and \field{op_vlf} is struct
        virtio_crypto_sym_data_vlf_stateless.
    \item If VIRTIO_CRYPTO_F_CIPHER_STATELESS_MODE is NOT negotiated, \field{op_flf}
        is struct virtio_crypto_sym_data_flf if VIRTIO_CRYPTO_F_REVISION_1 is negotiated
        and struct virtio_crypto_sym_data_flf is padded to 48 bytes if NOT negotiated,
        and \field{op_vlf} is struct virtio_crypto_sym_data_vlf.
    \end{itemize*}
\item If the the opcode (in \field{header}) is VIRTIO_CRYPTO_HASH:
    \begin{itemize*}
    \item If VIRTIO_CRYPTO_F_HASH_STATELESS_MODE is negotiated, \field{op_flf} is
        struct virtio_crypto_hash_data_flf_stateless, and \field{op_vlf} is struct
        virtio_crypto_hash_data_vlf_stateless.
    \item If VIRTIO_CRYPTO_F_HASH_STATELESS_MODE is NOT negotiated, \field{op_flf}
        is struct virtio_crypto_hash_data_flf if VIRTIO_CRYPTO_F_REVISION_1 is negotiated
        and struct virtio_crypto_hash_data_flf is padded to 48 bytes if NOT negotiated,
        and \field{op_vlf} is struct virtio_crypto_hash_data_vlf.
    \end{itemize*}
\item If the the opcode (in \field{header}) is VIRTIO_CRYPTO_MAC:
    \begin{itemize*}
    \item If VIRTIO_CRYPTO_F_MAC_STATELESS_MODE is negotiated, \field{op_flf} is
        struct virtio_crypto_mac_data_flf_stateless, and \field{op_vlf} is struct
        virtio_crypto_mac_data_vlf_stateless.
    \item If VIRTIO_CRYPTO_F_MAC_STATELESS_MODE is NOT negotiated, \field{op_flf}
        is struct virtio_crypto_mac_data_flf if VIRTIO_CRYPTO_F_REVISION_1 is negotiated
        and struct virtio_crypto_mac_data_flf is padded to 48 bytes if NOT negotiated,
        and \field{op_vlf} is struct virtio_crypto_mac_data_vlf.
    \end{itemize*}
\item If the the opcode (in \field{header}) is VIRTIO_CRYPTO_AEAD_ENCRYPT
    or VIRTIO_CRYPTO_AEAD_DECRYPT then:
    \begin{itemize*}
    \item If VIRTIO_CRYPTO_F_AEAD_STATELESS_MODE is negotiated, \field{op_flf} is
        struct virtio_crypto_aead_data_flf_stateless, and \field{op_vlf} is struct
        virtio_crypto_aead_data_vlf_stateless.
    \item If VIRTIO_CRYPTO_F_AEAD_STATELESS_MODE is NOT negotiated, \field{op_flf}
        is struct virtio_crypto_aead_data_flf if VIRTIO_CRYPTO_F_REVISION_1 is negotiated
        and struct virtio_crypto_aead_data_flf is padded to 48 bytes if NOT negotiated,
        and \field{op_vlf} is struct virtio_crypto_aead_data_vlf.
    \end{itemize*}
\item If the opcode (in \field{header}) is VIRTIO_CRYPTO_AKCIPHER_ENCRYPT, VIRTIO_CRYPTO_AKCIPHER_DECRYPT,
    VIRTIO_CRYPTO_AKCIPHER_SIGN or VIRTIO_CRYPTO_AKCIPHER_VERIFY then:
    \begin{itemize*}
    \item If VIRTIO_CRYPTO_F_AKCIPHER_STATELESS_MODE is negotiated, \field{op_flf} is
        struct virtio_crypto_akcipher_data_flf_statless, and \field{op_vlf} is struct
        virtio_crypto_akcipher_data_vlf_stateless.
    \item If VIRTIO_CRYPTO_F_AKCIPHER_STATELESS_MODE is NOT negotiated, \field{op_flf}
        is struct virtio_crypto_akcipher_data_flf if VIRTIO_CRYPTO_F_REVISION_1 is negotiated
        and struct virtio_crypto_akcipher_data_flf is padded to 48 bytes if NOT negotiated,
        and \field{op_vlf} is struct virtio_crypto_akcipher_data_vlf.
    \end{itemize*}
\end{itemize*}

\field{inhdr} is a unified input header that used to return the status of
the operations, is defined as follows:

\begin{lstlisting}
struct virtio_crypto_inhdr {
    u8 status;
};
\end{lstlisting}

\subsubsection{HASH Service Operation}\label{sec:Device Types / Crypto Device / Device Operation / HASH Service Operation}

Session mode HASH service requests are as follows:

\begin{lstlisting}
struct virtio_crypto_hash_data_flf {
    /* length of source data */
    le32 src_data_len;
    /* hash result length */
    le32 hash_result_len;
};

struct virtio_crypto_hash_data_vlf {
    /* Device read only portion */
    /* Source data */
    u8 src_data[src_data_len];

    /* Device write only portion */
    /* Hash result data */
    u8 hash_result[hash_result_len];
};
\end{lstlisting}

Each data request uses the virtio_crypto_hash_data_flf structure and the
virtio_crypto_hash_data_vlf structure to store information used to run the
HASH operations.

\field{src_data} is the source data that will be processed.
\field{src_data_len} is the length of source data.
\field{hash_result} is the result data and \field{hash_result_len} is the length
of it.

Stateless mode HASH service requests are as follows:

\begin{lstlisting}
struct virtio_crypto_hash_data_flf_stateless {
    struct {
        /* See VIRTIO_CRYPTO_HASH_* above */
        le32 algo;
    } sess_para;

    /* length of source data */
    le32 src_data_len;
    /* hash result length */
    le32 hash_result_len;
    le32 reserved;
};
struct virtio_crypto_hash_data_vlf_stateless {
    /* Device read only portion */
    /* Source data */
    u8 src_data[src_data_len];

    /* Device write only portion */
    /* Hash result data */
    u8 hash_result[hash_result_len];
};
\end{lstlisting}

\drivernormative{\paragraph}{HASH Service Operation}{Device Types / Crypto Device / Device Operation / HASH Service Operation}

\begin{itemize*}
\item If the driver uses the session mode, then the driver MUST set \field{session_id}
    in struct virtio_crypto_op_header to a valid value assigned by the device when the
    session was created.
\item If the VIRTIO_CRYPTO_F_HASH_STATELESS_MODE feature bit is negotiated, 1) if the
    driver uses the stateless mode, then the driver MUST set the \field{flag} field in
    struct virtio_crypto_op_header to ZERO and MUST set the fields in struct
    virtio_crypto_hash_data_flf_stateless.sess_para, 2) if the driver uses the session
    mode, then the driver MUST set the \field{flag} field in struct virtio_crypto_op_header
    to VIRTIO_CRYPTO_FLAG_SESSION_MODE.
\item The driver MUST set \field{opcode} in struct virtio_crypto_op_header to VIRTIO_CRYPTO_HASH.
\end{itemize*}

\devicenormative{\paragraph}{HASH Service Operation}{Device Types / Crypto Device / Device Operation / HASH Service Operation}

\begin{itemize*}
\item The device MUST use the corresponding structure according to the \field{opcode}
    in the data general header.
\item If the VIRTIO_CRYPTO_F_HASH_STATELESS_MODE feature bit is negotiated, the device
    MUST parse \field{flag} field in struct virtio_crypto_op_header in order to decide
    which mode the driver uses.
\item The device MUST copy the results of HASH operations in the hash_result[] if HASH
    operations success.
\item The device MUST set \field{status} in struct virtio_crypto_inhdr to one of the
    following values of enum VIRTIO_CRYPTO_STATUS:
\begin{itemize*}
\item VIRTIO_CRYPTO_OK if the operation success.
\item VIRTIO_CRYPTO_NOTSUPP if the requested algorithm or operation is unsupported.
\item VIRTIO_CRYPTO_INVSESS if the session ID invalid when in session mode.
\item VIRTIO_CRYPTO_ERR if any failure not mentioned above occurs.
\end{itemize*}
\end{itemize*}


\subsubsection{MAC Service Operation}\label{sec:Device Types / Crypto Device / Device Operation / MAC Service Operation}

Session mode MAC service requests are as follows:

\begin{lstlisting}
struct virtio_crypto_mac_data_flf {
    struct virtio_crypto_hash_data_flf hdr;
};

struct virtio_crypto_mac_data_vlf {
    /* Device read only portion */
    /* Source data */
    u8 src_data[src_data_len];

    /* Device write only portion */
    /* Hash result data */
    u8 hash_result[hash_result_len];
};
\end{lstlisting}

Each request uses the virtio_crypto_mac_data_flf structure and the
virtio_crypto_mac_data_vlf structure to store information used to run the
MAC operations.

\field{src_data} is the source data that will be processed.
\field{src_data_len} is the length of source data.
\field{hash_result} is the result data and \field{hash_result_len} is the length
of it.

Stateless mode MAC service requests are as follows:

\begin{lstlisting}
struct virtio_crypto_mac_data_flf_stateless {
    struct {
        /* See VIRTIO_CRYPTO_MAC_* above */
        le32 algo;
        /* length of authenticated key */
        le32 auth_key_len;
    } sess_para;

    /* length of source data */
    le32 src_data_len;
    /* hash result length */
    le32 hash_result_len;
};

struct virtio_crypto_mac_data_vlf_stateless {
    /* Device read only portion */
    /* The authenticated key */
    u8 auth_key[auth_key_len];
    /* Source data */
    u8 src_data[src_data_len];

    /* Device write only portion */
    /* Hash result data */
    u8 hash_result[hash_result_len];
};
\end{lstlisting}

\field{auth_key} is the authenticated key that will be used during the process.
\field{auth_key_len} is the length of the key.

\drivernormative{\paragraph}{MAC Service Operation}{Device Types / Crypto Device / Device Operation / MAC Service Operation}

\begin{itemize*}
\item If the driver uses the session mode, then the driver MUST set \field{session_id}
    in struct virtio_crypto_op_header to a valid value assigned by the device when the
    session was created.
\item If the VIRTIO_CRYPTO_F_MAC_STATELESS_MODE feature bit is negotiated, 1) if the
    driver uses the stateless mode, then the driver MUST set the \field{flag} field
    in struct virtio_crypto_op_header to ZERO and MUST set the fields in struct
    virtio_crypto_mac_data_flf_stateless.sess_para, 2) if the driver uses the session
    mode, then the driver MUST set the \field{flag} field in struct virtio_crypto_op_header
    to VIRTIO_CRYPTO_FLAG_SESSION_MODE.
\item The driver MUST set \field{opcode} in struct virtio_crypto_op_header to VIRTIO_CRYPTO_MAC.
\end{itemize*}

\devicenormative{\paragraph}{MAC Service Operation}{Device Types / Crypto Device / Device Operation / MAC Service Operation}

\begin{itemize*}
\item If the VIRTIO_CRYPTO_F_MAC_STATELESS_MODE feature bit is negotiated, the device
    MUST parse \field{flag} field in struct virtio_crypto_op_header in order to decide
	which mode the driver uses.
\item The device MUST copy the results of MAC operations in the hash_result[] if HASH
    operations success.
\item The device MUST set \field{status} in struct virtio_crypto_inhdr to one of the
    following values of enum VIRTIO_CRYPTO_STATUS:
\begin{itemize*}
\item VIRTIO_CRYPTO_OK if the operation success.
\item VIRTIO_CRYPTO_NOTSUPP if the requested algorithm or operation is unsupported.
\item VIRTIO_CRYPTO_INVSESS if the session ID invalid when in session mode.
\item VIRTIO_CRYPTO_ERR if any failure not mentioned above occurs.
\end{itemize*}
\end{itemize*}

\subsubsection{Symmetric algorithms Operation}\label{sec:Device Types / Crypto Device / Device Operation / Symmetric algorithms Operation}

Session mode CIPHER service requests are as follows:

\begin{lstlisting}
struct virtio_crypto_cipher_data_flf {
    /*
     * Byte Length of valid IV/Counter data pointed to by the below iv data.
     *
     * For block ciphers in CBC or F8 mode, or for Kasumi in F8 mode, or for
     *   SNOW3G in UEA2 mode, this is the length of the IV (which
     *   must be the same as the block length of the cipher).
     * For block ciphers in CTR mode, this is the length of the counter
     *   (which must be the same as the block length of the cipher).
     */
    le32 iv_len;
    /* length of source data */
    le32 src_data_len;
    /* length of destination data */
    le32 dst_data_len;
    le32 padding;
};

struct virtio_crypto_cipher_data_vlf {
    /* Device read only portion */

    /*
     * Initialization Vector or Counter data.
     *
     * For block ciphers in CBC or F8 mode, or for Kasumi in F8 mode, or for
     *   SNOW3G in UEA2 mode, this is the Initialization Vector (IV)
     *   value.
     * For block ciphers in CTR mode, this is the counter.
     * For AES-XTS, this is the 128bit tweak, i, from IEEE Std 1619-2007.
     *
     * The IV/Counter will be updated after every partial cryptographic
     * operation.
     */
    u8 iv[iv_len];
    /* Source data */
    u8 src_data[src_data_len];

    /* Device write only portion */
    /* Destination data */
    u8 dst_data[dst_data_len];
};
\end{lstlisting}

Session mode requests of algorithm chaining are as follows:

\begin{lstlisting}
struct virtio_crypto_alg_chain_data_flf {
    le32 iv_len;
    /* Length of source data */
    le32 src_data_len;
    /* Length of destination data */
    le32 dst_data_len;
    /* Starting point for cipher processing in source data */
    le32 cipher_start_src_offset;
    /* Length of the source data that the cipher will be computed on */
    le32 len_to_cipher;
    /* Starting point for hash processing in source data */
    le32 hash_start_src_offset;
    /* Length of the source data that the hash will be computed on */
    le32 len_to_hash;
    /* Length of the additional auth data */
    le32 aad_len;
    /* Length of the hash result */
    le32 hash_result_len;
    le32 reserved;
};

struct virtio_crypto_alg_chain_data_vlf {
    /* Device read only portion */

    /* Initialization Vector or Counter data */
    u8 iv[iv_len];
    /* Source data */
    u8 src_data[src_data_len];
    /* Additional authenticated data if exists */
    u8 aad[aad_len];

    /* Device write only portion */

    /* Destination data */
    u8 dst_data[dst_data_len];
    /* Hash result data */
    u8 hash_result[hash_result_len];
};
\end{lstlisting}

Session mode requests of symmetric algorithm are as follows:

\begin{lstlisting}
struct virtio_crypto_sym_data_flf {
    /* Device read only portion */

#define VIRTIO_CRYPTO_SYM_DATA_REQ_HDR_SIZE    40
    u8 op_type_flf[VIRTIO_CRYPTO_SYM_DATA_REQ_HDR_SIZE];

    /* See above VIRTIO_CRYPTO_SYM_OP_* */
    le32 op_type;
    le32 padding;
};

struct virtio_crypto_sym_data_vlf {
    u8 op_type_vlf[sym_para_len];
};
\end{lstlisting}

Each request uses the virtio_crypto_sym_data_flf structure and the
virtio_crypto_sym_data_flf structure to store information used to run the
CIPHER operations.

\field{op_type_flf} is the \field{op_type} specific header, it MUST starts
with or be one of the following structures:
\begin{itemize*}
\item struct virtio_crypto_cipher_data_flf
\item struct virtio_crypto_alg_chain_data_flf
\end{itemize*}

The length of \field{op_type_flf} is fixed to 40 bytes, the data of unused
part (if has) will be ignored.

\field{op_type_vlf} is the \field{op_type} specific parameters, it MUST starts
with or be one of the following structures:
\begin{itemize*}
\item struct virtio_crypto_cipher_data_vlf
\item struct virtio_crypto_alg_chain_data_vlf
\end{itemize*}

\field{sym_para_len} is the size of the specific structure used.

Stateless mode CIPHER service requests are as follows:

\begin{lstlisting}
struct virtio_crypto_cipher_data_flf_stateless {
    struct {
        /* See VIRTIO_CRYPTO_CIPHER* above */
        le32 algo;
        /* length of key */
        le32 key_len;

        /* See VIRTIO_CRYPTO_OP_* above */
        le32 op;
    } sess_para;

    /*
     * Byte Length of valid IV/Counter data pointed to by the below iv data.
     */
    le32 iv_len;
    /* length of source data */
    le32 src_data_len;
    /* length of destination data */
    le32 dst_data_len;
};

struct virtio_crypto_cipher_data_vlf_stateless {
    /* Device read only portion */

    /* The cipher key */
    u8 cipher_key[key_len];

    /* Initialization Vector or Counter data. */
    u8 iv[iv_len];
    /* Source data */
    u8 src_data[src_data_len];

    /* Device write only portion */
    /* Destination data */
    u8 dst_data[dst_data_len];
};
\end{lstlisting}

Stateless mode requests of algorithm chaining are as follows:

\begin{lstlisting}
struct virtio_crypto_alg_chain_data_flf_stateless {
    struct {
        /* See VIRTIO_CRYPTO_SYM_ALG_CHAIN_ORDER_* above */
        le32 alg_chain_order;
        /* length of the additional authenticated data in bytes */
        le32 aad_len;

        struct {
            /* See VIRTIO_CRYPTO_CIPHER* above */
            le32 algo;
            /* length of key */
            le32 key_len;
            /* See VIRTIO_CRYPTO_OP_* above */
            le32 op;
        } cipher;

        struct {
            /* See VIRTIO_CRYPTO_HASH_* or VIRTIO_CRYPTO_MAC_* above */
            le32 algo;
            /* length of authenticated key */
            le32 auth_key_len;
            /* See VIRTIO_CRYPTO_SYM_HASH_MODE_* above */
            le32 hash_mode;
        } hash;
    } sess_para;

    le32 iv_len;
    /* Length of source data */
    le32 src_data_len;
    /* Length of destination data */
    le32 dst_data_len;
    /* Starting point for cipher processing in source data */
    le32 cipher_start_src_offset;
    /* Length of the source data that the cipher will be computed on */
    le32 len_to_cipher;
    /* Starting point for hash processing in source data */
    le32 hash_start_src_offset;
    /* Length of the source data that the hash will be computed on */
    le32 len_to_hash;
    /* Length of the additional auth data */
    le32 aad_len;
    /* Length of the hash result */
    le32 hash_result_len;
    le32 reserved;
};

struct virtio_crypto_alg_chain_data_vlf_stateless {
    /* Device read only portion */

    /* The cipher key */
    u8 cipher_key[key_len];
    /* The auth key */
    u8 auth_key[auth_key_len];
    /* Initialization Vector or Counter data */
    u8 iv[iv_len];
    /* Additional authenticated data if exists */
    u8 aad[aad_len];
    /* Source data */
    u8 src_data[src_data_len];

    /* Device write only portion */

    /* Destination data */
    u8 dst_data[dst_data_len];
    /* Hash result data */
    u8 hash_result[hash_result_len];
};
\end{lstlisting}

Stateless mode requests of symmetric algorithm are as follows:

\begin{lstlisting}
struct virtio_crypto_sym_data_flf_stateless {
    /* Device read only portion */
#define VIRTIO_CRYPTO_SYM_DATE_REQ_HDR_STATELESS_SIZE    72
    u8 op_type_flf[VIRTIO_CRYPTO_SYM_DATE_REQ_HDR_STATELESS_SIZE];

    /* Device write only portion */
    /* See above VIRTIO_CRYPTO_SYM_OP_* */
    le32 op_type;
};

struct virtio_crypto_sym_data_vlf_stateless {
    u8 op_type_vlf[sym_para_len];
};
\end{lstlisting}

\field{op_type_flf} is the \field{op_type} specific header, it MUST starts
with or be one of the following structures:
\begin{itemize*}
\item struct virtio_crypto_cipher_data_flf_stateless
\item struct virtio_crypto_alg_chain_data_flf_stateless
\end{itemize*}

The length of \field{op_type_flf} is fixed to 72 bytes, the data of unused
part (if has) will be ignored.

\field{op_type_vlf} is the \field{op_type} specific parameters, it MUST starts
with or be one of the following structures:
\begin{itemize*}
\item struct virtio_crypto_cipher_data_vlf_stateless
\item struct virtio_crypto_alg_chain_data_vlf_stateless
\end{itemize*}

\field{sym_para_len} is the size of the specific structure used.

\drivernormative{\paragraph}{Symmetric algorithms Operation}{Device Types / Crypto Device / Device Operation / Symmetric algorithms Operation}

\begin{itemize*}
\item If the driver uses the session mode, then the driver MUST set \field{session_id}
    in struct virtio_crypto_op_header to a valid value assigned by the device when the
    session was created.
\item If the VIRTIO_CRYPTO_F_CIPHER_STATELESS_MODE feature bit is negotiated, 1) if the
    driver uses the stateless mode, then the driver MUST set the \field{flag} field in
    struct virtio_crypto_op_header to ZERO and MUST set the fields in struct
    virtio_crypto_cipher_data_flf_stateless.sess_para or struct
    virtio_crypto_alg_chain_data_flf_stateless.sess_para, 2) if the driver uses the
    session mode, then the driver MUST set the \field{flag} field in struct
    virtio_crypto_op_header to VIRTIO_CRYPTO_FLAG_SESSION_MODE.
\item The driver MUST set the \field{opcode} field in struct virtio_crypto_op_header
    to VIRTIO_CRYPTO_CIPHER_ENCRYPT or VIRTIO_CRYPTO_CIPHER_DECRYPT.
\item The driver MUST specify the fields of struct virtio_crypto_cipher_data_flf in
    struct virtio_crypto_sym_data_flf and struct virtio_crypto_cipher_data_vlf in
    struct virtio_crypto_sym_data_vlf if the request is based on VIRTIO_CRYPTO_SYM_OP_CIPHER.
\item The driver MUST specify the fields of struct virtio_crypto_alg_chain_data_flf
    in struct virtio_crypto_sym_data_flf and struct virtio_crypto_alg_chain_data_vlf
    in struct virtio_crypto_sym_data_vlf if the request is of the VIRTIO_CRYPTO_SYM_OP_ALGORITHM_CHAINING
    type.
\end{itemize*}

\devicenormative{\paragraph}{Symmetric algorithms Operation}{Device Types / Crypto Device / Device Operation / Symmetric algorithms Operation}

\begin{itemize*}
\item If the VIRTIO_CRYPTO_F_CIPHER_STATELESS_MODE feature bit is negotiated, the device
    MUST parse \field{flag} field in struct virtio_crypto_op_header in order to decide
	which mode the driver uses.
\item The device MUST parse the virtio_crypto_sym_data_req based on the \field{opcode}
    field in general header.
\item The device MUST parse the fields of struct virtio_crypto_cipher_data_flf in
    struct virtio_crypto_sym_data_flf and struct virtio_crypto_cipher_data_vlf in
    struct virtio_crypto_sym_data_vlf if the request is based on VIRTIO_CRYPTO_SYM_OP_CIPHER.
\item The device MUST parse the fields of struct virtio_crypto_alg_chain_data_flf
    in struct virtio_crypto_sym_data_flf and struct virtio_crypto_alg_chain_data_vlf
    in struct virtio_crypto_sym_data_vlf if the request is of the VIRTIO_CRYPTO_SYM_OP_ALGORITHM_CHAINING
    type.
\item The device MUST copy the result of cryptographic operation in the dst_data[] in
    both plain CIPHER mode and algorithms chain mode.
\item The device MUST check the \field{para}.\field{add_len} is bigger than 0 before
    parse the additional authenticated data in plain algorithms chain mode.
\item The device MUST copy the result of HASH/MAC operation in the hash_result[] is
    of the VIRTIO_CRYPTO_SYM_OP_ALGORITHM_CHAINING type.
\item The device MUST set the \field{status} field in struct virtio_crypto_inhdr to
    one of the following values of enum VIRTIO_CRYPTO_STATUS:
\begin{itemize*}
\item VIRTIO_CRYPTO_OK if the operation success.
\item VIRTIO_CRYPTO_NOTSUPP if the requested algorithm or operation is unsupported.
\item VIRTIO_CRYPTO_INVSESS if the session ID is invalid in session mode.
\item VIRTIO_CRYPTO_ERR if failure not mentioned above occurs.
\end{itemize*}
\end{itemize*}

\subsubsection{AEAD Service Operation}\label{sec:Device Types / Crypto Device / Device Operation / AEAD Service Operation}

Session mode requests of symmetric algorithm are as follows:

\begin{lstlisting}
struct virtio_crypto_aead_data_flf {
    /*
     * Byte Length of valid IV data.
     *
     * For GCM mode, this is either 12 (for 96-bit IVs) or 16, in which
     *   case iv points to J0.
     * For CCM mode, this is the length of the nonce, which can be in the
     *   range 7 to 13 inclusive.
     */
    le32 iv_len;
    /* length of additional auth data */
    le32 aad_len;
    /* length of source data */
    le32 src_data_len;
    /* length of dst data, this should be at least src_data_len + tag_len */
    le32 dst_data_len;
    /* Authentication tag length */
    le32 tag_len;
    le32 reserved;
};

struct virtio_crypto_aead_data_vlf {
    /* Device read only portion */

    /*
     * Initialization Vector data.
     *
     * For GCM mode, this is either the IV (if the length is 96 bits) or J0
     *   (for other sizes), where J0 is as defined by NIST SP800-38D.
     *   Regardless of the IV length, a full 16 bytes needs to be allocated.
     * For CCM mode, the first byte is reserved, and the nonce should be
     *   written starting at &iv[1] (to allow space for the implementation
     *   to write in the flags in the first byte).  Note that a full 16 bytes
     *   should be allocated, even though the iv_len field will have
     *   a value less than this.
     *
     * The IV will be updated after every partial cryptographic operation.
     */
    u8 iv[iv_len];
    /* Source data */
    u8 src_data[src_data_len];
    /* Additional authenticated data if exists */
    u8 aad[aad_len];

    /* Device write only portion */
    /* Pointer to output data */
    u8 dst_data[dst_data_len];
};
\end{lstlisting}

Each request uses the virtio_crypto_aead_data_flf structure and the
virtio_crypto_aead_data_flf structure to store information used to run the
AEAD operations.

Stateless mode AEAD service requests are as follows:

\begin{lstlisting}
struct virtio_crypto_aead_data_flf_stateless {
    struct {
        /* See VIRTIO_CRYPTO_AEAD_* above */
        le32 algo;
        /* length of key */
        le32 key_len;
        /* encrypt or decrypt, See above VIRTIO_CRYPTO_OP_* */
        le32 op;
    } sess_para;

    /* Byte Length of valid IV data. */
    le32 iv_len;
    /* Authentication tag length */
    le32 tag_len;
    /* length of additional auth data */
    le32 aad_len;
    /* length of source data */
    le32 src_data_len;
    /* length of dst data, this should be at least src_data_len + tag_len */
    le32 dst_data_len;
};

struct virtio_crypto_aead_data_vlf_stateless {
    /* Device read only portion */

    /* The cipher key */
    u8 key[key_len];
    /* Initialization Vector data. */
    u8 iv[iv_len];
    /* Source data */
    u8 src_data[src_data_len];
    /* Additional authenticated data if exists */
    u8 aad[aad_len];

    /* Device write only portion */
    /* Pointer to output data */
    u8 dst_data[dst_data_len];
};
\end{lstlisting}

\drivernormative{\paragraph}{AEAD Service Operation}{Device Types / Crypto Device / Device Operation / AEAD Service Operation}

\begin{itemize*}
\item If the driver uses the session mode, then the driver MUST set
    \field{session_id} in struct virtio_crypto_op_header to a valid value assigned
    by the device when the session was created.
\item If the VIRTIO_CRYPTO_F_AEAD_STATELESS_MODE feature bit is negotiated, 1) if
    the driver uses the stateless mode, then the driver MUST set the \field{flag}
    field in struct virtio_crypto_op_header to ZERO and MUST set the fields in
    struct virtio_crypto_aead_data_flf_stateless.sess_para, 2) if the driver uses
    the session mode, then the driver MUST set the \field{flag} field in struct
    virtio_crypto_op_header to VIRTIO_CRYPTO_FLAG_SESSION_MODE.
\item The driver MUST set the \field{opcode} field in struct virtio_crypto_op_header
    to VIRTIO_CRYPTO_AEAD_ENCRYPT or VIRTIO_CRYPTO_AEAD_DECRYPT.
\end{itemize*}

\devicenormative{\paragraph}{AEAD Service Operation}{Device Types / Crypto Device / Device Operation / AEAD Service Operation}

\begin{itemize*}
\item If the VIRTIO_CRYPTO_F_AEAD_STATELESS_MODE feature bit is negotiated, the
    device MUST parse the virtio_crypto_aead_data_vlf_stateless based on the \field{opcode}
	field in general header.
\item The device MUST copy the result of cryptographic operation in the dst_data[].
\item The device MUST copy the authentication tag in the dst_data[] offset the cipher result.
\item The device MUST set the \field{status} field in struct virtio_crypto_inhdr to
    one of the following values of enum VIRTIO_CRYPTO_STATUS:
\item When the \field{opcode} field is VIRTIO_CRYPTO_AEAD_DECRYPT, the device MUST
    verify and return the verification result to the driver.
\begin{itemize*}
\item VIRTIO_CRYPTO_OK if the operation success.
\item VIRTIO_CRYPTO_NOTSUPP if the requested algorithm or operation is unsupported.
\item VIRTIO_CRYPTO_BADMSG if the verification result is incorrect.
\item VIRTIO_CRYPTO_INVSESS if the session ID invalid when in session mode.
\item VIRTIO_CRYPTO_ERR if any failure not mentioned above occurs.
\end{itemize*}
\end{itemize*}

\subsubsection{AKCIPHER Service Operation}\label{sec:Device Types / Crypto Device / Device Operation / AKCIPHER Service Operation}

Session mode AKCIPHER requests are as follows:

\begin{lstlisting}
struct virtio_crypto_akcipher_data_flf {
    /* length of source data */
    le32 src_data_len;
    /* length of dst data */
    le32 dst_data_len;
};

struct virtio_crypto_akcipher_data_vlf {
    /* Device read only portion */
    /* Source data */
    u8 src_data[src_data_len];

    /* Device write only portion */
    /* Pointer to output data */
    u8 dst_data[dst_data_len];
};
\end{lstlisting}

Each data request uses the virtio_crypto_akcipher_flf structure and the virtio_crypto_akcipher_data_vlf
structure to store information used to run the AKCIPHER operations.

For encryption, decryption, and signing:
\field{src_data} is the source data that will be processed, note that for signing operations,
src_data stores the data to be signed, which usually is the digest of some data rather than the
data itself.
\field{src_data_len} is the length of source data.
\field{dst_result} is the result data and \field{dst_data_len} is the length of it. Note that the
length of the result is not always exactly equal to dst_data_len, the driver needs to check how
many bytes the device has written and calculate the actual length of the result.

For verification:
\field{src_data_len} refers to the length of the signature, and \field{dst_data_len} refers to
the length of signed data, where the signed data is usually the digest of some data.
\field{src_data} is spliced by the signature and the signed data, the src_data with the lower
address stores the signature, and the higher address stores the signed data.
\field{dst_data} is always empty for verification.

Different algorithms have different signature formats.
For the RSA algorithm, the result is determined by the padding algorithm specified by
\field{padding_algo} in structure virtio_crypto_rsa_session_para.

For the ECDSA algorithm, the signature is composed of the following
ASN.1 structure (see \hyperref[intro:rfc3279]{RFC3279})
and MUST be DER encoded (see \hyperref[intro:rfc6025]{rfc6025}).

\begin{lstlisting}
Ecdsa-Sig-Value ::= SEQUENCE {
    r INTEGER,
    s INTEGER
}
\end{lstlisting}

Stateless mode AKCIPHER service requests are as follows:
\begin{lstlisting}
struct virtio_crypto_akcipher_data_flf_stateless {
    struct {
        /* See VIRTIO_CYRPTO_AKCIPHER* above */
        le32 algo;
        /* See VIRTIO_CRYPTO_AKCIPHER_KEY_TYPE_* above */
        le32 key_type;
        /* length of key */
        le32 key_len;

        /* algothrim specific parameters described above */
        union {
            struct virtio_crypto_rsa_session_para rsa;
            struct virtio_crypto_ecdsa_session_para ecdsa;
        } u;
    } sess_para;

    /* length of source data */
    le32 src_data_len;
    /* length of destination data */
    le32 dst_data_len;
};

struct virtio_crypto_akcipher_data_vlf_stateless {
    /* Device read only portion */
    u8 akcipher_key[key_len];

    /* Source data */
    u8 src_data[src_data_len];

    /* Device write only portion */
    u8 dst_data[dst_data_len];
};
\end{lstlisting}

In stateless mode, the format of key and signature, the meaning of src_data and dst_data, are all the same
with session mode.

\drivernormative{\paragraph}{AKCIPHER Service Operation}{Device Types / Crypto Device / Device Operation / AKCIPHER Service Operation}

\begin{itemize*}
\item If the driver uses the session mode, then the driver MUST set
    \field{session_id} in struct virtio_crypto_op_header to a valid
    value assigned by the device when the session was created.
\item If the VIRTIO_CRYPTO_F_AKCIPHER_STATELESS_MODE feature bit is negotiated, 1) if the
    driver uses the stateless mode, then the driver MUST set the \field{flag} field in
    struct virtio_crypto_op_header to ZERO and MUST set the fields in struct
    virtio_crypto_akcipher_flf_stateless.sess_para, 2) if the driver uses the session
    mode, then the driver MUST set the \field{flag} field in struct virtio_crypto_op_header
    to VIRTIO_CRYPTO_FLAG_SESSION_MODE.
\item The driver MUST set the \field{opcode} field in struct virtio_crypto_op_header
    to one of VIRTIO_CRYPTO_AKCIPHER_ENCRYPT, VIRTIO_CRYPTO_AKCIPHER_DESTROY_SESSION,
    VIRTIO_CRYPTO_AKCIPHER_SIGN, and VIRTIO_CRYPTO_AKCIPHER_VERIFY.
\end{itemize*}

\devicenormative{\paragraph}{AKCIPHER Service Operation}{Device Types / Crypto Device / Device Operation / AKCIPHER Service Operation}

\begin{itemize*}
\item If the VIRTIO_CRYPTO_F_AKCIPHER_STATELESS_MODE feature bit is negotiated, the
    device MUST parse the virtio_crypto_akcipher_data_vlf_stateless based on the \field{opcode}
    field in general header.
\item The device MUST copy the result of cryptographic operation in the dst_data[].
\item The device MUST set the \field{status} field in struct virtio_crypto_inhdr to
    one of the following values of enum VIRTIO_CRYPTO_STATUS:
\begin{itemize*}
\item VIRTIO_CRYPTO_OK if the operation success.
\item VIRTIO_CRYPTO_NOTSUPP if the requested algorithm or operation is unsupported.
\item VIRTIO_CRYPTO_BADMSG if the verification result is incorrect.
\item VIRTIO_CRYPTO_INVSESS if the session ID invalid when in session mode.
\item VIRTIO_CRYPTO_KEY_REJECTED if the signature verification failed.
\item VIRTIO_CRYPTO_ERR if any failure not mentioned above occurs.
\end{itemize*}
\end{itemize*}

\section{Crypto Device}\label{sec:Device Types / Crypto Device}

The virtio crypto device is a virtual cryptography device as well as a
virtual cryptographic accelerator. The virtio crypto device provides the
following crypto services: CIPHER, MAC, HASH, AEAD and AKCIPHER. Virtio crypto
devices have a single control queue and at least one data queue. Crypto
operation requests are placed into a data queue, and serviced by the
device. Some crypto operation requests are only valid in the context of a
session. The role of the control queue is facilitating control operation
requests. Sessions management is realized with control operation
requests.

\subsection{Device ID}\label{sec:Device Types / Crypto Device / Device ID}

20

\subsection{Virtqueues}\label{sec:Device Types / Crypto Device / Virtqueues}

\begin{description}
\item[0] dataq1
\item[\ldots]
\item[N-1] dataqN
\item[N] controlq
\end{description}

N is set by \field{max_dataqueues}.

\subsection{Feature bits}\label{sec:Device Types / Crypto Device / Feature bits}

\begin{description}
\item VIRTIO_CRYPTO_F_REVISION_1 (0) revision 1. Revision 1 has a specific
    request format and other enhancements (which result in some additional
    requirements).
\item VIRTIO_CRYPTO_F_CIPHER_STATELESS_MODE (1) stateless mode requests are
    supported by the CIPHER service.
\item VIRTIO_CRYPTO_F_HASH_STATELESS_MODE (2) stateless mode requests are
    supported by the HASH service.
\item VIRTIO_CRYPTO_F_MAC_STATELESS_MODE (3) stateless mode requests are
    supported by the MAC service.
\item VIRTIO_CRYPTO_F_AEAD_STATELESS_MODE (4) stateless mode requests are
    supported by the AEAD service.
\item VIRTIO_CRYPTO_F_AKCIPHER_STATELESS_MODE (5) stateless mode requests are
    supported by the AKCIPHER service.
\end{description}


\subsubsection{Feature bit requirements}\label{sec:Device Types / Crypto Device / Feature bit requirements}

Some crypto feature bits require other crypto feature bits
(see \ref{drivernormative:Basic Facilities of a Virtio Device / Feature Bits}):

\begin{description}
\item[VIRTIO_CRYPTO_F_CIPHER_STATELESS_MODE] Requires VIRTIO_CRYPTO_F_REVISION_1.
\item[VIRTIO_CRYPTO_F_HASH_STATELESS_MODE] Requires VIRTIO_CRYPTO_F_REVISION_1.
\item[VIRTIO_CRYPTO_F_MAC_STATELESS_MODE] Requires VIRTIO_CRYPTO_F_REVISION_1.
\item[VIRTIO_CRYPTO_F_AEAD_STATELESS_MODE] Requires VIRTIO_CRYPTO_F_REVISION_1.
\item[VIRTIO_CRYPTO_F_AKCIPHER_STATELESS_MODE] Requires VIRTIO_CRYPTO_F_REVISION_1.
\end{description}

\subsection{Supported crypto services}\label{sec:Device Types / Crypto Device / Supported crypto services}

The following crypto services are defined:

\begin{lstlisting}
/* CIPHER (Symmetric Key Cipher) service */
#define VIRTIO_CRYPTO_SERVICE_CIPHER 0
/* HASH service */
#define VIRTIO_CRYPTO_SERVICE_HASH   1
/* MAC (Message Authentication Codes) service */
#define VIRTIO_CRYPTO_SERVICE_MAC    2
/* AEAD (Authenticated Encryption with Associated Data) service */
#define VIRTIO_CRYPTO_SERVICE_AEAD   3
/* AKCIPHER (Asymmetric Key Cipher) service */
#define VIRTIO_CRYPTO_SERVICE_AKCIPHER 4
\end{lstlisting}

The above constants designate bits used to indicate the which of crypto services are
offered by the device as described in, see \ref{sec:Device Types / Crypto Device / Device configuration layout}.

\subsubsection{CIPHER services}\label{sec:Device Types / Crypto Device / Supported crypto services / CIPHER services}

The following CIPHER algorithms are defined:

\begin{lstlisting}
#define VIRTIO_CRYPTO_NO_CIPHER                 0
#define VIRTIO_CRYPTO_CIPHER_ARC4               1
#define VIRTIO_CRYPTO_CIPHER_AES_ECB            2
#define VIRTIO_CRYPTO_CIPHER_AES_CBC            3
#define VIRTIO_CRYPTO_CIPHER_AES_CTR            4
#define VIRTIO_CRYPTO_CIPHER_DES_ECB            5
#define VIRTIO_CRYPTO_CIPHER_DES_CBC            6
#define VIRTIO_CRYPTO_CIPHER_3DES_ECB           7
#define VIRTIO_CRYPTO_CIPHER_3DES_CBC           8
#define VIRTIO_CRYPTO_CIPHER_3DES_CTR           9
#define VIRTIO_CRYPTO_CIPHER_KASUMI_F8          10
#define VIRTIO_CRYPTO_CIPHER_SNOW3G_UEA2        11
#define VIRTIO_CRYPTO_CIPHER_AES_F8             12
#define VIRTIO_CRYPTO_CIPHER_AES_XTS            13
#define VIRTIO_CRYPTO_CIPHER_ZUC_EEA3           14
\end{lstlisting}

The above constants have two usages:
\begin{enumerate}
\item As bit numbers, used to tell the driver which CIPHER algorithms
are supported by the device, see \ref{sec:Device Types / Crypto Device / Device configuration layout}.
\item As values, used to designate the algorithm in (CIPHER type) crypto
operation requests, see \ref{sec:Device Types / Crypto Device / Device Operation / Control Virtqueue / Session operation}.
\end{enumerate}

\subsubsection{HASH services}\label{sec:Device Types / Crypto Device / Supported crypto services / HASH services}

The following HASH algorithms are defined:

\begin{lstlisting}
#define VIRTIO_CRYPTO_NO_HASH            0
#define VIRTIO_CRYPTO_HASH_MD5           1
#define VIRTIO_CRYPTO_HASH_SHA1          2
#define VIRTIO_CRYPTO_HASH_SHA_224       3
#define VIRTIO_CRYPTO_HASH_SHA_256       4
#define VIRTIO_CRYPTO_HASH_SHA_384       5
#define VIRTIO_CRYPTO_HASH_SHA_512       6
#define VIRTIO_CRYPTO_HASH_SHA3_224      7
#define VIRTIO_CRYPTO_HASH_SHA3_256      8
#define VIRTIO_CRYPTO_HASH_SHA3_384      9
#define VIRTIO_CRYPTO_HASH_SHA3_512      10
#define VIRTIO_CRYPTO_HASH_SHA3_SHAKE128      11
#define VIRTIO_CRYPTO_HASH_SHA3_SHAKE256      12
\end{lstlisting}

The above constants have two usages:
\begin{enumerate}
\item As bit numbers, used to tell the driver which HASH algorithms
are supported by the device, see \ref{sec:Device Types / Crypto Device / Device configuration layout}.
\item As values, used to designate the algorithm in (HASH type) crypto
operation requires, see \ref{sec:Device Types / Crypto Device / Device Operation / Control Virtqueue / Session operation}.
\end{enumerate}

\subsubsection{MAC services}\label{sec:Device Types / Crypto Device / Supported crypto services / MAC services}

The following MAC algorithms are defined:

\begin{lstlisting}
#define VIRTIO_CRYPTO_NO_MAC                       0
#define VIRTIO_CRYPTO_MAC_HMAC_MD5                 1
#define VIRTIO_CRYPTO_MAC_HMAC_SHA1                2
#define VIRTIO_CRYPTO_MAC_HMAC_SHA_224             3
#define VIRTIO_CRYPTO_MAC_HMAC_SHA_256             4
#define VIRTIO_CRYPTO_MAC_HMAC_SHA_384             5
#define VIRTIO_CRYPTO_MAC_HMAC_SHA_512             6
#define VIRTIO_CRYPTO_MAC_CMAC_3DES                25
#define VIRTIO_CRYPTO_MAC_CMAC_AES                 26
#define VIRTIO_CRYPTO_MAC_KASUMI_F9                27
#define VIRTIO_CRYPTO_MAC_SNOW3G_UIA2              28
#define VIRTIO_CRYPTO_MAC_GMAC_AES                 41
#define VIRTIO_CRYPTO_MAC_GMAC_TWOFISH             42
#define VIRTIO_CRYPTO_MAC_CBCMAC_AES               49
#define VIRTIO_CRYPTO_MAC_CBCMAC_KASUMI_F9         50
#define VIRTIO_CRYPTO_MAC_XCBC_AES                 53
#define VIRTIO_CRYPTO_MAC_ZUC_EIA3                 54
\end{lstlisting}

The above constants have two usages:
\begin{enumerate}
\item As bit numbers, used to tell the driver which MAC algorithms
are supported by the device, see \ref{sec:Device Types / Crypto Device / Device configuration layout}.
\item As values, used to designate the algorithm in (MAC type) crypto
operation requests, see \ref{sec:Device Types / Crypto Device / Device Operation / Control Virtqueue / Session operation}.
\end{enumerate}

\subsubsection{AEAD services}\label{sec:Device Types / Crypto Device / Supported crypto services / AEAD services}

The following AEAD algorithms are defined:

\begin{lstlisting}
#define VIRTIO_CRYPTO_NO_AEAD     0
#define VIRTIO_CRYPTO_AEAD_GCM    1
#define VIRTIO_CRYPTO_AEAD_CCM    2
#define VIRTIO_CRYPTO_AEAD_CHACHA20_POLY1305  3
\end{lstlisting}

The above constants have two usages:
\begin{enumerate}
\item As bit numbers, used to tell the driver which AEAD algorithms
are supported by the device, see \ref{sec:Device Types / Crypto Device / Device configuration layout}.
\item As values, used to designate the algorithm in (DEAD type) crypto
operation requests, see \ref{sec:Device Types / Crypto Device / Device Operation / Control Virtqueue / Session operation}.
\end{enumerate}

\subsubsection{AKCIPHER services}\label{sec: Device Types / Crypto Device / Supported crypto services / AKCIPHER services}

The following AKCIPHER algorithms are defined:
\begin{lstlisting}
#define VIRTIO_CRYPTO_NO_AKCIPHER 0
#define VIRTIO_CRYPTO_AKCIPHER_RSA   1
#define VIRTIO_CRYPTO_AKCIPHER_ECDSA 2
\end{lstlisting}

The above constants have two usages:
\begin{enumerate}
\item As bit numbers, used to tell the driver which AKCIPHER algorithms
are supported by the device, see \ref{sec:Device Types / Crypto Device / Device configuration layout}.
\item As values, used to designate the algorithm in asymmetric crypto operation requests,
see \ref{sec:Device Types / Crypto Device / Device Operation / Control Virtqueue / Session operation}.
\end{enumerate}


\subsection{Device configuration layout}\label{sec:Device Types / Crypto Device / Device configuration layout}

Crypto device configuration uses the following layout structure:

\begin{lstlisting}
struct virtio_crypto_config {
    le32 status;
    le32 max_dataqueues;
    le32 crypto_services;
    /* Detailed algorithms mask */
    le32 cipher_algo_l;
    le32 cipher_algo_h;
    le32 hash_algo;
    le32 mac_algo_l;
    le32 mac_algo_h;
    le32 aead_algo;
    /* Maximum length of cipher key in bytes */
    le32 max_cipher_key_len;
    /* Maximum length of authenticated key in bytes */
    le32 max_auth_key_len;
    le32 akcipher_algo;
    /* Maximum size of each crypto request's content in bytes */
    le64 max_size;
};
\end{lstlisting}

\begin{description}
\item Currently, only one \field{status} bit is defined: VIRTIO_CRYPTO_S_HW_READY
    set indicates that the device is ready to process requests, this bit is read-only
    for the driver
\begin{lstlisting}
#define VIRTIO_CRYPTO_S_HW_READY  (1 << 0)
\end{lstlisting}

\item [\field{max_dataqueues}] is the maximum number of data virtqueues that can
    be configured by the device. The driver MAY use only one data queue, or it
    can use more to achieve better performance.

\item [\field{crypto_services}] crypto service offered, see \ref{sec:Device Types / Crypto Device / Supported crypto services}.

\item [\field{cipher_algo_l}] CIPHER algorithms bits 0-31, see \ref{sec:Device Types / Crypto Device / Supported crypto services  / CIPHER services}.

\item [\field{cipher_algo_h}] CIPHER algorithms bits 32-63, see \ref{sec:Device Types / Crypto Device / Supported crypto services  / CIPHER services}.

\item [\field{hash_algo}] HASH algorithms bits, see \ref{sec:Device Types / Crypto Device / Supported crypto services  / HASH services}.

\item [\field{mac_algo_l}] MAC algorithms bits 0-31, see \ref{sec:Device Types / Crypto Device / Supported crypto services  / MAC services}.

\item [\field{mac_algo_h}] MAC algorithms bits 32-63, see \ref{sec:Device Types / Crypto Device / Supported crypto services  / MAC services}.

\item [\field{aead_algo}] AEAD algorithms bits, see \ref{sec:Device Types / Crypto Device / Supported crypto services  / AEAD services}.

\item [\field{max_cipher_key_len}] is the maximum length of cipher key supported by the device.

\item [\field{max_auth_key_len}] is the maximum length of authenticated key supported by the device.

\item [\field{akcipher_algo}] AKCIPHER algorithms bit 0-31, see \ref{sec: Device Types / Crypto Device / Supported crypto services / AKCIPHER services}.

\item [\field{max_size}] is the maximum size of the variable-length parameters of
    data operation of each crypto request's content supported by the device.
\end{description}

\begin{note}
Unless explicitly stated otherwise all lengths and sizes are in bytes.
\end{note}

\devicenormative{\subsubsection}{Device configuration layout}{Device Types / Crypto Device / Device configuration layout}

\begin{itemize*}
\item The device MUST set \field{max_dataqueues} to between 1 and 65535 inclusive.
\item The device MUST set the \field{status} with valid flags, undefined flags MUST NOT be set.
\item The device MUST accept and handle requests after \field{status} is set to VIRTIO_CRYPTO_S_HW_READY.
\item The device MUST set \field{crypto_services} based on the crypto services the device offers.
\item The device MUST set detailed algorithms masks for each service advertised by \field{crypto_services}.
    The device MUST NOT set the not defined algorithms bits.
\item The device MUST set \field{max_size} to show the maximum size of crypto request the device supports.
\item The device MUST set \field{max_cipher_key_len} to show the maximum length of cipher key if the
    device supports CIPHER service.
\item The device MUST set \field{max_auth_key_len} to show the maximum length of authenticated key if
    the device supports MAC service.
\end{itemize*}

\drivernormative{\subsubsection}{Device configuration layout}{Device Types / Crypto Device / Device configuration layout}

\begin{itemize*}
\item The driver MUST read the \field{status} from the bottom bit of status to check whether the
    VIRTIO_CRYPTO_S_HW_READY is set, and the driver MUST reread it after device reset.
\item The driver MUST NOT transmit any requests to the device if the VIRTIO_CRYPTO_S_HW_READY is not set.
\item The driver MUST read \field{max_dataqueues} field to discover the number of data queues the device supports.
\item The driver MUST read \field{crypto_services} field to discover which services the device is able to offer.
\item The driver SHOULD ignore the not defined algorithms bits.
\item The driver MUST read the detailed algorithms fields based on \field{crypto_services} field.
\item The driver SHOULD read \field{max_size} to discover the maximum size of the variable-length
    parameters of data operation of the crypto request's content the device supports and MUST
    guarantee that the size of each crypto request's content is within the \field{max_size}, otherwise
    the request will fail and the driver MUST reset the device.
\item The driver SHOULD read \field{max_cipher_key_len} to discover the maximum length of cipher key
    the device supports and MUST guarantee that the \field{key_len} (CIPHER service or AEAD service) is within
    the \field{max_cipher_key_len} of the device configuration, otherwise the request will fail.
\item The driver SHOULD read \field{max_auth_key_len} to discover the maximum length of authenticated
    key the device supports and MUST guarantee that the \field{auth_key_len} (MAC service) is within the
    \field{max_auth_key_len} of the device configuration, otherwise the request will fail.
\end{itemize*}

\subsection{Device Initialization}\label{sec:Device Types / Crypto Device / Device Initialization}

\drivernormative{\subsubsection}{Device Initialization}{Device Types / Crypto Device / Device Initialization}

\begin{itemize*}
\item The driver MUST configure and initialize all virtqueues.
\item The driver MUST read the supported crypto services from bits of \field{crypto_services}.
\item The driver MUST read the supported algorithms based on \field{crypto_services} field.
\end{itemize*}

\subsection{Device Operation}\label{sec:Device Types / Crypto Device / Device Operation}

The operation of a virtio crypto device is driven by requests placed on the virtqueues.
Requests consist of a queue-type specific header (specifying among others the operation)
and an operation specific payload.

If VIRTIO_CRYPTO_F_REVISION_1 is negotiated the device may support both session mode
(See \ref{sec:Device Types / Crypto Device / Device Operation / Control Virtqueue / Session operation})
and stateless mode operation requests.
In stateless mode all operation parameters are supplied as a part of each request,
while in session mode, some or all operation parameters are managed within the
session. Stateless mode is guarded by feature bits 0-4 on a service level. If
stateless mode is negotiated for a service, the service accepts both session
mode and stateless requests; otherwise stateless mode requests are rejected
(via operation status).

\subsubsection{Operation Status}\label{sec:Device Types / Crypto Device / Device Operation / Operation status}
The device MUST return a status code as part of the operation (both session
operation and service operation) result. The valid operation status as follows:

\begin{lstlisting}
enum VIRTIO_CRYPTO_STATUS {
    VIRTIO_CRYPTO_OK = 0,
    VIRTIO_CRYPTO_ERR = 1,
    VIRTIO_CRYPTO_BADMSG = 2,
    VIRTIO_CRYPTO_NOTSUPP = 3,
    VIRTIO_CRYPTO_INVSESS = 4,
    VIRTIO_CRYPTO_NOSPC = 5,
    VIRTIO_CRYPTO_KEY_REJECTED = 6,
    VIRTIO_CRYPTO_MAX
};
\end{lstlisting}

\begin{itemize*}
\item VIRTIO_CRYPTO_OK: success.
\item VIRTIO_CRYPTO_BADMSG: authentication failed (only when AEAD decryption).
\item VIRTIO_CRYPTO_NOTSUPP: operation or algorithm is unsupported.
\item VIRTIO_CRYPTO_INVSESS: invalid session ID when executing crypto operations.
\item VIRTIO_CRYPTO_NOSPC: no free session ID (only when the VIRTIO_CRYPTO_F_REVISION_1
    feature bit is negotiated).
\item VIRTIO_CRYPTO_KEY_REJECTED: signature verification failed (only when AKCIPHER verification).
\item VIRTIO_CRYPTO_ERR: any failure not mentioned above occurs.
\end{itemize*}

\subsubsection{Control Virtqueue}\label{sec:Device Types / Crypto Device / Device Operation / Control Virtqueue}

The driver uses the control virtqueue to send control commands to the
device, such as session operations (See \ref{sec:Device Types / Crypto Device / Device
Operation / Control Virtqueue / Session operation}).

The header for controlq is of the following form:
\begin{lstlisting}
#define VIRTIO_CRYPTO_OPCODE(service, op)   (((service) << 8) | (op))

struct virtio_crypto_ctrl_header {
#define VIRTIO_CRYPTO_CIPHER_CREATE_SESSION \
       VIRTIO_CRYPTO_OPCODE(VIRTIO_CRYPTO_SERVICE_CIPHER, 0x02)
#define VIRTIO_CRYPTO_CIPHER_DESTROY_SESSION \
       VIRTIO_CRYPTO_OPCODE(VIRTIO_CRYPTO_SERVICE_CIPHER, 0x03)
#define VIRTIO_CRYPTO_HASH_CREATE_SESSION \
       VIRTIO_CRYPTO_OPCODE(VIRTIO_CRYPTO_SERVICE_HASH, 0x02)
#define VIRTIO_CRYPTO_HASH_DESTROY_SESSION \
       VIRTIO_CRYPTO_OPCODE(VIRTIO_CRYPTO_SERVICE_HASH, 0x03)
#define VIRTIO_CRYPTO_MAC_CREATE_SESSION \
       VIRTIO_CRYPTO_OPCODE(VIRTIO_CRYPTO_SERVICE_MAC, 0x02)
#define VIRTIO_CRYPTO_MAC_DESTROY_SESSION \
       VIRTIO_CRYPTO_OPCODE(VIRTIO_CRYPTO_SERVICE_MAC, 0x03)
#define VIRTIO_CRYPTO_AEAD_CREATE_SESSION \
       VIRTIO_CRYPTO_OPCODE(VIRTIO_CRYPTO_SERVICE_AEAD, 0x02)
#define VIRTIO_CRYPTO_AEAD_DESTROY_SESSION \
       VIRTIO_CRYPTO_OPCODE(VIRTIO_CRYPTO_SERVICE_AEAD, 0x03)
#define VIRTIO_CRYPTO_AKCIPHER_CREATE_SESSION \
       VIRTIO_CRYPTO_OPCODE(VIRTIO_CRYPTO_SERVICE_AKCIPHER, 0x04)
#define VIRTIO_CRYPTO_AKCIPHER_DESTROY_SESSION \
       VIRTIO_CRYPTO_OPCDE(VIRTIO_CRYPTO_SERVICE_AKCIPHER, 0x05)
    le32 opcode;
    /* algo should be service-specific algorithms */
    le32 algo;
    le32 flag;
    le32 reserved;
};
\end{lstlisting}

The controlq request is composed of four parts:
\begin{lstlisting}
struct virtio_crypto_op_ctrl_req {
    /* Device read only portion */

    struct virtio_crypto_ctrl_header header;

#define VIRTIO_CRYPTO_CTRLQ_OP_SPEC_HDR_LEGACY 56
    /* fixed length fields, opcode specific */
    u8 op_flf[flf_len];

    /* variable length fields, opcode specific */
    u8 op_vlf[vlf_len];

    /* Device write only portion */

    /* op result or completion status */
    u8 op_outcome[outcome_len];
};
\end{lstlisting}

\field{header} is a general header (see above).

\field{op_flf} is the opcode (in \field{header}) specific fixed-length parameters.

\field{flf_len} depends on the VIRTIO_CRYPTO_F_REVISION_1 feature bit (see below).

\field{op_vlf} is the opcode (in \field{header}) specific variable-length parameters.

\field{vlf_len} is the size of the specific structure used.
\begin{note}
The \field{vlf_len} of session-destroy operation and the hash-session-create
operation is ZERO.
\end{note}

\begin{itemize*}
\item If the opcode (in \field{header}) is VIRTIO_CRYPTO_CIPHER_CREATE_SESSION
    then \field{op_flf} is struct virtio_crypto_sym_create_session_flf if
    VIRTIO_CRYPTO_F_REVISION_1 is negotiated and struct virtio_crypto_sym_create_session_flf is
    padded to 56 bytes if NOT negotiated, and \field{op_vlf} is struct
    virtio_crypto_sym_create_session_vlf.
\item If the opcode (in \field{header}) is VIRTIO_CRYPTO_HASH_CREATE_SESSION
    then \field{op_flf} is struct virtio_crypto_hash_create_session_flf if
    VIRTIO_CRYPTO_F_REVISION_1 is negotiated and struct virtio_crypto_hash_create_session_flf is
    padded to 56 bytes if NOT negotiated.
\item If the opcode (in \field{header}) is VIRTIO_CRYPTO_MAC_CREATE_SESSION
    then \field{op_flf} is struct virtio_crypto_mac_create_session_flf if
    VIRTIO_CRYPTO_F_REVISION_1 is negotiated and struct virtio_crypto_mac_create_session_flf is
    padded to 56 bytes if NOT negotiated, and \field{op_vlf} is struct
    virtio_crypto_mac_create_session_vlf.
\item If the opcode (in \field{header}) is VIRTIO_CRYPTO_AEAD_CREATE_SESSION
    then \field{op_flf} is struct virtio_crypto_aead_create_session_flf if
    VIRTIO_CRYPTO_F_REVISION_1 is negotiated and struct virtio_crypto_aead_create_session_flf is
    padded to 56 bytes if NOT negotiated, and \field{op_vlf} is struct
    virtio_crypto_aead_create_session_vlf.
\item If the opcode (in \field{header}) is VIRTIO_CRYPTO_AKCIPHER_CREATE_SESSION
    then \field{op_flf} is struct virtio_crypto_akcipher_create_session_flf if
    VIRTIO_CRYPTO_F_REVISION_1 is negotiated and struct virtio_crypto_akcipher_create_session_flf is
    padded to 56 bytes if NOT negotiated, and \field{op_vlf} is struct
    virtio_crypto_akcipher_create_session_vlf.
\item If the opcode (in \field{header}) is VIRTIO_CRYPTO_CIPHER_DESTROY_SESSION
    or VIRTIO_CRYPTO_HASH_DESTROY_SESSION or VIRTIO_CRYPTO_MAC_DESTROY_SESSION or
    VIRTIO_CRYPTO_AEAD_DESTROY_SESSION then \field{op_flf} is struct
    virtio_crypto_destroy_session_flf if VIRTIO_CRYPTO_F_REVISION_1 is negotiated and
    struct virtio_crypto_destroy_session_flf is padded to 56 bytes if NOT negotiated.
\end{itemize*}

\field{op_outcome} stores the result of operation and must be struct
virtio_crypto_destroy_session_input for destroy session or
struct virtio_crypto_create_session_input for create session.

\field{outcome_len} is the size of the structure used.


\paragraph{Session operation}\label{sec:Device Types / Crypto Device / Device
Operation / Control Virtqueue / Session operation}

The session is a handle which describes the cryptographic parameters to be
applied to a number of buffers.

The following structure stores the result of session creation set by the device:

\begin{lstlisting}
struct virtio_crypto_create_session_input {
    le64 session_id;
    le32 status;
    le32 padding;
};
\end{lstlisting}

A request to destroy a session includes the following information:

\begin{lstlisting}
struct virtio_crypto_destroy_session_flf {
    /* Device read only portion */
    le64  session_id;
};

struct virtio_crypto_destroy_session_input {
    /* Device write only portion */
    u8  status;
};
\end{lstlisting}


\subparagraph{Session operation: HASH session}\label{sec:Device Types / Crypto Device / Device
Operation / Control Virtqueue / Session operation / Session operation: HASH session}

The fixed-length parameters of HASH session requests is as follows:

\begin{lstlisting}
struct virtio_crypto_hash_create_session_flf {
    /* Device read only portion */

    /* See VIRTIO_CRYPTO_HASH_* above */
    le32 algo;
    /* hash result length */
    le32 hash_result_len;
};
\end{lstlisting}


\subparagraph{Session operation: MAC session}\label{sec:Device Types / Crypto Device / Device
Operation / Control Virtqueue / Session operation / Session operation: MAC session}

The fixed-length and the variable-length parameters of MAC session requests are as follows:

\begin{lstlisting}
struct virtio_crypto_mac_create_session_flf {
    /* Device read only portion */

    /* See VIRTIO_CRYPTO_MAC_* above */
    le32 algo;
    /* hash result length */
    le32 hash_result_len;
    /* length of authenticated key */
    le32 auth_key_len;
    le32 padding;
};

struct virtio_crypto_mac_create_session_vlf {
    /* Device read only portion */

    /* The authenticated key */
    u8 auth_key[auth_key_len];
};
\end{lstlisting}

The length of \field{auth_key} is specified in \field{auth_key_len} in the struct
virtio_crypto_mac_create_session_flf.


\subparagraph{Session operation: Symmetric algorithms session}\label{sec:Device Types / Crypto Device / Device
Operation / Control Virtqueue / Session operation / Session operation: Symmetric algorithms session}

The request of symmetric session could be the CIPHER algorithms request
or the chain algorithms (chaining CIPHER and HASH/MAC) request.

The fixed-length and the variable-length parameters of CIPHER session requests are as follows:

\begin{lstlisting}
struct virtio_crypto_cipher_session_flf {
    /* Device read only portion */

    /* See VIRTIO_CRYPTO_CIPHER* above */
    le32 algo;
    /* length of key */
    le32 key_len;
#define VIRTIO_CRYPTO_OP_ENCRYPT  1
#define VIRTIO_CRYPTO_OP_DECRYPT  2
    /* encryption or decryption */
    le32 op;
    le32 padding;
};

struct virtio_crypto_cipher_session_vlf {
    /* Device read only portion */

    /* The cipher key */
    u8 cipher_key[key_len];
};
\end{lstlisting}

The length of \field{cipher_key} is specified in \field{key_len} in the struct
virtio_crypto_cipher_session_flf.

The fixed-length and the variable-length parameters of Chain session requests are as follows:

\begin{lstlisting}
struct virtio_crypto_alg_chain_session_flf {
    /* Device read only portion */

#define VIRTIO_CRYPTO_SYM_ALG_CHAIN_ORDER_HASH_THEN_CIPHER  1
#define VIRTIO_CRYPTO_SYM_ALG_CHAIN_ORDER_CIPHER_THEN_HASH  2
    le32 alg_chain_order;
/* Plain hash */
#define VIRTIO_CRYPTO_SYM_HASH_MODE_PLAIN    1
/* Authenticated hash (mac) */
#define VIRTIO_CRYPTO_SYM_HASH_MODE_AUTH     2
/* Nested hash */
#define VIRTIO_CRYPTO_SYM_HASH_MODE_NESTED   3
    le32 hash_mode;
    struct virtio_crypto_cipher_session_flf cipher_hdr;

#define VIRTIO_CRYPTO_ALG_CHAIN_SESS_OP_SPEC_HDR_SIZE  16
    /* fixed length fields, algo specific */
    u8 algo_flf[VIRTIO_CRYPTO_ALG_CHAIN_SESS_OP_SPEC_HDR_SIZE];

    /* length of the additional authenticated data (AAD) in bytes */
    le32 aad_len;
    le32 padding;
};

struct virtio_crypto_alg_chain_session_vlf {
    /* Device read only portion */

    /* The cipher key */
    u8 cipher_key[key_len];
    /* The authenticated key */
    u8 auth_key[auth_key_len];
};
\end{lstlisting}

\field{hash_mode} decides the type used by \field{algo_flf}.

\field{algo_flf} is fixed to 16 bytes and MUST contains or be one of
the following types:
\begin{itemize*}
\item struct virtio_crypto_hash_create_session_flf
\item struct virtio_crypto_mac_create_session_flf
\end{itemize*}
The data of unused part (if has) in \field{algo_flf} will be ignored.

The length of \field{cipher_key} is specified in \field{key_len} in \field{cipher_hdr}.

The length of \field{auth_key} is specified in \field{auth_key_len} in struct
virtio_crypto_mac_create_session_flf.

The fixed-length parameters of Symmetric session requests are as follows:

\begin{lstlisting}
struct virtio_crypto_sym_create_session_flf {
    /* Device read only portion */

#define VIRTIO_CRYPTO_SYM_SESS_OP_SPEC_HDR_SIZE  48
    /* fixed length fields, opcode specific */
    u8 op_flf[VIRTIO_CRYPTO_SYM_SESS_OP_SPEC_HDR_SIZE];

/* No operation */
#define VIRTIO_CRYPTO_SYM_OP_NONE  0
/* Cipher only operation on the data */
#define VIRTIO_CRYPTO_SYM_OP_CIPHER  1
/* Chain any cipher with any hash or mac operation. The order
   depends on the value of alg_chain_order param */
#define VIRTIO_CRYPTO_SYM_OP_ALGORITHM_CHAINING  2
    le32 op_type;
    le32 padding;
};
\end{lstlisting}

\field{op_flf} is fixed to 48 bytes, MUST contains or be one of
the following types:
\begin{itemize*}
\item struct virtio_crypto_cipher_session_flf
\item struct virtio_crypto_alg_chain_session_flf
\end{itemize*}
The data of unused part (if has) in \field{op_flf} will be ignored.

\field{op_type} decides the type used by \field{op_flf}.

The variable-length parameters of Symmetric session requests are as follows:

\begin{lstlisting}
struct virtio_crypto_sym_create_session_vlf {
    /* Device read only portion */
    /* variable length fields, opcode specific */
    u8 op_vlf[vlf_len];
};
\end{lstlisting}

\field{op_vlf} MUST contains or be one of the following types:
\begin{itemize*}
\item struct virtio_crypto_cipher_session_vlf
\item struct virtio_crypto_alg_chain_session_vlf
\end{itemize*}

\field{op_type} in struct virtio_crypto_sym_create_session_flf decides the
type used by \field{op_vlf}.

\field{vlf_len} is the size of the specific structure used.


\subparagraph{Session operation: AEAD session}\label{sec:Device Types / Crypto Device / Device
Operation / Control Virtqueue / Session operation / Session operation: AEAD session}

The fixed-length and the variable-length parameters of AEAD session requests are as follows:

\begin{lstlisting}
struct virtio_crypto_aead_create_session_flf {
    /* Device read only portion */

    /* See VIRTIO_CRYPTO_AEAD_* above */
    le32 algo;
    /* length of key */
    le32 key_len;
    /* Authentication tag length */
    le32 tag_len;
    /* The length of the additional authenticated data (AAD) in bytes */
    le32 aad_len;
    /* encryption or decryption, See above VIRTIO_CRYPTO_OP_* */
    le32 op;
    le32 padding;
};

struct virtio_crypto_aead_create_session_vlf {
    /* Device read only portion */
    u8 key[key_len];
};
\end{lstlisting}

The length of \field{key} is specified in \field{key_len} in struct
virtio_crypto_aead_create_session_flf.

\subparagraph{Session operation: AKCIPHER session}\label{sec:Device Types / Crypto Device / Device
Operation / Control Virtqueue / Session operation / Session operation: AKCIPHER session}

Due to the complexity of asymmetric key algorithms, different algorithms
require different parameters. The following data structures are used as
supplementary parameters to describe the asymmetric algorithm sessions.

For the RSA algorithm, the extra parameters are as follows:
\begin{lstlisting}
struct virtio_crypto_rsa_session_para {
#define VIRTIO_CRYPTO_RSA_RAW_PADDING   0
#define VIRTIO_CRYPTO_RSA_PKCS1_PADDING 1
    le32 padding_algo;

#define VIRTIO_CRYPTO_RSA_NO_HASH   0
#define VIRTIO_CRYPTO_RSA_MD2       1
#define VIRTIO_CRYPTO_RSA_MD3       2
#define VIRTIO_CRYPTO_RSA_MD4       3
#define VIRTIO_CRYPTO_RSA_MD5       4
#define VIRTIO_CRYPTO_RSA_SHA1      5
#define VIRTIO_CRYPTO_RSA_SHA256    6
#define VIRTIO_CRYPTO_RSA_SHA384    7
#define VIRTIO_CRYPTO_RSA_SHA512    8
#define VIRTIO_CRYPTO_RSA_SHA224    9
    le32 hash_algo;
};
\end{lstlisting}

\field{padding_algo} specifies the padding method used by RSA sessions.
\begin{itemize*}
\item If VIRTIO_CRYPTO_RSA_RAW_PADDING is specified, 1) \field{hash_algo}
is ignored, 2) ciphertext and plaintext MUST be padded with leading zeros,
3) and RSA sessions with VIRTIO_CRYPTO_RSA_RAW_PADDING MUST not be used
for verification and signing operations.
\item If VIRTIO_CRYPTO_RSA_PKCS1_PADDING is specified, EMSA-PKCS1-v1_5 padding method
is used (see \hyperref[intro:rfc3447]{PKCS\#1}), \field{hash_algo} specifies how the
digest of the data passed to RSA sessions is calculated when verifying and signing.
It only affects the padding algorithm and is ignored during encryption and decryption.
\end{itemize*}

The ECC algorithms such as the ECDSA algorithm, cannot use custom curves, only the
following known curves can be used (see \hyperref[intro:NIST]{NIST-recommended curves}).

\begin{lstlisting}
#define VIRTIO_CRYPTO_CURVE_UNKNOWN   0
#define VIRTIO_CRYPTO_CURVE_NIST_P192 1
#define VIRTIO_CRYPTO_CURVE_NIST_P224 2
#define VIRTIO_CRYPTO_CURVE_NIST_P256 3
#define VIRTIO_CRYPTO_CURVE_NIST_P384 4
#define VIRTIO_CRYPTO_CURVE_NIST_P521 5
\end{lstlisting}

For the ECDSA algorithm, the extra parameters are as follows:
\begin{lstlisting}
struct virtio_crypto_ecdsa_session_para {
    /* See VIRTIO_CRYPTO_CURVE_* above */
    le32 curve_id;
};
\end{lstlisting}

The fixed-length and the variable-length parameters of AKCIPHER session requests are as follows:
\begin{lstlisting}
struct virtio_crypto_akcipher_create_session_flf {
    /* Device read only portion */

    /* See VIRTIO_CRYPTO_AKCIPHER_* above */
    le32 algo;
#define VIRTIO_CRYPTO_AKCIPHER_KEY_TYPE_PUBLIC 1
#define VIRTIO_CRYPTO_AKCIPHER_KEY_TYPE_PRIVATE 2
    le32 key_type;
    /* length of key */
    le32 key_len;

#define VIRTIO_CRYPTO_AKCIPHER_SESS_ALGO_SPEC_HDR_SIZE 44
    u8 algo_flf[VIRTIO_CRYPTO_AKCIPHER_SESS_ALGO_SPEC_HDR_SIZE];
};

struct virtio_crypto_akcipher_create_session_vlf {
    /* Device read only portion */
    u8 key[key_len];
};
\end{lstlisting}

\field{algo} decides the type used by \field{algo_flf}.
\field{algo_flf} is fixed to 44 bytes and MUST contains of be one the
following structures:
\begin{itemize*}
\item struct virtio_crypto_rsa_session_para
\item struct virtio_crypto_ecdsa_session_para
\end{itemize*}

The length of \field{key} is specified in \field{key_len} in the struct
virtio_crypto_akcipher_create_session_flf.

For the RSA algorithm, the key needs to be encoded according to
\hyperref[intro:rfc3447]{PKCS\#1}. The private key is described with the
RSAPrivateKey structure, and the public key is described with the RSAPublicKey
structure. These ASN.1 structures are encoded in DER encoding rules (see
\hyperref[intro:rfc6025]{rfc6025}).

\begin{lstlisting}
RSAPrivateKey ::= SEQUENCE {
    version          INTEGER,
    modulus          INTEGER,
    publicExponent   INTEGER,
    privateExponent  INTEGER,
    prime1           INTEGER,
    prime2           INTEGER,
    exponent1        INTEGER,
    exponent1        INTEGER,
    coefficient      INTEGER,
    otherPrimeInfos  OtherPrimeInfos OPTIONAL
}

OtherPrimeInfos ::= SEQUENCE SIZE(1...MAX) OF OtherPrimeInfo

OtherPrimeINfo ::= SEQUENCE {
    prime           INTEGER,
    exponent        INTEGER,
    coefficient     INTEGER
}

RSAPublicKey ::= SEQUENCE {
    modulus         INTEGER,
    publicExponent  INTEGER
}
\end{lstlisting}

For the ECDSA algorithm, the private key is encoded according to
\hyperref[intro:rfc5915]{RFC5915}, the private key of the ECDSA algorithm
is described by the ASN.1 structure ECPrivateKey and encoded with DER
encoding rules (see \hyperref[intro:rfc6025]{rfc6025}).

\begin{lstlisting}
ECPrivateKey ::= SEQUNCE {
    version         INTEGER,
    privateKey      OCTET STRING,
    parameters [0]  ECParameters {{ NamedCurve }} OPTIONAL,
    publicKey  [1]  BIT STRING OPTIONAL
}
\end{lstlisting}

The public key of the ECDSA algorithm is encoded according to \hyperref[intro:SEC1]{SEC1},
and the public key of ECDSA is described by the ASN.1 structure ECPoint.
When initializing a session with ECDSA public key, the ECPoint is DER encoded and the
\field{key} only contains the value part of ECPoint, that is, the header part of the
OCTET STRING will be omitted (see \hyperref[intro:rfc6025]{rfc6025}).

\begin{lstlisting}
ECPoint ::= OCTET STRING
\end{lstlisting}

The length of \field{key} is specified in \field{key_len} in
struct virtio_crypto_akcipher_create_session_flf.

\drivernormative{\subparagraph}{Session operation: create session}{Device Types / Crypto Device / Device
Operation / Control Virtqueue / Session operation / Session operation: create session}

\begin{itemize*}
\item The driver MUST set the \field{opcode} field based on service type: CIPHER, HASH, MAC, AEAD or AKCIPHER.
\item The driver MUST set the control general header, the opcode specific header,
    the opcode specific extra parameters and the opcode specific outcome buffer in turn.
    See \ref{sec:Device Types / Crypto Device / Device Operation / Control Virtqueue}.
\item The driver MUST set the \field{reversed} field to zero.
\end{itemize*}

\devicenormative{\subparagraph}{Session operation: create session}{Device Types / Crypto Device / Device
Operation / Control Virtqueue / Session operation / Session operation: create session}

\begin{itemize*}
\item The device MUST use the corresponding opcode specific structure according to the
    \field{opcode} in the control general header.
\item The device MUST extract extra parameters according to the structures used.
\item The device MUST set the \field{status} field to one of the following values of enum
    VIRTIO_CRYPTO_STATUS after finish a session creation:
\begin{itemize*}
\item VIRTIO_CRYPTO_OK if a session is created successfully.
\item VIRTIO_CRYPTO_NOTSUPP if the requested algorithm or operation is unsupported.
\item VIRTIO_CRYPTO_NOSPC if no free session ID (only when the VIRTIO_CRYPTO_F_REVISION_1
    feature bit is negotiated).
\item VIRTIO_CRYPTO_ERR if failure not mentioned above occurs.
\end{itemize*}
\item The device MUST set the \field{session_id} field to a unique session identifier only
    if the status is set to VIRTIO_CRYPTO_OK.
\end{itemize*}

\drivernormative{\subparagraph}{Session operation: destroy session}{Device Types / Crypto Device / Device
Operation / Control Virtqueue / Session operation / Session operation: destroy session}

\begin{itemize*}
\item The driver MUST set the \field{opcode} field based on service type: CIPHER, HASH, MAC, AEAD or AKCIPHER.
\item The driver MUST set the \field{session_id} to a valid value assigned by the device
    when the session was created.
\end{itemize*}

\devicenormative{\subparagraph}{Session operation: destroy session}{Device Types / Crypto Device / Device
Operation / Control Virtqueue / Session operation / Session operation: destroy session}

\begin{itemize*}
\item The device MUST set the \field{status} field to one of the following values of enum VIRTIO_CRYPTO_STATUS.
\begin{itemize*}
\item VIRTIO_CRYPTO_OK if a session is created successfully.
\item VIRTIO_CRYPTO_ERR if any failure occurs.
\end{itemize*}
\end{itemize*}


\subsubsection{Data Virtqueue}\label{sec:Device Types / Crypto Device / Device Operation / Data Virtqueue}

The driver uses the data virtqueues to transmit crypto operation requests to the device,
and completes the crypto operations.

The header for dataq is as follows:

\begin{lstlisting}
struct virtio_crypto_op_header {
#define VIRTIO_CRYPTO_CIPHER_ENCRYPT \
    VIRTIO_CRYPTO_OPCODE(VIRTIO_CRYPTO_SERVICE_CIPHER, 0x00)
#define VIRTIO_CRYPTO_CIPHER_DECRYPT \
    VIRTIO_CRYPTO_OPCODE(VIRTIO_CRYPTO_SERVICE_CIPHER, 0x01)
#define VIRTIO_CRYPTO_HASH \
    VIRTIO_CRYPTO_OPCODE(VIRTIO_CRYPTO_SERVICE_HASH, 0x00)
#define VIRTIO_CRYPTO_MAC \
    VIRTIO_CRYPTO_OPCODE(VIRTIO_CRYPTO_SERVICE_MAC, 0x00)
#define VIRTIO_CRYPTO_AEAD_ENCRYPT \
    VIRTIO_CRYPTO_OPCODE(VIRTIO_CRYPTO_SERVICE_AEAD, 0x00)
#define VIRTIO_CRYPTO_AEAD_DECRYPT \
    VIRTIO_CRYPTO_OPCODE(VIRTIO_CRYPTO_SERVICE_AEAD, 0x01)
#define VIRTIO_CRYPTO_AKCIPHER_ENCRYPT \
    VIRTIO_CRYPTO_OPCODE(VIRTIO_CRYPTO_SERVICE_AKCIPHER, 0x00)
#define VIRTIO_CRYPTO_AKCIPHER_DECRYPT \
    VIRTIO_CRYPTO_OPCODE(VIRTIO_CRYPTO_SERVICE_AKCIPHER, 0x01)
#define VIRTIO_CRYPTO_AKCIPHER_SIGN \
    VIRTIO_CRYPTO_OPCODE(VIRTIO_CRYPTO_SERVICE_AKCIPHER, 0x02)
#define VIRTIO_CRYPTO_AKCIPHER_VERIFY \
    VIRTIO_CRYPTO_OPCODE(VIRTIO_CRYPTO_SERVICE_AKCIPHER, 0x03)
    le32 opcode;
    /* algo should be service-specific algorithms */
    le32 algo;
    le64 session_id;
#define VIRTIO_CRYPTO_FLAG_SESSION_MODE 1
    /* control flag to control the request */
    le32 flag;
    le32 padding;
};
\end{lstlisting}

\begin{note}
If VIRTIO_CRYPTO_F_REVISION_1 is not negotiated the \field{flag} is ignored.

If VIRTIO_CRYPTO_F_REVISION_1 is negotiated but VIRTIO_CRYPTO_F_<SERVICE>_STATELESS_MODE
is not negotiated, then the device SHOULD reject <SERVICE> requests if
VIRTIO_CRYPTO_FLAG_SESSION_MODE is not set (in \field{flag}).
\end{note}

The dataq request is composed of four parts:
\begin{lstlisting}
struct virtio_crypto_op_data_req {
    /* Device read only portion */

    struct virtio_crypto_op_header header;

#define VIRTIO_CRYPTO_DATAQ_OP_SPEC_HDR_LEGACY 48
    /* fixed length fields, opcode specific */
    u8 op_flf[flf_len];

    /* Device read && write portion */
    /* variable length fields, opcode specific */
    u8 op_vlf[vlf_len];

    /* Device write only portion */
    struct virtio_crypto_inhdr inhdr;
};
\end{lstlisting}

\field{header} is a general header (see above).

\field{op_flf} is the opcode (in \field{header}) specific header.

\field{flf_len} depends on the VIRTIO_CRYPTO_F_REVISION_1 feature bit
(see below).

\field{op_vlf} is the opcode (in \field{header}) specific parameters.

\field{vlf_len} is the size of the specific structure used.

\begin{itemize*}
\item If the the opcode (in \field{header}) is VIRTIO_CRYPTO_CIPHER_ENCRYPT
    or VIRTIO_CRYPTO_CIPHER_DECRYPT then:
    \begin{itemize*}
    \item If VIRTIO_CRYPTO_F_CIPHER_STATELESS_MODE is negotiated, \field{op_flf} is
        struct virtio_crypto_sym_data_flf_stateless, and \field{op_vlf} is struct
        virtio_crypto_sym_data_vlf_stateless.
    \item If VIRTIO_CRYPTO_F_CIPHER_STATELESS_MODE is NOT negotiated, \field{op_flf}
        is struct virtio_crypto_sym_data_flf if VIRTIO_CRYPTO_F_REVISION_1 is negotiated
        and struct virtio_crypto_sym_data_flf is padded to 48 bytes if NOT negotiated,
        and \field{op_vlf} is struct virtio_crypto_sym_data_vlf.
    \end{itemize*}
\item If the the opcode (in \field{header}) is VIRTIO_CRYPTO_HASH:
    \begin{itemize*}
    \item If VIRTIO_CRYPTO_F_HASH_STATELESS_MODE is negotiated, \field{op_flf} is
        struct virtio_crypto_hash_data_flf_stateless, and \field{op_vlf} is struct
        virtio_crypto_hash_data_vlf_stateless.
    \item If VIRTIO_CRYPTO_F_HASH_STATELESS_MODE is NOT negotiated, \field{op_flf}
        is struct virtio_crypto_hash_data_flf if VIRTIO_CRYPTO_F_REVISION_1 is negotiated
        and struct virtio_crypto_hash_data_flf is padded to 48 bytes if NOT negotiated,
        and \field{op_vlf} is struct virtio_crypto_hash_data_vlf.
    \end{itemize*}
\item If the the opcode (in \field{header}) is VIRTIO_CRYPTO_MAC:
    \begin{itemize*}
    \item If VIRTIO_CRYPTO_F_MAC_STATELESS_MODE is negotiated, \field{op_flf} is
        struct virtio_crypto_mac_data_flf_stateless, and \field{op_vlf} is struct
        virtio_crypto_mac_data_vlf_stateless.
    \item If VIRTIO_CRYPTO_F_MAC_STATELESS_MODE is NOT negotiated, \field{op_flf}
        is struct virtio_crypto_mac_data_flf if VIRTIO_CRYPTO_F_REVISION_1 is negotiated
        and struct virtio_crypto_mac_data_flf is padded to 48 bytes if NOT negotiated,
        and \field{op_vlf} is struct virtio_crypto_mac_data_vlf.
    \end{itemize*}
\item If the the opcode (in \field{header}) is VIRTIO_CRYPTO_AEAD_ENCRYPT
    or VIRTIO_CRYPTO_AEAD_DECRYPT then:
    \begin{itemize*}
    \item If VIRTIO_CRYPTO_F_AEAD_STATELESS_MODE is negotiated, \field{op_flf} is
        struct virtio_crypto_aead_data_flf_stateless, and \field{op_vlf} is struct
        virtio_crypto_aead_data_vlf_stateless.
    \item If VIRTIO_CRYPTO_F_AEAD_STATELESS_MODE is NOT negotiated, \field{op_flf}
        is struct virtio_crypto_aead_data_flf if VIRTIO_CRYPTO_F_REVISION_1 is negotiated
        and struct virtio_crypto_aead_data_flf is padded to 48 bytes if NOT negotiated,
        and \field{op_vlf} is struct virtio_crypto_aead_data_vlf.
    \end{itemize*}
\item If the opcode (in \field{header}) is VIRTIO_CRYPTO_AKCIPHER_ENCRYPT, VIRTIO_CRYPTO_AKCIPHER_DECRYPT,
    VIRTIO_CRYPTO_AKCIPHER_SIGN or VIRTIO_CRYPTO_AKCIPHER_VERIFY then:
    \begin{itemize*}
    \item If VIRTIO_CRYPTO_F_AKCIPHER_STATELESS_MODE is negotiated, \field{op_flf} is
        struct virtio_crypto_akcipher_data_flf_statless, and \field{op_vlf} is struct
        virtio_crypto_akcipher_data_vlf_stateless.
    \item If VIRTIO_CRYPTO_F_AKCIPHER_STATELESS_MODE is NOT negotiated, \field{op_flf}
        is struct virtio_crypto_akcipher_data_flf if VIRTIO_CRYPTO_F_REVISION_1 is negotiated
        and struct virtio_crypto_akcipher_data_flf is padded to 48 bytes if NOT negotiated,
        and \field{op_vlf} is struct virtio_crypto_akcipher_data_vlf.
    \end{itemize*}
\end{itemize*}

\field{inhdr} is a unified input header that used to return the status of
the operations, is defined as follows:

\begin{lstlisting}
struct virtio_crypto_inhdr {
    u8 status;
};
\end{lstlisting}

\subsubsection{HASH Service Operation}\label{sec:Device Types / Crypto Device / Device Operation / HASH Service Operation}

Session mode HASH service requests are as follows:

\begin{lstlisting}
struct virtio_crypto_hash_data_flf {
    /* length of source data */
    le32 src_data_len;
    /* hash result length */
    le32 hash_result_len;
};

struct virtio_crypto_hash_data_vlf {
    /* Device read only portion */
    /* Source data */
    u8 src_data[src_data_len];

    /* Device write only portion */
    /* Hash result data */
    u8 hash_result[hash_result_len];
};
\end{lstlisting}

Each data request uses the virtio_crypto_hash_data_flf structure and the
virtio_crypto_hash_data_vlf structure to store information used to run the
HASH operations.

\field{src_data} is the source data that will be processed.
\field{src_data_len} is the length of source data.
\field{hash_result} is the result data and \field{hash_result_len} is the length
of it.

Stateless mode HASH service requests are as follows:

\begin{lstlisting}
struct virtio_crypto_hash_data_flf_stateless {
    struct {
        /* See VIRTIO_CRYPTO_HASH_* above */
        le32 algo;
    } sess_para;

    /* length of source data */
    le32 src_data_len;
    /* hash result length */
    le32 hash_result_len;
    le32 reserved;
};
struct virtio_crypto_hash_data_vlf_stateless {
    /* Device read only portion */
    /* Source data */
    u8 src_data[src_data_len];

    /* Device write only portion */
    /* Hash result data */
    u8 hash_result[hash_result_len];
};
\end{lstlisting}

\drivernormative{\paragraph}{HASH Service Operation}{Device Types / Crypto Device / Device Operation / HASH Service Operation}

\begin{itemize*}
\item If the driver uses the session mode, then the driver MUST set \field{session_id}
    in struct virtio_crypto_op_header to a valid value assigned by the device when the
    session was created.
\item If the VIRTIO_CRYPTO_F_HASH_STATELESS_MODE feature bit is negotiated, 1) if the
    driver uses the stateless mode, then the driver MUST set the \field{flag} field in
    struct virtio_crypto_op_header to ZERO and MUST set the fields in struct
    virtio_crypto_hash_data_flf_stateless.sess_para, 2) if the driver uses the session
    mode, then the driver MUST set the \field{flag} field in struct virtio_crypto_op_header
    to VIRTIO_CRYPTO_FLAG_SESSION_MODE.
\item The driver MUST set \field{opcode} in struct virtio_crypto_op_header to VIRTIO_CRYPTO_HASH.
\end{itemize*}

\devicenormative{\paragraph}{HASH Service Operation}{Device Types / Crypto Device / Device Operation / HASH Service Operation}

\begin{itemize*}
\item The device MUST use the corresponding structure according to the \field{opcode}
    in the data general header.
\item If the VIRTIO_CRYPTO_F_HASH_STATELESS_MODE feature bit is negotiated, the device
    MUST parse \field{flag} field in struct virtio_crypto_op_header in order to decide
    which mode the driver uses.
\item The device MUST copy the results of HASH operations in the hash_result[] if HASH
    operations success.
\item The device MUST set \field{status} in struct virtio_crypto_inhdr to one of the
    following values of enum VIRTIO_CRYPTO_STATUS:
\begin{itemize*}
\item VIRTIO_CRYPTO_OK if the operation success.
\item VIRTIO_CRYPTO_NOTSUPP if the requested algorithm or operation is unsupported.
\item VIRTIO_CRYPTO_INVSESS if the session ID invalid when in session mode.
\item VIRTIO_CRYPTO_ERR if any failure not mentioned above occurs.
\end{itemize*}
\end{itemize*}


\subsubsection{MAC Service Operation}\label{sec:Device Types / Crypto Device / Device Operation / MAC Service Operation}

Session mode MAC service requests are as follows:

\begin{lstlisting}
struct virtio_crypto_mac_data_flf {
    struct virtio_crypto_hash_data_flf hdr;
};

struct virtio_crypto_mac_data_vlf {
    /* Device read only portion */
    /* Source data */
    u8 src_data[src_data_len];

    /* Device write only portion */
    /* Hash result data */
    u8 hash_result[hash_result_len];
};
\end{lstlisting}

Each request uses the virtio_crypto_mac_data_flf structure and the
virtio_crypto_mac_data_vlf structure to store information used to run the
MAC operations.

\field{src_data} is the source data that will be processed.
\field{src_data_len} is the length of source data.
\field{hash_result} is the result data and \field{hash_result_len} is the length
of it.

Stateless mode MAC service requests are as follows:

\begin{lstlisting}
struct virtio_crypto_mac_data_flf_stateless {
    struct {
        /* See VIRTIO_CRYPTO_MAC_* above */
        le32 algo;
        /* length of authenticated key */
        le32 auth_key_len;
    } sess_para;

    /* length of source data */
    le32 src_data_len;
    /* hash result length */
    le32 hash_result_len;
};

struct virtio_crypto_mac_data_vlf_stateless {
    /* Device read only portion */
    /* The authenticated key */
    u8 auth_key[auth_key_len];
    /* Source data */
    u8 src_data[src_data_len];

    /* Device write only portion */
    /* Hash result data */
    u8 hash_result[hash_result_len];
};
\end{lstlisting}

\field{auth_key} is the authenticated key that will be used during the process.
\field{auth_key_len} is the length of the key.

\drivernormative{\paragraph}{MAC Service Operation}{Device Types / Crypto Device / Device Operation / MAC Service Operation}

\begin{itemize*}
\item If the driver uses the session mode, then the driver MUST set \field{session_id}
    in struct virtio_crypto_op_header to a valid value assigned by the device when the
    session was created.
\item If the VIRTIO_CRYPTO_F_MAC_STATELESS_MODE feature bit is negotiated, 1) if the
    driver uses the stateless mode, then the driver MUST set the \field{flag} field
    in struct virtio_crypto_op_header to ZERO and MUST set the fields in struct
    virtio_crypto_mac_data_flf_stateless.sess_para, 2) if the driver uses the session
    mode, then the driver MUST set the \field{flag} field in struct virtio_crypto_op_header
    to VIRTIO_CRYPTO_FLAG_SESSION_MODE.
\item The driver MUST set \field{opcode} in struct virtio_crypto_op_header to VIRTIO_CRYPTO_MAC.
\end{itemize*}

\devicenormative{\paragraph}{MAC Service Operation}{Device Types / Crypto Device / Device Operation / MAC Service Operation}

\begin{itemize*}
\item If the VIRTIO_CRYPTO_F_MAC_STATELESS_MODE feature bit is negotiated, the device
    MUST parse \field{flag} field in struct virtio_crypto_op_header in order to decide
	which mode the driver uses.
\item The device MUST copy the results of MAC operations in the hash_result[] if HASH
    operations success.
\item The device MUST set \field{status} in struct virtio_crypto_inhdr to one of the
    following values of enum VIRTIO_CRYPTO_STATUS:
\begin{itemize*}
\item VIRTIO_CRYPTO_OK if the operation success.
\item VIRTIO_CRYPTO_NOTSUPP if the requested algorithm or operation is unsupported.
\item VIRTIO_CRYPTO_INVSESS if the session ID invalid when in session mode.
\item VIRTIO_CRYPTO_ERR if any failure not mentioned above occurs.
\end{itemize*}
\end{itemize*}

\subsubsection{Symmetric algorithms Operation}\label{sec:Device Types / Crypto Device / Device Operation / Symmetric algorithms Operation}

Session mode CIPHER service requests are as follows:

\begin{lstlisting}
struct virtio_crypto_cipher_data_flf {
    /*
     * Byte Length of valid IV/Counter data pointed to by the below iv data.
     *
     * For block ciphers in CBC or F8 mode, or for Kasumi in F8 mode, or for
     *   SNOW3G in UEA2 mode, this is the length of the IV (which
     *   must be the same as the block length of the cipher).
     * For block ciphers in CTR mode, this is the length of the counter
     *   (which must be the same as the block length of the cipher).
     */
    le32 iv_len;
    /* length of source data */
    le32 src_data_len;
    /* length of destination data */
    le32 dst_data_len;
    le32 padding;
};

struct virtio_crypto_cipher_data_vlf {
    /* Device read only portion */

    /*
     * Initialization Vector or Counter data.
     *
     * For block ciphers in CBC or F8 mode, or for Kasumi in F8 mode, or for
     *   SNOW3G in UEA2 mode, this is the Initialization Vector (IV)
     *   value.
     * For block ciphers in CTR mode, this is the counter.
     * For AES-XTS, this is the 128bit tweak, i, from IEEE Std 1619-2007.
     *
     * The IV/Counter will be updated after every partial cryptographic
     * operation.
     */
    u8 iv[iv_len];
    /* Source data */
    u8 src_data[src_data_len];

    /* Device write only portion */
    /* Destination data */
    u8 dst_data[dst_data_len];
};
\end{lstlisting}

Session mode requests of algorithm chaining are as follows:

\begin{lstlisting}
struct virtio_crypto_alg_chain_data_flf {
    le32 iv_len;
    /* Length of source data */
    le32 src_data_len;
    /* Length of destination data */
    le32 dst_data_len;
    /* Starting point for cipher processing in source data */
    le32 cipher_start_src_offset;
    /* Length of the source data that the cipher will be computed on */
    le32 len_to_cipher;
    /* Starting point for hash processing in source data */
    le32 hash_start_src_offset;
    /* Length of the source data that the hash will be computed on */
    le32 len_to_hash;
    /* Length of the additional auth data */
    le32 aad_len;
    /* Length of the hash result */
    le32 hash_result_len;
    le32 reserved;
};

struct virtio_crypto_alg_chain_data_vlf {
    /* Device read only portion */

    /* Initialization Vector or Counter data */
    u8 iv[iv_len];
    /* Source data */
    u8 src_data[src_data_len];
    /* Additional authenticated data if exists */
    u8 aad[aad_len];

    /* Device write only portion */

    /* Destination data */
    u8 dst_data[dst_data_len];
    /* Hash result data */
    u8 hash_result[hash_result_len];
};
\end{lstlisting}

Session mode requests of symmetric algorithm are as follows:

\begin{lstlisting}
struct virtio_crypto_sym_data_flf {
    /* Device read only portion */

#define VIRTIO_CRYPTO_SYM_DATA_REQ_HDR_SIZE    40
    u8 op_type_flf[VIRTIO_CRYPTO_SYM_DATA_REQ_HDR_SIZE];

    /* See above VIRTIO_CRYPTO_SYM_OP_* */
    le32 op_type;
    le32 padding;
};

struct virtio_crypto_sym_data_vlf {
    u8 op_type_vlf[sym_para_len];
};
\end{lstlisting}

Each request uses the virtio_crypto_sym_data_flf structure and the
virtio_crypto_sym_data_flf structure to store information used to run the
CIPHER operations.

\field{op_type_flf} is the \field{op_type} specific header, it MUST starts
with or be one of the following structures:
\begin{itemize*}
\item struct virtio_crypto_cipher_data_flf
\item struct virtio_crypto_alg_chain_data_flf
\end{itemize*}

The length of \field{op_type_flf} is fixed to 40 bytes, the data of unused
part (if has) will be ignored.

\field{op_type_vlf} is the \field{op_type} specific parameters, it MUST starts
with or be one of the following structures:
\begin{itemize*}
\item struct virtio_crypto_cipher_data_vlf
\item struct virtio_crypto_alg_chain_data_vlf
\end{itemize*}

\field{sym_para_len} is the size of the specific structure used.

Stateless mode CIPHER service requests are as follows:

\begin{lstlisting}
struct virtio_crypto_cipher_data_flf_stateless {
    struct {
        /* See VIRTIO_CRYPTO_CIPHER* above */
        le32 algo;
        /* length of key */
        le32 key_len;

        /* See VIRTIO_CRYPTO_OP_* above */
        le32 op;
    } sess_para;

    /*
     * Byte Length of valid IV/Counter data pointed to by the below iv data.
     */
    le32 iv_len;
    /* length of source data */
    le32 src_data_len;
    /* length of destination data */
    le32 dst_data_len;
};

struct virtio_crypto_cipher_data_vlf_stateless {
    /* Device read only portion */

    /* The cipher key */
    u8 cipher_key[key_len];

    /* Initialization Vector or Counter data. */
    u8 iv[iv_len];
    /* Source data */
    u8 src_data[src_data_len];

    /* Device write only portion */
    /* Destination data */
    u8 dst_data[dst_data_len];
};
\end{lstlisting}

Stateless mode requests of algorithm chaining are as follows:

\begin{lstlisting}
struct virtio_crypto_alg_chain_data_flf_stateless {
    struct {
        /* See VIRTIO_CRYPTO_SYM_ALG_CHAIN_ORDER_* above */
        le32 alg_chain_order;
        /* length of the additional authenticated data in bytes */
        le32 aad_len;

        struct {
            /* See VIRTIO_CRYPTO_CIPHER* above */
            le32 algo;
            /* length of key */
            le32 key_len;
            /* See VIRTIO_CRYPTO_OP_* above */
            le32 op;
        } cipher;

        struct {
            /* See VIRTIO_CRYPTO_HASH_* or VIRTIO_CRYPTO_MAC_* above */
            le32 algo;
            /* length of authenticated key */
            le32 auth_key_len;
            /* See VIRTIO_CRYPTO_SYM_HASH_MODE_* above */
            le32 hash_mode;
        } hash;
    } sess_para;

    le32 iv_len;
    /* Length of source data */
    le32 src_data_len;
    /* Length of destination data */
    le32 dst_data_len;
    /* Starting point for cipher processing in source data */
    le32 cipher_start_src_offset;
    /* Length of the source data that the cipher will be computed on */
    le32 len_to_cipher;
    /* Starting point for hash processing in source data */
    le32 hash_start_src_offset;
    /* Length of the source data that the hash will be computed on */
    le32 len_to_hash;
    /* Length of the additional auth data */
    le32 aad_len;
    /* Length of the hash result */
    le32 hash_result_len;
    le32 reserved;
};

struct virtio_crypto_alg_chain_data_vlf_stateless {
    /* Device read only portion */

    /* The cipher key */
    u8 cipher_key[key_len];
    /* The auth key */
    u8 auth_key[auth_key_len];
    /* Initialization Vector or Counter data */
    u8 iv[iv_len];
    /* Additional authenticated data if exists */
    u8 aad[aad_len];
    /* Source data */
    u8 src_data[src_data_len];

    /* Device write only portion */

    /* Destination data */
    u8 dst_data[dst_data_len];
    /* Hash result data */
    u8 hash_result[hash_result_len];
};
\end{lstlisting}

Stateless mode requests of symmetric algorithm are as follows:

\begin{lstlisting}
struct virtio_crypto_sym_data_flf_stateless {
    /* Device read only portion */
#define VIRTIO_CRYPTO_SYM_DATE_REQ_HDR_STATELESS_SIZE    72
    u8 op_type_flf[VIRTIO_CRYPTO_SYM_DATE_REQ_HDR_STATELESS_SIZE];

    /* Device write only portion */
    /* See above VIRTIO_CRYPTO_SYM_OP_* */
    le32 op_type;
};

struct virtio_crypto_sym_data_vlf_stateless {
    u8 op_type_vlf[sym_para_len];
};
\end{lstlisting}

\field{op_type_flf} is the \field{op_type} specific header, it MUST starts
with or be one of the following structures:
\begin{itemize*}
\item struct virtio_crypto_cipher_data_flf_stateless
\item struct virtio_crypto_alg_chain_data_flf_stateless
\end{itemize*}

The length of \field{op_type_flf} is fixed to 72 bytes, the data of unused
part (if has) will be ignored.

\field{op_type_vlf} is the \field{op_type} specific parameters, it MUST starts
with or be one of the following structures:
\begin{itemize*}
\item struct virtio_crypto_cipher_data_vlf_stateless
\item struct virtio_crypto_alg_chain_data_vlf_stateless
\end{itemize*}

\field{sym_para_len} is the size of the specific structure used.

\drivernormative{\paragraph}{Symmetric algorithms Operation}{Device Types / Crypto Device / Device Operation / Symmetric algorithms Operation}

\begin{itemize*}
\item If the driver uses the session mode, then the driver MUST set \field{session_id}
    in struct virtio_crypto_op_header to a valid value assigned by the device when the
    session was created.
\item If the VIRTIO_CRYPTO_F_CIPHER_STATELESS_MODE feature bit is negotiated, 1) if the
    driver uses the stateless mode, then the driver MUST set the \field{flag} field in
    struct virtio_crypto_op_header to ZERO and MUST set the fields in struct
    virtio_crypto_cipher_data_flf_stateless.sess_para or struct
    virtio_crypto_alg_chain_data_flf_stateless.sess_para, 2) if the driver uses the
    session mode, then the driver MUST set the \field{flag} field in struct
    virtio_crypto_op_header to VIRTIO_CRYPTO_FLAG_SESSION_MODE.
\item The driver MUST set the \field{opcode} field in struct virtio_crypto_op_header
    to VIRTIO_CRYPTO_CIPHER_ENCRYPT or VIRTIO_CRYPTO_CIPHER_DECRYPT.
\item The driver MUST specify the fields of struct virtio_crypto_cipher_data_flf in
    struct virtio_crypto_sym_data_flf and struct virtio_crypto_cipher_data_vlf in
    struct virtio_crypto_sym_data_vlf if the request is based on VIRTIO_CRYPTO_SYM_OP_CIPHER.
\item The driver MUST specify the fields of struct virtio_crypto_alg_chain_data_flf
    in struct virtio_crypto_sym_data_flf and struct virtio_crypto_alg_chain_data_vlf
    in struct virtio_crypto_sym_data_vlf if the request is of the VIRTIO_CRYPTO_SYM_OP_ALGORITHM_CHAINING
    type.
\end{itemize*}

\devicenormative{\paragraph}{Symmetric algorithms Operation}{Device Types / Crypto Device / Device Operation / Symmetric algorithms Operation}

\begin{itemize*}
\item If the VIRTIO_CRYPTO_F_CIPHER_STATELESS_MODE feature bit is negotiated, the device
    MUST parse \field{flag} field in struct virtio_crypto_op_header in order to decide
	which mode the driver uses.
\item The device MUST parse the virtio_crypto_sym_data_req based on the \field{opcode}
    field in general header.
\item The device MUST parse the fields of struct virtio_crypto_cipher_data_flf in
    struct virtio_crypto_sym_data_flf and struct virtio_crypto_cipher_data_vlf in
    struct virtio_crypto_sym_data_vlf if the request is based on VIRTIO_CRYPTO_SYM_OP_CIPHER.
\item The device MUST parse the fields of struct virtio_crypto_alg_chain_data_flf
    in struct virtio_crypto_sym_data_flf and struct virtio_crypto_alg_chain_data_vlf
    in struct virtio_crypto_sym_data_vlf if the request is of the VIRTIO_CRYPTO_SYM_OP_ALGORITHM_CHAINING
    type.
\item The device MUST copy the result of cryptographic operation in the dst_data[] in
    both plain CIPHER mode and algorithms chain mode.
\item The device MUST check the \field{para}.\field{add_len} is bigger than 0 before
    parse the additional authenticated data in plain algorithms chain mode.
\item The device MUST copy the result of HASH/MAC operation in the hash_result[] is
    of the VIRTIO_CRYPTO_SYM_OP_ALGORITHM_CHAINING type.
\item The device MUST set the \field{status} field in struct virtio_crypto_inhdr to
    one of the following values of enum VIRTIO_CRYPTO_STATUS:
\begin{itemize*}
\item VIRTIO_CRYPTO_OK if the operation success.
\item VIRTIO_CRYPTO_NOTSUPP if the requested algorithm or operation is unsupported.
\item VIRTIO_CRYPTO_INVSESS if the session ID is invalid in session mode.
\item VIRTIO_CRYPTO_ERR if failure not mentioned above occurs.
\end{itemize*}
\end{itemize*}

\subsubsection{AEAD Service Operation}\label{sec:Device Types / Crypto Device / Device Operation / AEAD Service Operation}

Session mode requests of symmetric algorithm are as follows:

\begin{lstlisting}
struct virtio_crypto_aead_data_flf {
    /*
     * Byte Length of valid IV data.
     *
     * For GCM mode, this is either 12 (for 96-bit IVs) or 16, in which
     *   case iv points to J0.
     * For CCM mode, this is the length of the nonce, which can be in the
     *   range 7 to 13 inclusive.
     */
    le32 iv_len;
    /* length of additional auth data */
    le32 aad_len;
    /* length of source data */
    le32 src_data_len;
    /* length of dst data, this should be at least src_data_len + tag_len */
    le32 dst_data_len;
    /* Authentication tag length */
    le32 tag_len;
    le32 reserved;
};

struct virtio_crypto_aead_data_vlf {
    /* Device read only portion */

    /*
     * Initialization Vector data.
     *
     * For GCM mode, this is either the IV (if the length is 96 bits) or J0
     *   (for other sizes), where J0 is as defined by NIST SP800-38D.
     *   Regardless of the IV length, a full 16 bytes needs to be allocated.
     * For CCM mode, the first byte is reserved, and the nonce should be
     *   written starting at &iv[1] (to allow space for the implementation
     *   to write in the flags in the first byte).  Note that a full 16 bytes
     *   should be allocated, even though the iv_len field will have
     *   a value less than this.
     *
     * The IV will be updated after every partial cryptographic operation.
     */
    u8 iv[iv_len];
    /* Source data */
    u8 src_data[src_data_len];
    /* Additional authenticated data if exists */
    u8 aad[aad_len];

    /* Device write only portion */
    /* Pointer to output data */
    u8 dst_data[dst_data_len];
};
\end{lstlisting}

Each request uses the virtio_crypto_aead_data_flf structure and the
virtio_crypto_aead_data_flf structure to store information used to run the
AEAD operations.

Stateless mode AEAD service requests are as follows:

\begin{lstlisting}
struct virtio_crypto_aead_data_flf_stateless {
    struct {
        /* See VIRTIO_CRYPTO_AEAD_* above */
        le32 algo;
        /* length of key */
        le32 key_len;
        /* encrypt or decrypt, See above VIRTIO_CRYPTO_OP_* */
        le32 op;
    } sess_para;

    /* Byte Length of valid IV data. */
    le32 iv_len;
    /* Authentication tag length */
    le32 tag_len;
    /* length of additional auth data */
    le32 aad_len;
    /* length of source data */
    le32 src_data_len;
    /* length of dst data, this should be at least src_data_len + tag_len */
    le32 dst_data_len;
};

struct virtio_crypto_aead_data_vlf_stateless {
    /* Device read only portion */

    /* The cipher key */
    u8 key[key_len];
    /* Initialization Vector data. */
    u8 iv[iv_len];
    /* Source data */
    u8 src_data[src_data_len];
    /* Additional authenticated data if exists */
    u8 aad[aad_len];

    /* Device write only portion */
    /* Pointer to output data */
    u8 dst_data[dst_data_len];
};
\end{lstlisting}

\drivernormative{\paragraph}{AEAD Service Operation}{Device Types / Crypto Device / Device Operation / AEAD Service Operation}

\begin{itemize*}
\item If the driver uses the session mode, then the driver MUST set
    \field{session_id} in struct virtio_crypto_op_header to a valid value assigned
    by the device when the session was created.
\item If the VIRTIO_CRYPTO_F_AEAD_STATELESS_MODE feature bit is negotiated, 1) if
    the driver uses the stateless mode, then the driver MUST set the \field{flag}
    field in struct virtio_crypto_op_header to ZERO and MUST set the fields in
    struct virtio_crypto_aead_data_flf_stateless.sess_para, 2) if the driver uses
    the session mode, then the driver MUST set the \field{flag} field in struct
    virtio_crypto_op_header to VIRTIO_CRYPTO_FLAG_SESSION_MODE.
\item The driver MUST set the \field{opcode} field in struct virtio_crypto_op_header
    to VIRTIO_CRYPTO_AEAD_ENCRYPT or VIRTIO_CRYPTO_AEAD_DECRYPT.
\end{itemize*}

\devicenormative{\paragraph}{AEAD Service Operation}{Device Types / Crypto Device / Device Operation / AEAD Service Operation}

\begin{itemize*}
\item If the VIRTIO_CRYPTO_F_AEAD_STATELESS_MODE feature bit is negotiated, the
    device MUST parse the virtio_crypto_aead_data_vlf_stateless based on the \field{opcode}
	field in general header.
\item The device MUST copy the result of cryptographic operation in the dst_data[].
\item The device MUST copy the authentication tag in the dst_data[] offset the cipher result.
\item The device MUST set the \field{status} field in struct virtio_crypto_inhdr to
    one of the following values of enum VIRTIO_CRYPTO_STATUS:
\item When the \field{opcode} field is VIRTIO_CRYPTO_AEAD_DECRYPT, the device MUST
    verify and return the verification result to the driver.
\begin{itemize*}
\item VIRTIO_CRYPTO_OK if the operation success.
\item VIRTIO_CRYPTO_NOTSUPP if the requested algorithm or operation is unsupported.
\item VIRTIO_CRYPTO_BADMSG if the verification result is incorrect.
\item VIRTIO_CRYPTO_INVSESS if the session ID invalid when in session mode.
\item VIRTIO_CRYPTO_ERR if any failure not mentioned above occurs.
\end{itemize*}
\end{itemize*}

\subsubsection{AKCIPHER Service Operation}\label{sec:Device Types / Crypto Device / Device Operation / AKCIPHER Service Operation}

Session mode AKCIPHER requests are as follows:

\begin{lstlisting}
struct virtio_crypto_akcipher_data_flf {
    /* length of source data */
    le32 src_data_len;
    /* length of dst data */
    le32 dst_data_len;
};

struct virtio_crypto_akcipher_data_vlf {
    /* Device read only portion */
    /* Source data */
    u8 src_data[src_data_len];

    /* Device write only portion */
    /* Pointer to output data */
    u8 dst_data[dst_data_len];
};
\end{lstlisting}

Each data request uses the virtio_crypto_akcipher_flf structure and the virtio_crypto_akcipher_data_vlf
structure to store information used to run the AKCIPHER operations.

For encryption, decryption, and signing:
\field{src_data} is the source data that will be processed, note that for signing operations,
src_data stores the data to be signed, which usually is the digest of some data rather than the
data itself.
\field{src_data_len} is the length of source data.
\field{dst_result} is the result data and \field{dst_data_len} is the length of it. Note that the
length of the result is not always exactly equal to dst_data_len, the driver needs to check how
many bytes the device has written and calculate the actual length of the result.

For verification:
\field{src_data_len} refers to the length of the signature, and \field{dst_data_len} refers to
the length of signed data, where the signed data is usually the digest of some data.
\field{src_data} is spliced by the signature and the signed data, the src_data with the lower
address stores the signature, and the higher address stores the signed data.
\field{dst_data} is always empty for verification.

Different algorithms have different signature formats.
For the RSA algorithm, the result is determined by the padding algorithm specified by
\field{padding_algo} in structure virtio_crypto_rsa_session_para.

For the ECDSA algorithm, the signature is composed of the following
ASN.1 structure (see \hyperref[intro:rfc3279]{RFC3279})
and MUST be DER encoded (see \hyperref[intro:rfc6025]{rfc6025}).

\begin{lstlisting}
Ecdsa-Sig-Value ::= SEQUENCE {
    r INTEGER,
    s INTEGER
}
\end{lstlisting}

Stateless mode AKCIPHER service requests are as follows:
\begin{lstlisting}
struct virtio_crypto_akcipher_data_flf_stateless {
    struct {
        /* See VIRTIO_CYRPTO_AKCIPHER* above */
        le32 algo;
        /* See VIRTIO_CRYPTO_AKCIPHER_KEY_TYPE_* above */
        le32 key_type;
        /* length of key */
        le32 key_len;

        /* algothrim specific parameters described above */
        union {
            struct virtio_crypto_rsa_session_para rsa;
            struct virtio_crypto_ecdsa_session_para ecdsa;
        } u;
    } sess_para;

    /* length of source data */
    le32 src_data_len;
    /* length of destination data */
    le32 dst_data_len;
};

struct virtio_crypto_akcipher_data_vlf_stateless {
    /* Device read only portion */
    u8 akcipher_key[key_len];

    /* Source data */
    u8 src_data[src_data_len];

    /* Device write only portion */
    u8 dst_data[dst_data_len];
};
\end{lstlisting}

In stateless mode, the format of key and signature, the meaning of src_data and dst_data, are all the same
with session mode.

\drivernormative{\paragraph}{AKCIPHER Service Operation}{Device Types / Crypto Device / Device Operation / AKCIPHER Service Operation}

\begin{itemize*}
\item If the driver uses the session mode, then the driver MUST set
    \field{session_id} in struct virtio_crypto_op_header to a valid
    value assigned by the device when the session was created.
\item If the VIRTIO_CRYPTO_F_AKCIPHER_STATELESS_MODE feature bit is negotiated, 1) if the
    driver uses the stateless mode, then the driver MUST set the \field{flag} field in
    struct virtio_crypto_op_header to ZERO and MUST set the fields in struct
    virtio_crypto_akcipher_flf_stateless.sess_para, 2) if the driver uses the session
    mode, then the driver MUST set the \field{flag} field in struct virtio_crypto_op_header
    to VIRTIO_CRYPTO_FLAG_SESSION_MODE.
\item The driver MUST set the \field{opcode} field in struct virtio_crypto_op_header
    to one of VIRTIO_CRYPTO_AKCIPHER_ENCRYPT, VIRTIO_CRYPTO_AKCIPHER_DESTROY_SESSION,
    VIRTIO_CRYPTO_AKCIPHER_SIGN, and VIRTIO_CRYPTO_AKCIPHER_VERIFY.
\end{itemize*}

\devicenormative{\paragraph}{AKCIPHER Service Operation}{Device Types / Crypto Device / Device Operation / AKCIPHER Service Operation}

\begin{itemize*}
\item If the VIRTIO_CRYPTO_F_AKCIPHER_STATELESS_MODE feature bit is negotiated, the
    device MUST parse the virtio_crypto_akcipher_data_vlf_stateless based on the \field{opcode}
    field in general header.
\item The device MUST copy the result of cryptographic operation in the dst_data[].
\item The device MUST set the \field{status} field in struct virtio_crypto_inhdr to
    one of the following values of enum VIRTIO_CRYPTO_STATUS:
\begin{itemize*}
\item VIRTIO_CRYPTO_OK if the operation success.
\item VIRTIO_CRYPTO_NOTSUPP if the requested algorithm or operation is unsupported.
\item VIRTIO_CRYPTO_BADMSG if the verification result is incorrect.
\item VIRTIO_CRYPTO_INVSESS if the session ID invalid when in session mode.
\item VIRTIO_CRYPTO_KEY_REJECTED if the signature verification failed.
\item VIRTIO_CRYPTO_ERR if any failure not mentioned above occurs.
\end{itemize*}
\end{itemize*}

\section{Crypto Device}\label{sec:Device Types / Crypto Device}

The virtio crypto device is a virtual cryptography device as well as a
virtual cryptographic accelerator. The virtio crypto device provides the
following crypto services: CIPHER, MAC, HASH, AEAD and AKCIPHER. Virtio crypto
devices have a single control queue and at least one data queue. Crypto
operation requests are placed into a data queue, and serviced by the
device. Some crypto operation requests are only valid in the context of a
session. The role of the control queue is facilitating control operation
requests. Sessions management is realized with control operation
requests.

\subsection{Device ID}\label{sec:Device Types / Crypto Device / Device ID}

20

\subsection{Virtqueues}\label{sec:Device Types / Crypto Device / Virtqueues}

\begin{description}
\item[0] dataq1
\item[\ldots]
\item[N-1] dataqN
\item[N] controlq
\end{description}

N is set by \field{max_dataqueues}.

\subsection{Feature bits}\label{sec:Device Types / Crypto Device / Feature bits}

\begin{description}
\item VIRTIO_CRYPTO_F_REVISION_1 (0) revision 1. Revision 1 has a specific
    request format and other enhancements (which result in some additional
    requirements).
\item VIRTIO_CRYPTO_F_CIPHER_STATELESS_MODE (1) stateless mode requests are
    supported by the CIPHER service.
\item VIRTIO_CRYPTO_F_HASH_STATELESS_MODE (2) stateless mode requests are
    supported by the HASH service.
\item VIRTIO_CRYPTO_F_MAC_STATELESS_MODE (3) stateless mode requests are
    supported by the MAC service.
\item VIRTIO_CRYPTO_F_AEAD_STATELESS_MODE (4) stateless mode requests are
    supported by the AEAD service.
\item VIRTIO_CRYPTO_F_AKCIPHER_STATELESS_MODE (5) stateless mode requests are
    supported by the AKCIPHER service.
\end{description}


\subsubsection{Feature bit requirements}\label{sec:Device Types / Crypto Device / Feature bit requirements}

Some crypto feature bits require other crypto feature bits
(see \ref{drivernormative:Basic Facilities of a Virtio Device / Feature Bits}):

\begin{description}
\item[VIRTIO_CRYPTO_F_CIPHER_STATELESS_MODE] Requires VIRTIO_CRYPTO_F_REVISION_1.
\item[VIRTIO_CRYPTO_F_HASH_STATELESS_MODE] Requires VIRTIO_CRYPTO_F_REVISION_1.
\item[VIRTIO_CRYPTO_F_MAC_STATELESS_MODE] Requires VIRTIO_CRYPTO_F_REVISION_1.
\item[VIRTIO_CRYPTO_F_AEAD_STATELESS_MODE] Requires VIRTIO_CRYPTO_F_REVISION_1.
\item[VIRTIO_CRYPTO_F_AKCIPHER_STATELESS_MODE] Requires VIRTIO_CRYPTO_F_REVISION_1.
\end{description}

\subsection{Supported crypto services}\label{sec:Device Types / Crypto Device / Supported crypto services}

The following crypto services are defined:

\begin{lstlisting}
/* CIPHER (Symmetric Key Cipher) service */
#define VIRTIO_CRYPTO_SERVICE_CIPHER 0
/* HASH service */
#define VIRTIO_CRYPTO_SERVICE_HASH   1
/* MAC (Message Authentication Codes) service */
#define VIRTIO_CRYPTO_SERVICE_MAC    2
/* AEAD (Authenticated Encryption with Associated Data) service */
#define VIRTIO_CRYPTO_SERVICE_AEAD   3
/* AKCIPHER (Asymmetric Key Cipher) service */
#define VIRTIO_CRYPTO_SERVICE_AKCIPHER 4
\end{lstlisting}

The above constants designate bits used to indicate the which of crypto services are
offered by the device as described in, see \ref{sec:Device Types / Crypto Device / Device configuration layout}.

\subsubsection{CIPHER services}\label{sec:Device Types / Crypto Device / Supported crypto services / CIPHER services}

The following CIPHER algorithms are defined:

\begin{lstlisting}
#define VIRTIO_CRYPTO_NO_CIPHER                 0
#define VIRTIO_CRYPTO_CIPHER_ARC4               1
#define VIRTIO_CRYPTO_CIPHER_AES_ECB            2
#define VIRTIO_CRYPTO_CIPHER_AES_CBC            3
#define VIRTIO_CRYPTO_CIPHER_AES_CTR            4
#define VIRTIO_CRYPTO_CIPHER_DES_ECB            5
#define VIRTIO_CRYPTO_CIPHER_DES_CBC            6
#define VIRTIO_CRYPTO_CIPHER_3DES_ECB           7
#define VIRTIO_CRYPTO_CIPHER_3DES_CBC           8
#define VIRTIO_CRYPTO_CIPHER_3DES_CTR           9
#define VIRTIO_CRYPTO_CIPHER_KASUMI_F8          10
#define VIRTIO_CRYPTO_CIPHER_SNOW3G_UEA2        11
#define VIRTIO_CRYPTO_CIPHER_AES_F8             12
#define VIRTIO_CRYPTO_CIPHER_AES_XTS            13
#define VIRTIO_CRYPTO_CIPHER_ZUC_EEA3           14
\end{lstlisting}

The above constants have two usages:
\begin{enumerate}
\item As bit numbers, used to tell the driver which CIPHER algorithms
are supported by the device, see \ref{sec:Device Types / Crypto Device / Device configuration layout}.
\item As values, used to designate the algorithm in (CIPHER type) crypto
operation requests, see \ref{sec:Device Types / Crypto Device / Device Operation / Control Virtqueue / Session operation}.
\end{enumerate}

\subsubsection{HASH services}\label{sec:Device Types / Crypto Device / Supported crypto services / HASH services}

The following HASH algorithms are defined:

\begin{lstlisting}
#define VIRTIO_CRYPTO_NO_HASH            0
#define VIRTIO_CRYPTO_HASH_MD5           1
#define VIRTIO_CRYPTO_HASH_SHA1          2
#define VIRTIO_CRYPTO_HASH_SHA_224       3
#define VIRTIO_CRYPTO_HASH_SHA_256       4
#define VIRTIO_CRYPTO_HASH_SHA_384       5
#define VIRTIO_CRYPTO_HASH_SHA_512       6
#define VIRTIO_CRYPTO_HASH_SHA3_224      7
#define VIRTIO_CRYPTO_HASH_SHA3_256      8
#define VIRTIO_CRYPTO_HASH_SHA3_384      9
#define VIRTIO_CRYPTO_HASH_SHA3_512      10
#define VIRTIO_CRYPTO_HASH_SHA3_SHAKE128      11
#define VIRTIO_CRYPTO_HASH_SHA3_SHAKE256      12
\end{lstlisting}

The above constants have two usages:
\begin{enumerate}
\item As bit numbers, used to tell the driver which HASH algorithms
are supported by the device, see \ref{sec:Device Types / Crypto Device / Device configuration layout}.
\item As values, used to designate the algorithm in (HASH type) crypto
operation requires, see \ref{sec:Device Types / Crypto Device / Device Operation / Control Virtqueue / Session operation}.
\end{enumerate}

\subsubsection{MAC services}\label{sec:Device Types / Crypto Device / Supported crypto services / MAC services}

The following MAC algorithms are defined:

\begin{lstlisting}
#define VIRTIO_CRYPTO_NO_MAC                       0
#define VIRTIO_CRYPTO_MAC_HMAC_MD5                 1
#define VIRTIO_CRYPTO_MAC_HMAC_SHA1                2
#define VIRTIO_CRYPTO_MAC_HMAC_SHA_224             3
#define VIRTIO_CRYPTO_MAC_HMAC_SHA_256             4
#define VIRTIO_CRYPTO_MAC_HMAC_SHA_384             5
#define VIRTIO_CRYPTO_MAC_HMAC_SHA_512             6
#define VIRTIO_CRYPTO_MAC_CMAC_3DES                25
#define VIRTIO_CRYPTO_MAC_CMAC_AES                 26
#define VIRTIO_CRYPTO_MAC_KASUMI_F9                27
#define VIRTIO_CRYPTO_MAC_SNOW3G_UIA2              28
#define VIRTIO_CRYPTO_MAC_GMAC_AES                 41
#define VIRTIO_CRYPTO_MAC_GMAC_TWOFISH             42
#define VIRTIO_CRYPTO_MAC_CBCMAC_AES               49
#define VIRTIO_CRYPTO_MAC_CBCMAC_KASUMI_F9         50
#define VIRTIO_CRYPTO_MAC_XCBC_AES                 53
#define VIRTIO_CRYPTO_MAC_ZUC_EIA3                 54
\end{lstlisting}

The above constants have two usages:
\begin{enumerate}
\item As bit numbers, used to tell the driver which MAC algorithms
are supported by the device, see \ref{sec:Device Types / Crypto Device / Device configuration layout}.
\item As values, used to designate the algorithm in (MAC type) crypto
operation requests, see \ref{sec:Device Types / Crypto Device / Device Operation / Control Virtqueue / Session operation}.
\end{enumerate}

\subsubsection{AEAD services}\label{sec:Device Types / Crypto Device / Supported crypto services / AEAD services}

The following AEAD algorithms are defined:

\begin{lstlisting}
#define VIRTIO_CRYPTO_NO_AEAD     0
#define VIRTIO_CRYPTO_AEAD_GCM    1
#define VIRTIO_CRYPTO_AEAD_CCM    2
#define VIRTIO_CRYPTO_AEAD_CHACHA20_POLY1305  3
\end{lstlisting}

The above constants have two usages:
\begin{enumerate}
\item As bit numbers, used to tell the driver which AEAD algorithms
are supported by the device, see \ref{sec:Device Types / Crypto Device / Device configuration layout}.
\item As values, used to designate the algorithm in (DEAD type) crypto
operation requests, see \ref{sec:Device Types / Crypto Device / Device Operation / Control Virtqueue / Session operation}.
\end{enumerate}

\subsubsection{AKCIPHER services}\label{sec: Device Types / Crypto Device / Supported crypto services / AKCIPHER services}

The following AKCIPHER algorithms are defined:
\begin{lstlisting}
#define VIRTIO_CRYPTO_NO_AKCIPHER 0
#define VIRTIO_CRYPTO_AKCIPHER_RSA   1
#define VIRTIO_CRYPTO_AKCIPHER_ECDSA 2
\end{lstlisting}

The above constants have two usages:
\begin{enumerate}
\item As bit numbers, used to tell the driver which AKCIPHER algorithms
are supported by the device, see \ref{sec:Device Types / Crypto Device / Device configuration layout}.
\item As values, used to designate the algorithm in asymmetric crypto operation requests,
see \ref{sec:Device Types / Crypto Device / Device Operation / Control Virtqueue / Session operation}.
\end{enumerate}


\subsection{Device configuration layout}\label{sec:Device Types / Crypto Device / Device configuration layout}

Crypto device configuration uses the following layout structure:

\begin{lstlisting}
struct virtio_crypto_config {
    le32 status;
    le32 max_dataqueues;
    le32 crypto_services;
    /* Detailed algorithms mask */
    le32 cipher_algo_l;
    le32 cipher_algo_h;
    le32 hash_algo;
    le32 mac_algo_l;
    le32 mac_algo_h;
    le32 aead_algo;
    /* Maximum length of cipher key in bytes */
    le32 max_cipher_key_len;
    /* Maximum length of authenticated key in bytes */
    le32 max_auth_key_len;
    le32 akcipher_algo;
    /* Maximum size of each crypto request's content in bytes */
    le64 max_size;
};
\end{lstlisting}

\begin{description}
\item Currently, only one \field{status} bit is defined: VIRTIO_CRYPTO_S_HW_READY
    set indicates that the device is ready to process requests, this bit is read-only
    for the driver
\begin{lstlisting}
#define VIRTIO_CRYPTO_S_HW_READY  (1 << 0)
\end{lstlisting}

\item [\field{max_dataqueues}] is the maximum number of data virtqueues that can
    be configured by the device. The driver MAY use only one data queue, or it
    can use more to achieve better performance.

\item [\field{crypto_services}] crypto service offered, see \ref{sec:Device Types / Crypto Device / Supported crypto services}.

\item [\field{cipher_algo_l}] CIPHER algorithms bits 0-31, see \ref{sec:Device Types / Crypto Device / Supported crypto services  / CIPHER services}.

\item [\field{cipher_algo_h}] CIPHER algorithms bits 32-63, see \ref{sec:Device Types / Crypto Device / Supported crypto services  / CIPHER services}.

\item [\field{hash_algo}] HASH algorithms bits, see \ref{sec:Device Types / Crypto Device / Supported crypto services  / HASH services}.

\item [\field{mac_algo_l}] MAC algorithms bits 0-31, see \ref{sec:Device Types / Crypto Device / Supported crypto services  / MAC services}.

\item [\field{mac_algo_h}] MAC algorithms bits 32-63, see \ref{sec:Device Types / Crypto Device / Supported crypto services  / MAC services}.

\item [\field{aead_algo}] AEAD algorithms bits, see \ref{sec:Device Types / Crypto Device / Supported crypto services  / AEAD services}.

\item [\field{max_cipher_key_len}] is the maximum length of cipher key supported by the device.

\item [\field{max_auth_key_len}] is the maximum length of authenticated key supported by the device.

\item [\field{akcipher_algo}] AKCIPHER algorithms bit 0-31, see \ref{sec: Device Types / Crypto Device / Supported crypto services / AKCIPHER services}.

\item [\field{max_size}] is the maximum size of the variable-length parameters of
    data operation of each crypto request's content supported by the device.
\end{description}

\begin{note}
Unless explicitly stated otherwise all lengths and sizes are in bytes.
\end{note}

\devicenormative{\subsubsection}{Device configuration layout}{Device Types / Crypto Device / Device configuration layout}

\begin{itemize*}
\item The device MUST set \field{max_dataqueues} to between 1 and 65535 inclusive.
\item The device MUST set the \field{status} with valid flags, undefined flags MUST NOT be set.
\item The device MUST accept and handle requests after \field{status} is set to VIRTIO_CRYPTO_S_HW_READY.
\item The device MUST set \field{crypto_services} based on the crypto services the device offers.
\item The device MUST set detailed algorithms masks for each service advertised by \field{crypto_services}.
    The device MUST NOT set the not defined algorithms bits.
\item The device MUST set \field{max_size} to show the maximum size of crypto request the device supports.
\item The device MUST set \field{max_cipher_key_len} to show the maximum length of cipher key if the
    device supports CIPHER service.
\item The device MUST set \field{max_auth_key_len} to show the maximum length of authenticated key if
    the device supports MAC service.
\end{itemize*}

\drivernormative{\subsubsection}{Device configuration layout}{Device Types / Crypto Device / Device configuration layout}

\begin{itemize*}
\item The driver MUST read the \field{status} from the bottom bit of status to check whether the
    VIRTIO_CRYPTO_S_HW_READY is set, and the driver MUST reread it after device reset.
\item The driver MUST NOT transmit any requests to the device if the VIRTIO_CRYPTO_S_HW_READY is not set.
\item The driver MUST read \field{max_dataqueues} field to discover the number of data queues the device supports.
\item The driver MUST read \field{crypto_services} field to discover which services the device is able to offer.
\item The driver SHOULD ignore the not defined algorithms bits.
\item The driver MUST read the detailed algorithms fields based on \field{crypto_services} field.
\item The driver SHOULD read \field{max_size} to discover the maximum size of the variable-length
    parameters of data operation of the crypto request's content the device supports and MUST
    guarantee that the size of each crypto request's content is within the \field{max_size}, otherwise
    the request will fail and the driver MUST reset the device.
\item The driver SHOULD read \field{max_cipher_key_len} to discover the maximum length of cipher key
    the device supports and MUST guarantee that the \field{key_len} (CIPHER service or AEAD service) is within
    the \field{max_cipher_key_len} of the device configuration, otherwise the request will fail.
\item The driver SHOULD read \field{max_auth_key_len} to discover the maximum length of authenticated
    key the device supports and MUST guarantee that the \field{auth_key_len} (MAC service) is within the
    \field{max_auth_key_len} of the device configuration, otherwise the request will fail.
\end{itemize*}

\subsection{Device Initialization}\label{sec:Device Types / Crypto Device / Device Initialization}

\drivernormative{\subsubsection}{Device Initialization}{Device Types / Crypto Device / Device Initialization}

\begin{itemize*}
\item The driver MUST configure and initialize all virtqueues.
\item The driver MUST read the supported crypto services from bits of \field{crypto_services}.
\item The driver MUST read the supported algorithms based on \field{crypto_services} field.
\end{itemize*}

\subsection{Device Operation}\label{sec:Device Types / Crypto Device / Device Operation}

The operation of a virtio crypto device is driven by requests placed on the virtqueues.
Requests consist of a queue-type specific header (specifying among others the operation)
and an operation specific payload.

If VIRTIO_CRYPTO_F_REVISION_1 is negotiated the device may support both session mode
(See \ref{sec:Device Types / Crypto Device / Device Operation / Control Virtqueue / Session operation})
and stateless mode operation requests.
In stateless mode all operation parameters are supplied as a part of each request,
while in session mode, some or all operation parameters are managed within the
session. Stateless mode is guarded by feature bits 0-4 on a service level. If
stateless mode is negotiated for a service, the service accepts both session
mode and stateless requests; otherwise stateless mode requests are rejected
(via operation status).

\subsubsection{Operation Status}\label{sec:Device Types / Crypto Device / Device Operation / Operation status}
The device MUST return a status code as part of the operation (both session
operation and service operation) result. The valid operation status as follows:

\begin{lstlisting}
enum VIRTIO_CRYPTO_STATUS {
    VIRTIO_CRYPTO_OK = 0,
    VIRTIO_CRYPTO_ERR = 1,
    VIRTIO_CRYPTO_BADMSG = 2,
    VIRTIO_CRYPTO_NOTSUPP = 3,
    VIRTIO_CRYPTO_INVSESS = 4,
    VIRTIO_CRYPTO_NOSPC = 5,
    VIRTIO_CRYPTO_KEY_REJECTED = 6,
    VIRTIO_CRYPTO_MAX
};
\end{lstlisting}

\begin{itemize*}
\item VIRTIO_CRYPTO_OK: success.
\item VIRTIO_CRYPTO_BADMSG: authentication failed (only when AEAD decryption).
\item VIRTIO_CRYPTO_NOTSUPP: operation or algorithm is unsupported.
\item VIRTIO_CRYPTO_INVSESS: invalid session ID when executing crypto operations.
\item VIRTIO_CRYPTO_NOSPC: no free session ID (only when the VIRTIO_CRYPTO_F_REVISION_1
    feature bit is negotiated).
\item VIRTIO_CRYPTO_KEY_REJECTED: signature verification failed (only when AKCIPHER verification).
\item VIRTIO_CRYPTO_ERR: any failure not mentioned above occurs.
\end{itemize*}

\subsubsection{Control Virtqueue}\label{sec:Device Types / Crypto Device / Device Operation / Control Virtqueue}

The driver uses the control virtqueue to send control commands to the
device, such as session operations (See \ref{sec:Device Types / Crypto Device / Device
Operation / Control Virtqueue / Session operation}).

The header for controlq is of the following form:
\begin{lstlisting}
#define VIRTIO_CRYPTO_OPCODE(service, op)   (((service) << 8) | (op))

struct virtio_crypto_ctrl_header {
#define VIRTIO_CRYPTO_CIPHER_CREATE_SESSION \
       VIRTIO_CRYPTO_OPCODE(VIRTIO_CRYPTO_SERVICE_CIPHER, 0x02)
#define VIRTIO_CRYPTO_CIPHER_DESTROY_SESSION \
       VIRTIO_CRYPTO_OPCODE(VIRTIO_CRYPTO_SERVICE_CIPHER, 0x03)
#define VIRTIO_CRYPTO_HASH_CREATE_SESSION \
       VIRTIO_CRYPTO_OPCODE(VIRTIO_CRYPTO_SERVICE_HASH, 0x02)
#define VIRTIO_CRYPTO_HASH_DESTROY_SESSION \
       VIRTIO_CRYPTO_OPCODE(VIRTIO_CRYPTO_SERVICE_HASH, 0x03)
#define VIRTIO_CRYPTO_MAC_CREATE_SESSION \
       VIRTIO_CRYPTO_OPCODE(VIRTIO_CRYPTO_SERVICE_MAC, 0x02)
#define VIRTIO_CRYPTO_MAC_DESTROY_SESSION \
       VIRTIO_CRYPTO_OPCODE(VIRTIO_CRYPTO_SERVICE_MAC, 0x03)
#define VIRTIO_CRYPTO_AEAD_CREATE_SESSION \
       VIRTIO_CRYPTO_OPCODE(VIRTIO_CRYPTO_SERVICE_AEAD, 0x02)
#define VIRTIO_CRYPTO_AEAD_DESTROY_SESSION \
       VIRTIO_CRYPTO_OPCODE(VIRTIO_CRYPTO_SERVICE_AEAD, 0x03)
#define VIRTIO_CRYPTO_AKCIPHER_CREATE_SESSION \
       VIRTIO_CRYPTO_OPCODE(VIRTIO_CRYPTO_SERVICE_AKCIPHER, 0x04)
#define VIRTIO_CRYPTO_AKCIPHER_DESTROY_SESSION \
       VIRTIO_CRYPTO_OPCDE(VIRTIO_CRYPTO_SERVICE_AKCIPHER, 0x05)
    le32 opcode;
    /* algo should be service-specific algorithms */
    le32 algo;
    le32 flag;
    le32 reserved;
};
\end{lstlisting}

The controlq request is composed of four parts:
\begin{lstlisting}
struct virtio_crypto_op_ctrl_req {
    /* Device read only portion */

    struct virtio_crypto_ctrl_header header;

#define VIRTIO_CRYPTO_CTRLQ_OP_SPEC_HDR_LEGACY 56
    /* fixed length fields, opcode specific */
    u8 op_flf[flf_len];

    /* variable length fields, opcode specific */
    u8 op_vlf[vlf_len];

    /* Device write only portion */

    /* op result or completion status */
    u8 op_outcome[outcome_len];
};
\end{lstlisting}

\field{header} is a general header (see above).

\field{op_flf} is the opcode (in \field{header}) specific fixed-length parameters.

\field{flf_len} depends on the VIRTIO_CRYPTO_F_REVISION_1 feature bit (see below).

\field{op_vlf} is the opcode (in \field{header}) specific variable-length parameters.

\field{vlf_len} is the size of the specific structure used.
\begin{note}
The \field{vlf_len} of session-destroy operation and the hash-session-create
operation is ZERO.
\end{note}

\begin{itemize*}
\item If the opcode (in \field{header}) is VIRTIO_CRYPTO_CIPHER_CREATE_SESSION
    then \field{op_flf} is struct virtio_crypto_sym_create_session_flf if
    VIRTIO_CRYPTO_F_REVISION_1 is negotiated and struct virtio_crypto_sym_create_session_flf is
    padded to 56 bytes if NOT negotiated, and \field{op_vlf} is struct
    virtio_crypto_sym_create_session_vlf.
\item If the opcode (in \field{header}) is VIRTIO_CRYPTO_HASH_CREATE_SESSION
    then \field{op_flf} is struct virtio_crypto_hash_create_session_flf if
    VIRTIO_CRYPTO_F_REVISION_1 is negotiated and struct virtio_crypto_hash_create_session_flf is
    padded to 56 bytes if NOT negotiated.
\item If the opcode (in \field{header}) is VIRTIO_CRYPTO_MAC_CREATE_SESSION
    then \field{op_flf} is struct virtio_crypto_mac_create_session_flf if
    VIRTIO_CRYPTO_F_REVISION_1 is negotiated and struct virtio_crypto_mac_create_session_flf is
    padded to 56 bytes if NOT negotiated, and \field{op_vlf} is struct
    virtio_crypto_mac_create_session_vlf.
\item If the opcode (in \field{header}) is VIRTIO_CRYPTO_AEAD_CREATE_SESSION
    then \field{op_flf} is struct virtio_crypto_aead_create_session_flf if
    VIRTIO_CRYPTO_F_REVISION_1 is negotiated and struct virtio_crypto_aead_create_session_flf is
    padded to 56 bytes if NOT negotiated, and \field{op_vlf} is struct
    virtio_crypto_aead_create_session_vlf.
\item If the opcode (in \field{header}) is VIRTIO_CRYPTO_AKCIPHER_CREATE_SESSION
    then \field{op_flf} is struct virtio_crypto_akcipher_create_session_flf if
    VIRTIO_CRYPTO_F_REVISION_1 is negotiated and struct virtio_crypto_akcipher_create_session_flf is
    padded to 56 bytes if NOT negotiated, and \field{op_vlf} is struct
    virtio_crypto_akcipher_create_session_vlf.
\item If the opcode (in \field{header}) is VIRTIO_CRYPTO_CIPHER_DESTROY_SESSION
    or VIRTIO_CRYPTO_HASH_DESTROY_SESSION or VIRTIO_CRYPTO_MAC_DESTROY_SESSION or
    VIRTIO_CRYPTO_AEAD_DESTROY_SESSION then \field{op_flf} is struct
    virtio_crypto_destroy_session_flf if VIRTIO_CRYPTO_F_REVISION_1 is negotiated and
    struct virtio_crypto_destroy_session_flf is padded to 56 bytes if NOT negotiated.
\end{itemize*}

\field{op_outcome} stores the result of operation and must be struct
virtio_crypto_destroy_session_input for destroy session or
struct virtio_crypto_create_session_input for create session.

\field{outcome_len} is the size of the structure used.


\paragraph{Session operation}\label{sec:Device Types / Crypto Device / Device
Operation / Control Virtqueue / Session operation}

The session is a handle which describes the cryptographic parameters to be
applied to a number of buffers.

The following structure stores the result of session creation set by the device:

\begin{lstlisting}
struct virtio_crypto_create_session_input {
    le64 session_id;
    le32 status;
    le32 padding;
};
\end{lstlisting}

A request to destroy a session includes the following information:

\begin{lstlisting}
struct virtio_crypto_destroy_session_flf {
    /* Device read only portion */
    le64  session_id;
};

struct virtio_crypto_destroy_session_input {
    /* Device write only portion */
    u8  status;
};
\end{lstlisting}


\subparagraph{Session operation: HASH session}\label{sec:Device Types / Crypto Device / Device
Operation / Control Virtqueue / Session operation / Session operation: HASH session}

The fixed-length parameters of HASH session requests is as follows:

\begin{lstlisting}
struct virtio_crypto_hash_create_session_flf {
    /* Device read only portion */

    /* See VIRTIO_CRYPTO_HASH_* above */
    le32 algo;
    /* hash result length */
    le32 hash_result_len;
};
\end{lstlisting}


\subparagraph{Session operation: MAC session}\label{sec:Device Types / Crypto Device / Device
Operation / Control Virtqueue / Session operation / Session operation: MAC session}

The fixed-length and the variable-length parameters of MAC session requests are as follows:

\begin{lstlisting}
struct virtio_crypto_mac_create_session_flf {
    /* Device read only portion */

    /* See VIRTIO_CRYPTO_MAC_* above */
    le32 algo;
    /* hash result length */
    le32 hash_result_len;
    /* length of authenticated key */
    le32 auth_key_len;
    le32 padding;
};

struct virtio_crypto_mac_create_session_vlf {
    /* Device read only portion */

    /* The authenticated key */
    u8 auth_key[auth_key_len];
};
\end{lstlisting}

The length of \field{auth_key} is specified in \field{auth_key_len} in the struct
virtio_crypto_mac_create_session_flf.


\subparagraph{Session operation: Symmetric algorithms session}\label{sec:Device Types / Crypto Device / Device
Operation / Control Virtqueue / Session operation / Session operation: Symmetric algorithms session}

The request of symmetric session could be the CIPHER algorithms request
or the chain algorithms (chaining CIPHER and HASH/MAC) request.

The fixed-length and the variable-length parameters of CIPHER session requests are as follows:

\begin{lstlisting}
struct virtio_crypto_cipher_session_flf {
    /* Device read only portion */

    /* See VIRTIO_CRYPTO_CIPHER* above */
    le32 algo;
    /* length of key */
    le32 key_len;
#define VIRTIO_CRYPTO_OP_ENCRYPT  1
#define VIRTIO_CRYPTO_OP_DECRYPT  2
    /* encryption or decryption */
    le32 op;
    le32 padding;
};

struct virtio_crypto_cipher_session_vlf {
    /* Device read only portion */

    /* The cipher key */
    u8 cipher_key[key_len];
};
\end{lstlisting}

The length of \field{cipher_key} is specified in \field{key_len} in the struct
virtio_crypto_cipher_session_flf.

The fixed-length and the variable-length parameters of Chain session requests are as follows:

\begin{lstlisting}
struct virtio_crypto_alg_chain_session_flf {
    /* Device read only portion */

#define VIRTIO_CRYPTO_SYM_ALG_CHAIN_ORDER_HASH_THEN_CIPHER  1
#define VIRTIO_CRYPTO_SYM_ALG_CHAIN_ORDER_CIPHER_THEN_HASH  2
    le32 alg_chain_order;
/* Plain hash */
#define VIRTIO_CRYPTO_SYM_HASH_MODE_PLAIN    1
/* Authenticated hash (mac) */
#define VIRTIO_CRYPTO_SYM_HASH_MODE_AUTH     2
/* Nested hash */
#define VIRTIO_CRYPTO_SYM_HASH_MODE_NESTED   3
    le32 hash_mode;
    struct virtio_crypto_cipher_session_flf cipher_hdr;

#define VIRTIO_CRYPTO_ALG_CHAIN_SESS_OP_SPEC_HDR_SIZE  16
    /* fixed length fields, algo specific */
    u8 algo_flf[VIRTIO_CRYPTO_ALG_CHAIN_SESS_OP_SPEC_HDR_SIZE];

    /* length of the additional authenticated data (AAD) in bytes */
    le32 aad_len;
    le32 padding;
};

struct virtio_crypto_alg_chain_session_vlf {
    /* Device read only portion */

    /* The cipher key */
    u8 cipher_key[key_len];
    /* The authenticated key */
    u8 auth_key[auth_key_len];
};
\end{lstlisting}

\field{hash_mode} decides the type used by \field{algo_flf}.

\field{algo_flf} is fixed to 16 bytes and MUST contains or be one of
the following types:
\begin{itemize*}
\item struct virtio_crypto_hash_create_session_flf
\item struct virtio_crypto_mac_create_session_flf
\end{itemize*}
The data of unused part (if has) in \field{algo_flf} will be ignored.

The length of \field{cipher_key} is specified in \field{key_len} in \field{cipher_hdr}.

The length of \field{auth_key} is specified in \field{auth_key_len} in struct
virtio_crypto_mac_create_session_flf.

The fixed-length parameters of Symmetric session requests are as follows:

\begin{lstlisting}
struct virtio_crypto_sym_create_session_flf {
    /* Device read only portion */

#define VIRTIO_CRYPTO_SYM_SESS_OP_SPEC_HDR_SIZE  48
    /* fixed length fields, opcode specific */
    u8 op_flf[VIRTIO_CRYPTO_SYM_SESS_OP_SPEC_HDR_SIZE];

/* No operation */
#define VIRTIO_CRYPTO_SYM_OP_NONE  0
/* Cipher only operation on the data */
#define VIRTIO_CRYPTO_SYM_OP_CIPHER  1
/* Chain any cipher with any hash or mac operation. The order
   depends on the value of alg_chain_order param */
#define VIRTIO_CRYPTO_SYM_OP_ALGORITHM_CHAINING  2
    le32 op_type;
    le32 padding;
};
\end{lstlisting}

\field{op_flf} is fixed to 48 bytes, MUST contains or be one of
the following types:
\begin{itemize*}
\item struct virtio_crypto_cipher_session_flf
\item struct virtio_crypto_alg_chain_session_flf
\end{itemize*}
The data of unused part (if has) in \field{op_flf} will be ignored.

\field{op_type} decides the type used by \field{op_flf}.

The variable-length parameters of Symmetric session requests are as follows:

\begin{lstlisting}
struct virtio_crypto_sym_create_session_vlf {
    /* Device read only portion */
    /* variable length fields, opcode specific */
    u8 op_vlf[vlf_len];
};
\end{lstlisting}

\field{op_vlf} MUST contains or be one of the following types:
\begin{itemize*}
\item struct virtio_crypto_cipher_session_vlf
\item struct virtio_crypto_alg_chain_session_vlf
\end{itemize*}

\field{op_type} in struct virtio_crypto_sym_create_session_flf decides the
type used by \field{op_vlf}.

\field{vlf_len} is the size of the specific structure used.


\subparagraph{Session operation: AEAD session}\label{sec:Device Types / Crypto Device / Device
Operation / Control Virtqueue / Session operation / Session operation: AEAD session}

The fixed-length and the variable-length parameters of AEAD session requests are as follows:

\begin{lstlisting}
struct virtio_crypto_aead_create_session_flf {
    /* Device read only portion */

    /* See VIRTIO_CRYPTO_AEAD_* above */
    le32 algo;
    /* length of key */
    le32 key_len;
    /* Authentication tag length */
    le32 tag_len;
    /* The length of the additional authenticated data (AAD) in bytes */
    le32 aad_len;
    /* encryption or decryption, See above VIRTIO_CRYPTO_OP_* */
    le32 op;
    le32 padding;
};

struct virtio_crypto_aead_create_session_vlf {
    /* Device read only portion */
    u8 key[key_len];
};
\end{lstlisting}

The length of \field{key} is specified in \field{key_len} in struct
virtio_crypto_aead_create_session_flf.

\subparagraph{Session operation: AKCIPHER session}\label{sec:Device Types / Crypto Device / Device
Operation / Control Virtqueue / Session operation / Session operation: AKCIPHER session}

Due to the complexity of asymmetric key algorithms, different algorithms
require different parameters. The following data structures are used as
supplementary parameters to describe the asymmetric algorithm sessions.

For the RSA algorithm, the extra parameters are as follows:
\begin{lstlisting}
struct virtio_crypto_rsa_session_para {
#define VIRTIO_CRYPTO_RSA_RAW_PADDING   0
#define VIRTIO_CRYPTO_RSA_PKCS1_PADDING 1
    le32 padding_algo;

#define VIRTIO_CRYPTO_RSA_NO_HASH   0
#define VIRTIO_CRYPTO_RSA_MD2       1
#define VIRTIO_CRYPTO_RSA_MD3       2
#define VIRTIO_CRYPTO_RSA_MD4       3
#define VIRTIO_CRYPTO_RSA_MD5       4
#define VIRTIO_CRYPTO_RSA_SHA1      5
#define VIRTIO_CRYPTO_RSA_SHA256    6
#define VIRTIO_CRYPTO_RSA_SHA384    7
#define VIRTIO_CRYPTO_RSA_SHA512    8
#define VIRTIO_CRYPTO_RSA_SHA224    9
    le32 hash_algo;
};
\end{lstlisting}

\field{padding_algo} specifies the padding method used by RSA sessions.
\begin{itemize*}
\item If VIRTIO_CRYPTO_RSA_RAW_PADDING is specified, 1) \field{hash_algo}
is ignored, 2) ciphertext and plaintext MUST be padded with leading zeros,
3) and RSA sessions with VIRTIO_CRYPTO_RSA_RAW_PADDING MUST not be used
for verification and signing operations.
\item If VIRTIO_CRYPTO_RSA_PKCS1_PADDING is specified, EMSA-PKCS1-v1_5 padding method
is used (see \hyperref[intro:rfc3447]{PKCS\#1}), \field{hash_algo} specifies how the
digest of the data passed to RSA sessions is calculated when verifying and signing.
It only affects the padding algorithm and is ignored during encryption and decryption.
\end{itemize*}

The ECC algorithms such as the ECDSA algorithm, cannot use custom curves, only the
following known curves can be used (see \hyperref[intro:NIST]{NIST-recommended curves}).

\begin{lstlisting}
#define VIRTIO_CRYPTO_CURVE_UNKNOWN   0
#define VIRTIO_CRYPTO_CURVE_NIST_P192 1
#define VIRTIO_CRYPTO_CURVE_NIST_P224 2
#define VIRTIO_CRYPTO_CURVE_NIST_P256 3
#define VIRTIO_CRYPTO_CURVE_NIST_P384 4
#define VIRTIO_CRYPTO_CURVE_NIST_P521 5
\end{lstlisting}

For the ECDSA algorithm, the extra parameters are as follows:
\begin{lstlisting}
struct virtio_crypto_ecdsa_session_para {
    /* See VIRTIO_CRYPTO_CURVE_* above */
    le32 curve_id;
};
\end{lstlisting}

The fixed-length and the variable-length parameters of AKCIPHER session requests are as follows:
\begin{lstlisting}
struct virtio_crypto_akcipher_create_session_flf {
    /* Device read only portion */

    /* See VIRTIO_CRYPTO_AKCIPHER_* above */
    le32 algo;
#define VIRTIO_CRYPTO_AKCIPHER_KEY_TYPE_PUBLIC 1
#define VIRTIO_CRYPTO_AKCIPHER_KEY_TYPE_PRIVATE 2
    le32 key_type;
    /* length of key */
    le32 key_len;

#define VIRTIO_CRYPTO_AKCIPHER_SESS_ALGO_SPEC_HDR_SIZE 44
    u8 algo_flf[VIRTIO_CRYPTO_AKCIPHER_SESS_ALGO_SPEC_HDR_SIZE];
};

struct virtio_crypto_akcipher_create_session_vlf {
    /* Device read only portion */
    u8 key[key_len];
};
\end{lstlisting}

\field{algo} decides the type used by \field{algo_flf}.
\field{algo_flf} is fixed to 44 bytes and MUST contains of be one the
following structures:
\begin{itemize*}
\item struct virtio_crypto_rsa_session_para
\item struct virtio_crypto_ecdsa_session_para
\end{itemize*}

The length of \field{key} is specified in \field{key_len} in the struct
virtio_crypto_akcipher_create_session_flf.

For the RSA algorithm, the key needs to be encoded according to
\hyperref[intro:rfc3447]{PKCS\#1}. The private key is described with the
RSAPrivateKey structure, and the public key is described with the RSAPublicKey
structure. These ASN.1 structures are encoded in DER encoding rules (see
\hyperref[intro:rfc6025]{rfc6025}).

\begin{lstlisting}
RSAPrivateKey ::= SEQUENCE {
    version          INTEGER,
    modulus          INTEGER,
    publicExponent   INTEGER,
    privateExponent  INTEGER,
    prime1           INTEGER,
    prime2           INTEGER,
    exponent1        INTEGER,
    exponent1        INTEGER,
    coefficient      INTEGER,
    otherPrimeInfos  OtherPrimeInfos OPTIONAL
}

OtherPrimeInfos ::= SEQUENCE SIZE(1...MAX) OF OtherPrimeInfo

OtherPrimeINfo ::= SEQUENCE {
    prime           INTEGER,
    exponent        INTEGER,
    coefficient     INTEGER
}

RSAPublicKey ::= SEQUENCE {
    modulus         INTEGER,
    publicExponent  INTEGER
}
\end{lstlisting}

For the ECDSA algorithm, the private key is encoded according to
\hyperref[intro:rfc5915]{RFC5915}, the private key of the ECDSA algorithm
is described by the ASN.1 structure ECPrivateKey and encoded with DER
encoding rules (see \hyperref[intro:rfc6025]{rfc6025}).

\begin{lstlisting}
ECPrivateKey ::= SEQUNCE {
    version         INTEGER,
    privateKey      OCTET STRING,
    parameters [0]  ECParameters {{ NamedCurve }} OPTIONAL,
    publicKey  [1]  BIT STRING OPTIONAL
}
\end{lstlisting}

The public key of the ECDSA algorithm is encoded according to \hyperref[intro:SEC1]{SEC1},
and the public key of ECDSA is described by the ASN.1 structure ECPoint.
When initializing a session with ECDSA public key, the ECPoint is DER encoded and the
\field{key} only contains the value part of ECPoint, that is, the header part of the
OCTET STRING will be omitted (see \hyperref[intro:rfc6025]{rfc6025}).

\begin{lstlisting}
ECPoint ::= OCTET STRING
\end{lstlisting}

The length of \field{key} is specified in \field{key_len} in
struct virtio_crypto_akcipher_create_session_flf.

\drivernormative{\subparagraph}{Session operation: create session}{Device Types / Crypto Device / Device
Operation / Control Virtqueue / Session operation / Session operation: create session}

\begin{itemize*}
\item The driver MUST set the \field{opcode} field based on service type: CIPHER, HASH, MAC, AEAD or AKCIPHER.
\item The driver MUST set the control general header, the opcode specific header,
    the opcode specific extra parameters and the opcode specific outcome buffer in turn.
    See \ref{sec:Device Types / Crypto Device / Device Operation / Control Virtqueue}.
\item The driver MUST set the \field{reversed} field to zero.
\end{itemize*}

\devicenormative{\subparagraph}{Session operation: create session}{Device Types / Crypto Device / Device
Operation / Control Virtqueue / Session operation / Session operation: create session}

\begin{itemize*}
\item The device MUST use the corresponding opcode specific structure according to the
    \field{opcode} in the control general header.
\item The device MUST extract extra parameters according to the structures used.
\item The device MUST set the \field{status} field to one of the following values of enum
    VIRTIO_CRYPTO_STATUS after finish a session creation:
\begin{itemize*}
\item VIRTIO_CRYPTO_OK if a session is created successfully.
\item VIRTIO_CRYPTO_NOTSUPP if the requested algorithm or operation is unsupported.
\item VIRTIO_CRYPTO_NOSPC if no free session ID (only when the VIRTIO_CRYPTO_F_REVISION_1
    feature bit is negotiated).
\item VIRTIO_CRYPTO_ERR if failure not mentioned above occurs.
\end{itemize*}
\item The device MUST set the \field{session_id} field to a unique session identifier only
    if the status is set to VIRTIO_CRYPTO_OK.
\end{itemize*}

\drivernormative{\subparagraph}{Session operation: destroy session}{Device Types / Crypto Device / Device
Operation / Control Virtqueue / Session operation / Session operation: destroy session}

\begin{itemize*}
\item The driver MUST set the \field{opcode} field based on service type: CIPHER, HASH, MAC, AEAD or AKCIPHER.
\item The driver MUST set the \field{session_id} to a valid value assigned by the device
    when the session was created.
\end{itemize*}

\devicenormative{\subparagraph}{Session operation: destroy session}{Device Types / Crypto Device / Device
Operation / Control Virtqueue / Session operation / Session operation: destroy session}

\begin{itemize*}
\item The device MUST set the \field{status} field to one of the following values of enum VIRTIO_CRYPTO_STATUS.
\begin{itemize*}
\item VIRTIO_CRYPTO_OK if a session is created successfully.
\item VIRTIO_CRYPTO_ERR if any failure occurs.
\end{itemize*}
\end{itemize*}


\subsubsection{Data Virtqueue}\label{sec:Device Types / Crypto Device / Device Operation / Data Virtqueue}

The driver uses the data virtqueues to transmit crypto operation requests to the device,
and completes the crypto operations.

The header for dataq is as follows:

\begin{lstlisting}
struct virtio_crypto_op_header {
#define VIRTIO_CRYPTO_CIPHER_ENCRYPT \
    VIRTIO_CRYPTO_OPCODE(VIRTIO_CRYPTO_SERVICE_CIPHER, 0x00)
#define VIRTIO_CRYPTO_CIPHER_DECRYPT \
    VIRTIO_CRYPTO_OPCODE(VIRTIO_CRYPTO_SERVICE_CIPHER, 0x01)
#define VIRTIO_CRYPTO_HASH \
    VIRTIO_CRYPTO_OPCODE(VIRTIO_CRYPTO_SERVICE_HASH, 0x00)
#define VIRTIO_CRYPTO_MAC \
    VIRTIO_CRYPTO_OPCODE(VIRTIO_CRYPTO_SERVICE_MAC, 0x00)
#define VIRTIO_CRYPTO_AEAD_ENCRYPT \
    VIRTIO_CRYPTO_OPCODE(VIRTIO_CRYPTO_SERVICE_AEAD, 0x00)
#define VIRTIO_CRYPTO_AEAD_DECRYPT \
    VIRTIO_CRYPTO_OPCODE(VIRTIO_CRYPTO_SERVICE_AEAD, 0x01)
#define VIRTIO_CRYPTO_AKCIPHER_ENCRYPT \
    VIRTIO_CRYPTO_OPCODE(VIRTIO_CRYPTO_SERVICE_AKCIPHER, 0x00)
#define VIRTIO_CRYPTO_AKCIPHER_DECRYPT \
    VIRTIO_CRYPTO_OPCODE(VIRTIO_CRYPTO_SERVICE_AKCIPHER, 0x01)
#define VIRTIO_CRYPTO_AKCIPHER_SIGN \
    VIRTIO_CRYPTO_OPCODE(VIRTIO_CRYPTO_SERVICE_AKCIPHER, 0x02)
#define VIRTIO_CRYPTO_AKCIPHER_VERIFY \
    VIRTIO_CRYPTO_OPCODE(VIRTIO_CRYPTO_SERVICE_AKCIPHER, 0x03)
    le32 opcode;
    /* algo should be service-specific algorithms */
    le32 algo;
    le64 session_id;
#define VIRTIO_CRYPTO_FLAG_SESSION_MODE 1
    /* control flag to control the request */
    le32 flag;
    le32 padding;
};
\end{lstlisting}

\begin{note}
If VIRTIO_CRYPTO_F_REVISION_1 is not negotiated the \field{flag} is ignored.

If VIRTIO_CRYPTO_F_REVISION_1 is negotiated but VIRTIO_CRYPTO_F_<SERVICE>_STATELESS_MODE
is not negotiated, then the device SHOULD reject <SERVICE> requests if
VIRTIO_CRYPTO_FLAG_SESSION_MODE is not set (in \field{flag}).
\end{note}

The dataq request is composed of four parts:
\begin{lstlisting}
struct virtio_crypto_op_data_req {
    /* Device read only portion */

    struct virtio_crypto_op_header header;

#define VIRTIO_CRYPTO_DATAQ_OP_SPEC_HDR_LEGACY 48
    /* fixed length fields, opcode specific */
    u8 op_flf[flf_len];

    /* Device read && write portion */
    /* variable length fields, opcode specific */
    u8 op_vlf[vlf_len];

    /* Device write only portion */
    struct virtio_crypto_inhdr inhdr;
};
\end{lstlisting}

\field{header} is a general header (see above).

\field{op_flf} is the opcode (in \field{header}) specific header.

\field{flf_len} depends on the VIRTIO_CRYPTO_F_REVISION_1 feature bit
(see below).

\field{op_vlf} is the opcode (in \field{header}) specific parameters.

\field{vlf_len} is the size of the specific structure used.

\begin{itemize*}
\item If the the opcode (in \field{header}) is VIRTIO_CRYPTO_CIPHER_ENCRYPT
    or VIRTIO_CRYPTO_CIPHER_DECRYPT then:
    \begin{itemize*}
    \item If VIRTIO_CRYPTO_F_CIPHER_STATELESS_MODE is negotiated, \field{op_flf} is
        struct virtio_crypto_sym_data_flf_stateless, and \field{op_vlf} is struct
        virtio_crypto_sym_data_vlf_stateless.
    \item If VIRTIO_CRYPTO_F_CIPHER_STATELESS_MODE is NOT negotiated, \field{op_flf}
        is struct virtio_crypto_sym_data_flf if VIRTIO_CRYPTO_F_REVISION_1 is negotiated
        and struct virtio_crypto_sym_data_flf is padded to 48 bytes if NOT negotiated,
        and \field{op_vlf} is struct virtio_crypto_sym_data_vlf.
    \end{itemize*}
\item If the the opcode (in \field{header}) is VIRTIO_CRYPTO_HASH:
    \begin{itemize*}
    \item If VIRTIO_CRYPTO_F_HASH_STATELESS_MODE is negotiated, \field{op_flf} is
        struct virtio_crypto_hash_data_flf_stateless, and \field{op_vlf} is struct
        virtio_crypto_hash_data_vlf_stateless.
    \item If VIRTIO_CRYPTO_F_HASH_STATELESS_MODE is NOT negotiated, \field{op_flf}
        is struct virtio_crypto_hash_data_flf if VIRTIO_CRYPTO_F_REVISION_1 is negotiated
        and struct virtio_crypto_hash_data_flf is padded to 48 bytes if NOT negotiated,
        and \field{op_vlf} is struct virtio_crypto_hash_data_vlf.
    \end{itemize*}
\item If the the opcode (in \field{header}) is VIRTIO_CRYPTO_MAC:
    \begin{itemize*}
    \item If VIRTIO_CRYPTO_F_MAC_STATELESS_MODE is negotiated, \field{op_flf} is
        struct virtio_crypto_mac_data_flf_stateless, and \field{op_vlf} is struct
        virtio_crypto_mac_data_vlf_stateless.
    \item If VIRTIO_CRYPTO_F_MAC_STATELESS_MODE is NOT negotiated, \field{op_flf}
        is struct virtio_crypto_mac_data_flf if VIRTIO_CRYPTO_F_REVISION_1 is negotiated
        and struct virtio_crypto_mac_data_flf is padded to 48 bytes if NOT negotiated,
        and \field{op_vlf} is struct virtio_crypto_mac_data_vlf.
    \end{itemize*}
\item If the the opcode (in \field{header}) is VIRTIO_CRYPTO_AEAD_ENCRYPT
    or VIRTIO_CRYPTO_AEAD_DECRYPT then:
    \begin{itemize*}
    \item If VIRTIO_CRYPTO_F_AEAD_STATELESS_MODE is negotiated, \field{op_flf} is
        struct virtio_crypto_aead_data_flf_stateless, and \field{op_vlf} is struct
        virtio_crypto_aead_data_vlf_stateless.
    \item If VIRTIO_CRYPTO_F_AEAD_STATELESS_MODE is NOT negotiated, \field{op_flf}
        is struct virtio_crypto_aead_data_flf if VIRTIO_CRYPTO_F_REVISION_1 is negotiated
        and struct virtio_crypto_aead_data_flf is padded to 48 bytes if NOT negotiated,
        and \field{op_vlf} is struct virtio_crypto_aead_data_vlf.
    \end{itemize*}
\item If the opcode (in \field{header}) is VIRTIO_CRYPTO_AKCIPHER_ENCRYPT, VIRTIO_CRYPTO_AKCIPHER_DECRYPT,
    VIRTIO_CRYPTO_AKCIPHER_SIGN or VIRTIO_CRYPTO_AKCIPHER_VERIFY then:
    \begin{itemize*}
    \item If VIRTIO_CRYPTO_F_AKCIPHER_STATELESS_MODE is negotiated, \field{op_flf} is
        struct virtio_crypto_akcipher_data_flf_statless, and \field{op_vlf} is struct
        virtio_crypto_akcipher_data_vlf_stateless.
    \item If VIRTIO_CRYPTO_F_AKCIPHER_STATELESS_MODE is NOT negotiated, \field{op_flf}
        is struct virtio_crypto_akcipher_data_flf if VIRTIO_CRYPTO_F_REVISION_1 is negotiated
        and struct virtio_crypto_akcipher_data_flf is padded to 48 bytes if NOT negotiated,
        and \field{op_vlf} is struct virtio_crypto_akcipher_data_vlf.
    \end{itemize*}
\end{itemize*}

\field{inhdr} is a unified input header that used to return the status of
the operations, is defined as follows:

\begin{lstlisting}
struct virtio_crypto_inhdr {
    u8 status;
};
\end{lstlisting}

\subsubsection{HASH Service Operation}\label{sec:Device Types / Crypto Device / Device Operation / HASH Service Operation}

Session mode HASH service requests are as follows:

\begin{lstlisting}
struct virtio_crypto_hash_data_flf {
    /* length of source data */
    le32 src_data_len;
    /* hash result length */
    le32 hash_result_len;
};

struct virtio_crypto_hash_data_vlf {
    /* Device read only portion */
    /* Source data */
    u8 src_data[src_data_len];

    /* Device write only portion */
    /* Hash result data */
    u8 hash_result[hash_result_len];
};
\end{lstlisting}

Each data request uses the virtio_crypto_hash_data_flf structure and the
virtio_crypto_hash_data_vlf structure to store information used to run the
HASH operations.

\field{src_data} is the source data that will be processed.
\field{src_data_len} is the length of source data.
\field{hash_result} is the result data and \field{hash_result_len} is the length
of it.

Stateless mode HASH service requests are as follows:

\begin{lstlisting}
struct virtio_crypto_hash_data_flf_stateless {
    struct {
        /* See VIRTIO_CRYPTO_HASH_* above */
        le32 algo;
    } sess_para;

    /* length of source data */
    le32 src_data_len;
    /* hash result length */
    le32 hash_result_len;
    le32 reserved;
};
struct virtio_crypto_hash_data_vlf_stateless {
    /* Device read only portion */
    /* Source data */
    u8 src_data[src_data_len];

    /* Device write only portion */
    /* Hash result data */
    u8 hash_result[hash_result_len];
};
\end{lstlisting}

\drivernormative{\paragraph}{HASH Service Operation}{Device Types / Crypto Device / Device Operation / HASH Service Operation}

\begin{itemize*}
\item If the driver uses the session mode, then the driver MUST set \field{session_id}
    in struct virtio_crypto_op_header to a valid value assigned by the device when the
    session was created.
\item If the VIRTIO_CRYPTO_F_HASH_STATELESS_MODE feature bit is negotiated, 1) if the
    driver uses the stateless mode, then the driver MUST set the \field{flag} field in
    struct virtio_crypto_op_header to ZERO and MUST set the fields in struct
    virtio_crypto_hash_data_flf_stateless.sess_para, 2) if the driver uses the session
    mode, then the driver MUST set the \field{flag} field in struct virtio_crypto_op_header
    to VIRTIO_CRYPTO_FLAG_SESSION_MODE.
\item The driver MUST set \field{opcode} in struct virtio_crypto_op_header to VIRTIO_CRYPTO_HASH.
\end{itemize*}

\devicenormative{\paragraph}{HASH Service Operation}{Device Types / Crypto Device / Device Operation / HASH Service Operation}

\begin{itemize*}
\item The device MUST use the corresponding structure according to the \field{opcode}
    in the data general header.
\item If the VIRTIO_CRYPTO_F_HASH_STATELESS_MODE feature bit is negotiated, the device
    MUST parse \field{flag} field in struct virtio_crypto_op_header in order to decide
    which mode the driver uses.
\item The device MUST copy the results of HASH operations in the hash_result[] if HASH
    operations success.
\item The device MUST set \field{status} in struct virtio_crypto_inhdr to one of the
    following values of enum VIRTIO_CRYPTO_STATUS:
\begin{itemize*}
\item VIRTIO_CRYPTO_OK if the operation success.
\item VIRTIO_CRYPTO_NOTSUPP if the requested algorithm or operation is unsupported.
\item VIRTIO_CRYPTO_INVSESS if the session ID invalid when in session mode.
\item VIRTIO_CRYPTO_ERR if any failure not mentioned above occurs.
\end{itemize*}
\end{itemize*}


\subsubsection{MAC Service Operation}\label{sec:Device Types / Crypto Device / Device Operation / MAC Service Operation}

Session mode MAC service requests are as follows:

\begin{lstlisting}
struct virtio_crypto_mac_data_flf {
    struct virtio_crypto_hash_data_flf hdr;
};

struct virtio_crypto_mac_data_vlf {
    /* Device read only portion */
    /* Source data */
    u8 src_data[src_data_len];

    /* Device write only portion */
    /* Hash result data */
    u8 hash_result[hash_result_len];
};
\end{lstlisting}

Each request uses the virtio_crypto_mac_data_flf structure and the
virtio_crypto_mac_data_vlf structure to store information used to run the
MAC operations.

\field{src_data} is the source data that will be processed.
\field{src_data_len} is the length of source data.
\field{hash_result} is the result data and \field{hash_result_len} is the length
of it.

Stateless mode MAC service requests are as follows:

\begin{lstlisting}
struct virtio_crypto_mac_data_flf_stateless {
    struct {
        /* See VIRTIO_CRYPTO_MAC_* above */
        le32 algo;
        /* length of authenticated key */
        le32 auth_key_len;
    } sess_para;

    /* length of source data */
    le32 src_data_len;
    /* hash result length */
    le32 hash_result_len;
};

struct virtio_crypto_mac_data_vlf_stateless {
    /* Device read only portion */
    /* The authenticated key */
    u8 auth_key[auth_key_len];
    /* Source data */
    u8 src_data[src_data_len];

    /* Device write only portion */
    /* Hash result data */
    u8 hash_result[hash_result_len];
};
\end{lstlisting}

\field{auth_key} is the authenticated key that will be used during the process.
\field{auth_key_len} is the length of the key.

\drivernormative{\paragraph}{MAC Service Operation}{Device Types / Crypto Device / Device Operation / MAC Service Operation}

\begin{itemize*}
\item If the driver uses the session mode, then the driver MUST set \field{session_id}
    in struct virtio_crypto_op_header to a valid value assigned by the device when the
    session was created.
\item If the VIRTIO_CRYPTO_F_MAC_STATELESS_MODE feature bit is negotiated, 1) if the
    driver uses the stateless mode, then the driver MUST set the \field{flag} field
    in struct virtio_crypto_op_header to ZERO and MUST set the fields in struct
    virtio_crypto_mac_data_flf_stateless.sess_para, 2) if the driver uses the session
    mode, then the driver MUST set the \field{flag} field in struct virtio_crypto_op_header
    to VIRTIO_CRYPTO_FLAG_SESSION_MODE.
\item The driver MUST set \field{opcode} in struct virtio_crypto_op_header to VIRTIO_CRYPTO_MAC.
\end{itemize*}

\devicenormative{\paragraph}{MAC Service Operation}{Device Types / Crypto Device / Device Operation / MAC Service Operation}

\begin{itemize*}
\item If the VIRTIO_CRYPTO_F_MAC_STATELESS_MODE feature bit is negotiated, the device
    MUST parse \field{flag} field in struct virtio_crypto_op_header in order to decide
	which mode the driver uses.
\item The device MUST copy the results of MAC operations in the hash_result[] if HASH
    operations success.
\item The device MUST set \field{status} in struct virtio_crypto_inhdr to one of the
    following values of enum VIRTIO_CRYPTO_STATUS:
\begin{itemize*}
\item VIRTIO_CRYPTO_OK if the operation success.
\item VIRTIO_CRYPTO_NOTSUPP if the requested algorithm or operation is unsupported.
\item VIRTIO_CRYPTO_INVSESS if the session ID invalid when in session mode.
\item VIRTIO_CRYPTO_ERR if any failure not mentioned above occurs.
\end{itemize*}
\end{itemize*}

\subsubsection{Symmetric algorithms Operation}\label{sec:Device Types / Crypto Device / Device Operation / Symmetric algorithms Operation}

Session mode CIPHER service requests are as follows:

\begin{lstlisting}
struct virtio_crypto_cipher_data_flf {
    /*
     * Byte Length of valid IV/Counter data pointed to by the below iv data.
     *
     * For block ciphers in CBC or F8 mode, or for Kasumi in F8 mode, or for
     *   SNOW3G in UEA2 mode, this is the length of the IV (which
     *   must be the same as the block length of the cipher).
     * For block ciphers in CTR mode, this is the length of the counter
     *   (which must be the same as the block length of the cipher).
     */
    le32 iv_len;
    /* length of source data */
    le32 src_data_len;
    /* length of destination data */
    le32 dst_data_len;
    le32 padding;
};

struct virtio_crypto_cipher_data_vlf {
    /* Device read only portion */

    /*
     * Initialization Vector or Counter data.
     *
     * For block ciphers in CBC or F8 mode, or for Kasumi in F8 mode, or for
     *   SNOW3G in UEA2 mode, this is the Initialization Vector (IV)
     *   value.
     * For block ciphers in CTR mode, this is the counter.
     * For AES-XTS, this is the 128bit tweak, i, from IEEE Std 1619-2007.
     *
     * The IV/Counter will be updated after every partial cryptographic
     * operation.
     */
    u8 iv[iv_len];
    /* Source data */
    u8 src_data[src_data_len];

    /* Device write only portion */
    /* Destination data */
    u8 dst_data[dst_data_len];
};
\end{lstlisting}

Session mode requests of algorithm chaining are as follows:

\begin{lstlisting}
struct virtio_crypto_alg_chain_data_flf {
    le32 iv_len;
    /* Length of source data */
    le32 src_data_len;
    /* Length of destination data */
    le32 dst_data_len;
    /* Starting point for cipher processing in source data */
    le32 cipher_start_src_offset;
    /* Length of the source data that the cipher will be computed on */
    le32 len_to_cipher;
    /* Starting point for hash processing in source data */
    le32 hash_start_src_offset;
    /* Length of the source data that the hash will be computed on */
    le32 len_to_hash;
    /* Length of the additional auth data */
    le32 aad_len;
    /* Length of the hash result */
    le32 hash_result_len;
    le32 reserved;
};

struct virtio_crypto_alg_chain_data_vlf {
    /* Device read only portion */

    /* Initialization Vector or Counter data */
    u8 iv[iv_len];
    /* Source data */
    u8 src_data[src_data_len];
    /* Additional authenticated data if exists */
    u8 aad[aad_len];

    /* Device write only portion */

    /* Destination data */
    u8 dst_data[dst_data_len];
    /* Hash result data */
    u8 hash_result[hash_result_len];
};
\end{lstlisting}

Session mode requests of symmetric algorithm are as follows:

\begin{lstlisting}
struct virtio_crypto_sym_data_flf {
    /* Device read only portion */

#define VIRTIO_CRYPTO_SYM_DATA_REQ_HDR_SIZE    40
    u8 op_type_flf[VIRTIO_CRYPTO_SYM_DATA_REQ_HDR_SIZE];

    /* See above VIRTIO_CRYPTO_SYM_OP_* */
    le32 op_type;
    le32 padding;
};

struct virtio_crypto_sym_data_vlf {
    u8 op_type_vlf[sym_para_len];
};
\end{lstlisting}

Each request uses the virtio_crypto_sym_data_flf structure and the
virtio_crypto_sym_data_flf structure to store information used to run the
CIPHER operations.

\field{op_type_flf} is the \field{op_type} specific header, it MUST starts
with or be one of the following structures:
\begin{itemize*}
\item struct virtio_crypto_cipher_data_flf
\item struct virtio_crypto_alg_chain_data_flf
\end{itemize*}

The length of \field{op_type_flf} is fixed to 40 bytes, the data of unused
part (if has) will be ignored.

\field{op_type_vlf} is the \field{op_type} specific parameters, it MUST starts
with or be one of the following structures:
\begin{itemize*}
\item struct virtio_crypto_cipher_data_vlf
\item struct virtio_crypto_alg_chain_data_vlf
\end{itemize*}

\field{sym_para_len} is the size of the specific structure used.

Stateless mode CIPHER service requests are as follows:

\begin{lstlisting}
struct virtio_crypto_cipher_data_flf_stateless {
    struct {
        /* See VIRTIO_CRYPTO_CIPHER* above */
        le32 algo;
        /* length of key */
        le32 key_len;

        /* See VIRTIO_CRYPTO_OP_* above */
        le32 op;
    } sess_para;

    /*
     * Byte Length of valid IV/Counter data pointed to by the below iv data.
     */
    le32 iv_len;
    /* length of source data */
    le32 src_data_len;
    /* length of destination data */
    le32 dst_data_len;
};

struct virtio_crypto_cipher_data_vlf_stateless {
    /* Device read only portion */

    /* The cipher key */
    u8 cipher_key[key_len];

    /* Initialization Vector or Counter data. */
    u8 iv[iv_len];
    /* Source data */
    u8 src_data[src_data_len];

    /* Device write only portion */
    /* Destination data */
    u8 dst_data[dst_data_len];
};
\end{lstlisting}

Stateless mode requests of algorithm chaining are as follows:

\begin{lstlisting}
struct virtio_crypto_alg_chain_data_flf_stateless {
    struct {
        /* See VIRTIO_CRYPTO_SYM_ALG_CHAIN_ORDER_* above */
        le32 alg_chain_order;
        /* length of the additional authenticated data in bytes */
        le32 aad_len;

        struct {
            /* See VIRTIO_CRYPTO_CIPHER* above */
            le32 algo;
            /* length of key */
            le32 key_len;
            /* See VIRTIO_CRYPTO_OP_* above */
            le32 op;
        } cipher;

        struct {
            /* See VIRTIO_CRYPTO_HASH_* or VIRTIO_CRYPTO_MAC_* above */
            le32 algo;
            /* length of authenticated key */
            le32 auth_key_len;
            /* See VIRTIO_CRYPTO_SYM_HASH_MODE_* above */
            le32 hash_mode;
        } hash;
    } sess_para;

    le32 iv_len;
    /* Length of source data */
    le32 src_data_len;
    /* Length of destination data */
    le32 dst_data_len;
    /* Starting point for cipher processing in source data */
    le32 cipher_start_src_offset;
    /* Length of the source data that the cipher will be computed on */
    le32 len_to_cipher;
    /* Starting point for hash processing in source data */
    le32 hash_start_src_offset;
    /* Length of the source data that the hash will be computed on */
    le32 len_to_hash;
    /* Length of the additional auth data */
    le32 aad_len;
    /* Length of the hash result */
    le32 hash_result_len;
    le32 reserved;
};

struct virtio_crypto_alg_chain_data_vlf_stateless {
    /* Device read only portion */

    /* The cipher key */
    u8 cipher_key[key_len];
    /* The auth key */
    u8 auth_key[auth_key_len];
    /* Initialization Vector or Counter data */
    u8 iv[iv_len];
    /* Additional authenticated data if exists */
    u8 aad[aad_len];
    /* Source data */
    u8 src_data[src_data_len];

    /* Device write only portion */

    /* Destination data */
    u8 dst_data[dst_data_len];
    /* Hash result data */
    u8 hash_result[hash_result_len];
};
\end{lstlisting}

Stateless mode requests of symmetric algorithm are as follows:

\begin{lstlisting}
struct virtio_crypto_sym_data_flf_stateless {
    /* Device read only portion */
#define VIRTIO_CRYPTO_SYM_DATE_REQ_HDR_STATELESS_SIZE    72
    u8 op_type_flf[VIRTIO_CRYPTO_SYM_DATE_REQ_HDR_STATELESS_SIZE];

    /* Device write only portion */
    /* See above VIRTIO_CRYPTO_SYM_OP_* */
    le32 op_type;
};

struct virtio_crypto_sym_data_vlf_stateless {
    u8 op_type_vlf[sym_para_len];
};
\end{lstlisting}

\field{op_type_flf} is the \field{op_type} specific header, it MUST starts
with or be one of the following structures:
\begin{itemize*}
\item struct virtio_crypto_cipher_data_flf_stateless
\item struct virtio_crypto_alg_chain_data_flf_stateless
\end{itemize*}

The length of \field{op_type_flf} is fixed to 72 bytes, the data of unused
part (if has) will be ignored.

\field{op_type_vlf} is the \field{op_type} specific parameters, it MUST starts
with or be one of the following structures:
\begin{itemize*}
\item struct virtio_crypto_cipher_data_vlf_stateless
\item struct virtio_crypto_alg_chain_data_vlf_stateless
\end{itemize*}

\field{sym_para_len} is the size of the specific structure used.

\drivernormative{\paragraph}{Symmetric algorithms Operation}{Device Types / Crypto Device / Device Operation / Symmetric algorithms Operation}

\begin{itemize*}
\item If the driver uses the session mode, then the driver MUST set \field{session_id}
    in struct virtio_crypto_op_header to a valid value assigned by the device when the
    session was created.
\item If the VIRTIO_CRYPTO_F_CIPHER_STATELESS_MODE feature bit is negotiated, 1) if the
    driver uses the stateless mode, then the driver MUST set the \field{flag} field in
    struct virtio_crypto_op_header to ZERO and MUST set the fields in struct
    virtio_crypto_cipher_data_flf_stateless.sess_para or struct
    virtio_crypto_alg_chain_data_flf_stateless.sess_para, 2) if the driver uses the
    session mode, then the driver MUST set the \field{flag} field in struct
    virtio_crypto_op_header to VIRTIO_CRYPTO_FLAG_SESSION_MODE.
\item The driver MUST set the \field{opcode} field in struct virtio_crypto_op_header
    to VIRTIO_CRYPTO_CIPHER_ENCRYPT or VIRTIO_CRYPTO_CIPHER_DECRYPT.
\item The driver MUST specify the fields of struct virtio_crypto_cipher_data_flf in
    struct virtio_crypto_sym_data_flf and struct virtio_crypto_cipher_data_vlf in
    struct virtio_crypto_sym_data_vlf if the request is based on VIRTIO_CRYPTO_SYM_OP_CIPHER.
\item The driver MUST specify the fields of struct virtio_crypto_alg_chain_data_flf
    in struct virtio_crypto_sym_data_flf and struct virtio_crypto_alg_chain_data_vlf
    in struct virtio_crypto_sym_data_vlf if the request is of the VIRTIO_CRYPTO_SYM_OP_ALGORITHM_CHAINING
    type.
\end{itemize*}

\devicenormative{\paragraph}{Symmetric algorithms Operation}{Device Types / Crypto Device / Device Operation / Symmetric algorithms Operation}

\begin{itemize*}
\item If the VIRTIO_CRYPTO_F_CIPHER_STATELESS_MODE feature bit is negotiated, the device
    MUST parse \field{flag} field in struct virtio_crypto_op_header in order to decide
	which mode the driver uses.
\item The device MUST parse the virtio_crypto_sym_data_req based on the \field{opcode}
    field in general header.
\item The device MUST parse the fields of struct virtio_crypto_cipher_data_flf in
    struct virtio_crypto_sym_data_flf and struct virtio_crypto_cipher_data_vlf in
    struct virtio_crypto_sym_data_vlf if the request is based on VIRTIO_CRYPTO_SYM_OP_CIPHER.
\item The device MUST parse the fields of struct virtio_crypto_alg_chain_data_flf
    in struct virtio_crypto_sym_data_flf and struct virtio_crypto_alg_chain_data_vlf
    in struct virtio_crypto_sym_data_vlf if the request is of the VIRTIO_CRYPTO_SYM_OP_ALGORITHM_CHAINING
    type.
\item The device MUST copy the result of cryptographic operation in the dst_data[] in
    both plain CIPHER mode and algorithms chain mode.
\item The device MUST check the \field{para}.\field{add_len} is bigger than 0 before
    parse the additional authenticated data in plain algorithms chain mode.
\item The device MUST copy the result of HASH/MAC operation in the hash_result[] is
    of the VIRTIO_CRYPTO_SYM_OP_ALGORITHM_CHAINING type.
\item The device MUST set the \field{status} field in struct virtio_crypto_inhdr to
    one of the following values of enum VIRTIO_CRYPTO_STATUS:
\begin{itemize*}
\item VIRTIO_CRYPTO_OK if the operation success.
\item VIRTIO_CRYPTO_NOTSUPP if the requested algorithm or operation is unsupported.
\item VIRTIO_CRYPTO_INVSESS if the session ID is invalid in session mode.
\item VIRTIO_CRYPTO_ERR if failure not mentioned above occurs.
\end{itemize*}
\end{itemize*}

\subsubsection{AEAD Service Operation}\label{sec:Device Types / Crypto Device / Device Operation / AEAD Service Operation}

Session mode requests of symmetric algorithm are as follows:

\begin{lstlisting}
struct virtio_crypto_aead_data_flf {
    /*
     * Byte Length of valid IV data.
     *
     * For GCM mode, this is either 12 (for 96-bit IVs) or 16, in which
     *   case iv points to J0.
     * For CCM mode, this is the length of the nonce, which can be in the
     *   range 7 to 13 inclusive.
     */
    le32 iv_len;
    /* length of additional auth data */
    le32 aad_len;
    /* length of source data */
    le32 src_data_len;
    /* length of dst data, this should be at least src_data_len + tag_len */
    le32 dst_data_len;
    /* Authentication tag length */
    le32 tag_len;
    le32 reserved;
};

struct virtio_crypto_aead_data_vlf {
    /* Device read only portion */

    /*
     * Initialization Vector data.
     *
     * For GCM mode, this is either the IV (if the length is 96 bits) or J0
     *   (for other sizes), where J0 is as defined by NIST SP800-38D.
     *   Regardless of the IV length, a full 16 bytes needs to be allocated.
     * For CCM mode, the first byte is reserved, and the nonce should be
     *   written starting at &iv[1] (to allow space for the implementation
     *   to write in the flags in the first byte).  Note that a full 16 bytes
     *   should be allocated, even though the iv_len field will have
     *   a value less than this.
     *
     * The IV will be updated after every partial cryptographic operation.
     */
    u8 iv[iv_len];
    /* Source data */
    u8 src_data[src_data_len];
    /* Additional authenticated data if exists */
    u8 aad[aad_len];

    /* Device write only portion */
    /* Pointer to output data */
    u8 dst_data[dst_data_len];
};
\end{lstlisting}

Each request uses the virtio_crypto_aead_data_flf structure and the
virtio_crypto_aead_data_flf structure to store information used to run the
AEAD operations.

Stateless mode AEAD service requests are as follows:

\begin{lstlisting}
struct virtio_crypto_aead_data_flf_stateless {
    struct {
        /* See VIRTIO_CRYPTO_AEAD_* above */
        le32 algo;
        /* length of key */
        le32 key_len;
        /* encrypt or decrypt, See above VIRTIO_CRYPTO_OP_* */
        le32 op;
    } sess_para;

    /* Byte Length of valid IV data. */
    le32 iv_len;
    /* Authentication tag length */
    le32 tag_len;
    /* length of additional auth data */
    le32 aad_len;
    /* length of source data */
    le32 src_data_len;
    /* length of dst data, this should be at least src_data_len + tag_len */
    le32 dst_data_len;
};

struct virtio_crypto_aead_data_vlf_stateless {
    /* Device read only portion */

    /* The cipher key */
    u8 key[key_len];
    /* Initialization Vector data. */
    u8 iv[iv_len];
    /* Source data */
    u8 src_data[src_data_len];
    /* Additional authenticated data if exists */
    u8 aad[aad_len];

    /* Device write only portion */
    /* Pointer to output data */
    u8 dst_data[dst_data_len];
};
\end{lstlisting}

\drivernormative{\paragraph}{AEAD Service Operation}{Device Types / Crypto Device / Device Operation / AEAD Service Operation}

\begin{itemize*}
\item If the driver uses the session mode, then the driver MUST set
    \field{session_id} in struct virtio_crypto_op_header to a valid value assigned
    by the device when the session was created.
\item If the VIRTIO_CRYPTO_F_AEAD_STATELESS_MODE feature bit is negotiated, 1) if
    the driver uses the stateless mode, then the driver MUST set the \field{flag}
    field in struct virtio_crypto_op_header to ZERO and MUST set the fields in
    struct virtio_crypto_aead_data_flf_stateless.sess_para, 2) if the driver uses
    the session mode, then the driver MUST set the \field{flag} field in struct
    virtio_crypto_op_header to VIRTIO_CRYPTO_FLAG_SESSION_MODE.
\item The driver MUST set the \field{opcode} field in struct virtio_crypto_op_header
    to VIRTIO_CRYPTO_AEAD_ENCRYPT or VIRTIO_CRYPTO_AEAD_DECRYPT.
\end{itemize*}

\devicenormative{\paragraph}{AEAD Service Operation}{Device Types / Crypto Device / Device Operation / AEAD Service Operation}

\begin{itemize*}
\item If the VIRTIO_CRYPTO_F_AEAD_STATELESS_MODE feature bit is negotiated, the
    device MUST parse the virtio_crypto_aead_data_vlf_stateless based on the \field{opcode}
	field in general header.
\item The device MUST copy the result of cryptographic operation in the dst_data[].
\item The device MUST copy the authentication tag in the dst_data[] offset the cipher result.
\item The device MUST set the \field{status} field in struct virtio_crypto_inhdr to
    one of the following values of enum VIRTIO_CRYPTO_STATUS:
\item When the \field{opcode} field is VIRTIO_CRYPTO_AEAD_DECRYPT, the device MUST
    verify and return the verification result to the driver.
\begin{itemize*}
\item VIRTIO_CRYPTO_OK if the operation success.
\item VIRTIO_CRYPTO_NOTSUPP if the requested algorithm or operation is unsupported.
\item VIRTIO_CRYPTO_BADMSG if the verification result is incorrect.
\item VIRTIO_CRYPTO_INVSESS if the session ID invalid when in session mode.
\item VIRTIO_CRYPTO_ERR if any failure not mentioned above occurs.
\end{itemize*}
\end{itemize*}

\subsubsection{AKCIPHER Service Operation}\label{sec:Device Types / Crypto Device / Device Operation / AKCIPHER Service Operation}

Session mode AKCIPHER requests are as follows:

\begin{lstlisting}
struct virtio_crypto_akcipher_data_flf {
    /* length of source data */
    le32 src_data_len;
    /* length of dst data */
    le32 dst_data_len;
};

struct virtio_crypto_akcipher_data_vlf {
    /* Device read only portion */
    /* Source data */
    u8 src_data[src_data_len];

    /* Device write only portion */
    /* Pointer to output data */
    u8 dst_data[dst_data_len];
};
\end{lstlisting}

Each data request uses the virtio_crypto_akcipher_flf structure and the virtio_crypto_akcipher_data_vlf
structure to store information used to run the AKCIPHER operations.

For encryption, decryption, and signing:
\field{src_data} is the source data that will be processed, note that for signing operations,
src_data stores the data to be signed, which usually is the digest of some data rather than the
data itself.
\field{src_data_len} is the length of source data.
\field{dst_result} is the result data and \field{dst_data_len} is the length of it. Note that the
length of the result is not always exactly equal to dst_data_len, the driver needs to check how
many bytes the device has written and calculate the actual length of the result.

For verification:
\field{src_data_len} refers to the length of the signature, and \field{dst_data_len} refers to
the length of signed data, where the signed data is usually the digest of some data.
\field{src_data} is spliced by the signature and the signed data, the src_data with the lower
address stores the signature, and the higher address stores the signed data.
\field{dst_data} is always empty for verification.

Different algorithms have different signature formats.
For the RSA algorithm, the result is determined by the padding algorithm specified by
\field{padding_algo} in structure virtio_crypto_rsa_session_para.

For the ECDSA algorithm, the signature is composed of the following
ASN.1 structure (see \hyperref[intro:rfc3279]{RFC3279})
and MUST be DER encoded (see \hyperref[intro:rfc6025]{rfc6025}).

\begin{lstlisting}
Ecdsa-Sig-Value ::= SEQUENCE {
    r INTEGER,
    s INTEGER
}
\end{lstlisting}

Stateless mode AKCIPHER service requests are as follows:
\begin{lstlisting}
struct virtio_crypto_akcipher_data_flf_stateless {
    struct {
        /* See VIRTIO_CYRPTO_AKCIPHER* above */
        le32 algo;
        /* See VIRTIO_CRYPTO_AKCIPHER_KEY_TYPE_* above */
        le32 key_type;
        /* length of key */
        le32 key_len;

        /* algothrim specific parameters described above */
        union {
            struct virtio_crypto_rsa_session_para rsa;
            struct virtio_crypto_ecdsa_session_para ecdsa;
        } u;
    } sess_para;

    /* length of source data */
    le32 src_data_len;
    /* length of destination data */
    le32 dst_data_len;
};

struct virtio_crypto_akcipher_data_vlf_stateless {
    /* Device read only portion */
    u8 akcipher_key[key_len];

    /* Source data */
    u8 src_data[src_data_len];

    /* Device write only portion */
    u8 dst_data[dst_data_len];
};
\end{lstlisting}

In stateless mode, the format of key and signature, the meaning of src_data and dst_data, are all the same
with session mode.

\drivernormative{\paragraph}{AKCIPHER Service Operation}{Device Types / Crypto Device / Device Operation / AKCIPHER Service Operation}

\begin{itemize*}
\item If the driver uses the session mode, then the driver MUST set
    \field{session_id} in struct virtio_crypto_op_header to a valid
    value assigned by the device when the session was created.
\item If the VIRTIO_CRYPTO_F_AKCIPHER_STATELESS_MODE feature bit is negotiated, 1) if the
    driver uses the stateless mode, then the driver MUST set the \field{flag} field in
    struct virtio_crypto_op_header to ZERO and MUST set the fields in struct
    virtio_crypto_akcipher_flf_stateless.sess_para, 2) if the driver uses the session
    mode, then the driver MUST set the \field{flag} field in struct virtio_crypto_op_header
    to VIRTIO_CRYPTO_FLAG_SESSION_MODE.
\item The driver MUST set the \field{opcode} field in struct virtio_crypto_op_header
    to one of VIRTIO_CRYPTO_AKCIPHER_ENCRYPT, VIRTIO_CRYPTO_AKCIPHER_DESTROY_SESSION,
    VIRTIO_CRYPTO_AKCIPHER_SIGN, and VIRTIO_CRYPTO_AKCIPHER_VERIFY.
\end{itemize*}

\devicenormative{\paragraph}{AKCIPHER Service Operation}{Device Types / Crypto Device / Device Operation / AKCIPHER Service Operation}

\begin{itemize*}
\item If the VIRTIO_CRYPTO_F_AKCIPHER_STATELESS_MODE feature bit is negotiated, the
    device MUST parse the virtio_crypto_akcipher_data_vlf_stateless based on the \field{opcode}
    field in general header.
\item The device MUST copy the result of cryptographic operation in the dst_data[].
\item The device MUST set the \field{status} field in struct virtio_crypto_inhdr to
    one of the following values of enum VIRTIO_CRYPTO_STATUS:
\begin{itemize*}
\item VIRTIO_CRYPTO_OK if the operation success.
\item VIRTIO_CRYPTO_NOTSUPP if the requested algorithm or operation is unsupported.
\item VIRTIO_CRYPTO_BADMSG if the verification result is incorrect.
\item VIRTIO_CRYPTO_INVSESS if the session ID invalid when in session mode.
\item VIRTIO_CRYPTO_KEY_REJECTED if the signature verification failed.
\item VIRTIO_CRYPTO_ERR if any failure not mentioned above occurs.
\end{itemize*}
\end{itemize*}

\section{Crypto Device}\label{sec:Device Types / Crypto Device}

The virtio crypto device is a virtual cryptography device as well as a
virtual cryptographic accelerator. The virtio crypto device provides the
following crypto services: CIPHER, MAC, HASH, AEAD and AKCIPHER. Virtio crypto
devices have a single control queue and at least one data queue. Crypto
operation requests are placed into a data queue, and serviced by the
device. Some crypto operation requests are only valid in the context of a
session. The role of the control queue is facilitating control operation
requests. Sessions management is realized with control operation
requests.

\subsection{Device ID}\label{sec:Device Types / Crypto Device / Device ID}

20

\subsection{Virtqueues}\label{sec:Device Types / Crypto Device / Virtqueues}

\begin{description}
\item[0] dataq1
\item[\ldots]
\item[N-1] dataqN
\item[N] controlq
\end{description}

N is set by \field{max_dataqueues}.

\subsection{Feature bits}\label{sec:Device Types / Crypto Device / Feature bits}

\begin{description}
\item VIRTIO_CRYPTO_F_REVISION_1 (0) revision 1. Revision 1 has a specific
    request format and other enhancements (which result in some additional
    requirements).
\item VIRTIO_CRYPTO_F_CIPHER_STATELESS_MODE (1) stateless mode requests are
    supported by the CIPHER service.
\item VIRTIO_CRYPTO_F_HASH_STATELESS_MODE (2) stateless mode requests are
    supported by the HASH service.
\item VIRTIO_CRYPTO_F_MAC_STATELESS_MODE (3) stateless mode requests are
    supported by the MAC service.
\item VIRTIO_CRYPTO_F_AEAD_STATELESS_MODE (4) stateless mode requests are
    supported by the AEAD service.
\item VIRTIO_CRYPTO_F_AKCIPHER_STATELESS_MODE (5) stateless mode requests are
    supported by the AKCIPHER service.
\end{description}


\subsubsection{Feature bit requirements}\label{sec:Device Types / Crypto Device / Feature bit requirements}

Some crypto feature bits require other crypto feature bits
(see \ref{drivernormative:Basic Facilities of a Virtio Device / Feature Bits}):

\begin{description}
\item[VIRTIO_CRYPTO_F_CIPHER_STATELESS_MODE] Requires VIRTIO_CRYPTO_F_REVISION_1.
\item[VIRTIO_CRYPTO_F_HASH_STATELESS_MODE] Requires VIRTIO_CRYPTO_F_REVISION_1.
\item[VIRTIO_CRYPTO_F_MAC_STATELESS_MODE] Requires VIRTIO_CRYPTO_F_REVISION_1.
\item[VIRTIO_CRYPTO_F_AEAD_STATELESS_MODE] Requires VIRTIO_CRYPTO_F_REVISION_1.
\item[VIRTIO_CRYPTO_F_AKCIPHER_STATELESS_MODE] Requires VIRTIO_CRYPTO_F_REVISION_1.
\end{description}

\subsection{Supported crypto services}\label{sec:Device Types / Crypto Device / Supported crypto services}

The following crypto services are defined:

\begin{lstlisting}
/* CIPHER (Symmetric Key Cipher) service */
#define VIRTIO_CRYPTO_SERVICE_CIPHER 0
/* HASH service */
#define VIRTIO_CRYPTO_SERVICE_HASH   1
/* MAC (Message Authentication Codes) service */
#define VIRTIO_CRYPTO_SERVICE_MAC    2
/* AEAD (Authenticated Encryption with Associated Data) service */
#define VIRTIO_CRYPTO_SERVICE_AEAD   3
/* AKCIPHER (Asymmetric Key Cipher) service */
#define VIRTIO_CRYPTO_SERVICE_AKCIPHER 4
\end{lstlisting}

The above constants designate bits used to indicate the which of crypto services are
offered by the device as described in, see \ref{sec:Device Types / Crypto Device / Device configuration layout}.

\subsubsection{CIPHER services}\label{sec:Device Types / Crypto Device / Supported crypto services / CIPHER services}

The following CIPHER algorithms are defined:

\begin{lstlisting}
#define VIRTIO_CRYPTO_NO_CIPHER                 0
#define VIRTIO_CRYPTO_CIPHER_ARC4               1
#define VIRTIO_CRYPTO_CIPHER_AES_ECB            2
#define VIRTIO_CRYPTO_CIPHER_AES_CBC            3
#define VIRTIO_CRYPTO_CIPHER_AES_CTR            4
#define VIRTIO_CRYPTO_CIPHER_DES_ECB            5
#define VIRTIO_CRYPTO_CIPHER_DES_CBC            6
#define VIRTIO_CRYPTO_CIPHER_3DES_ECB           7
#define VIRTIO_CRYPTO_CIPHER_3DES_CBC           8
#define VIRTIO_CRYPTO_CIPHER_3DES_CTR           9
#define VIRTIO_CRYPTO_CIPHER_KASUMI_F8          10
#define VIRTIO_CRYPTO_CIPHER_SNOW3G_UEA2        11
#define VIRTIO_CRYPTO_CIPHER_AES_F8             12
#define VIRTIO_CRYPTO_CIPHER_AES_XTS            13
#define VIRTIO_CRYPTO_CIPHER_ZUC_EEA3           14
\end{lstlisting}

The above constants have two usages:
\begin{enumerate}
\item As bit numbers, used to tell the driver which CIPHER algorithms
are supported by the device, see \ref{sec:Device Types / Crypto Device / Device configuration layout}.
\item As values, used to designate the algorithm in (CIPHER type) crypto
operation requests, see \ref{sec:Device Types / Crypto Device / Device Operation / Control Virtqueue / Session operation}.
\end{enumerate}

\subsubsection{HASH services}\label{sec:Device Types / Crypto Device / Supported crypto services / HASH services}

The following HASH algorithms are defined:

\begin{lstlisting}
#define VIRTIO_CRYPTO_NO_HASH            0
#define VIRTIO_CRYPTO_HASH_MD5           1
#define VIRTIO_CRYPTO_HASH_SHA1          2
#define VIRTIO_CRYPTO_HASH_SHA_224       3
#define VIRTIO_CRYPTO_HASH_SHA_256       4
#define VIRTIO_CRYPTO_HASH_SHA_384       5
#define VIRTIO_CRYPTO_HASH_SHA_512       6
#define VIRTIO_CRYPTO_HASH_SHA3_224      7
#define VIRTIO_CRYPTO_HASH_SHA3_256      8
#define VIRTIO_CRYPTO_HASH_SHA3_384      9
#define VIRTIO_CRYPTO_HASH_SHA3_512      10
#define VIRTIO_CRYPTO_HASH_SHA3_SHAKE128      11
#define VIRTIO_CRYPTO_HASH_SHA3_SHAKE256      12
\end{lstlisting}

The above constants have two usages:
\begin{enumerate}
\item As bit numbers, used to tell the driver which HASH algorithms
are supported by the device, see \ref{sec:Device Types / Crypto Device / Device configuration layout}.
\item As values, used to designate the algorithm in (HASH type) crypto
operation requires, see \ref{sec:Device Types / Crypto Device / Device Operation / Control Virtqueue / Session operation}.
\end{enumerate}

\subsubsection{MAC services}\label{sec:Device Types / Crypto Device / Supported crypto services / MAC services}

The following MAC algorithms are defined:

\begin{lstlisting}
#define VIRTIO_CRYPTO_NO_MAC                       0
#define VIRTIO_CRYPTO_MAC_HMAC_MD5                 1
#define VIRTIO_CRYPTO_MAC_HMAC_SHA1                2
#define VIRTIO_CRYPTO_MAC_HMAC_SHA_224             3
#define VIRTIO_CRYPTO_MAC_HMAC_SHA_256             4
#define VIRTIO_CRYPTO_MAC_HMAC_SHA_384             5
#define VIRTIO_CRYPTO_MAC_HMAC_SHA_512             6
#define VIRTIO_CRYPTO_MAC_CMAC_3DES                25
#define VIRTIO_CRYPTO_MAC_CMAC_AES                 26
#define VIRTIO_CRYPTO_MAC_KASUMI_F9                27
#define VIRTIO_CRYPTO_MAC_SNOW3G_UIA2              28
#define VIRTIO_CRYPTO_MAC_GMAC_AES                 41
#define VIRTIO_CRYPTO_MAC_GMAC_TWOFISH             42
#define VIRTIO_CRYPTO_MAC_CBCMAC_AES               49
#define VIRTIO_CRYPTO_MAC_CBCMAC_KASUMI_F9         50
#define VIRTIO_CRYPTO_MAC_XCBC_AES                 53
#define VIRTIO_CRYPTO_MAC_ZUC_EIA3                 54
\end{lstlisting}

The above constants have two usages:
\begin{enumerate}
\item As bit numbers, used to tell the driver which MAC algorithms
are supported by the device, see \ref{sec:Device Types / Crypto Device / Device configuration layout}.
\item As values, used to designate the algorithm in (MAC type) crypto
operation requests, see \ref{sec:Device Types / Crypto Device / Device Operation / Control Virtqueue / Session operation}.
\end{enumerate}

\subsubsection{AEAD services}\label{sec:Device Types / Crypto Device / Supported crypto services / AEAD services}

The following AEAD algorithms are defined:

\begin{lstlisting}
#define VIRTIO_CRYPTO_NO_AEAD     0
#define VIRTIO_CRYPTO_AEAD_GCM    1
#define VIRTIO_CRYPTO_AEAD_CCM    2
#define VIRTIO_CRYPTO_AEAD_CHACHA20_POLY1305  3
\end{lstlisting}

The above constants have two usages:
\begin{enumerate}
\item As bit numbers, used to tell the driver which AEAD algorithms
are supported by the device, see \ref{sec:Device Types / Crypto Device / Device configuration layout}.
\item As values, used to designate the algorithm in (DEAD type) crypto
operation requests, see \ref{sec:Device Types / Crypto Device / Device Operation / Control Virtqueue / Session operation}.
\end{enumerate}

\subsubsection{AKCIPHER services}\label{sec: Device Types / Crypto Device / Supported crypto services / AKCIPHER services}

The following AKCIPHER algorithms are defined:
\begin{lstlisting}
#define VIRTIO_CRYPTO_NO_AKCIPHER 0
#define VIRTIO_CRYPTO_AKCIPHER_RSA   1
#define VIRTIO_CRYPTO_AKCIPHER_ECDSA 2
\end{lstlisting}

The above constants have two usages:
\begin{enumerate}
\item As bit numbers, used to tell the driver which AKCIPHER algorithms
are supported by the device, see \ref{sec:Device Types / Crypto Device / Device configuration layout}.
\item As values, used to designate the algorithm in asymmetric crypto operation requests,
see \ref{sec:Device Types / Crypto Device / Device Operation / Control Virtqueue / Session operation}.
\end{enumerate}


\subsection{Device configuration layout}\label{sec:Device Types / Crypto Device / Device configuration layout}

Crypto device configuration uses the following layout structure:

\begin{lstlisting}
struct virtio_crypto_config {
    le32 status;
    le32 max_dataqueues;
    le32 crypto_services;
    /* Detailed algorithms mask */
    le32 cipher_algo_l;
    le32 cipher_algo_h;
    le32 hash_algo;
    le32 mac_algo_l;
    le32 mac_algo_h;
    le32 aead_algo;
    /* Maximum length of cipher key in bytes */
    le32 max_cipher_key_len;
    /* Maximum length of authenticated key in bytes */
    le32 max_auth_key_len;
    le32 akcipher_algo;
    /* Maximum size of each crypto request's content in bytes */
    le64 max_size;
};
\end{lstlisting}

\begin{description}
\item Currently, only one \field{status} bit is defined: VIRTIO_CRYPTO_S_HW_READY
    set indicates that the device is ready to process requests, this bit is read-only
    for the driver
\begin{lstlisting}
#define VIRTIO_CRYPTO_S_HW_READY  (1 << 0)
\end{lstlisting}

\item [\field{max_dataqueues}] is the maximum number of data virtqueues that can
    be configured by the device. The driver MAY use only one data queue, or it
    can use more to achieve better performance.

\item [\field{crypto_services}] crypto service offered, see \ref{sec:Device Types / Crypto Device / Supported crypto services}.

\item [\field{cipher_algo_l}] CIPHER algorithms bits 0-31, see \ref{sec:Device Types / Crypto Device / Supported crypto services  / CIPHER services}.

\item [\field{cipher_algo_h}] CIPHER algorithms bits 32-63, see \ref{sec:Device Types / Crypto Device / Supported crypto services  / CIPHER services}.

\item [\field{hash_algo}] HASH algorithms bits, see \ref{sec:Device Types / Crypto Device / Supported crypto services  / HASH services}.

\item [\field{mac_algo_l}] MAC algorithms bits 0-31, see \ref{sec:Device Types / Crypto Device / Supported crypto services  / MAC services}.

\item [\field{mac_algo_h}] MAC algorithms bits 32-63, see \ref{sec:Device Types / Crypto Device / Supported crypto services  / MAC services}.

\item [\field{aead_algo}] AEAD algorithms bits, see \ref{sec:Device Types / Crypto Device / Supported crypto services  / AEAD services}.

\item [\field{max_cipher_key_len}] is the maximum length of cipher key supported by the device.

\item [\field{max_auth_key_len}] is the maximum length of authenticated key supported by the device.

\item [\field{akcipher_algo}] AKCIPHER algorithms bit 0-31, see \ref{sec: Device Types / Crypto Device / Supported crypto services / AKCIPHER services}.

\item [\field{max_size}] is the maximum size of the variable-length parameters of
    data operation of each crypto request's content supported by the device.
\end{description}

\begin{note}
Unless explicitly stated otherwise all lengths and sizes are in bytes.
\end{note}

\devicenormative{\subsubsection}{Device configuration layout}{Device Types / Crypto Device / Device configuration layout}

\begin{itemize*}
\item The device MUST set \field{max_dataqueues} to between 1 and 65535 inclusive.
\item The device MUST set the \field{status} with valid flags, undefined flags MUST NOT be set.
\item The device MUST accept and handle requests after \field{status} is set to VIRTIO_CRYPTO_S_HW_READY.
\item The device MUST set \field{crypto_services} based on the crypto services the device offers.
\item The device MUST set detailed algorithms masks for each service advertised by \field{crypto_services}.
    The device MUST NOT set the not defined algorithms bits.
\item The device MUST set \field{max_size} to show the maximum size of crypto request the device supports.
\item The device MUST set \field{max_cipher_key_len} to show the maximum length of cipher key if the
    device supports CIPHER service.
\item The device MUST set \field{max_auth_key_len} to show the maximum length of authenticated key if
    the device supports MAC service.
\end{itemize*}

\drivernormative{\subsubsection}{Device configuration layout}{Device Types / Crypto Device / Device configuration layout}

\begin{itemize*}
\item The driver MUST read the \field{status} from the bottom bit of status to check whether the
    VIRTIO_CRYPTO_S_HW_READY is set, and the driver MUST reread it after device reset.
\item The driver MUST NOT transmit any requests to the device if the VIRTIO_CRYPTO_S_HW_READY is not set.
\item The driver MUST read \field{max_dataqueues} field to discover the number of data queues the device supports.
\item The driver MUST read \field{crypto_services} field to discover which services the device is able to offer.
\item The driver SHOULD ignore the not defined algorithms bits.
\item The driver MUST read the detailed algorithms fields based on \field{crypto_services} field.
\item The driver SHOULD read \field{max_size} to discover the maximum size of the variable-length
    parameters of data operation of the crypto request's content the device supports and MUST
    guarantee that the size of each crypto request's content is within the \field{max_size}, otherwise
    the request will fail and the driver MUST reset the device.
\item The driver SHOULD read \field{max_cipher_key_len} to discover the maximum length of cipher key
    the device supports and MUST guarantee that the \field{key_len} (CIPHER service or AEAD service) is within
    the \field{max_cipher_key_len} of the device configuration, otherwise the request will fail.
\item The driver SHOULD read \field{max_auth_key_len} to discover the maximum length of authenticated
    key the device supports and MUST guarantee that the \field{auth_key_len} (MAC service) is within the
    \field{max_auth_key_len} of the device configuration, otherwise the request will fail.
\end{itemize*}

\subsection{Device Initialization}\label{sec:Device Types / Crypto Device / Device Initialization}

\drivernormative{\subsubsection}{Device Initialization}{Device Types / Crypto Device / Device Initialization}

\begin{itemize*}
\item The driver MUST configure and initialize all virtqueues.
\item The driver MUST read the supported crypto services from bits of \field{crypto_services}.
\item The driver MUST read the supported algorithms based on \field{crypto_services} field.
\end{itemize*}

\subsection{Device Operation}\label{sec:Device Types / Crypto Device / Device Operation}

The operation of a virtio crypto device is driven by requests placed on the virtqueues.
Requests consist of a queue-type specific header (specifying among others the operation)
and an operation specific payload.

If VIRTIO_CRYPTO_F_REVISION_1 is negotiated the device may support both session mode
(See \ref{sec:Device Types / Crypto Device / Device Operation / Control Virtqueue / Session operation})
and stateless mode operation requests.
In stateless mode all operation parameters are supplied as a part of each request,
while in session mode, some or all operation parameters are managed within the
session. Stateless mode is guarded by feature bits 0-4 on a service level. If
stateless mode is negotiated for a service, the service accepts both session
mode and stateless requests; otherwise stateless mode requests are rejected
(via operation status).

\subsubsection{Operation Status}\label{sec:Device Types / Crypto Device / Device Operation / Operation status}
The device MUST return a status code as part of the operation (both session
operation and service operation) result. The valid operation status as follows:

\begin{lstlisting}
enum VIRTIO_CRYPTO_STATUS {
    VIRTIO_CRYPTO_OK = 0,
    VIRTIO_CRYPTO_ERR = 1,
    VIRTIO_CRYPTO_BADMSG = 2,
    VIRTIO_CRYPTO_NOTSUPP = 3,
    VIRTIO_CRYPTO_INVSESS = 4,
    VIRTIO_CRYPTO_NOSPC = 5,
    VIRTIO_CRYPTO_KEY_REJECTED = 6,
    VIRTIO_CRYPTO_MAX
};
\end{lstlisting}

\begin{itemize*}
\item VIRTIO_CRYPTO_OK: success.
\item VIRTIO_CRYPTO_BADMSG: authentication failed (only when AEAD decryption).
\item VIRTIO_CRYPTO_NOTSUPP: operation or algorithm is unsupported.
\item VIRTIO_CRYPTO_INVSESS: invalid session ID when executing crypto operations.
\item VIRTIO_CRYPTO_NOSPC: no free session ID (only when the VIRTIO_CRYPTO_F_REVISION_1
    feature bit is negotiated).
\item VIRTIO_CRYPTO_KEY_REJECTED: signature verification failed (only when AKCIPHER verification).
\item VIRTIO_CRYPTO_ERR: any failure not mentioned above occurs.
\end{itemize*}

\subsubsection{Control Virtqueue}\label{sec:Device Types / Crypto Device / Device Operation / Control Virtqueue}

The driver uses the control virtqueue to send control commands to the
device, such as session operations (See \ref{sec:Device Types / Crypto Device / Device
Operation / Control Virtqueue / Session operation}).

The header for controlq is of the following form:
\begin{lstlisting}
#define VIRTIO_CRYPTO_OPCODE(service, op)   (((service) << 8) | (op))

struct virtio_crypto_ctrl_header {
#define VIRTIO_CRYPTO_CIPHER_CREATE_SESSION \
       VIRTIO_CRYPTO_OPCODE(VIRTIO_CRYPTO_SERVICE_CIPHER, 0x02)
#define VIRTIO_CRYPTO_CIPHER_DESTROY_SESSION \
       VIRTIO_CRYPTO_OPCODE(VIRTIO_CRYPTO_SERVICE_CIPHER, 0x03)
#define VIRTIO_CRYPTO_HASH_CREATE_SESSION \
       VIRTIO_CRYPTO_OPCODE(VIRTIO_CRYPTO_SERVICE_HASH, 0x02)
#define VIRTIO_CRYPTO_HASH_DESTROY_SESSION \
       VIRTIO_CRYPTO_OPCODE(VIRTIO_CRYPTO_SERVICE_HASH, 0x03)
#define VIRTIO_CRYPTO_MAC_CREATE_SESSION \
       VIRTIO_CRYPTO_OPCODE(VIRTIO_CRYPTO_SERVICE_MAC, 0x02)
#define VIRTIO_CRYPTO_MAC_DESTROY_SESSION \
       VIRTIO_CRYPTO_OPCODE(VIRTIO_CRYPTO_SERVICE_MAC, 0x03)
#define VIRTIO_CRYPTO_AEAD_CREATE_SESSION \
       VIRTIO_CRYPTO_OPCODE(VIRTIO_CRYPTO_SERVICE_AEAD, 0x02)
#define VIRTIO_CRYPTO_AEAD_DESTROY_SESSION \
       VIRTIO_CRYPTO_OPCODE(VIRTIO_CRYPTO_SERVICE_AEAD, 0x03)
#define VIRTIO_CRYPTO_AKCIPHER_CREATE_SESSION \
       VIRTIO_CRYPTO_OPCODE(VIRTIO_CRYPTO_SERVICE_AKCIPHER, 0x04)
#define VIRTIO_CRYPTO_AKCIPHER_DESTROY_SESSION \
       VIRTIO_CRYPTO_OPCDE(VIRTIO_CRYPTO_SERVICE_AKCIPHER, 0x05)
    le32 opcode;
    /* algo should be service-specific algorithms */
    le32 algo;
    le32 flag;
    le32 reserved;
};
\end{lstlisting}

The controlq request is composed of four parts:
\begin{lstlisting}
struct virtio_crypto_op_ctrl_req {
    /* Device read only portion */

    struct virtio_crypto_ctrl_header header;

#define VIRTIO_CRYPTO_CTRLQ_OP_SPEC_HDR_LEGACY 56
    /* fixed length fields, opcode specific */
    u8 op_flf[flf_len];

    /* variable length fields, opcode specific */
    u8 op_vlf[vlf_len];

    /* Device write only portion */

    /* op result or completion status */
    u8 op_outcome[outcome_len];
};
\end{lstlisting}

\field{header} is a general header (see above).

\field{op_flf} is the opcode (in \field{header}) specific fixed-length parameters.

\field{flf_len} depends on the VIRTIO_CRYPTO_F_REVISION_1 feature bit (see below).

\field{op_vlf} is the opcode (in \field{header}) specific variable-length parameters.

\field{vlf_len} is the size of the specific structure used.
\begin{note}
The \field{vlf_len} of session-destroy operation and the hash-session-create
operation is ZERO.
\end{note}

\begin{itemize*}
\item If the opcode (in \field{header}) is VIRTIO_CRYPTO_CIPHER_CREATE_SESSION
    then \field{op_flf} is struct virtio_crypto_sym_create_session_flf if
    VIRTIO_CRYPTO_F_REVISION_1 is negotiated and struct virtio_crypto_sym_create_session_flf is
    padded to 56 bytes if NOT negotiated, and \field{op_vlf} is struct
    virtio_crypto_sym_create_session_vlf.
\item If the opcode (in \field{header}) is VIRTIO_CRYPTO_HASH_CREATE_SESSION
    then \field{op_flf} is struct virtio_crypto_hash_create_session_flf if
    VIRTIO_CRYPTO_F_REVISION_1 is negotiated and struct virtio_crypto_hash_create_session_flf is
    padded to 56 bytes if NOT negotiated.
\item If the opcode (in \field{header}) is VIRTIO_CRYPTO_MAC_CREATE_SESSION
    then \field{op_flf} is struct virtio_crypto_mac_create_session_flf if
    VIRTIO_CRYPTO_F_REVISION_1 is negotiated and struct virtio_crypto_mac_create_session_flf is
    padded to 56 bytes if NOT negotiated, and \field{op_vlf} is struct
    virtio_crypto_mac_create_session_vlf.
\item If the opcode (in \field{header}) is VIRTIO_CRYPTO_AEAD_CREATE_SESSION
    then \field{op_flf} is struct virtio_crypto_aead_create_session_flf if
    VIRTIO_CRYPTO_F_REVISION_1 is negotiated and struct virtio_crypto_aead_create_session_flf is
    padded to 56 bytes if NOT negotiated, and \field{op_vlf} is struct
    virtio_crypto_aead_create_session_vlf.
\item If the opcode (in \field{header}) is VIRTIO_CRYPTO_AKCIPHER_CREATE_SESSION
    then \field{op_flf} is struct virtio_crypto_akcipher_create_session_flf if
    VIRTIO_CRYPTO_F_REVISION_1 is negotiated and struct virtio_crypto_akcipher_create_session_flf is
    padded to 56 bytes if NOT negotiated, and \field{op_vlf} is struct
    virtio_crypto_akcipher_create_session_vlf.
\item If the opcode (in \field{header}) is VIRTIO_CRYPTO_CIPHER_DESTROY_SESSION
    or VIRTIO_CRYPTO_HASH_DESTROY_SESSION or VIRTIO_CRYPTO_MAC_DESTROY_SESSION or
    VIRTIO_CRYPTO_AEAD_DESTROY_SESSION then \field{op_flf} is struct
    virtio_crypto_destroy_session_flf if VIRTIO_CRYPTO_F_REVISION_1 is negotiated and
    struct virtio_crypto_destroy_session_flf is padded to 56 bytes if NOT negotiated.
\end{itemize*}

\field{op_outcome} stores the result of operation and must be struct
virtio_crypto_destroy_session_input for destroy session or
struct virtio_crypto_create_session_input for create session.

\field{outcome_len} is the size of the structure used.


\paragraph{Session operation}\label{sec:Device Types / Crypto Device / Device
Operation / Control Virtqueue / Session operation}

The session is a handle which describes the cryptographic parameters to be
applied to a number of buffers.

The following structure stores the result of session creation set by the device:

\begin{lstlisting}
struct virtio_crypto_create_session_input {
    le64 session_id;
    le32 status;
    le32 padding;
};
\end{lstlisting}

A request to destroy a session includes the following information:

\begin{lstlisting}
struct virtio_crypto_destroy_session_flf {
    /* Device read only portion */
    le64  session_id;
};

struct virtio_crypto_destroy_session_input {
    /* Device write only portion */
    u8  status;
};
\end{lstlisting}


\subparagraph{Session operation: HASH session}\label{sec:Device Types / Crypto Device / Device
Operation / Control Virtqueue / Session operation / Session operation: HASH session}

The fixed-length parameters of HASH session requests is as follows:

\begin{lstlisting}
struct virtio_crypto_hash_create_session_flf {
    /* Device read only portion */

    /* See VIRTIO_CRYPTO_HASH_* above */
    le32 algo;
    /* hash result length */
    le32 hash_result_len;
};
\end{lstlisting}


\subparagraph{Session operation: MAC session}\label{sec:Device Types / Crypto Device / Device
Operation / Control Virtqueue / Session operation / Session operation: MAC session}

The fixed-length and the variable-length parameters of MAC session requests are as follows:

\begin{lstlisting}
struct virtio_crypto_mac_create_session_flf {
    /* Device read only portion */

    /* See VIRTIO_CRYPTO_MAC_* above */
    le32 algo;
    /* hash result length */
    le32 hash_result_len;
    /* length of authenticated key */
    le32 auth_key_len;
    le32 padding;
};

struct virtio_crypto_mac_create_session_vlf {
    /* Device read only portion */

    /* The authenticated key */
    u8 auth_key[auth_key_len];
};
\end{lstlisting}

The length of \field{auth_key} is specified in \field{auth_key_len} in the struct
virtio_crypto_mac_create_session_flf.


\subparagraph{Session operation: Symmetric algorithms session}\label{sec:Device Types / Crypto Device / Device
Operation / Control Virtqueue / Session operation / Session operation: Symmetric algorithms session}

The request of symmetric session could be the CIPHER algorithms request
or the chain algorithms (chaining CIPHER and HASH/MAC) request.

The fixed-length and the variable-length parameters of CIPHER session requests are as follows:

\begin{lstlisting}
struct virtio_crypto_cipher_session_flf {
    /* Device read only portion */

    /* See VIRTIO_CRYPTO_CIPHER* above */
    le32 algo;
    /* length of key */
    le32 key_len;
#define VIRTIO_CRYPTO_OP_ENCRYPT  1
#define VIRTIO_CRYPTO_OP_DECRYPT  2
    /* encryption or decryption */
    le32 op;
    le32 padding;
};

struct virtio_crypto_cipher_session_vlf {
    /* Device read only portion */

    /* The cipher key */
    u8 cipher_key[key_len];
};
\end{lstlisting}

The length of \field{cipher_key} is specified in \field{key_len} in the struct
virtio_crypto_cipher_session_flf.

The fixed-length and the variable-length parameters of Chain session requests are as follows:

\begin{lstlisting}
struct virtio_crypto_alg_chain_session_flf {
    /* Device read only portion */

#define VIRTIO_CRYPTO_SYM_ALG_CHAIN_ORDER_HASH_THEN_CIPHER  1
#define VIRTIO_CRYPTO_SYM_ALG_CHAIN_ORDER_CIPHER_THEN_HASH  2
    le32 alg_chain_order;
/* Plain hash */
#define VIRTIO_CRYPTO_SYM_HASH_MODE_PLAIN    1
/* Authenticated hash (mac) */
#define VIRTIO_CRYPTO_SYM_HASH_MODE_AUTH     2
/* Nested hash */
#define VIRTIO_CRYPTO_SYM_HASH_MODE_NESTED   3
    le32 hash_mode;
    struct virtio_crypto_cipher_session_flf cipher_hdr;

#define VIRTIO_CRYPTO_ALG_CHAIN_SESS_OP_SPEC_HDR_SIZE  16
    /* fixed length fields, algo specific */
    u8 algo_flf[VIRTIO_CRYPTO_ALG_CHAIN_SESS_OP_SPEC_HDR_SIZE];

    /* length of the additional authenticated data (AAD) in bytes */
    le32 aad_len;
    le32 padding;
};

struct virtio_crypto_alg_chain_session_vlf {
    /* Device read only portion */

    /* The cipher key */
    u8 cipher_key[key_len];
    /* The authenticated key */
    u8 auth_key[auth_key_len];
};
\end{lstlisting}

\field{hash_mode} decides the type used by \field{algo_flf}.

\field{algo_flf} is fixed to 16 bytes and MUST contains or be one of
the following types:
\begin{itemize*}
\item struct virtio_crypto_hash_create_session_flf
\item struct virtio_crypto_mac_create_session_flf
\end{itemize*}
The data of unused part (if has) in \field{algo_flf} will be ignored.

The length of \field{cipher_key} is specified in \field{key_len} in \field{cipher_hdr}.

The length of \field{auth_key} is specified in \field{auth_key_len} in struct
virtio_crypto_mac_create_session_flf.

The fixed-length parameters of Symmetric session requests are as follows:

\begin{lstlisting}
struct virtio_crypto_sym_create_session_flf {
    /* Device read only portion */

#define VIRTIO_CRYPTO_SYM_SESS_OP_SPEC_HDR_SIZE  48
    /* fixed length fields, opcode specific */
    u8 op_flf[VIRTIO_CRYPTO_SYM_SESS_OP_SPEC_HDR_SIZE];

/* No operation */
#define VIRTIO_CRYPTO_SYM_OP_NONE  0
/* Cipher only operation on the data */
#define VIRTIO_CRYPTO_SYM_OP_CIPHER  1
/* Chain any cipher with any hash or mac operation. The order
   depends on the value of alg_chain_order param */
#define VIRTIO_CRYPTO_SYM_OP_ALGORITHM_CHAINING  2
    le32 op_type;
    le32 padding;
};
\end{lstlisting}

\field{op_flf} is fixed to 48 bytes, MUST contains or be one of
the following types:
\begin{itemize*}
\item struct virtio_crypto_cipher_session_flf
\item struct virtio_crypto_alg_chain_session_flf
\end{itemize*}
The data of unused part (if has) in \field{op_flf} will be ignored.

\field{op_type} decides the type used by \field{op_flf}.

The variable-length parameters of Symmetric session requests are as follows:

\begin{lstlisting}
struct virtio_crypto_sym_create_session_vlf {
    /* Device read only portion */
    /* variable length fields, opcode specific */
    u8 op_vlf[vlf_len];
};
\end{lstlisting}

\field{op_vlf} MUST contains or be one of the following types:
\begin{itemize*}
\item struct virtio_crypto_cipher_session_vlf
\item struct virtio_crypto_alg_chain_session_vlf
\end{itemize*}

\field{op_type} in struct virtio_crypto_sym_create_session_flf decides the
type used by \field{op_vlf}.

\field{vlf_len} is the size of the specific structure used.


\subparagraph{Session operation: AEAD session}\label{sec:Device Types / Crypto Device / Device
Operation / Control Virtqueue / Session operation / Session operation: AEAD session}

The fixed-length and the variable-length parameters of AEAD session requests are as follows:

\begin{lstlisting}
struct virtio_crypto_aead_create_session_flf {
    /* Device read only portion */

    /* See VIRTIO_CRYPTO_AEAD_* above */
    le32 algo;
    /* length of key */
    le32 key_len;
    /* Authentication tag length */
    le32 tag_len;
    /* The length of the additional authenticated data (AAD) in bytes */
    le32 aad_len;
    /* encryption or decryption, See above VIRTIO_CRYPTO_OP_* */
    le32 op;
    le32 padding;
};

struct virtio_crypto_aead_create_session_vlf {
    /* Device read only portion */
    u8 key[key_len];
};
\end{lstlisting}

The length of \field{key} is specified in \field{key_len} in struct
virtio_crypto_aead_create_session_flf.

\subparagraph{Session operation: AKCIPHER session}\label{sec:Device Types / Crypto Device / Device
Operation / Control Virtqueue / Session operation / Session operation: AKCIPHER session}

Due to the complexity of asymmetric key algorithms, different algorithms
require different parameters. The following data structures are used as
supplementary parameters to describe the asymmetric algorithm sessions.

For the RSA algorithm, the extra parameters are as follows:
\begin{lstlisting}
struct virtio_crypto_rsa_session_para {
#define VIRTIO_CRYPTO_RSA_RAW_PADDING   0
#define VIRTIO_CRYPTO_RSA_PKCS1_PADDING 1
    le32 padding_algo;

#define VIRTIO_CRYPTO_RSA_NO_HASH   0
#define VIRTIO_CRYPTO_RSA_MD2       1
#define VIRTIO_CRYPTO_RSA_MD3       2
#define VIRTIO_CRYPTO_RSA_MD4       3
#define VIRTIO_CRYPTO_RSA_MD5       4
#define VIRTIO_CRYPTO_RSA_SHA1      5
#define VIRTIO_CRYPTO_RSA_SHA256    6
#define VIRTIO_CRYPTO_RSA_SHA384    7
#define VIRTIO_CRYPTO_RSA_SHA512    8
#define VIRTIO_CRYPTO_RSA_SHA224    9
    le32 hash_algo;
};
\end{lstlisting}

\field{padding_algo} specifies the padding method used by RSA sessions.
\begin{itemize*}
\item If VIRTIO_CRYPTO_RSA_RAW_PADDING is specified, 1) \field{hash_algo}
is ignored, 2) ciphertext and plaintext MUST be padded with leading zeros,
3) and RSA sessions with VIRTIO_CRYPTO_RSA_RAW_PADDING MUST not be used
for verification and signing operations.
\item If VIRTIO_CRYPTO_RSA_PKCS1_PADDING is specified, EMSA-PKCS1-v1_5 padding method
is used (see \hyperref[intro:rfc3447]{PKCS\#1}), \field{hash_algo} specifies how the
digest of the data passed to RSA sessions is calculated when verifying and signing.
It only affects the padding algorithm and is ignored during encryption and decryption.
\end{itemize*}

The ECC algorithms such as the ECDSA algorithm, cannot use custom curves, only the
following known curves can be used (see \hyperref[intro:NIST]{NIST-recommended curves}).

\begin{lstlisting}
#define VIRTIO_CRYPTO_CURVE_UNKNOWN   0
#define VIRTIO_CRYPTO_CURVE_NIST_P192 1
#define VIRTIO_CRYPTO_CURVE_NIST_P224 2
#define VIRTIO_CRYPTO_CURVE_NIST_P256 3
#define VIRTIO_CRYPTO_CURVE_NIST_P384 4
#define VIRTIO_CRYPTO_CURVE_NIST_P521 5
\end{lstlisting}

For the ECDSA algorithm, the extra parameters are as follows:
\begin{lstlisting}
struct virtio_crypto_ecdsa_session_para {
    /* See VIRTIO_CRYPTO_CURVE_* above */
    le32 curve_id;
};
\end{lstlisting}

The fixed-length and the variable-length parameters of AKCIPHER session requests are as follows:
\begin{lstlisting}
struct virtio_crypto_akcipher_create_session_flf {
    /* Device read only portion */

    /* See VIRTIO_CRYPTO_AKCIPHER_* above */
    le32 algo;
#define VIRTIO_CRYPTO_AKCIPHER_KEY_TYPE_PUBLIC 1
#define VIRTIO_CRYPTO_AKCIPHER_KEY_TYPE_PRIVATE 2
    le32 key_type;
    /* length of key */
    le32 key_len;

#define VIRTIO_CRYPTO_AKCIPHER_SESS_ALGO_SPEC_HDR_SIZE 44
    u8 algo_flf[VIRTIO_CRYPTO_AKCIPHER_SESS_ALGO_SPEC_HDR_SIZE];
};

struct virtio_crypto_akcipher_create_session_vlf {
    /* Device read only portion */
    u8 key[key_len];
};
\end{lstlisting}

\field{algo} decides the type used by \field{algo_flf}.
\field{algo_flf} is fixed to 44 bytes and MUST contains of be one the
following structures:
\begin{itemize*}
\item struct virtio_crypto_rsa_session_para
\item struct virtio_crypto_ecdsa_session_para
\end{itemize*}

The length of \field{key} is specified in \field{key_len} in the struct
virtio_crypto_akcipher_create_session_flf.

For the RSA algorithm, the key needs to be encoded according to
\hyperref[intro:rfc3447]{PKCS\#1}. The private key is described with the
RSAPrivateKey structure, and the public key is described with the RSAPublicKey
structure. These ASN.1 structures are encoded in DER encoding rules (see
\hyperref[intro:rfc6025]{rfc6025}).

\begin{lstlisting}
RSAPrivateKey ::= SEQUENCE {
    version          INTEGER,
    modulus          INTEGER,
    publicExponent   INTEGER,
    privateExponent  INTEGER,
    prime1           INTEGER,
    prime2           INTEGER,
    exponent1        INTEGER,
    exponent1        INTEGER,
    coefficient      INTEGER,
    otherPrimeInfos  OtherPrimeInfos OPTIONAL
}

OtherPrimeInfos ::= SEQUENCE SIZE(1...MAX) OF OtherPrimeInfo

OtherPrimeINfo ::= SEQUENCE {
    prime           INTEGER,
    exponent        INTEGER,
    coefficient     INTEGER
}

RSAPublicKey ::= SEQUENCE {
    modulus         INTEGER,
    publicExponent  INTEGER
}
\end{lstlisting}

For the ECDSA algorithm, the private key is encoded according to
\hyperref[intro:rfc5915]{RFC5915}, the private key of the ECDSA algorithm
is described by the ASN.1 structure ECPrivateKey and encoded with DER
encoding rules (see \hyperref[intro:rfc6025]{rfc6025}).

\begin{lstlisting}
ECPrivateKey ::= SEQUNCE {
    version         INTEGER,
    privateKey      OCTET STRING,
    parameters [0]  ECParameters {{ NamedCurve }} OPTIONAL,
    publicKey  [1]  BIT STRING OPTIONAL
}
\end{lstlisting}

The public key of the ECDSA algorithm is encoded according to \hyperref[intro:SEC1]{SEC1},
and the public key of ECDSA is described by the ASN.1 structure ECPoint.
When initializing a session with ECDSA public key, the ECPoint is DER encoded and the
\field{key} only contains the value part of ECPoint, that is, the header part of the
OCTET STRING will be omitted (see \hyperref[intro:rfc6025]{rfc6025}).

\begin{lstlisting}
ECPoint ::= OCTET STRING
\end{lstlisting}

The length of \field{key} is specified in \field{key_len} in
struct virtio_crypto_akcipher_create_session_flf.

\drivernormative{\subparagraph}{Session operation: create session}{Device Types / Crypto Device / Device
Operation / Control Virtqueue / Session operation / Session operation: create session}

\begin{itemize*}
\item The driver MUST set the \field{opcode} field based on service type: CIPHER, HASH, MAC, AEAD or AKCIPHER.
\item The driver MUST set the control general header, the opcode specific header,
    the opcode specific extra parameters and the opcode specific outcome buffer in turn.
    See \ref{sec:Device Types / Crypto Device / Device Operation / Control Virtqueue}.
\item The driver MUST set the \field{reversed} field to zero.
\end{itemize*}

\devicenormative{\subparagraph}{Session operation: create session}{Device Types / Crypto Device / Device
Operation / Control Virtqueue / Session operation / Session operation: create session}

\begin{itemize*}
\item The device MUST use the corresponding opcode specific structure according to the
    \field{opcode} in the control general header.
\item The device MUST extract extra parameters according to the structures used.
\item The device MUST set the \field{status} field to one of the following values of enum
    VIRTIO_CRYPTO_STATUS after finish a session creation:
\begin{itemize*}
\item VIRTIO_CRYPTO_OK if a session is created successfully.
\item VIRTIO_CRYPTO_NOTSUPP if the requested algorithm or operation is unsupported.
\item VIRTIO_CRYPTO_NOSPC if no free session ID (only when the VIRTIO_CRYPTO_F_REVISION_1
    feature bit is negotiated).
\item VIRTIO_CRYPTO_ERR if failure not mentioned above occurs.
\end{itemize*}
\item The device MUST set the \field{session_id} field to a unique session identifier only
    if the status is set to VIRTIO_CRYPTO_OK.
\end{itemize*}

\drivernormative{\subparagraph}{Session operation: destroy session}{Device Types / Crypto Device / Device
Operation / Control Virtqueue / Session operation / Session operation: destroy session}

\begin{itemize*}
\item The driver MUST set the \field{opcode} field based on service type: CIPHER, HASH, MAC, AEAD or AKCIPHER.
\item The driver MUST set the \field{session_id} to a valid value assigned by the device
    when the session was created.
\end{itemize*}

\devicenormative{\subparagraph}{Session operation: destroy session}{Device Types / Crypto Device / Device
Operation / Control Virtqueue / Session operation / Session operation: destroy session}

\begin{itemize*}
\item The device MUST set the \field{status} field to one of the following values of enum VIRTIO_CRYPTO_STATUS.
\begin{itemize*}
\item VIRTIO_CRYPTO_OK if a session is created successfully.
\item VIRTIO_CRYPTO_ERR if any failure occurs.
\end{itemize*}
\end{itemize*}


\subsubsection{Data Virtqueue}\label{sec:Device Types / Crypto Device / Device Operation / Data Virtqueue}

The driver uses the data virtqueues to transmit crypto operation requests to the device,
and completes the crypto operations.

The header for dataq is as follows:

\begin{lstlisting}
struct virtio_crypto_op_header {
#define VIRTIO_CRYPTO_CIPHER_ENCRYPT \
    VIRTIO_CRYPTO_OPCODE(VIRTIO_CRYPTO_SERVICE_CIPHER, 0x00)
#define VIRTIO_CRYPTO_CIPHER_DECRYPT \
    VIRTIO_CRYPTO_OPCODE(VIRTIO_CRYPTO_SERVICE_CIPHER, 0x01)
#define VIRTIO_CRYPTO_HASH \
    VIRTIO_CRYPTO_OPCODE(VIRTIO_CRYPTO_SERVICE_HASH, 0x00)
#define VIRTIO_CRYPTO_MAC \
    VIRTIO_CRYPTO_OPCODE(VIRTIO_CRYPTO_SERVICE_MAC, 0x00)
#define VIRTIO_CRYPTO_AEAD_ENCRYPT \
    VIRTIO_CRYPTO_OPCODE(VIRTIO_CRYPTO_SERVICE_AEAD, 0x00)
#define VIRTIO_CRYPTO_AEAD_DECRYPT \
    VIRTIO_CRYPTO_OPCODE(VIRTIO_CRYPTO_SERVICE_AEAD, 0x01)
#define VIRTIO_CRYPTO_AKCIPHER_ENCRYPT \
    VIRTIO_CRYPTO_OPCODE(VIRTIO_CRYPTO_SERVICE_AKCIPHER, 0x00)
#define VIRTIO_CRYPTO_AKCIPHER_DECRYPT \
    VIRTIO_CRYPTO_OPCODE(VIRTIO_CRYPTO_SERVICE_AKCIPHER, 0x01)
#define VIRTIO_CRYPTO_AKCIPHER_SIGN \
    VIRTIO_CRYPTO_OPCODE(VIRTIO_CRYPTO_SERVICE_AKCIPHER, 0x02)
#define VIRTIO_CRYPTO_AKCIPHER_VERIFY \
    VIRTIO_CRYPTO_OPCODE(VIRTIO_CRYPTO_SERVICE_AKCIPHER, 0x03)
    le32 opcode;
    /* algo should be service-specific algorithms */
    le32 algo;
    le64 session_id;
#define VIRTIO_CRYPTO_FLAG_SESSION_MODE 1
    /* control flag to control the request */
    le32 flag;
    le32 padding;
};
\end{lstlisting}

\begin{note}
If VIRTIO_CRYPTO_F_REVISION_1 is not negotiated the \field{flag} is ignored.

If VIRTIO_CRYPTO_F_REVISION_1 is negotiated but VIRTIO_CRYPTO_F_<SERVICE>_STATELESS_MODE
is not negotiated, then the device SHOULD reject <SERVICE> requests if
VIRTIO_CRYPTO_FLAG_SESSION_MODE is not set (in \field{flag}).
\end{note}

The dataq request is composed of four parts:
\begin{lstlisting}
struct virtio_crypto_op_data_req {
    /* Device read only portion */

    struct virtio_crypto_op_header header;

#define VIRTIO_CRYPTO_DATAQ_OP_SPEC_HDR_LEGACY 48
    /* fixed length fields, opcode specific */
    u8 op_flf[flf_len];

    /* Device read && write portion */
    /* variable length fields, opcode specific */
    u8 op_vlf[vlf_len];

    /* Device write only portion */
    struct virtio_crypto_inhdr inhdr;
};
\end{lstlisting}

\field{header} is a general header (see above).

\field{op_flf} is the opcode (in \field{header}) specific header.

\field{flf_len} depends on the VIRTIO_CRYPTO_F_REVISION_1 feature bit
(see below).

\field{op_vlf} is the opcode (in \field{header}) specific parameters.

\field{vlf_len} is the size of the specific structure used.

\begin{itemize*}
\item If the the opcode (in \field{header}) is VIRTIO_CRYPTO_CIPHER_ENCRYPT
    or VIRTIO_CRYPTO_CIPHER_DECRYPT then:
    \begin{itemize*}
    \item If VIRTIO_CRYPTO_F_CIPHER_STATELESS_MODE is negotiated, \field{op_flf} is
        struct virtio_crypto_sym_data_flf_stateless, and \field{op_vlf} is struct
        virtio_crypto_sym_data_vlf_stateless.
    \item If VIRTIO_CRYPTO_F_CIPHER_STATELESS_MODE is NOT negotiated, \field{op_flf}
        is struct virtio_crypto_sym_data_flf if VIRTIO_CRYPTO_F_REVISION_1 is negotiated
        and struct virtio_crypto_sym_data_flf is padded to 48 bytes if NOT negotiated,
        and \field{op_vlf} is struct virtio_crypto_sym_data_vlf.
    \end{itemize*}
\item If the the opcode (in \field{header}) is VIRTIO_CRYPTO_HASH:
    \begin{itemize*}
    \item If VIRTIO_CRYPTO_F_HASH_STATELESS_MODE is negotiated, \field{op_flf} is
        struct virtio_crypto_hash_data_flf_stateless, and \field{op_vlf} is struct
        virtio_crypto_hash_data_vlf_stateless.
    \item If VIRTIO_CRYPTO_F_HASH_STATELESS_MODE is NOT negotiated, \field{op_flf}
        is struct virtio_crypto_hash_data_flf if VIRTIO_CRYPTO_F_REVISION_1 is negotiated
        and struct virtio_crypto_hash_data_flf is padded to 48 bytes if NOT negotiated,
        and \field{op_vlf} is struct virtio_crypto_hash_data_vlf.
    \end{itemize*}
\item If the the opcode (in \field{header}) is VIRTIO_CRYPTO_MAC:
    \begin{itemize*}
    \item If VIRTIO_CRYPTO_F_MAC_STATELESS_MODE is negotiated, \field{op_flf} is
        struct virtio_crypto_mac_data_flf_stateless, and \field{op_vlf} is struct
        virtio_crypto_mac_data_vlf_stateless.
    \item If VIRTIO_CRYPTO_F_MAC_STATELESS_MODE is NOT negotiated, \field{op_flf}
        is struct virtio_crypto_mac_data_flf if VIRTIO_CRYPTO_F_REVISION_1 is negotiated
        and struct virtio_crypto_mac_data_flf is padded to 48 bytes if NOT negotiated,
        and \field{op_vlf} is struct virtio_crypto_mac_data_vlf.
    \end{itemize*}
\item If the the opcode (in \field{header}) is VIRTIO_CRYPTO_AEAD_ENCRYPT
    or VIRTIO_CRYPTO_AEAD_DECRYPT then:
    \begin{itemize*}
    \item If VIRTIO_CRYPTO_F_AEAD_STATELESS_MODE is negotiated, \field{op_flf} is
        struct virtio_crypto_aead_data_flf_stateless, and \field{op_vlf} is struct
        virtio_crypto_aead_data_vlf_stateless.
    \item If VIRTIO_CRYPTO_F_AEAD_STATELESS_MODE is NOT negotiated, \field{op_flf}
        is struct virtio_crypto_aead_data_flf if VIRTIO_CRYPTO_F_REVISION_1 is negotiated
        and struct virtio_crypto_aead_data_flf is padded to 48 bytes if NOT negotiated,
        and \field{op_vlf} is struct virtio_crypto_aead_data_vlf.
    \end{itemize*}
\item If the opcode (in \field{header}) is VIRTIO_CRYPTO_AKCIPHER_ENCRYPT, VIRTIO_CRYPTO_AKCIPHER_DECRYPT,
    VIRTIO_CRYPTO_AKCIPHER_SIGN or VIRTIO_CRYPTO_AKCIPHER_VERIFY then:
    \begin{itemize*}
    \item If VIRTIO_CRYPTO_F_AKCIPHER_STATELESS_MODE is negotiated, \field{op_flf} is
        struct virtio_crypto_akcipher_data_flf_statless, and \field{op_vlf} is struct
        virtio_crypto_akcipher_data_vlf_stateless.
    \item If VIRTIO_CRYPTO_F_AKCIPHER_STATELESS_MODE is NOT negotiated, \field{op_flf}
        is struct virtio_crypto_akcipher_data_flf if VIRTIO_CRYPTO_F_REVISION_1 is negotiated
        and struct virtio_crypto_akcipher_data_flf is padded to 48 bytes if NOT negotiated,
        and \field{op_vlf} is struct virtio_crypto_akcipher_data_vlf.
    \end{itemize*}
\end{itemize*}

\field{inhdr} is a unified input header that used to return the status of
the operations, is defined as follows:

\begin{lstlisting}
struct virtio_crypto_inhdr {
    u8 status;
};
\end{lstlisting}

\subsubsection{HASH Service Operation}\label{sec:Device Types / Crypto Device / Device Operation / HASH Service Operation}

Session mode HASH service requests are as follows:

\begin{lstlisting}
struct virtio_crypto_hash_data_flf {
    /* length of source data */
    le32 src_data_len;
    /* hash result length */
    le32 hash_result_len;
};

struct virtio_crypto_hash_data_vlf {
    /* Device read only portion */
    /* Source data */
    u8 src_data[src_data_len];

    /* Device write only portion */
    /* Hash result data */
    u8 hash_result[hash_result_len];
};
\end{lstlisting}

Each data request uses the virtio_crypto_hash_data_flf structure and the
virtio_crypto_hash_data_vlf structure to store information used to run the
HASH operations.

\field{src_data} is the source data that will be processed.
\field{src_data_len} is the length of source data.
\field{hash_result} is the result data and \field{hash_result_len} is the length
of it.

Stateless mode HASH service requests are as follows:

\begin{lstlisting}
struct virtio_crypto_hash_data_flf_stateless {
    struct {
        /* See VIRTIO_CRYPTO_HASH_* above */
        le32 algo;
    } sess_para;

    /* length of source data */
    le32 src_data_len;
    /* hash result length */
    le32 hash_result_len;
    le32 reserved;
};
struct virtio_crypto_hash_data_vlf_stateless {
    /* Device read only portion */
    /* Source data */
    u8 src_data[src_data_len];

    /* Device write only portion */
    /* Hash result data */
    u8 hash_result[hash_result_len];
};
\end{lstlisting}

\drivernormative{\paragraph}{HASH Service Operation}{Device Types / Crypto Device / Device Operation / HASH Service Operation}

\begin{itemize*}
\item If the driver uses the session mode, then the driver MUST set \field{session_id}
    in struct virtio_crypto_op_header to a valid value assigned by the device when the
    session was created.
\item If the VIRTIO_CRYPTO_F_HASH_STATELESS_MODE feature bit is negotiated, 1) if the
    driver uses the stateless mode, then the driver MUST set the \field{flag} field in
    struct virtio_crypto_op_header to ZERO and MUST set the fields in struct
    virtio_crypto_hash_data_flf_stateless.sess_para, 2) if the driver uses the session
    mode, then the driver MUST set the \field{flag} field in struct virtio_crypto_op_header
    to VIRTIO_CRYPTO_FLAG_SESSION_MODE.
\item The driver MUST set \field{opcode} in struct virtio_crypto_op_header to VIRTIO_CRYPTO_HASH.
\end{itemize*}

\devicenormative{\paragraph}{HASH Service Operation}{Device Types / Crypto Device / Device Operation / HASH Service Operation}

\begin{itemize*}
\item The device MUST use the corresponding structure according to the \field{opcode}
    in the data general header.
\item If the VIRTIO_CRYPTO_F_HASH_STATELESS_MODE feature bit is negotiated, the device
    MUST parse \field{flag} field in struct virtio_crypto_op_header in order to decide
    which mode the driver uses.
\item The device MUST copy the results of HASH operations in the hash_result[] if HASH
    operations success.
\item The device MUST set \field{status} in struct virtio_crypto_inhdr to one of the
    following values of enum VIRTIO_CRYPTO_STATUS:
\begin{itemize*}
\item VIRTIO_CRYPTO_OK if the operation success.
\item VIRTIO_CRYPTO_NOTSUPP if the requested algorithm or operation is unsupported.
\item VIRTIO_CRYPTO_INVSESS if the session ID invalid when in session mode.
\item VIRTIO_CRYPTO_ERR if any failure not mentioned above occurs.
\end{itemize*}
\end{itemize*}


\subsubsection{MAC Service Operation}\label{sec:Device Types / Crypto Device / Device Operation / MAC Service Operation}

Session mode MAC service requests are as follows:

\begin{lstlisting}
struct virtio_crypto_mac_data_flf {
    struct virtio_crypto_hash_data_flf hdr;
};

struct virtio_crypto_mac_data_vlf {
    /* Device read only portion */
    /* Source data */
    u8 src_data[src_data_len];

    /* Device write only portion */
    /* Hash result data */
    u8 hash_result[hash_result_len];
};
\end{lstlisting}

Each request uses the virtio_crypto_mac_data_flf structure and the
virtio_crypto_mac_data_vlf structure to store information used to run the
MAC operations.

\field{src_data} is the source data that will be processed.
\field{src_data_len} is the length of source data.
\field{hash_result} is the result data and \field{hash_result_len} is the length
of it.

Stateless mode MAC service requests are as follows:

\begin{lstlisting}
struct virtio_crypto_mac_data_flf_stateless {
    struct {
        /* See VIRTIO_CRYPTO_MAC_* above */
        le32 algo;
        /* length of authenticated key */
        le32 auth_key_len;
    } sess_para;

    /* length of source data */
    le32 src_data_len;
    /* hash result length */
    le32 hash_result_len;
};

struct virtio_crypto_mac_data_vlf_stateless {
    /* Device read only portion */
    /* The authenticated key */
    u8 auth_key[auth_key_len];
    /* Source data */
    u8 src_data[src_data_len];

    /* Device write only portion */
    /* Hash result data */
    u8 hash_result[hash_result_len];
};
\end{lstlisting}

\field{auth_key} is the authenticated key that will be used during the process.
\field{auth_key_len} is the length of the key.

\drivernormative{\paragraph}{MAC Service Operation}{Device Types / Crypto Device / Device Operation / MAC Service Operation}

\begin{itemize*}
\item If the driver uses the session mode, then the driver MUST set \field{session_id}
    in struct virtio_crypto_op_header to a valid value assigned by the device when the
    session was created.
\item If the VIRTIO_CRYPTO_F_MAC_STATELESS_MODE feature bit is negotiated, 1) if the
    driver uses the stateless mode, then the driver MUST set the \field{flag} field
    in struct virtio_crypto_op_header to ZERO and MUST set the fields in struct
    virtio_crypto_mac_data_flf_stateless.sess_para, 2) if the driver uses the session
    mode, then the driver MUST set the \field{flag} field in struct virtio_crypto_op_header
    to VIRTIO_CRYPTO_FLAG_SESSION_MODE.
\item The driver MUST set \field{opcode} in struct virtio_crypto_op_header to VIRTIO_CRYPTO_MAC.
\end{itemize*}

\devicenormative{\paragraph}{MAC Service Operation}{Device Types / Crypto Device / Device Operation / MAC Service Operation}

\begin{itemize*}
\item If the VIRTIO_CRYPTO_F_MAC_STATELESS_MODE feature bit is negotiated, the device
    MUST parse \field{flag} field in struct virtio_crypto_op_header in order to decide
	which mode the driver uses.
\item The device MUST copy the results of MAC operations in the hash_result[] if HASH
    operations success.
\item The device MUST set \field{status} in struct virtio_crypto_inhdr to one of the
    following values of enum VIRTIO_CRYPTO_STATUS:
\begin{itemize*}
\item VIRTIO_CRYPTO_OK if the operation success.
\item VIRTIO_CRYPTO_NOTSUPP if the requested algorithm or operation is unsupported.
\item VIRTIO_CRYPTO_INVSESS if the session ID invalid when in session mode.
\item VIRTIO_CRYPTO_ERR if any failure not mentioned above occurs.
\end{itemize*}
\end{itemize*}

\subsubsection{Symmetric algorithms Operation}\label{sec:Device Types / Crypto Device / Device Operation / Symmetric algorithms Operation}

Session mode CIPHER service requests are as follows:

\begin{lstlisting}
struct virtio_crypto_cipher_data_flf {
    /*
     * Byte Length of valid IV/Counter data pointed to by the below iv data.
     *
     * For block ciphers in CBC or F8 mode, or for Kasumi in F8 mode, or for
     *   SNOW3G in UEA2 mode, this is the length of the IV (which
     *   must be the same as the block length of the cipher).
     * For block ciphers in CTR mode, this is the length of the counter
     *   (which must be the same as the block length of the cipher).
     */
    le32 iv_len;
    /* length of source data */
    le32 src_data_len;
    /* length of destination data */
    le32 dst_data_len;
    le32 padding;
};

struct virtio_crypto_cipher_data_vlf {
    /* Device read only portion */

    /*
     * Initialization Vector or Counter data.
     *
     * For block ciphers in CBC or F8 mode, or for Kasumi in F8 mode, or for
     *   SNOW3G in UEA2 mode, this is the Initialization Vector (IV)
     *   value.
     * For block ciphers in CTR mode, this is the counter.
     * For AES-XTS, this is the 128bit tweak, i, from IEEE Std 1619-2007.
     *
     * The IV/Counter will be updated after every partial cryptographic
     * operation.
     */
    u8 iv[iv_len];
    /* Source data */
    u8 src_data[src_data_len];

    /* Device write only portion */
    /* Destination data */
    u8 dst_data[dst_data_len];
};
\end{lstlisting}

Session mode requests of algorithm chaining are as follows:

\begin{lstlisting}
struct virtio_crypto_alg_chain_data_flf {
    le32 iv_len;
    /* Length of source data */
    le32 src_data_len;
    /* Length of destination data */
    le32 dst_data_len;
    /* Starting point for cipher processing in source data */
    le32 cipher_start_src_offset;
    /* Length of the source data that the cipher will be computed on */
    le32 len_to_cipher;
    /* Starting point for hash processing in source data */
    le32 hash_start_src_offset;
    /* Length of the source data that the hash will be computed on */
    le32 len_to_hash;
    /* Length of the additional auth data */
    le32 aad_len;
    /* Length of the hash result */
    le32 hash_result_len;
    le32 reserved;
};

struct virtio_crypto_alg_chain_data_vlf {
    /* Device read only portion */

    /* Initialization Vector or Counter data */
    u8 iv[iv_len];
    /* Source data */
    u8 src_data[src_data_len];
    /* Additional authenticated data if exists */
    u8 aad[aad_len];

    /* Device write only portion */

    /* Destination data */
    u8 dst_data[dst_data_len];
    /* Hash result data */
    u8 hash_result[hash_result_len];
};
\end{lstlisting}

Session mode requests of symmetric algorithm are as follows:

\begin{lstlisting}
struct virtio_crypto_sym_data_flf {
    /* Device read only portion */

#define VIRTIO_CRYPTO_SYM_DATA_REQ_HDR_SIZE    40
    u8 op_type_flf[VIRTIO_CRYPTO_SYM_DATA_REQ_HDR_SIZE];

    /* See above VIRTIO_CRYPTO_SYM_OP_* */
    le32 op_type;
    le32 padding;
};

struct virtio_crypto_sym_data_vlf {
    u8 op_type_vlf[sym_para_len];
};
\end{lstlisting}

Each request uses the virtio_crypto_sym_data_flf structure and the
virtio_crypto_sym_data_flf structure to store information used to run the
CIPHER operations.

\field{op_type_flf} is the \field{op_type} specific header, it MUST starts
with or be one of the following structures:
\begin{itemize*}
\item struct virtio_crypto_cipher_data_flf
\item struct virtio_crypto_alg_chain_data_flf
\end{itemize*}

The length of \field{op_type_flf} is fixed to 40 bytes, the data of unused
part (if has) will be ignored.

\field{op_type_vlf} is the \field{op_type} specific parameters, it MUST starts
with or be one of the following structures:
\begin{itemize*}
\item struct virtio_crypto_cipher_data_vlf
\item struct virtio_crypto_alg_chain_data_vlf
\end{itemize*}

\field{sym_para_len} is the size of the specific structure used.

Stateless mode CIPHER service requests are as follows:

\begin{lstlisting}
struct virtio_crypto_cipher_data_flf_stateless {
    struct {
        /* See VIRTIO_CRYPTO_CIPHER* above */
        le32 algo;
        /* length of key */
        le32 key_len;

        /* See VIRTIO_CRYPTO_OP_* above */
        le32 op;
    } sess_para;

    /*
     * Byte Length of valid IV/Counter data pointed to by the below iv data.
     */
    le32 iv_len;
    /* length of source data */
    le32 src_data_len;
    /* length of destination data */
    le32 dst_data_len;
};

struct virtio_crypto_cipher_data_vlf_stateless {
    /* Device read only portion */

    /* The cipher key */
    u8 cipher_key[key_len];

    /* Initialization Vector or Counter data. */
    u8 iv[iv_len];
    /* Source data */
    u8 src_data[src_data_len];

    /* Device write only portion */
    /* Destination data */
    u8 dst_data[dst_data_len];
};
\end{lstlisting}

Stateless mode requests of algorithm chaining are as follows:

\begin{lstlisting}
struct virtio_crypto_alg_chain_data_flf_stateless {
    struct {
        /* See VIRTIO_CRYPTO_SYM_ALG_CHAIN_ORDER_* above */
        le32 alg_chain_order;
        /* length of the additional authenticated data in bytes */
        le32 aad_len;

        struct {
            /* See VIRTIO_CRYPTO_CIPHER* above */
            le32 algo;
            /* length of key */
            le32 key_len;
            /* See VIRTIO_CRYPTO_OP_* above */
            le32 op;
        } cipher;

        struct {
            /* See VIRTIO_CRYPTO_HASH_* or VIRTIO_CRYPTO_MAC_* above */
            le32 algo;
            /* length of authenticated key */
            le32 auth_key_len;
            /* See VIRTIO_CRYPTO_SYM_HASH_MODE_* above */
            le32 hash_mode;
        } hash;
    } sess_para;

    le32 iv_len;
    /* Length of source data */
    le32 src_data_len;
    /* Length of destination data */
    le32 dst_data_len;
    /* Starting point for cipher processing in source data */
    le32 cipher_start_src_offset;
    /* Length of the source data that the cipher will be computed on */
    le32 len_to_cipher;
    /* Starting point for hash processing in source data */
    le32 hash_start_src_offset;
    /* Length of the source data that the hash will be computed on */
    le32 len_to_hash;
    /* Length of the additional auth data */
    le32 aad_len;
    /* Length of the hash result */
    le32 hash_result_len;
    le32 reserved;
};

struct virtio_crypto_alg_chain_data_vlf_stateless {
    /* Device read only portion */

    /* The cipher key */
    u8 cipher_key[key_len];
    /* The auth key */
    u8 auth_key[auth_key_len];
    /* Initialization Vector or Counter data */
    u8 iv[iv_len];
    /* Additional authenticated data if exists */
    u8 aad[aad_len];
    /* Source data */
    u8 src_data[src_data_len];

    /* Device write only portion */

    /* Destination data */
    u8 dst_data[dst_data_len];
    /* Hash result data */
    u8 hash_result[hash_result_len];
};
\end{lstlisting}

Stateless mode requests of symmetric algorithm are as follows:

\begin{lstlisting}
struct virtio_crypto_sym_data_flf_stateless {
    /* Device read only portion */
#define VIRTIO_CRYPTO_SYM_DATE_REQ_HDR_STATELESS_SIZE    72
    u8 op_type_flf[VIRTIO_CRYPTO_SYM_DATE_REQ_HDR_STATELESS_SIZE];

    /* Device write only portion */
    /* See above VIRTIO_CRYPTO_SYM_OP_* */
    le32 op_type;
};

struct virtio_crypto_sym_data_vlf_stateless {
    u8 op_type_vlf[sym_para_len];
};
\end{lstlisting}

\field{op_type_flf} is the \field{op_type} specific header, it MUST starts
with or be one of the following structures:
\begin{itemize*}
\item struct virtio_crypto_cipher_data_flf_stateless
\item struct virtio_crypto_alg_chain_data_flf_stateless
\end{itemize*}

The length of \field{op_type_flf} is fixed to 72 bytes, the data of unused
part (if has) will be ignored.

\field{op_type_vlf} is the \field{op_type} specific parameters, it MUST starts
with or be one of the following structures:
\begin{itemize*}
\item struct virtio_crypto_cipher_data_vlf_stateless
\item struct virtio_crypto_alg_chain_data_vlf_stateless
\end{itemize*}

\field{sym_para_len} is the size of the specific structure used.

\drivernormative{\paragraph}{Symmetric algorithms Operation}{Device Types / Crypto Device / Device Operation / Symmetric algorithms Operation}

\begin{itemize*}
\item If the driver uses the session mode, then the driver MUST set \field{session_id}
    in struct virtio_crypto_op_header to a valid value assigned by the device when the
    session was created.
\item If the VIRTIO_CRYPTO_F_CIPHER_STATELESS_MODE feature bit is negotiated, 1) if the
    driver uses the stateless mode, then the driver MUST set the \field{flag} field in
    struct virtio_crypto_op_header to ZERO and MUST set the fields in struct
    virtio_crypto_cipher_data_flf_stateless.sess_para or struct
    virtio_crypto_alg_chain_data_flf_stateless.sess_para, 2) if the driver uses the
    session mode, then the driver MUST set the \field{flag} field in struct
    virtio_crypto_op_header to VIRTIO_CRYPTO_FLAG_SESSION_MODE.
\item The driver MUST set the \field{opcode} field in struct virtio_crypto_op_header
    to VIRTIO_CRYPTO_CIPHER_ENCRYPT or VIRTIO_CRYPTO_CIPHER_DECRYPT.
\item The driver MUST specify the fields of struct virtio_crypto_cipher_data_flf in
    struct virtio_crypto_sym_data_flf and struct virtio_crypto_cipher_data_vlf in
    struct virtio_crypto_sym_data_vlf if the request is based on VIRTIO_CRYPTO_SYM_OP_CIPHER.
\item The driver MUST specify the fields of struct virtio_crypto_alg_chain_data_flf
    in struct virtio_crypto_sym_data_flf and struct virtio_crypto_alg_chain_data_vlf
    in struct virtio_crypto_sym_data_vlf if the request is of the VIRTIO_CRYPTO_SYM_OP_ALGORITHM_CHAINING
    type.
\end{itemize*}

\devicenormative{\paragraph}{Symmetric algorithms Operation}{Device Types / Crypto Device / Device Operation / Symmetric algorithms Operation}

\begin{itemize*}
\item If the VIRTIO_CRYPTO_F_CIPHER_STATELESS_MODE feature bit is negotiated, the device
    MUST parse \field{flag} field in struct virtio_crypto_op_header in order to decide
	which mode the driver uses.
\item The device MUST parse the virtio_crypto_sym_data_req based on the \field{opcode}
    field in general header.
\item The device MUST parse the fields of struct virtio_crypto_cipher_data_flf in
    struct virtio_crypto_sym_data_flf and struct virtio_crypto_cipher_data_vlf in
    struct virtio_crypto_sym_data_vlf if the request is based on VIRTIO_CRYPTO_SYM_OP_CIPHER.
\item The device MUST parse the fields of struct virtio_crypto_alg_chain_data_flf
    in struct virtio_crypto_sym_data_flf and struct virtio_crypto_alg_chain_data_vlf
    in struct virtio_crypto_sym_data_vlf if the request is of the VIRTIO_CRYPTO_SYM_OP_ALGORITHM_CHAINING
    type.
\item The device MUST copy the result of cryptographic operation in the dst_data[] in
    both plain CIPHER mode and algorithms chain mode.
\item The device MUST check the \field{para}.\field{add_len} is bigger than 0 before
    parse the additional authenticated data in plain algorithms chain mode.
\item The device MUST copy the result of HASH/MAC operation in the hash_result[] is
    of the VIRTIO_CRYPTO_SYM_OP_ALGORITHM_CHAINING type.
\item The device MUST set the \field{status} field in struct virtio_crypto_inhdr to
    one of the following values of enum VIRTIO_CRYPTO_STATUS:
\begin{itemize*}
\item VIRTIO_CRYPTO_OK if the operation success.
\item VIRTIO_CRYPTO_NOTSUPP if the requested algorithm or operation is unsupported.
\item VIRTIO_CRYPTO_INVSESS if the session ID is invalid in session mode.
\item VIRTIO_CRYPTO_ERR if failure not mentioned above occurs.
\end{itemize*}
\end{itemize*}

\subsubsection{AEAD Service Operation}\label{sec:Device Types / Crypto Device / Device Operation / AEAD Service Operation}

Session mode requests of symmetric algorithm are as follows:

\begin{lstlisting}
struct virtio_crypto_aead_data_flf {
    /*
     * Byte Length of valid IV data.
     *
     * For GCM mode, this is either 12 (for 96-bit IVs) or 16, in which
     *   case iv points to J0.
     * For CCM mode, this is the length of the nonce, which can be in the
     *   range 7 to 13 inclusive.
     */
    le32 iv_len;
    /* length of additional auth data */
    le32 aad_len;
    /* length of source data */
    le32 src_data_len;
    /* length of dst data, this should be at least src_data_len + tag_len */
    le32 dst_data_len;
    /* Authentication tag length */
    le32 tag_len;
    le32 reserved;
};

struct virtio_crypto_aead_data_vlf {
    /* Device read only portion */

    /*
     * Initialization Vector data.
     *
     * For GCM mode, this is either the IV (if the length is 96 bits) or J0
     *   (for other sizes), where J0 is as defined by NIST SP800-38D.
     *   Regardless of the IV length, a full 16 bytes needs to be allocated.
     * For CCM mode, the first byte is reserved, and the nonce should be
     *   written starting at &iv[1] (to allow space for the implementation
     *   to write in the flags in the first byte).  Note that a full 16 bytes
     *   should be allocated, even though the iv_len field will have
     *   a value less than this.
     *
     * The IV will be updated after every partial cryptographic operation.
     */
    u8 iv[iv_len];
    /* Source data */
    u8 src_data[src_data_len];
    /* Additional authenticated data if exists */
    u8 aad[aad_len];

    /* Device write only portion */
    /* Pointer to output data */
    u8 dst_data[dst_data_len];
};
\end{lstlisting}

Each request uses the virtio_crypto_aead_data_flf structure and the
virtio_crypto_aead_data_flf structure to store information used to run the
AEAD operations.

Stateless mode AEAD service requests are as follows:

\begin{lstlisting}
struct virtio_crypto_aead_data_flf_stateless {
    struct {
        /* See VIRTIO_CRYPTO_AEAD_* above */
        le32 algo;
        /* length of key */
        le32 key_len;
        /* encrypt or decrypt, See above VIRTIO_CRYPTO_OP_* */
        le32 op;
    } sess_para;

    /* Byte Length of valid IV data. */
    le32 iv_len;
    /* Authentication tag length */
    le32 tag_len;
    /* length of additional auth data */
    le32 aad_len;
    /* length of source data */
    le32 src_data_len;
    /* length of dst data, this should be at least src_data_len + tag_len */
    le32 dst_data_len;
};

struct virtio_crypto_aead_data_vlf_stateless {
    /* Device read only portion */

    /* The cipher key */
    u8 key[key_len];
    /* Initialization Vector data. */
    u8 iv[iv_len];
    /* Source data */
    u8 src_data[src_data_len];
    /* Additional authenticated data if exists */
    u8 aad[aad_len];

    /* Device write only portion */
    /* Pointer to output data */
    u8 dst_data[dst_data_len];
};
\end{lstlisting}

\drivernormative{\paragraph}{AEAD Service Operation}{Device Types / Crypto Device / Device Operation / AEAD Service Operation}

\begin{itemize*}
\item If the driver uses the session mode, then the driver MUST set
    \field{session_id} in struct virtio_crypto_op_header to a valid value assigned
    by the device when the session was created.
\item If the VIRTIO_CRYPTO_F_AEAD_STATELESS_MODE feature bit is negotiated, 1) if
    the driver uses the stateless mode, then the driver MUST set the \field{flag}
    field in struct virtio_crypto_op_header to ZERO and MUST set the fields in
    struct virtio_crypto_aead_data_flf_stateless.sess_para, 2) if the driver uses
    the session mode, then the driver MUST set the \field{flag} field in struct
    virtio_crypto_op_header to VIRTIO_CRYPTO_FLAG_SESSION_MODE.
\item The driver MUST set the \field{opcode} field in struct virtio_crypto_op_header
    to VIRTIO_CRYPTO_AEAD_ENCRYPT or VIRTIO_CRYPTO_AEAD_DECRYPT.
\end{itemize*}

\devicenormative{\paragraph}{AEAD Service Operation}{Device Types / Crypto Device / Device Operation / AEAD Service Operation}

\begin{itemize*}
\item If the VIRTIO_CRYPTO_F_AEAD_STATELESS_MODE feature bit is negotiated, the
    device MUST parse the virtio_crypto_aead_data_vlf_stateless based on the \field{opcode}
	field in general header.
\item The device MUST copy the result of cryptographic operation in the dst_data[].
\item The device MUST copy the authentication tag in the dst_data[] offset the cipher result.
\item The device MUST set the \field{status} field in struct virtio_crypto_inhdr to
    one of the following values of enum VIRTIO_CRYPTO_STATUS:
\item When the \field{opcode} field is VIRTIO_CRYPTO_AEAD_DECRYPT, the device MUST
    verify and return the verification result to the driver.
\begin{itemize*}
\item VIRTIO_CRYPTO_OK if the operation success.
\item VIRTIO_CRYPTO_NOTSUPP if the requested algorithm or operation is unsupported.
\item VIRTIO_CRYPTO_BADMSG if the verification result is incorrect.
\item VIRTIO_CRYPTO_INVSESS if the session ID invalid when in session mode.
\item VIRTIO_CRYPTO_ERR if any failure not mentioned above occurs.
\end{itemize*}
\end{itemize*}

\subsubsection{AKCIPHER Service Operation}\label{sec:Device Types / Crypto Device / Device Operation / AKCIPHER Service Operation}

Session mode AKCIPHER requests are as follows:

\begin{lstlisting}
struct virtio_crypto_akcipher_data_flf {
    /* length of source data */
    le32 src_data_len;
    /* length of dst data */
    le32 dst_data_len;
};

struct virtio_crypto_akcipher_data_vlf {
    /* Device read only portion */
    /* Source data */
    u8 src_data[src_data_len];

    /* Device write only portion */
    /* Pointer to output data */
    u8 dst_data[dst_data_len];
};
\end{lstlisting}

Each data request uses the virtio_crypto_akcipher_flf structure and the virtio_crypto_akcipher_data_vlf
structure to store information used to run the AKCIPHER operations.

For encryption, decryption, and signing:
\field{src_data} is the source data that will be processed, note that for signing operations,
src_data stores the data to be signed, which usually is the digest of some data rather than the
data itself.
\field{src_data_len} is the length of source data.
\field{dst_result} is the result data and \field{dst_data_len} is the length of it. Note that the
length of the result is not always exactly equal to dst_data_len, the driver needs to check how
many bytes the device has written and calculate the actual length of the result.

For verification:
\field{src_data_len} refers to the length of the signature, and \field{dst_data_len} refers to
the length of signed data, where the signed data is usually the digest of some data.
\field{src_data} is spliced by the signature and the signed data, the src_data with the lower
address stores the signature, and the higher address stores the signed data.
\field{dst_data} is always empty for verification.

Different algorithms have different signature formats.
For the RSA algorithm, the result is determined by the padding algorithm specified by
\field{padding_algo} in structure virtio_crypto_rsa_session_para.

For the ECDSA algorithm, the signature is composed of the following
ASN.1 structure (see \hyperref[intro:rfc3279]{RFC3279})
and MUST be DER encoded (see \hyperref[intro:rfc6025]{rfc6025}).

\begin{lstlisting}
Ecdsa-Sig-Value ::= SEQUENCE {
    r INTEGER,
    s INTEGER
}
\end{lstlisting}

Stateless mode AKCIPHER service requests are as follows:
\begin{lstlisting}
struct virtio_crypto_akcipher_data_flf_stateless {
    struct {
        /* See VIRTIO_CYRPTO_AKCIPHER* above */
        le32 algo;
        /* See VIRTIO_CRYPTO_AKCIPHER_KEY_TYPE_* above */
        le32 key_type;
        /* length of key */
        le32 key_len;

        /* algothrim specific parameters described above */
        union {
            struct virtio_crypto_rsa_session_para rsa;
            struct virtio_crypto_ecdsa_session_para ecdsa;
        } u;
    } sess_para;

    /* length of source data */
    le32 src_data_len;
    /* length of destination data */
    le32 dst_data_len;
};

struct virtio_crypto_akcipher_data_vlf_stateless {
    /* Device read only portion */
    u8 akcipher_key[key_len];

    /* Source data */
    u8 src_data[src_data_len];

    /* Device write only portion */
    u8 dst_data[dst_data_len];
};
\end{lstlisting}

In stateless mode, the format of key and signature, the meaning of src_data and dst_data, are all the same
with session mode.

\drivernormative{\paragraph}{AKCIPHER Service Operation}{Device Types / Crypto Device / Device Operation / AKCIPHER Service Operation}

\begin{itemize*}
\item If the driver uses the session mode, then the driver MUST set
    \field{session_id} in struct virtio_crypto_op_header to a valid
    value assigned by the device when the session was created.
\item If the VIRTIO_CRYPTO_F_AKCIPHER_STATELESS_MODE feature bit is negotiated, 1) if the
    driver uses the stateless mode, then the driver MUST set the \field{flag} field in
    struct virtio_crypto_op_header to ZERO and MUST set the fields in struct
    virtio_crypto_akcipher_flf_stateless.sess_para, 2) if the driver uses the session
    mode, then the driver MUST set the \field{flag} field in struct virtio_crypto_op_header
    to VIRTIO_CRYPTO_FLAG_SESSION_MODE.
\item The driver MUST set the \field{opcode} field in struct virtio_crypto_op_header
    to one of VIRTIO_CRYPTO_AKCIPHER_ENCRYPT, VIRTIO_CRYPTO_AKCIPHER_DESTROY_SESSION,
    VIRTIO_CRYPTO_AKCIPHER_SIGN, and VIRTIO_CRYPTO_AKCIPHER_VERIFY.
\end{itemize*}

\devicenormative{\paragraph}{AKCIPHER Service Operation}{Device Types / Crypto Device / Device Operation / AKCIPHER Service Operation}

\begin{itemize*}
\item If the VIRTIO_CRYPTO_F_AKCIPHER_STATELESS_MODE feature bit is negotiated, the
    device MUST parse the virtio_crypto_akcipher_data_vlf_stateless based on the \field{opcode}
    field in general header.
\item The device MUST copy the result of cryptographic operation in the dst_data[].
\item The device MUST set the \field{status} field in struct virtio_crypto_inhdr to
    one of the following values of enum VIRTIO_CRYPTO_STATUS:
\begin{itemize*}
\item VIRTIO_CRYPTO_OK if the operation success.
\item VIRTIO_CRYPTO_NOTSUPP if the requested algorithm or operation is unsupported.
\item VIRTIO_CRYPTO_BADMSG if the verification result is incorrect.
\item VIRTIO_CRYPTO_INVSESS if the session ID invalid when in session mode.
\item VIRTIO_CRYPTO_KEY_REJECTED if the signature verification failed.
\item VIRTIO_CRYPTO_ERR if any failure not mentioned above occurs.
\end{itemize*}
\end{itemize*}

\section{Crypto Device}\label{sec:Device Types / Crypto Device}

The virtio crypto device is a virtual cryptography device as well as a
virtual cryptographic accelerator. The virtio crypto device provides the
following crypto services: CIPHER, MAC, HASH, AEAD and AKCIPHER. Virtio crypto
devices have a single control queue and at least one data queue. Crypto
operation requests are placed into a data queue, and serviced by the
device. Some crypto operation requests are only valid in the context of a
session. The role of the control queue is facilitating control operation
requests. Sessions management is realized with control operation
requests.

\subsection{Device ID}\label{sec:Device Types / Crypto Device / Device ID}

20

\subsection{Virtqueues}\label{sec:Device Types / Crypto Device / Virtqueues}

\begin{description}
\item[0] dataq1
\item[\ldots]
\item[N-1] dataqN
\item[N] controlq
\end{description}

N is set by \field{max_dataqueues}.

\subsection{Feature bits}\label{sec:Device Types / Crypto Device / Feature bits}

\begin{description}
\item VIRTIO_CRYPTO_F_REVISION_1 (0) revision 1. Revision 1 has a specific
    request format and other enhancements (which result in some additional
    requirements).
\item VIRTIO_CRYPTO_F_CIPHER_STATELESS_MODE (1) stateless mode requests are
    supported by the CIPHER service.
\item VIRTIO_CRYPTO_F_HASH_STATELESS_MODE (2) stateless mode requests are
    supported by the HASH service.
\item VIRTIO_CRYPTO_F_MAC_STATELESS_MODE (3) stateless mode requests are
    supported by the MAC service.
\item VIRTIO_CRYPTO_F_AEAD_STATELESS_MODE (4) stateless mode requests are
    supported by the AEAD service.
\item VIRTIO_CRYPTO_F_AKCIPHER_STATELESS_MODE (5) stateless mode requests are
    supported by the AKCIPHER service.
\end{description}


\subsubsection{Feature bit requirements}\label{sec:Device Types / Crypto Device / Feature bit requirements}

Some crypto feature bits require other crypto feature bits
(see \ref{drivernormative:Basic Facilities of a Virtio Device / Feature Bits}):

\begin{description}
\item[VIRTIO_CRYPTO_F_CIPHER_STATELESS_MODE] Requires VIRTIO_CRYPTO_F_REVISION_1.
\item[VIRTIO_CRYPTO_F_HASH_STATELESS_MODE] Requires VIRTIO_CRYPTO_F_REVISION_1.
\item[VIRTIO_CRYPTO_F_MAC_STATELESS_MODE] Requires VIRTIO_CRYPTO_F_REVISION_1.
\item[VIRTIO_CRYPTO_F_AEAD_STATELESS_MODE] Requires VIRTIO_CRYPTO_F_REVISION_1.
\item[VIRTIO_CRYPTO_F_AKCIPHER_STATELESS_MODE] Requires VIRTIO_CRYPTO_F_REVISION_1.
\end{description}

\subsection{Supported crypto services}\label{sec:Device Types / Crypto Device / Supported crypto services}

The following crypto services are defined:

\begin{lstlisting}
/* CIPHER (Symmetric Key Cipher) service */
#define VIRTIO_CRYPTO_SERVICE_CIPHER 0
/* HASH service */
#define VIRTIO_CRYPTO_SERVICE_HASH   1
/* MAC (Message Authentication Codes) service */
#define VIRTIO_CRYPTO_SERVICE_MAC    2
/* AEAD (Authenticated Encryption with Associated Data) service */
#define VIRTIO_CRYPTO_SERVICE_AEAD   3
/* AKCIPHER (Asymmetric Key Cipher) service */
#define VIRTIO_CRYPTO_SERVICE_AKCIPHER 4
\end{lstlisting}

The above constants designate bits used to indicate the which of crypto services are
offered by the device as described in, see \ref{sec:Device Types / Crypto Device / Device configuration layout}.

\subsubsection{CIPHER services}\label{sec:Device Types / Crypto Device / Supported crypto services / CIPHER services}

The following CIPHER algorithms are defined:

\begin{lstlisting}
#define VIRTIO_CRYPTO_NO_CIPHER                 0
#define VIRTIO_CRYPTO_CIPHER_ARC4               1
#define VIRTIO_CRYPTO_CIPHER_AES_ECB            2
#define VIRTIO_CRYPTO_CIPHER_AES_CBC            3
#define VIRTIO_CRYPTO_CIPHER_AES_CTR            4
#define VIRTIO_CRYPTO_CIPHER_DES_ECB            5
#define VIRTIO_CRYPTO_CIPHER_DES_CBC            6
#define VIRTIO_CRYPTO_CIPHER_3DES_ECB           7
#define VIRTIO_CRYPTO_CIPHER_3DES_CBC           8
#define VIRTIO_CRYPTO_CIPHER_3DES_CTR           9
#define VIRTIO_CRYPTO_CIPHER_KASUMI_F8          10
#define VIRTIO_CRYPTO_CIPHER_SNOW3G_UEA2        11
#define VIRTIO_CRYPTO_CIPHER_AES_F8             12
#define VIRTIO_CRYPTO_CIPHER_AES_XTS            13
#define VIRTIO_CRYPTO_CIPHER_ZUC_EEA3           14
\end{lstlisting}

The above constants have two usages:
\begin{enumerate}
\item As bit numbers, used to tell the driver which CIPHER algorithms
are supported by the device, see \ref{sec:Device Types / Crypto Device / Device configuration layout}.
\item As values, used to designate the algorithm in (CIPHER type) crypto
operation requests, see \ref{sec:Device Types / Crypto Device / Device Operation / Control Virtqueue / Session operation}.
\end{enumerate}

\subsubsection{HASH services}\label{sec:Device Types / Crypto Device / Supported crypto services / HASH services}

The following HASH algorithms are defined:

\begin{lstlisting}
#define VIRTIO_CRYPTO_NO_HASH            0
#define VIRTIO_CRYPTO_HASH_MD5           1
#define VIRTIO_CRYPTO_HASH_SHA1          2
#define VIRTIO_CRYPTO_HASH_SHA_224       3
#define VIRTIO_CRYPTO_HASH_SHA_256       4
#define VIRTIO_CRYPTO_HASH_SHA_384       5
#define VIRTIO_CRYPTO_HASH_SHA_512       6
#define VIRTIO_CRYPTO_HASH_SHA3_224      7
#define VIRTIO_CRYPTO_HASH_SHA3_256      8
#define VIRTIO_CRYPTO_HASH_SHA3_384      9
#define VIRTIO_CRYPTO_HASH_SHA3_512      10
#define VIRTIO_CRYPTO_HASH_SHA3_SHAKE128      11
#define VIRTIO_CRYPTO_HASH_SHA3_SHAKE256      12
\end{lstlisting}

The above constants have two usages:
\begin{enumerate}
\item As bit numbers, used to tell the driver which HASH algorithms
are supported by the device, see \ref{sec:Device Types / Crypto Device / Device configuration layout}.
\item As values, used to designate the algorithm in (HASH type) crypto
operation requires, see \ref{sec:Device Types / Crypto Device / Device Operation / Control Virtqueue / Session operation}.
\end{enumerate}

\subsubsection{MAC services}\label{sec:Device Types / Crypto Device / Supported crypto services / MAC services}

The following MAC algorithms are defined:

\begin{lstlisting}
#define VIRTIO_CRYPTO_NO_MAC                       0
#define VIRTIO_CRYPTO_MAC_HMAC_MD5                 1
#define VIRTIO_CRYPTO_MAC_HMAC_SHA1                2
#define VIRTIO_CRYPTO_MAC_HMAC_SHA_224             3
#define VIRTIO_CRYPTO_MAC_HMAC_SHA_256             4
#define VIRTIO_CRYPTO_MAC_HMAC_SHA_384             5
#define VIRTIO_CRYPTO_MAC_HMAC_SHA_512             6
#define VIRTIO_CRYPTO_MAC_CMAC_3DES                25
#define VIRTIO_CRYPTO_MAC_CMAC_AES                 26
#define VIRTIO_CRYPTO_MAC_KASUMI_F9                27
#define VIRTIO_CRYPTO_MAC_SNOW3G_UIA2              28
#define VIRTIO_CRYPTO_MAC_GMAC_AES                 41
#define VIRTIO_CRYPTO_MAC_GMAC_TWOFISH             42
#define VIRTIO_CRYPTO_MAC_CBCMAC_AES               49
#define VIRTIO_CRYPTO_MAC_CBCMAC_KASUMI_F9         50
#define VIRTIO_CRYPTO_MAC_XCBC_AES                 53
#define VIRTIO_CRYPTO_MAC_ZUC_EIA3                 54
\end{lstlisting}

The above constants have two usages:
\begin{enumerate}
\item As bit numbers, used to tell the driver which MAC algorithms
are supported by the device, see \ref{sec:Device Types / Crypto Device / Device configuration layout}.
\item As values, used to designate the algorithm in (MAC type) crypto
operation requests, see \ref{sec:Device Types / Crypto Device / Device Operation / Control Virtqueue / Session operation}.
\end{enumerate}

\subsubsection{AEAD services}\label{sec:Device Types / Crypto Device / Supported crypto services / AEAD services}

The following AEAD algorithms are defined:

\begin{lstlisting}
#define VIRTIO_CRYPTO_NO_AEAD     0
#define VIRTIO_CRYPTO_AEAD_GCM    1
#define VIRTIO_CRYPTO_AEAD_CCM    2
#define VIRTIO_CRYPTO_AEAD_CHACHA20_POLY1305  3
\end{lstlisting}

The above constants have two usages:
\begin{enumerate}
\item As bit numbers, used to tell the driver which AEAD algorithms
are supported by the device, see \ref{sec:Device Types / Crypto Device / Device configuration layout}.
\item As values, used to designate the algorithm in (DEAD type) crypto
operation requests, see \ref{sec:Device Types / Crypto Device / Device Operation / Control Virtqueue / Session operation}.
\end{enumerate}

\subsubsection{AKCIPHER services}\label{sec: Device Types / Crypto Device / Supported crypto services / AKCIPHER services}

The following AKCIPHER algorithms are defined:
\begin{lstlisting}
#define VIRTIO_CRYPTO_NO_AKCIPHER 0
#define VIRTIO_CRYPTO_AKCIPHER_RSA   1
#define VIRTIO_CRYPTO_AKCIPHER_ECDSA 2
\end{lstlisting}

The above constants have two usages:
\begin{enumerate}
\item As bit numbers, used to tell the driver which AKCIPHER algorithms
are supported by the device, see \ref{sec:Device Types / Crypto Device / Device configuration layout}.
\item As values, used to designate the algorithm in asymmetric crypto operation requests,
see \ref{sec:Device Types / Crypto Device / Device Operation / Control Virtqueue / Session operation}.
\end{enumerate}


\subsection{Device configuration layout}\label{sec:Device Types / Crypto Device / Device configuration layout}

Crypto device configuration uses the following layout structure:

\begin{lstlisting}
struct virtio_crypto_config {
    le32 status;
    le32 max_dataqueues;
    le32 crypto_services;
    /* Detailed algorithms mask */
    le32 cipher_algo_l;
    le32 cipher_algo_h;
    le32 hash_algo;
    le32 mac_algo_l;
    le32 mac_algo_h;
    le32 aead_algo;
    /* Maximum length of cipher key in bytes */
    le32 max_cipher_key_len;
    /* Maximum length of authenticated key in bytes */
    le32 max_auth_key_len;
    le32 akcipher_algo;
    /* Maximum size of each crypto request's content in bytes */
    le64 max_size;
};
\end{lstlisting}

\begin{description}
\item Currently, only one \field{status} bit is defined: VIRTIO_CRYPTO_S_HW_READY
    set indicates that the device is ready to process requests, this bit is read-only
    for the driver
\begin{lstlisting}
#define VIRTIO_CRYPTO_S_HW_READY  (1 << 0)
\end{lstlisting}

\item [\field{max_dataqueues}] is the maximum number of data virtqueues that can
    be configured by the device. The driver MAY use only one data queue, or it
    can use more to achieve better performance.

\item [\field{crypto_services}] crypto service offered, see \ref{sec:Device Types / Crypto Device / Supported crypto services}.

\item [\field{cipher_algo_l}] CIPHER algorithms bits 0-31, see \ref{sec:Device Types / Crypto Device / Supported crypto services  / CIPHER services}.

\item [\field{cipher_algo_h}] CIPHER algorithms bits 32-63, see \ref{sec:Device Types / Crypto Device / Supported crypto services  / CIPHER services}.

\item [\field{hash_algo}] HASH algorithms bits, see \ref{sec:Device Types / Crypto Device / Supported crypto services  / HASH services}.

\item [\field{mac_algo_l}] MAC algorithms bits 0-31, see \ref{sec:Device Types / Crypto Device / Supported crypto services  / MAC services}.

\item [\field{mac_algo_h}] MAC algorithms bits 32-63, see \ref{sec:Device Types / Crypto Device / Supported crypto services  / MAC services}.

\item [\field{aead_algo}] AEAD algorithms bits, see \ref{sec:Device Types / Crypto Device / Supported crypto services  / AEAD services}.

\item [\field{max_cipher_key_len}] is the maximum length of cipher key supported by the device.

\item [\field{max_auth_key_len}] is the maximum length of authenticated key supported by the device.

\item [\field{akcipher_algo}] AKCIPHER algorithms bit 0-31, see \ref{sec: Device Types / Crypto Device / Supported crypto services / AKCIPHER services}.

\item [\field{max_size}] is the maximum size of the variable-length parameters of
    data operation of each crypto request's content supported by the device.
\end{description}

\begin{note}
Unless explicitly stated otherwise all lengths and sizes are in bytes.
\end{note}

\devicenormative{\subsubsection}{Device configuration layout}{Device Types / Crypto Device / Device configuration layout}

\begin{itemize*}
\item The device MUST set \field{max_dataqueues} to between 1 and 65535 inclusive.
\item The device MUST set the \field{status} with valid flags, undefined flags MUST NOT be set.
\item The device MUST accept and handle requests after \field{status} is set to VIRTIO_CRYPTO_S_HW_READY.
\item The device MUST set \field{crypto_services} based on the crypto services the device offers.
\item The device MUST set detailed algorithms masks for each service advertised by \field{crypto_services}.
    The device MUST NOT set the not defined algorithms bits.
\item The device MUST set \field{max_size} to show the maximum size of crypto request the device supports.
\item The device MUST set \field{max_cipher_key_len} to show the maximum length of cipher key if the
    device supports CIPHER service.
\item The device MUST set \field{max_auth_key_len} to show the maximum length of authenticated key if
    the device supports MAC service.
\end{itemize*}

\drivernormative{\subsubsection}{Device configuration layout}{Device Types / Crypto Device / Device configuration layout}

\begin{itemize*}
\item The driver MUST read the \field{status} from the bottom bit of status to check whether the
    VIRTIO_CRYPTO_S_HW_READY is set, and the driver MUST reread it after device reset.
\item The driver MUST NOT transmit any requests to the device if the VIRTIO_CRYPTO_S_HW_READY is not set.
\item The driver MUST read \field{max_dataqueues} field to discover the number of data queues the device supports.
\item The driver MUST read \field{crypto_services} field to discover which services the device is able to offer.
\item The driver SHOULD ignore the not defined algorithms bits.
\item The driver MUST read the detailed algorithms fields based on \field{crypto_services} field.
\item The driver SHOULD read \field{max_size} to discover the maximum size of the variable-length
    parameters of data operation of the crypto request's content the device supports and MUST
    guarantee that the size of each crypto request's content is within the \field{max_size}, otherwise
    the request will fail and the driver MUST reset the device.
\item The driver SHOULD read \field{max_cipher_key_len} to discover the maximum length of cipher key
    the device supports and MUST guarantee that the \field{key_len} (CIPHER service or AEAD service) is within
    the \field{max_cipher_key_len} of the device configuration, otherwise the request will fail.
\item The driver SHOULD read \field{max_auth_key_len} to discover the maximum length of authenticated
    key the device supports and MUST guarantee that the \field{auth_key_len} (MAC service) is within the
    \field{max_auth_key_len} of the device configuration, otherwise the request will fail.
\end{itemize*}

\subsection{Device Initialization}\label{sec:Device Types / Crypto Device / Device Initialization}

\drivernormative{\subsubsection}{Device Initialization}{Device Types / Crypto Device / Device Initialization}

\begin{itemize*}
\item The driver MUST configure and initialize all virtqueues.
\item The driver MUST read the supported crypto services from bits of \field{crypto_services}.
\item The driver MUST read the supported algorithms based on \field{crypto_services} field.
\end{itemize*}

\subsection{Device Operation}\label{sec:Device Types / Crypto Device / Device Operation}

The operation of a virtio crypto device is driven by requests placed on the virtqueues.
Requests consist of a queue-type specific header (specifying among others the operation)
and an operation specific payload.

If VIRTIO_CRYPTO_F_REVISION_1 is negotiated the device may support both session mode
(See \ref{sec:Device Types / Crypto Device / Device Operation / Control Virtqueue / Session operation})
and stateless mode operation requests.
In stateless mode all operation parameters are supplied as a part of each request,
while in session mode, some or all operation parameters are managed within the
session. Stateless mode is guarded by feature bits 0-4 on a service level. If
stateless mode is negotiated for a service, the service accepts both session
mode and stateless requests; otherwise stateless mode requests are rejected
(via operation status).

\subsubsection{Operation Status}\label{sec:Device Types / Crypto Device / Device Operation / Operation status}
The device MUST return a status code as part of the operation (both session
operation and service operation) result. The valid operation status as follows:

\begin{lstlisting}
enum VIRTIO_CRYPTO_STATUS {
    VIRTIO_CRYPTO_OK = 0,
    VIRTIO_CRYPTO_ERR = 1,
    VIRTIO_CRYPTO_BADMSG = 2,
    VIRTIO_CRYPTO_NOTSUPP = 3,
    VIRTIO_CRYPTO_INVSESS = 4,
    VIRTIO_CRYPTO_NOSPC = 5,
    VIRTIO_CRYPTO_KEY_REJECTED = 6,
    VIRTIO_CRYPTO_MAX
};
\end{lstlisting}

\begin{itemize*}
\item VIRTIO_CRYPTO_OK: success.
\item VIRTIO_CRYPTO_BADMSG: authentication failed (only when AEAD decryption).
\item VIRTIO_CRYPTO_NOTSUPP: operation or algorithm is unsupported.
\item VIRTIO_CRYPTO_INVSESS: invalid session ID when executing crypto operations.
\item VIRTIO_CRYPTO_NOSPC: no free session ID (only when the VIRTIO_CRYPTO_F_REVISION_1
    feature bit is negotiated).
\item VIRTIO_CRYPTO_KEY_REJECTED: signature verification failed (only when AKCIPHER verification).
\item VIRTIO_CRYPTO_ERR: any failure not mentioned above occurs.
\end{itemize*}

\subsubsection{Control Virtqueue}\label{sec:Device Types / Crypto Device / Device Operation / Control Virtqueue}

The driver uses the control virtqueue to send control commands to the
device, such as session operations (See \ref{sec:Device Types / Crypto Device / Device
Operation / Control Virtqueue / Session operation}).

The header for controlq is of the following form:
\begin{lstlisting}
#define VIRTIO_CRYPTO_OPCODE(service, op)   (((service) << 8) | (op))

struct virtio_crypto_ctrl_header {
#define VIRTIO_CRYPTO_CIPHER_CREATE_SESSION \
       VIRTIO_CRYPTO_OPCODE(VIRTIO_CRYPTO_SERVICE_CIPHER, 0x02)
#define VIRTIO_CRYPTO_CIPHER_DESTROY_SESSION \
       VIRTIO_CRYPTO_OPCODE(VIRTIO_CRYPTO_SERVICE_CIPHER, 0x03)
#define VIRTIO_CRYPTO_HASH_CREATE_SESSION \
       VIRTIO_CRYPTO_OPCODE(VIRTIO_CRYPTO_SERVICE_HASH, 0x02)
#define VIRTIO_CRYPTO_HASH_DESTROY_SESSION \
       VIRTIO_CRYPTO_OPCODE(VIRTIO_CRYPTO_SERVICE_HASH, 0x03)
#define VIRTIO_CRYPTO_MAC_CREATE_SESSION \
       VIRTIO_CRYPTO_OPCODE(VIRTIO_CRYPTO_SERVICE_MAC, 0x02)
#define VIRTIO_CRYPTO_MAC_DESTROY_SESSION \
       VIRTIO_CRYPTO_OPCODE(VIRTIO_CRYPTO_SERVICE_MAC, 0x03)
#define VIRTIO_CRYPTO_AEAD_CREATE_SESSION \
       VIRTIO_CRYPTO_OPCODE(VIRTIO_CRYPTO_SERVICE_AEAD, 0x02)
#define VIRTIO_CRYPTO_AEAD_DESTROY_SESSION \
       VIRTIO_CRYPTO_OPCODE(VIRTIO_CRYPTO_SERVICE_AEAD, 0x03)
#define VIRTIO_CRYPTO_AKCIPHER_CREATE_SESSION \
       VIRTIO_CRYPTO_OPCODE(VIRTIO_CRYPTO_SERVICE_AKCIPHER, 0x04)
#define VIRTIO_CRYPTO_AKCIPHER_DESTROY_SESSION \
       VIRTIO_CRYPTO_OPCDE(VIRTIO_CRYPTO_SERVICE_AKCIPHER, 0x05)
    le32 opcode;
    /* algo should be service-specific algorithms */
    le32 algo;
    le32 flag;
    le32 reserved;
};
\end{lstlisting}

The controlq request is composed of four parts:
\begin{lstlisting}
struct virtio_crypto_op_ctrl_req {
    /* Device read only portion */

    struct virtio_crypto_ctrl_header header;

#define VIRTIO_CRYPTO_CTRLQ_OP_SPEC_HDR_LEGACY 56
    /* fixed length fields, opcode specific */
    u8 op_flf[flf_len];

    /* variable length fields, opcode specific */
    u8 op_vlf[vlf_len];

    /* Device write only portion */

    /* op result or completion status */
    u8 op_outcome[outcome_len];
};
\end{lstlisting}

\field{header} is a general header (see above).

\field{op_flf} is the opcode (in \field{header}) specific fixed-length parameters.

\field{flf_len} depends on the VIRTIO_CRYPTO_F_REVISION_1 feature bit (see below).

\field{op_vlf} is the opcode (in \field{header}) specific variable-length parameters.

\field{vlf_len} is the size of the specific structure used.
\begin{note}
The \field{vlf_len} of session-destroy operation and the hash-session-create
operation is ZERO.
\end{note}

\begin{itemize*}
\item If the opcode (in \field{header}) is VIRTIO_CRYPTO_CIPHER_CREATE_SESSION
    then \field{op_flf} is struct virtio_crypto_sym_create_session_flf if
    VIRTIO_CRYPTO_F_REVISION_1 is negotiated and struct virtio_crypto_sym_create_session_flf is
    padded to 56 bytes if NOT negotiated, and \field{op_vlf} is struct
    virtio_crypto_sym_create_session_vlf.
\item If the opcode (in \field{header}) is VIRTIO_CRYPTO_HASH_CREATE_SESSION
    then \field{op_flf} is struct virtio_crypto_hash_create_session_flf if
    VIRTIO_CRYPTO_F_REVISION_1 is negotiated and struct virtio_crypto_hash_create_session_flf is
    padded to 56 bytes if NOT negotiated.
\item If the opcode (in \field{header}) is VIRTIO_CRYPTO_MAC_CREATE_SESSION
    then \field{op_flf} is struct virtio_crypto_mac_create_session_flf if
    VIRTIO_CRYPTO_F_REVISION_1 is negotiated and struct virtio_crypto_mac_create_session_flf is
    padded to 56 bytes if NOT negotiated, and \field{op_vlf} is struct
    virtio_crypto_mac_create_session_vlf.
\item If the opcode (in \field{header}) is VIRTIO_CRYPTO_AEAD_CREATE_SESSION
    then \field{op_flf} is struct virtio_crypto_aead_create_session_flf if
    VIRTIO_CRYPTO_F_REVISION_1 is negotiated and struct virtio_crypto_aead_create_session_flf is
    padded to 56 bytes if NOT negotiated, and \field{op_vlf} is struct
    virtio_crypto_aead_create_session_vlf.
\item If the opcode (in \field{header}) is VIRTIO_CRYPTO_AKCIPHER_CREATE_SESSION
    then \field{op_flf} is struct virtio_crypto_akcipher_create_session_flf if
    VIRTIO_CRYPTO_F_REVISION_1 is negotiated and struct virtio_crypto_akcipher_create_session_flf is
    padded to 56 bytes if NOT negotiated, and \field{op_vlf} is struct
    virtio_crypto_akcipher_create_session_vlf.
\item If the opcode (in \field{header}) is VIRTIO_CRYPTO_CIPHER_DESTROY_SESSION
    or VIRTIO_CRYPTO_HASH_DESTROY_SESSION or VIRTIO_CRYPTO_MAC_DESTROY_SESSION or
    VIRTIO_CRYPTO_AEAD_DESTROY_SESSION then \field{op_flf} is struct
    virtio_crypto_destroy_session_flf if VIRTIO_CRYPTO_F_REVISION_1 is negotiated and
    struct virtio_crypto_destroy_session_flf is padded to 56 bytes if NOT negotiated.
\end{itemize*}

\field{op_outcome} stores the result of operation and must be struct
virtio_crypto_destroy_session_input for destroy session or
struct virtio_crypto_create_session_input for create session.

\field{outcome_len} is the size of the structure used.


\paragraph{Session operation}\label{sec:Device Types / Crypto Device / Device
Operation / Control Virtqueue / Session operation}

The session is a handle which describes the cryptographic parameters to be
applied to a number of buffers.

The following structure stores the result of session creation set by the device:

\begin{lstlisting}
struct virtio_crypto_create_session_input {
    le64 session_id;
    le32 status;
    le32 padding;
};
\end{lstlisting}

A request to destroy a session includes the following information:

\begin{lstlisting}
struct virtio_crypto_destroy_session_flf {
    /* Device read only portion */
    le64  session_id;
};

struct virtio_crypto_destroy_session_input {
    /* Device write only portion */
    u8  status;
};
\end{lstlisting}


\subparagraph{Session operation: HASH session}\label{sec:Device Types / Crypto Device / Device
Operation / Control Virtqueue / Session operation / Session operation: HASH session}

The fixed-length parameters of HASH session requests is as follows:

\begin{lstlisting}
struct virtio_crypto_hash_create_session_flf {
    /* Device read only portion */

    /* See VIRTIO_CRYPTO_HASH_* above */
    le32 algo;
    /* hash result length */
    le32 hash_result_len;
};
\end{lstlisting}


\subparagraph{Session operation: MAC session}\label{sec:Device Types / Crypto Device / Device
Operation / Control Virtqueue / Session operation / Session operation: MAC session}

The fixed-length and the variable-length parameters of MAC session requests are as follows:

\begin{lstlisting}
struct virtio_crypto_mac_create_session_flf {
    /* Device read only portion */

    /* See VIRTIO_CRYPTO_MAC_* above */
    le32 algo;
    /* hash result length */
    le32 hash_result_len;
    /* length of authenticated key */
    le32 auth_key_len;
    le32 padding;
};

struct virtio_crypto_mac_create_session_vlf {
    /* Device read only portion */

    /* The authenticated key */
    u8 auth_key[auth_key_len];
};
\end{lstlisting}

The length of \field{auth_key} is specified in \field{auth_key_len} in the struct
virtio_crypto_mac_create_session_flf.


\subparagraph{Session operation: Symmetric algorithms session}\label{sec:Device Types / Crypto Device / Device
Operation / Control Virtqueue / Session operation / Session operation: Symmetric algorithms session}

The request of symmetric session could be the CIPHER algorithms request
or the chain algorithms (chaining CIPHER and HASH/MAC) request.

The fixed-length and the variable-length parameters of CIPHER session requests are as follows:

\begin{lstlisting}
struct virtio_crypto_cipher_session_flf {
    /* Device read only portion */

    /* See VIRTIO_CRYPTO_CIPHER* above */
    le32 algo;
    /* length of key */
    le32 key_len;
#define VIRTIO_CRYPTO_OP_ENCRYPT  1
#define VIRTIO_CRYPTO_OP_DECRYPT  2
    /* encryption or decryption */
    le32 op;
    le32 padding;
};

struct virtio_crypto_cipher_session_vlf {
    /* Device read only portion */

    /* The cipher key */
    u8 cipher_key[key_len];
};
\end{lstlisting}

The length of \field{cipher_key} is specified in \field{key_len} in the struct
virtio_crypto_cipher_session_flf.

The fixed-length and the variable-length parameters of Chain session requests are as follows:

\begin{lstlisting}
struct virtio_crypto_alg_chain_session_flf {
    /* Device read only portion */

#define VIRTIO_CRYPTO_SYM_ALG_CHAIN_ORDER_HASH_THEN_CIPHER  1
#define VIRTIO_CRYPTO_SYM_ALG_CHAIN_ORDER_CIPHER_THEN_HASH  2
    le32 alg_chain_order;
/* Plain hash */
#define VIRTIO_CRYPTO_SYM_HASH_MODE_PLAIN    1
/* Authenticated hash (mac) */
#define VIRTIO_CRYPTO_SYM_HASH_MODE_AUTH     2
/* Nested hash */
#define VIRTIO_CRYPTO_SYM_HASH_MODE_NESTED   3
    le32 hash_mode;
    struct virtio_crypto_cipher_session_flf cipher_hdr;

#define VIRTIO_CRYPTO_ALG_CHAIN_SESS_OP_SPEC_HDR_SIZE  16
    /* fixed length fields, algo specific */
    u8 algo_flf[VIRTIO_CRYPTO_ALG_CHAIN_SESS_OP_SPEC_HDR_SIZE];

    /* length of the additional authenticated data (AAD) in bytes */
    le32 aad_len;
    le32 padding;
};

struct virtio_crypto_alg_chain_session_vlf {
    /* Device read only portion */

    /* The cipher key */
    u8 cipher_key[key_len];
    /* The authenticated key */
    u8 auth_key[auth_key_len];
};
\end{lstlisting}

\field{hash_mode} decides the type used by \field{algo_flf}.

\field{algo_flf} is fixed to 16 bytes and MUST contains or be one of
the following types:
\begin{itemize*}
\item struct virtio_crypto_hash_create_session_flf
\item struct virtio_crypto_mac_create_session_flf
\end{itemize*}
The data of unused part (if has) in \field{algo_flf} will be ignored.

The length of \field{cipher_key} is specified in \field{key_len} in \field{cipher_hdr}.

The length of \field{auth_key} is specified in \field{auth_key_len} in struct
virtio_crypto_mac_create_session_flf.

The fixed-length parameters of Symmetric session requests are as follows:

\begin{lstlisting}
struct virtio_crypto_sym_create_session_flf {
    /* Device read only portion */

#define VIRTIO_CRYPTO_SYM_SESS_OP_SPEC_HDR_SIZE  48
    /* fixed length fields, opcode specific */
    u8 op_flf[VIRTIO_CRYPTO_SYM_SESS_OP_SPEC_HDR_SIZE];

/* No operation */
#define VIRTIO_CRYPTO_SYM_OP_NONE  0
/* Cipher only operation on the data */
#define VIRTIO_CRYPTO_SYM_OP_CIPHER  1
/* Chain any cipher with any hash or mac operation. The order
   depends on the value of alg_chain_order param */
#define VIRTIO_CRYPTO_SYM_OP_ALGORITHM_CHAINING  2
    le32 op_type;
    le32 padding;
};
\end{lstlisting}

\field{op_flf} is fixed to 48 bytes, MUST contains or be one of
the following types:
\begin{itemize*}
\item struct virtio_crypto_cipher_session_flf
\item struct virtio_crypto_alg_chain_session_flf
\end{itemize*}
The data of unused part (if has) in \field{op_flf} will be ignored.

\field{op_type} decides the type used by \field{op_flf}.

The variable-length parameters of Symmetric session requests are as follows:

\begin{lstlisting}
struct virtio_crypto_sym_create_session_vlf {
    /* Device read only portion */
    /* variable length fields, opcode specific */
    u8 op_vlf[vlf_len];
};
\end{lstlisting}

\field{op_vlf} MUST contains or be one of the following types:
\begin{itemize*}
\item struct virtio_crypto_cipher_session_vlf
\item struct virtio_crypto_alg_chain_session_vlf
\end{itemize*}

\field{op_type} in struct virtio_crypto_sym_create_session_flf decides the
type used by \field{op_vlf}.

\field{vlf_len} is the size of the specific structure used.


\subparagraph{Session operation: AEAD session}\label{sec:Device Types / Crypto Device / Device
Operation / Control Virtqueue / Session operation / Session operation: AEAD session}

The fixed-length and the variable-length parameters of AEAD session requests are as follows:

\begin{lstlisting}
struct virtio_crypto_aead_create_session_flf {
    /* Device read only portion */

    /* See VIRTIO_CRYPTO_AEAD_* above */
    le32 algo;
    /* length of key */
    le32 key_len;
    /* Authentication tag length */
    le32 tag_len;
    /* The length of the additional authenticated data (AAD) in bytes */
    le32 aad_len;
    /* encryption or decryption, See above VIRTIO_CRYPTO_OP_* */
    le32 op;
    le32 padding;
};

struct virtio_crypto_aead_create_session_vlf {
    /* Device read only portion */
    u8 key[key_len];
};
\end{lstlisting}

The length of \field{key} is specified in \field{key_len} in struct
virtio_crypto_aead_create_session_flf.

\subparagraph{Session operation: AKCIPHER session}\label{sec:Device Types / Crypto Device / Device
Operation / Control Virtqueue / Session operation / Session operation: AKCIPHER session}

Due to the complexity of asymmetric key algorithms, different algorithms
require different parameters. The following data structures are used as
supplementary parameters to describe the asymmetric algorithm sessions.

For the RSA algorithm, the extra parameters are as follows:
\begin{lstlisting}
struct virtio_crypto_rsa_session_para {
#define VIRTIO_CRYPTO_RSA_RAW_PADDING   0
#define VIRTIO_CRYPTO_RSA_PKCS1_PADDING 1
    le32 padding_algo;

#define VIRTIO_CRYPTO_RSA_NO_HASH   0
#define VIRTIO_CRYPTO_RSA_MD2       1
#define VIRTIO_CRYPTO_RSA_MD3       2
#define VIRTIO_CRYPTO_RSA_MD4       3
#define VIRTIO_CRYPTO_RSA_MD5       4
#define VIRTIO_CRYPTO_RSA_SHA1      5
#define VIRTIO_CRYPTO_RSA_SHA256    6
#define VIRTIO_CRYPTO_RSA_SHA384    7
#define VIRTIO_CRYPTO_RSA_SHA512    8
#define VIRTIO_CRYPTO_RSA_SHA224    9
    le32 hash_algo;
};
\end{lstlisting}

\field{padding_algo} specifies the padding method used by RSA sessions.
\begin{itemize*}
\item If VIRTIO_CRYPTO_RSA_RAW_PADDING is specified, 1) \field{hash_algo}
is ignored, 2) ciphertext and plaintext MUST be padded with leading zeros,
3) and RSA sessions with VIRTIO_CRYPTO_RSA_RAW_PADDING MUST not be used
for verification and signing operations.
\item If VIRTIO_CRYPTO_RSA_PKCS1_PADDING is specified, EMSA-PKCS1-v1_5 padding method
is used (see \hyperref[intro:rfc3447]{PKCS\#1}), \field{hash_algo} specifies how the
digest of the data passed to RSA sessions is calculated when verifying and signing.
It only affects the padding algorithm and is ignored during encryption and decryption.
\end{itemize*}

The ECC algorithms such as the ECDSA algorithm, cannot use custom curves, only the
following known curves can be used (see \hyperref[intro:NIST]{NIST-recommended curves}).

\begin{lstlisting}
#define VIRTIO_CRYPTO_CURVE_UNKNOWN   0
#define VIRTIO_CRYPTO_CURVE_NIST_P192 1
#define VIRTIO_CRYPTO_CURVE_NIST_P224 2
#define VIRTIO_CRYPTO_CURVE_NIST_P256 3
#define VIRTIO_CRYPTO_CURVE_NIST_P384 4
#define VIRTIO_CRYPTO_CURVE_NIST_P521 5
\end{lstlisting}

For the ECDSA algorithm, the extra parameters are as follows:
\begin{lstlisting}
struct virtio_crypto_ecdsa_session_para {
    /* See VIRTIO_CRYPTO_CURVE_* above */
    le32 curve_id;
};
\end{lstlisting}

The fixed-length and the variable-length parameters of AKCIPHER session requests are as follows:
\begin{lstlisting}
struct virtio_crypto_akcipher_create_session_flf {
    /* Device read only portion */

    /* See VIRTIO_CRYPTO_AKCIPHER_* above */
    le32 algo;
#define VIRTIO_CRYPTO_AKCIPHER_KEY_TYPE_PUBLIC 1
#define VIRTIO_CRYPTO_AKCIPHER_KEY_TYPE_PRIVATE 2
    le32 key_type;
    /* length of key */
    le32 key_len;

#define VIRTIO_CRYPTO_AKCIPHER_SESS_ALGO_SPEC_HDR_SIZE 44
    u8 algo_flf[VIRTIO_CRYPTO_AKCIPHER_SESS_ALGO_SPEC_HDR_SIZE];
};

struct virtio_crypto_akcipher_create_session_vlf {
    /* Device read only portion */
    u8 key[key_len];
};
\end{lstlisting}

\field{algo} decides the type used by \field{algo_flf}.
\field{algo_flf} is fixed to 44 bytes and MUST contains of be one the
following structures:
\begin{itemize*}
\item struct virtio_crypto_rsa_session_para
\item struct virtio_crypto_ecdsa_session_para
\end{itemize*}

The length of \field{key} is specified in \field{key_len} in the struct
virtio_crypto_akcipher_create_session_flf.

For the RSA algorithm, the key needs to be encoded according to
\hyperref[intro:rfc3447]{PKCS\#1}. The private key is described with the
RSAPrivateKey structure, and the public key is described with the RSAPublicKey
structure. These ASN.1 structures are encoded in DER encoding rules (see
\hyperref[intro:rfc6025]{rfc6025}).

\begin{lstlisting}
RSAPrivateKey ::= SEQUENCE {
    version          INTEGER,
    modulus          INTEGER,
    publicExponent   INTEGER,
    privateExponent  INTEGER,
    prime1           INTEGER,
    prime2           INTEGER,
    exponent1        INTEGER,
    exponent1        INTEGER,
    coefficient      INTEGER,
    otherPrimeInfos  OtherPrimeInfos OPTIONAL
}

OtherPrimeInfos ::= SEQUENCE SIZE(1...MAX) OF OtherPrimeInfo

OtherPrimeINfo ::= SEQUENCE {
    prime           INTEGER,
    exponent        INTEGER,
    coefficient     INTEGER
}

RSAPublicKey ::= SEQUENCE {
    modulus         INTEGER,
    publicExponent  INTEGER
}
\end{lstlisting}

For the ECDSA algorithm, the private key is encoded according to
\hyperref[intro:rfc5915]{RFC5915}, the private key of the ECDSA algorithm
is described by the ASN.1 structure ECPrivateKey and encoded with DER
encoding rules (see \hyperref[intro:rfc6025]{rfc6025}).

\begin{lstlisting}
ECPrivateKey ::= SEQUNCE {
    version         INTEGER,
    privateKey      OCTET STRING,
    parameters [0]  ECParameters {{ NamedCurve }} OPTIONAL,
    publicKey  [1]  BIT STRING OPTIONAL
}
\end{lstlisting}

The public key of the ECDSA algorithm is encoded according to \hyperref[intro:SEC1]{SEC1},
and the public key of ECDSA is described by the ASN.1 structure ECPoint.
When initializing a session with ECDSA public key, the ECPoint is DER encoded and the
\field{key} only contains the value part of ECPoint, that is, the header part of the
OCTET STRING will be omitted (see \hyperref[intro:rfc6025]{rfc6025}).

\begin{lstlisting}
ECPoint ::= OCTET STRING
\end{lstlisting}

The length of \field{key} is specified in \field{key_len} in
struct virtio_crypto_akcipher_create_session_flf.

\drivernormative{\subparagraph}{Session operation: create session}{Device Types / Crypto Device / Device
Operation / Control Virtqueue / Session operation / Session operation: create session}

\begin{itemize*}
\item The driver MUST set the \field{opcode} field based on service type: CIPHER, HASH, MAC, AEAD or AKCIPHER.
\item The driver MUST set the control general header, the opcode specific header,
    the opcode specific extra parameters and the opcode specific outcome buffer in turn.
    See \ref{sec:Device Types / Crypto Device / Device Operation / Control Virtqueue}.
\item The driver MUST set the \field{reversed} field to zero.
\end{itemize*}

\devicenormative{\subparagraph}{Session operation: create session}{Device Types / Crypto Device / Device
Operation / Control Virtqueue / Session operation / Session operation: create session}

\begin{itemize*}
\item The device MUST use the corresponding opcode specific structure according to the
    \field{opcode} in the control general header.
\item The device MUST extract extra parameters according to the structures used.
\item The device MUST set the \field{status} field to one of the following values of enum
    VIRTIO_CRYPTO_STATUS after finish a session creation:
\begin{itemize*}
\item VIRTIO_CRYPTO_OK if a session is created successfully.
\item VIRTIO_CRYPTO_NOTSUPP if the requested algorithm or operation is unsupported.
\item VIRTIO_CRYPTO_NOSPC if no free session ID (only when the VIRTIO_CRYPTO_F_REVISION_1
    feature bit is negotiated).
\item VIRTIO_CRYPTO_ERR if failure not mentioned above occurs.
\end{itemize*}
\item The device MUST set the \field{session_id} field to a unique session identifier only
    if the status is set to VIRTIO_CRYPTO_OK.
\end{itemize*}

\drivernormative{\subparagraph}{Session operation: destroy session}{Device Types / Crypto Device / Device
Operation / Control Virtqueue / Session operation / Session operation: destroy session}

\begin{itemize*}
\item The driver MUST set the \field{opcode} field based on service type: CIPHER, HASH, MAC, AEAD or AKCIPHER.
\item The driver MUST set the \field{session_id} to a valid value assigned by the device
    when the session was created.
\end{itemize*}

\devicenormative{\subparagraph}{Session operation: destroy session}{Device Types / Crypto Device / Device
Operation / Control Virtqueue / Session operation / Session operation: destroy session}

\begin{itemize*}
\item The device MUST set the \field{status} field to one of the following values of enum VIRTIO_CRYPTO_STATUS.
\begin{itemize*}
\item VIRTIO_CRYPTO_OK if a session is created successfully.
\item VIRTIO_CRYPTO_ERR if any failure occurs.
\end{itemize*}
\end{itemize*}


\subsubsection{Data Virtqueue}\label{sec:Device Types / Crypto Device / Device Operation / Data Virtqueue}

The driver uses the data virtqueues to transmit crypto operation requests to the device,
and completes the crypto operations.

The header for dataq is as follows:

\begin{lstlisting}
struct virtio_crypto_op_header {
#define VIRTIO_CRYPTO_CIPHER_ENCRYPT \
    VIRTIO_CRYPTO_OPCODE(VIRTIO_CRYPTO_SERVICE_CIPHER, 0x00)
#define VIRTIO_CRYPTO_CIPHER_DECRYPT \
    VIRTIO_CRYPTO_OPCODE(VIRTIO_CRYPTO_SERVICE_CIPHER, 0x01)
#define VIRTIO_CRYPTO_HASH \
    VIRTIO_CRYPTO_OPCODE(VIRTIO_CRYPTO_SERVICE_HASH, 0x00)
#define VIRTIO_CRYPTO_MAC \
    VIRTIO_CRYPTO_OPCODE(VIRTIO_CRYPTO_SERVICE_MAC, 0x00)
#define VIRTIO_CRYPTO_AEAD_ENCRYPT \
    VIRTIO_CRYPTO_OPCODE(VIRTIO_CRYPTO_SERVICE_AEAD, 0x00)
#define VIRTIO_CRYPTO_AEAD_DECRYPT \
    VIRTIO_CRYPTO_OPCODE(VIRTIO_CRYPTO_SERVICE_AEAD, 0x01)
#define VIRTIO_CRYPTO_AKCIPHER_ENCRYPT \
    VIRTIO_CRYPTO_OPCODE(VIRTIO_CRYPTO_SERVICE_AKCIPHER, 0x00)
#define VIRTIO_CRYPTO_AKCIPHER_DECRYPT \
    VIRTIO_CRYPTO_OPCODE(VIRTIO_CRYPTO_SERVICE_AKCIPHER, 0x01)
#define VIRTIO_CRYPTO_AKCIPHER_SIGN \
    VIRTIO_CRYPTO_OPCODE(VIRTIO_CRYPTO_SERVICE_AKCIPHER, 0x02)
#define VIRTIO_CRYPTO_AKCIPHER_VERIFY \
    VIRTIO_CRYPTO_OPCODE(VIRTIO_CRYPTO_SERVICE_AKCIPHER, 0x03)
    le32 opcode;
    /* algo should be service-specific algorithms */
    le32 algo;
    le64 session_id;
#define VIRTIO_CRYPTO_FLAG_SESSION_MODE 1
    /* control flag to control the request */
    le32 flag;
    le32 padding;
};
\end{lstlisting}

\begin{note}
If VIRTIO_CRYPTO_F_REVISION_1 is not negotiated the \field{flag} is ignored.

If VIRTIO_CRYPTO_F_REVISION_1 is negotiated but VIRTIO_CRYPTO_F_<SERVICE>_STATELESS_MODE
is not negotiated, then the device SHOULD reject <SERVICE> requests if
VIRTIO_CRYPTO_FLAG_SESSION_MODE is not set (in \field{flag}).
\end{note}

The dataq request is composed of four parts:
\begin{lstlisting}
struct virtio_crypto_op_data_req {
    /* Device read only portion */

    struct virtio_crypto_op_header header;

#define VIRTIO_CRYPTO_DATAQ_OP_SPEC_HDR_LEGACY 48
    /* fixed length fields, opcode specific */
    u8 op_flf[flf_len];

    /* Device read && write portion */
    /* variable length fields, opcode specific */
    u8 op_vlf[vlf_len];

    /* Device write only portion */
    struct virtio_crypto_inhdr inhdr;
};
\end{lstlisting}

\field{header} is a general header (see above).

\field{op_flf} is the opcode (in \field{header}) specific header.

\field{flf_len} depends on the VIRTIO_CRYPTO_F_REVISION_1 feature bit
(see below).

\field{op_vlf} is the opcode (in \field{header}) specific parameters.

\field{vlf_len} is the size of the specific structure used.

\begin{itemize*}
\item If the the opcode (in \field{header}) is VIRTIO_CRYPTO_CIPHER_ENCRYPT
    or VIRTIO_CRYPTO_CIPHER_DECRYPT then:
    \begin{itemize*}
    \item If VIRTIO_CRYPTO_F_CIPHER_STATELESS_MODE is negotiated, \field{op_flf} is
        struct virtio_crypto_sym_data_flf_stateless, and \field{op_vlf} is struct
        virtio_crypto_sym_data_vlf_stateless.
    \item If VIRTIO_CRYPTO_F_CIPHER_STATELESS_MODE is NOT negotiated, \field{op_flf}
        is struct virtio_crypto_sym_data_flf if VIRTIO_CRYPTO_F_REVISION_1 is negotiated
        and struct virtio_crypto_sym_data_flf is padded to 48 bytes if NOT negotiated,
        and \field{op_vlf} is struct virtio_crypto_sym_data_vlf.
    \end{itemize*}
\item If the the opcode (in \field{header}) is VIRTIO_CRYPTO_HASH:
    \begin{itemize*}
    \item If VIRTIO_CRYPTO_F_HASH_STATELESS_MODE is negotiated, \field{op_flf} is
        struct virtio_crypto_hash_data_flf_stateless, and \field{op_vlf} is struct
        virtio_crypto_hash_data_vlf_stateless.
    \item If VIRTIO_CRYPTO_F_HASH_STATELESS_MODE is NOT negotiated, \field{op_flf}
        is struct virtio_crypto_hash_data_flf if VIRTIO_CRYPTO_F_REVISION_1 is negotiated
        and struct virtio_crypto_hash_data_flf is padded to 48 bytes if NOT negotiated,
        and \field{op_vlf} is struct virtio_crypto_hash_data_vlf.
    \end{itemize*}
\item If the the opcode (in \field{header}) is VIRTIO_CRYPTO_MAC:
    \begin{itemize*}
    \item If VIRTIO_CRYPTO_F_MAC_STATELESS_MODE is negotiated, \field{op_flf} is
        struct virtio_crypto_mac_data_flf_stateless, and \field{op_vlf} is struct
        virtio_crypto_mac_data_vlf_stateless.
    \item If VIRTIO_CRYPTO_F_MAC_STATELESS_MODE is NOT negotiated, \field{op_flf}
        is struct virtio_crypto_mac_data_flf if VIRTIO_CRYPTO_F_REVISION_1 is negotiated
        and struct virtio_crypto_mac_data_flf is padded to 48 bytes if NOT negotiated,
        and \field{op_vlf} is struct virtio_crypto_mac_data_vlf.
    \end{itemize*}
\item If the the opcode (in \field{header}) is VIRTIO_CRYPTO_AEAD_ENCRYPT
    or VIRTIO_CRYPTO_AEAD_DECRYPT then:
    \begin{itemize*}
    \item If VIRTIO_CRYPTO_F_AEAD_STATELESS_MODE is negotiated, \field{op_flf} is
        struct virtio_crypto_aead_data_flf_stateless, and \field{op_vlf} is struct
        virtio_crypto_aead_data_vlf_stateless.
    \item If VIRTIO_CRYPTO_F_AEAD_STATELESS_MODE is NOT negotiated, \field{op_flf}
        is struct virtio_crypto_aead_data_flf if VIRTIO_CRYPTO_F_REVISION_1 is negotiated
        and struct virtio_crypto_aead_data_flf is padded to 48 bytes if NOT negotiated,
        and \field{op_vlf} is struct virtio_crypto_aead_data_vlf.
    \end{itemize*}
\item If the opcode (in \field{header}) is VIRTIO_CRYPTO_AKCIPHER_ENCRYPT, VIRTIO_CRYPTO_AKCIPHER_DECRYPT,
    VIRTIO_CRYPTO_AKCIPHER_SIGN or VIRTIO_CRYPTO_AKCIPHER_VERIFY then:
    \begin{itemize*}
    \item If VIRTIO_CRYPTO_F_AKCIPHER_STATELESS_MODE is negotiated, \field{op_flf} is
        struct virtio_crypto_akcipher_data_flf_statless, and \field{op_vlf} is struct
        virtio_crypto_akcipher_data_vlf_stateless.
    \item If VIRTIO_CRYPTO_F_AKCIPHER_STATELESS_MODE is NOT negotiated, \field{op_flf}
        is struct virtio_crypto_akcipher_data_flf if VIRTIO_CRYPTO_F_REVISION_1 is negotiated
        and struct virtio_crypto_akcipher_data_flf is padded to 48 bytes if NOT negotiated,
        and \field{op_vlf} is struct virtio_crypto_akcipher_data_vlf.
    \end{itemize*}
\end{itemize*}

\field{inhdr} is a unified input header that used to return the status of
the operations, is defined as follows:

\begin{lstlisting}
struct virtio_crypto_inhdr {
    u8 status;
};
\end{lstlisting}

\subsubsection{HASH Service Operation}\label{sec:Device Types / Crypto Device / Device Operation / HASH Service Operation}

Session mode HASH service requests are as follows:

\begin{lstlisting}
struct virtio_crypto_hash_data_flf {
    /* length of source data */
    le32 src_data_len;
    /* hash result length */
    le32 hash_result_len;
};

struct virtio_crypto_hash_data_vlf {
    /* Device read only portion */
    /* Source data */
    u8 src_data[src_data_len];

    /* Device write only portion */
    /* Hash result data */
    u8 hash_result[hash_result_len];
};
\end{lstlisting}

Each data request uses the virtio_crypto_hash_data_flf structure and the
virtio_crypto_hash_data_vlf structure to store information used to run the
HASH operations.

\field{src_data} is the source data that will be processed.
\field{src_data_len} is the length of source data.
\field{hash_result} is the result data and \field{hash_result_len} is the length
of it.

Stateless mode HASH service requests are as follows:

\begin{lstlisting}
struct virtio_crypto_hash_data_flf_stateless {
    struct {
        /* See VIRTIO_CRYPTO_HASH_* above */
        le32 algo;
    } sess_para;

    /* length of source data */
    le32 src_data_len;
    /* hash result length */
    le32 hash_result_len;
    le32 reserved;
};
struct virtio_crypto_hash_data_vlf_stateless {
    /* Device read only portion */
    /* Source data */
    u8 src_data[src_data_len];

    /* Device write only portion */
    /* Hash result data */
    u8 hash_result[hash_result_len];
};
\end{lstlisting}

\drivernormative{\paragraph}{HASH Service Operation}{Device Types / Crypto Device / Device Operation / HASH Service Operation}

\begin{itemize*}
\item If the driver uses the session mode, then the driver MUST set \field{session_id}
    in struct virtio_crypto_op_header to a valid value assigned by the device when the
    session was created.
\item If the VIRTIO_CRYPTO_F_HASH_STATELESS_MODE feature bit is negotiated, 1) if the
    driver uses the stateless mode, then the driver MUST set the \field{flag} field in
    struct virtio_crypto_op_header to ZERO and MUST set the fields in struct
    virtio_crypto_hash_data_flf_stateless.sess_para, 2) if the driver uses the session
    mode, then the driver MUST set the \field{flag} field in struct virtio_crypto_op_header
    to VIRTIO_CRYPTO_FLAG_SESSION_MODE.
\item The driver MUST set \field{opcode} in struct virtio_crypto_op_header to VIRTIO_CRYPTO_HASH.
\end{itemize*}

\devicenormative{\paragraph}{HASH Service Operation}{Device Types / Crypto Device / Device Operation / HASH Service Operation}

\begin{itemize*}
\item The device MUST use the corresponding structure according to the \field{opcode}
    in the data general header.
\item If the VIRTIO_CRYPTO_F_HASH_STATELESS_MODE feature bit is negotiated, the device
    MUST parse \field{flag} field in struct virtio_crypto_op_header in order to decide
    which mode the driver uses.
\item The device MUST copy the results of HASH operations in the hash_result[] if HASH
    operations success.
\item The device MUST set \field{status} in struct virtio_crypto_inhdr to one of the
    following values of enum VIRTIO_CRYPTO_STATUS:
\begin{itemize*}
\item VIRTIO_CRYPTO_OK if the operation success.
\item VIRTIO_CRYPTO_NOTSUPP if the requested algorithm or operation is unsupported.
\item VIRTIO_CRYPTO_INVSESS if the session ID invalid when in session mode.
\item VIRTIO_CRYPTO_ERR if any failure not mentioned above occurs.
\end{itemize*}
\end{itemize*}


\subsubsection{MAC Service Operation}\label{sec:Device Types / Crypto Device / Device Operation / MAC Service Operation}

Session mode MAC service requests are as follows:

\begin{lstlisting}
struct virtio_crypto_mac_data_flf {
    struct virtio_crypto_hash_data_flf hdr;
};

struct virtio_crypto_mac_data_vlf {
    /* Device read only portion */
    /* Source data */
    u8 src_data[src_data_len];

    /* Device write only portion */
    /* Hash result data */
    u8 hash_result[hash_result_len];
};
\end{lstlisting}

Each request uses the virtio_crypto_mac_data_flf structure and the
virtio_crypto_mac_data_vlf structure to store information used to run the
MAC operations.

\field{src_data} is the source data that will be processed.
\field{src_data_len} is the length of source data.
\field{hash_result} is the result data and \field{hash_result_len} is the length
of it.

Stateless mode MAC service requests are as follows:

\begin{lstlisting}
struct virtio_crypto_mac_data_flf_stateless {
    struct {
        /* See VIRTIO_CRYPTO_MAC_* above */
        le32 algo;
        /* length of authenticated key */
        le32 auth_key_len;
    } sess_para;

    /* length of source data */
    le32 src_data_len;
    /* hash result length */
    le32 hash_result_len;
};

struct virtio_crypto_mac_data_vlf_stateless {
    /* Device read only portion */
    /* The authenticated key */
    u8 auth_key[auth_key_len];
    /* Source data */
    u8 src_data[src_data_len];

    /* Device write only portion */
    /* Hash result data */
    u8 hash_result[hash_result_len];
};
\end{lstlisting}

\field{auth_key} is the authenticated key that will be used during the process.
\field{auth_key_len} is the length of the key.

\drivernormative{\paragraph}{MAC Service Operation}{Device Types / Crypto Device / Device Operation / MAC Service Operation}

\begin{itemize*}
\item If the driver uses the session mode, then the driver MUST set \field{session_id}
    in struct virtio_crypto_op_header to a valid value assigned by the device when the
    session was created.
\item If the VIRTIO_CRYPTO_F_MAC_STATELESS_MODE feature bit is negotiated, 1) if the
    driver uses the stateless mode, then the driver MUST set the \field{flag} field
    in struct virtio_crypto_op_header to ZERO and MUST set the fields in struct
    virtio_crypto_mac_data_flf_stateless.sess_para, 2) if the driver uses the session
    mode, then the driver MUST set the \field{flag} field in struct virtio_crypto_op_header
    to VIRTIO_CRYPTO_FLAG_SESSION_MODE.
\item The driver MUST set \field{opcode} in struct virtio_crypto_op_header to VIRTIO_CRYPTO_MAC.
\end{itemize*}

\devicenormative{\paragraph}{MAC Service Operation}{Device Types / Crypto Device / Device Operation / MAC Service Operation}

\begin{itemize*}
\item If the VIRTIO_CRYPTO_F_MAC_STATELESS_MODE feature bit is negotiated, the device
    MUST parse \field{flag} field in struct virtio_crypto_op_header in order to decide
	which mode the driver uses.
\item The device MUST copy the results of MAC operations in the hash_result[] if HASH
    operations success.
\item The device MUST set \field{status} in struct virtio_crypto_inhdr to one of the
    following values of enum VIRTIO_CRYPTO_STATUS:
\begin{itemize*}
\item VIRTIO_CRYPTO_OK if the operation success.
\item VIRTIO_CRYPTO_NOTSUPP if the requested algorithm or operation is unsupported.
\item VIRTIO_CRYPTO_INVSESS if the session ID invalid when in session mode.
\item VIRTIO_CRYPTO_ERR if any failure not mentioned above occurs.
\end{itemize*}
\end{itemize*}

\subsubsection{Symmetric algorithms Operation}\label{sec:Device Types / Crypto Device / Device Operation / Symmetric algorithms Operation}

Session mode CIPHER service requests are as follows:

\begin{lstlisting}
struct virtio_crypto_cipher_data_flf {
    /*
     * Byte Length of valid IV/Counter data pointed to by the below iv data.
     *
     * For block ciphers in CBC or F8 mode, or for Kasumi in F8 mode, or for
     *   SNOW3G in UEA2 mode, this is the length of the IV (which
     *   must be the same as the block length of the cipher).
     * For block ciphers in CTR mode, this is the length of the counter
     *   (which must be the same as the block length of the cipher).
     */
    le32 iv_len;
    /* length of source data */
    le32 src_data_len;
    /* length of destination data */
    le32 dst_data_len;
    le32 padding;
};

struct virtio_crypto_cipher_data_vlf {
    /* Device read only portion */

    /*
     * Initialization Vector or Counter data.
     *
     * For block ciphers in CBC or F8 mode, or for Kasumi in F8 mode, or for
     *   SNOW3G in UEA2 mode, this is the Initialization Vector (IV)
     *   value.
     * For block ciphers in CTR mode, this is the counter.
     * For AES-XTS, this is the 128bit tweak, i, from IEEE Std 1619-2007.
     *
     * The IV/Counter will be updated after every partial cryptographic
     * operation.
     */
    u8 iv[iv_len];
    /* Source data */
    u8 src_data[src_data_len];

    /* Device write only portion */
    /* Destination data */
    u8 dst_data[dst_data_len];
};
\end{lstlisting}

Session mode requests of algorithm chaining are as follows:

\begin{lstlisting}
struct virtio_crypto_alg_chain_data_flf {
    le32 iv_len;
    /* Length of source data */
    le32 src_data_len;
    /* Length of destination data */
    le32 dst_data_len;
    /* Starting point for cipher processing in source data */
    le32 cipher_start_src_offset;
    /* Length of the source data that the cipher will be computed on */
    le32 len_to_cipher;
    /* Starting point for hash processing in source data */
    le32 hash_start_src_offset;
    /* Length of the source data that the hash will be computed on */
    le32 len_to_hash;
    /* Length of the additional auth data */
    le32 aad_len;
    /* Length of the hash result */
    le32 hash_result_len;
    le32 reserved;
};

struct virtio_crypto_alg_chain_data_vlf {
    /* Device read only portion */

    /* Initialization Vector or Counter data */
    u8 iv[iv_len];
    /* Source data */
    u8 src_data[src_data_len];
    /* Additional authenticated data if exists */
    u8 aad[aad_len];

    /* Device write only portion */

    /* Destination data */
    u8 dst_data[dst_data_len];
    /* Hash result data */
    u8 hash_result[hash_result_len];
};
\end{lstlisting}

Session mode requests of symmetric algorithm are as follows:

\begin{lstlisting}
struct virtio_crypto_sym_data_flf {
    /* Device read only portion */

#define VIRTIO_CRYPTO_SYM_DATA_REQ_HDR_SIZE    40
    u8 op_type_flf[VIRTIO_CRYPTO_SYM_DATA_REQ_HDR_SIZE];

    /* See above VIRTIO_CRYPTO_SYM_OP_* */
    le32 op_type;
    le32 padding;
};

struct virtio_crypto_sym_data_vlf {
    u8 op_type_vlf[sym_para_len];
};
\end{lstlisting}

Each request uses the virtio_crypto_sym_data_flf structure and the
virtio_crypto_sym_data_flf structure to store information used to run the
CIPHER operations.

\field{op_type_flf} is the \field{op_type} specific header, it MUST starts
with or be one of the following structures:
\begin{itemize*}
\item struct virtio_crypto_cipher_data_flf
\item struct virtio_crypto_alg_chain_data_flf
\end{itemize*}

The length of \field{op_type_flf} is fixed to 40 bytes, the data of unused
part (if has) will be ignored.

\field{op_type_vlf} is the \field{op_type} specific parameters, it MUST starts
with or be one of the following structures:
\begin{itemize*}
\item struct virtio_crypto_cipher_data_vlf
\item struct virtio_crypto_alg_chain_data_vlf
\end{itemize*}

\field{sym_para_len} is the size of the specific structure used.

Stateless mode CIPHER service requests are as follows:

\begin{lstlisting}
struct virtio_crypto_cipher_data_flf_stateless {
    struct {
        /* See VIRTIO_CRYPTO_CIPHER* above */
        le32 algo;
        /* length of key */
        le32 key_len;

        /* See VIRTIO_CRYPTO_OP_* above */
        le32 op;
    } sess_para;

    /*
     * Byte Length of valid IV/Counter data pointed to by the below iv data.
     */
    le32 iv_len;
    /* length of source data */
    le32 src_data_len;
    /* length of destination data */
    le32 dst_data_len;
};

struct virtio_crypto_cipher_data_vlf_stateless {
    /* Device read only portion */

    /* The cipher key */
    u8 cipher_key[key_len];

    /* Initialization Vector or Counter data. */
    u8 iv[iv_len];
    /* Source data */
    u8 src_data[src_data_len];

    /* Device write only portion */
    /* Destination data */
    u8 dst_data[dst_data_len];
};
\end{lstlisting}

Stateless mode requests of algorithm chaining are as follows:

\begin{lstlisting}
struct virtio_crypto_alg_chain_data_flf_stateless {
    struct {
        /* See VIRTIO_CRYPTO_SYM_ALG_CHAIN_ORDER_* above */
        le32 alg_chain_order;
        /* length of the additional authenticated data in bytes */
        le32 aad_len;

        struct {
            /* See VIRTIO_CRYPTO_CIPHER* above */
            le32 algo;
            /* length of key */
            le32 key_len;
            /* See VIRTIO_CRYPTO_OP_* above */
            le32 op;
        } cipher;

        struct {
            /* See VIRTIO_CRYPTO_HASH_* or VIRTIO_CRYPTO_MAC_* above */
            le32 algo;
            /* length of authenticated key */
            le32 auth_key_len;
            /* See VIRTIO_CRYPTO_SYM_HASH_MODE_* above */
            le32 hash_mode;
        } hash;
    } sess_para;

    le32 iv_len;
    /* Length of source data */
    le32 src_data_len;
    /* Length of destination data */
    le32 dst_data_len;
    /* Starting point for cipher processing in source data */
    le32 cipher_start_src_offset;
    /* Length of the source data that the cipher will be computed on */
    le32 len_to_cipher;
    /* Starting point for hash processing in source data */
    le32 hash_start_src_offset;
    /* Length of the source data that the hash will be computed on */
    le32 len_to_hash;
    /* Length of the additional auth data */
    le32 aad_len;
    /* Length of the hash result */
    le32 hash_result_len;
    le32 reserved;
};

struct virtio_crypto_alg_chain_data_vlf_stateless {
    /* Device read only portion */

    /* The cipher key */
    u8 cipher_key[key_len];
    /* The auth key */
    u8 auth_key[auth_key_len];
    /* Initialization Vector or Counter data */
    u8 iv[iv_len];
    /* Additional authenticated data if exists */
    u8 aad[aad_len];
    /* Source data */
    u8 src_data[src_data_len];

    /* Device write only portion */

    /* Destination data */
    u8 dst_data[dst_data_len];
    /* Hash result data */
    u8 hash_result[hash_result_len];
};
\end{lstlisting}

Stateless mode requests of symmetric algorithm are as follows:

\begin{lstlisting}
struct virtio_crypto_sym_data_flf_stateless {
    /* Device read only portion */
#define VIRTIO_CRYPTO_SYM_DATE_REQ_HDR_STATELESS_SIZE    72
    u8 op_type_flf[VIRTIO_CRYPTO_SYM_DATE_REQ_HDR_STATELESS_SIZE];

    /* Device write only portion */
    /* See above VIRTIO_CRYPTO_SYM_OP_* */
    le32 op_type;
};

struct virtio_crypto_sym_data_vlf_stateless {
    u8 op_type_vlf[sym_para_len];
};
\end{lstlisting}

\field{op_type_flf} is the \field{op_type} specific header, it MUST starts
with or be one of the following structures:
\begin{itemize*}
\item struct virtio_crypto_cipher_data_flf_stateless
\item struct virtio_crypto_alg_chain_data_flf_stateless
\end{itemize*}

The length of \field{op_type_flf} is fixed to 72 bytes, the data of unused
part (if has) will be ignored.

\field{op_type_vlf} is the \field{op_type} specific parameters, it MUST starts
with or be one of the following structures:
\begin{itemize*}
\item struct virtio_crypto_cipher_data_vlf_stateless
\item struct virtio_crypto_alg_chain_data_vlf_stateless
\end{itemize*}

\field{sym_para_len} is the size of the specific structure used.

\drivernormative{\paragraph}{Symmetric algorithms Operation}{Device Types / Crypto Device / Device Operation / Symmetric algorithms Operation}

\begin{itemize*}
\item If the driver uses the session mode, then the driver MUST set \field{session_id}
    in struct virtio_crypto_op_header to a valid value assigned by the device when the
    session was created.
\item If the VIRTIO_CRYPTO_F_CIPHER_STATELESS_MODE feature bit is negotiated, 1) if the
    driver uses the stateless mode, then the driver MUST set the \field{flag} field in
    struct virtio_crypto_op_header to ZERO and MUST set the fields in struct
    virtio_crypto_cipher_data_flf_stateless.sess_para or struct
    virtio_crypto_alg_chain_data_flf_stateless.sess_para, 2) if the driver uses the
    session mode, then the driver MUST set the \field{flag} field in struct
    virtio_crypto_op_header to VIRTIO_CRYPTO_FLAG_SESSION_MODE.
\item The driver MUST set the \field{opcode} field in struct virtio_crypto_op_header
    to VIRTIO_CRYPTO_CIPHER_ENCRYPT or VIRTIO_CRYPTO_CIPHER_DECRYPT.
\item The driver MUST specify the fields of struct virtio_crypto_cipher_data_flf in
    struct virtio_crypto_sym_data_flf and struct virtio_crypto_cipher_data_vlf in
    struct virtio_crypto_sym_data_vlf if the request is based on VIRTIO_CRYPTO_SYM_OP_CIPHER.
\item The driver MUST specify the fields of struct virtio_crypto_alg_chain_data_flf
    in struct virtio_crypto_sym_data_flf and struct virtio_crypto_alg_chain_data_vlf
    in struct virtio_crypto_sym_data_vlf if the request is of the VIRTIO_CRYPTO_SYM_OP_ALGORITHM_CHAINING
    type.
\end{itemize*}

\devicenormative{\paragraph}{Symmetric algorithms Operation}{Device Types / Crypto Device / Device Operation / Symmetric algorithms Operation}

\begin{itemize*}
\item If the VIRTIO_CRYPTO_F_CIPHER_STATELESS_MODE feature bit is negotiated, the device
    MUST parse \field{flag} field in struct virtio_crypto_op_header in order to decide
	which mode the driver uses.
\item The device MUST parse the virtio_crypto_sym_data_req based on the \field{opcode}
    field in general header.
\item The device MUST parse the fields of struct virtio_crypto_cipher_data_flf in
    struct virtio_crypto_sym_data_flf and struct virtio_crypto_cipher_data_vlf in
    struct virtio_crypto_sym_data_vlf if the request is based on VIRTIO_CRYPTO_SYM_OP_CIPHER.
\item The device MUST parse the fields of struct virtio_crypto_alg_chain_data_flf
    in struct virtio_crypto_sym_data_flf and struct virtio_crypto_alg_chain_data_vlf
    in struct virtio_crypto_sym_data_vlf if the request is of the VIRTIO_CRYPTO_SYM_OP_ALGORITHM_CHAINING
    type.
\item The device MUST copy the result of cryptographic operation in the dst_data[] in
    both plain CIPHER mode and algorithms chain mode.
\item The device MUST check the \field{para}.\field{add_len} is bigger than 0 before
    parse the additional authenticated data in plain algorithms chain mode.
\item The device MUST copy the result of HASH/MAC operation in the hash_result[] is
    of the VIRTIO_CRYPTO_SYM_OP_ALGORITHM_CHAINING type.
\item The device MUST set the \field{status} field in struct virtio_crypto_inhdr to
    one of the following values of enum VIRTIO_CRYPTO_STATUS:
\begin{itemize*}
\item VIRTIO_CRYPTO_OK if the operation success.
\item VIRTIO_CRYPTO_NOTSUPP if the requested algorithm or operation is unsupported.
\item VIRTIO_CRYPTO_INVSESS if the session ID is invalid in session mode.
\item VIRTIO_CRYPTO_ERR if failure not mentioned above occurs.
\end{itemize*}
\end{itemize*}

\subsubsection{AEAD Service Operation}\label{sec:Device Types / Crypto Device / Device Operation / AEAD Service Operation}

Session mode requests of symmetric algorithm are as follows:

\begin{lstlisting}
struct virtio_crypto_aead_data_flf {
    /*
     * Byte Length of valid IV data.
     *
     * For GCM mode, this is either 12 (for 96-bit IVs) or 16, in which
     *   case iv points to J0.
     * For CCM mode, this is the length of the nonce, which can be in the
     *   range 7 to 13 inclusive.
     */
    le32 iv_len;
    /* length of additional auth data */
    le32 aad_len;
    /* length of source data */
    le32 src_data_len;
    /* length of dst data, this should be at least src_data_len + tag_len */
    le32 dst_data_len;
    /* Authentication tag length */
    le32 tag_len;
    le32 reserved;
};

struct virtio_crypto_aead_data_vlf {
    /* Device read only portion */

    /*
     * Initialization Vector data.
     *
     * For GCM mode, this is either the IV (if the length is 96 bits) or J0
     *   (for other sizes), where J0 is as defined by NIST SP800-38D.
     *   Regardless of the IV length, a full 16 bytes needs to be allocated.
     * For CCM mode, the first byte is reserved, and the nonce should be
     *   written starting at &iv[1] (to allow space for the implementation
     *   to write in the flags in the first byte).  Note that a full 16 bytes
     *   should be allocated, even though the iv_len field will have
     *   a value less than this.
     *
     * The IV will be updated after every partial cryptographic operation.
     */
    u8 iv[iv_len];
    /* Source data */
    u8 src_data[src_data_len];
    /* Additional authenticated data if exists */
    u8 aad[aad_len];

    /* Device write only portion */
    /* Pointer to output data */
    u8 dst_data[dst_data_len];
};
\end{lstlisting}

Each request uses the virtio_crypto_aead_data_flf structure and the
virtio_crypto_aead_data_flf structure to store information used to run the
AEAD operations.

Stateless mode AEAD service requests are as follows:

\begin{lstlisting}
struct virtio_crypto_aead_data_flf_stateless {
    struct {
        /* See VIRTIO_CRYPTO_AEAD_* above */
        le32 algo;
        /* length of key */
        le32 key_len;
        /* encrypt or decrypt, See above VIRTIO_CRYPTO_OP_* */
        le32 op;
    } sess_para;

    /* Byte Length of valid IV data. */
    le32 iv_len;
    /* Authentication tag length */
    le32 tag_len;
    /* length of additional auth data */
    le32 aad_len;
    /* length of source data */
    le32 src_data_len;
    /* length of dst data, this should be at least src_data_len + tag_len */
    le32 dst_data_len;
};

struct virtio_crypto_aead_data_vlf_stateless {
    /* Device read only portion */

    /* The cipher key */
    u8 key[key_len];
    /* Initialization Vector data. */
    u8 iv[iv_len];
    /* Source data */
    u8 src_data[src_data_len];
    /* Additional authenticated data if exists */
    u8 aad[aad_len];

    /* Device write only portion */
    /* Pointer to output data */
    u8 dst_data[dst_data_len];
};
\end{lstlisting}

\drivernormative{\paragraph}{AEAD Service Operation}{Device Types / Crypto Device / Device Operation / AEAD Service Operation}

\begin{itemize*}
\item If the driver uses the session mode, then the driver MUST set
    \field{session_id} in struct virtio_crypto_op_header to a valid value assigned
    by the device when the session was created.
\item If the VIRTIO_CRYPTO_F_AEAD_STATELESS_MODE feature bit is negotiated, 1) if
    the driver uses the stateless mode, then the driver MUST set the \field{flag}
    field in struct virtio_crypto_op_header to ZERO and MUST set the fields in
    struct virtio_crypto_aead_data_flf_stateless.sess_para, 2) if the driver uses
    the session mode, then the driver MUST set the \field{flag} field in struct
    virtio_crypto_op_header to VIRTIO_CRYPTO_FLAG_SESSION_MODE.
\item The driver MUST set the \field{opcode} field in struct virtio_crypto_op_header
    to VIRTIO_CRYPTO_AEAD_ENCRYPT or VIRTIO_CRYPTO_AEAD_DECRYPT.
\end{itemize*}

\devicenormative{\paragraph}{AEAD Service Operation}{Device Types / Crypto Device / Device Operation / AEAD Service Operation}

\begin{itemize*}
\item If the VIRTIO_CRYPTO_F_AEAD_STATELESS_MODE feature bit is negotiated, the
    device MUST parse the virtio_crypto_aead_data_vlf_stateless based on the \field{opcode}
	field in general header.
\item The device MUST copy the result of cryptographic operation in the dst_data[].
\item The device MUST copy the authentication tag in the dst_data[] offset the cipher result.
\item The device MUST set the \field{status} field in struct virtio_crypto_inhdr to
    one of the following values of enum VIRTIO_CRYPTO_STATUS:
\item When the \field{opcode} field is VIRTIO_CRYPTO_AEAD_DECRYPT, the device MUST
    verify and return the verification result to the driver.
\begin{itemize*}
\item VIRTIO_CRYPTO_OK if the operation success.
\item VIRTIO_CRYPTO_NOTSUPP if the requested algorithm or operation is unsupported.
\item VIRTIO_CRYPTO_BADMSG if the verification result is incorrect.
\item VIRTIO_CRYPTO_INVSESS if the session ID invalid when in session mode.
\item VIRTIO_CRYPTO_ERR if any failure not mentioned above occurs.
\end{itemize*}
\end{itemize*}

\subsubsection{AKCIPHER Service Operation}\label{sec:Device Types / Crypto Device / Device Operation / AKCIPHER Service Operation}

Session mode AKCIPHER requests are as follows:

\begin{lstlisting}
struct virtio_crypto_akcipher_data_flf {
    /* length of source data */
    le32 src_data_len;
    /* length of dst data */
    le32 dst_data_len;
};

struct virtio_crypto_akcipher_data_vlf {
    /* Device read only portion */
    /* Source data */
    u8 src_data[src_data_len];

    /* Device write only portion */
    /* Pointer to output data */
    u8 dst_data[dst_data_len];
};
\end{lstlisting}

Each data request uses the virtio_crypto_akcipher_flf structure and the virtio_crypto_akcipher_data_vlf
structure to store information used to run the AKCIPHER operations.

For encryption, decryption, and signing:
\field{src_data} is the source data that will be processed, note that for signing operations,
src_data stores the data to be signed, which usually is the digest of some data rather than the
data itself.
\field{src_data_len} is the length of source data.
\field{dst_result} is the result data and \field{dst_data_len} is the length of it. Note that the
length of the result is not always exactly equal to dst_data_len, the driver needs to check how
many bytes the device has written and calculate the actual length of the result.

For verification:
\field{src_data_len} refers to the length of the signature, and \field{dst_data_len} refers to
the length of signed data, where the signed data is usually the digest of some data.
\field{src_data} is spliced by the signature and the signed data, the src_data with the lower
address stores the signature, and the higher address stores the signed data.
\field{dst_data} is always empty for verification.

Different algorithms have different signature formats.
For the RSA algorithm, the result is determined by the padding algorithm specified by
\field{padding_algo} in structure virtio_crypto_rsa_session_para.

For the ECDSA algorithm, the signature is composed of the following
ASN.1 structure (see \hyperref[intro:rfc3279]{RFC3279})
and MUST be DER encoded (see \hyperref[intro:rfc6025]{rfc6025}).

\begin{lstlisting}
Ecdsa-Sig-Value ::= SEQUENCE {
    r INTEGER,
    s INTEGER
}
\end{lstlisting}

Stateless mode AKCIPHER service requests are as follows:
\begin{lstlisting}
struct virtio_crypto_akcipher_data_flf_stateless {
    struct {
        /* See VIRTIO_CYRPTO_AKCIPHER* above */
        le32 algo;
        /* See VIRTIO_CRYPTO_AKCIPHER_KEY_TYPE_* above */
        le32 key_type;
        /* length of key */
        le32 key_len;

        /* algothrim specific parameters described above */
        union {
            struct virtio_crypto_rsa_session_para rsa;
            struct virtio_crypto_ecdsa_session_para ecdsa;
        } u;
    } sess_para;

    /* length of source data */
    le32 src_data_len;
    /* length of destination data */
    le32 dst_data_len;
};

struct virtio_crypto_akcipher_data_vlf_stateless {
    /* Device read only portion */
    u8 akcipher_key[key_len];

    /* Source data */
    u8 src_data[src_data_len];

    /* Device write only portion */
    u8 dst_data[dst_data_len];
};
\end{lstlisting}

In stateless mode, the format of key and signature, the meaning of src_data and dst_data, are all the same
with session mode.

\drivernormative{\paragraph}{AKCIPHER Service Operation}{Device Types / Crypto Device / Device Operation / AKCIPHER Service Operation}

\begin{itemize*}
\item If the driver uses the session mode, then the driver MUST set
    \field{session_id} in struct virtio_crypto_op_header to a valid
    value assigned by the device when the session was created.
\item If the VIRTIO_CRYPTO_F_AKCIPHER_STATELESS_MODE feature bit is negotiated, 1) if the
    driver uses the stateless mode, then the driver MUST set the \field{flag} field in
    struct virtio_crypto_op_header to ZERO and MUST set the fields in struct
    virtio_crypto_akcipher_flf_stateless.sess_para, 2) if the driver uses the session
    mode, then the driver MUST set the \field{flag} field in struct virtio_crypto_op_header
    to VIRTIO_CRYPTO_FLAG_SESSION_MODE.
\item The driver MUST set the \field{opcode} field in struct virtio_crypto_op_header
    to one of VIRTIO_CRYPTO_AKCIPHER_ENCRYPT, VIRTIO_CRYPTO_AKCIPHER_DESTROY_SESSION,
    VIRTIO_CRYPTO_AKCIPHER_SIGN, and VIRTIO_CRYPTO_AKCIPHER_VERIFY.
\end{itemize*}

\devicenormative{\paragraph}{AKCIPHER Service Operation}{Device Types / Crypto Device / Device Operation / AKCIPHER Service Operation}

\begin{itemize*}
\item If the VIRTIO_CRYPTO_F_AKCIPHER_STATELESS_MODE feature bit is negotiated, the
    device MUST parse the virtio_crypto_akcipher_data_vlf_stateless based on the \field{opcode}
    field in general header.
\item The device MUST copy the result of cryptographic operation in the dst_data[].
\item The device MUST set the \field{status} field in struct virtio_crypto_inhdr to
    one of the following values of enum VIRTIO_CRYPTO_STATUS:
\begin{itemize*}
\item VIRTIO_CRYPTO_OK if the operation success.
\item VIRTIO_CRYPTO_NOTSUPP if the requested algorithm or operation is unsupported.
\item VIRTIO_CRYPTO_BADMSG if the verification result is incorrect.
\item VIRTIO_CRYPTO_INVSESS if the session ID invalid when in session mode.
\item VIRTIO_CRYPTO_KEY_REJECTED if the signature verification failed.
\item VIRTIO_CRYPTO_ERR if any failure not mentioned above occurs.
\end{itemize*}
\end{itemize*}

\section{Crypto Device}\label{sec:Device Types / Crypto Device}

The virtio crypto device is a virtual cryptography device as well as a
virtual cryptographic accelerator. The virtio crypto device provides the
following crypto services: CIPHER, MAC, HASH, AEAD and AKCIPHER. Virtio crypto
devices have a single control queue and at least one data queue. Crypto
operation requests are placed into a data queue, and serviced by the
device. Some crypto operation requests are only valid in the context of a
session. The role of the control queue is facilitating control operation
requests. Sessions management is realized with control operation
requests.

\subsection{Device ID}\label{sec:Device Types / Crypto Device / Device ID}

20

\subsection{Virtqueues}\label{sec:Device Types / Crypto Device / Virtqueues}

\begin{description}
\item[0] dataq1
\item[\ldots]
\item[N-1] dataqN
\item[N] controlq
\end{description}

N is set by \field{max_dataqueues}.

\subsection{Feature bits}\label{sec:Device Types / Crypto Device / Feature bits}

\begin{description}
\item VIRTIO_CRYPTO_F_REVISION_1 (0) revision 1. Revision 1 has a specific
    request format and other enhancements (which result in some additional
    requirements).
\item VIRTIO_CRYPTO_F_CIPHER_STATELESS_MODE (1) stateless mode requests are
    supported by the CIPHER service.
\item VIRTIO_CRYPTO_F_HASH_STATELESS_MODE (2) stateless mode requests are
    supported by the HASH service.
\item VIRTIO_CRYPTO_F_MAC_STATELESS_MODE (3) stateless mode requests are
    supported by the MAC service.
\item VIRTIO_CRYPTO_F_AEAD_STATELESS_MODE (4) stateless mode requests are
    supported by the AEAD service.
\item VIRTIO_CRYPTO_F_AKCIPHER_STATELESS_MODE (5) stateless mode requests are
    supported by the AKCIPHER service.
\end{description}


\subsubsection{Feature bit requirements}\label{sec:Device Types / Crypto Device / Feature bit requirements}

Some crypto feature bits require other crypto feature bits
(see \ref{drivernormative:Basic Facilities of a Virtio Device / Feature Bits}):

\begin{description}
\item[VIRTIO_CRYPTO_F_CIPHER_STATELESS_MODE] Requires VIRTIO_CRYPTO_F_REVISION_1.
\item[VIRTIO_CRYPTO_F_HASH_STATELESS_MODE] Requires VIRTIO_CRYPTO_F_REVISION_1.
\item[VIRTIO_CRYPTO_F_MAC_STATELESS_MODE] Requires VIRTIO_CRYPTO_F_REVISION_1.
\item[VIRTIO_CRYPTO_F_AEAD_STATELESS_MODE] Requires VIRTIO_CRYPTO_F_REVISION_1.
\item[VIRTIO_CRYPTO_F_AKCIPHER_STATELESS_MODE] Requires VIRTIO_CRYPTO_F_REVISION_1.
\end{description}

\subsection{Supported crypto services}\label{sec:Device Types / Crypto Device / Supported crypto services}

The following crypto services are defined:

\begin{lstlisting}
/* CIPHER (Symmetric Key Cipher) service */
#define VIRTIO_CRYPTO_SERVICE_CIPHER 0
/* HASH service */
#define VIRTIO_CRYPTO_SERVICE_HASH   1
/* MAC (Message Authentication Codes) service */
#define VIRTIO_CRYPTO_SERVICE_MAC    2
/* AEAD (Authenticated Encryption with Associated Data) service */
#define VIRTIO_CRYPTO_SERVICE_AEAD   3
/* AKCIPHER (Asymmetric Key Cipher) service */
#define VIRTIO_CRYPTO_SERVICE_AKCIPHER 4
\end{lstlisting}

The above constants designate bits used to indicate the which of crypto services are
offered by the device as described in, see \ref{sec:Device Types / Crypto Device / Device configuration layout}.

\subsubsection{CIPHER services}\label{sec:Device Types / Crypto Device / Supported crypto services / CIPHER services}

The following CIPHER algorithms are defined:

\begin{lstlisting}
#define VIRTIO_CRYPTO_NO_CIPHER                 0
#define VIRTIO_CRYPTO_CIPHER_ARC4               1
#define VIRTIO_CRYPTO_CIPHER_AES_ECB            2
#define VIRTIO_CRYPTO_CIPHER_AES_CBC            3
#define VIRTIO_CRYPTO_CIPHER_AES_CTR            4
#define VIRTIO_CRYPTO_CIPHER_DES_ECB            5
#define VIRTIO_CRYPTO_CIPHER_DES_CBC            6
#define VIRTIO_CRYPTO_CIPHER_3DES_ECB           7
#define VIRTIO_CRYPTO_CIPHER_3DES_CBC           8
#define VIRTIO_CRYPTO_CIPHER_3DES_CTR           9
#define VIRTIO_CRYPTO_CIPHER_KASUMI_F8          10
#define VIRTIO_CRYPTO_CIPHER_SNOW3G_UEA2        11
#define VIRTIO_CRYPTO_CIPHER_AES_F8             12
#define VIRTIO_CRYPTO_CIPHER_AES_XTS            13
#define VIRTIO_CRYPTO_CIPHER_ZUC_EEA3           14
\end{lstlisting}

The above constants have two usages:
\begin{enumerate}
\item As bit numbers, used to tell the driver which CIPHER algorithms
are supported by the device, see \ref{sec:Device Types / Crypto Device / Device configuration layout}.
\item As values, used to designate the algorithm in (CIPHER type) crypto
operation requests, see \ref{sec:Device Types / Crypto Device / Device Operation / Control Virtqueue / Session operation}.
\end{enumerate}

\subsubsection{HASH services}\label{sec:Device Types / Crypto Device / Supported crypto services / HASH services}

The following HASH algorithms are defined:

\begin{lstlisting}
#define VIRTIO_CRYPTO_NO_HASH            0
#define VIRTIO_CRYPTO_HASH_MD5           1
#define VIRTIO_CRYPTO_HASH_SHA1          2
#define VIRTIO_CRYPTO_HASH_SHA_224       3
#define VIRTIO_CRYPTO_HASH_SHA_256       4
#define VIRTIO_CRYPTO_HASH_SHA_384       5
#define VIRTIO_CRYPTO_HASH_SHA_512       6
#define VIRTIO_CRYPTO_HASH_SHA3_224      7
#define VIRTIO_CRYPTO_HASH_SHA3_256      8
#define VIRTIO_CRYPTO_HASH_SHA3_384      9
#define VIRTIO_CRYPTO_HASH_SHA3_512      10
#define VIRTIO_CRYPTO_HASH_SHA3_SHAKE128      11
#define VIRTIO_CRYPTO_HASH_SHA3_SHAKE256      12
\end{lstlisting}

The above constants have two usages:
\begin{enumerate}
\item As bit numbers, used to tell the driver which HASH algorithms
are supported by the device, see \ref{sec:Device Types / Crypto Device / Device configuration layout}.
\item As values, used to designate the algorithm in (HASH type) crypto
operation requires, see \ref{sec:Device Types / Crypto Device / Device Operation / Control Virtqueue / Session operation}.
\end{enumerate}

\subsubsection{MAC services}\label{sec:Device Types / Crypto Device / Supported crypto services / MAC services}

The following MAC algorithms are defined:

\begin{lstlisting}
#define VIRTIO_CRYPTO_NO_MAC                       0
#define VIRTIO_CRYPTO_MAC_HMAC_MD5                 1
#define VIRTIO_CRYPTO_MAC_HMAC_SHA1                2
#define VIRTIO_CRYPTO_MAC_HMAC_SHA_224             3
#define VIRTIO_CRYPTO_MAC_HMAC_SHA_256             4
#define VIRTIO_CRYPTO_MAC_HMAC_SHA_384             5
#define VIRTIO_CRYPTO_MAC_HMAC_SHA_512             6
#define VIRTIO_CRYPTO_MAC_CMAC_3DES                25
#define VIRTIO_CRYPTO_MAC_CMAC_AES                 26
#define VIRTIO_CRYPTO_MAC_KASUMI_F9                27
#define VIRTIO_CRYPTO_MAC_SNOW3G_UIA2              28
#define VIRTIO_CRYPTO_MAC_GMAC_AES                 41
#define VIRTIO_CRYPTO_MAC_GMAC_TWOFISH             42
#define VIRTIO_CRYPTO_MAC_CBCMAC_AES               49
#define VIRTIO_CRYPTO_MAC_CBCMAC_KASUMI_F9         50
#define VIRTIO_CRYPTO_MAC_XCBC_AES                 53
#define VIRTIO_CRYPTO_MAC_ZUC_EIA3                 54
\end{lstlisting}

The above constants have two usages:
\begin{enumerate}
\item As bit numbers, used to tell the driver which MAC algorithms
are supported by the device, see \ref{sec:Device Types / Crypto Device / Device configuration layout}.
\item As values, used to designate the algorithm in (MAC type) crypto
operation requests, see \ref{sec:Device Types / Crypto Device / Device Operation / Control Virtqueue / Session operation}.
\end{enumerate}

\subsubsection{AEAD services}\label{sec:Device Types / Crypto Device / Supported crypto services / AEAD services}

The following AEAD algorithms are defined:

\begin{lstlisting}
#define VIRTIO_CRYPTO_NO_AEAD     0
#define VIRTIO_CRYPTO_AEAD_GCM    1
#define VIRTIO_CRYPTO_AEAD_CCM    2
#define VIRTIO_CRYPTO_AEAD_CHACHA20_POLY1305  3
\end{lstlisting}

The above constants have two usages:
\begin{enumerate}
\item As bit numbers, used to tell the driver which AEAD algorithms
are supported by the device, see \ref{sec:Device Types / Crypto Device / Device configuration layout}.
\item As values, used to designate the algorithm in (DEAD type) crypto
operation requests, see \ref{sec:Device Types / Crypto Device / Device Operation / Control Virtqueue / Session operation}.
\end{enumerate}

\subsubsection{AKCIPHER services}\label{sec: Device Types / Crypto Device / Supported crypto services / AKCIPHER services}

The following AKCIPHER algorithms are defined:
\begin{lstlisting}
#define VIRTIO_CRYPTO_NO_AKCIPHER 0
#define VIRTIO_CRYPTO_AKCIPHER_RSA   1
#define VIRTIO_CRYPTO_AKCIPHER_ECDSA 2
\end{lstlisting}

The above constants have two usages:
\begin{enumerate}
\item As bit numbers, used to tell the driver which AKCIPHER algorithms
are supported by the device, see \ref{sec:Device Types / Crypto Device / Device configuration layout}.
\item As values, used to designate the algorithm in asymmetric crypto operation requests,
see \ref{sec:Device Types / Crypto Device / Device Operation / Control Virtqueue / Session operation}.
\end{enumerate}


\subsection{Device configuration layout}\label{sec:Device Types / Crypto Device / Device configuration layout}

Crypto device configuration uses the following layout structure:

\begin{lstlisting}
struct virtio_crypto_config {
    le32 status;
    le32 max_dataqueues;
    le32 crypto_services;
    /* Detailed algorithms mask */
    le32 cipher_algo_l;
    le32 cipher_algo_h;
    le32 hash_algo;
    le32 mac_algo_l;
    le32 mac_algo_h;
    le32 aead_algo;
    /* Maximum length of cipher key in bytes */
    le32 max_cipher_key_len;
    /* Maximum length of authenticated key in bytes */
    le32 max_auth_key_len;
    le32 akcipher_algo;
    /* Maximum size of each crypto request's content in bytes */
    le64 max_size;
};
\end{lstlisting}

\begin{description}
\item Currently, only one \field{status} bit is defined: VIRTIO_CRYPTO_S_HW_READY
    set indicates that the device is ready to process requests, this bit is read-only
    for the driver
\begin{lstlisting}
#define VIRTIO_CRYPTO_S_HW_READY  (1 << 0)
\end{lstlisting}

\item [\field{max_dataqueues}] is the maximum number of data virtqueues that can
    be configured by the device. The driver MAY use only one data queue, or it
    can use more to achieve better performance.

\item [\field{crypto_services}] crypto service offered, see \ref{sec:Device Types / Crypto Device / Supported crypto services}.

\item [\field{cipher_algo_l}] CIPHER algorithms bits 0-31, see \ref{sec:Device Types / Crypto Device / Supported crypto services  / CIPHER services}.

\item [\field{cipher_algo_h}] CIPHER algorithms bits 32-63, see \ref{sec:Device Types / Crypto Device / Supported crypto services  / CIPHER services}.

\item [\field{hash_algo}] HASH algorithms bits, see \ref{sec:Device Types / Crypto Device / Supported crypto services  / HASH services}.

\item [\field{mac_algo_l}] MAC algorithms bits 0-31, see \ref{sec:Device Types / Crypto Device / Supported crypto services  / MAC services}.

\item [\field{mac_algo_h}] MAC algorithms bits 32-63, see \ref{sec:Device Types / Crypto Device / Supported crypto services  / MAC services}.

\item [\field{aead_algo}] AEAD algorithms bits, see \ref{sec:Device Types / Crypto Device / Supported crypto services  / AEAD services}.

\item [\field{max_cipher_key_len}] is the maximum length of cipher key supported by the device.

\item [\field{max_auth_key_len}] is the maximum length of authenticated key supported by the device.

\item [\field{akcipher_algo}] AKCIPHER algorithms bit 0-31, see \ref{sec: Device Types / Crypto Device / Supported crypto services / AKCIPHER services}.

\item [\field{max_size}] is the maximum size of the variable-length parameters of
    data operation of each crypto request's content supported by the device.
\end{description}

\begin{note}
Unless explicitly stated otherwise all lengths and sizes are in bytes.
\end{note}

\devicenormative{\subsubsection}{Device configuration layout}{Device Types / Crypto Device / Device configuration layout}

\begin{itemize*}
\item The device MUST set \field{max_dataqueues} to between 1 and 65535 inclusive.
\item The device MUST set the \field{status} with valid flags, undefined flags MUST NOT be set.
\item The device MUST accept and handle requests after \field{status} is set to VIRTIO_CRYPTO_S_HW_READY.
\item The device MUST set \field{crypto_services} based on the crypto services the device offers.
\item The device MUST set detailed algorithms masks for each service advertised by \field{crypto_services}.
    The device MUST NOT set the not defined algorithms bits.
\item The device MUST set \field{max_size} to show the maximum size of crypto request the device supports.
\item The device MUST set \field{max_cipher_key_len} to show the maximum length of cipher key if the
    device supports CIPHER service.
\item The device MUST set \field{max_auth_key_len} to show the maximum length of authenticated key if
    the device supports MAC service.
\end{itemize*}

\drivernormative{\subsubsection}{Device configuration layout}{Device Types / Crypto Device / Device configuration layout}

\begin{itemize*}
\item The driver MUST read the \field{status} from the bottom bit of status to check whether the
    VIRTIO_CRYPTO_S_HW_READY is set, and the driver MUST reread it after device reset.
\item The driver MUST NOT transmit any requests to the device if the VIRTIO_CRYPTO_S_HW_READY is not set.
\item The driver MUST read \field{max_dataqueues} field to discover the number of data queues the device supports.
\item The driver MUST read \field{crypto_services} field to discover which services the device is able to offer.
\item The driver SHOULD ignore the not defined algorithms bits.
\item The driver MUST read the detailed algorithms fields based on \field{crypto_services} field.
\item The driver SHOULD read \field{max_size} to discover the maximum size of the variable-length
    parameters of data operation of the crypto request's content the device supports and MUST
    guarantee that the size of each crypto request's content is within the \field{max_size}, otherwise
    the request will fail and the driver MUST reset the device.
\item The driver SHOULD read \field{max_cipher_key_len} to discover the maximum length of cipher key
    the device supports and MUST guarantee that the \field{key_len} (CIPHER service or AEAD service) is within
    the \field{max_cipher_key_len} of the device configuration, otherwise the request will fail.
\item The driver SHOULD read \field{max_auth_key_len} to discover the maximum length of authenticated
    key the device supports and MUST guarantee that the \field{auth_key_len} (MAC service) is within the
    \field{max_auth_key_len} of the device configuration, otherwise the request will fail.
\end{itemize*}

\subsection{Device Initialization}\label{sec:Device Types / Crypto Device / Device Initialization}

\drivernormative{\subsubsection}{Device Initialization}{Device Types / Crypto Device / Device Initialization}

\begin{itemize*}
\item The driver MUST configure and initialize all virtqueues.
\item The driver MUST read the supported crypto services from bits of \field{crypto_services}.
\item The driver MUST read the supported algorithms based on \field{crypto_services} field.
\end{itemize*}

\subsection{Device Operation}\label{sec:Device Types / Crypto Device / Device Operation}

The operation of a virtio crypto device is driven by requests placed on the virtqueues.
Requests consist of a queue-type specific header (specifying among others the operation)
and an operation specific payload.

If VIRTIO_CRYPTO_F_REVISION_1 is negotiated the device may support both session mode
(See \ref{sec:Device Types / Crypto Device / Device Operation / Control Virtqueue / Session operation})
and stateless mode operation requests.
In stateless mode all operation parameters are supplied as a part of each request,
while in session mode, some or all operation parameters are managed within the
session. Stateless mode is guarded by feature bits 0-4 on a service level. If
stateless mode is negotiated for a service, the service accepts both session
mode and stateless requests; otherwise stateless mode requests are rejected
(via operation status).

\subsubsection{Operation Status}\label{sec:Device Types / Crypto Device / Device Operation / Operation status}
The device MUST return a status code as part of the operation (both session
operation and service operation) result. The valid operation status as follows:

\begin{lstlisting}
enum VIRTIO_CRYPTO_STATUS {
    VIRTIO_CRYPTO_OK = 0,
    VIRTIO_CRYPTO_ERR = 1,
    VIRTIO_CRYPTO_BADMSG = 2,
    VIRTIO_CRYPTO_NOTSUPP = 3,
    VIRTIO_CRYPTO_INVSESS = 4,
    VIRTIO_CRYPTO_NOSPC = 5,
    VIRTIO_CRYPTO_KEY_REJECTED = 6,
    VIRTIO_CRYPTO_MAX
};
\end{lstlisting}

\begin{itemize*}
\item VIRTIO_CRYPTO_OK: success.
\item VIRTIO_CRYPTO_BADMSG: authentication failed (only when AEAD decryption).
\item VIRTIO_CRYPTO_NOTSUPP: operation or algorithm is unsupported.
\item VIRTIO_CRYPTO_INVSESS: invalid session ID when executing crypto operations.
\item VIRTIO_CRYPTO_NOSPC: no free session ID (only when the VIRTIO_CRYPTO_F_REVISION_1
    feature bit is negotiated).
\item VIRTIO_CRYPTO_KEY_REJECTED: signature verification failed (only when AKCIPHER verification).
\item VIRTIO_CRYPTO_ERR: any failure not mentioned above occurs.
\end{itemize*}

\subsubsection{Control Virtqueue}\label{sec:Device Types / Crypto Device / Device Operation / Control Virtqueue}

The driver uses the control virtqueue to send control commands to the
device, such as session operations (See \ref{sec:Device Types / Crypto Device / Device
Operation / Control Virtqueue / Session operation}).

The header for controlq is of the following form:
\begin{lstlisting}
#define VIRTIO_CRYPTO_OPCODE(service, op)   (((service) << 8) | (op))

struct virtio_crypto_ctrl_header {
#define VIRTIO_CRYPTO_CIPHER_CREATE_SESSION \
       VIRTIO_CRYPTO_OPCODE(VIRTIO_CRYPTO_SERVICE_CIPHER, 0x02)
#define VIRTIO_CRYPTO_CIPHER_DESTROY_SESSION \
       VIRTIO_CRYPTO_OPCODE(VIRTIO_CRYPTO_SERVICE_CIPHER, 0x03)
#define VIRTIO_CRYPTO_HASH_CREATE_SESSION \
       VIRTIO_CRYPTO_OPCODE(VIRTIO_CRYPTO_SERVICE_HASH, 0x02)
#define VIRTIO_CRYPTO_HASH_DESTROY_SESSION \
       VIRTIO_CRYPTO_OPCODE(VIRTIO_CRYPTO_SERVICE_HASH, 0x03)
#define VIRTIO_CRYPTO_MAC_CREATE_SESSION \
       VIRTIO_CRYPTO_OPCODE(VIRTIO_CRYPTO_SERVICE_MAC, 0x02)
#define VIRTIO_CRYPTO_MAC_DESTROY_SESSION \
       VIRTIO_CRYPTO_OPCODE(VIRTIO_CRYPTO_SERVICE_MAC, 0x03)
#define VIRTIO_CRYPTO_AEAD_CREATE_SESSION \
       VIRTIO_CRYPTO_OPCODE(VIRTIO_CRYPTO_SERVICE_AEAD, 0x02)
#define VIRTIO_CRYPTO_AEAD_DESTROY_SESSION \
       VIRTIO_CRYPTO_OPCODE(VIRTIO_CRYPTO_SERVICE_AEAD, 0x03)
#define VIRTIO_CRYPTO_AKCIPHER_CREATE_SESSION \
       VIRTIO_CRYPTO_OPCODE(VIRTIO_CRYPTO_SERVICE_AKCIPHER, 0x04)
#define VIRTIO_CRYPTO_AKCIPHER_DESTROY_SESSION \
       VIRTIO_CRYPTO_OPCDE(VIRTIO_CRYPTO_SERVICE_AKCIPHER, 0x05)
    le32 opcode;
    /* algo should be service-specific algorithms */
    le32 algo;
    le32 flag;
    le32 reserved;
};
\end{lstlisting}

The controlq request is composed of four parts:
\begin{lstlisting}
struct virtio_crypto_op_ctrl_req {
    /* Device read only portion */

    struct virtio_crypto_ctrl_header header;

#define VIRTIO_CRYPTO_CTRLQ_OP_SPEC_HDR_LEGACY 56
    /* fixed length fields, opcode specific */
    u8 op_flf[flf_len];

    /* variable length fields, opcode specific */
    u8 op_vlf[vlf_len];

    /* Device write only portion */

    /* op result or completion status */
    u8 op_outcome[outcome_len];
};
\end{lstlisting}

\field{header} is a general header (see above).

\field{op_flf} is the opcode (in \field{header}) specific fixed-length parameters.

\field{flf_len} depends on the VIRTIO_CRYPTO_F_REVISION_1 feature bit (see below).

\field{op_vlf} is the opcode (in \field{header}) specific variable-length parameters.

\field{vlf_len} is the size of the specific structure used.
\begin{note}
The \field{vlf_len} of session-destroy operation and the hash-session-create
operation is ZERO.
\end{note}

\begin{itemize*}
\item If the opcode (in \field{header}) is VIRTIO_CRYPTO_CIPHER_CREATE_SESSION
    then \field{op_flf} is struct virtio_crypto_sym_create_session_flf if
    VIRTIO_CRYPTO_F_REVISION_1 is negotiated and struct virtio_crypto_sym_create_session_flf is
    padded to 56 bytes if NOT negotiated, and \field{op_vlf} is struct
    virtio_crypto_sym_create_session_vlf.
\item If the opcode (in \field{header}) is VIRTIO_CRYPTO_HASH_CREATE_SESSION
    then \field{op_flf} is struct virtio_crypto_hash_create_session_flf if
    VIRTIO_CRYPTO_F_REVISION_1 is negotiated and struct virtio_crypto_hash_create_session_flf is
    padded to 56 bytes if NOT negotiated.
\item If the opcode (in \field{header}) is VIRTIO_CRYPTO_MAC_CREATE_SESSION
    then \field{op_flf} is struct virtio_crypto_mac_create_session_flf if
    VIRTIO_CRYPTO_F_REVISION_1 is negotiated and struct virtio_crypto_mac_create_session_flf is
    padded to 56 bytes if NOT negotiated, and \field{op_vlf} is struct
    virtio_crypto_mac_create_session_vlf.
\item If the opcode (in \field{header}) is VIRTIO_CRYPTO_AEAD_CREATE_SESSION
    then \field{op_flf} is struct virtio_crypto_aead_create_session_flf if
    VIRTIO_CRYPTO_F_REVISION_1 is negotiated and struct virtio_crypto_aead_create_session_flf is
    padded to 56 bytes if NOT negotiated, and \field{op_vlf} is struct
    virtio_crypto_aead_create_session_vlf.
\item If the opcode (in \field{header}) is VIRTIO_CRYPTO_AKCIPHER_CREATE_SESSION
    then \field{op_flf} is struct virtio_crypto_akcipher_create_session_flf if
    VIRTIO_CRYPTO_F_REVISION_1 is negotiated and struct virtio_crypto_akcipher_create_session_flf is
    padded to 56 bytes if NOT negotiated, and \field{op_vlf} is struct
    virtio_crypto_akcipher_create_session_vlf.
\item If the opcode (in \field{header}) is VIRTIO_CRYPTO_CIPHER_DESTROY_SESSION
    or VIRTIO_CRYPTO_HASH_DESTROY_SESSION or VIRTIO_CRYPTO_MAC_DESTROY_SESSION or
    VIRTIO_CRYPTO_AEAD_DESTROY_SESSION then \field{op_flf} is struct
    virtio_crypto_destroy_session_flf if VIRTIO_CRYPTO_F_REVISION_1 is negotiated and
    struct virtio_crypto_destroy_session_flf is padded to 56 bytes if NOT negotiated.
\end{itemize*}

\field{op_outcome} stores the result of operation and must be struct
virtio_crypto_destroy_session_input for destroy session or
struct virtio_crypto_create_session_input for create session.

\field{outcome_len} is the size of the structure used.


\paragraph{Session operation}\label{sec:Device Types / Crypto Device / Device
Operation / Control Virtqueue / Session operation}

The session is a handle which describes the cryptographic parameters to be
applied to a number of buffers.

The following structure stores the result of session creation set by the device:

\begin{lstlisting}
struct virtio_crypto_create_session_input {
    le64 session_id;
    le32 status;
    le32 padding;
};
\end{lstlisting}

A request to destroy a session includes the following information:

\begin{lstlisting}
struct virtio_crypto_destroy_session_flf {
    /* Device read only portion */
    le64  session_id;
};

struct virtio_crypto_destroy_session_input {
    /* Device write only portion */
    u8  status;
};
\end{lstlisting}


\subparagraph{Session operation: HASH session}\label{sec:Device Types / Crypto Device / Device
Operation / Control Virtqueue / Session operation / Session operation: HASH session}

The fixed-length parameters of HASH session requests is as follows:

\begin{lstlisting}
struct virtio_crypto_hash_create_session_flf {
    /* Device read only portion */

    /* See VIRTIO_CRYPTO_HASH_* above */
    le32 algo;
    /* hash result length */
    le32 hash_result_len;
};
\end{lstlisting}


\subparagraph{Session operation: MAC session}\label{sec:Device Types / Crypto Device / Device
Operation / Control Virtqueue / Session operation / Session operation: MAC session}

The fixed-length and the variable-length parameters of MAC session requests are as follows:

\begin{lstlisting}
struct virtio_crypto_mac_create_session_flf {
    /* Device read only portion */

    /* See VIRTIO_CRYPTO_MAC_* above */
    le32 algo;
    /* hash result length */
    le32 hash_result_len;
    /* length of authenticated key */
    le32 auth_key_len;
    le32 padding;
};

struct virtio_crypto_mac_create_session_vlf {
    /* Device read only portion */

    /* The authenticated key */
    u8 auth_key[auth_key_len];
};
\end{lstlisting}

The length of \field{auth_key} is specified in \field{auth_key_len} in the struct
virtio_crypto_mac_create_session_flf.


\subparagraph{Session operation: Symmetric algorithms session}\label{sec:Device Types / Crypto Device / Device
Operation / Control Virtqueue / Session operation / Session operation: Symmetric algorithms session}

The request of symmetric session could be the CIPHER algorithms request
or the chain algorithms (chaining CIPHER and HASH/MAC) request.

The fixed-length and the variable-length parameters of CIPHER session requests are as follows:

\begin{lstlisting}
struct virtio_crypto_cipher_session_flf {
    /* Device read only portion */

    /* See VIRTIO_CRYPTO_CIPHER* above */
    le32 algo;
    /* length of key */
    le32 key_len;
#define VIRTIO_CRYPTO_OP_ENCRYPT  1
#define VIRTIO_CRYPTO_OP_DECRYPT  2
    /* encryption or decryption */
    le32 op;
    le32 padding;
};

struct virtio_crypto_cipher_session_vlf {
    /* Device read only portion */

    /* The cipher key */
    u8 cipher_key[key_len];
};
\end{lstlisting}

The length of \field{cipher_key} is specified in \field{key_len} in the struct
virtio_crypto_cipher_session_flf.

The fixed-length and the variable-length parameters of Chain session requests are as follows:

\begin{lstlisting}
struct virtio_crypto_alg_chain_session_flf {
    /* Device read only portion */

#define VIRTIO_CRYPTO_SYM_ALG_CHAIN_ORDER_HASH_THEN_CIPHER  1
#define VIRTIO_CRYPTO_SYM_ALG_CHAIN_ORDER_CIPHER_THEN_HASH  2
    le32 alg_chain_order;
/* Plain hash */
#define VIRTIO_CRYPTO_SYM_HASH_MODE_PLAIN    1
/* Authenticated hash (mac) */
#define VIRTIO_CRYPTO_SYM_HASH_MODE_AUTH     2
/* Nested hash */
#define VIRTIO_CRYPTO_SYM_HASH_MODE_NESTED   3
    le32 hash_mode;
    struct virtio_crypto_cipher_session_flf cipher_hdr;

#define VIRTIO_CRYPTO_ALG_CHAIN_SESS_OP_SPEC_HDR_SIZE  16
    /* fixed length fields, algo specific */
    u8 algo_flf[VIRTIO_CRYPTO_ALG_CHAIN_SESS_OP_SPEC_HDR_SIZE];

    /* length of the additional authenticated data (AAD) in bytes */
    le32 aad_len;
    le32 padding;
};

struct virtio_crypto_alg_chain_session_vlf {
    /* Device read only portion */

    /* The cipher key */
    u8 cipher_key[key_len];
    /* The authenticated key */
    u8 auth_key[auth_key_len];
};
\end{lstlisting}

\field{hash_mode} decides the type used by \field{algo_flf}.

\field{algo_flf} is fixed to 16 bytes and MUST contains or be one of
the following types:
\begin{itemize*}
\item struct virtio_crypto_hash_create_session_flf
\item struct virtio_crypto_mac_create_session_flf
\end{itemize*}
The data of unused part (if has) in \field{algo_flf} will be ignored.

The length of \field{cipher_key} is specified in \field{key_len} in \field{cipher_hdr}.

The length of \field{auth_key} is specified in \field{auth_key_len} in struct
virtio_crypto_mac_create_session_flf.

The fixed-length parameters of Symmetric session requests are as follows:

\begin{lstlisting}
struct virtio_crypto_sym_create_session_flf {
    /* Device read only portion */

#define VIRTIO_CRYPTO_SYM_SESS_OP_SPEC_HDR_SIZE  48
    /* fixed length fields, opcode specific */
    u8 op_flf[VIRTIO_CRYPTO_SYM_SESS_OP_SPEC_HDR_SIZE];

/* No operation */
#define VIRTIO_CRYPTO_SYM_OP_NONE  0
/* Cipher only operation on the data */
#define VIRTIO_CRYPTO_SYM_OP_CIPHER  1
/* Chain any cipher with any hash or mac operation. The order
   depends on the value of alg_chain_order param */
#define VIRTIO_CRYPTO_SYM_OP_ALGORITHM_CHAINING  2
    le32 op_type;
    le32 padding;
};
\end{lstlisting}

\field{op_flf} is fixed to 48 bytes, MUST contains or be one of
the following types:
\begin{itemize*}
\item struct virtio_crypto_cipher_session_flf
\item struct virtio_crypto_alg_chain_session_flf
\end{itemize*}
The data of unused part (if has) in \field{op_flf} will be ignored.

\field{op_type} decides the type used by \field{op_flf}.

The variable-length parameters of Symmetric session requests are as follows:

\begin{lstlisting}
struct virtio_crypto_sym_create_session_vlf {
    /* Device read only portion */
    /* variable length fields, opcode specific */
    u8 op_vlf[vlf_len];
};
\end{lstlisting}

\field{op_vlf} MUST contains or be one of the following types:
\begin{itemize*}
\item struct virtio_crypto_cipher_session_vlf
\item struct virtio_crypto_alg_chain_session_vlf
\end{itemize*}

\field{op_type} in struct virtio_crypto_sym_create_session_flf decides the
type used by \field{op_vlf}.

\field{vlf_len} is the size of the specific structure used.


\subparagraph{Session operation: AEAD session}\label{sec:Device Types / Crypto Device / Device
Operation / Control Virtqueue / Session operation / Session operation: AEAD session}

The fixed-length and the variable-length parameters of AEAD session requests are as follows:

\begin{lstlisting}
struct virtio_crypto_aead_create_session_flf {
    /* Device read only portion */

    /* See VIRTIO_CRYPTO_AEAD_* above */
    le32 algo;
    /* length of key */
    le32 key_len;
    /* Authentication tag length */
    le32 tag_len;
    /* The length of the additional authenticated data (AAD) in bytes */
    le32 aad_len;
    /* encryption or decryption, See above VIRTIO_CRYPTO_OP_* */
    le32 op;
    le32 padding;
};

struct virtio_crypto_aead_create_session_vlf {
    /* Device read only portion */
    u8 key[key_len];
};
\end{lstlisting}

The length of \field{key} is specified in \field{key_len} in struct
virtio_crypto_aead_create_session_flf.

\subparagraph{Session operation: AKCIPHER session}\label{sec:Device Types / Crypto Device / Device
Operation / Control Virtqueue / Session operation / Session operation: AKCIPHER session}

Due to the complexity of asymmetric key algorithms, different algorithms
require different parameters. The following data structures are used as
supplementary parameters to describe the asymmetric algorithm sessions.

For the RSA algorithm, the extra parameters are as follows:
\begin{lstlisting}
struct virtio_crypto_rsa_session_para {
#define VIRTIO_CRYPTO_RSA_RAW_PADDING   0
#define VIRTIO_CRYPTO_RSA_PKCS1_PADDING 1
    le32 padding_algo;

#define VIRTIO_CRYPTO_RSA_NO_HASH   0
#define VIRTIO_CRYPTO_RSA_MD2       1
#define VIRTIO_CRYPTO_RSA_MD3       2
#define VIRTIO_CRYPTO_RSA_MD4       3
#define VIRTIO_CRYPTO_RSA_MD5       4
#define VIRTIO_CRYPTO_RSA_SHA1      5
#define VIRTIO_CRYPTO_RSA_SHA256    6
#define VIRTIO_CRYPTO_RSA_SHA384    7
#define VIRTIO_CRYPTO_RSA_SHA512    8
#define VIRTIO_CRYPTO_RSA_SHA224    9
    le32 hash_algo;
};
\end{lstlisting}

\field{padding_algo} specifies the padding method used by RSA sessions.
\begin{itemize*}
\item If VIRTIO_CRYPTO_RSA_RAW_PADDING is specified, 1) \field{hash_algo}
is ignored, 2) ciphertext and plaintext MUST be padded with leading zeros,
3) and RSA sessions with VIRTIO_CRYPTO_RSA_RAW_PADDING MUST not be used
for verification and signing operations.
\item If VIRTIO_CRYPTO_RSA_PKCS1_PADDING is specified, EMSA-PKCS1-v1_5 padding method
is used (see \hyperref[intro:rfc3447]{PKCS\#1}), \field{hash_algo} specifies how the
digest of the data passed to RSA sessions is calculated when verifying and signing.
It only affects the padding algorithm and is ignored during encryption and decryption.
\end{itemize*}

The ECC algorithms such as the ECDSA algorithm, cannot use custom curves, only the
following known curves can be used (see \hyperref[intro:NIST]{NIST-recommended curves}).

\begin{lstlisting}
#define VIRTIO_CRYPTO_CURVE_UNKNOWN   0
#define VIRTIO_CRYPTO_CURVE_NIST_P192 1
#define VIRTIO_CRYPTO_CURVE_NIST_P224 2
#define VIRTIO_CRYPTO_CURVE_NIST_P256 3
#define VIRTIO_CRYPTO_CURVE_NIST_P384 4
#define VIRTIO_CRYPTO_CURVE_NIST_P521 5
\end{lstlisting}

For the ECDSA algorithm, the extra parameters are as follows:
\begin{lstlisting}
struct virtio_crypto_ecdsa_session_para {
    /* See VIRTIO_CRYPTO_CURVE_* above */
    le32 curve_id;
};
\end{lstlisting}

The fixed-length and the variable-length parameters of AKCIPHER session requests are as follows:
\begin{lstlisting}
struct virtio_crypto_akcipher_create_session_flf {
    /* Device read only portion */

    /* See VIRTIO_CRYPTO_AKCIPHER_* above */
    le32 algo;
#define VIRTIO_CRYPTO_AKCIPHER_KEY_TYPE_PUBLIC 1
#define VIRTIO_CRYPTO_AKCIPHER_KEY_TYPE_PRIVATE 2
    le32 key_type;
    /* length of key */
    le32 key_len;

#define VIRTIO_CRYPTO_AKCIPHER_SESS_ALGO_SPEC_HDR_SIZE 44
    u8 algo_flf[VIRTIO_CRYPTO_AKCIPHER_SESS_ALGO_SPEC_HDR_SIZE];
};

struct virtio_crypto_akcipher_create_session_vlf {
    /* Device read only portion */
    u8 key[key_len];
};
\end{lstlisting}

\field{algo} decides the type used by \field{algo_flf}.
\field{algo_flf} is fixed to 44 bytes and MUST contains of be one the
following structures:
\begin{itemize*}
\item struct virtio_crypto_rsa_session_para
\item struct virtio_crypto_ecdsa_session_para
\end{itemize*}

The length of \field{key} is specified in \field{key_len} in the struct
virtio_crypto_akcipher_create_session_flf.

For the RSA algorithm, the key needs to be encoded according to
\hyperref[intro:rfc3447]{PKCS\#1}. The private key is described with the
RSAPrivateKey structure, and the public key is described with the RSAPublicKey
structure. These ASN.1 structures are encoded in DER encoding rules (see
\hyperref[intro:rfc6025]{rfc6025}).

\begin{lstlisting}
RSAPrivateKey ::= SEQUENCE {
    version          INTEGER,
    modulus          INTEGER,
    publicExponent   INTEGER,
    privateExponent  INTEGER,
    prime1           INTEGER,
    prime2           INTEGER,
    exponent1        INTEGER,
    exponent1        INTEGER,
    coefficient      INTEGER,
    otherPrimeInfos  OtherPrimeInfos OPTIONAL
}

OtherPrimeInfos ::= SEQUENCE SIZE(1...MAX) OF OtherPrimeInfo

OtherPrimeINfo ::= SEQUENCE {
    prime           INTEGER,
    exponent        INTEGER,
    coefficient     INTEGER
}

RSAPublicKey ::= SEQUENCE {
    modulus         INTEGER,
    publicExponent  INTEGER
}
\end{lstlisting}

For the ECDSA algorithm, the private key is encoded according to
\hyperref[intro:rfc5915]{RFC5915}, the private key of the ECDSA algorithm
is described by the ASN.1 structure ECPrivateKey and encoded with DER
encoding rules (see \hyperref[intro:rfc6025]{rfc6025}).

\begin{lstlisting}
ECPrivateKey ::= SEQUNCE {
    version         INTEGER,
    privateKey      OCTET STRING,
    parameters [0]  ECParameters {{ NamedCurve }} OPTIONAL,
    publicKey  [1]  BIT STRING OPTIONAL
}
\end{lstlisting}

The public key of the ECDSA algorithm is encoded according to \hyperref[intro:SEC1]{SEC1},
and the public key of ECDSA is described by the ASN.1 structure ECPoint.
When initializing a session with ECDSA public key, the ECPoint is DER encoded and the
\field{key} only contains the value part of ECPoint, that is, the header part of the
OCTET STRING will be omitted (see \hyperref[intro:rfc6025]{rfc6025}).

\begin{lstlisting}
ECPoint ::= OCTET STRING
\end{lstlisting}

The length of \field{key} is specified in \field{key_len} in
struct virtio_crypto_akcipher_create_session_flf.

\drivernormative{\subparagraph}{Session operation: create session}{Device Types / Crypto Device / Device
Operation / Control Virtqueue / Session operation / Session operation: create session}

\begin{itemize*}
\item The driver MUST set the \field{opcode} field based on service type: CIPHER, HASH, MAC, AEAD or AKCIPHER.
\item The driver MUST set the control general header, the opcode specific header,
    the opcode specific extra parameters and the opcode specific outcome buffer in turn.
    See \ref{sec:Device Types / Crypto Device / Device Operation / Control Virtqueue}.
\item The driver MUST set the \field{reversed} field to zero.
\end{itemize*}

\devicenormative{\subparagraph}{Session operation: create session}{Device Types / Crypto Device / Device
Operation / Control Virtqueue / Session operation / Session operation: create session}

\begin{itemize*}
\item The device MUST use the corresponding opcode specific structure according to the
    \field{opcode} in the control general header.
\item The device MUST extract extra parameters according to the structures used.
\item The device MUST set the \field{status} field to one of the following values of enum
    VIRTIO_CRYPTO_STATUS after finish a session creation:
\begin{itemize*}
\item VIRTIO_CRYPTO_OK if a session is created successfully.
\item VIRTIO_CRYPTO_NOTSUPP if the requested algorithm or operation is unsupported.
\item VIRTIO_CRYPTO_NOSPC if no free session ID (only when the VIRTIO_CRYPTO_F_REVISION_1
    feature bit is negotiated).
\item VIRTIO_CRYPTO_ERR if failure not mentioned above occurs.
\end{itemize*}
\item The device MUST set the \field{session_id} field to a unique session identifier only
    if the status is set to VIRTIO_CRYPTO_OK.
\end{itemize*}

\drivernormative{\subparagraph}{Session operation: destroy session}{Device Types / Crypto Device / Device
Operation / Control Virtqueue / Session operation / Session operation: destroy session}

\begin{itemize*}
\item The driver MUST set the \field{opcode} field based on service type: CIPHER, HASH, MAC, AEAD or AKCIPHER.
\item The driver MUST set the \field{session_id} to a valid value assigned by the device
    when the session was created.
\end{itemize*}

\devicenormative{\subparagraph}{Session operation: destroy session}{Device Types / Crypto Device / Device
Operation / Control Virtqueue / Session operation / Session operation: destroy session}

\begin{itemize*}
\item The device MUST set the \field{status} field to one of the following values of enum VIRTIO_CRYPTO_STATUS.
\begin{itemize*}
\item VIRTIO_CRYPTO_OK if a session is created successfully.
\item VIRTIO_CRYPTO_ERR if any failure occurs.
\end{itemize*}
\end{itemize*}


\subsubsection{Data Virtqueue}\label{sec:Device Types / Crypto Device / Device Operation / Data Virtqueue}

The driver uses the data virtqueues to transmit crypto operation requests to the device,
and completes the crypto operations.

The header for dataq is as follows:

\begin{lstlisting}
struct virtio_crypto_op_header {
#define VIRTIO_CRYPTO_CIPHER_ENCRYPT \
    VIRTIO_CRYPTO_OPCODE(VIRTIO_CRYPTO_SERVICE_CIPHER, 0x00)
#define VIRTIO_CRYPTO_CIPHER_DECRYPT \
    VIRTIO_CRYPTO_OPCODE(VIRTIO_CRYPTO_SERVICE_CIPHER, 0x01)
#define VIRTIO_CRYPTO_HASH \
    VIRTIO_CRYPTO_OPCODE(VIRTIO_CRYPTO_SERVICE_HASH, 0x00)
#define VIRTIO_CRYPTO_MAC \
    VIRTIO_CRYPTO_OPCODE(VIRTIO_CRYPTO_SERVICE_MAC, 0x00)
#define VIRTIO_CRYPTO_AEAD_ENCRYPT \
    VIRTIO_CRYPTO_OPCODE(VIRTIO_CRYPTO_SERVICE_AEAD, 0x00)
#define VIRTIO_CRYPTO_AEAD_DECRYPT \
    VIRTIO_CRYPTO_OPCODE(VIRTIO_CRYPTO_SERVICE_AEAD, 0x01)
#define VIRTIO_CRYPTO_AKCIPHER_ENCRYPT \
    VIRTIO_CRYPTO_OPCODE(VIRTIO_CRYPTO_SERVICE_AKCIPHER, 0x00)
#define VIRTIO_CRYPTO_AKCIPHER_DECRYPT \
    VIRTIO_CRYPTO_OPCODE(VIRTIO_CRYPTO_SERVICE_AKCIPHER, 0x01)
#define VIRTIO_CRYPTO_AKCIPHER_SIGN \
    VIRTIO_CRYPTO_OPCODE(VIRTIO_CRYPTO_SERVICE_AKCIPHER, 0x02)
#define VIRTIO_CRYPTO_AKCIPHER_VERIFY \
    VIRTIO_CRYPTO_OPCODE(VIRTIO_CRYPTO_SERVICE_AKCIPHER, 0x03)
    le32 opcode;
    /* algo should be service-specific algorithms */
    le32 algo;
    le64 session_id;
#define VIRTIO_CRYPTO_FLAG_SESSION_MODE 1
    /* control flag to control the request */
    le32 flag;
    le32 padding;
};
\end{lstlisting}

\begin{note}
If VIRTIO_CRYPTO_F_REVISION_1 is not negotiated the \field{flag} is ignored.

If VIRTIO_CRYPTO_F_REVISION_1 is negotiated but VIRTIO_CRYPTO_F_<SERVICE>_STATELESS_MODE
is not negotiated, then the device SHOULD reject <SERVICE> requests if
VIRTIO_CRYPTO_FLAG_SESSION_MODE is not set (in \field{flag}).
\end{note}

The dataq request is composed of four parts:
\begin{lstlisting}
struct virtio_crypto_op_data_req {
    /* Device read only portion */

    struct virtio_crypto_op_header header;

#define VIRTIO_CRYPTO_DATAQ_OP_SPEC_HDR_LEGACY 48
    /* fixed length fields, opcode specific */
    u8 op_flf[flf_len];

    /* Device read && write portion */
    /* variable length fields, opcode specific */
    u8 op_vlf[vlf_len];

    /* Device write only portion */
    struct virtio_crypto_inhdr inhdr;
};
\end{lstlisting}

\field{header} is a general header (see above).

\field{op_flf} is the opcode (in \field{header}) specific header.

\field{flf_len} depends on the VIRTIO_CRYPTO_F_REVISION_1 feature bit
(see below).

\field{op_vlf} is the opcode (in \field{header}) specific parameters.

\field{vlf_len} is the size of the specific structure used.

\begin{itemize*}
\item If the the opcode (in \field{header}) is VIRTIO_CRYPTO_CIPHER_ENCRYPT
    or VIRTIO_CRYPTO_CIPHER_DECRYPT then:
    \begin{itemize*}
    \item If VIRTIO_CRYPTO_F_CIPHER_STATELESS_MODE is negotiated, \field{op_flf} is
        struct virtio_crypto_sym_data_flf_stateless, and \field{op_vlf} is struct
        virtio_crypto_sym_data_vlf_stateless.
    \item If VIRTIO_CRYPTO_F_CIPHER_STATELESS_MODE is NOT negotiated, \field{op_flf}
        is struct virtio_crypto_sym_data_flf if VIRTIO_CRYPTO_F_REVISION_1 is negotiated
        and struct virtio_crypto_sym_data_flf is padded to 48 bytes if NOT negotiated,
        and \field{op_vlf} is struct virtio_crypto_sym_data_vlf.
    \end{itemize*}
\item If the the opcode (in \field{header}) is VIRTIO_CRYPTO_HASH:
    \begin{itemize*}
    \item If VIRTIO_CRYPTO_F_HASH_STATELESS_MODE is negotiated, \field{op_flf} is
        struct virtio_crypto_hash_data_flf_stateless, and \field{op_vlf} is struct
        virtio_crypto_hash_data_vlf_stateless.
    \item If VIRTIO_CRYPTO_F_HASH_STATELESS_MODE is NOT negotiated, \field{op_flf}
        is struct virtio_crypto_hash_data_flf if VIRTIO_CRYPTO_F_REVISION_1 is negotiated
        and struct virtio_crypto_hash_data_flf is padded to 48 bytes if NOT negotiated,
        and \field{op_vlf} is struct virtio_crypto_hash_data_vlf.
    \end{itemize*}
\item If the the opcode (in \field{header}) is VIRTIO_CRYPTO_MAC:
    \begin{itemize*}
    \item If VIRTIO_CRYPTO_F_MAC_STATELESS_MODE is negotiated, \field{op_flf} is
        struct virtio_crypto_mac_data_flf_stateless, and \field{op_vlf} is struct
        virtio_crypto_mac_data_vlf_stateless.
    \item If VIRTIO_CRYPTO_F_MAC_STATELESS_MODE is NOT negotiated, \field{op_flf}
        is struct virtio_crypto_mac_data_flf if VIRTIO_CRYPTO_F_REVISION_1 is negotiated
        and struct virtio_crypto_mac_data_flf is padded to 48 bytes if NOT negotiated,
        and \field{op_vlf} is struct virtio_crypto_mac_data_vlf.
    \end{itemize*}
\item If the the opcode (in \field{header}) is VIRTIO_CRYPTO_AEAD_ENCRYPT
    or VIRTIO_CRYPTO_AEAD_DECRYPT then:
    \begin{itemize*}
    \item If VIRTIO_CRYPTO_F_AEAD_STATELESS_MODE is negotiated, \field{op_flf} is
        struct virtio_crypto_aead_data_flf_stateless, and \field{op_vlf} is struct
        virtio_crypto_aead_data_vlf_stateless.
    \item If VIRTIO_CRYPTO_F_AEAD_STATELESS_MODE is NOT negotiated, \field{op_flf}
        is struct virtio_crypto_aead_data_flf if VIRTIO_CRYPTO_F_REVISION_1 is negotiated
        and struct virtio_crypto_aead_data_flf is padded to 48 bytes if NOT negotiated,
        and \field{op_vlf} is struct virtio_crypto_aead_data_vlf.
    \end{itemize*}
\item If the opcode (in \field{header}) is VIRTIO_CRYPTO_AKCIPHER_ENCRYPT, VIRTIO_CRYPTO_AKCIPHER_DECRYPT,
    VIRTIO_CRYPTO_AKCIPHER_SIGN or VIRTIO_CRYPTO_AKCIPHER_VERIFY then:
    \begin{itemize*}
    \item If VIRTIO_CRYPTO_F_AKCIPHER_STATELESS_MODE is negotiated, \field{op_flf} is
        struct virtio_crypto_akcipher_data_flf_statless, and \field{op_vlf} is struct
        virtio_crypto_akcipher_data_vlf_stateless.
    \item If VIRTIO_CRYPTO_F_AKCIPHER_STATELESS_MODE is NOT negotiated, \field{op_flf}
        is struct virtio_crypto_akcipher_data_flf if VIRTIO_CRYPTO_F_REVISION_1 is negotiated
        and struct virtio_crypto_akcipher_data_flf is padded to 48 bytes if NOT negotiated,
        and \field{op_vlf} is struct virtio_crypto_akcipher_data_vlf.
    \end{itemize*}
\end{itemize*}

\field{inhdr} is a unified input header that used to return the status of
the operations, is defined as follows:

\begin{lstlisting}
struct virtio_crypto_inhdr {
    u8 status;
};
\end{lstlisting}

\subsubsection{HASH Service Operation}\label{sec:Device Types / Crypto Device / Device Operation / HASH Service Operation}

Session mode HASH service requests are as follows:

\begin{lstlisting}
struct virtio_crypto_hash_data_flf {
    /* length of source data */
    le32 src_data_len;
    /* hash result length */
    le32 hash_result_len;
};

struct virtio_crypto_hash_data_vlf {
    /* Device read only portion */
    /* Source data */
    u8 src_data[src_data_len];

    /* Device write only portion */
    /* Hash result data */
    u8 hash_result[hash_result_len];
};
\end{lstlisting}

Each data request uses the virtio_crypto_hash_data_flf structure and the
virtio_crypto_hash_data_vlf structure to store information used to run the
HASH operations.

\field{src_data} is the source data that will be processed.
\field{src_data_len} is the length of source data.
\field{hash_result} is the result data and \field{hash_result_len} is the length
of it.

Stateless mode HASH service requests are as follows:

\begin{lstlisting}
struct virtio_crypto_hash_data_flf_stateless {
    struct {
        /* See VIRTIO_CRYPTO_HASH_* above */
        le32 algo;
    } sess_para;

    /* length of source data */
    le32 src_data_len;
    /* hash result length */
    le32 hash_result_len;
    le32 reserved;
};
struct virtio_crypto_hash_data_vlf_stateless {
    /* Device read only portion */
    /* Source data */
    u8 src_data[src_data_len];

    /* Device write only portion */
    /* Hash result data */
    u8 hash_result[hash_result_len];
};
\end{lstlisting}

\drivernormative{\paragraph}{HASH Service Operation}{Device Types / Crypto Device / Device Operation / HASH Service Operation}

\begin{itemize*}
\item If the driver uses the session mode, then the driver MUST set \field{session_id}
    in struct virtio_crypto_op_header to a valid value assigned by the device when the
    session was created.
\item If the VIRTIO_CRYPTO_F_HASH_STATELESS_MODE feature bit is negotiated, 1) if the
    driver uses the stateless mode, then the driver MUST set the \field{flag} field in
    struct virtio_crypto_op_header to ZERO and MUST set the fields in struct
    virtio_crypto_hash_data_flf_stateless.sess_para, 2) if the driver uses the session
    mode, then the driver MUST set the \field{flag} field in struct virtio_crypto_op_header
    to VIRTIO_CRYPTO_FLAG_SESSION_MODE.
\item The driver MUST set \field{opcode} in struct virtio_crypto_op_header to VIRTIO_CRYPTO_HASH.
\end{itemize*}

\devicenormative{\paragraph}{HASH Service Operation}{Device Types / Crypto Device / Device Operation / HASH Service Operation}

\begin{itemize*}
\item The device MUST use the corresponding structure according to the \field{opcode}
    in the data general header.
\item If the VIRTIO_CRYPTO_F_HASH_STATELESS_MODE feature bit is negotiated, the device
    MUST parse \field{flag} field in struct virtio_crypto_op_header in order to decide
    which mode the driver uses.
\item The device MUST copy the results of HASH operations in the hash_result[] if HASH
    operations success.
\item The device MUST set \field{status} in struct virtio_crypto_inhdr to one of the
    following values of enum VIRTIO_CRYPTO_STATUS:
\begin{itemize*}
\item VIRTIO_CRYPTO_OK if the operation success.
\item VIRTIO_CRYPTO_NOTSUPP if the requested algorithm or operation is unsupported.
\item VIRTIO_CRYPTO_INVSESS if the session ID invalid when in session mode.
\item VIRTIO_CRYPTO_ERR if any failure not mentioned above occurs.
\end{itemize*}
\end{itemize*}


\subsubsection{MAC Service Operation}\label{sec:Device Types / Crypto Device / Device Operation / MAC Service Operation}

Session mode MAC service requests are as follows:

\begin{lstlisting}
struct virtio_crypto_mac_data_flf {
    struct virtio_crypto_hash_data_flf hdr;
};

struct virtio_crypto_mac_data_vlf {
    /* Device read only portion */
    /* Source data */
    u8 src_data[src_data_len];

    /* Device write only portion */
    /* Hash result data */
    u8 hash_result[hash_result_len];
};
\end{lstlisting}

Each request uses the virtio_crypto_mac_data_flf structure and the
virtio_crypto_mac_data_vlf structure to store information used to run the
MAC operations.

\field{src_data} is the source data that will be processed.
\field{src_data_len} is the length of source data.
\field{hash_result} is the result data and \field{hash_result_len} is the length
of it.

Stateless mode MAC service requests are as follows:

\begin{lstlisting}
struct virtio_crypto_mac_data_flf_stateless {
    struct {
        /* See VIRTIO_CRYPTO_MAC_* above */
        le32 algo;
        /* length of authenticated key */
        le32 auth_key_len;
    } sess_para;

    /* length of source data */
    le32 src_data_len;
    /* hash result length */
    le32 hash_result_len;
};

struct virtio_crypto_mac_data_vlf_stateless {
    /* Device read only portion */
    /* The authenticated key */
    u8 auth_key[auth_key_len];
    /* Source data */
    u8 src_data[src_data_len];

    /* Device write only portion */
    /* Hash result data */
    u8 hash_result[hash_result_len];
};
\end{lstlisting}

\field{auth_key} is the authenticated key that will be used during the process.
\field{auth_key_len} is the length of the key.

\drivernormative{\paragraph}{MAC Service Operation}{Device Types / Crypto Device / Device Operation / MAC Service Operation}

\begin{itemize*}
\item If the driver uses the session mode, then the driver MUST set \field{session_id}
    in struct virtio_crypto_op_header to a valid value assigned by the device when the
    session was created.
\item If the VIRTIO_CRYPTO_F_MAC_STATELESS_MODE feature bit is negotiated, 1) if the
    driver uses the stateless mode, then the driver MUST set the \field{flag} field
    in struct virtio_crypto_op_header to ZERO and MUST set the fields in struct
    virtio_crypto_mac_data_flf_stateless.sess_para, 2) if the driver uses the session
    mode, then the driver MUST set the \field{flag} field in struct virtio_crypto_op_header
    to VIRTIO_CRYPTO_FLAG_SESSION_MODE.
\item The driver MUST set \field{opcode} in struct virtio_crypto_op_header to VIRTIO_CRYPTO_MAC.
\end{itemize*}

\devicenormative{\paragraph}{MAC Service Operation}{Device Types / Crypto Device / Device Operation / MAC Service Operation}

\begin{itemize*}
\item If the VIRTIO_CRYPTO_F_MAC_STATELESS_MODE feature bit is negotiated, the device
    MUST parse \field{flag} field in struct virtio_crypto_op_header in order to decide
	which mode the driver uses.
\item The device MUST copy the results of MAC operations in the hash_result[] if HASH
    operations success.
\item The device MUST set \field{status} in struct virtio_crypto_inhdr to one of the
    following values of enum VIRTIO_CRYPTO_STATUS:
\begin{itemize*}
\item VIRTIO_CRYPTO_OK if the operation success.
\item VIRTIO_CRYPTO_NOTSUPP if the requested algorithm or operation is unsupported.
\item VIRTIO_CRYPTO_INVSESS if the session ID invalid when in session mode.
\item VIRTIO_CRYPTO_ERR if any failure not mentioned above occurs.
\end{itemize*}
\end{itemize*}

\subsubsection{Symmetric algorithms Operation}\label{sec:Device Types / Crypto Device / Device Operation / Symmetric algorithms Operation}

Session mode CIPHER service requests are as follows:

\begin{lstlisting}
struct virtio_crypto_cipher_data_flf {
    /*
     * Byte Length of valid IV/Counter data pointed to by the below iv data.
     *
     * For block ciphers in CBC or F8 mode, or for Kasumi in F8 mode, or for
     *   SNOW3G in UEA2 mode, this is the length of the IV (which
     *   must be the same as the block length of the cipher).
     * For block ciphers in CTR mode, this is the length of the counter
     *   (which must be the same as the block length of the cipher).
     */
    le32 iv_len;
    /* length of source data */
    le32 src_data_len;
    /* length of destination data */
    le32 dst_data_len;
    le32 padding;
};

struct virtio_crypto_cipher_data_vlf {
    /* Device read only portion */

    /*
     * Initialization Vector or Counter data.
     *
     * For block ciphers in CBC or F8 mode, or for Kasumi in F8 mode, or for
     *   SNOW3G in UEA2 mode, this is the Initialization Vector (IV)
     *   value.
     * For block ciphers in CTR mode, this is the counter.
     * For AES-XTS, this is the 128bit tweak, i, from IEEE Std 1619-2007.
     *
     * The IV/Counter will be updated after every partial cryptographic
     * operation.
     */
    u8 iv[iv_len];
    /* Source data */
    u8 src_data[src_data_len];

    /* Device write only portion */
    /* Destination data */
    u8 dst_data[dst_data_len];
};
\end{lstlisting}

Session mode requests of algorithm chaining are as follows:

\begin{lstlisting}
struct virtio_crypto_alg_chain_data_flf {
    le32 iv_len;
    /* Length of source data */
    le32 src_data_len;
    /* Length of destination data */
    le32 dst_data_len;
    /* Starting point for cipher processing in source data */
    le32 cipher_start_src_offset;
    /* Length of the source data that the cipher will be computed on */
    le32 len_to_cipher;
    /* Starting point for hash processing in source data */
    le32 hash_start_src_offset;
    /* Length of the source data that the hash will be computed on */
    le32 len_to_hash;
    /* Length of the additional auth data */
    le32 aad_len;
    /* Length of the hash result */
    le32 hash_result_len;
    le32 reserved;
};

struct virtio_crypto_alg_chain_data_vlf {
    /* Device read only portion */

    /* Initialization Vector or Counter data */
    u8 iv[iv_len];
    /* Source data */
    u8 src_data[src_data_len];
    /* Additional authenticated data if exists */
    u8 aad[aad_len];

    /* Device write only portion */

    /* Destination data */
    u8 dst_data[dst_data_len];
    /* Hash result data */
    u8 hash_result[hash_result_len];
};
\end{lstlisting}

Session mode requests of symmetric algorithm are as follows:

\begin{lstlisting}
struct virtio_crypto_sym_data_flf {
    /* Device read only portion */

#define VIRTIO_CRYPTO_SYM_DATA_REQ_HDR_SIZE    40
    u8 op_type_flf[VIRTIO_CRYPTO_SYM_DATA_REQ_HDR_SIZE];

    /* See above VIRTIO_CRYPTO_SYM_OP_* */
    le32 op_type;
    le32 padding;
};

struct virtio_crypto_sym_data_vlf {
    u8 op_type_vlf[sym_para_len];
};
\end{lstlisting}

Each request uses the virtio_crypto_sym_data_flf structure and the
virtio_crypto_sym_data_flf structure to store information used to run the
CIPHER operations.

\field{op_type_flf} is the \field{op_type} specific header, it MUST starts
with or be one of the following structures:
\begin{itemize*}
\item struct virtio_crypto_cipher_data_flf
\item struct virtio_crypto_alg_chain_data_flf
\end{itemize*}

The length of \field{op_type_flf} is fixed to 40 bytes, the data of unused
part (if has) will be ignored.

\field{op_type_vlf} is the \field{op_type} specific parameters, it MUST starts
with or be one of the following structures:
\begin{itemize*}
\item struct virtio_crypto_cipher_data_vlf
\item struct virtio_crypto_alg_chain_data_vlf
\end{itemize*}

\field{sym_para_len} is the size of the specific structure used.

Stateless mode CIPHER service requests are as follows:

\begin{lstlisting}
struct virtio_crypto_cipher_data_flf_stateless {
    struct {
        /* See VIRTIO_CRYPTO_CIPHER* above */
        le32 algo;
        /* length of key */
        le32 key_len;

        /* See VIRTIO_CRYPTO_OP_* above */
        le32 op;
    } sess_para;

    /*
     * Byte Length of valid IV/Counter data pointed to by the below iv data.
     */
    le32 iv_len;
    /* length of source data */
    le32 src_data_len;
    /* length of destination data */
    le32 dst_data_len;
};

struct virtio_crypto_cipher_data_vlf_stateless {
    /* Device read only portion */

    /* The cipher key */
    u8 cipher_key[key_len];

    /* Initialization Vector or Counter data. */
    u8 iv[iv_len];
    /* Source data */
    u8 src_data[src_data_len];

    /* Device write only portion */
    /* Destination data */
    u8 dst_data[dst_data_len];
};
\end{lstlisting}

Stateless mode requests of algorithm chaining are as follows:

\begin{lstlisting}
struct virtio_crypto_alg_chain_data_flf_stateless {
    struct {
        /* See VIRTIO_CRYPTO_SYM_ALG_CHAIN_ORDER_* above */
        le32 alg_chain_order;
        /* length of the additional authenticated data in bytes */
        le32 aad_len;

        struct {
            /* See VIRTIO_CRYPTO_CIPHER* above */
            le32 algo;
            /* length of key */
            le32 key_len;
            /* See VIRTIO_CRYPTO_OP_* above */
            le32 op;
        } cipher;

        struct {
            /* See VIRTIO_CRYPTO_HASH_* or VIRTIO_CRYPTO_MAC_* above */
            le32 algo;
            /* length of authenticated key */
            le32 auth_key_len;
            /* See VIRTIO_CRYPTO_SYM_HASH_MODE_* above */
            le32 hash_mode;
        } hash;
    } sess_para;

    le32 iv_len;
    /* Length of source data */
    le32 src_data_len;
    /* Length of destination data */
    le32 dst_data_len;
    /* Starting point for cipher processing in source data */
    le32 cipher_start_src_offset;
    /* Length of the source data that the cipher will be computed on */
    le32 len_to_cipher;
    /* Starting point for hash processing in source data */
    le32 hash_start_src_offset;
    /* Length of the source data that the hash will be computed on */
    le32 len_to_hash;
    /* Length of the additional auth data */
    le32 aad_len;
    /* Length of the hash result */
    le32 hash_result_len;
    le32 reserved;
};

struct virtio_crypto_alg_chain_data_vlf_stateless {
    /* Device read only portion */

    /* The cipher key */
    u8 cipher_key[key_len];
    /* The auth key */
    u8 auth_key[auth_key_len];
    /* Initialization Vector or Counter data */
    u8 iv[iv_len];
    /* Additional authenticated data if exists */
    u8 aad[aad_len];
    /* Source data */
    u8 src_data[src_data_len];

    /* Device write only portion */

    /* Destination data */
    u8 dst_data[dst_data_len];
    /* Hash result data */
    u8 hash_result[hash_result_len];
};
\end{lstlisting}

Stateless mode requests of symmetric algorithm are as follows:

\begin{lstlisting}
struct virtio_crypto_sym_data_flf_stateless {
    /* Device read only portion */
#define VIRTIO_CRYPTO_SYM_DATE_REQ_HDR_STATELESS_SIZE    72
    u8 op_type_flf[VIRTIO_CRYPTO_SYM_DATE_REQ_HDR_STATELESS_SIZE];

    /* Device write only portion */
    /* See above VIRTIO_CRYPTO_SYM_OP_* */
    le32 op_type;
};

struct virtio_crypto_sym_data_vlf_stateless {
    u8 op_type_vlf[sym_para_len];
};
\end{lstlisting}

\field{op_type_flf} is the \field{op_type} specific header, it MUST starts
with or be one of the following structures:
\begin{itemize*}
\item struct virtio_crypto_cipher_data_flf_stateless
\item struct virtio_crypto_alg_chain_data_flf_stateless
\end{itemize*}

The length of \field{op_type_flf} is fixed to 72 bytes, the data of unused
part (if has) will be ignored.

\field{op_type_vlf} is the \field{op_type} specific parameters, it MUST starts
with or be one of the following structures:
\begin{itemize*}
\item struct virtio_crypto_cipher_data_vlf_stateless
\item struct virtio_crypto_alg_chain_data_vlf_stateless
\end{itemize*}

\field{sym_para_len} is the size of the specific structure used.

\drivernormative{\paragraph}{Symmetric algorithms Operation}{Device Types / Crypto Device / Device Operation / Symmetric algorithms Operation}

\begin{itemize*}
\item If the driver uses the session mode, then the driver MUST set \field{session_id}
    in struct virtio_crypto_op_header to a valid value assigned by the device when the
    session was created.
\item If the VIRTIO_CRYPTO_F_CIPHER_STATELESS_MODE feature bit is negotiated, 1) if the
    driver uses the stateless mode, then the driver MUST set the \field{flag} field in
    struct virtio_crypto_op_header to ZERO and MUST set the fields in struct
    virtio_crypto_cipher_data_flf_stateless.sess_para or struct
    virtio_crypto_alg_chain_data_flf_stateless.sess_para, 2) if the driver uses the
    session mode, then the driver MUST set the \field{flag} field in struct
    virtio_crypto_op_header to VIRTIO_CRYPTO_FLAG_SESSION_MODE.
\item The driver MUST set the \field{opcode} field in struct virtio_crypto_op_header
    to VIRTIO_CRYPTO_CIPHER_ENCRYPT or VIRTIO_CRYPTO_CIPHER_DECRYPT.
\item The driver MUST specify the fields of struct virtio_crypto_cipher_data_flf in
    struct virtio_crypto_sym_data_flf and struct virtio_crypto_cipher_data_vlf in
    struct virtio_crypto_sym_data_vlf if the request is based on VIRTIO_CRYPTO_SYM_OP_CIPHER.
\item The driver MUST specify the fields of struct virtio_crypto_alg_chain_data_flf
    in struct virtio_crypto_sym_data_flf and struct virtio_crypto_alg_chain_data_vlf
    in struct virtio_crypto_sym_data_vlf if the request is of the VIRTIO_CRYPTO_SYM_OP_ALGORITHM_CHAINING
    type.
\end{itemize*}

\devicenormative{\paragraph}{Symmetric algorithms Operation}{Device Types / Crypto Device / Device Operation / Symmetric algorithms Operation}

\begin{itemize*}
\item If the VIRTIO_CRYPTO_F_CIPHER_STATELESS_MODE feature bit is negotiated, the device
    MUST parse \field{flag} field in struct virtio_crypto_op_header in order to decide
	which mode the driver uses.
\item The device MUST parse the virtio_crypto_sym_data_req based on the \field{opcode}
    field in general header.
\item The device MUST parse the fields of struct virtio_crypto_cipher_data_flf in
    struct virtio_crypto_sym_data_flf and struct virtio_crypto_cipher_data_vlf in
    struct virtio_crypto_sym_data_vlf if the request is based on VIRTIO_CRYPTO_SYM_OP_CIPHER.
\item The device MUST parse the fields of struct virtio_crypto_alg_chain_data_flf
    in struct virtio_crypto_sym_data_flf and struct virtio_crypto_alg_chain_data_vlf
    in struct virtio_crypto_sym_data_vlf if the request is of the VIRTIO_CRYPTO_SYM_OP_ALGORITHM_CHAINING
    type.
\item The device MUST copy the result of cryptographic operation in the dst_data[] in
    both plain CIPHER mode and algorithms chain mode.
\item The device MUST check the \field{para}.\field{add_len} is bigger than 0 before
    parse the additional authenticated data in plain algorithms chain mode.
\item The device MUST copy the result of HASH/MAC operation in the hash_result[] is
    of the VIRTIO_CRYPTO_SYM_OP_ALGORITHM_CHAINING type.
\item The device MUST set the \field{status} field in struct virtio_crypto_inhdr to
    one of the following values of enum VIRTIO_CRYPTO_STATUS:
\begin{itemize*}
\item VIRTIO_CRYPTO_OK if the operation success.
\item VIRTIO_CRYPTO_NOTSUPP if the requested algorithm or operation is unsupported.
\item VIRTIO_CRYPTO_INVSESS if the session ID is invalid in session mode.
\item VIRTIO_CRYPTO_ERR if failure not mentioned above occurs.
\end{itemize*}
\end{itemize*}

\subsubsection{AEAD Service Operation}\label{sec:Device Types / Crypto Device / Device Operation / AEAD Service Operation}

Session mode requests of symmetric algorithm are as follows:

\begin{lstlisting}
struct virtio_crypto_aead_data_flf {
    /*
     * Byte Length of valid IV data.
     *
     * For GCM mode, this is either 12 (for 96-bit IVs) or 16, in which
     *   case iv points to J0.
     * For CCM mode, this is the length of the nonce, which can be in the
     *   range 7 to 13 inclusive.
     */
    le32 iv_len;
    /* length of additional auth data */
    le32 aad_len;
    /* length of source data */
    le32 src_data_len;
    /* length of dst data, this should be at least src_data_len + tag_len */
    le32 dst_data_len;
    /* Authentication tag length */
    le32 tag_len;
    le32 reserved;
};

struct virtio_crypto_aead_data_vlf {
    /* Device read only portion */

    /*
     * Initialization Vector data.
     *
     * For GCM mode, this is either the IV (if the length is 96 bits) or J0
     *   (for other sizes), where J0 is as defined by NIST SP800-38D.
     *   Regardless of the IV length, a full 16 bytes needs to be allocated.
     * For CCM mode, the first byte is reserved, and the nonce should be
     *   written starting at &iv[1] (to allow space for the implementation
     *   to write in the flags in the first byte).  Note that a full 16 bytes
     *   should be allocated, even though the iv_len field will have
     *   a value less than this.
     *
     * The IV will be updated after every partial cryptographic operation.
     */
    u8 iv[iv_len];
    /* Source data */
    u8 src_data[src_data_len];
    /* Additional authenticated data if exists */
    u8 aad[aad_len];

    /* Device write only portion */
    /* Pointer to output data */
    u8 dst_data[dst_data_len];
};
\end{lstlisting}

Each request uses the virtio_crypto_aead_data_flf structure and the
virtio_crypto_aead_data_flf structure to store information used to run the
AEAD operations.

Stateless mode AEAD service requests are as follows:

\begin{lstlisting}
struct virtio_crypto_aead_data_flf_stateless {
    struct {
        /* See VIRTIO_CRYPTO_AEAD_* above */
        le32 algo;
        /* length of key */
        le32 key_len;
        /* encrypt or decrypt, See above VIRTIO_CRYPTO_OP_* */
        le32 op;
    } sess_para;

    /* Byte Length of valid IV data. */
    le32 iv_len;
    /* Authentication tag length */
    le32 tag_len;
    /* length of additional auth data */
    le32 aad_len;
    /* length of source data */
    le32 src_data_len;
    /* length of dst data, this should be at least src_data_len + tag_len */
    le32 dst_data_len;
};

struct virtio_crypto_aead_data_vlf_stateless {
    /* Device read only portion */

    /* The cipher key */
    u8 key[key_len];
    /* Initialization Vector data. */
    u8 iv[iv_len];
    /* Source data */
    u8 src_data[src_data_len];
    /* Additional authenticated data if exists */
    u8 aad[aad_len];

    /* Device write only portion */
    /* Pointer to output data */
    u8 dst_data[dst_data_len];
};
\end{lstlisting}

\drivernormative{\paragraph}{AEAD Service Operation}{Device Types / Crypto Device / Device Operation / AEAD Service Operation}

\begin{itemize*}
\item If the driver uses the session mode, then the driver MUST set
    \field{session_id} in struct virtio_crypto_op_header to a valid value assigned
    by the device when the session was created.
\item If the VIRTIO_CRYPTO_F_AEAD_STATELESS_MODE feature bit is negotiated, 1) if
    the driver uses the stateless mode, then the driver MUST set the \field{flag}
    field in struct virtio_crypto_op_header to ZERO and MUST set the fields in
    struct virtio_crypto_aead_data_flf_stateless.sess_para, 2) if the driver uses
    the session mode, then the driver MUST set the \field{flag} field in struct
    virtio_crypto_op_header to VIRTIO_CRYPTO_FLAG_SESSION_MODE.
\item The driver MUST set the \field{opcode} field in struct virtio_crypto_op_header
    to VIRTIO_CRYPTO_AEAD_ENCRYPT or VIRTIO_CRYPTO_AEAD_DECRYPT.
\end{itemize*}

\devicenormative{\paragraph}{AEAD Service Operation}{Device Types / Crypto Device / Device Operation / AEAD Service Operation}

\begin{itemize*}
\item If the VIRTIO_CRYPTO_F_AEAD_STATELESS_MODE feature bit is negotiated, the
    device MUST parse the virtio_crypto_aead_data_vlf_stateless based on the \field{opcode}
	field in general header.
\item The device MUST copy the result of cryptographic operation in the dst_data[].
\item The device MUST copy the authentication tag in the dst_data[] offset the cipher result.
\item The device MUST set the \field{status} field in struct virtio_crypto_inhdr to
    one of the following values of enum VIRTIO_CRYPTO_STATUS:
\item When the \field{opcode} field is VIRTIO_CRYPTO_AEAD_DECRYPT, the device MUST
    verify and return the verification result to the driver.
\begin{itemize*}
\item VIRTIO_CRYPTO_OK if the operation success.
\item VIRTIO_CRYPTO_NOTSUPP if the requested algorithm or operation is unsupported.
\item VIRTIO_CRYPTO_BADMSG if the verification result is incorrect.
\item VIRTIO_CRYPTO_INVSESS if the session ID invalid when in session mode.
\item VIRTIO_CRYPTO_ERR if any failure not mentioned above occurs.
\end{itemize*}
\end{itemize*}

\subsubsection{AKCIPHER Service Operation}\label{sec:Device Types / Crypto Device / Device Operation / AKCIPHER Service Operation}

Session mode AKCIPHER requests are as follows:

\begin{lstlisting}
struct virtio_crypto_akcipher_data_flf {
    /* length of source data */
    le32 src_data_len;
    /* length of dst data */
    le32 dst_data_len;
};

struct virtio_crypto_akcipher_data_vlf {
    /* Device read only portion */
    /* Source data */
    u8 src_data[src_data_len];

    /* Device write only portion */
    /* Pointer to output data */
    u8 dst_data[dst_data_len];
};
\end{lstlisting}

Each data request uses the virtio_crypto_akcipher_flf structure and the virtio_crypto_akcipher_data_vlf
structure to store information used to run the AKCIPHER operations.

For encryption, decryption, and signing:
\field{src_data} is the source data that will be processed, note that for signing operations,
src_data stores the data to be signed, which usually is the digest of some data rather than the
data itself.
\field{src_data_len} is the length of source data.
\field{dst_result} is the result data and \field{dst_data_len} is the length of it. Note that the
length of the result is not always exactly equal to dst_data_len, the driver needs to check how
many bytes the device has written and calculate the actual length of the result.

For verification:
\field{src_data_len} refers to the length of the signature, and \field{dst_data_len} refers to
the length of signed data, where the signed data is usually the digest of some data.
\field{src_data} is spliced by the signature and the signed data, the src_data with the lower
address stores the signature, and the higher address stores the signed data.
\field{dst_data} is always empty for verification.

Different algorithms have different signature formats.
For the RSA algorithm, the result is determined by the padding algorithm specified by
\field{padding_algo} in structure virtio_crypto_rsa_session_para.

For the ECDSA algorithm, the signature is composed of the following
ASN.1 structure (see \hyperref[intro:rfc3279]{RFC3279})
and MUST be DER encoded (see \hyperref[intro:rfc6025]{rfc6025}).

\begin{lstlisting}
Ecdsa-Sig-Value ::= SEQUENCE {
    r INTEGER,
    s INTEGER
}
\end{lstlisting}

Stateless mode AKCIPHER service requests are as follows:
\begin{lstlisting}
struct virtio_crypto_akcipher_data_flf_stateless {
    struct {
        /* See VIRTIO_CYRPTO_AKCIPHER* above */
        le32 algo;
        /* See VIRTIO_CRYPTO_AKCIPHER_KEY_TYPE_* above */
        le32 key_type;
        /* length of key */
        le32 key_len;

        /* algothrim specific parameters described above */
        union {
            struct virtio_crypto_rsa_session_para rsa;
            struct virtio_crypto_ecdsa_session_para ecdsa;
        } u;
    } sess_para;

    /* length of source data */
    le32 src_data_len;
    /* length of destination data */
    le32 dst_data_len;
};

struct virtio_crypto_akcipher_data_vlf_stateless {
    /* Device read only portion */
    u8 akcipher_key[key_len];

    /* Source data */
    u8 src_data[src_data_len];

    /* Device write only portion */
    u8 dst_data[dst_data_len];
};
\end{lstlisting}

In stateless mode, the format of key and signature, the meaning of src_data and dst_data, are all the same
with session mode.

\drivernormative{\paragraph}{AKCIPHER Service Operation}{Device Types / Crypto Device / Device Operation / AKCIPHER Service Operation}

\begin{itemize*}
\item If the driver uses the session mode, then the driver MUST set
    \field{session_id} in struct virtio_crypto_op_header to a valid
    value assigned by the device when the session was created.
\item If the VIRTIO_CRYPTO_F_AKCIPHER_STATELESS_MODE feature bit is negotiated, 1) if the
    driver uses the stateless mode, then the driver MUST set the \field{flag} field in
    struct virtio_crypto_op_header to ZERO and MUST set the fields in struct
    virtio_crypto_akcipher_flf_stateless.sess_para, 2) if the driver uses the session
    mode, then the driver MUST set the \field{flag} field in struct virtio_crypto_op_header
    to VIRTIO_CRYPTO_FLAG_SESSION_MODE.
\item The driver MUST set the \field{opcode} field in struct virtio_crypto_op_header
    to one of VIRTIO_CRYPTO_AKCIPHER_ENCRYPT, VIRTIO_CRYPTO_AKCIPHER_DESTROY_SESSION,
    VIRTIO_CRYPTO_AKCIPHER_SIGN, and VIRTIO_CRYPTO_AKCIPHER_VERIFY.
\end{itemize*}

\devicenormative{\paragraph}{AKCIPHER Service Operation}{Device Types / Crypto Device / Device Operation / AKCIPHER Service Operation}

\begin{itemize*}
\item If the VIRTIO_CRYPTO_F_AKCIPHER_STATELESS_MODE feature bit is negotiated, the
    device MUST parse the virtio_crypto_akcipher_data_vlf_stateless based on the \field{opcode}
    field in general header.
\item The device MUST copy the result of cryptographic operation in the dst_data[].
\item The device MUST set the \field{status} field in struct virtio_crypto_inhdr to
    one of the following values of enum VIRTIO_CRYPTO_STATUS:
\begin{itemize*}
\item VIRTIO_CRYPTO_OK if the operation success.
\item VIRTIO_CRYPTO_NOTSUPP if the requested algorithm or operation is unsupported.
\item VIRTIO_CRYPTO_BADMSG if the verification result is incorrect.
\item VIRTIO_CRYPTO_INVSESS if the session ID invalid when in session mode.
\item VIRTIO_CRYPTO_KEY_REJECTED if the signature verification failed.
\item VIRTIO_CRYPTO_ERR if any failure not mentioned above occurs.
\end{itemize*}
\end{itemize*}


\chapter{Reserved Feature Bits}\label{sec:Reserved Feature Bits}

Currently these device-independent feature bits are defined:

\begin{description}
  \item[VIRTIO_F_INDIRECT_DESC (28)] Negotiating this feature indicates
  that the driver can use descriptors with the VIRTQ_DESC_F_INDIRECT
  flag set, as described in \ref{sec:Basic Facilities of a Virtio
Device / Virtqueues / The Virtqueue Descriptor Table / Indirect
Descriptors}~\nameref{sec:Basic Facilities of a Virtio Device /
Virtqueues / The Virtqueue Descriptor Table / Indirect
Descriptors} and \ref{sec:Packed Virtqueues / Indirect Flag: Scatter-Gather Support}~\nameref{sec:Packed Virtqueues / Indirect Flag: Scatter-Gather Support}.
  \item[VIRTIO_F_EVENT_IDX(29)] This feature enables the \field{used_event}
  and the \field{avail_event} fields as described in
\ref{sec:Basic Facilities of a Virtio Device / Virtqueues / Used Buffer Notification Suppression}, \ref{sec:Basic Facilities of a Virtio Device / Virtqueues / The Virtqueue Used Ring} and \ref{sec:Packed Virtqueues / Driver and Device Event Suppression}.


  \item[VIRTIO_F_VERSION_1(32)] This indicates compliance with this
    specification, giving a simple way to detect legacy devices or drivers.

  \item[VIRTIO_F_ACCESS_PLATFORM(33)] This feature indicates that
  the device can be used on a platform where device access to data
  in memory is limited and/or translated. E.g. this is the case if the device can be located
  behind an IOMMU that translates bus addresses from the device into physical
  addresses in memory, if the device can be limited to only access
  certain memory addresses or if special commands such as
  a cache flush can be needed to synchronise data in memory with
  the device. Whether accesses are actually limited or translated
  is described by platform-specific means.
  If this feature bit is set to 0, then the device
  has same access to memory addresses supplied to it as the
  driver has.
  In particular, the device will always use physical addresses
  matching addresses used by the driver (typically meaning
  physical addresses used by the CPU)
  and not translated further, and can access any address supplied to it by
  the driver. When clear, this overrides any platform-specific description of
  whether device access is limited or translated in any way, e.g.
  whether an IOMMU may be present.
  \item[VIRTIO_F_RING_PACKED(34)] This feature indicates
  support for the packed virtqueue layout as described in
  \ref{sec:Basic Facilities of a Virtio Device / Packed Virtqueues}~\nameref{sec:Basic Facilities of a Virtio Device / Packed Virtqueues}.
  \item[VIRTIO_F_IN_ORDER(35)] This feature indicates
  that all buffers are used by the device in the same
  order in which they have been made available.
  \item[VIRTIO_F_ORDER_PLATFORM(36)] This feature indicates
  that memory accesses by the driver and the device are ordered
  in a way described by the platform.

  If this feature bit is negotiated, the ordering in effect for any
  memory accesses by the driver that need to be ordered in a specific way
  with respect to accesses by the device is the one suitable for devices
  described by the platform. This implies that the driver needs to use
  memory barriers suitable for devices described by the platform; e.g.
  for the PCI transport in the case of hardware PCI devices.

  If this feature bit is not negotiated, then the device
  and driver are assumed to be implemented in software, that is
  they can be assumed to run on identical CPUs
  in an SMP configuration.
  Thus a weaker form of memory barriers is sufficient
  to yield better performance.
  \item[VIRTIO_F_SR_IOV(37)] This feature indicates that
  the device supports Single Root I/O Virtualization.
  Currently only PCI devices support this feature.
  \item[VIRTIO_F_NOTIFICATION_DATA(38)] This feature indicates
  that the driver passes extra data (besides identifying the virtqueue)
  in its device notifications.
  See \ref{sec:Basic Facilities of a Virtio Device / Driver notifications}~\nameref{sec:Basic Facilities of a Virtio Device / Driver notifications}.

  \item[VIRTIO_F_NOTIF_CONFIG_DATA(39)] This feature indicates that the driver
  uses the data provided by the device as a virtqueue identifier in available
  buffer notifications.
  As mentioned in section \ref{sec:Basic Facilities of a Virtio Device / Driver notifications}, when the
  driver is required to send an available buffer notification to the device, it
  sends the virtqueue index to be notified. The method of delivering
  notifications is transport specific.
  With the PCI transport, the device can optionally provide a per-virtqueue value
  for the driver to use in driver notifications, instead of the virtqueue index.
  Some devices may benefit from this flexibility by providing, for example,
  an internal virtqueue identifier, or an internal offset related to the
  virtqueue index.

  This feature indicates the availability of such value. The definition of the
  data to be provided in driver notification and the delivery method is
  transport specific.
  For more details about driver notifications over PCI see \ref{sec:Virtio Transport Options / Virtio Over PCI Bus / PCI-specific Initialization And Device Operation / Available Buffer Notifications}.

  \item[VIRTIO_F_RING_RESET(40)] This feature indicates
  that the driver can reset a queue individually.
  See \ref{sec:Basic Facilities of a Virtio Device / Virtqueues / Virtqueue Reset}.

  \item[VIRTIO_F_ADMIN_VQ(41)] This feature indicates that the device exposes one or more
  administration virtqueues.
  At the moment this feature is only supported for devices using
  \ref{sec:Virtio Transport Options / Virtio Over PCI
	  Bus}~\nameref{sec:Virtio Transport Options / Virtio Over PCI Bus}
	  as the transport and is reserved for future use for
	  devices using other transports (see
	  \ref{drivernormative:Basic Facilities of a Virtio Device / Feature Bits}
	and
	\ref{devicenormative:Basic Facilities of a Virtio Device / Feature Bits} for
	handling features reserved for future use.

\end{description}

\drivernormative{\section}{Reserved Feature Bits}{Reserved Feature Bits}

A driver MUST accept VIRTIO_F_VERSION_1 if it is offered.  A driver
MAY fail to operate further if VIRTIO_F_VERSION_1 is not offered.

A driver SHOULD accept VIRTIO_F_ACCESS_PLATFORM if it is offered, and it MUST
then either disable the IOMMU or configure the IOMMU to translate bus addresses
passed to the device into physical addresses in memory.  If
VIRTIO_F_ACCESS_PLATFORM is not offered, then a driver MUST pass only physical
addresses to the device.

A driver SHOULD accept VIRTIO_F_RING_PACKED if it is offered.

A driver SHOULD accept VIRTIO_F_ORDER_PLATFORM if it is offered.
If VIRTIO_F_ORDER_PLATFORM has been negotiated, a driver MUST use
the barriers suitable for hardware devices.

If VIRTIO_F_SR_IOV has been negotiated, a driver MAY enable
virtual functions through the device's PCI SR-IOV capability
structure.  A driver MUST NOT negotiate VIRTIO_F_SR_IOV if
the device does not have a PCI SR-IOV capability structure
or is not a PCI device.  A driver MUST negotiate
VIRTIO_F_SR_IOV and complete the feature negotiation
(including checking the FEATURES_OK \field{device status}
bit) before enabling virtual functions through the device's
PCI SR-IOV capability structure.  After once successfully
negotiating VIRTIO_F_SR_IOV, the driver MAY enable virtual
functions through the device's PCI SR-IOV capability
structure even if the device or the system has been fully
or partially reset, and even without re-negotiating
VIRTIO_F_SR_IOV after the reset.

A driver SHOULD accept VIRTIO_F_NOTIF_CONFIG_DATA if it is offered.

\devicenormative{\section}{Reserved Feature Bits}{Reserved Feature Bits}

A device MUST offer VIRTIO_F_VERSION_1.  A device MAY fail to operate further
if VIRTIO_F_VERSION_1 is not accepted.

A device SHOULD offer VIRTIO_F_ACCESS_PLATFORM if its access to
memory is through bus addresses distinct from and translated
by the platform to physical addresses used by the driver, and/or
if it can only access certain memory addresses with said access
specified and/or granted by the platform.
A device MAY fail to operate further if VIRTIO_F_ACCESS_PLATFORM is not
accepted.

If VIRTIO_F_IN_ORDER has been negotiated, a device MUST use
buffers in the same order in which they have been available.

A device MAY fail to operate further if
VIRTIO_F_ORDER_PLATFORM is offered but not accepted.
A device MAY operate in a slower emulation mode if
VIRTIO_F_ORDER_PLATFORM is offered but not accepted.

It is RECOMMENDED that an add-in card based PCI device
offers both VIRTIO_F_ACCESS_PLATFORM and
VIRTIO_F_ORDER_PLATFORM for maximum portability.

A device SHOULD offer VIRTIO_F_SR_IOV if it is a PCI device
and presents a PCI SR-IOV capability structure, otherwise
it MUST NOT offer VIRTIO_F_SR_IOV.

\section{Legacy Interface: Reserved Feature Bits}\label{sec:Reserved Feature Bits / Legacy Interface: Reserved Feature Bits}

Transitional devices MAY offer the following:
\begin{description}
\item[VIRTIO_F_NOTIFY_ON_EMPTY (24)] If this feature
  has been negotiated by driver, the device MUST issue
  a used buffer notification if the device runs
  out of available descriptors on a virtqueue, even though
  notifications are suppressed using the VIRTQ_AVAIL_F_NO_INTERRUPT
  flag or the \field{used_event} field.
\begin{note}
  An example of a driver using this feature is the legacy
  networking driver: it doesn't need to know every time a packet
  is transmitted, but it does need to free the transmitted
  packets a finite time after they are transmitted. It can avoid
  using a timer if the device notifies it when all the packets
  are transmitted.
\end{note}
\end{description}

Transitional devices MUST offer, and if offered by the device
transitional drivers MUST accept the following:
\begin{description}
\item[VIRTIO_F_ANY_LAYOUT (27)] This feature indicates that the device
  accepts arbitrary descriptor layouts, as described in Section
  \ref{sec:Basic Facilities of a Virtio Device / Virtqueues / Message Framing / Legacy Interface: Message Framing}~\nameref{sec:Basic Facilities of a Virtio Device / Virtqueues / Message Framing / Legacy Interface: Message Framing}.

\item[UNUSED (30)] Bit 30 is used by qemu's implementation to check
  for experimental early versions of virtio which did not perform
  correct feature negotiation, and SHOULD NOT be negotiated.
\end{description}
