\chapter{Creating New Transports}\label{sec:Creating New Transports}

Devices and drivers utilize various transport methods to facilitate
communication, such as PCI, MMIO, or Channel I/O. These transport
methods determine aspects of the interaction between the device and the
driver, including device discovery, capability exchange, interrupt
handling, and data transfer. For instance, in a host/guest architecture,
the host might expose a device to the guest via a virtual PCI bus, and
the guest would use a PCI device driver to interface with the device.

This section outlines the mandatory requirements that a transport method
implements.

A transport provides a mechanism to implement configuration space for
the device.

A transport provides a mechanism for the driver to identify the device
type.

A transport provides a mechanism for the driver to read the device's
FEATURES_OK and DEVICE_NEEDS_RESET status bits.

A transport provides a mechanism for the driver to modify the device's
status.

A transport provides a mechanism for the driver to read and modify the
device's feature bits.

A transport allows one or more virtqueues to be implemented by the
device. The number of virtqueues is device specific and not specified by
the transport.

A transport provides a mechanism for the driver to communicate virtqueue
configuration and memory location to the device.

A transport provides a mechanism for the device to send device
notifications to the driver, such as used buffer notifications.

A transport provides a mechanism for the driver to send driver
notifications to the device, such as available buffer notifications.

A transport provides a mechanism for the driver to initiate a device
reset.
